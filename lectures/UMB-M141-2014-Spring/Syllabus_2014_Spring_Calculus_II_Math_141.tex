\documentclass{article}
\usepackage{amsmath, amsfonts, amssymb, verbatim, hyperref}
\usepackage{enumitem}
\usepackage{pst-plot}
\usepackage{pstricks}
\addtolength{\hoffset}{-3.5cm}
\addtolength{\textwidth}{6.8cm}
\addtolength{\voffset}{-3.3cm}
\addtolength{\textheight}{6.3cm}
\title{Math 141 Calculus II \\ Spring 2014}
\begin{document}
%\color{green}
\maketitle
%\noindent\textbf{Time and place.}
%Monday, Wednesday, Friday 10-10:50, McCormack, Room 417, first floor. Monday 11:00-11:50,

\noindent \textbf{Instructor.} Todor Milev, \href{mailto:todor.milev@umb.edu}{\nolinkurl{todor.milev@umb.edu}} \quad \quad \quad .

\medskip
\noindent \textbf{Office hours. } Monday, Wednesday 11:00-12:00, 12:30-13:50, Friday 11:00-12:00, room S-3-65, or by email appointment.

\medskip
\noindent \textbf{Online resources. }  \url{https://piazza.com/umb/spring2014/m141/home}  \quad \quad \quad .


\medskip\noindent \textbf{Textbook. }  James Stewart, Calculus, 7th edition, published by Brooks Cole, 2012. ISBN-13: 978-0-538-49781-7
ISBN-10: 0-538-49781-5.

\medskip \noindent \textbf{Lecture slides. } \url{https://piazza.com/umb/spring2014/m141/resources} \quad \quad \quad .

\medskip\noindent Lecture slides will become available as the course progresses.

%\medskip
%\noindent \textbf{Prerequisite. } A standard pre-calculus course or equivalent.


\medskip
\noindent \textbf{Grades.} Your grade will consist of three tests and a final exam, and a weekly quiz. 
\begin{itemize}
\item The quizzes will account for 10\% of your total grade.
\item The tests will account for 60\% (20\% each) of your total grade.
\item The final will account for 30\% of your total grade.
\end{itemize}
Please note that missed tests can not be made up, unless there is a valid medical reason accompanied with an official signed document from a medical doctor. Letter grades will be assigned as follows. 

\begin{center}
\begin{tabular}{cc|cc}
A & 85-100 & C & 65-69 \\
A-& 82-84 & C- & 62-64 \\
B+& 80-81 & D+ & 60-61 \\
B & 75-79& D & 55-59\\
B-& 72-74& D- & 50-54\\
C+& 70-71& F & below 50\\
\end{tabular}

\end{center}

No books, notes, calculators or any other electronic device (such as mobile phones) are allowed during any exam unless otherwise stated.

\medskip
\noindent \textbf{Homework.} You will be assigned weekly homework, which will be posted on

\url{https://piazza.com/umb/spring2014/m141/resources} \quad \quad \quad .

\noindent I will not collect homework, but I will expect you to complete it in a separate notebook/pieces of paper. I will proofread your homework only if you submit it to me.
 
 \medskip
\noindent \textbf{Weekly quizzes} You will be given 1 (or more if announced) weekly quiz. Your quiz problem will be one of your homework problems, verbatim. Quiz dates will be announced in class. 



\medskip
\noindent \textbf{Student conduct.} Students  are required to adhere the University Policy on Academic Standards and Cheating, to the University Statement of Plagiarism and the Documentation of Written Work, and to the Code of Student Conduct as described in the catalog of Undergraduate programs, pages 44-45 and 48-52. The code is available at the following web-page.

\noindent\url{http://www.umb.edu/life_on_campus/policies/code/}

\medskip
\noindent \textbf{Lectures.} 
\renewcommand{\theenumii}{\arabic{enumii}}
\begin{enumerate}
\item Exponents, extended review.
\begin{enumerate}
\item Definition of exponent.
\item Derivatives of exponential functions
\end{enumerate}
\item Inverse functions, review.
\item Logarithmic functions, review.
\begin{enumerate}
\item Definition of logarithm.
\item Natural logarithms.
\item Derivatives of logarithms.
\item The number $e$ as a limit.
\item Derivatives of arbitrary exponents with arbitrary base.
\end{enumerate}
\item Inverse trigonometric functions.
\begin{enumerate}
\item Definition of inverse trigonometric functions.
\item Derivatives of inverse trigonometric functions.
\end{enumerate}
\item Integration, Review
\begin{enumerate}
\item The Fundamental Theorem of Calculus.
\item Differential forms.
\item Integration and logarithms
\end{enumerate}
\item Techniques of integration
\begin{enumerate}
\item Integration by parts.
\item Integration of rational functions.
\begin{enumerate}
\item Building block integrals.
\item Partial fractions.
\end{enumerate}
\item Trigonometric integrals.
\item Integrals of radicals of quadratics
\begin{enumerate}
\item Trigonometric substitutions.
\item Euler substitutions corresponding to trig substitutions.
\end{enumerate}
\end{enumerate}
\item L'Hospital's rule.
\item Improper integrals.
\item Polar coordinates.
\item Curves
\begin{enumerate}
\item Curve images and parametric curves definitions.
\item Curve (arc) length.
\item Curves in polar coordinates.
\item Area locked by curves.
\end{enumerate}
\item Differential equations
\begin{enumerate}
\item Direction fields.
\item Separable equations.
\item The logistic (population growth) equation.
\end{enumerate}
\item Sequences.
\item Series.
\begin{enumerate}
\item Geometric and arithmetic sums.
\item Comparison test.
\item Integral test.
\item Absolute convergence, alternating series.
\item Ratio and root tests.
\end{enumerate}
\item Power series
\begin{enumerate}
\item Radius and interval of convergence.
\item Maclaurin and Taylor series.
\item Integrating and differentiating Maclaurin and Taylor series. 
\item Maclaurin series of $\ln(1+x), e^x, \sin x, \cos x, \arctan x, \arcsin x$.
\end{enumerate}
\end{enumerate}
\end{document}