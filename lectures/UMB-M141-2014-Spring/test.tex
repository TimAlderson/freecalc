\documentclass%
%[handout]
{beamer}
% % % % % % % %
% % % % % % % %
% % % % % % % %
%IMPORTANT
%compiles with 
%pdflatex -shell-escape 
%IMPORTANT
% % % % % % % %
% % % % % % % %
% % % % % % % %
\mode<presentation>
{
\useinnertheme{rounded}
\useoutertheme{infolines}
\usecolortheme{orchid}
\usecolortheme{whale}
}

\usepackage[english]{babel}
\usepackage[latin1]{inputenc}
\usepackage[all,cmtip]{xy}
\usepackage{times}
\usepackage[T1]{fontenc}
\usepackage{../example-templates}
\usepackage{../pstricks-commands}

\usepackage{auto-pst-pdf}
\usepackage{pst-plot}
%\usepackage{pstricks-add} 

% Or whatever. Note that the encoding and the font should match. If T1
% does not look nice, try deleting the line with the fontenc.

\graphicspath{{../../modules/}}

\newtheoremstyle{partialproof}{3pt}{3pt}{}{}{}{.}{.5em}{}
\theoremstyle{partialproof} \newtheorem{partialproof}[theorem]{Proof.}
%\DeclareMathOperator{\diff}{d}
\setbeamertemplate{navigation symbols}{}

\includeonlylecture{1}

\newcommand{\lect}[3]{
  \date{#1}
  \lecture[#1]{#2}{#3}
}

\setbeamertemplate{footline}
{
  \leavevmode%
  \hbox{%
  \begin{beamercolorbox}[wd=.333333\paperwidth,ht=2.25ex,dp=1ex,center]{author in head/foot}%
    \usebeamerfont{author in head/foot}\insertshortauthor
  \end{beamercolorbox}%
  \begin{beamercolorbox}[wd=.333333\paperwidth,ht=2.25ex,dp=1ex,center]{title in head/foot}%
    \usebeamerfont{title in head/foot}\insertshorttitle
  \end{beamercolorbox}%
  \begin{beamercolorbox}[wd=.333333\paperwidth,ht=2.25ex,dp=1ex,center]{date in head/foot}%
    \usebeamerfont{date in head/foot}\insertshortdate{}
  \end{beamercolorbox}}%
  \vskip0pt%
}

% If you have a file called "university-logo-filename.xxx", where xxx
% is a graphic format that can be processed by latex or pdflatex,
% resp., then you can add a logo as follows:

%\pgfdeclareimage[height=0.8cm]{logo}{bluelogo}
%\logo{\pgfuseimage{logo}}
\renewcommand{\Arcsin}{\arcsin}
\renewcommand{\Arccos}{\arccos}
\renewcommand{\Arccot}{\arccot}
\renewcommand{\Arctan}{\arctan}


\begin{document}

\AtBeginLecture{%

\title[\insertlecture]{FreeCalc}
\subtitle{\insertlecture}
\author[FreeCalc]{}
\institute[UMass Boston]{University of Massachusetts Boston}
\date{\insertshortlecture}
\begin{frame}
  \titlepage
\end{frame}
}%

% begin lecture
\lect{\today}{Sample}{1}
%begin module partial-fractions-building-blocks-intro
\begin{frame}
\frametitle{Integrating arbitrary rational functions}
Let $\frac{P(x)}{Q(x)}$ be an arbitrary rational function, i.e., a quotient of polynomials.
\begin{question}
Can we integrate $\displaystyle\int \frac{P(x)}{Q(x)}dx$?
\end{question}
\begin{itemize}
\item Yes. We will now proceed to learn how.
\item The algorithm for integration is roughly:
\begin{itemize}
\item We use algebra to split $\frac{P(x)}{Q(x)}$ into smaller pieces (``partial fractions''). 
\item We use linear substitutions to transform each piece to one of 4 basic building block integrals.
\item We solve each building block integral and collect the terms.
\end{itemize}
\item We study the algorithm ``from the ground up'': we start with the building blocks.
\end{itemize}
\end{frame}

%end module partial-fractions-building-blocks-intro
%begin module partial-fractions-building-blocks-3-and-4-intro
\begin{frame}
\frametitle{The building blocks}
Let $n$ be a positive integer.
\begin{itemize}
\item (Building block I) The first building block integral is:  

$\displaystyle \int \frac{1}{x^n }\diff x\quad .$
\item<2-> (Building block II) The second building block integral is: 

$\displaystyle \int \frac{\alert<4>{x}}{(1+\alert<3>{x^2})^n }\alert<4>{\diff x}\quad .$ \quad \uncover<3->{ (Note: $\alert<3>{u=x^2}, \alert<4>{xdx=\frac{1}{2}\diff u}$ transforms II to I).}
\item<5-> (Building block III) The third building block integral is: 

$\displaystyle \int \frac{1}{(1+x^2)^n }\diff x\quad .$
\item<6-> The case $n=1$ is special for each of the building blocks: 

$\displaystyle \int \frac{1}{x}\diff x$, $\displaystyle \int \frac{x}{1+x^2 }\diff x$ and $\displaystyle \int \frac{1}{1+x^2 }\diff x$.
\item<7-> The case $n=1$ we call respectively building block Ia, IIa and IIIa. 
\uncover<8-> {The case $n>1$ we call respectively building block Ib, IIb and IIIb.} \uncover<9->{ This ``building block'' terminology serves our convenience, and is not a part of standard mathematical terminology. }
\end{itemize}

\end{frame}
%end module partial-fractions-building-blocks-3-and-4-intro
%begin module partial-fractions-building-block-1a
\begin{frame}
\frametitle{Building block Ia}
Building block Ia: $\displaystyle \int \frac{1}{x }\diff x$. 
\begin{example} Integrate building block Ia
\[
\int \frac{1}{x }\diff x \uncover<2->{\alert<2,3>{ =}}\uncover<3->{\alert<3>{\ln | x | +C} }
\]
\end{example}
\end{frame}
\begin{frame}
\frametitle{Linear substitutions leading to building block Ia}
Building block Ia: $\displaystyle \int \frac{1}{x }\diff x=\ln |x|+ C$. 
\begin{example} Integrate 
\[
\begin{array}{rcll|l}
\displaystyle \alert<9>{\int \frac{1}{-4x+5 }\diff x} \uncover<2->{&=&\displaystyle \int \frac{1}{(-4x+5) }\frac{\alert<3>{ \diff (\alert<2>{-4} x)}}{ (\alert<2>{-4})}} \\
\uncover<3->{&=&\displaystyle \int \frac{1}{\alert<4>{(-4x+5)} } \frac{\alert<3>{\diff (\alert<4>{-4x+5})}}{ (-4)} \uncover<4->{&&\text{Set } \alert<4,8>{u=-4x+5}}}\\
\uncover<4->{&=&\displaystyle \int \alert<6>{\frac{1}{\alert<4>{u}}}\frac{\diff \alert<4>{u}}{(\alert<5>{-4})}}\\
\uncover<5->{&=&\displaystyle \alert<5>{-\frac{1}{4}} \alert<7>{\int \alert<6>{u^{-1}} \diff u} }\uncover<7->{=-\frac{1}{4}\alert<7>{\ln |\alert<8>{u}|}+C}\\
\uncover<8->{&=&\displaystyle \alert<9>{-\frac{1}{4}\ln |\alert<8>{-4x+5}|  +C}}\quad .
\end{array}
\]

\end{example}
\end{frame}
\begin{frame}
\frametitle{Lin. subst. leading to building block Ia: general case}
Building block Ia: $\displaystyle \int \frac{1}{x }\diff x=\ln |x|+ C$. 
\begin{example} Integrate 
\[
\begin{array}{rcll|l}
\displaystyle \alert<1>{ \int \frac{1}{ax+b }\diff x}&=&\displaystyle \int \frac{1}{(ax+b) }\frac{\diff (a x)}{a} \\
&=&\displaystyle \int \frac{1}{(ax+b) }\frac{\diff (ax+b)}{a} &&\text{Set }u=ax+b\\
&=&\displaystyle \int\frac{1}{u}\frac{\diff u}{a}\\
&=&\displaystyle \frac{1}{a}\int u^{-1} \diff u =\frac{1}{a}\ln |u|+C\\
&=&\displaystyle \alert<1>{\frac{1}{a}\ln |ax+b|  +C}\quad .
\end{array}
\]

\end{example}
\end{frame}

%end module partial-fractions-building-block-1a



%begin module partial-fractions-building-block-1b
\begin{frame}
\frametitle{Building block Ib}
Building block Ib: $\displaystyle \int \frac{1}{x^n }\diff x=\int x^{-n}\diff x$, $n\neq 1$. 
\begin{example} Integrate the building block integral Ib
\[
\int \frac{1}{x^n }\diff x\quad , n\neq 1.
\]

\[
\int \frac{1}{x^n}\diff x \uncover<2->{\alert<2,3>{= \int x^{-n}\diff x}}\uncover<3->{\alert<3>{=}} \uncover<4->{\alert<4>{\frac{x^{-n+1}}{-n+1} +C}}
\]
\end{example}
\end{frame}
\begin{frame}
\frametitle{Linear substitutions leading to building block Ib}
Building block Ib: $\displaystyle \int \frac{1}{x^n }\diff x=\int x^{-n}\diff x= \frac{x^{-n+1}}{-n+1} +C$, $n\neq 1$. 
\begin{example} Integrate 
\[
\begin{array}{rcll|l}
\alert<8>{\displaystyle \int \frac{1}{(3x+5)^3 }\diff x} \uncover<2->{&=&\displaystyle \int \frac{1}{(3x+5)^3 }\frac{\alert<3>{ \diff (\alert<2>{3} x)}}{\alert<2>{3}}} \\
\uncover<3->{&=&\displaystyle \int \frac{1}{(\alert<4>{3x+5})^3 }\frac{\alert<3>{\diff ( \alert<4>{3 x+5})}}{3}} \uncover<4->{&&\text{Set }\alert<4,7>{ u=3x+5}{~~~~~~~~~~~~~~~~~~~} } \\
\uncover<4->{&=&\displaystyle \int\alert<5>{ \frac{1}{{\alert<4>{u}}^3} } \frac{\diff \alert<4>{u}}{3}}\\
\uncover<5->{ &=&\displaystyle \frac{1}{3} \alert<6>{\int \alert<5>{ u^{-3}} \diff u}} \uncover<6->{ =\frac{1}{3} \alert<6>{ \frac{{\alert<7>{u}}^{-2}}{(-2)}}+C}\\
\uncover<7->{&=&\displaystyle \alert<8>{-\frac{1}{6(\alert<7>{3x+5})^2}+C}\quad .}
\end{array}
\]

\end{example}
\end{frame}
\begin{frame}
\frametitle{Lin. subst. leading to building block Ib: general case}
Building block Ib: $\displaystyle \int \frac{1}{x^n }\diff x=\int x^{-n}\diff x= \frac{x^{-n+1}}{-n+1} +C$, $n\neq 1$. 
\begin{example} Let $n\neq 1$. Integrate 
\[
\begin{array}{rcll|l}
\displaystyle \alert<1>{\int \frac{1}{(ax+b)^n }\diff x} &=&\displaystyle \int \frac{1}{(ax+b)^n }\frac{\diff (a x)}{a} \\
&=&\displaystyle \int \frac{1}{(ax+b)^n }\frac{\diff (a x+b)}{a} &&\text{Set }u=ax+b\\
&=&\displaystyle \int\frac{1}{u^3}\frac{\diff u}{a}\\
&=&\displaystyle \frac{1}{a}\int u^{-n} \diff u =-\frac{1}{a} \frac{u^{-n+1}}{(n-1)}+C\\
&=&\displaystyle \alert<1>{-\frac{1}{ a(n-1)(ax+b)^{n-1}}+C}\quad .
\end{array}
\]

\end{example}
\end{frame}

%end module partial-fractions-building-block-1b
%begin module partial-fractions-building-blocks-2a-and-3a
%This module may be too long, perhaps a split is needed.


%\begin{comment}






%\end{comment}
%end module partial-fractions-building-blocks-2a-and-3a
%begin module Building block IIb

\begin{frame}
\frametitle{Building blocks IIa and IIb}
We solve building block IIb. For completeness, we solve block IIa again as well.
\begin{example}
\[
\begin{array}{rcl}
\displaystyle \int \frac{\alertNoH{2}{ x} }{(x^2+1)^n} \alertNoH{2}{\diff x} \uncover<2->{&=&\displaystyle  \int \frac{1}{(\alertNoH{3}{x^2+1})^n} \alertNoH{2}{\frac{\diff \left(\alertNoH{3}{x^2+1} \right )}{2}}} \\
\uncover<3->{&=& \displaystyle \alertNoH{4-7}{ \frac{1}{2}\int {\alertNoH{3}{u}}^{-n}\diff \alertNoH{3}{ u}}} \\
\uncover<4->{&=&\left\{\begin{array}{ll}\displaystyle
\fcAnswer{5}{ \frac{1}{2}\ln (x^2+1) +C} & \alertNoH{4,5}{\text{if }n=1} \\ \fcAnswer{7}{\displaystyle \frac{1}{2} \frac{(x^2+1)^{-n+1}}{(-n+1)}+C} &\alertNoH{6}{ \text{if }n\neq 1}
\end{array}
\right. ,}
\end{array}
\]
\uncover<3->{where we used the substitution $\alertNoH{3}{u=x^2+1}$.}

\end{example}

\end{frame}

%end module Building block IIb

%begin module Building block IIIb

\begin{frame}
\frametitle{Building block IIIb: example illustrating main idea}
\begin{example}
Integrate $\int \frac{\diff x}{(x^2+1)^2}$. We start with an already known integral:
\[
\begin{array}{rcl}
\uncover<2->{\alert<12,13>{\alert<15>{\Arctan x}+C}} 
\only<1-12>{
\uncover<2->{
&=&\displaystyle \int \alert<3>{\frac{1}{x^2+1}}\diff \alert<4>{ x}}\\
\uncover<3->{&=&\displaystyle \alert<3>{\frac{1}{x^2+1}} \alert<4>{ x}-\int \alert<4>{x} \alert<5,6>{ \diff \left(\alert<3>{\frac{1}{x^2+1}}\right)} }\\
\uncover<5->{&=&\displaystyle \frac{x}{x^2+1} \alert<7>{-} \int \alert<7>{ x} \only<5>{\alert<5>{\textbf{?}}} \uncover<6->{\alert<6>{ \left( \alert<7>{-} \frac{\alert<7>{ 2x} }{ (x^2+1)^2}\right)\diff x}}}\\
\uncover<7->{&=&\displaystyle \frac{x}{x^2+1}\alert<7>{ +2} \int \frac{ \uncover<8->{\alert<8>{-\alert<10>{1}}+}\alert<9>{ \alert<7>{ x^2} \uncover<8->{\alert<8>{+1}}}}{\alert<9,10>{(x^2+1)^2}}\diff x }\\
\uncover<9->{&=&\displaystyle  \frac{x}{x^2+1}+2\alert<11>{\int \alert<9>{\frac{1}{x^2+1}} \diff x}-2\int \alert<10>{\frac{1}{(x^2+1)^2}}\diff x}\\
} %only<1-12>
\uncover<11->{ &\alert<12,13>{=}&\displaystyle \alert<12,13>{ \frac{x}{x^2+1}+ \alert<15>{2\alert<11>{\Arctan x} }\alert<14>{ -2 \int \frac{\diff x}{(x^2+1)^2}}} {~~~~~~~~~~~~~~~}} 
\end{array}
\]
\only<13->{
\uncover<14->{Rearrange terms \uncover<16->{and divide by $2$ to get the desired integral:}
\[
\alert<14>{\uncover<14,15>{2} \int \frac{\diff x}{(1+x^2)^2}}=\uncover<16->{\frac{1}{2}} \left(\frac{x}{x^2+1}+ \alert<15>{\Arctan x}  \right)+\uncover<14,15>{L}\uncover<16->{K}\quad .
\]
}%uncover14
}%uncover13
\end{example}
\vspace{8cm}
\end{frame}

\begin{frame}
\frametitle{Building block IIIb}
\begin{itemize}
\item<1-> Building block IIIa: 
\[
\uncover<6->{\alert<6>{J(1)=}} \int \frac{1}{(x^2+1)}\diff x=\alert<6>{\arctan x+C}\quad .
\] 
\item<2-> Block IIIb:
\[
\uncover<4->{\alert<4>{J(n)=}} \alert<4>{\int \frac{1}{(x^2+1)^n}\diff x}
\] 
\item<3-> Unlike other cases, IIIb is much harder than IIIa.
\item<4-> Set $\alert<4>{J(n)=\int \frac{1}{(x^2+1)^n}\diff x}$. \uncover<5->{We are looking for a formula for $J(n)$.} \uncover<6->{We know $\alert<6>{J(1)=\arctan x+C}$ (this is block IIIa).}
\item<7-> We start by $J(n-1) =\int \frac{1}{(x^2+1)^{n-1}} \diff x$ and integrate by parts.
\item<8-> In this way we end up expressing $J(n)$ via $J(n-1)$.
\item<9-> We work our way from $J(n)$ to $J(n-1)$, from $J(n-1)$ to $J(n-2)$, and so on, until we get to $J(1)$.
\end{itemize} 
\end{frame}

\begin{frame}
\begin{example}
Recall that $\alert<11,12>{J(n)=\int \frac{1}{(x^2+1)^{n}}\diff x}$. %\uncover<3->{Set $\alert<3>{u=\frac{1}{(1+x^2)^{n-1}}}$.} 
\uncover<2->{We have that:}
\[
\begin{array}{rcl}
\uncover<2->{\alert<13,14,16>{J(n-1)}}
\only<1-13>{\uncover<2->{&\alert<13>{=} & 
\displaystyle \int \alert<3>{\frac{1}{(x^2+1)^{n-1 }}} \diff \alert<4>{x} } \\
\uncover<3->{&=&\displaystyle  \alert<3>{\frac{1}{(x^2+1)^{n-1}} } \alert<4>{x}-\int  \alert<4>{x} \alert<5,5>{ \diff \left(\alert<3>{ \frac{1}{ (1+x^2)^{ n-1}}}\right)}}\\
\uncover<5->{&=&\displaystyle  \frac{x}{(x^2+1)^{n-1}} \alert<7>{-} \int \alert<7>{ x}  \only<5>{\alert<5>{\textbf{?}}} \uncover<6->{\alert<6>{\frac{ \alert<7>{(-n+1) 2 x}}{ (1+x^2 )^{n}} \diff x}}} \\
\uncover<7->{ &=&\displaystyle  \frac{x}{(x^2+1)^{n-1}} \alert<7>{+ 2(n-1)} \int \frac{\uncover<8->{ \alert<8,9>{1+}} \alert<7,9>{x^2} \uncover<8->{\alert<8,10>{-1}}}{ \alert<9,10>{ (1+x^2)^n} }\diff x}\\
\uncover<9->{ &=&\displaystyle  \frac{x}{(x^2+1)^{n-1}}+ 2(n-1)\alert<11>{ \int \alert<9>{\frac{1}{(1+x^2)^{n-1}}} \diff x}} \\
\uncover<9->{&&\displaystyle \alert<10>{-} 2(n-1)\alert<12>{ \int \alert<10>{\frac{1}{(1+x^2)^n}}\diff x} }\\
} %only<1-13>
\uncover<11->{&\alert<13,14>{=}&\displaystyle \alert<13,14>{ \frac{x}{(x^2+1)^{n-1}}+\alert<16>{ 2(n-1)\alert<11>{ J(n-1)}}  \alert<15>{ -2(n-1)\alert<12>{J(n)}}}\quad .}
\end{array}
\]

\only<14->{
\uncover<15->{
Rearrange to get:
\[
\begin{array}{rcl}
\alert<15>{ \alert<17>{2(n-1)}J(n) } &=& \displaystyle \frac{x}{(x^2+1)^{n-1}}+\alert<16>{(2n-3) J(n-1)} \\
\uncover<17->{ \alert<19,20,21>{J(n)}&\alert<19,20,21>{=}&\displaystyle \alert<19,20,21>{ \frac{x}{ \alert<17>{(2n-2)} (x^2+ 1)^{ n-1}}+ \frac{2n-3}{\alert<17>{ 2n-2}}J(n-1)} \quad .}
\end{array}
\]

\uncover<18->{In this way we expressed $J(n)$ using $J(n-1)$.} \uncover<19->{We apply the above formula consecutively:

$
\alert<19>{ J(n)=  \frac{x}{ (2n-2) (x^2+ 1)^{ n-1}}+   \frac{2n-3}{ 2n-2}\only<19,20>{\alert<20>{J(n-1)}  \phantom{\left(\frac{x}{(2n-4)(x^2+1)^{n-2}}\right)} 
}} \only<21->{\alert<21>{\left(\frac{x}{(2n-4)(x^2+1)^{n-2}}+\frac{2n-5 }{2n-4} \alert<22,23>{ J(n-2)} \right)}}  \uncover<22->{\alert<22>{=\dots}}
$
}

\noindent \uncover<22->{\alert<22>{and so on.}} \uncover<23->{A formula for the final result can be written using the above (found in Calculus for beginners, Chapter ``Techniques of integration'').}

} %uncover15
} %uncover14
\end{example}


\vspace{8cm}
\end{frame}
%end module Building block IIIb

\begin{frame}
\frametitle{Building block integral summary}
\begin{tabular}{|c|c|c|c|c|}\hline
Type & a & b & Type a, lin. sub. & Type b, lin. sub \\\hline
I &   $\only<2>{\color{red}} \int \frac{1}{x}\diff x $  &$\only<2>{\color{red}}\int \frac{1}{x^n}\diff x$ & $\only<5>{ \color{red}}\int \frac{A}{ax+b}\diff x $ &  $\only<5>{ \color{red}}\int \frac{A}{(a x+ b)^n}\diff x $\\\hline
II & $\only<2>{\color{red}}\int \frac{x}{x^2+1}\diff x\only<3->{\color{black}}$ &$\only<2>{\color{red}}\int \frac{x}{\left(x^2+1\right)^n}\diff x\only<3->{ \color{black}} $ & $\only<6>{ \color{red}}\int \frac{A\left(x+ \frac{b}{2a} \right)}{\alertNoH{7}{ax^2+bx+c}}\diff x $ &  $\only<9->{\color{gray}} \int \frac{A\left(x+ \frac{b}{2a} \right)}{\left(\alertNoH{8}{ ax^2 +bx+c}\right)^n}\diff x  $\\\hline
III & $\only<2>{\color{red}}\int \frac{1}{x^2+1}\diff x\only<3->{\color{black}}$ &$\only<2,3>{\color{red}}\int \frac{1}{\left(x^2+1\right)^n}\diff x$ & $\only<6>{ \color{red}}\int \frac{B}{\alertNoH{7}{ax^2+bx+c}}\diff x $ &  $\only<9->{\color{gray}}\int \frac{B}{\left( \alertNoH{8}{a x^2+bx+ c}\right)^n}\diff x $\\\hline
\end{tabular}
where $A,B$ are arbitrary constants and $a,b,c$ are constants with \alertNoH{6,7}{$b^2-4ac<0$}. The quadratics in the denominators have no real roots.


\begin{itemize}
\item<2-> We solved building blocks I, II and III \uncover<3->{\alertNoH{3}{in almost complete detail.}}
\item<4-> The types in the remaining columns can be transformed to building block ones:
\begin{itemize}
\item<5-> Block I, linear substitutions: done in full detail.
\item<6-> Block IIa, IIIa, linear substitutions: done in full detail, \uncover<7->{\alertNoH{7}{by means of completing the square.}}
\item<8-> Block IIb, IIIb, linear substitutions: \alertNoH{8}{done by means of completing the square;} \uncover<9->{\alertNoH{9}{computations are analogous and we leave them for exercise.}}
\end{itemize}
\end{itemize}


\end{frame}

% begin module from-building-blocks-to-complete-algorithm-intro
\begin{frame}
\frametitle{From building blocks to all rational functions: example}
\begin{itemize}
\item We know how to solve $\displaystyle \int \frac{2}{x-1}\diff x$ and $\displaystyle \int \frac{1}{x+2}\diff x$.
\item Consider the difference
\[
\alertNoH{5}{\frac{2}{x-1} - \frac{1}{x+2} } = %
\uncover<2->{%
\frac{2(x+2) - (x-1)}{(x-1)(x+2)} = %
}%
\uncover<3->{%
\alertNoH{5}{ \frac{x + 5}{x^2+x-2} }\quad .
}%
\]
\item<4->

We can now solve the following integral:
\[
\int \alertNoH{5}{ \frac{x+5}{x^2+x-2}}\diff x = %
\uncover<5->{%
\int \left(\alertNoH{5}{\frac{2}{x-1} - \frac{1}{x+2}} \right) \diff x = %
}%
\uncover<6->{%
2\ln | x - 1| - \ln | x + 2| + C
}%
\]
\item<7-> From  (linear substitutions of) basic building blocks we constructed a larger example, which we can therefore solve.
\item<8-> We now learn how to do the reverse procedure: given a rational function, split it into ``partial fractions''.
\end{itemize}
\end{frame}
% end module from-building-blocks-to-complete-algorithm-intro

\begin{frame}
\frametitle{Partial fractions definition}
\begin{definition} 
A partial fraction is rational function of the form of one of the 2 forms below.
\begin{itemize}
\item $\frac{A}{(ax+b)^n} $, $n\geq 1$.
\item $\frac{Ax+B}{ax^{2}+bx+c}$, where $b^2-4ac<0$ and $n\geq 1$.
\end{itemize}
\end{definition}
\uncover<2->{
\begin{theorem}
Every rational function can be written as a sum of a polynomial and partial fractions.
\end{theorem}
}
\begin{itemize}
\item<3-> We already learned know how to integrate all partial fractions (using linear substitutions and building blocks I, II and III). 
\item<4-> Thus, if we can produce the partial fractions whose existence is promised by the theorem, we can integrate all rational functions.
\end{itemize}
\end{frame}

% begin module partial-fractions-long-division
\begin{frame}
\frametitle{Review of polynomial notation}
\begin{itemize}
\item Recall that a rational function is a function of the form
\[
f(x) = \frac{P(x)}{Q(x)}
\]
where $P$ and $Q\neq 0$ are polynomials. 
\item<2->Recall that the degree of $P$ is the highest power of $x$ in $P$ that has a non-zero coefficient.
\end{itemize}
\end{frame}

\begin{frame}\frametitle{Ensure denominator degree $>$ numerator degree}
\begin{itemize}
\item To decompose $\frac{P(x)}{Q(x)}$ in partial fractions we ensure first the degree of the numerator is smaller than the degree of the denominator.
\item<2-> We recall that to divide the \alertNoH{4}{dividend $P(x)$} by the \alertNoH{5}{divisor $Q(x)$} to get \alertNoH{6}{quotient $S(x)$} with \alertNoH{7}{remainder $R(x)$} means \uncover<3->{ to find polynomials  $S(x), R(x)$ such that \alertNoH{7}{$\deg R<\deg Q$} and
\[
\begin{array}{rcll|l}
\displaystyle \alertNoH{4} {P(x)}& =&\displaystyle  \alertNoH{6}{S(x)} \alertNoH{5}{Q(x)} + \alertNoH{7}{ R(x)} \uncover<8->{&&\alertNoH{8}{ \text{divide by } Q(x)}}\\
\displaystyle \uncover<8->{\frac{P(x)}{\alertNoH{8}{Q(x)} }&=&\displaystyle  \frac{S(x)\fcCancel{9}{Q(x)} }{\fcCancel{9}{\alertNoH{8}{Q(x)}}} +\frac{R(x)}{ \alertNoH{8}{ Q(x)}}} \\
\uncover<9->{
\displaystyle \frac{P(x)}{Q(x) }&=&\displaystyle  S(x) +\frac{R(x)}{ Q(x)}} \\
\end{array}
\]
}
\item<10-> The above transforms $\frac{P(x)}{Q(x)}$ to a polynomial plus a fraction in which the numerator has degree smaller than the denominator.
\item<11-> The polynomials $Q(x)$ and $S(x)$ are computed via polynomial long division. We recall the procedure through examples.
\end{itemize}
\end{frame}
% end module partial-fractions-long-division

% begin module partial-fractions-long-division-ex1
\begin{frame}
\begin{example} %[Example 1, p. 510]
Find $\int \frac{x^3 + x}{x - 1}\diff x$.
\begin{columns}[t]
\column{.45\textwidth}
\uncover<2->{%
\[
\begin{array}{r@{}r@{}c@{}r@{}c@{}c@{}c@{}r@{}r@{}}
& & & %
\uncover<4->{\alert<handout:0| 4-6,23>{x^2}} & %
\uncover<11->{\alert<handout:0| 11-13,23>{+}} & %
\uncover<11->{\alert<handout:0| 11-13,23>{ x}} & %
\uncover<17->{\alert<handout:0| 17-19,23>{+}} & %
\uncover<17->{\alert<handout:0| 17-19,23>{ 2}} & \\%
\cline{3-8}
\alert<handout:0| 3-6,10-13,16-19>{x} & %
\alert<handout:0| 5-6,12-13,18-19>{-1} & %
\Big) & %
\alert<handout:0| 3-4,7-8>{x^3} & %
 & %
 & %
\alert<handout:0| 9>{+} & %
\alert<handout:0| 9>{ x} & \\%
& & & %
\uncover<6->{\alert<handout:0| 6-8>{x^3}} & %
\uncover<6->{\alert<handout:0| 6-8>{-}} & %
\uncover<6->{\alert<handout:0| 6-8>{ x^2}} & %
&  & \\%
\cline{4-6}%
& & & %
&  & %
\uncover<8->{\alert<handout:0| 8,10-11,14-15>{x^2}} & %
\uncover<9->{\alert<handout:0| 9,14-15>{+}} & %
\uncover<9->{\alert<handout:0| 9,14-15>{ x}} & \\%
& & & %
&  & %
\uncover<13->{\alert<handout:0| 13-15>{x^2}} & %
\uncover<13->{\alert<handout:0| 13-15>{-}} & %
\uncover<13->{\alert<handout:0| 13-15>{ x}} & \\%
\cline{6-8}
& & & %
 & %
 & %
 & %
 & %
\uncover<15->{\alert<handout:0| 15-17,20-21>{2x}} & \\%
& & & %
 & %
 & %
 & %
 & %
\uncover<19->{\alert<handout:0| 19-21>{2x}} & %
\uncover<19->{\alert<handout:0| 19-21>{- 2}} \\%
\cline{8-9}%
& & & %
 & %
 & %
 & %
 & %
 & %
\uncover<21->{\alert<handout:0| 21,24>{2}} \\%
\end{array}
\]
}%

\only<handout:0| -4,10-11,16-17>{\uncover<3->{%
Divide %
}}%
\only<handout:0| 5-6,12-13,18-19>{%
Multiply %
}%
\only<handout:0| 7-8,14-15,20-21>{%
Subtract %
}%
\only<handout:0| 3-4>{%
$x^3$ %
}%
\only<handout:0| 10-11>{%
$x^2$ %
}%
\only<handout:0| 16-17>{%
$2x$ %
}%
\only<handout:0| 5-6>{%
$x^2$ %
}%
\only<handout:0| 12-13>{%
$x$ %
}%
\only<handout:0| 18-19>{%
$2$ %
}%
\only<handout:0| 7-8>{%
$x^3-x^2$ %
}%
\only<handout:0| 14-15>{%
$x^2-x$ %
}%
\only<handout:0| 20-21>{%
$2x-2$ %
}%
\only<handout:0| -4,10-11,16-17>{\uncover<3->{%
by %
}}%
\only<handout:0| 5-6,12-13,18-19>{%
by %
}%
\only<handout:0| 7-8,14-15,20-21>{%
from %
}%
\only<handout:0| 3-4,10-11,16-17>{%
$x$ %
}%
\only<handout:0| 5-6,12-13,18-19>{%
$x-1$ %
}%
\only<handout:0| 7-8>{%
$x^3$ %
}%
\only<handout:0| 14-15>{%
$x^2+x$ %
}%
\only<handout:0| 20-21>{%
$2x$ %
}%
\only<handout:0| 9>{%
Bring down the $x$%
}%
\invisible<1->{%
y%
}%
\column{.55\textwidth}
\begin{eqnarray*}
& & %
\uncover<22->{%
\int \frac{x^3 + x}{x - 1}\diff x %
}\\%
& \uncover<22->{ = } & %
\uncover<22->{%
\int \left( \alert<handout:0| 23>{x^2 + x + 2} + \frac{\alert<handout:0| 24>{2}}{x - 1}\right) \diff x
}\\%
& \uncover<25->{ = } & %
\uncover<25->{%
\frac{x^3}{3} + \frac{x^2}{2} + 2x %
}\\%
& & \uncover<25->{%
\qquad + 2\ln | x - 1 | + C %
}%
\end{eqnarray*}
\end{columns}
\end{example}
\end{frame}
% end module partial-fractions-long-division-ex1


\begin{frame}
\begin{itemize}
\item The next step in producing a partial fraction decomposition is to factor the denominator $Q(x)$.
\item Factoring of $Q(x)$ can always be done in quadratic and linear terms: 
\begin{corollary} [Corollary to the Fundamental Theorem of Algebra]
Let $Q(x)$ be a polynomial (with real coefficients). Then $Q(x)$ can be factored as a product of terms of the form $(ax+b)^n$ (powers of linear terms) and product of terms of the form $(ax^2+bx+c)^n$ with $b^2-4ac<0$ (powers of quadratic terms).
\end{corollary}  
\item The above result is a corollary to the Fundamental Theorem of Algebra. We state the Fundamental Theorem of algebra without proving it.
\begin{theorem}[The Fundamental Theorem of Algebra]
Every polynomial has at least one complex root.
\end{theorem}
\end{itemize}
\end{frame}

\begin{frame}
Suppose we have already factored the denominator $Q(x)$ into factors of the form 
\[
(ax+b)^N\qquad \text{ and }\qquad (ax^2+bx+c)^N
\]
\uncover<2->{Then we can split the fraction $R(x)/Q(x)$ into sum of partial fractions of the form 
\[
\frac{A}{(ax+b)^i} \qquad \text{or}\qquad \frac{Ax+B}{(ax^2+bx+c)^i}\quad ,
\]
where the exponent $i$ in the partial fraction does not exceed the exponent $N$ of the corresponding term in $Q(x)$.
}

\uncover<3->{
The cases when the factorization of $Q(x)$ has terms appearing with power $N>1$ are treated differently from the the cases where all terms of the factorization of $Q(x)$ are distinct.
}
\end{frame}



% begin module partial-fractions-cases
\begin{frame}
The second step is to factor the denominator $Q(x)$ as far as possible.  In fact, any polynomial $Q$ can be factored as a product of linear factors (of the form $ax + b$) and irreducible quadratic factors (of the form $ax^2 + bx + c$, where $b^2 - 4ac < 0$).

\uncover<2->{%
The third step is to express the rational function $R(x)/Q(x)$ as a sum of partial fractions of the form
\[
\frac{A}{(ax+b)^i} \qquad \textrm{or}\qquad \frac{Ax+B}{(ax^2+bx+c)^j}
\]
This is always possible to do.
}%

\uncover<3->{%
There are four cases that occur:
\begin{enumerate}
\item  $Q(x)$ is a product of distinct linear factors.
\item  $Q(x)$ is a product of linear factors, some of which are repeated.
\item  $Q(x)$ contains irreducible quadratic factors, none of which is repeated.
\item  $Q(x)$ contains irreducible quadratic factors, some of which are repeated.
\end{enumerate}
}%
\end{frame}
% end module partial-fractions-cases

% begin module partial-fractions-case1
\begin{frame}
Suppose $Q(x)$ is a product of distinct linear factors.

This means we can write
\[
Q(x) = (a_1x+b_1)(a_2x+b_2) \cdots (a_kx+b_k)
\]
where no factor is repeated (and no factor is a constant multiple of another).

Then there exist constants $A_1, A_2, \ldots , A_k$ such that
\[
\frac{R(x)}{Q(x)} = \frac{A_1}{a_1x+b_1} + \frac{A_2}{a_2x+b_2} + \cdots + \frac{A_k}{a_kx+b_k}
\]

The next example shows how to find $A_1, A_2, \cdots , A_k$.
\end{frame}
% end module partial-fractions-case1

% begin module partial-fractions-case1-ex2
\begin{frame}
\begin{example} %[Example 2, p. 511]
Find $\int \frac{x^2+2x-1}{2x^3+3x^2-2x}\diff x$.
\begin{itemize}
\item<2->  deg$(x^2+2x-1) < $ deg$(2x^3+3x^2-2x)$: don't divide.
\item<3->  Factor denominator: $2x^3 + 3x^2-2x = x(2x-1)(x+2)$.
\end{itemize}
\abovedisplayskip=0pt
\belowdisplayskip=0pt
\begin{eqnarray*}
\uncover<4->{%
\frac{x^2+2x-1}{x(2x-1)(x+2)}%
}%
& \uncover<4->{ = } & %
\uncover<4->{%
\frac{\alert<handout:0| 12>{A}}{x} + \frac{\alert<handout:0| 13>{B}}{2x-1} + \frac{\alert<handout:0| 14>{C}}{x+2}%
}\\%
\uncover<5->{%
x^2+2x-1%
}%
& \uncover<5->{ = } & %
\uncover<5->{%
A(2x-1)(x+2) + Bx(x+2) + Cx(2x-1)%
}\\%
\uncover<6->{%
\alert<handout:0| 7>{x^2}+\alert<handout:0| 8>{2}x\alert<handout:0| 9>{-1}%
}%
& \uncover<6->{ = } & %
\uncover<6->{%
\alert<handout:0| 7>{(2A + B + 2C)x^2} + \alert<handout:0| 8>{(3A + 2B - C)x} \alert<handout:0| 9>{- 2A}%
}%
\end{eqnarray*}
\begin{columns}[t]
\column{.4\textwidth}
\abovedisplayskip=0pt
\belowdisplayskip=0pt
\[
\begin{array}{r@{}r@{}r@{}r@{}r@{}c@{}r}
\uncover<7->{\alert<handout:0| 7>{2A}} & %
\uncover<7->{\alert<handout:0| 7>{+}} & %
\uncover<7->{\alert<handout:0| 7>{ B}} & %
\uncover<7->{\alert<handout:0| 7>{+}} & %
\uncover<7->{\alert<handout:0| 7>{2C}} & %
\uncover<7->{\alert<handout:0| 7>{=}} & %
\uncover<7->{\alert<handout:0| 7>{1}} \\ %
\uncover<8->{\alert<handout:0| 8>{3A}} & %
\uncover<8->{\alert<handout:0| 8>{+}} & %
\uncover<8->{\alert<handout:0| 8>{2B}} & %
\uncover<8->{\alert<handout:0| 8>{-}} & %
\uncover<8->{\alert<handout:0| 8>{ C}} & %
\uncover<8->{\alert<handout:0| 8>{=}} & %
\uncover<8->{\alert<handout:0| 8>{2}} \\ %
\uncover<9->{\alert<handout:0| 9>{-2A}} & %
 & %
 & %
 & %
 & %
\uncover<9->{\alert<handout:0| 9>{=}} & %
\uncover<9->{\alert<handout:0| 9>{-1}} \\ %
\end{array}
\]
\uncover<10->{Solution:\\ $\alert<handout:0| 12>{A = \frac{1}{2}}, \alert<handout:0| 13>{B = \frac{1}{5}}, \alert<handout:0| 14>{C = -\frac{1}{10}}$.}
\column{.5\textwidth}
\abovedisplayskip=0pt
\belowdisplayskip=0pt
\[
\begin{array}{c@{}l}
&  \uncover<11->{\int \frac{x^2+2x-1}{2x^3+3x^2-2x}\diff x}\\%
\uncover<11->{ = } & %
\uncover<11->{%
\int \left( \alert<handout:0| 12>{\frac{1}{2}} \ \frac{1}{x} + \alert<handout:0| 13>{\frac{1}{5}} \ \frac{1}{2x-1} \alert<handout:0| 14>{- \frac{1}{10}} \ \frac{1}{x+2}\right) \diff x%
}\\%
\uncover<15->{ = } & %
\uncover<15->{%
\frac{1}{2}\ln |x| + \frac{1}{10}\ln |2x-1|%
}\\%
 & %
\uncover<15->{%
\qquad -\frac{1}{10}\ln |x+2| + K%
}%
\end{array}
\]
\end{columns}
\end{example}
\end{frame}
% end module partial-fractions-case1-ex2

% begin module partial-fractions-case1-quick-trick
\begin{frame}
NOTE:  There is a quick trick to find $A, B$, and $C$.
\abovedisplayskip=2pt
\belowdisplayskip=0pt
\[
x^2 + 2x - 1 = A(\alert<handout:0| 3>{2x-1})(\alert<handout:0| 4>{x+2}) + B\alert<handout:0| 2>{x}(\alert<handout:0| 4>{x+2}) + C\alert<handout:0| 2>{x}(\alert<handout:0| 3>{2x-1})
\]
\uncover<2->{\alert<5-7>{ To find $A$, set \alert<handout:0| 2>{$x=0$};}} \uncover<3->{\alert<8-10>{to find $B$, set \alert<handout:0| 3>{$x=\frac{1}{2}$}; }} \uncover<4->{\alert<11-13>{to find $C$, set \alert<handout:0| 4>{$x=-2$}}.}

\[
\begin{array}{rcl}
\uncover<5->{%
0^2 + 2\cdot 0 - 1%
}%
& \uncover<5->{ = } & %
\uncover<5->{%
A(2\cdot 0 - 1)(0 + 2)%
}\\%
\uncover<6->{%
 - 1%
}%
& \uncover<6->{ = } & %
\uncover<6->{%
-2 A%
}\\%
\uncover<7->{%
 A%
}%
& \uncover<7->{ = } & %
\uncover<7->{%
\frac{1}{2}%
}\\ ~\\ ~\\
\uncover<8->{%
\left( \frac{1}{2}\right)^2 + 2\cdot \frac{1}{2} - 1%
}%
& \uncover<8->{ = } & %
\uncover<8->{%
B\left( \frac{1}{2}\right)\left(\frac{1}{2} + 2\right)%
}\\%
\uncover<9->{%
 \frac{1}{4}%
}%
& \uncover<9->{ = } & %
\uncover<9->{%
\frac{5}{4} B%
}\\%
\uncover<10->{%
 B%
}%
& \uncover<10->{ = } & %
\uncover<10->{%
\frac{1}{5}%
}\\ ~\\ ~\\
\uncover<11->{%
(-2)^2 + 2(-2) - 1%
}%
& \uncover<11->{ = } & %
\uncover<11->{%
C(-2)(2(-2) - 1)%
}\\%
\uncover<12->{%
 - 1%
}%
& \uncover<12->{ = } & %
\uncover<12->{%
10C%
}\\%
\uncover<13->{%
 C%
}%
& \uncover<13->{ = } & %
\uncover<13->{%
-\frac{1}{10}%
}\\%
\end{array}
\]
\end{frame}
% end module partial-fractions-case1-quick-trick

% begin module partial-fractions-case2
\begin{frame}
\frametitle{$Q(x)$ has linear factors with higher multiplicity}
\begin{itemize}
\item Suppose $Q(x)$ is a product of linear factors, some of which appear with power greater than $1$.
\item<2-> For example suppose the first linear factor has power $r$, that is, $(a_1x+b_1)^r$ occurs in the factorization of $Q(x)$.
\item<3-> Then instead of a single term $\frac{A}{a_1x+b_1}$ we use
\[
\frac{A_1}{a_1x+b_1}%
 + \frac{A_2}{(a_1x+b_1)^2}%
 + \cdots %
 + \frac{A_r}{(a_1x+b_1)^r}%
\]
\item<4-> In a similar fashion we add more partial fractions to account for all other terms of the form $(a_sx+b_s)^{t}$.
\end{itemize}
\end{frame}
% end module partial-fractions-case2

% begin module partial-fractions-case2-ex4
\begin{frame}
\begin{example}[Example 4, p. 513]
Find $\int \frac{x^4-2x^2+4x+1}{x^3-x^2-x+1}\diff x$.
\begin{itemize}
\item<2->  Divide: $\frac{x^4-2x^2+4x+1}{x^3-x^2-x+1} = x + 1 + \frac{4x}{x^3-x^2-x+1}$.
\item<3->  Factor denominator: $x^3-x^2-x+1 = (x-1)^2(x+1)$.
\end{itemize}
\abovedisplayskip=0pt
\belowdisplayskip=0pt
\begin{eqnarray*}
\uncover<4->{%
\frac{4x}{(x-1)^2(x+1)}%
}%
& \uncover<4->{ = } & %
\uncover<4->{%
\frac{A}{x-1} + \frac{B}{(x-1)^2} + \frac{C}{x+1}%
}\\%
\uncover<5->{%
4x%
}%
& \uncover<5->{ = } & %
\uncover<5->{%
A(\alert<handout:0| 8>{x-1})(\alert<handout:0| 6>{x+1}) + B(\alert<handout:0| 6>{x+1}) + C(\alert<handout:0| 8>{x-1})^2%
}\\%
\end{eqnarray*}
\vspace{-.4in}
\begin{itemize}
\item<6->  Plug in $-1$: \uncover<7->{$4(-1) = C(-1-1)^2$, therefore $C = -1$.}
\item<8->  Plug in $1$: \uncover<9->{$4(1) = B(1+1)$ therefore $B = 2$.}
\item<10->  Plug in $0$: $4(0) = A(0-1)(0+1) + 2(0+1) + (-1)(0-1)^2$.
\item<11->  Therefore $A = 1$.
\end{itemize}
\abovedisplayskip=0pt
\belowdisplayskip=0pt
\[
\begin{array}{r@{ \ }c@{ \ }l}
\uncover<12->{%
\int \frac{x^4-2x^2+4x+1}{x^3-x^2-x+1}\diff x%
}%
& \uncover<12->{ = } & %
\uncover<12->{%
\int \left( x + 1 + \frac{1}{x-1} + \frac{2}{(x-1)^2} - \frac{1}{x+1}\right) \diff x%
}\\%
& \uncover<13->{ = } & %
\uncover<13->{%
\frac{x^2}{2} + x + \ln |x-1| - \frac{2}{x-1} -\ln |x+1| + K%
}%
\end{array}
\]
\end{example}
\end{frame}
% end module partial-fractions-case2-ex4

% begin module partial-fractions-case3
\begin{frame}
\begin{enumerate}
\setcounter{enumi}{2}
\item  $Q(x)$ contains irreducible quadratic factors, none of which is repeated.
\end{enumerate}
If $Q(x)$ has the factor $ax^2 + bx + c$, where $b^2-4ac < 0$, then, in addition to the partial fractions arising from linear factors, the expression for $R(x)/Q(x)$ will have a term of the form
\[
\frac{Ax+B}{ax^2+bx+c}
\]

This term can be integrated by completing the square and using the formula
\[
\int \frac{\diff x}{x^2+a^2} = \frac{1}{a} \tan^{-1} \left( \frac{x}{a}\right) + C
\]
\end{frame}
% end module partial-fractions-case3

% begin module partial-fractions-case3-ex5
\begin{frame}
\begin{example}[Example 5, p. 514]
Find $\int \frac{2x^2-x+4}{x^3+4x}\diff x$.
\begin{itemize}
\item<2->  deg$(2x^2-x+4) < $ deg$(x^3+4x)$: don't divide.
\item<3->  Factor denominator: $x^3+4x = x(x^2+4)$.
\end{itemize}
\abovedisplayskip=0pt
\belowdisplayskip=0pt
\begin{eqnarray*}
\uncover<4->{%
\frac{2x^2-x+4}{x(x^2+4)}%
}%
& \uncover<4->{ = } & %
\uncover<4->{%
\frac{A}{x} + \frac{Bx+C}{(x^2+4)}%
}\\%
\uncover<5->{%
2x^2-x+4%
}%
& \uncover<5->{ = } & %
\uncover<5->{%
A(x^2+4) + (Bx+C)x%
}\\%
\uncover<6->{%
\alert<handout:0| 9>{2x^2}\alert<handout:0| 8>{-x}+\alert<handout:0| 7>{4}%
}%
& \uncover<6->{ = } & %
\uncover<6->{%
\alert<handout:0| 9>{(A+B)x^2} + \alert<handout:0| 8>{Cx} + \alert<handout:0| 7>{4A}
}\\%
\end{eqnarray*}
\abovedisplayskip=0pt
\belowdisplayskip=0pt
\vspace{-.4in}
\[
\uncover<7->{\alert<handout:0| 7>{A = 1}}\qquad%
\uncover<8->{\alert<handout:0| 8>{C = -1}}\qquad%
\uncover<9->{\alert<handout:0| 9-10>{A+B = 2}}%
\uncover<10->{\alert<handout:0| 10>{,\textrm{ therefore } B = 1}}%
\]
\abovedisplayskip=0pt
\belowdisplayskip=0pt
\vspace{-.2in}
\begin{eqnarray*}
%\[
%\begin{array}{r@{ \ }c@{ \ }l}
\uncover<11->{%
\int \frac{2x^2-x+4}{x(x^2+4)}\diff x%
}%
& \uncover<11->{ = } & %
\uncover<11->{%
\int \left( \frac{1}{x} + \frac{x-1}{x^2+4}\right) \diff x%
}\\%
& \uncover<12->{ = } & %
\uncover<12->{%
\int  \frac{1}{x}\diff x + \int \frac{x}{x^2+4}\diff x - \int \frac{1}{x^2+4}\diff x%
}\\%
& \uncover<13->{ = } & %
\uncover<13->{%
\ln |x| + \frac{1}{2}\ln (x^2 + 4) - \frac{1}{2} \tan^{-1} \left( \frac{x}{2}\right) + K%
}\\%
%\end{array}
%\]
\end{eqnarray*}
\vspace{-.3in}
\end{example}
\end{frame}
% end module partial-fractions-case3-ex5

% begin module partial-fractions-case4
\begin{frame}
%\begin{enumerate}
%\setcounter{enumi}{3}
%\item  
Suppose $Q(x)$ contains irreducible quadratic factors, some of which are repeated.
%\end{enumerate}

If $Q(x)$ has the factor $(ax^2 + bx+c)^r$, where $b^2-4ac < 0$, then instead of the single term $(Ax+B)/(ax^2+bx+c)$ we use  
\[
\frac{A_1x+B_1}{ax^2+bx+c} + %
\frac{A_2x+B_2}{(ax^2+bx+c)^2} + %
 \cdots + %
\frac{A_rx+B_r}{(ax^2+bx+c)^r} %
\]
in the partial fraction decomposition of $R(x)/Q(x)$.

These terms can be integrated by completing the square.
\end{frame}
% end module partial-fractions-case4

% begin module partial-fractions-case4-ex7
\begin{frame}
\begin{example} %[Example 7, p. 516]
Write out the form of the partial fraction decomposition of
\[
\frac{x^3+x^2+1}{\alert<handout:0| 2-3>{x}\alert<handout:0| 4-5>{(x-1)}\alert<handout:0| 6-7>{(x^2+x+1)}\alert<handout:0| 8-9>{(x^2+1)^3}}
\]
\[
\uncover<3->{\alert<handout:0| 3>{%
 = \frac{A}{x} + %
}}%
\uncover<5->{\alert<handout:0| 5>{%
 \frac{B}{x-1} + %
}}%
\uncover<7->{\alert<handout:0| 7>{%
 \frac{Cx+D}{x^2+x+1} + %
}}%
\uncover<9->{\alert<handout:0| 9>{%
 \frac{Ex+F}{x^2+1} + %
 \frac{Gx+H}{(x^2+1)^2} + %
 \frac{Ix+J}{(x^2+1)^3} %
}}%
\]
\end{example}
\end{frame}
% end module partial-fractions-case4-ex7

\end{document}
