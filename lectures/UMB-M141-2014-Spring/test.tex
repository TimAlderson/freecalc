\documentclass%
%[handout]
{beamer}
% % % % % % % %
% % % % % % % %
% % % % % % % %
%IMPORTANT
%compiles with 
%pdflatex -shell-escape 
%IMPORTANT
% % % % % % % %
% % % % % % % %
% % % % % % % %
\mode<presentation>
{
\useinnertheme{rounded}
\useoutertheme{infolines}
\usecolortheme{orchid}
\usecolortheme{whale}
}

\usepackage[english]{babel}
\usepackage[latin1]{inputenc}
\usepackage[all,cmtip]{xy}
\usepackage{times}
\usepackage[T1]{fontenc}
\usepackage{../example-templates}
\usepackage{../pstricks-commands}

\usepackage{auto-pst-pdf}
\usepackage{pst-plot}
%\usepackage{pstricks-add} 

% Or whatever. Note that the encoding and the font should match. If T1
% does not look nice, try deleting the line with the fontenc.

\graphicspath{{../../modules/}}

\newtheoremstyle{partialproof}{3pt}{3pt}{}{}{}{.}{.5em}{}
\theoremstyle{partialproof} \newtheorem{partialproof}[theorem]{Proof.}
%\DeclareMathOperator{\diff}{d}
\setbeamertemplate{navigation symbols}{}

\includeonlylecture{1}

\newcommand{\lect}[3]{
  \date{#1}
  \lecture[#1]{#2}{#3}
}

\setbeamertemplate{footline}
{
  \leavevmode%
  \hbox{%
  \begin{beamercolorbox}[wd=.333333\paperwidth,ht=2.25ex,dp=1ex,center]{author in head/foot}%
    \usebeamerfont{author in head/foot}\insertshortauthor
  \end{beamercolorbox}%
  \begin{beamercolorbox}[wd=.333333\paperwidth,ht=2.25ex,dp=1ex,center]{title in head/foot}%
    \usebeamerfont{title in head/foot}\insertshorttitle
  \end{beamercolorbox}%
  \begin{beamercolorbox}[wd=.333333\paperwidth,ht=2.25ex,dp=1ex,center]{date in head/foot}%
    \usebeamerfont{date in head/foot}\insertshortdate{}
  \end{beamercolorbox}}%
  \vskip0pt%
}

% If you have a file called "university-logo-filename.xxx", where xxx
% is a graphic format that can be processed by latex or pdflatex,
% resp., then you can add a logo as follows:

%\pgfdeclareimage[height=0.8cm]{logo}{bluelogo}
%\logo{\pgfuseimage{logo}}
\renewcommand{\Arcsin}{\arcsin}
\renewcommand{\Arccos}{\arccos}
\renewcommand{\Arccot}{\arccot}
\renewcommand{\Arctan}{\arctan}


\begin{document}

\AtBeginLecture{%

\title[\insertlecture]{FreeCalc}
\subtitle{\insertlecture}
\author[FreeCalc]{}
\institute[UMass Boston]{University of Massachusetts Boston}
\date{\insertshortlecture}
\begin{frame}
  \titlepage
\end{frame}
}%

% begin lecture
\lect{\today}{Sample}{1}
%%begin module partial-fractions-building-blocks-intro
\begin{frame}
\frametitle{Integrating arbitrary rational functions}
Let $\frac{P(x)}{Q(x)}$ be an arbitrary rational function, i.e., a quotient of polynomials.
\begin{question}
Can we integrate $\displaystyle\int \frac{P(x)}{Q(x)}dx$?
\end{question}
\begin{itemize}
\item Yes. We will now proceed to learn how.
\item The algorithm for integration is roughly:
\begin{itemize}
\item We use algebra to split $\frac{P(x)}{Q(x)}$ into smaller pieces (``partial fractions''). 
\item We use linear substitutions to transform each piece to one of 4 basic building block integrals.
\item We solve each building block integral and collect the terms.
\end{itemize}
\item We study the algorithm ``from the ground up'': we start with the building blocks.
\end{itemize}
\end{frame}

%end module partial-fractions-building-blocks-intro
%%begin module partial-fractions-building-block-1
\begin{frame}

\end{frame}
%end module partial-fractions-building-block-1
%%begin module partial-fractions-building-block-2
\begin{frame}
\frametitle{Building block II}
The second building block integral is: $\displaystyle \int \frac{1}{x }\diff x$. 
\begin{example} Integrate the second building block integral 
\[
\int \frac{1}{x }\diff x \uncover<2->{\alert<2,3>{ =}}\uncover<3->{\alert<3>{\ln | x | +C} }
\]
\end{example}
\end{frame}
\begin{frame}
\frametitle{Linear substitutions leading to building block II}
Second building block: $\displaystyle \int \frac{1}{x }\diff x=\ln |x|+ C$. 
\begin{example} Integrate 
\[
\begin{array}{rcll|l}
\displaystyle \alert<9>{\int \frac{1}{-4x+5 }\diff x} \uncover<2->{&=&\displaystyle \int \frac{1}{(-4x+5) }\frac{\alert<3>{ \diff (\alert<2>{-4} x)}}{ (\alert<2>{-4})}} \\
\uncover<3->{&=&\displaystyle \int \frac{1}{\alert<4>{(-4x+5)} } \frac{\alert<3>{\diff (\alert<4>{-4x+5})}}{ (-4)} \uncover<4->{&&\text{Set } \alert<4,8>{u=-4x+5}}}\\
\uncover<4->{&=&\displaystyle \int \alert<6>{\frac{1}{\alert<4>{u}}}\frac{\diff \alert<4>{u}}{(\alert<5>{-4})}}\\
\uncover<5->{&=&\displaystyle \alert<5>{-\frac{1}{4}} \alert<7>{\int \alert<6>{u^{-1}} \diff u} }\uncover<7->{=-\frac{1}{4}\alert<7>{\ln |\alert<8>{u}|}+C}\\
\uncover<8->{&=&\displaystyle \alert<9>{-\frac{1}{4}\ln |\alert<8>{-4x+5}|  +C}}\quad .
\end{array}
\]

\end{example}
\end{frame}
\begin{frame}
\frametitle{Lin. subst. leading to building block II: general case}
Second building block: $\displaystyle \int \frac{1}{x }\diff x=\ln |x|+ C$. 
\begin{example} Integrate 
\[
\begin{array}{rcll|l}
\displaystyle \alert<1>{ \int \frac{1}{ax+b }\diff x}&=&\displaystyle \int \frac{1}{(ax+b) }\frac{\diff (a x)}{a} \\
&=&\displaystyle \int \frac{1}{(ax+b) }\frac{\diff (ax+b)}{a} &&\text{Set }u=ax+b\\
&=&\displaystyle \int\frac{1}{u}\frac{\diff u}{a}\\
&=&\displaystyle \frac{1}{a}\int u^{-1} \diff u =\frac{1}{a}\ln |u|+C\\
&=&\displaystyle \alert<1>{\frac{1}{a}\ln |ax+b|  +C}\quad .
\end{array}
\]

\end{example}
\end{frame}

%end module partial-fractions-building-block-2
%%begin module partial-fractions-building-blocks-3-and-4-intro
\begin{frame}
\frametitle{Simplified ang general building blocks III and IV}
\begin{itemize}
\item The third building block integral is: 

$\displaystyle \int \frac{x}{(1+x^2)^n }\diff x.$

\item<2-> We will first treat the special case $n=1$:

$\displaystyle \int \frac{x}{1+x^2 }\diff x$. 
\item<3-> The fourth building block integral is:

$\displaystyle \int \frac{1}{(1+x^2)^n }\diff x$
\item<4->  We will first treat the special case $n=1$:

$\displaystyle \int \frac{1}{1+x^2 }\diff x.$
\end{itemize}

\end{frame}
%end module partial-fractions-building-blocks-3-and-4-intro
%%begin module partial-fractions-building-blocks-3-and-4-intro
\begin{frame}
\frametitle{Building blocks III and IV for $n=1$}
Special case building block III: $\int \frac{x}{1+x^2}\diff x.$

Special case building block IV: $\int \frac{1}{1+x^2 }\diff x.$
\begin{example}
Integrate 
\[
\begin{array}{rcl}
\displaystyle \int \frac{x}{1+x^2 }\diff x &=&\displaystyle \int \frac{1}{(1+x^2) }\frac{\diff (x^2)}{2} = \int \frac{1}{1+x^2 }\frac{\diff (1+x^2)}{2} \\
&=&\displaystyle\frac12\ln (1+x^2)+C\quad .
\end{array}
\]
\end{example}
\begin{example}
Integrate 
\[
\displaystyle \int \frac{1}{1+x^2 }\diff x =\Arctan x +C
\]
\end{example}
\vspace{2cm} 
\end{frame}
\begin{frame}
\frametitle{Linear substitutions leading to blocks III and IV}
Special case building block III: $ \int \frac{x}{1+x^2}\diff x = \frac{1}{2}\ln(1+x^2)+C$.

Special case building block IV: $ \int \frac{1}{1+x^2 }\diff x=\Arctan x+C.$


\begin{example}
\[
\begin{array}{rcll|l}
\displaystyle \int \frac{1}{x^2+2}\diff x&=&\displaystyle \int\frac{1}{2 \left(\frac{1}{2}x^2+1\right)}\diff x \\
&=&\displaystyle \int \frac{1}{2\left(\left(\frac{x}{\sqrt{2}}\right)^2+1  \right)} \sqrt{2}\diff\left(\frac{ x}{\sqrt{2}}\right) &&\text{ Set } u=\frac{x}{\sqrt{2}} \\
&=&\displaystyle \frac{\sqrt{2}}{2}\int \frac{1}{1+u^2}du\\
&=&\frac{\sqrt{2}}{2}\Arctan (u)+C \\&=&\frac{\sqrt{2}}{2} \Arctan\left(\frac{x}{\sqrt{2}}\right)+C
\end{array}
\]

\end{example}
\vspace{2cm}

\end{frame}

\begin{frame}
\frametitle{Linear substitutions leading to blocks III and IV}
Special case building block III: $ \int \frac{x}{1+x^2}\diff x = \frac{1}{2}\ln(1+x^2)+C$.

Special case building block IV: $ \int \frac{1}{1+x^2 }\diff x=\Arctan x+C.$


\begin{example}
\[
\begin{array}{rcll|l}
\displaystyle \int \frac{x}{2x^2+3}\diff x&=&\displaystyle \int\frac{1}{2x^2+3}\diff \left(\frac{x^2}{2}\right) \\
&=&\displaystyle \int\frac{1}{2x^2+3}\diff \left(\frac{2x^2+3}{4}\right)&&\text{Set } u=2x^2+3 \\
&=&\displaystyle \frac{1}{4}\int \frac{1}{u}du\\
&=&\frac{1}{4}\ln (u)+C\\
&=&\frac{1}{4}\ln (2x^2+3)+C
\end{array}
\]

\end{example}
\vspace{4cm}

\end{frame}
\begin{frame}
\frametitle{Linear substitutions leading to blocks III and IV}
Special case building block III: $ \int \frac{x}{1+x^2}\diff x = \frac{1}{2}\ln(1+x^2)+C$.

Special case building block IV: $\int \frac{1}{1+x^2 }\diff x=\Arctan x+C.$

\begin{itemize}

\item Let $ax^2+bx+c$ be a quadratic with no real roots.
\item Then $ax^2+bx+c$ can be transformed to the form $r(u^2+1)$ by a linear substitution $u=px+q$ (for some number $r, p, q$). 
\item To find such substitution start by completing  the square. 
\item After completing the square proceed as in preceding examples.
\item In this way, integrals of the form $\displaystyle \int \frac{Ax+B}{ax^2+bx+c} \diff x$ can be expressed via building blocks III and IV.

\item We show examples, and then proceed in full generality.
\end{itemize}
\vspace{5cm}
\end{frame}


\begin{frame}
\frametitle{Linear substitutions leading to blocks III and IV}
Special case building block III: $ \int \frac{x}{1+x^2}\diff x = \frac{1}{2}\ln(1+x^2)+C$.

Special case building block IV: $\int \frac{1}{1+x^2 }\diff x=\Arctan x+C.$


\begin{example}
\uncover<7->{Let \alert<7,27,30>{$u= x+ \frac{1}{2} $}}\uncover<16->{, let \alert<16,28>{$z=\frac{2u}{\sqrt{3}}$}.} Integrate 
\[
\begin{array}{rcl}
\displaystyle \int\frac{x}{x^2+\alert<3>{x}+1}\diff x 
\only<1>{{~~~~~~~~~~~~~~~~~~~~~~~~~~~~~~~~~~~~~~~~~~~~~~~~~~~~~~} {~~~~~~~~~~~~~~~~~~~~~~~~~~~~~~~~~~~~~~~~~~~~~~~~~~~~~~} }
\only<2-10>{&=& \displaystyle \int \frac{x}{\alert<4>{ x^2+\alert<3>{2\frac{1}{2}x} +\frac{1}{4} } \alert<5>{-\frac{1}{4} +1} } \alert<6>{\diff x}  {~~~~~~~~~~~~~~~~~~~~~~~~~~~~~~~~~~~~~~~~~~~~~~~~~~~~~~} \\
\uncover<4->{&=&\displaystyle \int \frac{\alert<7>{x+\frac{1}{2}}-\frac{1}{2}}{ \alert<4>{\left(\alert<7>{x+\frac{1}2}\right)^2}+\alert<5>{\frac{3}{4}} }\alert<6>{\diff \left(\alert<7>{x+\frac{1}{2}}\right) }}\\
\uncover<7->{&=&\displaystyle \int \frac{\alert<7,8>{u} \alert<9>{ -\frac{1}{2}} }{{\alert<7>{u}}^2+\frac{3}{4}}\diff \alert<7>{u}} \\
\uncover<8->{&=&\alert<10>{ \displaystyle \int \frac{\alert<8>{u}}{u^2+\frac{3}{4}}\diff u\alert<9>{-\frac{1}{2}}\int \frac{1}{u^2+\frac{3}{4}}\diff u}}
}

\only<11->{
&=&\alert<11>{\displaystyle\alert<25>{ \int \frac{u}{u^2+\frac{3}{4}}\diff u} -\frac{1}{2} \alert<12,18>{\int \frac{1}{u^2+\frac{3}{4}}\diff u}} {~~~~~~~~~~~~~~~~~~~~~~~~~~~~~~~~~~~~~~~~~~~~~~~~~~~~~~} \\
}
\only<11-18>{
\only<12->{\displaystyle \alert<12,18>{ \int \frac{1}{\alert<13>{u^2+\frac{3}{4}} }\diff u}\uncover<13->{&=& \displaystyle \int \frac{1}{\alert<13>{\frac{3}{4}\left( \alert<14>{\frac{4}{3}u^2} +1\right)}}\alert<15>{\diff u}}}\\
\uncover<14->{&=&\displaystyle \int \frac{1}{\frac{3}{4}\left(\alert<14>{ \left(\alert<16>{ \frac{2u}{ \sqrt{3}}}\right)^2} +1\right)} \alert<15>{ \frac{\sqrt{3} }{2} \diff \left(\alert<16>{ \frac{2u}{\sqrt{3}}}\right) }}\\
\uncover<16->{&=&\displaystyle \frac{2\sqrt{3}}{3}\int \frac{1}{\alert<16>{z}^2+1}\diff \alert<16>{z}}\uncover<17->{ = \alert<18>{\frac{2\sqrt{3}}{3} \Arctan z}+C}
}
\only<19->{
&=&\displaystyle \only<19-24>{\alert<20>{\int \frac{u}{u^2+\frac{3}{4}}\diff u}} \only<25->{\alert<25>{ \frac{1}{2}\ln \left(\alert<27>{u}^2+\frac{3}{4}\right) } }-\frac{1}{2}\alert<19>{ \frac{2\sqrt{3}}{3} \Arctan \alert<28>{z} }+C\\
\only<26->{
\uncover<27->{&=&\displaystyle \frac{1}{2}\ln  \left( \alert<29>{ {\alert<27>{\left(x+\frac{1}{2}\right)}}^2 + \frac{3}{4}}\right) - \frac{\sqrt{3}}{3} \Arctan \left( \alert<28>{ \frac{\alert<30>{2u }}{ \sqrt{3}}} \right)+C} \\
\uncover<29->{&=&\displaystyle \frac{1}{2}\ln \left(\alert<29>{x^2+x+1}\right) - \frac{\sqrt{3}}{3} \Arctan \left(\frac{\alert<30>{2x+1}}{\sqrt{3}}\right)+C
}
}
}
\only<20-25>{
\displaystyle \alert<20,25>{\int \frac{\alert<21>{ u} }{u^2+\frac{3}{4}}\alert<21>{ \diff u} } \uncover<21->{ &=& \displaystyle\int \frac{1}{u^2+\frac{3}{4}}\diff \alert<21>{\left(\frac{u^2}{\alert<22>{2}}\right)}}\\
\uncover<22->{&=&\displaystyle \alert<22>{\frac{1}{2}}\int \frac{1}{\alert<24>{u^2+\frac{3}{4}}}\diff \left(\alert<24>{ u^2\uncover<23->{ \alert<23>{ +\frac{3}{4}} }} \right)}\uncover<24->{ =\alert<25>{ \frac{1}{2}\ln \left(\alert<24>{u^2 +\frac{3}{4}} \right)}+C}
}
\end{array}
\]
\end{example}

\vspace{8cm}

\end{frame}

%end module partial-fractions-building-blocks-3-and-4-intro

%%begin module Building block III

\begin{frame}
\frametitle{Building block III}
In the preceding slides we solved the partial case of building block integral III: $\int \frac{x}{x^2+1}\diff x$. The general case is not much harder.

\begin{example}
\[
\begin{array}{rcl}
\displaystyle \int \frac{\alert<2>{ x} }{(x^2+1)^n} \alert<2>{\diff x} \uncover<2->{&=&\displaystyle  \int \frac{1}{(\alert<3>{x^2+1})^n} \alert<2>{\frac{\diff \left(\alert<3>{x^2+1} \right )}{2}}} \\
\uncover<3->{&=& \displaystyle \frac{1}{2}\int {\alert<3>{u}}^{-n}\diff \alert<3>{ u}} \\
\uncover<4->{&=&\left\{\begin{array}{ll}\displaystyle
\uncover<5->{\alert<5>{ \frac{1}{2}\ln (x^2+1) +C}} & \alert<4,5>{\text{if }n=1} \\
\uncover<7->{\alert<7>{\displaystyle \frac{1}{2} \frac{(x^2+1)^{-n+1}}{(-n+1)}+C}} &\alert<6>{ \text{if }n\neq -1}
\end{array} 
\right. ,}
\end{array}
\]
\uncover<3->{where we used the substitution $\alert<3>{u=x^2+1}$.}

\end{example}

\end{frame}

%end module Building block III
%begin module Building block IV

\begin{frame}
\frametitle{Building block IV}
Building block integral IV: $\int \frac{1}{(x^2+1)}\diff x=\arctan x+C$. Unlike previous cases, the general case $\int \frac{1}{(x^2+1)^n}\diff x$ is much harder.

\begin{example}
\[
\begin{array}{rcl}
\int \frac{1}{(x^2+1)^n}\diff x&=&
\end{array}
\]

\end{example}

\end{frame}

%end module Building block IV

%% begin module partial-fractions-intro
\begin{frame}
\frametitle{Integration of Rational Functions by Partial Fractions}
A rational function is a function which can be written as ratio of polynomials, $\frac{P(x)}{Q(x)}$. We will show how to integrate any rational function. To do so, we express $\frac{P(x)}{Q(x)}$ as a sum of simple fractions, called partial fractions. We start with an example. Put $2/(x-1)$ and $1/(x+2)$ over a common denominator:
\[
\alert<5>{\frac{2}{x-1} - \frac{1}{x+2} } = %
\uncover<2->{%
\frac{2(x+2) - (x-1)}{(x-1)(x+2)} = %
}%
\uncover<3->{%
\alert<5>{ \frac{x + 5}{x^2+x-2} }%
}%
\]

\uncover<4->{%
We can now solve the following integral:
\[
\int \alert<5>{ \frac{x+5}{x^2+x-2}}\diff x = %
\uncover<5->{%
\int \left(\alert<5>{\frac{2}{x-1} - \frac{1}{x+2}} \right) \diff x = %
}%
\uncover<6->{%
2\ln | x - 1| - \ln | x + 2| + C
}%
\]
}%

\end{frame}
% end module partial-fractions-intro

%% begin module partial-fractions-long-division
\begin{frame}
\frametitle{Review of polynomial notation}
\begin{itemize}
\item Recall that a rational function is a function of the form
\[
f(x) = \frac{P(x)}{Q(x)}
\]
where $P$ and $Q\neq 0$ are polynomials. 
\item<2->Recall that the degree of $P$ is the highest power of $x$ in $P$ that has a non-zero coefficient.
\end{itemize}
\end{frame}

\begin{frame}\frametitle{Ensure denominator degree $>$ numerator degree}
\begin{itemize}
\item To decompose $\frac{P(x)}{Q(x)}$ in partial fractions we ensure first the degree of the numerator is smaller than the degree of the denominator.
\item<2-> We recall that to divide the \alertNoH{4}{dividend $P(x)$} by the \alertNoH{5}{divisor $Q(x)$} to get \alertNoH{6}{quotient $S(x)$} with \alertNoH{7}{remainder $R(x)$} means \uncover<3->{ to find polynomials  $S(x), R(x)$ such that \alertNoH{7}{$\deg R<\deg Q$} and
\[
\begin{array}{rcll|l}
\displaystyle \alertNoH{4} {P(x)}& =&\displaystyle  \alertNoH{6}{S(x)} \alertNoH{5}{Q(x)} + \alertNoH{7}{ R(x)} \uncover<8->{&&\alertNoH{8}{ \text{divide by } Q(x)}}\\
\displaystyle \uncover<8->{\frac{P(x)}{\alertNoH{8}{Q(x)} }&=&\displaystyle  \frac{S(x)\fcCancel{9}{Q(x)} }{\fcCancel{9}{\alertNoH{8}{Q(x)}}} +\frac{R(x)}{ \alertNoH{8}{ Q(x)}}} \\
\uncover<9->{
\displaystyle \frac{P(x)}{Q(x) }&=&\displaystyle  S(x) +\frac{R(x)}{ Q(x)}} \\
\end{array}
\]
}
\item<10-> The above transforms $\frac{P(x)}{Q(x)}$ to a polynomial plus a fraction in which the numerator has degree smaller than the denominator.
\item<11-> The polynomials $Q(x)$ and $S(x)$ are computed via polynomial long division. We recall the procedure through examples.
\end{itemize}
\end{frame}
% end module partial-fractions-long-division

%% begin module partial-fractions-long-division-ex1
\begin{frame}
\begin{example} %[Example 1, p. 510]
Find $\int \frac{x^3 + x}{x - 1}\diff x$.
\begin{columns}[t]
\column{.45\textwidth}
\uncover<2->{%
\[
\begin{array}{r@{}r@{}c@{}r@{}c@{}c@{}c@{}r@{}r@{}}
& & & %
\uncover<4->{\alert<handout:0| 4-6,23>{x^2}} & %
\uncover<11->{\alert<handout:0| 11-13,23>{+}} & %
\uncover<11->{\alert<handout:0| 11-13,23>{ x}} & %
\uncover<17->{\alert<handout:0| 17-19,23>{+}} & %
\uncover<17->{\alert<handout:0| 17-19,23>{ 2}} & \\%
\cline{3-8}
\alert<handout:0| 3-6,10-13,16-19>{x} & %
\alert<handout:0| 5-6,12-13,18-19>{-1} & %
\Big) & %
\alert<handout:0| 3-4,7-8>{x^3} & %
 & %
 & %
\alert<handout:0| 9>{+} & %
\alert<handout:0| 9>{ x} & \\%
& & & %
\uncover<6->{\alert<handout:0| 6-8>{x^3}} & %
\uncover<6->{\alert<handout:0| 6-8>{-}} & %
\uncover<6->{\alert<handout:0| 6-8>{ x^2}} & %
&  & \\%
\cline{4-6}%
& & & %
&  & %
\uncover<8->{\alert<handout:0| 8,10-11,14-15>{x^2}} & %
\uncover<9->{\alert<handout:0| 9,14-15>{+}} & %
\uncover<9->{\alert<handout:0| 9,14-15>{ x}} & \\%
& & & %
&  & %
\uncover<13->{\alert<handout:0| 13-15>{x^2}} & %
\uncover<13->{\alert<handout:0| 13-15>{-}} & %
\uncover<13->{\alert<handout:0| 13-15>{ x}} & \\%
\cline{6-8}
& & & %
 & %
 & %
 & %
 & %
\uncover<15->{\alert<handout:0| 15-17,20-21>{2x}} & \\%
& & & %
 & %
 & %
 & %
 & %
\uncover<19->{\alert<handout:0| 19-21>{2x}} & %
\uncover<19->{\alert<handout:0| 19-21>{- 2}} \\%
\cline{8-9}%
& & & %
 & %
 & %
 & %
 & %
 & %
\uncover<21->{\alert<handout:0| 21,24>{2}} \\%
\end{array}
\]
}%

\only<handout:0| -4,10-11,16-17>{\uncover<3->{%
Divide %
}}%
\only<handout:0| 5-6,12-13,18-19>{%
Multiply %
}%
\only<handout:0| 7-8,14-15,20-21>{%
Subtract %
}%
\only<handout:0| 3-4>{%
$x^3$ %
}%
\only<handout:0| 10-11>{%
$x^2$ %
}%
\only<handout:0| 16-17>{%
$2x$ %
}%
\only<handout:0| 5-6>{%
$x^2$ %
}%
\only<handout:0| 12-13>{%
$x$ %
}%
\only<handout:0| 18-19>{%
$2$ %
}%
\only<handout:0| 7-8>{%
$x^3-x^2$ %
}%
\only<handout:0| 14-15>{%
$x^2-x$ %
}%
\only<handout:0| 20-21>{%
$2x-2$ %
}%
\only<handout:0| -4,10-11,16-17>{\uncover<3->{%
by %
}}%
\only<handout:0| 5-6,12-13,18-19>{%
by %
}%
\only<handout:0| 7-8,14-15,20-21>{%
from %
}%
\only<handout:0| 3-4,10-11,16-17>{%
$x$ %
}%
\only<handout:0| 5-6,12-13,18-19>{%
$x-1$ %
}%
\only<handout:0| 7-8>{%
$x^3$ %
}%
\only<handout:0| 14-15>{%
$x^2+x$ %
}%
\only<handout:0| 20-21>{%
$2x$ %
}%
\only<handout:0| 9>{%
Bring down the $x$%
}%
\invisible<1->{%
y%
}%
\column{.55\textwidth}
\begin{eqnarray*}
& & %
\uncover<22->{%
\int \frac{x^3 + x}{x - 1}\diff x %
}\\%
& \uncover<22->{ = } & %
\uncover<22->{%
\int \left( \alert<handout:0| 23>{x^2 + x + 2} + \frac{\alert<handout:0| 24>{2}}{x - 1}\right) \diff x
}\\%
& \uncover<25->{ = } & %
\uncover<25->{%
\frac{x^3}{3} + \frac{x^2}{2} + 2x %
}\\%
& & \uncover<25->{%
\qquad + 2\ln | x - 1 | + C %
}%
\end{eqnarray*}
\end{columns}
\end{example}
\end{frame}
% end module partial-fractions-long-division-ex1


\end{document}
