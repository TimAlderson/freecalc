\documentclass%
%[handout]
{beamer}
% % % % % % % %
% % % % % % % %
% % % % % % % %
%IMPORTANT
%compiles with 
%pdflatex -shell-escape 
%IMPORTANT
% % % % % % % %
% % % % % % % %
% % % % % % % %
\mode<presentation>
{
\useinnertheme{rounded}
\useoutertheme{infolines}
\usecolortheme{orchid}
\usecolortheme{whale}
}

\usepackage[english]{babel}
\usepackage[latin1]{inputenc}
\usepackage[all,cmtip]{xy}
\usepackage{times}
\usepackage[T1]{fontenc}
\usepackage{../example-templates}
\usepackage{../pstricks-commands}

\usepackage{auto-pst-pdf}
\usepackage{pst-plot}
%\usepackage{pstricks-add} 

% Or whatever. Note that the encoding and the font should match. If T1
% does not look nice, try deleting the line with the fontenc.

\graphicspath{{../../modules/}}

\newtheoremstyle{partialproof}{3pt}{3pt}{}{}{}{.}{.5em}{}
\theoremstyle{partialproof} \newtheorem{partialproof}[theorem]{Proof.}
%\DeclareMathOperator{\diff}{d}
\newcommand{\diff}{\text{d}}
\setbeamertemplate{navigation symbols}{}

\includeonlylecture{1}

\newcommand{\lect}[3]{
  \date{#1}
  \lecture[#1]{#2}{#3}
}

\setbeamertemplate{footline}
{
  \leavevmode%
  \hbox{%
  \begin{beamercolorbox}[wd=.333333\paperwidth,ht=2.25ex,dp=1ex,center]{author in head/foot}%
    \usebeamerfont{author in head/foot}\insertshortauthor
  \end{beamercolorbox}%
  \begin{beamercolorbox}[wd=.333333\paperwidth,ht=2.25ex,dp=1ex,center]{title in head/foot}%
    \usebeamerfont{title in head/foot}\insertshorttitle
  \end{beamercolorbox}%
  \begin{beamercolorbox}[wd=.333333\paperwidth,ht=2.25ex,dp=1ex,center]{date in head/foot}%
    \usebeamerfont{date in head/foot}\insertshortdate{}
  \end{beamercolorbox}}%
  \vskip0pt%
}

% If you have a file called "university-logo-filename.xxx", where xxx
% is a graphic format that can be processed by latex or pdflatex,
% resp., then you can add a logo as follows:

%\pgfdeclareimage[height=0.8cm]{logo}{bluelogo}
%\logo{\pgfuseimage{logo}}
\renewcommand{\Arcsin}{\arcsin}
\renewcommand{\Arccos}{\arccos}
\renewcommand{\Arccot}{\arccot}
\renewcommand{\Arctan}{\arctan}


\begin{document}

\AtBeginLecture{%

\title[\insertlecture]{FreeCalc}
\subtitle{\insertlecture}
\author[FreeCalc]{}
\institute[UMass Boston]{University of Massachusetts Boston}
\date{\insertshortlecture}
\begin{frame}
  \titlepage
\end{frame}
}%

% begin lecture
\lect{\today}{Sample}{1}
%% begin module e-limit
\begin{frame}
\begin{theorem}[The Number $e$ as a Limit]
\[
e = \lim_{x\rightarrow 0} (1 + x)^{\frac{1}{x}} = \lim_{y\to \infty} \left(1+\frac{1}{y}\right)^y.
\]
\end{theorem}
\begin{proof}
\uncover<2->{Let $f(x) = \ln x$.  }%
\uncover<3->{Then $f'(x) = \frac{1}{x}$, so $f'(1) = 1$.}%
\abovedisplayskip=0pt
\belowdisplayskip=0pt
\abovedisplayshortskip=0pt
\belowdisplayshortskip=0pt
\begin{align*}
\uncover<4->{\alertNoH{ 10}{1} = f'(1)} & \uncover<4->{=} %
\uncover<4->{\lim_{h\rightarrow 0}\frac{f(1+h)-f(1)}{h}}%
\uncover<5->{ = \lim_{x\rightarrow 0}\frac{f(1+x)-f(1)}{x}}\\%
& \uncover<6->{=}  %
\uncover<6->{\lim_{x\rightarrow 0}\frac{\ln (1+x)-\ln (1)}{x}}%
\uncover<7->{ = \lim_{x\rightarrow 0}\frac{1}{x}\ln (1 + x)}\\%
& \uncover<8->{\alertNoH{ 10}{=}}  %
\uncover<8->{\alertNoH{ 10}{\lim_{x\rightarrow 0}\ln (1+x)^{\frac{1}{x}}}.}
\end{align*}
\uncover<9->{Then use the fact that \alertNoH{ 11}{the exponential function is continuous}:}
\[
\uncover<9->{e = e^{\alertNoH{ 10}{1}} =}%
\uncover<10->{\alertNoH{ 11}{e^{\alertNoH{ 10}{\lim\limits_{x\rightarrow 0}\ln (1+x)^{\frac{1}{x}}}} =}}%
\uncover<11->{\alertNoH{ 11}{\lim\limits_{x\rightarrow 0}e^{\ln (1+x)^{\frac{1}{x}}}} =}%
\uncover<12->{\lim\limits_{x\rightarrow 0} (1+x)^{\frac{1}{x}}.}\qedhere
\]
\end{proof}
\end{frame}
% end module e-limit

%%begin module e-limit-problems-ex1
\begin{frame}
\begin{example}
Compute
\[
\begin{array}{rcll|l}
\displaystyle\lim\limits_{x\to \infty}\left(\alertNoH{2}{ \frac{x+3}{x}} \right)^x
\uncover<2->{
&=&\displaystyle  \lim\limits_{x\to \infty} \left(\alertNoH{2}{1 +\alertNoH{3}{\frac{3}{x}} } \right)^{\alertNoH{4}{x}} } \\
\uncover<3->{&=& \displaystyle \lim\limits_{x\to \infty}\left(1+\frac{1}{\alertNoH{3,6}{ \frac{x}{3}} }\right)^{\alertNoH{4}{3\alertNoH{6}{\frac{x}{3}}} } } \uncover<5->{ && \text{Set } \alertNoH{6}{\frac{x}{3}=y}}\\
\uncover<6->{&=&\displaystyle \lim\limits_{\substack{\alertNoH{7}{x\to \infty} \\ \uncover<7->{\alertNoH{7}{\frac{x}{3}=y\to \infty} } }}\left(1+\frac{1}{\alertNoH{6}{y}}\right)^{\alertNoH{8}{3\alertNoH{6}{y}}} } \\
\uncover<8->{ &=&\displaystyle \alertNoH{9,10}{ \alertNoH{11}{\lim\limits_{y\to \infty}}\left(\alertNoH{11}{\left(1+\frac{1}{y}\right)^{\alertNoH{8}{y}}}\right)^{\alertNoH{8}{3}}}} \uncover<9->{\alertNoH{9,10}{=}}\uncover<10->{\alertNoH{10}{\alertNoH{11}{ e}^3} \quad .}
\end{array}
\]

\end{example}
\end{frame}
\begin{frame}
\begin{example}
Compute
\[
\begin{array}{rll|l}
&\displaystyle \lim_{x\to \infty} \left(\frac{\alertNoH{2}{x}}{x-2} \right)^{2x+2}\\
\uncover<2->{ =&\displaystyle
\lim\limits_{x\to \infty}\left(\alertNoH{3}{ \frac{\alertNoH{2}{x-2 +2}} {x-2} } \right)^{2x+2}}
\uncover<3->{= \lim\limits_{x\to \infty}\left(\alertNoH{3}{1+\alertNoH{4}{\frac{2}{x-2}} } \right)^{2\alertNoH{5}{x}+2}} \\
\uncover<4->{=&\displaystyle \lim\limits_{x\to \infty} \left( 1+ \alertNoH{4}{\frac{1}{ \frac{x-2}{2}}} \right)^{\alertNoH{6}{ 2( \alertNoH{5}{ x-2+2} )+2}} }\\
\uncover<6->{ =&\displaystyle \lim\limits_{x\to \infty} \left( 1+ \frac{1}{ \alertNoH{8}{ \frac{x-2}{2}} } \right)^{ \alertNoH{6}{ 4\alertNoH{8}{\frac{x-2}{2}} +6}}} \uncover<8->{= \lim \limits_{\substack{ \alertNoH{8}{\frac{ x-2 }{2}
= y} \\ y\to\infty}} {\alertNoH{9,10}{\left(1+\frac{1}{\alertNoH{8}{ y} }\right)}}^{\alertNoH{9}{4\alertNoH{8}{y}} +\alertNoH{10}{6}} } \uncover<7->{&& \alertNoH{7}{\text{Set } \alertNoH{8}{y=\frac{x-2}{2}} }}\\
\uncover<9->{=& \displaystyle \alertNoH{12}{ \lim\limits_{y\to \infty} }  \alertNoH{9}{ \left( \alertNoH{12}{\left(1+\frac{1}{y} \right)^{y}}\right)^4} \alertNoH{13}{\lim\limits_{ y\to\infty}} \alertNoH{10}{\left(1+\alertNoH{13}{\frac{1}{y}} \right)^6} } \\ \uncover<11->{ =&\alertNoH{14}{ \alertNoH{12}{e}^4\cdot (1+\alertNoH{13}{0})^6}}\uncover<14->{
= \alertNoH{14}{e^4} \quad . }
\end{array}
\]

\end{example}
\end{frame}
%end module e-limit-problems-ex1

%% begin module general-exponential-derivative
\begin{frame}
\begin{theorem}[The Derivative of $a^x$]
\[
\frac{\diff}{\diff x} (a^x) = a^x \ln a .
\]
\end{theorem}
\begin{proof}
\uncover<2->{
Use the fact that $\alert<handout:0| 4,8>{a = e^{\ln a}}$.
}
\begin{eqnarray*}
\uncover<3->{\frac{\diff}{\diff x} (\alert<handout:0| 4>{a^x})}%
& \uncover<3->{ = } & %
\uncover<4->{\frac{\diff}{\diff x} (\alert<handout:0| 4>{e^{\ln a}})^x}\\%
& \uncover<5->{ = } & %
\uncover<5->{\frac{\diff}{\diff x} e^{(\ln a)x}}\\%
& \uncover<6->{ = } & %
\uncover<6->{ e^{(\ln a)x}\frac{\diff}{\diff x}(\ln a)x }\\%
& \uncover<7->{ = } & %
\uncover<7->{ (\alert<handout:0| 8>{e^{\ln a}})^x(\ln a) }\\%
& \uncover<8->{ = } & %
\uncover<8->{ \alert<handout:0| 8>{a}^x(\ln a).}%
\end{eqnarray*}
\end{proof}
\end{frame}
% end module general-exponential-derivative

%begin module differentials-rules
\begin{frame}
\begin{itemize}
\item All rules for computing with derivatives have analogues for computing with differential forms.
\item<2-> The rules for computing differential forms are a direct consequence of the corresponding derivative rules and the transformation law $\diff (f(x))=f'(x)\diff x$.
\end{itemize}
\end{frame}
\begin{frame}
Rule name: \phantom{p}
\only<1,2>{\alertNoH{1,2}{product rule. }}
\only<3,4>{\alertNoH{3,4}{constant derivative rule. }}
\only<7,8>{\alertNoH{7,8}{sum rule. }}
\only<9,10>{\alertNoH{9,10}{chain rule. }}
\only<11-12>{\alertNoH{11-12}{power rule. }}
\only<13-14>{\alertNoH{13-14}{exponent derivative rule. }}
%\only<5,6>{\alertNoH{5,6}{ Constant derivative rule. }}

\begin{tabular}{lll}
Differential rule & Derivative rule\\
\uncover<2->{$\diff (fg)=g \diff f +f \diff g$} &
\uncover<1->{$(fg)'=f'g +f g'$} \\
\uncover<4->{$\diff c=0 {\color{gray!50}{ =0\diff x}}$} &
\uncover<3->{$(c)'=0$ & $c$-const.}\\
\uncover<6->{$\diff (cf)=c\diff f $} &
\uncover<5->{$(cf)'=cf'$ &$c$-const. }\\
\uncover<8->{$\diff (f+g)=\diff f +\diff g$} &
\uncover<7->{$(f+g)'=f'+g'$}\\
\uncover<10->{$\diff f(g(x))=$ $ f'(g(x))\diff g(x) $}\\
\uncover<10->{$\phantom{\diff f(g(x))}=$ $ f'(g(x))g'(x)\diff x$} &
\uncover<9->{$(f(g(x)))'= f'(g(x))g'(x)$} \\
\uncover<10->{$\diff f(g)\phantom{(x)}=f'(g)\diff g$} \\\hline
\uncover<12->{$\diff \left( x^n\right)= nx^{n-1}\diff x$} &
\uncover<11->{$(x^n)'=nx^{n-1}$}\\
\uncover<14->{$\diff  \left(e^x\right)= e^x \diff x$} &
\uncover<13->{$\left(e^x\right)'=e^x$}\\
\uncover<16->{$\diff \left(\sin x\right) = \cos x \diff x$} &
\uncover<15->{$(\sin x)'= \cos x$}\\
\uncover<18->{$\diff \left(\cos x \right)= -\sin x \diff x$} &
\uncover<17->{$(\cos x)'= -\sin x$}\\
\uncover<20->{$\displaystyle \left(\diff \ln x\right) =\frac{1}{x}\diff x$} &
\uncover<19->{$\left(\ln x\right)' =\displaystyle \frac1x$}\\
\end{tabular}


\end{frame}
%end module differentials-rules

%begin module differentials-rules
\begin{frame}
\begin{itemize}
\item All rules for computing with derivatives have analogues for computing with differential forms.
\item<2-> The rules for computing differential forms are a direct consequence of the corresponding derivative rules and the transformation law $\diff (f(x))=f'(x)\diff x$.
\end{itemize}
\end{frame}
\begin{frame}
\uncover<3->{Let $c$ be a constant.} Rule name: \phantom{p}
\only<1,2>{\alert<1,2>{product rule. }}
\only<3,4>{\alert<3,4>{constant derivative rule. }}
\only<7,8>{\alert<7,8>{sum rule. }}
\only<9,10>{\alert<9,10>{chain rule. }}
\only<11-12>{\alert<11-12>{power rule. }}
\only<13-14>{\alert<13-14>{exponent derivative rule. }}
%\only<5,6>{\alert<5,6>{ Constant derivative rule. }}
\only<22>{\alert<22>{Integration by parts.}}
\only<23>{\alert<23>{Integration is linear.}}
\only<24>{\alert<24>{Substitution rule.}}

\uncover<21->{Corresponding \alert<21>{integration rules.} \uncover<25->{\alert<25>{ Integration rules justified via the Fundamental Theorem of Calculus}}}

\begin{tabular}{ll}
\only<1-20>{Differential}\only<21->{\alert<21>{Integration}} rule & Derivative rule \\
\uncover<2->{\alert<22>{$\uncover<21->{\alert<21>{\int}} \diff (fg)=\uncover<21->{\alert<21>{\int}} g \diff f +\uncover<21->{\alert<21>{\int}} f \diff g$} }& 
\uncover<1->{$ (fg)'=f'g +f g'$} \\
\uncover<4->{$\uncover<21->{\alert<21>{\int}}\diff c= 0$} &
\uncover<3->{$(c)'=0$}\\
\uncover<6->{\alert<23>{$\uncover<21->{\alert<21>{\int}}\diff (cf)=c \uncover<21-> {\alert<21>{\int}}\diff f $}} & 
\uncover<5->{$(cf)'=cf'$} \\
\uncover<8->{\alert<23>{$\uncover<21->{\alert<21>{\int}}\diff (f+g) = \uncover<21->{\alert<21>{\int}}\diff f +\uncover<21->{\alert<21>{\int}}\diff g$}} & 
\uncover<7->{$(f+g)'=f'+g'$}\\
\uncover<10->{\alert<24>{$\uncover<21->{\alert<21>{\int}}\diff f(g(x))=$ $ \uncover<21->{\alert<21>{\int}}f'(g(x))\diff g(x) $} }\\
\uncover<10->{\alert<24>{$\phantom{\int \diff f(g(x))}=$ $\uncover<21->{\alert<21>{\int}}f'(g(x))g'(x)\diff x$} } & 
\uncover<9->{$(f(g(x)))'= f'(g(x))g'(x)$} \\ 
\uncover<10->{\alert<24>{$\uncover<21->{\alert<21>{\int}} \diff f(g)\phantom{(x)}= \uncover<21->{\alert<21>{\int}} f'(g) \diff g$} } \\\hline
\uncover<12->{$\diff  x^n= nx^{n-1}\diff x$} & 
\uncover<11->{$(x^n)'=nx^{n-1}$}\\
\uncover<14->{$\diff  e^x= e^x \diff x$} & 
\uncover<13->{$\left(e^x\right)'=e^x$}\\
\uncover<16->{$\diff \sin x = \cos x \diff x$} & 
\uncover<15->{$(\sin x)'= \cos x$}\\
\uncover<18->{$\diff \cos x = -\sin x \diff x$} & 
\uncover<17->{$(\cos x)'= -\sin x$}\\
\uncover<20->{$\displaystyle \diff \ln x=\frac{1}{x}\diff x$} & 
\uncover<19->{$(\ln x)'=\displaystyle \frac1x$}\\
\end{tabular}


\end{frame}
%end module differentials-rules
\end{document}
