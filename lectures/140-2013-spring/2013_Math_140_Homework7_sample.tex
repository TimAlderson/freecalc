\documentclass{article}
\usepackage{amsmath, amsfonts, amssymb, verbatim, hyperref}
\usepackage{auto-pst-pdf}
\usepackage{pst-plot}
\usepackage{multicol}
\addtolength{\hoffset}{-3.5cm}
\addtolength{\textwidth}{6.8cm}
\addtolength{\voffset}{-3.3cm}
\addtolength{\textheight}{6.3cm}
\renewcommand{\Re}{\mathrm{Re~}}
\renewcommand{\Im}{\mathrm{Im~}}
\newcommand{\doublebrace}[4]{\left\{\begin{array}{ll} #1 & #2 \\#3 & #4  \end{array} \right.}
\newcommand{\triplebrace}[6]{\left\{\begin{array}{ll} #1 & #2 \\#3 & #4  \\#5 & #6\end{array} \right.}
\date{}
\newtheorem{problem}{Problem}
\title{
Solution to a problem discussed in class.
}
\begin{document}
\maketitle
\begin{problem}
Find the inverse function of the function $f$. You are asked to do the algebra only; you are not asked to determine the domain and range of the function or its inverse. 
$f(x)= 3x^2+4x-7$, where $x\geq -\frac{2}{3}$.
\end{problem}
\textbf{Solution.} As studied in class, we have the an algebraic recipe for computing $f^{-1}(y)$. Namely, write $y=f(x)=3x^2+4x-7$ and solve for $x$. Indeed, we carry out the recipe:

\begin{eqnarray*}
y&=& 3x^2+4x-7\\
3x^2+4x-7-y&=&0 \\
x_{1,2} &=& \frac{-4 \pm \sqrt{4^2-4*3*(-y-7) }}{6} \quad \quad \text{using the formula } x_{1,2}=\frac{-b\pm \sqrt{b^2-4ac}}{2a} \\
~&=& \frac{-2 \pm \sqrt{25+3y}}{3}\quad .
\end{eqnarray*}
The problem gives us that $x\geq -\frac{2}{3}$. At the same time, we know that $\sqrt{25+3y} \geq 0$, or in other words, $\frac{-2-\sqrt{25+3y}}{3} \leq -\frac{2}{3}$. If $y=-\frac{25}{3}$, then $x_1=x_2=-\frac{2}3$. Therefore $y>-\frac{25}3 $, we have that $x_2=\frac{-2-\sqrt{25+3y}}{3} $ contradicts the condition  that $x\geq -\frac{2}3$. Therefore $x=\frac{-2 +\sqrt{25+3y}}{3}$ is the only possible value for $f^{-1}(y)$. In order for $x$ to be defined, we must have that $25+3y\geq 0$, or $y\geq -\frac{25}{3}$.
Therefore $f^{-1}(y) $ has domain $y\geq -\frac{25}3$ and range $x\geq -\frac{2}{3}$. This completes the solution of the problem, with an answer 
\[
f^{-1}(y)=\frac{-2+\sqrt{25+3y}}{3}, \quad \quad y\geq -\frac{25}3\quad . 
\]

Here is how to plot $f^{-1}(x)= \frac{-2+ \sqrt{25+3x}}{3}$. The plot of an arbitrary parabola $y=ax^2+bx+c$ has vertex at $x=-\frac{b}{2a}$, $y= a\left(-\frac{b}{2a}\right)^2 + b\left(-\frac{b}{2a}\right) + c= -\frac{(b^2-4ac)}{4a}$.  For our particular parabola, the vertex is at $x=-\frac{2}{3}$, $y=-\frac{4^2- 4* 3*(-7)}{4*3}= -\frac{100}{12}= -\frac{25}{3}$. Therefore the plot of $y=3x^{2}+4x-7$, $x\geq -2/3$ is given by the following graph. 
 
\psset{xunit=0.5cm, yunit=0.5cm}
\begin{pspicture}(-5, -5)(5,5) 
\psframe*[linecolor=white](-5,-5)(5,5) 
\psaxes[ticks=none, labels=none]{<->}(0,0)(-9,-9)(4.5,4.5)
%Function formula: -7+4 (x)+3 ((x)^{2}) 
\psplot[linecolor=red, plotpoints=1000]{-0.66}{1.5}{x 2 exp 3 mul x 4 mul add -7 add }
\psplot[linestyle=dashed, linecolor=gray!50, plotpoints=1000]{-2.82}{-0.67}{x 2 exp 3 mul x 4 mul add -7 add }
\psline[linecolor=blue, linestyle=dashed](-4.5, -4.5)(3,3)
%Function formula: 1/3 (sqrt{}(3 (x)+25))-2/3 
\psplot[linecolor=red, plotpoints=1000]{-8.33333}{5.75}{-0.666667 25 x 3 mul add sqrt 0.333333 mul add }
\psplot[linecolor=gray!50, linestyle=dashed, plotpoints=1000]{-8.33333}{5.75}{-0.666667 25 x 3 mul add sqrt -0.333333 mul add }
\rput[lb](3, 0.3){$f^{-1}(x)= \frac{-2+\sqrt{25+3x}}{3} $}
\rput[l](1.4, 4){$f(x)=3x^2+4x-7  $, $x\geq -2/3$}
\pscircle*(-0.666667,-8.33333){0.05}
\rput[l](2,-7){parabola vertex}
\pscurve[linestyle=dotted]{->}(2, -7)(1, -8.5)(-0.666667,-8.33333)
\rput (-3, -8 ){$(-\frac{2}{3}, -\frac{25}{3})$}

\pscircle*(-8.33333, -0.666667){0.05}
\rput (-8, -2 ){$(-\frac{25}{3}, -\frac{2}{3})$}
\end{pspicture} 

\end{document}