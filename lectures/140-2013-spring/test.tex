\documentclass%
%[handout]
{beamer}
% % % % % % % %
% % % % % % % %
% % % % % % % %
%IMPORTANT
%compiles with 
%pdflatex -shell-escape 
%IMPORTANT
% % % % % % % %
% % % % % % % %
% % % % % % % %
\mode<presentation>
{
\useinnertheme{rounded}
\useoutertheme{infolines}
\usecolortheme{orchid}
\usecolortheme{whale}
}

\usepackage[english]{babel}
\usepackage[latin1]{inputenc}
\usepackage[all,cmtip]{xy}
\usepackage{times}
\usepackage[T1]{fontenc}
\usepackage{../example-templates}
\usepackage{auto-pst-pdf}
\usepackage{pst-plot}
\usepackage{cancel}


% Or whatever. Note that the encoding and the font should match. If T1
% does not look nice, try deleting the line with the fontenc.

\graphicspath{{../../modules/}}

\newtheoremstyle{partialproof}{3pt}{3pt}{}{}{}{.}{.5em}{}
\theoremstyle{partialproof} \newtheorem{partialproof}[theorem]{Proof.}
%\DeclareMathOperator{\diff}{d}
\newcommand{\diff}{\text{d}}
\setbeamertemplate{navigation symbols}{}

\includeonlylecture{1}

\newcommand{\lect}[3]{
  \date{#1}
  \lecture[#1]{#2}{#3}
}

\setbeamertemplate{footline}
{
  \leavevmode%
  \hbox{%
  \begin{beamercolorbox}[wd=.333333\paperwidth,ht=2.25ex,dp=1ex,center]{author in head/foot}%
    \usebeamerfont{author in head/foot}\insertshortauthor
  \end{beamercolorbox}%
  \begin{beamercolorbox}[wd=.333333\paperwidth,ht=2.25ex,dp=1ex,center]{title in head/foot}%
    \usebeamerfont{title in head/foot}\insertshorttitle
  \end{beamercolorbox}%
  \begin{beamercolorbox}[wd=.333333\paperwidth,ht=2.25ex,dp=1ex,center]{date in head/foot}%
    \usebeamerfont{date in head/foot}\insertshortdate{}
  \end{beamercolorbox}}%
  \vskip0pt%
}

% If you have a file called "university-logo-filename.xxx", where xxx
% is a graphic format that can be processed by latex or pdflatex,
% resp., then you can add a logo as follows:

%\pgfdeclareimage[height=0.8cm]{logo}{bluelogo}
%\logo{\pgfuseimage{logo}}

\begin{document}
\newcommand{\psHollowDot}[2]{
\pscircle*[fillcolor=white, linecolor=red](#1, #2){0.07}
\pscircle*[fillcolor=white, linecolor=white](#1, #2){0.04}
}
\newcommand{\psHollowDotBlue}[2]{
\pscircle*[fillcolor=white, linecolor=blue](#1, #2){0.07}
\pscircle*[fillcolor=white, linecolor=white](#1, #2){0.04}
}
\newcommand{\psFullDot}[2]{
\pscircle*[fillcolor=white, linecolor=red](#1, #2){0.07}
}
\newcommand{\psFullDotBlack}[2]{
\pscircle*[fillcolor=white, linecolor=black](#1, #2){0.07}
}
\newcommand{\psFullDotBlue}[2]{
\pscircle*[fillcolor=white, linecolor=blue](#1, #2){0.07}
}
\newcommand{\psLabelXOne}{\psline(1, -0.1)(1,0.1) \rput[t](1, -0.2 ) { $1$} }
\newcommand{\psLabelYOne}{\psline(-0.1, 1)(0.1, 1) \rput[r](-0.2, 1 ) { $1$} }

\AtBeginLecture{%

\title[\insertlecture]{FreeCalc}
\subtitle{\insertlecture}
\author[FreeCalc]{}
\institute[UMass Boston]{University of Massachusetts Boston}
\date{\insertshortlecture}
\begin{frame}
  \titlepage
\end{frame}
}%

% begin lecture
\lect{\today}{Sample}{1}
% begin module areas-intro


\begin{frame}
\frametitle{The Area Problem}
\begin{itemize}
\item  How can we find the area under $y = x^2$ between $x = 0$ and $x = 1$?
\item<handout:2-| 2->  We can approximate it using rectangles.
\item<handout:2-| 3->  Divide $[0,1]$ into three strips of width $\frac{1}{3}$, and draw rectangles in those strips, the heights of which are the same as the height of the function at the right end of that strip.
\item<handout:3-| 4->  Four strips gives a better approximation. \uncover<handout:0| 5->{Five is even better.}
\item<handout:6-| 11-19>  We could use the left endpoints to find the heights instead.
\end{itemize}

\psset{xunit=2cm, yunit=2cm}
\begin{pspicture}(-5, -5)(5,5) 
\psline{linecolor=red!1}(0, -0.3)(0, -0.31)
\psframe*[linecolor=white](-5,-5)(5,5) 
\psaxes[ticks=none, labels=none]{<->}(0,0)(-0.1,-0.1)(1.4,1.4)
\tiny
%Function formula: x^{2} 
\rput(0.9,1.3){$y=x^{2}$} 
\uncover<3,12>{
%approximation 1/3
\psline*[linecolor=cyan, linewidth=0.1pt](0.333333, 0)(0.333333, 0.111111)(0, 0.111111)(0, 0)(0.666667, 0)(0.666667, 0.444444)(0.333333, 0.444444)(0.333333, 0)(1, 0)(1, 1)(0.666667, 1)(0.666667, 0)
\psline[linecolor=blue, linewidth=0.1pt](0.333333, 0)(0.333333, 0.111111)(0, 0.111111)(0, 0)(0.666667, 0)(0.666667, 0.444444)(0.333333, 0.444444)(0.333333, 0)(1, 0)(1, 1)(0.666667, 1)(0.666667, 0)
\rput[t](0.333333,-0.03){$\frac{1}{3}$}\rput[t](0.666667,-0.03){$\frac{2}{3}$}\rput[t](1,-0.03){$1$}
}
\uncover<4,13>{
%approximation 1/4
\psline*[linecolor=cyan, linewidth=0.1pt](0.25, 0)(0.25, 0.0625)(0, 0.0625)(0, 0)(0.5, 0)(0.5, 0.25)(0.25, 0.25)(0.25, 0)(0.75, 0)(0.75, 0.5625)(0.5, 0.5625)(0.5, 0)(1, 0)(1, 1)(0.75, 1)(0.75, 0)
\psline[linecolor=blue, linewidth=0.1pt](0.25, 0)(0.25, 0.0625)(0, 0.0625)(0, 0)(0.5, 0)(0.5, 0.25)(0.25, 0.25)(0.25, 0)(0.75, 0)(0.75, 0.5625)(0.5, 0.5625)(0.5, 0)(1, 0)(1, 1)(0.75, 1)(0.75, 0)
\rput[t](0.25,-0.03){$\frac{1}{4}$}\rput[t](0.5,-0.03){$\frac{1}{2}$}\rput[t](0.75,-0.03){$\frac{3}{4}$}\rput[t](1,-0.03){$1$}
}
\uncover<5,14>{
%approximation 1/5
\psline*[linecolor=cyan, linewidth=0.1pt](0.2, 0)(0.2, 0.04)(0, 0.04)(0, 0)(0.4, 0)(0.4, 0.16)(0.2, 0.16)(0.2, 0)(0.6, 0)(0.6, 0.36)(0.4, 0.36)(0.4, 0)(0.8, 0)(0.8, 0.64)(0.6, 0.64)(0.6, 0)(1, 0)(1, 1)(0.8, 1)(0.8, 0)
\psline[linecolor=blue, linewidth=0.1pt](0.2, 0)(0.2, 0.04)(0, 0.04)(0, 0)(0.4, 0)(0.4, 0.16)(0.2, 0.16)(0.2, 0)(0.6, 0)(0.6, 0.36)(0.4, 0.36)(0.4, 0)(0.8, 0)(0.8, 0.64)(0.6, 0.64)(0.6, 0)(1, 0)(1, 1)(0.8, 1)(0.8, 0)
\rput[t](0.2,-0.03){$\frac{1}{5}$}\rput[t](0.4,-0.03){$\frac{2}{5}$}\rput[t](0.6,-0.03){$\frac{3}{5}$}\rput[t](0.8,-0.03){$\frac{4}{5}$}\rput[t](1,-0.03){$1$}
}
\uncover<6,15>{
%approximation 1/8
\psline*[linecolor=cyan, linewidth=0.1pt](0.125, 0)(0.125, 0.015625)(0, 0.015625)(0, 0)(0.25, 0)(0.25, 0.0625)(0.125, 0.0625)(0.125, 0)(0.375, 0)(0.375, 0.140625)(0.25, 0.140625)(0.25, 0)(0.5, 0)(0.5, 0.25)(0.375, 0.25)(0.375, 0)(0.625, 0)(0.625, 0.390625)(0.5, 0.390625)(0.5, 0)(0.75, 0)(0.75, 0.5625)(0.625, 0.5625)(0.625, 0)(0.875, 0)(0.875, 0.765625)(0.75, 0.765625)(0.75, 0)(1, 0)(1, 1)(0.875, 1)(0.875, 0)
\psline[linecolor=blue, linewidth=0.1pt](0.125, 0)(0.125, 0.015625)(0, 0.015625)(0, 0)(0.25, 0)(0.25, 0.0625)(0.125, 0.0625)(0.125, 0)(0.375, 0)(0.375, 0.140625)(0.25, 0.140625)(0.25, 0)(0.5, 0)(0.5, 0.25)(0.375, 0.25)(0.375, 0)(0.625, 0)(0.625, 0.390625)(0.5, 0.390625)(0.5, 0)(0.75, 0)(0.75, 0.5625)(0.625, 0.5625)(0.625, 0)(0.875, 0)(0.875, 0.765625)(0.75, 0.765625)(0.75, 0)(1, 0)(1, 1)(0.875, 1)(0.875, 0)
\rput[t](0.125,-0.03){$\frac{1}{8}$}\rput[t](0.25,-0.03){$\frac{1}{4}$}\rput[t](0.375,-0.03){$\frac{3}{8}$}\rput[t](0.5,-0.03){$\frac{1}{2}$}\rput[t](0.625,-0.03){$\frac{5}{8}$}\rput[t](0.75,-0.03){$\frac{3}{4}$}\rput[t](0.875,-0.03){$\frac{7}{8}$}\rput[t](1,-0.03){$1$}
}
\uncover<7,16>{
%approximation 1/10
\psline*[linecolor=cyan, linewidth=0.1pt](0.1, 0)(0.1, 0.01)(0, 0.01)(0, 0)(0.2, 0)(0.2, 0.04)(0.1, 0.04)(0.1, 0)(0.3, 0)(0.3, 0.09)(0.2, 0.09)(0.2, 0)(0.4, 0)(0.4, 0.16)(0.3, 0.16)(0.3, 0)(0.5, 0)(0.5, 0.25)(0.4, 0.25)(0.4, 0)(0.6, 0)(0.6, 0.36)(0.5, 0.36)(0.5, 0)(0.7, 0)(0.7, 0.49)(0.6, 0.49)(0.6, 0)(0.8, 0)(0.8, 0.64)(0.7, 0.64)(0.7, 0)(0.9, 0)(0.9, 0.81)(0.8, 0.81)(0.8, 0)(1, 0)(1, 1)(0.9, 1)(0.9, 0)
\psline[linecolor=blue, linewidth=0.1pt](0.1, 0)(0.1, 0.01)(0, 0.01)(0, 0)(0.2, 0)(0.2, 0.04)(0.1, 0.04)(0.1, 0)(0.3, 0)(0.3, 0.09)(0.2, 0.09)(0.2, 0)(0.4, 0)(0.4, 0.16)(0.3, 0.16)(0.3, 0)(0.5, 0)(0.5, 0.25)(0.4, 0.25)(0.4, 0)(0.6, 0)(0.6, 0.36)(0.5, 0.36)(0.5, 0)(0.7, 0)(0.7, 0.49)(0.6, 0.49)(0.6, 0)(0.8, 0)(0.8, 0.64)(0.7, 0.64)(0.7, 0)(0.9, 0)(0.9, 0.81)(0.8, 0.81)(0.8, 0)(1, 0)(1, 1)(0.9, 1)(0.9, 0)
\rput[t](0.1,-0.03){$\frac{1}{10}$}\rput[t](0.2,-0.03){$\frac{1}{5}$}\rput[t](0.3,-0.03){$\frac{3}{10}$}\rput[t](0.4,-0.03){$\frac{2}{5}$}\rput[t](0.5,-0.03){$\frac{1}{2}$}\rput[t](0.6,-0.03){$\frac{3}{5}$}\rput[t](0.7,-0.03){$\frac{7}{10}$}\rput[t](0.8,-0.03){$\frac{4}{5}$}\rput[t](0.9,-0.03){$\frac{9}{10}$}\rput[t](1,-0.03){$1$}
}
\uncover<8,17>{
%approximation 1/20
\psline*[linecolor=cyan, linewidth=0.1pt](0.05, 0)(0.05, 0.0025)(0, 0.0025)(0, 0)(0.1, 0)(0.1, 0.01)(0.05, 0.01)(0.05, 0)(0.15, 0)(0.15, 0.0225)(0.1, 0.0225)(0.1, 0)(0.2, 0)(0.2, 0.04)(0.15, 0.04)(0.15, 0)(0.25, 0)(0.25, 0.0625)(0.2, 0.0625)(0.2, 0)(0.3, 0)(0.3, 0.09)(0.25, 0.09)(0.25, 0)(0.35, 0)(0.35, 0.1225)(0.3, 0.1225)(0.3, 0)(0.4, 0)(0.4, 0.16)(0.35, 0.16)(0.35, 0)(0.45, 0)(0.45, 0.2025)(0.4, 0.2025)(0.4, 0)(0.5, 0)(0.5, 0.25)(0.45, 0.25)(0.45, 0)(0.55, 0)(0.55, 0.3025)(0.5, 0.3025)(0.5, 0)(0.6, 0)(0.6, 0.36)(0.55, 0.36)(0.55, 0)(0.65, 0)(0.65, 0.4225)(0.6, 0.4225)(0.6, 0)(0.7, 0)(0.7, 0.49)(0.65, 0.49)(0.65, 0)(0.75, 0)(0.75, 0.5625)(0.7, 0.5625)(0.7, 0)(0.8, 0)(0.8, 0.64)(0.75, 0.64)(0.75, 0)(0.85, 0)(0.85, 0.7225)(0.8, 0.7225)(0.8, 0)(0.9, 0)(0.9, 0.81)(0.85, 0.81)(0.85, 0)(0.95, 0)(0.95, 0.9025)(0.9, 0.9025)(0.9, 0)(1, 0)(1, 1)(0.95, 1)(0.95, 0)
\psline[linecolor=blue, linewidth=0.1pt](0.05, 0)(0.05, 0.0025)(0, 0.0025)(0, 0)(0.1, 0)(0.1, 0.01)(0.05, 0.01)(0.05, 0)(0.15, 0)(0.15, 0.0225)(0.1, 0.0225)(0.1, 0)(0.2, 0)(0.2, 0.04)(0.15, 0.04)(0.15, 0)(0.25, 0)(0.25, 0.0625)(0.2, 0.0625)(0.2, 0)(0.3, 0)(0.3, 0.09)(0.25, 0.09)(0.25, 0)(0.35, 0)(0.35, 0.1225)(0.3, 0.1225)(0.3, 0)(0.4, 0)(0.4, 0.16)(0.35, 0.16)(0.35, 0)(0.45, 0)(0.45, 0.2025)(0.4, 0.2025)(0.4, 0)(0.5, 0)(0.5, 0.25)(0.45, 0.25)(0.45, 0)(0.55, 0)(0.55, 0.3025)(0.5, 0.3025)(0.5, 0)(0.6, 0)(0.6, 0.36)(0.55, 0.36)(0.55, 0)(0.65, 0)(0.65, 0.4225)(0.6, 0.4225)(0.6, 0)(0.7, 0)(0.7, 0.49)(0.65, 0.49)(0.65, 0)(0.75, 0)(0.75, 0.5625)(0.7, 0.5625)(0.7, 0)(0.8, 0)(0.8, 0.64)(0.75, 0.64)(0.75, 0)(0.85, 0)(0.85, 0.7225)(0.8, 0.7225)(0.8, 0)(0.9, 0)(0.9, 0.81)(0.85, 0.81)(0.85, 0)(0.95, 0)(0.95, 0.9025)(0.9, 0.9025)(0.9, 0)(1, 0)(1, 1)(0.95, 1)(0.95, 0)
}
\uncover<9,18>{
%approximation 1/30
\psline*[linecolor=cyan, linewidth=0.1pt](0.0333333, 0)(0.0333333, 0.00111111)(0, 0.00111111)(0, 0)(0.0666667, 0)(0.0666667, 0.00444444)(0.0333333, 0.00444444)(0.0333333, 0)(0.1, 0)(0.1, 0.01)(0.0666667, 0.01)(0.0666667, 0)(0.133333, 0)(0.133333, 0.0177778)(0.1, 0.0177778)(0.1, 0)(0.166667, 0)(0.166667, 0.0277778)(0.133333, 0.0277778)(0.133333, 0)(0.2, 0)(0.2, 0.04)(0.166667, 0.04)(0.166667, 0)(0.233333, 0)(0.233333, 0.0544444)(0.2, 0.0544444)(0.2, 0)(0.266667, 0)(0.266667, 0.0711111)(0.233333, 0.0711111)(0.233333, 0)(0.3, 0)(0.3, 0.09)(0.266667, 0.09)(0.266667, 0)(0.333333, 0)(0.333333, 0.111111)(0.3, 0.111111)(0.3, 0)(0.366667, 0)(0.366667, 0.134444)(0.333333, 0.134444)(0.333333, 0)(0.4, 0)(0.4, 0.16)(0.366667, 0.16)(0.366667, 0)(0.433333, 0)(0.433333, 0.187778)(0.4, 0.187778)(0.4, 0)(0.466667, 0)(0.466667, 0.217778)(0.433333, 0.217778)(0.433333, 0)(0.5, 0)(0.5, 0.25)(0.466667, 0.25)(0.466667, 0)(0.533333, 0)(0.533333, 0.284444)(0.5, 0.284444)(0.5, 0)(0.566667, 0)(0.566667, 0.321111)(0.533333, 0.321111)(0.533333, 0)(0.6, 0)(0.6, 0.36)(0.566667, 0.36)(0.566667, 0)(0.633333, 0)(0.633333, 0.401111)(0.6, 0.401111)(0.6, 0)(0.666667, 0)(0.666667, 0.444444)(0.633333, 0.444444)(0.633333, 0)(0.7, 0)(0.7, 0.49)(0.666667, 0.49)(0.666667, 0)(0.733333, 0)(0.733333, 0.537778)(0.7, 0.537778)(0.7, 0)(0.766667, 0)(0.766667, 0.587778)(0.733333, 0.587778)(0.733333, 0)(0.8, 0)(0.8, 0.64)(0.766667, 0.64)(0.766667, 0)(0.833333, 0)(0.833333, 0.694444)(0.8, 0.694444)(0.8, 0)(0.866667, 0)(0.866667, 0.751111)(0.833333, 0.751111)(0.833333, 0)(0.9, 0)(0.9, 0.81)(0.866667, 0.81)(0.866667, 0)(0.933333, 0)(0.933333, 0.871111)(0.9, 0.871111)(0.9, 0)(0.966667, 0)(0.966667, 0.934444)(0.933333, 0.934444)(0.933333, 0)(1, 0)(1, 1)(0.966667, 1)(0.966667, 0)
\psline[linecolor=blue, linewidth=0.1pt](0.0333333, 0)(0.0333333, 0.00111111)(0, 0.00111111)(0, 0)(0.0666667, 0)(0.0666667, 0.00444444)(0.0333333, 0.00444444)(0.0333333, 0)(0.1, 0)(0.1, 0.01)(0.0666667, 0.01)(0.0666667, 0)(0.133333, 0)(0.133333, 0.0177778)(0.1, 0.0177778)(0.1, 0)(0.166667, 0)(0.166667, 0.0277778)(0.133333, 0.0277778)(0.133333, 0)(0.2, 0)(0.2, 0.04)(0.166667, 0.04)(0.166667, 0)(0.233333, 0)(0.233333, 0.0544444)(0.2, 0.0544444)(0.2, 0)(0.266667, 0)(0.266667, 0.0711111)(0.233333, 0.0711111)(0.233333, 0)(0.3, 0)(0.3, 0.09)(0.266667, 0.09)(0.266667, 0)(0.333333, 0)(0.333333, 0.111111)(0.3, 0.111111)(0.3, 0)(0.366667, 0)(0.366667, 0.134444)(0.333333, 0.134444)(0.333333, 0)(0.4, 0)(0.4, 0.16)(0.366667, 0.16)(0.366667, 0)(0.433333, 0)(0.433333, 0.187778)(0.4, 0.187778)(0.4, 0)(0.466667, 0)(0.466667, 0.217778)(0.433333, 0.217778)(0.433333, 0)(0.5, 0)(0.5, 0.25)(0.466667, 0.25)(0.466667, 0)(0.533333, 0)(0.533333, 0.284444)(0.5, 0.284444)(0.5, 0)(0.566667, 0)(0.566667, 0.321111)(0.533333, 0.321111)(0.533333, 0)(0.6, 0)(0.6, 0.36)(0.566667, 0.36)(0.566667, 0)(0.633333, 0)(0.633333, 0.401111)(0.6, 0.401111)(0.6, 0)(0.666667, 0)(0.666667, 0.444444)(0.633333, 0.444444)(0.633333, 0)(0.7, 0)(0.7, 0.49)(0.666667, 0.49)(0.666667, 0)(0.733333, 0)(0.733333, 0.537778)(0.7, 0.537778)(0.7, 0)(0.766667, 0)(0.766667, 0.587778)(0.733333, 0.587778)(0.733333, 0)(0.8, 0)(0.8, 0.64)(0.766667, 0.64)(0.766667, 0)(0.833333, 0)(0.833333, 0.694444)(0.8, 0.694444)(0.8, 0)(0.866667, 0)(0.866667, 0.751111)(0.833333, 0.751111)(0.833333, 0)(0.9, 0)(0.9, 0.81)(0.866667, 0.81)(0.866667, 0)(0.933333, 0)(0.933333, 0.871111)(0.9, 0.871111)(0.9, 0)(0.966667, 0)(0.966667, 0.934444)(0.933333, 0.934444)(0.933333, 0)(1, 0)(1, 1)(0.966667, 1)(0.966667, 0)
}
\uncover<10,19>{
%approximation 1/40
\psline*[linecolor=cyan, linewidth=0.1pt](0.025, 0)(0.025, 0.000625)(0, 0.000625)(0, 0)(0.05, 0)(0.05, 0.0025)(0.025, 0.0025)(0.025, 0)(0.075, 0)(0.075, 0.005625)(0.05, 0.005625)(0.05, 0)(0.1, 0)(0.1, 0.01)(0.075, 0.01)(0.075, 0)(0.125, 0)(0.125, 0.015625)(0.1, 0.015625)(0.1, 0)(0.15, 0)(0.15, 0.0225)(0.125, 0.0225)(0.125, 0)(0.175, 0)(0.175, 0.030625)(0.15, 0.030625)(0.15, 0)(0.2, 0)(0.2, 0.04)(0.175, 0.04)(0.175, 0)(0.225, 0)(0.225, 0.050625)(0.2, 0.050625)(0.2, 0)(0.25, 0)(0.25, 0.0625)(0.225, 0.0625)(0.225, 0)(0.275, 0)(0.275, 0.075625)(0.25, 0.075625)(0.25, 0)(0.3, 0)(0.3, 0.09)(0.275, 0.09)(0.275, 0)(0.325, 0)(0.325, 0.105625)(0.3, 0.105625)(0.3, 0)(0.35, 0)(0.35, 0.1225)(0.325, 0.1225)(0.325, 0)(0.375, 0)(0.375, 0.140625)(0.35, 0.140625)(0.35, 0)(0.4, 0)(0.4, 0.16)(0.375, 0.16)(0.375, 0)(0.425, 0)(0.425, 0.180625)(0.4, 0.180625)(0.4, 0)(0.45, 0)(0.45, 0.2025)(0.425, 0.2025)(0.425, 0)(0.475, 0)(0.475, 0.225625)(0.45, 0.225625)(0.45, 0)(0.5, 0)(0.5, 0.25)(0.475, 0.25)(0.475, 0)(0.525, 0)(0.525, 0.275625)(0.5, 0.275625)(0.5, 0)(0.55, 0)(0.55, 0.3025)(0.525, 0.3025)(0.525, 0)(0.575, 0)(0.575, 0.330625)(0.55, 0.330625)(0.55, 0)(0.6, 0)(0.6, 0.36)(0.575, 0.36)(0.575, 0)(0.625, 0)(0.625, 0.390625)(0.6, 0.390625)(0.6, 0)(0.65, 0)(0.65, 0.4225)(0.625, 0.4225)(0.625, 0)(0.675, 0)(0.675, 0.455625)(0.65, 0.455625)(0.65, 0)(0.7, 0)(0.7, 0.49)(0.675, 0.49)(0.675, 0)(0.725, 0)(0.725, 0.525625)(0.7, 0.525625)(0.7, 0)(0.75, 0)(0.75, 0.5625)(0.725, 0.5625)(0.725, 0)(0.775, 0)(0.775, 0.600625)(0.75, 0.600625)(0.75, 0)(0.8, 0)(0.8, 0.64)(0.775, 0.64)(0.775, 0)(0.825, 0)(0.825, 0.680625)(0.8, 0.680625)(0.8, 0)(0.85, 0)(0.85, 0.7225)(0.825, 0.7225)(0.825, 0)(0.875, 0)(0.875, 0.765625)(0.85, 0.765625)(0.85, 0)(0.9, 0)(0.9, 0.81)(0.875, 0.81)(0.875, 0)(0.925, 0)(0.925, 0.855625)(0.9, 0.855625)(0.9, 0)(0.95, 0)(0.95, 0.9025)(0.925, 0.9025)(0.925, 0)(0.975, 0)(0.975, 0.950625)(0.95, 0.950625)(0.95, 0)(1, 0)(1, 1)(0.975, 1)(0.975, 0)
\psline[linecolor=blue, linewidth=0.1pt](0.025, 0)(0.025, 0.000625)(0, 0.000625)(0, 0)(0.05, 0)(0.05, 0.0025)(0.025, 0.0025)(0.025, 0)(0.075, 0)(0.075, 0.005625)(0.05, 0.005625)(0.05, 0)(0.1, 0)(0.1, 0.01)(0.075, 0.01)(0.075, 0)(0.125, 0)(0.125, 0.015625)(0.1, 0.015625)(0.1, 0)(0.15, 0)(0.15, 0.0225)(0.125, 0.0225)(0.125, 0)(0.175, 0)(0.175, 0.030625)(0.15, 0.030625)(0.15, 0)(0.2, 0)(0.2, 0.04)(0.175, 0.04)(0.175, 0)(0.225, 0)(0.225, 0.050625)(0.2, 0.050625)(0.2, 0)(0.25, 0)(0.25, 0.0625)(0.225, 0.0625)(0.225, 0)(0.275, 0)(0.275, 0.075625)(0.25, 0.075625)(0.25, 0)(0.3, 0)(0.3, 0.09)(0.275, 0.09)(0.275, 0)(0.325, 0)(0.325, 0.105625)(0.3, 0.105625)(0.3, 0)(0.35, 0)(0.35, 0.1225)(0.325, 0.1225)(0.325, 0)(0.375, 0)(0.375, 0.140625)(0.35, 0.140625)(0.35, 0)(0.4, 0)(0.4, 0.16)(0.375, 0.16)(0.375, 0)(0.425, 0)(0.425, 0.180625)(0.4, 0.180625)(0.4, 0)(0.45, 0)(0.45, 0.2025)(0.425, 0.2025)(0.425, 0)(0.475, 0)(0.475, 0.225625)(0.45, 0.225625)(0.45, 0)(0.5, 0)(0.5, 0.25)(0.475, 0.25)(0.475, 0)(0.525, 0)(0.525, 0.275625)(0.5, 0.275625)(0.5, 0)(0.55, 0)(0.55, 0.3025)(0.525, 0.3025)(0.525, 0)(0.575, 0)(0.575, 0.330625)(0.55, 0.330625)(0.55, 0)(0.6, 0)(0.6, 0.36)(0.575, 0.36)(0.575, 0)(0.625, 0)(0.625, 0.390625)(0.6, 0.390625)(0.6, 0)(0.65, 0)(0.65, 0.4225)(0.625, 0.4225)(0.625, 0)(0.675, 0)(0.675, 0.455625)(0.65, 0.455625)(0.65, 0)(0.7, 0)(0.7, 0.49)(0.675, 0.49)(0.675, 0)(0.725, 0)(0.725, 0.525625)(0.7, 0.525625)(0.7, 0)(0.75, 0)(0.75, 0.5625)(0.725, 0.5625)(0.725, 0)(0.775, 0)(0.775, 0.600625)(0.75, 0.600625)(0.75, 0)(0.8, 0)(0.8, 0.64)(0.775, 0.64)(0.775, 0)(0.825, 0)(0.825, 0.680625)(0.8, 0.680625)(0.8, 0)(0.85, 0)(0.85, 0.7225)(0.825, 0.7225)(0.825, 0)(0.875, 0)(0.875, 0.765625)(0.85, 0.765625)(0.85, 0)(0.9, 0)(0.9, 0.81)(0.875, 0.81)(0.875, 0)(0.925, 0)(0.925, 0.855625)(0.9, 0.855625)(0.9, 0)(0.95, 0)(0.95, 0.9025)(0.925, 0.9025)(0.925, 0)(0.975, 0)(0.975, 0.950625)(0.95, 0.950625)(0.95, 0)(1, 0)(1, 1)(0.975, 1)(0.975, 0)
}
\uncover<1,2, 11>{
\pscustom*[linecolor=cyan, linewidth=0.1pt]{\psplot[linecolor=red, plotpoints=1000]{0}{1}{x 2 exp }\psline(1,1)(1,0)}
}
\psplot[linecolor=red, plotpoints=1000]{0}{1.15}{x 2 exp }
\end{pspicture}
\
\uncover<11->{
\psset{xunit=2cm, yunit=2cm}
\begin{pspicture}(-5, -5)(5,5) 
\psline{linecolor=red!1}(0, -0.3)(0, -0.031)
\psframe*[linecolor=white](-5,-5)(5,5) 
\psaxes[ticks=none, labels=none]{<->}(0,0)(-0.1,-0.1)(1.4,1.4)
\tiny
%Function formula: x^{2} 
\rput(0.9,1.3){$y=x^{2}$} 
\uncover<12>{
%approximation 1/3
\psline*[linecolor=cyan, linewidth=0.1pt](0, 0)(0, 0)(0.333333, 0)(0.333333, 0)(0.333333, 0)(0.333333, 0.111111)(0.666667, 0.111111)(0.666667, 0)(0.666667, 0)(0.666667, 0.444444)(1, 0.444444)(1, 0)
\psline[linecolor=blue, linewidth=0.1pt](0, 0)(0, 0)(0.333333, 0)(0.333333, 0)(0.333333, 0)(0.333333, 0.111111)(0.666667, 0.111111)(0.666667, 0)(0.666667, 0)(0.666667, 0.444444)(1, 0.444444)(1, 0)
\rput[t](0,-0.03){$0$}\rput[t](0.333333,-0.03){$\frac{1}{3}$}\rput[t](0.666667,-0.03){$\frac{2}{3}$}\rput[t](1,-0.03){$1$}
}
\uncover<13>{
%approximation 1/4
\psline*[linecolor=cyan, linewidth=0.1pt](0, 0)(0, 0)(0.25, 0)(0.25, 0)(0.25, 0)(0.25, 0.0625)(0.5, 0.0625)(0.5, 0)(0.5, 0)(0.5, 0.25)(0.75, 0.25)(0.75, 0)(0.75, 0)(0.75, 0.5625)(1, 0.5625)(1, 0)
\psline[linecolor=blue, linewidth=0.1pt](0, 0)(0, 0)(0.25, 0)(0.25, 0)(0.25, 0)(0.25, 0.0625)(0.5, 0.0625)(0.5, 0)(0.5, 0)(0.5, 0.25)(0.75, 0.25)(0.75, 0)(0.75, 0)(0.75, 0.5625)(1, 0.5625)(1, 0)
\rput[t](0,-0.03){$0$}\rput[t](0.25,-0.03){$\frac{1}{4}$}\rput[t](0.5,-0.03){$\frac{1}{2}$}\rput[t](0.75,-0.03){$\frac{3}{4}$}\rput[t](1,-0.03){$1$}
}
\uncover<14>{
%approximation 1/5
\psline*[linecolor=cyan, linewidth=0.1pt](0, 0)(0, 0)(0.2, 0)(0.2, 0)(0.2, 0)(0.2, 0.04)(0.4, 0.04)(0.4, 0)(0.4, 0)(0.4, 0.16)(0.6, 0.16)(0.6, 0)(0.6, 0)(0.6, 0.36)(0.8, 0.36)(0.8, 0)(0.8, 0)(0.8, 0.64)(1, 0.64)(1, 0)
\psline[linecolor=blue, linewidth=0.1pt](0, 0)(0, 0)(0.2, 0)(0.2, 0)(0.2, 0)(0.2, 0.04)(0.4, 0.04)(0.4, 0)(0.4, 0)(0.4, 0.16)(0.6, 0.16)(0.6, 0)(0.6, 0)(0.6, 0.36)(0.8, 0.36)(0.8, 0)(0.8, 0)(0.8, 0.64)(1, 0.64)(1, 0)
\rput[t](0,-0.03){$0$}\rput[t](0.2,-0.03){$\frac{1}{5}$}\rput[t](0.4,-0.03){$\frac{2}{5}$}\rput[t](0.6,-0.03){$\frac{3}{5}$}\rput[t](0.8,-0.03){$\frac{4}{5}$}\rput[t](1,-0.03){$1$}
}
\uncover<15>{
%approximation 1/8
\psline*[linecolor=cyan, linewidth=0.1pt](0, 0)(0, 0)(0.125, 0)(0.125, 0)(0.125, 0)(0.125, 0.015625)(0.25, 0.015625)(0.25, 0)(0.25, 0)(0.25, 0.0625)(0.375, 0.0625)(0.375, 0)(0.375, 0)(0.375, 0.140625)(0.5, 0.140625)(0.5, 0)(0.5, 0)(0.5, 0.25)(0.625, 0.25)(0.625, 0)(0.625, 0)(0.625, 0.390625)(0.75, 0.390625)(0.75, 0)(0.75, 0)(0.75, 0.5625)(0.875, 0.5625)(0.875, 0)(0.875, 0)(0.875, 0.765625)(1, 0.765625)(1, 0)
\psline[linecolor=blue, linewidth=0.1pt](0, 0)(0, 0)(0.125, 0)(0.125, 0)(0.125, 0)(0.125, 0.015625)(0.25, 0.015625)(0.25, 0)(0.25, 0)(0.25, 0.0625)(0.375, 0.0625)(0.375, 0)(0.375, 0)(0.375, 0.140625)(0.5, 0.140625)(0.5, 0)(0.5, 0)(0.5, 0.25)(0.625, 0.25)(0.625, 0)(0.625, 0)(0.625, 0.390625)(0.75, 0.390625)(0.75, 0)(0.75, 0)(0.75, 0.5625)(0.875, 0.5625)(0.875, 0)(0.875, 0)(0.875, 0.765625)(1, 0.765625)(1, 0)
\rput[t](0,-0.03){$0$}\rput[t](0.125,-0.03){$\frac{1}{8}$}\rput[t](0.25,-0.03){$\frac{1}{4}$}\rput[t](0.375,-0.03){$\frac{3}{8}$}\rput[t](0.5,-0.03){$\frac{1}{2}$}\rput[t](0.625,-0.03){$\frac{5}{8}$}\rput[t](0.75,-0.03){$\frac{3}{4}$}\rput[t](0.875,-0.03){$\frac{7}{8}$}\rput[t](1,-0.03){$1$}
}
\uncover<16>{
%approximation 1/10
\psline*[linecolor=cyan, linewidth=0.1pt](0, 0)(0, 0)(0.1, 0)(0.1, 0)(0.1, 0)(0.1, 0.01)(0.2, 0.01)(0.2, 0)(0.2, 0)(0.2, 0.04)(0.3, 0.04)(0.3, 0)(0.3, 0)(0.3, 0.09)(0.4, 0.09)(0.4, 0)(0.4, 0)(0.4, 0.16)(0.5, 0.16)(0.5, 0)(0.5, 0)(0.5, 0.25)(0.6, 0.25)(0.6, 0)(0.6, 0)(0.6, 0.36)(0.7, 0.36)(0.7, 0)(0.7, 0)(0.7, 0.49)(0.8, 0.49)(0.8, 0)(0.8, 0)(0.8, 0.64)(0.9, 0.64)(0.9, 0)(0.9, 0)(0.9, 0.81)(1, 0.81)(1, 0)
\psline[linecolor=blue, linewidth=0.1pt](0, 0)(0, 0)(0.1, 0)(0.1, 0)(0.1, 0)(0.1, 0.01)(0.2, 0.01)(0.2, 0)(0.2, 0)(0.2, 0.04)(0.3, 0.04)(0.3, 0)(0.3, 0)(0.3, 0.09)(0.4, 0.09)(0.4, 0)(0.4, 0)(0.4, 0.16)(0.5, 0.16)(0.5, 0)(0.5, 0)(0.5, 0.25)(0.6, 0.25)(0.6, 0)(0.6, 0)(0.6, 0.36)(0.7, 0.36)(0.7, 0)(0.7, 0)(0.7, 0.49)(0.8, 0.49)(0.8, 0)(0.8, 0)(0.8, 0.64)(0.9, 0.64)(0.9, 0)(0.9, 0)(0.9, 0.81)(1, 0.81)(1, 0)
\rput[t](0,-0.03){$0$}\rput[t](0.1,-0.03){$\frac{1}{10}$}\rput[t](0.2,-0.03){$\frac{1}{5}$}\rput[t](0.3,-0.03){$\frac{3}{10}$}\rput[t](0.4,-0.03){$\frac{2}{5}$}\rput[t](0.5,-0.03){$\frac{1}{2}$}\rput[t](0.6,-0.03){$\frac{3}{5}$}\rput[t](0.7,-0.03){$\frac{7}{10}$}\rput[t](0.8,-0.03){$\frac{4}{5}$}\rput[t](0.9,-0.03){$\frac{9}{10}$}\rput[t](1,-0.03){$1$}
}
\uncover<17>{
%approximation 1/20
\psline*[linecolor=cyan, linewidth=0.1pt](0, 0)(0, 0)(0.05, 0)(0.05, 0)(0.05, 0)(0.05, 0.0025)(0.1, 0.0025)(0.1, 0)(0.1, 0)(0.1, 0.01)(0.15, 0.01)(0.15, 0)(0.15, 0)(0.15, 0.0225)(0.2, 0.0225)(0.2, 0)(0.2, 0)(0.2, 0.04)(0.25, 0.04)(0.25, 0)(0.25, 0)(0.25, 0.0625)(0.3, 0.0625)(0.3, 0)(0.3, 0)(0.3, 0.09)(0.35, 0.09)(0.35, 0)(0.35, 0)(0.35, 0.1225)(0.4, 0.1225)(0.4, 0)(0.4, 0)(0.4, 0.16)(0.45, 0.16)(0.45, 0)(0.45, 0)(0.45, 0.2025)(0.5, 0.2025)(0.5, 0)(0.5, 0)(0.5, 0.25)(0.55, 0.25)(0.55, 0)(0.55, 0)(0.55, 0.3025)(0.6, 0.3025)(0.6, 0)(0.6, 0)(0.6, 0.36)(0.65, 0.36)(0.65, 0)(0.65, 0)(0.65, 0.4225)(0.7, 0.4225)(0.7, 0)(0.7, 0)(0.7, 0.49)(0.75, 0.49)(0.75, 0)(0.75, 0)(0.75, 0.5625)(0.8, 0.5625)(0.8, 0)(0.8, 0)(0.8, 0.64)(0.85, 0.64)(0.85, 0)(0.85, 0)(0.85, 0.7225)(0.9, 0.7225)(0.9, 0)(0.9, 0)(0.9, 0.81)(0.95, 0.81)(0.95, 0)(0.95, 0)(0.95, 0.9025)(1, 0.9025)(1, 0)
\psline[linecolor=blue, linewidth=0.1pt](0, 0)(0, 0)(0.05, 0)(0.05, 0)(0.05, 0)(0.05, 0.0025)(0.1, 0.0025)(0.1, 0)(0.1, 0)(0.1, 0.01)(0.15, 0.01)(0.15, 0)(0.15, 0)(0.15, 0.0225)(0.2, 0.0225)(0.2, 0)(0.2, 0)(0.2, 0.04)(0.25, 0.04)(0.25, 0)(0.25, 0)(0.25, 0.0625)(0.3, 0.0625)(0.3, 0)(0.3, 0)(0.3, 0.09)(0.35, 0.09)(0.35, 0)(0.35, 0)(0.35, 0.1225)(0.4, 0.1225)(0.4, 0)(0.4, 0)(0.4, 0.16)(0.45, 0.16)(0.45, 0)(0.45, 0)(0.45, 0.2025)(0.5, 0.2025)(0.5, 0)(0.5, 0)(0.5, 0.25)(0.55, 0.25)(0.55, 0)(0.55, 0)(0.55, 0.3025)(0.6, 0.3025)(0.6, 0)(0.6, 0)(0.6, 0.36)(0.65, 0.36)(0.65, 0)(0.65, 0)(0.65, 0.4225)(0.7, 0.4225)(0.7, 0)(0.7, 0)(0.7, 0.49)(0.75, 0.49)(0.75, 0)(0.75, 0)(0.75, 0.5625)(0.8, 0.5625)(0.8, 0)(0.8, 0)(0.8, 0.64)(0.85, 0.64)(0.85, 0)(0.85, 0)(0.85, 0.7225)(0.9, 0.7225)(0.9, 0)(0.9, 0)(0.9, 0.81)(0.95, 0.81)(0.95, 0)(0.95, 0)(0.95, 0.9025)(1, 0.9025)(1, 0)
}
\uncover<18>{
%approximation 1/30
\psline*[linecolor=cyan, linewidth=0.1pt](0, 0)(0, 0)(0.0333333, 0)(0.0333333, 0)(0.0333333, 0)(0.0333333, 0.00111111)(0.0666667, 0.00111111)(0.0666667, 0)(0.0666667, 0)(0.0666667, 0.00444444)(0.1, 0.00444444)(0.1, 0)(0.1, 0)(0.1, 0.01)(0.133333, 0.01)(0.133333, 0)(0.133333, 0)(0.133333, 0.0177778)(0.166667, 0.0177778)(0.166667, 0)(0.166667, 0)(0.166667, 0.0277778)(0.2, 0.0277778)(0.2, 0)(0.2, 0)(0.2, 0.04)(0.233333, 0.04)(0.233333, 0)(0.233333, 0)(0.233333, 0.0544444)(0.266667, 0.0544444)(0.266667, 0)(0.266667, 0)(0.266667, 0.0711111)(0.3, 0.0711111)(0.3, 0)(0.3, 0)(0.3, 0.09)(0.333333, 0.09)(0.333333, 0)(0.333333, 0)(0.333333, 0.111111)(0.366667, 0.111111)(0.366667, 0)(0.366667, 0)(0.366667, 0.134444)(0.4, 0.134444)(0.4, 0)(0.4, 0)(0.4, 0.16)(0.433333, 0.16)(0.433333, 0)(0.433333, 0)(0.433333, 0.187778)(0.466667, 0.187778)(0.466667, 0)(0.466667, 0)(0.466667, 0.217778)(0.5, 0.217778)(0.5, 0)(0.5, 0)(0.5, 0.25)(0.533333, 0.25)(0.533333, 0)(0.533333, 0)(0.533333, 0.284444)(0.566667, 0.284444)(0.566667, 0)(0.566667, 0)(0.566667, 0.321111)(0.6, 0.321111)(0.6, 0)(0.6, 0)(0.6, 0.36)(0.633333, 0.36)(0.633333, 0)(0.633333, 0)(0.633333, 0.401111)(0.666667, 0.401111)(0.666667, 0)(0.666667, 0)(0.666667, 0.444444)(0.7, 0.444444)(0.7, 0)(0.7, 0)(0.7, 0.49)(0.733333, 0.49)(0.733333, 0)(0.733333, 0)(0.733333, 0.537778)(0.766667, 0.537778)(0.766667, 0)(0.766667, 0)(0.766667, 0.587778)(0.8, 0.587778)(0.8, 0)(0.8, 0)(0.8, 0.64)(0.833333, 0.64)(0.833333, 0)(0.833333, 0)(0.833333, 0.694444)(0.866667, 0.694444)(0.866667, 0)(0.866667, 0)(0.866667, 0.751111)(0.9, 0.751111)(0.9, 0)(0.9, 0)(0.9, 0.81)(0.933333, 0.81)(0.933333, 0)(0.933333, 0)(0.933333, 0.871111)(0.966667, 0.871111)(0.966667, 0)(0.966667, 0)(0.966667, 0.934444)(1, 0.934444)(1, 0)
\psline[linecolor=blue, linewidth=0.1pt](0, 0)(0, 0)(0.0333333, 0)(0.0333333, 0)(0.0333333, 0)(0.0333333, 0.00111111)(0.0666667, 0.00111111)(0.0666667, 0)(0.0666667, 0)(0.0666667, 0.00444444)(0.1, 0.00444444)(0.1, 0)(0.1, 0)(0.1, 0.01)(0.133333, 0.01)(0.133333, 0)(0.133333, 0)(0.133333, 0.0177778)(0.166667, 0.0177778)(0.166667, 0)(0.166667, 0)(0.166667, 0.0277778)(0.2, 0.0277778)(0.2, 0)(0.2, 0)(0.2, 0.04)(0.233333, 0.04)(0.233333, 0)(0.233333, 0)(0.233333, 0.0544444)(0.266667, 0.0544444)(0.266667, 0)(0.266667, 0)(0.266667, 0.0711111)(0.3, 0.0711111)(0.3, 0)(0.3, 0)(0.3, 0.09)(0.333333, 0.09)(0.333333, 0)(0.333333, 0)(0.333333, 0.111111)(0.366667, 0.111111)(0.366667, 0)(0.366667, 0)(0.366667, 0.134444)(0.4, 0.134444)(0.4, 0)(0.4, 0)(0.4, 0.16)(0.433333, 0.16)(0.433333, 0)(0.433333, 0)(0.433333, 0.187778)(0.466667, 0.187778)(0.466667, 0)(0.466667, 0)(0.466667, 0.217778)(0.5, 0.217778)(0.5, 0)(0.5, 0)(0.5, 0.25)(0.533333, 0.25)(0.533333, 0)(0.533333, 0)(0.533333, 0.284444)(0.566667, 0.284444)(0.566667, 0)(0.566667, 0)(0.566667, 0.321111)(0.6, 0.321111)(0.6, 0)(0.6, 0)(0.6, 0.36)(0.633333, 0.36)(0.633333, 0)(0.633333, 0)(0.633333, 0.401111)(0.666667, 0.401111)(0.666667, 0)(0.666667, 0)(0.666667, 0.444444)(0.7, 0.444444)(0.7, 0)(0.7, 0)(0.7, 0.49)(0.733333, 0.49)(0.733333, 0)(0.733333, 0)(0.733333, 0.537778)(0.766667, 0.537778)(0.766667, 0)(0.766667, 0)(0.766667, 0.587778)(0.8, 0.587778)(0.8, 0)(0.8, 0)(0.8, 0.64)(0.833333, 0.64)(0.833333, 0)(0.833333, 0)(0.833333, 0.694444)(0.866667, 0.694444)(0.866667, 0)(0.866667, 0)(0.866667, 0.751111)(0.9, 0.751111)(0.9, 0)(0.9, 0)(0.9, 0.81)(0.933333, 0.81)(0.933333, 0)(0.933333, 0)(0.933333, 0.871111)(0.966667, 0.871111)(0.966667, 0)(0.966667, 0)(0.966667, 0.934444)(1, 0.934444)(1, 0)
}
\uncover<19>{
%approximation 1/40
\psline*[linecolor=cyan, linewidth=0.1pt](0, 0)(0, 0)(0.025, 0)(0.025, 0)(0.025, 0)(0.025, 0.000625)(0.05, 0.000625)(0.05, 0)(0.05, 0)(0.05, 0.0025)(0.075, 0.0025)(0.075, 0)(0.075, 0)(0.075, 0.005625)(0.1, 0.005625)(0.1, 0)(0.1, 0)(0.1, 0.01)(0.125, 0.01)(0.125, 0)(0.125, 0)(0.125, 0.015625)(0.15, 0.015625)(0.15, 0)(0.15, 0)(0.15, 0.0225)(0.175, 0.0225)(0.175, 0)(0.175, 0)(0.175, 0.030625)(0.2, 0.030625)(0.2, 0)(0.2, 0)(0.2, 0.04)(0.225, 0.04)(0.225, 0)(0.225, 0)(0.225, 0.050625)(0.25, 0.050625)(0.25, 0)(0.25, 0)(0.25, 0.0625)(0.275, 0.0625)(0.275, 0)(0.275, 0)(0.275, 0.075625)(0.3, 0.075625)(0.3, 0)(0.3, 0)(0.3, 0.09)(0.325, 0.09)(0.325, 0)(0.325, 0)(0.325, 0.105625)(0.35, 0.105625)(0.35, 0)(0.35, 0)(0.35, 0.1225)(0.375, 0.1225)(0.375, 0)(0.375, 0)(0.375, 0.140625)(0.4, 0.140625)(0.4, 0)(0.4, 0)(0.4, 0.16)(0.425, 0.16)(0.425, 0)(0.425, 0)(0.425, 0.180625)(0.45, 0.180625)(0.45, 0)(0.45, 0)(0.45, 0.2025)(0.475, 0.2025)(0.475, 0)(0.475, 0)(0.475, 0.225625)(0.5, 0.225625)(0.5, 0)(0.5, 0)(0.5, 0.25)(0.525, 0.25)(0.525, 0)(0.525, 0)(0.525, 0.275625)(0.55, 0.275625)(0.55, 0)(0.55, 0)(0.55, 0.3025)(0.575, 0.3025)(0.575, 0)(0.575, 0)(0.575, 0.330625)(0.6, 0.330625)(0.6, 0)(0.6, 0)(0.6, 0.36)(0.625, 0.36)(0.625, 0)(0.625, 0)(0.625, 0.390625)(0.65, 0.390625)(0.65, 0)(0.65, 0)(0.65, 0.4225)(0.675, 0.4225)(0.675, 0)(0.675, 0)(0.675, 0.455625)(0.7, 0.455625)(0.7, 0)(0.7, 0)(0.7, 0.49)(0.725, 0.49)(0.725, 0)(0.725, 0)(0.725, 0.525625)(0.75, 0.525625)(0.75, 0)(0.75, 0)(0.75, 0.5625)(0.775, 0.5625)(0.775, 0)(0.775, 0)(0.775, 0.600625)(0.8, 0.600625)(0.8, 0)(0.8, 0)(0.8, 0.64)(0.825, 0.64)(0.825, 0)(0.825, 0)(0.825, 0.680625)(0.85, 0.680625)(0.85, 0)(0.85, 0)(0.85, 0.7225)(0.875, 0.7225)(0.875, 0)(0.875, 0)(0.875, 0.765625)(0.9, 0.765625)(0.9, 0)(0.9, 0)(0.9, 0.81)(0.925, 0.81)(0.925, 0)(0.925, 0)(0.925, 0.855625)(0.95, 0.855625)(0.95, 0)(0.95, 0)(0.95, 0.9025)(0.975, 0.9025)(0.975, 0)(0.975, 0)(0.975, 0.950625)(1, 0.950625)(1, 0)
\psline[linecolor=blue, linewidth=0.1pt](0, 0)(0, 0)(0.025, 0)(0.025, 0)(0.025, 0)(0.025, 0.000625)(0.05, 0.000625)(0.05, 0)(0.05, 0)(0.05, 0.0025)(0.075, 0.0025)(0.075, 0)(0.075, 0)(0.075, 0.005625)(0.1, 0.005625)(0.1, 0)(0.1, 0)(0.1, 0.01)(0.125, 0.01)(0.125, 0)(0.125, 0)(0.125, 0.015625)(0.15, 0.015625)(0.15, 0)(0.15, 0)(0.15, 0.0225)(0.175, 0.0225)(0.175, 0)(0.175, 0)(0.175, 0.030625)(0.2, 0.030625)(0.2, 0)(0.2, 0)(0.2, 0.04)(0.225, 0.04)(0.225, 0)(0.225, 0)(0.225, 0.050625)(0.25, 0.050625)(0.25, 0)(0.25, 0)(0.25, 0.0625)(0.275, 0.0625)(0.275, 0)(0.275, 0)(0.275, 0.075625)(0.3, 0.075625)(0.3, 0)(0.3, 0)(0.3, 0.09)(0.325, 0.09)(0.325, 0)(0.325, 0)(0.325, 0.105625)(0.35, 0.105625)(0.35, 0)(0.35, 0)(0.35, 0.1225)(0.375, 0.1225)(0.375, 0)(0.375, 0)(0.375, 0.140625)(0.4, 0.140625)(0.4, 0)(0.4, 0)(0.4, 0.16)(0.425, 0.16)(0.425, 0)(0.425, 0)(0.425, 0.180625)(0.45, 0.180625)(0.45, 0)(0.45, 0)(0.45, 0.2025)(0.475, 0.2025)(0.475, 0)(0.475, 0)(0.475, 0.225625)(0.5, 0.225625)(0.5, 0)(0.5, 0)(0.5, 0.25)(0.525, 0.25)(0.525, 0)(0.525, 0)(0.525, 0.275625)(0.55, 0.275625)(0.55, 0)(0.55, 0)(0.55, 0.3025)(0.575, 0.3025)(0.575, 0)(0.575, 0)(0.575, 0.330625)(0.6, 0.330625)(0.6, 0)(0.6, 0)(0.6, 0.36)(0.625, 0.36)(0.625, 0)(0.625, 0)(0.625, 0.390625)(0.65, 0.390625)(0.65, 0)(0.65, 0)(0.65, 0.4225)(0.675, 0.4225)(0.675, 0)(0.675, 0)(0.675, 0.455625)(0.7, 0.455625)(0.7, 0)(0.7, 0)(0.7, 0.49)(0.725, 0.49)(0.725, 0)(0.725, 0)(0.725, 0.525625)(0.75, 0.525625)(0.75, 0)(0.75, 0)(0.75, 0.5625)(0.775, 0.5625)(0.775, 0)(0.775, 0)(0.775, 0.600625)(0.8, 0.600625)(0.8, 0)(0.8, 0)(0.8, 0.64)(0.825, 0.64)(0.825, 0)(0.825, 0)(0.825, 0.680625)(0.85, 0.680625)(0.85, 0)(0.85, 0)(0.85, 0.7225)(0.875, 0.7225)(0.875, 0)(0.875, 0)(0.875, 0.765625)(0.9, 0.765625)(0.9, 0)(0.9, 0)(0.9, 0.81)(0.925, 0.81)(0.925, 0)(0.925, 0)(0.925, 0.855625)(0.95, 0.855625)(0.95, 0)(0.95, 0)(0.95, 0.9025)(0.975, 0.9025)(0.975, 0)(0.975, 0)(0.975, 0.950625)(1, 0.950625)(1, 0)
}
\uncover<11>{
\pscustom*[linecolor=cyan, linewidth=0.1pt]{\psplot[linecolor=red, plotpoints=1000]{0}{1}{x 2 exp }\psline(1,1)(1,0)}
}
\psplot[linecolor=red, plotpoints=1000]{0}{1.15}{x 2 exp }
\end{pspicture}
}
\end{frame}
% end module areas-intro

% end lecture

\end{document}
