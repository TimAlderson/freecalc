\documentclass%
%[handout]
{beamer}
% % % % % % % %
% % % % % % % %
% % % % % % % %
%IMPORTANT
%compiles with 
%pdflatex -shell-escape 
%IMPORTANT
% % % % % % % %
% % % % % % % %
% % % % % % % %
\mode<presentation>
{
\useinnertheme{rounded}
\useoutertheme{infolines}
\usecolortheme{orchid}
\usecolortheme{whale}
}

\usepackage[english]{babel}
\usepackage[latin1]{inputenc}
\usepackage{times}
\usepackage[T1]{fontenc}
\usepackage{../example-templates}
\usepackage{auto-pst-pdf}
\usepackage{pst-plot}

% Or whatever. Note that the encoding and the font should match. If T1
% does not look nice, try deleting the line with the fontenc.

\graphicspath{{../../modules/}}

\newtheoremstyle{partialproof}{3pt}{3pt}{}{}{}{.}{.5em}{}
\theoremstyle{partialproof} \newtheorem{partialproof}[theorem]{Proof.}
%\DeclareMathOperator{\diff}{d}
\newcommand{\diff}{\text{d}}
\setbeamertemplate{navigation symbols}{}

\includeonlylecture{1}

\newcommand{\lect}[3]{
  \date{#1}
  \lecture[#1]{#2}{#3}
}

\setbeamertemplate{footline}
{
  \leavevmode%
  \hbox{%
  \begin{beamercolorbox}[wd=.333333\paperwidth,ht=2.25ex,dp=1ex,center]{author in head/foot}%
    \usebeamerfont{author in head/foot}\insertshortauthor
  \end{beamercolorbox}%
  \begin{beamercolorbox}[wd=.333333\paperwidth,ht=2.25ex,dp=1ex,center]{title in head/foot}%
    \usebeamerfont{title in head/foot}\insertshorttitle
  \end{beamercolorbox}%
  \begin{beamercolorbox}[wd=.333333\paperwidth,ht=2.25ex,dp=1ex,center]{date in head/foot}%
    \usebeamerfont{date in head/foot}\insertshortdate{}
  \end{beamercolorbox}}%
  \vskip0pt%
}

% If you have a file called "university-logo-filename.xxx", where xxx
% is a graphic format that can be processed by latex or pdflatex,
% resp., then you can add a logo as follows:

%\pgfdeclareimage[height=0.8cm]{logo}{bluelogo}
%\logo{\pgfuseimage{logo}}

\begin{document}


\AtBeginLecture{%

\title[\insertlecture]{FreeCalc}
\subtitle{\insertlecture}
\author[FreeCalc]{}
\institute[UMass Boston]{University of Massachusetts Boston}
\date{\insertshortlecture}
\begin{frame}
  \titlepage
\end{frame}
}%

% begin lecture
\lect{\today}{Sample}{1}
% begin module limit-laws
\begin{frame}
\frametitle{Calculating Limits Using Limit Laws}
\begin{theorem}[Limit Laws]
Suppose that $c$ is a constant and that the limits %
$\displaystyle \lim_{x\rightarrow a} f(x)$ and $\displaystyle \lim_{x\rightarrow a}g(x)$ %
exist ($\pm\infty$ \textbf{not allowed}).  Then
\begin{enumerate}
\item<1-| alert@2>  $\displaystyle \lim_{x\rightarrow a}[f(x) + g(x)] = \lim_{x\rightarrow a} f(x) + \lim_{x\rightarrow a} g(x)$.
\item<1-| alert@3>  $\displaystyle \lim_{x\rightarrow a}[f(x) - g(x)] = \lim_{x\rightarrow a} f(x) - \lim_{x\rightarrow a} g(x)$.
\item<1-| alert@4>  $\displaystyle \lim_{x\rightarrow a}[cf(x)] = c\lim_{x\rightarrow a} f(x)$. 
\item<1-| alert@5>  $\displaystyle \lim_{x\rightarrow a}[f(x)g(x)] = \lim_{x\rightarrow a} f(x) \cdot \lim_{x\rightarrow a} g(x)$.
\item<1-| alert@6>  $\displaystyle \lim_{x\rightarrow a}\frac{f(x)}{g(x)} = \frac{\lim\limits_{x\rightarrow a} f(x)}{\lim\limits_{x\rightarrow a} g(x)}$ \ if \ $\displaystyle \lim\limits_{x\rightarrow a}g(x) \neq 0$.
\end{enumerate}
\end{theorem}
\alert<2->{
\only<handout:0| -2>{\uncover<2>{
Sum Law
}}
\only<handout:0| 3>{Difference Law}
\only<handout:0| 4>{Constant Multiple Law}
\only<handout:0| 5>{Product Law}
\only<handout:0| 6->{Quotient Law}\invisible<1->{p}
}
\end{frame}


\begin{frame}
Here are some other useful limit laws:

\begin{enumerate}
\setcounter{enumi}{5}
\item<1-| alert@2> $\displaystyle \lim_{x\rightarrow a} [f(x)]^n = [\lim_{x\rightarrow a} f(x)]^n$
\item<1-| alert@4> $\displaystyle \lim_{x\rightarrow a} c = c$.
\item<1-| alert@4> $\displaystyle \lim_{x\rightarrow a} x = a$.
\item<1-| alert@4> $\displaystyle \lim_{x\rightarrow a} x^n = a^n$.
\item<1-| alert@4> $\displaystyle \lim_{x\rightarrow a} \sqrt[n]{x} = \sqrt[n]{a}$.
\item<1-| alert@3> $\displaystyle \lim_{x\rightarrow a} \sqrt[n]{f(x)} = \sqrt[n]{\lim_{x\rightarrow a}f(x)}$. 
\end{enumerate}
\alert{
\only<handout:0| -2>{
\uncover<2>{
Power Law
}}
\only<handout:0| 3>{Root Law}
\only<handout:0| 4>{Direct Substitution}\invisible<1->{p}
}
\end{frame}
% end module limit-laws

% end lecture

\end{document}
