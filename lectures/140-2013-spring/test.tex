\documentclass%
%[handout]
{beamer}
% % % % % % % %
% % % % % % % %
% % % % % % % %
%IMPORTANT
%compiles with 
%pdflatex -shell-escape 
%IMPORTANT
% % % % % % % %
% % % % % % % %
% % % % % % % %
\mode<presentation>
{
\useinnertheme{rounded}
\useoutertheme{infolines}
\usecolortheme{orchid}
\usecolortheme{whale}
}

\usepackage[english]{babel}
\usepackage[latin1]{inputenc}
\usepackage[all,cmtip]{xy}
\usepackage{times}
\usepackage[T1]{fontenc}
\usepackage{../example-templates}
\usepackage{auto-pst-pdf}
\usepackage{pst-plot}

% Or whatever. Note that the encoding and the font should match. If T1
% does not look nice, try deleting the line with the fontenc.

\graphicspath{{../../modules/}}

\newtheoremstyle{partialproof}{3pt}{3pt}{}{}{}{.}{.5em}{}
\theoremstyle{partialproof} \newtheorem{partialproof}[theorem]{Proof.}
%\DeclareMathOperator{\diff}{d}
\newcommand{\diff}{\text{d}}
\setbeamertemplate{navigation symbols}{}

\includeonlylecture{1}

\newcommand{\lect}[3]{
  \date{#1}
  \lecture[#1]{#2}{#3}
}

\setbeamertemplate{footline}
{
  \leavevmode%
  \hbox{%
  \begin{beamercolorbox}[wd=.333333\paperwidth,ht=2.25ex,dp=1ex,center]{author in head/foot}%
    \usebeamerfont{author in head/foot}\insertshortauthor
  \end{beamercolorbox}%
  \begin{beamercolorbox}[wd=.333333\paperwidth,ht=2.25ex,dp=1ex,center]{title in head/foot}%
    \usebeamerfont{title in head/foot}\insertshorttitle
  \end{beamercolorbox}%
  \begin{beamercolorbox}[wd=.333333\paperwidth,ht=2.25ex,dp=1ex,center]{date in head/foot}%
    \usebeamerfont{date in head/foot}\insertshortdate{}
  \end{beamercolorbox}}%
  \vskip0pt%
}

% If you have a file called "university-logo-filename.xxx", where xxx
% is a graphic format that can be processed by latex or pdflatex,
% resp., then you can add a logo as follows:

%\pgfdeclareimage[height=0.8cm]{logo}{bluelogo}
%\logo{\pgfuseimage{logo}}

\begin{document}
\newcommand{\psHollowDot}[2]{
\pscircle*[fillcolor=white, linecolor=red](#1, #2){0.07}
\pscircle*[fillcolor=white, linecolor=white](#1, #2){0.04}
}
\newcommand{\psFullDot}[2]{
\pscircle*[fillcolor=white, linecolor=red](#1, #2){0.07}
}
\newcommand{\psFullDotBlack}[2]{
\pscircle*[fillcolor=white, linecolor=black](#1, #2){0.07}
}
\newcommand{\psLabelXOne}{\psline(1, -0.1)(1,0.1) \rput[t](1, -0.2 ) { $1$} }
\newcommand{\psLabelYOne}{\psline(-0.1, 1)(0.1, 1) \rput[r](-0.2, 1 ) { $1$} }

\AtBeginLecture{%

\title[\insertlecture]{FreeCalc}
\subtitle{\insertlecture}
\author[FreeCalc]{}
\institute[UMass Boston]{University of Massachusetts Boston}
\date{\insertshortlecture}
\begin{frame}
  \titlepage
\end{frame}
}%

% begin lecture
\lect{\today}{Sample}{1}
% begin module derivatives-as-function-ex2
\begin{frame}
\begin{example}
If $f(x) = x^3-x$, find the formula for $f'(x)$.
\begin{columns}[c]
\column{.25\textwidth}

\psset{xunit=0.7cm, yunit=0.7cm}
\begin{pspicture}(-5, -5)(5,5) 
\psframe*[linecolor=white](-5,-5)(5,5) 
\psaxes[ticks=none, labels=none]{<->}(0,0)(-2,-2.5)(2,2.5)
%Function formula: - (x)+(x)^{3} 
\psplot[linecolor=red, plotpoints=1000]{-1.5}{1.5}{x 3 exp x -1 mul add }
\tiny
\psLabelXOne
\psLabelYOne
\rput[l](-1.3, 0.6){$y=f(x)$}
\end{pspicture} 

\psset{xunit=0.7cm, yunit=0.7cm}
\begin{pspicture}(-5, -5)(5,5) 
\psframe*[linecolor=white](-5,-5)(5,5) 
\psaxes[ticks=none, labels=none]{<->}(0,0)(-2,-2.5)(2,2.5)
\tiny
\psLabelXOne
\psLabelYOne
%Function formula: 3 ((x)^{2})-1 
\uncover<8->{
\psplot[linecolor=blue, plotpoints=1000]{-1}{1}{-1 x 2 exp 3 mul add }
\rput[l](-1.3, -1.5){$y=f(x)$}
}
\end{pspicture} 
\column{.75\textwidth}
\begin{align*}
&\uncover<2->{f'(x)}\\%
 & \uncover<2->{ = } %
\uncover<2->{\lim_{h\rightarrow 0} \frac{f(x+h)-f(x)}{h}}\\%
 & \uncover<3->{ = } %
\uncover<3->{\lim_{h\rightarrow 0} \frac{[(x+h)^3 - (x+h)]-[x^3-x]}{h}}\\%
 & \uncover<4->{ = } %
\uncover<4->{\lim_{h\rightarrow 0} \frac{x^3 + 3x^2h+3xh^2+h^3-x-h-x^3+x}{h}}\\%
 & \uncover<5->{ = } %
\uncover<5->{\lim_{h\rightarrow 0} \frac{3x^2h+3xh^2+h^3-h}{h}}\\%
 & \uncover<6->{ = } %
\uncover<6->{\lim_{h\rightarrow 0} (3x^2+3xh+h^2-1)}\\%
 & \uncover<7->{ = } %
\uncover<7,8->{3x^2-1}%
\end{align*}
\end{columns}
\end{example}
\end{frame}
% end module derivatives-as-function-ex2

% end lecture

\end{document}
