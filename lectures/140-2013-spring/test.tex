\documentclass%
%[handout]
{beamer}
% % % % % % % %
% % % % % % % %
% % % % % % % %
%IMPORTANT
%compiles with 
%pdflatex -shell-escape 
%IMPORTANT
% % % % % % % %
% % % % % % % %
% % % % % % % %
\mode<presentation>
{
\useinnertheme{rounded}
\useoutertheme{infolines}
\usecolortheme{orchid}
\usecolortheme{whale}
}

\usepackage[english]{babel}
\usepackage[latin1]{inputenc}
\usepackage[all,cmtip]{xy}
\usepackage{times}
\usepackage[T1]{fontenc}
\usepackage{../example-templates}
\usepackage{auto-pst-pdf}
\usepackage{pst-plot}

% Or whatever. Note that the encoding and the font should match. If T1
% does not look nice, try deleting the line with the fontenc.

\graphicspath{{../../modules/}}

\newtheoremstyle{partialproof}{3pt}{3pt}{}{}{}{.}{.5em}{}
\theoremstyle{partialproof} \newtheorem{partialproof}[theorem]{Proof.}
%\DeclareMathOperator{\diff}{d}
\newcommand{\diff}{\text{d}}
\setbeamertemplate{navigation symbols}{}

\includeonlylecture{1}

\newcommand{\lect}[3]{
  \date{#1}
  \lecture[#1]{#2}{#3}
}

\setbeamertemplate{footline}
{
  \leavevmode%
  \hbox{%
  \begin{beamercolorbox}[wd=.333333\paperwidth,ht=2.25ex,dp=1ex,center]{author in head/foot}%
    \usebeamerfont{author in head/foot}\insertshortauthor
  \end{beamercolorbox}%
  \begin{beamercolorbox}[wd=.333333\paperwidth,ht=2.25ex,dp=1ex,center]{title in head/foot}%
    \usebeamerfont{title in head/foot}\insertshorttitle
  \end{beamercolorbox}%
  \begin{beamercolorbox}[wd=.333333\paperwidth,ht=2.25ex,dp=1ex,center]{date in head/foot}%
    \usebeamerfont{date in head/foot}\insertshortdate{}
  \end{beamercolorbox}}%
  \vskip0pt%
}

% If you have a file called "university-logo-filename.xxx", where xxx
% is a graphic format that can be processed by latex or pdflatex,
% resp., then you can add a logo as follows:

%\pgfdeclareimage[height=0.8cm]{logo}{bluelogo}
%\logo{\pgfuseimage{logo}}

\begin{document}
\newcommand{\psHollowDot}[2]{
\pscircle*[fillcolor=white, linecolor=red](#1, #2){0.07}
\pscircle*[fillcolor=white, linecolor=white](#1, #2){0.04}
}
\newcommand{\psHollowDotBlue}[2]{
\pscircle*[fillcolor=white, linecolor=blue](#1, #2){0.07}
\pscircle*[fillcolor=white, linecolor=white](#1, #2){0.04}
}
\newcommand{\psFullDot}[2]{
\pscircle*[fillcolor=white, linecolor=red](#1, #2){0.07}
}
\newcommand{\psFullDotBlack}[2]{
\pscircle*[fillcolor=white, linecolor=black](#1, #2){0.07}
}
\newcommand{\psLabelXOne}{\psline(1, -0.1)(1,0.1) \rput[t](1, -0.2 ) { $1$} }
\newcommand{\psLabelYOne}{\psline(-0.1, 1)(0.1, 1) \rput[r](-0.2, 1 ) { $1$} }

\AtBeginLecture{%

\title[\insertlecture]{FreeCalc}
\subtitle{\insertlecture}
\author[FreeCalc]{}
\institute[UMass Boston]{University of Massachusetts Boston}
\date{\insertshortlecture}
\begin{frame}
  \titlepage
\end{frame}
}%

% begin lecture
\lect{\today}{Sample}{1}
% begin module differentiation-formulas-ex1
\begin{frame}
\alert<1>{You will not be tested on the material in the following slide.}
\end{frame}
\begin{frame}
\frametitle{Derivative ball volume =surface area}
The relationship between surface area and volume of a ball.

\footnotesize
\begin{tabular}{|p{0.7cm}p{2cm}p{1cm}p{1cm}p{1cm}p{1cm}p{2.5cm}|}\hline
\alert<0>{Di-men-sion} & \alert<2,14,26>{Pts. at distance $\leq r$ from origin} &  \alert<4,16,28>{Inside-volume name} & \alert<6, 18,30>{Inside-volume f-la} & \alert<8,20,32>{Boundary name} & \alert<10,22,34>{Boundary area f-la} & \alert<12,24,36>{Derivative inside-volume}\\\hline
%
\alert<2>{3} & \uncover<3->{\alert<3>{ball}} & \uncover<5->{\alert<5>{ball volume}} &  \uncover<7->{\alert<7, 12>{$\frac {4}{3}\pi r^3$}} & \uncover<9->{\alert<9>{sphere surface area} } & \uncover<11->{\alert<11, 13>{$4\pi r^2$}} & \uncover<12->{$\alert<12>{\frac{d}{dr}\left(\frac {4}{3}\pi r^3\right)=}\uncover<13->{\alert<13>{4\pi r^2}}$} \\\hline
%
\alert<14>{2} & \uncover<15->{\alert<15>{disk, circle}} & \uncover<17->{\alert<17>{circle area}} & \uncover<19->{\alert<19,24>{$\pi r^2$}} & \uncover<21->{\alert<21>{circle circum-ference}} & \uncover<23->{\alert<23,25>{$2\pi r$}} & \uncover<24->{${\alert<24>{\frac{d}{dr}\left(\pi r^2\right)=}} \uncover<25->{\alert<25>{2\pi r}}$} \\\hline
%
\alert<26>{1} & \uncover<27->{\alert<27>{interval}} & \uncover<29->{\alert<29>{length}} & \uncover<31->{\alert<31>{$2r$}} & \uncover<33->{\alert<33>{the two endpoints}} & \uncover<35->{\alert<35,37>{$2$}} &\uncover<36->{$\alert<36>{\frac{d}{dr}(2r)=} \uncover<37->{\alert<37>{2}}$} \\
\hline
\end{tabular}
\end{frame}
% end module differentiation-formulas-ex1

% end lecture

\end{document}
