\documentclass%
%[handout]
{beamer}
% % % % % % % %
% % % % % % % %
% % % % % % % %
%IMPORTANT
%compiles with 
%pdflatex -shell-escape 
%IMPORTANT
% % % % % % % %
% % % % % % % %
% % % % % % % %
\mode<presentation>
{
\useinnertheme{rounded}
\useoutertheme{infolines}
\usecolortheme{orchid}
\usecolortheme{whale}
}

\usepackage[english]{babel}
\usepackage[latin1]{inputenc}
\usepackage{times}
\usepackage[T1]{fontenc}
\usepackage{../example-templates}
\usepackage{auto-pst-pdf}
\usepackage{pst-plot}

% Or whatever. Note that the encoding and the font should match. If T1
% does not look nice, try deleting the line with the fontenc.

\graphicspath{{../../modules/}}

\newtheoremstyle{partialproof}{3pt}{3pt}{}{}{}{.}{.5em}{}
\theoremstyle{partialproof} \newtheorem{partialproof}[theorem]{Proof.}
%\DeclareMathOperator{\diff}{d}
\newcommand{\diff}{\text{d}}
\setbeamertemplate{navigation symbols}{}

\includeonlylecture{1}

\newcommand{\lect}[3]{
  \date{#1}
  \lecture[#1]{#2}{#3}
}

\setbeamertemplate{footline}
{
  \leavevmode%
  \hbox{%
  \begin{beamercolorbox}[wd=.333333\paperwidth,ht=2.25ex,dp=1ex,center]{author in head/foot}%
    \usebeamerfont{author in head/foot}\insertshortauthor
  \end{beamercolorbox}%
  \begin{beamercolorbox}[wd=.333333\paperwidth,ht=2.25ex,dp=1ex,center]{title in head/foot}%
    \usebeamerfont{title in head/foot}\insertshorttitle
  \end{beamercolorbox}%
  \begin{beamercolorbox}[wd=.333333\paperwidth,ht=2.25ex,dp=1ex,center]{date in head/foot}%
    \usebeamerfont{date in head/foot}\insertshortdate{}
  \end{beamercolorbox}}%
  \vskip0pt%
}

% If you have a file called "university-logo-filename.xxx", where xxx
% is a graphic format that can be processed by latex or pdflatex,
% resp., then you can add a logo as follows:

%\pgfdeclareimage[height=0.8cm]{logo}{bluelogo}
%\logo{\pgfuseimage{logo}}

\begin{document}


\AtBeginLecture{%

\title[\insertlecture]{FreeCalc}
\subtitle{\insertlecture}
\author[FreeCalc]{}
\institute[UMass Boston]{University of Massachusetts Boston}
\date{\insertshortlecture}
\begin{frame}
  \titlepage
\end{frame}
}%

% begin lecture
\lect{\today}{Sample}{1}
% begin module power-functions-def
\begin{frame}
\frametitle{Power Functions}
\begin{definition}[Power Function]
Let $x>0$, $a$ - arbitrary real number. The power function is defined as
\[
f(x) \uncover<4->{\alert<handout: 0| 4>{=e^{a\ln x} } }= \alert<2>{x}^{\alert<3>{a}} \quad .
\]
\uncover<2->{$x$ = \alert<2>{base}. } \uncover<3->{$a$ = \alert<3>{exponent} or \alert<3>{power}. }
\uncover<4->{\alert<handout:0| 4>{First equality = one {\tiny(the best)} of ways to define for non-integer $a$ (we study $\ln x$, $e^x$ later). } }
\end{definition}
\begin{tabular}{l}
\uncover<5->{
If $a$ - positive integer ($1, 2, 3, \ldots$) \\
then $x^a$ = polynomial function.
}\\
\uncover<6->{
Rules (studied in high \\
school for integer exponents): \\
$x^{a+b}=x^ax^b $ \\
$x^{-a}=\frac{1}{x^a}$\\
$x^{n}=\underbrace{x\dots x }_{n~\mathrm{times}} $ when $n$-integer. \\
~\\~\\~\\~\\~\\~\\~\\
}
\end{tabular}
\uncover<7-11>{
\psset{xunit=0.4cm,yunit=0.4cm}
\begin{pspicture}(-5,-5)(5,5)
\psaxes[labels=none]{<->}(0,0)(-5,-5)(5,5)
\rput[r](0,5){\tiny{$y$}}
\rput[l](5,0){\tiny{$x$}}
\only<7>{
\psplot[linecolor=red]{-5}{5}{ x 1 exp }
\rput( 3, 1){$y=x^{\phantom{1}}$}
} %only
\only<8>{
\psplot[linecolor=red]{-2.23}{2.23}{ x 2 exp }
\rput( 3, 1){$y=x^2$}
}
\only<9>{
\psplot[linecolor=red]{-1.7}{1.7}{ x 3 exp }
\rput( 3, 1){$y=x^3$}
}
\only<10>{
\psplot[linecolor=red]{-1.49}{1.49}{ x 4 exp }
\rput( 3, 1){$y=x^4$}
}
\only<11>{
\psplot[linecolor=red]{-1.37}{1.37}{ x 5 exp }
\rput( 3, 1){$y=x^5$}
}
\end{pspicture}
}
\end{frame}
% end module power-functions-def

% end lecture

\end{document}
