\documentclass%
%[handout]
{beamer}
% % % % % % % %
% % % % % % % %
% % % % % % % %
%IMPORTANT
%compiles with 
%pdflatex -shell-escape 
%IMPORTANT
% % % % % % % %
% % % % % % % %
% % % % % % % %
\mode<presentation>
{
\useinnertheme{rounded}
\useoutertheme{infolines}
\usecolortheme{orchid}
\usecolortheme{whale}
}

\usepackage[english]{babel}
\usepackage[latin1]{inputenc}
\usepackage[all,cmtip]{xy}
\usepackage{times}
\usepackage[T1]{fontenc}
\usepackage{../example-templates}
\usepackage{auto-pst-pdf}
\usepackage{pst-plot}
\usepackage{cancel}


% Or whatever. Note that the encoding and the font should match. If T1
% does not look nice, try deleting the line with the fontenc.

\graphicspath{{../../modules/}}

\newtheoremstyle{partialproof}{3pt}{3pt}{}{}{}{.}{.5em}{}
\theoremstyle{partialproof} \newtheorem{partialproof}[theorem]{Proof.}
%\DeclareMathOperator{\diff}{d}
\newcommand{\diff}{\text{d}}
\setbeamertemplate{navigation symbols}{}

\includeonlylecture{1}

\newcommand{\lect}[3]{
  \date{#1}
  \lecture[#1]{#2}{#3}
}

\setbeamertemplate{footline}
{
  \leavevmode%
  \hbox{%
  \begin{beamercolorbox}[wd=.333333\paperwidth,ht=2.25ex,dp=1ex,center]{author in head/foot}%
    \usebeamerfont{author in head/foot}\insertshortauthor
  \end{beamercolorbox}%
  \begin{beamercolorbox}[wd=.333333\paperwidth,ht=2.25ex,dp=1ex,center]{title in head/foot}%
    \usebeamerfont{title in head/foot}\insertshorttitle
  \end{beamercolorbox}%
  \begin{beamercolorbox}[wd=.333333\paperwidth,ht=2.25ex,dp=1ex,center]{date in head/foot}%
    \usebeamerfont{date in head/foot}\insertshortdate{}
  \end{beamercolorbox}}%
  \vskip0pt%
}

% If you have a file called "university-logo-filename.xxx", where xxx
% is a graphic format that can be processed by latex or pdflatex,
% resp., then you can add a logo as follows:

%\pgfdeclareimage[height=0.8cm]{logo}{bluelogo}
%\logo{\pgfuseimage{logo}}

\begin{document}
\newcommand{\psHollowDot}[2]{
\pscircle*[fillcolor=white, linecolor=red](#1, #2){0.07}
\pscircle*[fillcolor=white, linecolor=white](#1, #2){0.04}
}
\newcommand{\psHollowDotBlue}[2]{
\pscircle*[fillcolor=white, linecolor=blue](#1, #2){0.07}
\pscircle*[fillcolor=white, linecolor=white](#1, #2){0.04}
}
\newcommand{\psFullDot}[2]{
\pscircle*[fillcolor=white, linecolor=red](#1, #2){0.07}
}
\newcommand{\psFullDotBlack}[2]{
\pscircle*[fillcolor=white, linecolor=black](#1, #2){0.07}
}
\newcommand{\psFullDotBlue}[2]{
\pscircle*[fillcolor=white, linecolor=blue](#1, #2){0.07}
}
\newcommand{\psLabelXOne}{\psline(1, -0.1)(1,0.1) \rput[t](1, -0.2 ) { $1$} }
\newcommand{\psLabelYOne}{\psline(-0.1, 1)(0.1, 1) \rput[r](-0.2, 1 ) { $1$} }

\AtBeginLecture{%

\title[\insertlecture]{FreeCalc}
\subtitle{\insertlecture}
\author[FreeCalc]{}
\institute[UMass Boston]{University of Massachusetts Boston}
\date{\insertshortlecture}
\begin{frame}
  \titlepage
\end{frame}
}%

% begin lecture
\lect{\today}{Sample}{1}
% begin module linearization-def
\begin{frame}
\begin{definition}[Linearization of $f$ at $a$]
The linear function whose graph is the tangent line at $(a,f(a))$ is called the linearization of $f$ at $a$.  Its equation is
\[
L(x) = f(a) + f'(a)(x-a).
\]
\end{definition}
\begin{definition}[Linear Approximation of $f(x)$ near $a$]
The approximation
\[
f(x) \approx f(a) + f'(a)(x-a)
\]
is called the linear approximation of $f$ at $a$.
\end{definition}
Let $y=f(x)$, $\Delta y:= f(x)-f(a)$, and $\Delta x:= x-a$.
\begin{definition}[Linear approx. $y=f(x)$ near $a$, alternative notation]
\[
\Delta y\approx \frac{d y}{d x}\Delta x\quad .
\]
\end{definition}
\end{frame}
\begin{frame}
\frametitle{Linear approximations}
\begin{center}%

\psset{xunit=1.5cm, yunit=0.8cm}
\begin{pspicture}(-5, -5)(5,5)
\tiny
\psframe*[linecolor=white](-5,-5)(5,5)
\psaxes[ticks=none, labels=none]{<->}(0,0)(-0.5,-0.5)(5,4.5)
%Function formula: 1/2+x
\psplot[linecolor=blue, plotpoints=1000]{-0.2}{4.3}{x 0.5 add }
\rput(3,4.2){$y=L(x)$}
 %Function formula: 7/2- ((-2+1/2 (x))^{2})
\rput(4.3,3.1){$y=f(x)$}
\psplot[linecolor=black, plotpoints=1000]{-0.2}{5}{x 0.5 mul -2 add 2 exp -1 mul 3.5 add }
\rput[br](2,2.6){$(x,f(x))$}
\fcFullDotBlack{2}{2.5}
\psline[linestyle=dashed](2, 2.5)(2,0)
\psline[linestyle=dashed](3, 3.25)(3,0)

\only<handout:1|1>{\psline(2, 2.5)(3, 2.5) (3, 3.25)}
\only<handout:0|2>{\psline[linecolor=red](2, 2.5)(3, 2.5) (3, 3.25)}
\only<handout:2|3>{\psline[linecolor=red](2, 2.5)(3, 2.5) (3, 3.5)}

\rput[t](2.5, 2.4 ){\alertNoH{2-3}{$\Delta x$}}

\only<handout:1|1-2>{ \rput[l](3.1,2.875){\alertNoH{2}{$\Delta f$}}
}
\only<handout:2|3>{ \rput[l](3.1,3){\alertNoH{3}{$\Delta L$}}
}

\rput[t](2,-0.1){$x$}
\rput[t](3,-0.1){$x+\Delta x$}
\end{pspicture}

\begin{tabular}{|l|c|c|}
\hline
Function & \alert<handout:1| 2>{$f$} & \alert<handout:2| 3>{$L$}\\
\hline
Run & \alert<handout:1| 2>{$\Delta x$} & \alert<handout:2| 3>{$\Delta x$}\\
Rise & \alert<handout:1| 2>{$\Delta f$} & \alert<handout:2| 3>{$\Delta L$}\\
Formula & \alert<handout:1| 2>{$\Delta f = f(x+\Delta x) - f(x)$} & \alert<handout:2| 3>{$\Delta L =(\Delta x)  f'(x) $}\\
\hline
\end{tabular}
\end{center}%

\end{frame}
% end module linearization-def

% begin module differential-def
\begin{frame}
\frametitle{Differentials}
\begin{itemize}
\item<1-> From previous slides:
\alert<12>{$\displaystyle \Delta y\approx \frac{dy}{dx} \Delta x$}
\item $\displaystyle \only<1-2>{\Delta y} 
\only<3->{\alert<3,6> {\alert<7>{d}y}} \only<1-3>{\approx}
\only<4->{\alert<3> =} \only<1->{\frac{dy}{\alert<5>{dx}}}
\only<1-3>{ \Delta x} 
\only<4->{\alert<4,5,8>{\alert<7>{d}x}}
\only<6->{=\alert<6>{\alert<7>{d}y} }
$
\item<2-> If we substitute \alert<3>{$\Delta y $ by the formal expression $dy$} and \alert<4>{$\Delta x$ by the formal expression $dx$}, the expression \alert<5>{$dx$ appears to ``cancel''} to give a \alert<6>{formal identity}.
\item<7-> Define the \alert<7,11>{\emph{differential $d$}} %\uncover<8->
{ and the \alert<8,10>{\emph{differential forms $dx$, $d(f(x))$}}} %\uncover<9->
{by requesting that \alert<9>{$d$ and $dx$ satisfy the transformation law} 
\[
\alert<9>{\alert<8>{\alert<7>{d}(f(x))}=f'(x) \alert<8>{\alert<7>{d}x}}
\] 
for any differentiable function $f(x)$.} In abbreviated notation:
\[ 
\alert<9>{ \alert<7>{d}f = f' \alert<8>{\alert<7>{d}x}}
\]
\uncover<10->{Expressions containing expression of the form $\alert<10>{\alert<11>{d}(something)}$ are called \alert<10>{differential forms}.}
\item<12-> Differential forms ``encode'' linear approximations.
\end{itemize}
\end{frame}
\begin{frame}
\begin{itemize}
\item $\alert<4,8,11>{\alert<3>{ \alert<2>{d} f(x)}= f' \alert<3>{\alert<2>{d}x}}$.
\item<2-> On the previous slide we stated the \alert<2>{differential $d$} and the \alert<3>{differential forms} $\alert<3>{dx, d f(x)}$ are \alert<4,7>{formal expressions related by a transformation law}.
\item<5-> One can construct differential forms and differentials  directly but this is outside of the scope of Calculus I and II. 
\item<6-> Courses such as ``Integration and Manifolds'' or ``Differential geometry'' usually give the precise definitions.
\item<7-> Nonetheless, \alert<7>{what we studied} is \alert<8>{completely sufficient} for practical purposes and \alert<8>{carrying out computations}.
\item<9-> \alert<9,10>{\textbf{Do not confuse differentials with derivatives.}} \uncover<11->{\alert<11>{The correct equality is this.}}

\[
\only<9>{\alert<9,10>{ df(x) = f'(x)}} \only<10->{\alert<9,10>{ \xcancel{df(x) = f'(x)}}}
\quad \quad \quad\quad \quad\uncover<11>{\alert<11>{ df(x)=f'(x)dx}}
\]
\end{itemize}
\end{frame}

% end lecture

\end{document}
