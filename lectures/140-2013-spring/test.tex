\documentclass%
%[handout]
{beamer}
% % % % % % % %
% % % % % % % %
% % % % % % % %
%IMPORTANT
%compiles with 
%pdflatex -shell-escape 
%IMPORTANT
% % % % % % % %
% % % % % % % %
% % % % % % % %
\mode<presentation>
{
\useinnertheme{rounded}
\useoutertheme{infolines}
\usecolortheme{orchid}
\usecolortheme{whale}
}

\usepackage[english]{babel}
\usepackage[latin1]{inputenc}
\usepackage[all,cmtip]{xy}
\usepackage{times}
\usepackage[T1]{fontenc}
\usepackage{../example-templates}
\usepackage{auto-pst-pdf}
\usepackage{pst-plot}

% Or whatever. Note that the encoding and the font should match. If T1
% does not look nice, try deleting the line with the fontenc.

\graphicspath{{../../modules/}}

\newtheoremstyle{partialproof}{3pt}{3pt}{}{}{}{.}{.5em}{}
\theoremstyle{partialproof} \newtheorem{partialproof}[theorem]{Proof.}
%\DeclareMathOperator{\diff}{d}
\newcommand{\diff}{\text{d}}
\setbeamertemplate{navigation symbols}{}

\includeonlylecture{1}

\newcommand{\lect}[3]{
  \date{#1}
  \lecture[#1]{#2}{#3}
}

\setbeamertemplate{footline}
{
  \leavevmode%
  \hbox{%
  \begin{beamercolorbox}[wd=.333333\paperwidth,ht=2.25ex,dp=1ex,center]{author in head/foot}%
    \usebeamerfont{author in head/foot}\insertshortauthor
  \end{beamercolorbox}%
  \begin{beamercolorbox}[wd=.333333\paperwidth,ht=2.25ex,dp=1ex,center]{title in head/foot}%
    \usebeamerfont{title in head/foot}\insertshorttitle
  \end{beamercolorbox}%
  \begin{beamercolorbox}[wd=.333333\paperwidth,ht=2.25ex,dp=1ex,center]{date in head/foot}%
    \usebeamerfont{date in head/foot}\insertshortdate{}
  \end{beamercolorbox}}%
  \vskip0pt%
}

% If you have a file called "university-logo-filename.xxx", where xxx
% is a graphic format that can be processed by latex or pdflatex,
% resp., then you can add a logo as follows:

%\pgfdeclareimage[height=0.8cm]{logo}{bluelogo}
%\logo{\pgfuseimage{logo}}

\begin{document}
\newcommand{\psHollowDot}[2]{
\pscircle*[fillcolor=white, linecolor=red](#1, #2){0.07}
\pscircle*[fillcolor=white, linecolor=white](#1, #2){0.04}
}
\newcommand{\psFullDot}[2]{
\pscircle*[fillcolor=white, linecolor=red](#1, #2){0.07}
}
\newcommand{\psLabelXOne}{\psline(1, -0.1)(1,0.1) \rput[t](1, -0.2 ) {\footnotesize $1$} }
\newcommand{\psLabelYOne}{\psline(-0.1, 1)(0.1, 1) \rput[r](-0.2, 1 ) {\footnotesize $1$} }

\AtBeginLecture{%

\title[\insertlecture]{FreeCalc}
\subtitle{\insertlecture}
\author[FreeCalc]{}
\institute[UMass Boston]{University of Massachusetts Boston}
\date{\insertshortlecture}
\begin{frame}
  \titlepage
\end{frame}
}%

% begin lecture
\lect{\today}{Sample}{1}
% begin module limits-ex9
\begin{frame}
\begin{example}
Find $\lim_{x\rightarrow 3^+} \frac{2x}{x-3}$ and $\lim_{x\rightarrow 3^-}\frac{2x}{x-3}$.
\begin{columns}[c]
\column{.42\textwidth}
\ \ \uncover<4->{
$\lim_{x\rightarrow 3^+}2x/(x-3) = \infty$.
}

\ \ \uncover<7->{
$\lim_{x\rightarrow 3^-}2x/(x-3) =-\infty$.
}

\uncover<8->{%
\psset{xunit=0.15cm, yunit=0.15cm}\begin{pspicture}(-2, -5)(5,5) \psframe*[linecolor=white](-2,-9)(30,33) 
\psaxes[ticks=none, labels=none]{<->}(0,0)(-2,-4.5)(18,30)
\psplot[linecolor=red, plotpoints=1000]{3.2}{30}{x 2 mul x -3 add div }
\psplot[linecolor=red, plotpoints=1000]{-3}{2.6}{x 2 mul x -3 add div }
\psline(-0.3,5)(0.3,5)
\rput(-1, 5){$5$}
\psline[linestyle=dotted](3, -12)(3,33)
\rput(10, 10){$y=\frac{2x}{x-3}$}
\rput[lt](3, -4){$x=3$}
\end{pspicture} 
}%
\column{.58\textwidth}
\begin{itemize}
\item<2->  If $x$ is near 3 but larger than 3, the denominator $x-3$ is a small positive number and $2x$ is close to 6.
\item<3->  So the quotient $2x/(x-3)$ is a large positive number.
\item<5->  If $x$ is near 3 but smaller than 3, the denominator $x-3$ is a negative number with small absolute value and $2x$ is close to 6.
\item<6->  So  $2x/(x-3)$ is a negative number with large absolute value.
\item<8->  $x = 3$ is a vertical asymptote for $f(x) = 2x/(x-3)$.
\end{itemize}
\end{columns}
\end{example}
\end{frame}
% end module limits-ex9

% end lecture

\end{document}
