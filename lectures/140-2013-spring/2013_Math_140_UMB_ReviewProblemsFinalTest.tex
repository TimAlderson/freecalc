\documentclass{article}
\usepackage{amsmath, amsfonts, amssymb, verbatim, hyperref}
\usepackage{auto-pst-pdf}
\usepackage{pst-plot}
%\usepackage{pst-solides3d}
%\usepackage{pst-3dplot}
\usepackage{pstricks}
\usepackage{rotating}

\usepackage{multicol}
\addtolength{\hoffset}{-3.5cm}
\addtolength{\textwidth}{6.8cm}
\addtolength{\voffset}{-3.3cm}
\addtolength{\textheight}{6.3cm}
\renewcommand{\Re}{\mathrm{Re~}}
\renewcommand{\Im}{\mathrm{Im~}}
\newcommand{\doublebrace}[4]{\left\{\begin{array}{ll} #1 & #2 \\#3 & #4  \end{array} \right.}
\newcommand{\triplebrace}[6]{\left\{\begin{array}{ll} #1 & #2 \\#3 & #4  \\#5 & #6\end{array} \right.}
\newtheorem{problem}{Problem}
\newcommand{\fcHollowDot}[2]{
\pscircle*[fillcolor=white, linecolor=red](#1, #2){0.07}
\pscircle*[fillcolor=white, linecolor=white](#1, #2){0.04}
}
\newcommand{\fcHollowDotBlue}[2]{
\pscircle*[fillcolor=white, linecolor=blue](#1, #2){0.07}
\pscircle*[fillcolor=white, linecolor=white](#1, #2){0.04}
}
\newcommand{\fcFullDot}[2]{
\pscircle*[fillcolor=white, linecolor=red](#1, #2){0.07}
}
\newcommand{\fcFullDotBlack}[2]{
\pscircle*[fillcolor=white, linecolor=black](#1, #2){0.07}
}
\newcommand{\fcLabelXOne}{\psline(1, -0.1)(1,0.1) \rput[t](1, -0.2 ) {\footnotesize $1$} }
\newcommand{\fcLabelYOne}{\psline(-0.1, 1)(0.1, 1) \rput[r](-0.2, 1 ) {\footnotesize $1$} }


\begin{document}
\begin{center}
\Large
Review problems \\
Final Math 140 \\
\normalsize Instructors: S. Cai, G. Cunningham, J. Greenough A. Leisinger, T. Milev
\end{center}
\begin{problem}
Define horizontal and vertical asymptote.
\end{problem}
\begin{problem}
Compute the horizontal and vertical asymptotes of
\begin{enumerate}
\item $
\displaystyle
f(x)=\frac{x^2+x+1}{-x^2- 3x+ 4}
$
\hfill{~}
\rotatebox{180}{
answer: horizontal asymptote: $y= -1$, vertical asymptotes $x=1, x=-4$.
}
\item $
\displaystyle
f(x)=\frac{x^2+x+1}{x^2- 2x- 3}
$
\hfill{~}
\rotatebox{180}{
answer: horizontal asymptote: $y= 1$, vertical asymptotes $x=-1, x=3$.
}
\end{enumerate}
\end{problem}
\begin{problem}
Define derivative.
\end{problem}
\begin{problem}
Find the limit
\begin{enumerate}
\item $\displaystyle \lim\limits_{h\to 0} \frac{ \sin^2 (x+h)-\sin^2 (x)}{h}$.
\hfill{~}
\rotatebox{180}{
answer: $2\sin x \cos x$.
}
\item $\displaystyle \lim\limits_{h\to 0} \frac{ \frac{1}{(x+h)^3}-\frac{1}{x^3}}{h}$, where $x\neq 0$.
\hfill{~}
\rotatebox{180}{
answer: $\frac{-3}{x^4}$.
}
\end{enumerate}
\end{problem}

\begin{problem}Find the derivative.
\begin{enumerate}
\item
 $x^{x^x} $.
\hfill{~}
\rotatebox{180}{
answer: $x^{x^{x}+x-1}+x^{x^{x}+x} \ln{}x
 +x^{x^{x}+x} (\ln{}x)^2 $.
}
\item $\sqrt[3]{\ln x}$
\hfill{~}
\rotatebox{180}{
answer: $\frac{(\ln{}x)^{-2/3}}{3x} $.
}
\end{enumerate}
\end{problem}

\begin{problem}
Consider the curve given by the equation $x^3+ x^2y^2+2y^3=4$.
\begin{itemize}
\item Use implicit differentiation to express $\frac{dy}{dx}$ via $y $ and $x$ on the curve.
\item Find the equation of the tangent line to the curve at the point $(x,y)=(1,1)$.
\end{itemize}
\end{problem}

\begin{problem}
State the fundamental theorem of Calculus (both parts).
\end{problem}

\begin{problem}
Compute the derivative.
\begin{multicols}{2}
\begin{enumerate}
\item $\displaystyle \int\limits_{\sqrt{x}}^{x^2} e^t ~dt$.
\item $\displaystyle \int\limits_{\ln{x}}^{-x} e^t ~dt$.
\end{enumerate}
\end{multicols}
\end{problem}


\begin{problem}
Compute the integral
\begin{multicols}{2}
\begin{enumerate}
\item $\displaystyle \int_{-1}^{1}\frac{x^3}{1+x^2}dx$
\item  $\displaystyle \int_{-2}^{2}\frac{x^5}{1+x^2}dx$
\end{enumerate}
\end{multicols}
\end{problem}

\begin{problem}
Find intervals of increasing, decreasing, maxima, minima (local and absolute if those exist), concavity, inflection points. Finally, plot the function roughly by hand. Compare your answer with the output of a graphic calculator (you may use an online graphic calculator as well).
\begin{multicols}{2}
\begin{enumerate}
\item $f(x)=x^4 - 3x^3+2$.
\item $f(x)=x^4 - 2x^3-3$.
\end{enumerate}
\end{multicols}
\end{problem}

\begin{problem}
A cone-shaped drinking cup is made from a circular piece of paper of radius $r$ by cutting out a sector and joining the edges $OA$ and $OB$. Find the maximum capacity of such a cup.
\begin{pspicture}(0,0)(1,1)
\pswedge*[linecolor=cyan](0,0){1}{120}{60}
\pswedge[linecolor=blue](0,0){1}{120}{60}
\rput[t] (0,-0.2){$O$}
\rput[b] (0.5,1){$B$}
\rput[b] (0.15,0.4){$r$}
\rput[b] (-0.5,1){$A$}
\end{pspicture}
\end{problem}

\begin{problem}
The minute hand on a watch is 0.5cm long and the hour hand is 0.25cm long. How fast is the distance between the tips of the hands changing at 2 o'clock?
\end{problem}

\begin{problem}Define the linearization of a function at a point.
\end{problem}

\begin{problem}
Explain the difference between definite integral and indefinite integral.
\end{problem}

\begin{problem}
Define Riemann sum. Define definite integral. Give an example of a Riemann sum.
\end{problem}

\begin{problem}
State Rolle's theorem. Use Rolle's theorem to prove that $x^3 +x^2+ 5x- 4$ has only one real root.
\end{problem}

\begin{problem}
Consider the parabola $y=1-x^2$.
\begin{enumerate}
\item Find the area locked between the parabola and the $x$ axis.
\item Find the volume of the solid of revolution obtained by rotating the parabola around the $x$ axis.
\end{enumerate}
\end{problem}

\end{document}
