\documentclass{article}
\usepackage{amsmath, amsfonts, amssymb, verbatim, hyperref}
\usepackage{auto-pst-pdf}
\usepackage{pst-plot}
\usepackage{cancel}
\addtolength{\hoffset}{-3.5cm}
\addtolength{\textwidth}{6.8cm}
\addtolength{\voffset}{-3.3cm}
\addtolength{\textheight}{6.3cm}
\renewcommand{\Re}{\mathrm{Re~}}
\renewcommand{\Im}{\mathrm{Im~}}
\newcommand{\doublebrace}[4]{\left\{\begin{array}{ll} #1 & #2 \\#3 & #4  \end{array} \right.}
\newcommand{\fcFullDot}[2]{
\pscircle*[fillcolor=white, linecolor=red](#1, #2){0.07}
}
\newcommand{\fcHollowDot}[2]{
\pscircle*[fillcolor=white, linecolor=red](#1, #2){0.07}
\pscircle*[fillcolor=white, linecolor=white](#1, #2){0.04}
}
\date{}
\newtheorem{problem}{Problem}
\title{Exam I\\ Math 140 Calculus I }
\pagestyle{empty}
\begin{document}
\begin{center}
\LARGE
Exam I

Calculus I, Math 140

March 7, 2013
\end{center}

\noindent \textbf{Name:}
\medskip
%\maketitle
%\color{green}

\begin{problem}
(7.5 + 7.5 pts) Find an expression for the function $(f\circ g)(x)$ and $(g\circ f)(x)$, where
\[
f(x)= \frac{3x-1}{x-2},\quad  g(y)=\frac{y-2 }{2y-4}
\] Simplify your answer to a single fraction.
\end{problem}
\textbf{Solution. }
Note that for $y\neq 2$, we have that $g(y)= \frac{y-2}{2y-4}= \frac{\cancel{(y-2)}}{2\cancel{(y-2)}}= \frac{1}{2}$. Therefore
\[
f(g(x))=f(\frac{1}{2})= \frac{3\times\frac{1}{2}-1}{\frac{1}{2}-2}= \frac{\frac{1}2}{-\frac{3}2}= -\frac{1}3\quad .
\]
Furthermore, since $g(y)=\frac{1}{2}$, we need not compute anything to get a formula for $g(f(x))$ - indeed, the value of $g$ does not depend on its argument:
\[
g(f(x))=\frac{1}{2}\quad .
\]
Final answer: $(f\circ g) (x)=-\frac{1}{3}$, $(g\circ f)(x)=\frac{1}2$.
\begin{problem} (15 pts)
Find all solutions in the interval $[0,2\pi]$ of the equation.
$\sqrt{ 3}\sin x=  \sin 2x$
\end{problem}
\textbf{Solution. }
We have the formula $\sin 2x=2\sin x \cos x $. Therefore
\[\begin{array}{rcl}
\sqrt{3} sin x&=& 2\sin x \cos x\\
\sqrt{3}\sin x - 2\sin x \cos x &=& 0\\
\sin x (\sqrt{3}- 2\cos x)&=&0\quad .
\end{array}
\]
Thus either $\sin x =0$ or $\sqrt{3}- 2\cos x=0$. As studied in the lecture on trigonometry, for $x\in [0,2\pi]$, we have that $\sin x=0$
if $x=0,\pi$ or $2\pi$. On the other hand $\sqrt 3-2\cos x=0$ is equivalent to $\cos x=\frac{\sqrt{3}}{2}$, which, together with $x\in [0,2\pi]$, implies $x=\frac{\pi}{6} $ or $x=\frac{11\pi}{6}$.

Final answer: $x\in \{0, \frac{\pi}{6},\pi  , \frac{11\pi}{6},2\pi \}$.
\begin{problem} (15 pts) Evaluate the limit if it exists.
\[\lim\limits_{x\to 2} \frac{3x^2-4x-4}{x^3-4x}
\]
\end{problem}
\textbf{Solution.}
\[
\lim\limits_{x\to 2} \frac{3x^2-4x-4}{x^3-4x}=
\lim\limits_{x\to 2}\frac{(3x+2 ) \cancel{(x-2)}} {x\cancel{(x-2)}(x+2)} = \lim\limits_{x\to 2} \frac{3x+2}{x(x+2)}= \frac{3\times 2+2 }{2(2+2)}= \frac{8}{8}=1.
\]
\begin{problem} (10 pts)
Evaluate the limit.
\[\lim\limits_{x\to 2^+} \frac{\sqrt{x^3-4x }}{3x^2 -4x-4 }\quad .
\]
\end{problem}
\textbf{Solution. }
\[
\lim\limits_{x\to 2^+} \frac{\sqrt{x^3-4x }}{3x^2 -4x-4 }=
\lim\limits_{x\to 2^+} \frac{\left( x(x-2)(x+2)\right)^{\frac{1}2}} {(3x+2)(x-2)} = \lim\limits_{x\to 2^+}\frac{ x^{\frac{1}2}(x+2)^{\frac{1}2}}{(3x+2)(x-2)^{\frac{1}2}}\quad .
\]
As the denominator of the above fraction tends to $0$ without changing sign as $x$ tends to $2$, and the numerator is non-zero, the limit must be equal to either $\infty$ or $-\infty$. On the other hand $x\to 2^{+}$ implies that $x>2$ and therefore multiplicands in the above limit are positive. Therefore the limit equals $+\infty$.

Final answer: $\lim\limits_{x\to 2^+} \frac{\sqrt{x^3-4x }}{3x^2 -4x-4 } =\infty$.
\begin{problem} (10 pts)
Use the intermediate value theorem to show that the equation has a solution in the interval $(-2,0)$.
\begin{equation}\label{eqProbTest1Prob5}
e^{2x}+x+1 =\sin  (-x) \quad .
\end{equation}
\end{problem}
\textbf{Solution.} Let $f(x) =  e^{2x}+x+1-\sin(-x)= e^{2x}+x+1+\sin(+x)$. A number $x$  is a solution to the equation \eqref{eqProbTest1Prob5} if and only if $f(x)=0$. On the other hand $f(-2)= e^{-4}-2+1+\sin(-2) = 1-e^{-4}-\sin 2$. We know that $e^{-4}=\frac{1}{e^4}<1$ and so $1-e^4<0$. Furthermore as $0<2<\pi$ we have that $\sin 2 >0 $  and $-\sin 2<0 $ (recall that $\sin 2$ is measured in radians). Therefore $ 1-e^{-4}-\sin 2 <0$. On the other hand, $f(0)= e^0+0+1+\sin 0=2>0$. We have that $f(x)$ is continuous: we studied in class that $1, x, \sin x$ and $ e^x$ are all continuous functions, and we also studied that sum of continuous functions is continuous. Furthermore $f(-2)<0<f(0)$. Therefore by the Intermediate Value Theorem we have that there exists a number $c\in (-2, 0)$ for which $f(c)=0$, which proves that \eqref{eqProbTest1Prob5} has a solution in the desired interval.
\begin{problem} (10pts)
Solve the equation.
\[e^{-4x}+5e^{-2x}-6=0\quad .
\]
\end{problem}
Set $z=e^{-2x}$. The equation becomes
\[
\begin{array}{rcl}
z^2+5z-6&=&0\\
(z+6)(z-1)&=&0\quad .
\end{array}
\]
Therefore $e^{-2x}=z=1$ or $e^{-2x}=z=-6$. The latter case, $e^{-2x}=-6$, is not possible for real $x$. Therefore $e^{-2x}=1$, and $-2x=\ln 1=0$, so $x=0$.

Final answer: $x=0$.
\begin{problem} (15 pts)
Plot roughly the function
\[
f(x)= x^{2}+2x+2
\] with domain $x>-1$. Compute the inverse function $f^{-1}(y) $. Plot roughly $f^{-1}(x)$. Explain what is the relationship between the graphs of $f^{-1}(x)$ and $f(x)$.
\end{problem}
To compute $f^{-1}(y)$, we solve the equation $y=f(x)$ for $y$:
\[
\begin{array}{rcl}
x^2+2x+2&=&y\\
x^2+2x+2-y&=&0\\
(x+1)^2 &=& y-1\\
x+1&=&\pm \sqrt{y-1}\\
x&=& -1\pm \sqrt{y-1}\quad .
\end{array}
\]
On the other hand we are given that $x>-1$, and therefore, as  $\sqrt{y-1}>0$ (square roots are positive by definition), we have that $x=f^{-1}(y)=-1+\sqrt{ y-1}$,  $y>1$. Therefore, after relabelling the dummy argument variable of $f^{-1}$ from $y$ to $x$, we get
\[
f^{-1}(x)=-1+\sqrt{x-1}, \quad \quad \quad x>1\quad .
\]
The function $f(x)$, $x>-1$ is easy to plot - that is half of a parabola with a vertex at $x=-1$, $y=f(-1)=1$. On the other hand, the graph of $f^{-1}(x)$ is the reflection of $ f(x)$ across the line $y=x$. This is easy to plot roughly by hand, as expected of you on the test. Using the explicit formulas for $f(x)$ and $f^{-1}(x)$ above, it is very easy to plot the functions by computer, as shown below.

\psset{xunit=1cm, yunit=1cm}
\begin{pspicture}(-5, -5)(5,5)
\psframe*[linecolor=white](-5,-5)(5,5)
\psaxes[ticks=none, labels=none]{<->}(0,0)(-1.5,-1.5)(4.5,4.5)
%Function formula: -1+sqrt{}(-1+x)
\rput(3,1){$y=-1+\sqrt{-1+x}$}
\psplot[linecolor=red, plotpoints=1000]{1}{4.5}{x -1 add sqrt -1 add } %Function formula: 2+2 (x)+(x)^{2}
\rput[l](1,4){$y=x^{2}+2 x+2$}
\psplot[linecolor=red, plotpoints=1000]{-1}{0.870828693}{x 2 exp x 2 mul add 2 add }
\psline[linecolor=blue, linestyle=dashed] (-1.5, -1.5)(4.5, 4.5)
\fcHollowDot{-1}{1}
\fcHollowDot{1}{-1}
\end{pspicture}

\begin{problem} (10 pts)\label{probProblem8Test1}
Compute the limit.
\[\lim\limits_{x\to\infty}\sqrt{x^2+3x}-\sqrt{x^2-\frac{x}3}\quad .
\]
\end{problem}
\textbf{Solution. }
\[
\begin{array}{rcl}
\lim\limits_{x\to\infty}\sqrt{x^2+3x}-\sqrt{x^2-\frac{x}3} &=&
\lim\limits_{x\to\infty}\frac{\left(\sqrt{x^2+3x}-\sqrt{x^2-\frac{x}3}\right)\left(\sqrt{x^2+3x}+\sqrt{x^2-\frac{x}3}\right)}{\sqrt{x^2+3x}+\sqrt{x^2-\frac{x}3}}=\lim\limits_{x\to\infty} \frac{\sqrt{x^2+3x}^2-\sqrt{x^2-\frac{x}3}^2 }{\sqrt{x^2+3x}+\sqrt{x^2-\frac{x}3}}\\
&=& \lim\limits_{x\to\infty} \frac{ \cancel{x^2}+3x\cancel{-x^2}+\frac{x}{3} }{\sqrt{x^2+3x}+\sqrt{x^2-\frac{x}3}}=\lim\limits_{x\to \infty} \frac{\frac{10}3\cancel{x} \frac{1}{\cancel{x}}}{\left( \sqrt{x^2+3x}+\sqrt{x^2-\frac{x}3}\right) \frac{1}x } =
\lim\limits_{x\to \infty} \frac{\frac{10}3}{\sqrt{1+\frac{3}{x}}+\sqrt{1-\frac{1}{3x}}}\\
&=&\frac{\frac{10}3}{\sqrt{1+\lim\limits_{x\to \infty}\frac{3}x}+ \sqrt{1-\lim\limits_{x\to \infty}\frac{1}{3x}}}= \frac{\frac{10}{3}}{\sqrt{1+0}+\sqrt{1-0}}= \frac{10}6= \frac{5}3\quad .
\end{array}
\]

%\noindent {The problems so far are worth 100\% of the grade. Solving this problem provides extra 15\%. }
\begin{problem} (15 pts)
Compute the limit.
\[
\lim_{x\to \infty} \sqrt{ 3^{2x}+3^{x+1}}-\sqrt{3^{2x}-3^{x-1}}\quad .
\]
\end{problem}
\textbf{Solution. } The function mapping $x$ to $3^x$ is one to one and increasing.  Furthermore, as $x\to \infty$, we have hat $3^x\to \infty$. Therefore we have the right to change variables $u=3^x$. Therefore we have that
\[
\lim\limits_{\substack{x\to \infty \\ u=3^x \\ u\to \infty}} \sqrt{ 3^{2x}+3^{x+1}}-\sqrt{3^{2x}-3^{x-1}} = \lim\limits_{ 3^x=u\to \infty}  \sqrt{ (3^{x})^2+3\times 3^{x}}-\sqrt{(3^{x})^2-\frac{3^{x}}{3}}= \lim \limits_{u\to \infty} \sqrt{ u^2+3u}-\sqrt{u^2-\frac{u}{3}}= \frac{5}{3}\quad,
\]
where we have noticed that last limit is identical to that given in Problem \ref{probProblem8Test1}, and we have already computed it in the solution to the previous problem.
\end{document}
