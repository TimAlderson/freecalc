\documentclass{article}[12pt]
\pagestyle{empty}

\usepackage{array}
\usepackage{amsmath}
\usepackage{graphicx}
\usepackage{../homework-problems}

\newcommand{\ds}{\displaystyle}
\newcommand{\dx}{\; dx}

\begin{document}


\begin{center}
MATH 140 Fall 2013 Final exam practice problems
\end{center}


\begin{problem}
Find the following limits, or show that they do not exist:
\begin{multicols}{2}
\begin{enumerate}
\item ${\ds \lim_{x \to 2} \frac{x^2-4}{x^2-x-2}}$
\answer{$\frac43$}
\item ${\ds \lim_{x \to -\infty} \frac{5x^3+x-1}{2x^3-7}}$
\answer{$5/2$}
\item ${\ds \lim_{x \to 1^{+}} \frac{x-3}{x-1}}$
\answer{$-\infty$}
\item ${\ds \lim_{h \to 0} \frac{2(x+h)^3 - 2x^3}{h}}$
\answer{$6x^2$}
\item ${\ds \lim_{x \to \infty} \frac{\sqrt{9x^2-2}}{x+4}}$
\answer{$3$}
\item ${\ds \lim_{x \to -1} \frac{2x+3}{x+1}}$
\answer{Does not exist}
\end{enumerate}
\end{multicols}
\end{problem}

\begin{problem}
Find the derivative of the following functions.
\begin{multicols}{2}
\begin{enumerate}
\item ${\ds \frac{\sin x}{x^2}}$
\answer{${\ds \frac{x^2 \cos x - 2x \sin x}{x^4}}$}
\item ${\ds e^{\sqrt{x^2 + 1}}}$
\answer{${\ds e^{\sqrt{x^2 + 1}} \cdot (1/2)(x^2+1)^{-1/2} \cdot 2x = \frac{x e^{\sqrt{x^2+1}}}{\sqrt{x^2+1}}}$} 
\item ${\ds \ln (x-1/x)}$
\answer{${\ds \frac{1}{x-1/x} \cdot (1 + 1/x^2)}$} 
\item ${\ds \sqrt[3]{x} \ln x}$
\answer{${\ds (1/3)x^{-2/3} \ln x + \sqrt[3]{x} \cdot (1/x)}$} 
\item ${\ds \cos(e^x)}$
\answer{${\ds -\sin(e^x) \cdot e^x}$} 
\item ${\ds \sin^3(2x)}$
\answer{${\ds 3 \sin^2(2x) \cdot \cos(2x) \cdot 2 = 6 \sin^2(2x) \cos(2x)}$} 
\item ${\ds f(x) = \int_x^1 (2+t^4)^5 \; dt}$
\answer{${\ds -(2+x^4)^5}$} 
\item ${\ds g(x) = \int_{0}^{x^3} \cos^2 t \; dt}$
\answer{${\ds 3x^2 \cos^2(x^3)}$} 
\item Find $y'$ if $2x^2 + x + xy = 1$.
\answer{${\ds y' = \frac{-4x-1-y}{x}}$} 
\item Find $y'$ if $x \sin y + y \sin x = 4$.
\answer{${\ds y' = \frac{-\sin y - y \cos x}{\sin x + x \cos y}}$} 
\end{enumerate}
\end{multicols}
\end{problem}
% Integral problems, including various subs

\begin{problem}
Evaluate the following integrals.
\begin{multicols}{2}
\begin{enumerate}
\item ${\ds \int e^x \sqrt{e^x + 1} \dx}$
\answer{${\ds (2/3) (e^x + 1)^{3/2} + C}$}
\item ${\ds \int \frac{x^4 + 3x}{x^2} \dx}$
\answer{${\ds x^3/3 + 3 \ln |x| + C}$}
\item ${\ds \int x^2 e^{x^3} \dx}$
\answer{${\ds (1/3) e^{x^3} + C}$}
\item ${\ds \int \frac{\cos x}{\sin x} \dx}$
\answer{${\ds \ln |\sin x| + C}$}
\end{enumerate}
\end{multicols}
\end{problem}

\begin{problem}~
\begin{enumerate}
\item Find $f(x)$ if $f'(x) = 3 + 1/x$ and $f(1) = 2$.
\answer{$f(x) = 3x + \ln |x| - 1$}
\item Find $g(x)$ if $g'(x) = x - \sin x$ and $f(0) = 7$.
\answer{$g(x) = x^2/2 + \cos x + 6$}
\end{enumerate}
\end{problem}

% Graphing problems
\begin{problem}~
\begin{enumerate}
\item Sketch the graph of $y = x^4 - 8x^2 + 8$ by determining the intervals of increase and decrease, finding the local mins and maxes, determining where the graph is concave up and concave down, and plotting a few key points.

\answer{Check your graph with a calculator or online graphing program. Local max at 0, local mins at 2 and -2. Concave down
between $-\sqrt{4/3}$ and $\sqrt{4/3}$, and concave up otherwise.}

\item Sketch the graph of $y = \frac{x-1}{x^2-9}$ by graphing any vertical and horizontal asymptotes, finding the $x$- and
$y$-intercepts, and then sketching a graph that fits this information.

\answer{Check your graph with a calculator or online graphing program. Vertical asymptotes at $x = 3$ and $x = -3$. Horizontal
asymptote at $y = 0$. $y$-intercept of $1/9$; $x$-intercept of $1$.}
\end{enumerate}
\end{problem}
% Linear approx

\begin{problem}
\begin{enumerate}
\item Find the linearization of the function $f(x) = \sqrt{x}$ at $a = 100$, and then use your new function to approximate
$\sqrt{99.8}$.

\answer{$L(x) = 10 + 0.05(x-100)$. Therefore $\sqrt{99.8} \approx L(99.8) = 9.99$.}
\item Use a linear approximation to estimate $(1.001)^9$. 

\answer{$(1.001)^9 \approx 1.009$.}
\end{enumerate}
\end{problem}

\begin{problem}~
\begin{enumerate}
% Riemann sums
\item Estimate $\int_0^4 \sqrt{8x+1}$ using a Riemann sum with four intervals of equal width, and taking the sample point to be the left endpoint. (That is, estimate the integral using 4 rectangles and left endpoints.)

\answer{ $\Delta x = 1$ and $f(x) = \sqrt{8x+1}$. ${\ds \int_0^4 f(x) \dx \approx 9 + \sqrt{17}}$.}
\item Estimate $\int_0^6 \frac{1}{x^2+1}$ using a Riemann sum with three intervals of equal width, and taking the sample point to be the left endpoint. (That is, estimate the integral using 3 rectangles and left endpoints.)

\answer{ $\Delta x = 2$ and $f(x) = \frac{1}{x^2+1}$. ${\ds \int_0^6 f(x) \dx \approx 214/85}$.}
\end{enumerate}
\end{problem}

\begin{problem}~
\begin{enumerate}
% Optimization
\item What is the $x$-coordinate of the point on the hyperbola $x^2 - 4y^2 = 16$ that is closest to the point $(1, 0)$? 

\answer{$x = 4/5$}
\item You want to build a rectangular box with a square base out of sheet metal. You are going to use 2 pieces of sheet metal for the bottom of the box to reinforce it, and only a single piece of sheet metal for all of the sides and the top. If you want
to use no more than $36$ sq. ft. of material, what is the largest possible volume you can enclose?

\answer{12 cubic feet.}
\end{enumerate}
\end{problem}

\begin{problem}~
\begin{enumerate}
% MVT
\item Use the Intermediate Value Theorem to show that $f(x) = x^3 + 4x - 7$ has at {\bf least} one zero. (That is, show that 
there is at least one solution to $x^3 + 4x - 7 = 0$.) See the example after the statement of the Intermediate Value Theorem (Lecture 6).

\answer{$f(0) = -7$ and $f(2) = 9$. Since $f(x)$ is continuous and it has both negative and positive outputs, it must
have a zero. In other words, for some $c$ between $0$ and $2$, $f(c) = 0$.}

\item Use the Mean Value Theorem or Rolle's Theorem to show that $f(x) = x^3 + 4x - 7$ has at {\bf most} one zero. 
See example after the statement of the Mean Value Theorem (Lecture 19).

\answer{
\begin{tabular}{l}
If there were solutions $x = a$ and $x = b$, then we would have $f(a) = f(b)$, and Rolle's Theorem would guarantee
that for some $x$-value, $f'(x) = 0$. \\
However, $f'(x) = 3x^2 + 4$, which is never 0. Therefore there cannot be 2 or more
solutions. 
(In other words: Note that $f'(x) = 3x^2 + 4$, \\which is always positive. Therefore, the graph of $f$ is increasing 
from left to right. So once the graph crosses the $x$-axis, \\it can never turn around and cross again, so there can only be a single zero
(that is, a single solution to $f(x) = 0$).)
\end{tabular}
}
\end{enumerate}
\end{problem}

\begin{problem}~
\begin{enumerate}
% Related Rates
\item A spherical soap bubble is slowly shrinking. If its surface area is decreasing at a rate of 50 square millimeters per
second, how quickly is the radius decreasing when the surface area is 1000 square millimeters?
\answer{$dr/dt = \frac{-50}{8\pi \sqrt{250/\pi}} = \frac{\sqrt{10}}{8 \pi}$.}

\item A car drives along an elliptical track. The track can be modeled by the equation $x^2 + 5y^2 = 14$, where $x$ and $y$ are 
measured in kilometers of distance from the center of the track. As the car passes the point $(3, 1)$, the $x$-coordinate is 
increasing at a rate of $1.5$ km/min. How quickly is the $y$-coordinate changing at that point?

\answer{$dy/dt = -9/10$ km/min.}
\end{enumerate}
\end{problem}



\begin{problem}~
\begin{enumerate}
% Area problems
\item Find the area of the region bounded by the curves $y = 2x^2$ and $y = 4 + x^2$.

\answer{$32/3$}
\item Find the area of the region bounded by the curves $x = 4 - y^2$ and $y = 2 - x$

\answer{$9/2$}
\end{enumerate}
\end{problem}

\begin{problem}~
\begin{enumerate}
% Volume problems
\item Consider the region bounded by the curves $y = 1-x^2$ and $y =0$. What is the volume of the solid obtained by
rotating this region about the line $y = 0$?

\answer{$16 \pi / 15$}
\item Consider the region bounded by the curves $y = x^2$ and $x = y^2$. What is the volume of the solid obtained by
rotating this region about the line $x = 2$?

\answer{ $31 \pi / 30$}
\end{enumerate}
\end{problem}



\end{document}

