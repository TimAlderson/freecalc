\documentclass{article}
\usepackage{amsmath, amsfonts, amssymb, verbatim, hyperref}
\usepackage{auto-pst-pdf}
\usepackage{pst-plot}
\usepackage{multicol}
\addtolength{\hoffset}{-3.5cm}
\addtolength{\textwidth}{6.8cm}
\addtolength{\voffset}{-3.3cm}
\addtolength{\textheight}{6.3cm}
\renewcommand{\Re}{\mathrm{Re~}}
\renewcommand{\Im}{\mathrm{Im~}}
\newcommand{\doublebrace}[4]{\left\{\begin{array}{ll} #1 & #2 \\#3 & #4  \end{array} \right.}
%Don't know where is the proper place to add such commands, adding here for lack of better placement.
\newcommand{\psHollowDot}[2]{
\pscircle*[fillcolor=white, linecolor=red](#1, #2){0.07}
\pscircle*[fillcolor=white, linecolor=white](#1, #2){0.04}
}
\newcommand{\psHollowDotBlue}[2]{
\pscircle*[fillcolor=white, linecolor=blue](#1, #2){0.07}
\pscircle*[fillcolor=white, linecolor=white](#1, #2){0.04}
}
\newcommand{\psFullDot}[2]{
\pscircle*[fillcolor=white, linecolor=red](#1, #2){0.07}
}
\newcommand{\psFullDotBlack}[2]{
\pscircle*[fillcolor=white, linecolor=black](#1, #2){0.07}
}
\newcommand{\psFullDotBlue}[2]{
\pscircle*[fillcolor=white, linecolor=blue](#1, #2){0.07}
}

\newcommand{\psLabelXOne}{\psline(1, -0.1)(1,0.1) \rput[t](1, -0.2 ) {\footnotesize $1$} }
\newcommand{\psLabelYOne}{\psline(-0.1, 1)(0.1, 1) \rput[r](-0.2, 1 ) {\footnotesize $1$} }


\newcommand{\bigFatWarning}{ %\textbf{This homework contains copyrighted material from  James Stewart, Calculus, 7th edition, 2012. You are not permitted to copy this file for any purpose other than completing your homework. You are not allowed to give a copy of this file to anyone outside of our course. }
}
\newcommand{\answer}[1]{ \hfill{~} \rotatebox{180}{ answer:#1}}
\newcommand{\inputOptional}[1]{}

\date{}
\newtheorem{problem}{Problem}
\newcommand{\homeworkEnd}{\end{document}}
\newcommand{\homeworkStart}[1]{\title{#1}\begin{document}\maketitle\bigFatWarning}

\begin{comment}
\homeworkStart{Special homework}
\begin{problem}Compute the expressions $(f\circ g)(x)$, $(g\circ f)(x)$ and simplify to a single fraction. 

\begin{enumerate}
\item $f{}({{x}})=\frac{x+2}{x-2},
g{}({{x}})=\frac{x-1}{x+2}$.
\answer{$(f\circ g)(x)= \frac{3+3 x}{-5- x}$, $(g\circ f)(x)=\frac{4}{-2+3 x}$  }
\item 
$f{}({{x}})=\frac{x+1}{3x-2},
g{}({{x}})=\frac{x-2}{x-1}
$.
\answer{
$(f\circ g)(x)= \frac{-3+2 x}{-4+x}
$, 
$(g\circ f)(x)=\frac{5-5 x}{3-2 x}
$  }

\item 
$f{}({{x}})=\frac{2x+1}{3x-1},
g{}({{x}})=\frac{x-2}{2x-1}
$.
\answer{
$(f\circ g)(x)=\frac{-5+4 x}{-5+x}
$, 
$(g\circ f)(x)=\frac{3-4 x}{3+x}
$  }

\item 
$f{}({{x}})=\frac{x+1}{x-2},
g{}({{x}})=\frac{x+2}{2x-1}
$.
\answer{
$(f\circ g)(x)= \frac{1+3 x}{4-3 x}
$, 
$(g\circ f)(x)=\frac{-3+3 x}{4+x}
$  }

\item 
$f{}({{x}})=\frac{5x+1}{4x-1},
g{}({{x}})=\frac{4x-1}{3x+1}
$.
\answer{
$(f\circ g)(x)= \frac{-4+23 x}{-5+13 x}
$, 
$(g\circ f)(x)=\frac{5+16 x}{2+19 x}
$  }


\end{enumerate}
\end{problem}

\homeworkEnd
%\homeworkStart{Homework Math 140 on Lectures 1 and 2 \\Will be quizzed Friday September 6}
\begin{problem}(Stewart, 7th ed., page 21, 31-37) Find the implied domain of the function
\begin{multicols}{2}
\begin{enumerate}
\item $f(x)=\frac{x+4}{x^2-9}$.
\item $f(x)=\frac{2x^3-5}{x^2+x-6}$.
\item $f(t)=\sqrt[3]{2t-1}$.
\item $g(t)=\sqrt{3-t}-\sqrt{2+t}$.
\item $h(x)=\frac{1}{\sqrt[4]{x^2-5x}}$.
\item $f(u)=\frac{u+1}{1+\frac{1}{u+1}}$.
\item $F(p)=\sqrt{2-{\sqrt{p}}}$.
\end{enumerate}
\end{multicols}
\end{problem}
\begin{problem} Find the functions $f\circ g$, $g\circ f$, $f\circ f$ and $g\circ g$ and their implied domains.

\begin{enumerate}
\item $f(x)=x^2+1$, $g(x)=x+1$. \answer{ in some order: $(1+x)^{2}+1, (x)^{2}+2, ((x)^{2}+1)^{2}+1, 2+x$ }
\item $f(x)=\sqrt{x+1}$, $g(x)=x+1$. \answer{ in some order:$\sqrt{2+x}, 1+\sqrt{1+x}, \sqrt{1+\sqrt{1+x}}, 2+x$}
\item $f(x)= 2x$, $g(x)= \tan x$.
\answer{ in some order:$2 \tan{}x, \tan{}(2 x), 4 x, \tan{}(\tan{}x) $}
\item $f(x)=\frac{x+1}{x-1}$, $g(x)=\frac{x-1}{x+1}$.
\answer{ in some order:$- x, \frac{1}{x}, x, -\frac{1}{x} $}
\end{enumerate}
\end{problem}

Evaluate the difference and simplify your answer.
\begin{multicols}{2}
\begin{enumerate}
\item $\frac{f(2+h)-f(2)}{h}$, where $f(x)=x^2-x-1$.

\answer{$ h+5 $}
\item $\frac{f(a+h)-f(a)}{h}$, where $f(x)= x^2$.

\answer{$ h+2 a  $}
\item $\frac{f(a+h)-f(a)}{h}$, where $f(x)= x^3$.

\answer{$h^{2}+3 a^{2}+3 a h  $}
\item $\frac{f(a+h)-f(a)}{h}$, where $f(x)= x^4$.

\answer{$ 6 a^{2} h+4 a h^{2}+h^{3}+4 a^{3}$}
\item $\frac{f(x)-f(a)}{x-a}$, where $f(x)=\frac{1}{x}$.

\answer{$-\frac{1}{ax}$.}
\item $\frac{f(x)-f(1)}{x-1}$, where $f(x)=\frac{x-1}{x+1}$.
\answer{$\frac{1}{x+1}$.}
\end{enumerate}
\end{multicols}

\begin{problem}(Stewart, 7ed., page 21, 45, 46, 49)
Plot the piecewise defined functions.
\begin{multicols}{2}
\begin{enumerate}
\item $G(x)=\frac{3x+|x|}x$.
\item $g(x)=|x|-x$.
\item $f(x)=\doublebrace{x+2}{x\leq -1}{x^2}{x\geq -1}$.
\end{enumerate}
\end{multicols}
\end{problem}
\begin{problem}(Stewart, 7ed, page 21, 55-56)
Write down formulas for function whose graphs are as follows. The graphs are up to scale. The arc is a part of a circle.
\begin{multicols}{2}
\begin{enumerate}
\psset{xunit=0.4cm, yunit=0.4cm}
\item 
\tiny
\begin{pspicture}(-1,-1)(6,5)
\psaxes{->}(0,0)(-1,-1)(6,5)
\psline[linecolor=red](0,3)(3, 0)(5, 4)
\psFullDot{5}{4}
\rput[r](4.9, 4){$(5, 4)$}
\rput[b](6,0.1 ){$x$}
\rput[l](0.1,5 ){$y$}
\end{pspicture}
\normalsize
\item 
\tiny
\psset{xunit=0.4cm, yunit=0.4cm}
\begin{pspicture}(-4,-1)(4,5)
\psaxes{->}(0,0)(-4.5,-1)(4.5,4)
\psplot[linecolor=red]{-2}{2}{4 x x mul sub sqrt }
\psline[linecolor=red](2,0)(4,3)
\psline[linecolor=red](-2,0)(-4,3)
\rput[b](4.5,0.1 ){$x$}
\rput[l](0.1,5 ){$y$}
\psFullDot{4}{3}
\rput[r](3.9, 3){$(4, 3)$}
\psFullDot{-4}{3}
\rput[l](-3.9, 3){$(-4, 3)$}

\end{pspicture}
\normalsize
\end{enumerate}


\end{multicols}
\end{problem}
Graph the functions roughly by hand, by applying consecutively the transformations learned in class. Each consequtive graph is a transformations of the preceding one. Compare your answer with the graph produced by a graphic calculator.
\begin{multicols}{2}
\begin{enumerate}
\item $y=\frac{1}{x}$.
\item $y=\frac{1}{x+1}$.
\item $y=\frac{1}{2x+1}$.
\item $y=\frac{3}{2x+1}$.
\item $y=\frac{3+x}{2x+1}$.
\item $y=\left|\frac{3+x}{2x+1}\right|$.
\end{enumerate}
\end{multicols}

%\homeworkEnd
\homeworkStart{Homework Math 140 on Lectures 3 \\Will be quizzed Friday September 13}
\begin{problem} (Textbook page A32, problems 1-6). 
Convert from degrees to radians.
\begin{multicols}{3}
\begin{enumerate}
\item $210^\circ$.
\item $300^\circ$.
\item $9^\circ$.
\item $-315^\circ$.
\item $900^\circ$.
\item $36^\circ$.
\end{enumerate}
\end{multicols}
\end{problem}
\begin{problem} (Textbook page A32, problems 7-12). 
Convert from radians to degrees.
\begin{multicols}{3}
\begin{enumerate}
\item $4\pi$.
\item $-7/2\pi$.
\item $5/12\pi$.
\item $8/3\pi$.
\item $-3/8\pi$.
\item $5$.
\end{enumerate}
\end{multicols}
\end{problem}
Derive the trigonometry identities.
\begin{multicols}{3}
\begin{enumerate}
\item $\sin \theta\cot \theta =\cos \theta$.
\item $(\sin \theta +\cos \theta)^2=1+\sin(2\theta)$.
\item $\sec \theta - \cos \theta= \tan \theta \sin \theta$.
\item $\tan^2 \theta-\sin^2 \theta=\tan^2\theta\sin^2\theta$.
\item $\cot^2\theta+\sec^2\theta=\tan^2\theta+\csc^2\theta$.
\item $2\csc 2\theta= \sec \theta \csc \theta$.
\item $\tan (2\theta) =\frac{2\tan \theta}{1-\tan^2\theta} $.
\item $\frac{1}{1-\sin \theta}+ \frac{1}{1+\sin \theta}=2\sec^2\theta$.
\item $\tan \alpha + \tan \beta = \frac{\sin (\alpha+\beta)}{\cos \alpha \cos \beta}$.
\item $\tan (\alpha+\beta)= \frac{\tan \alpha +\tan \beta}{1-\tan \alpha\tan \beta}$.
\item $\sin (3\theta) +\sin \theta = 2 \sin 2\theta \cos \theta $.
\item $\cos (3\theta) = 4\cos^3\theta-3\cos \theta $.
\end{enumerate} 
\end{multicols}

(Textbook page A33, problems 65-72).
Find all values of $x$ in the interval $[0,2\pi]$ that satisfy the 
equation.
\begin{multicols}{2}
\begin{enumerate}
\item $2\cos x - 1=0$. \\ \answer{$x=\frac{\pi}{3}$ or $x=\frac{5\pi}3$} 
\item $3\cot^2 x= 1$. \\ \answer{$x=\frac{\pi}{3}$, $x=\frac{2\pi}3$, $x=\frac{4\pi}3$, or $x=\frac{5\pi}3$}
\item $2\sin^2 x= 1$. \\ \answer{$x=\frac{\pi}{4}$, $x=\frac{3\pi}4$, $x=\frac{5\pi}4$, or $x=\frac{7\pi}4$}
\item $|\tan x|=1 $. \\ \answer{$x=\frac{\pi}{4}$, $x=\frac{3\pi}4$, $x=\frac{5\pi}4$, or $x=\frac{7\pi}4$}
\item $\sin 2x = \cos x $.  \\ \answer{$x=\frac{\pi}{2}$, $x=\frac{3\pi}2$, $x=\frac{\pi}6$, or $x=\frac{5\pi}6$}
\item $2\cos x +\sin 2x=0$.\\ \answer{$x=\frac{\pi}{2}$, $x=\frac{3\pi}2$}
\item $\sin x =\tan x$. \\ \answer{$x=0$, $x=\pi$, or $x=2\pi$}
\item $2+\cos 2x = 3 \cos x$. \\ \answer{$x=0$, $x=2\pi$, $x=\frac{\pi}{3}$, or $x=\frac{5\pi}{3}$}
\end{enumerate}
\end{multicols}

\homeworkEnd
\end{comment}
\homeworkStart{Homework Math 140, Lectures 4 and 5. \\ Will be quizzed Friday September 20}
\begin{problem}(Textbook page 69, problems 3-9). 
Evaluate the limits. Justify your computations.
\begin{multicols}{3}
\begin{enumerate}
\item $\displaystyle\lim\limits_{x\to 3} 5x^3-3x^2+x-6$.
\item $\displaystyle\lim\limits_{x\to -1} (x^4-3x)(x^2+5x+3)$.
\item $\displaystyle\lim\limits_{t\to -2}\frac{t^4-2}{2t^2-3t+2} $.
\item $\displaystyle\lim\limits_{u\to -2}\sqrt{u^4+3u +6}$.
\item $\displaystyle\lim\limits_{x \to 8}(1+\sqrt[3]{x})(2-6x^2+x^3)$.
\item $\displaystyle\lim\limits_{t \to 2}\left(\frac{t^2-2}{t^3-3t+5} \right)^2$.
\item $\displaystyle\lim\limits_{x\to 2}\sqrt{\frac{2x^2+1}{3x-2}}$.
\end{enumerate}
\end{multicols}
\end{problem}
\begin{problem}(Textbook page 70, problems 11-32). 
Evaluate the limit if it exists.
\begin{multicols}{3}
\begin{enumerate}
\item $\displaystyle\lim\limits_{x\to 5}\frac{x^2-6x+5}{x-5} $. 
\answer{4}
\item $\displaystyle\lim\limits_{x\to 4}\frac{x^2-4x}{x^2-3x-4} $.
\answer{$\frac{4}5$}
\item $\displaystyle\lim\limits_{x\to 5}\frac{x^2-5x+6}{x-5} $.
\answer{DNE}
\item $\displaystyle\lim\limits_{x\to -1}\frac{x^2-4x}{x^{2}-3x-4} $.
\answer{DNE}
\item $\displaystyle\lim\limits_{t\to -3}\frac{t^2-9}{2t^2+7t+3} $.
\answer{$\frac{6}{5}$}
\item $\displaystyle\lim\limits_{x\to -1}\frac{2x^2+3x+1}{x^2-2x-3} $.
\answer{$\frac{1}{4}$}
\item $\displaystyle\lim\limits_{h\to 0}\frac{(-5+h)^2-25}{h} $.
\answer{$-10$}
\item $\displaystyle\lim\limits_{h\to 0}\frac{(2+h)^3-8}{h} $.
\answer{$12$}
\item $\displaystyle\lim\limits_{x\to -2}\frac{x+2}{x^3+8} $.
\answer{$\frac{1}{12}$}
\item $\displaystyle\lim\limits_{t\to 1}\frac{t^4-1}{t^3-1} $.
\answer{$\frac{4}{3}$}
\item $\displaystyle\lim\limits_{h\to 0}\frac{\sqrt{9+h}-3}{h} $.
\answer{$\frac{1}{6}$}
\item $\displaystyle\lim\limits_{u\to 2} \frac{\sqrt{4u+1}-3}{u-2}$.
\answer{$\frac{2}{3}$}
\item $\displaystyle\lim\limits_{x\to -4} \frac{\frac{1}{4}+ \frac{1}{x}} {4+x}$.
\answer{$-\frac{1}{16}$}
\item $\displaystyle\lim\limits_{x\to -1} \frac{x^2+2x+1}{x^4-1}$.
\answer{$0$}
\item $\displaystyle\lim\limits_{t\to 0} \frac{\sqrt{1+t}- \sqrt{1-t}}{t}$.
\answer{$2$}
\item $\displaystyle\lim\limits_{t\to 0}\left(\frac{1}t -\frac{1}{t^2+t}\right)$.
\answer{$1$}
\item $\displaystyle\lim\limits_{x\to 16} \frac{4-\sqrt{x}}{16x-x^2}$.
\answer{$\frac{1}{128}$}
\item $\displaystyle\lim\limits_{h \to 0}\frac{(3+h)^{-1}-3^{-1}}{h} $.
\answer{$-\frac{1}{9}$}
\item $\displaystyle\lim\limits_{t\to 0} \left(\frac{1}{t\sqrt{1+t}}-\frac{1}{t} \right)$.
\answer{$-\frac{1}{2}$}
\item $\displaystyle\lim\limits_{x\to -4} \frac{\sqrt{x^2+9}-5}{x+4}$.
\answer{$-\frac{4}{5}$}
\item $\displaystyle\lim\limits_{h\to 0}\frac{(x+h)^3-x^3}{h} $.
\answer{$3x^2$}
\item $\displaystyle\lim\limits_{h\to 0}\frac{\frac{1}{(x+h)^2}-\frac{1}{x^2}}{h} $.
\answer{$-\frac{2}{x^3}$}
\end{enumerate}
\end{multicols}
\end{problem}
\homeworkEnd
Evaluate the definite integral.
\begin{multicols}{3}
\begin{enumerate}
\item $\displaystyle \int\limits_{-2}^{3} \left(x^2-1 \right)  \diff x$.

\answer{$\left[ \frac{1}{3} x^{3}- x \right]_{-2}^3=\frac{20}{3}$}

\item $\displaystyle \int\limits_{1}^{2} \left(4x^3+3x^2+2x+1\right)  \diff x$.

\answer{$\left[ x^{4}+x^{3}+x^{2}+x\right]_{1}^{2}=26$}
\item $\displaystyle \int\limits_{0}^{2}(x-1)(x^2+1)  \diff x$.

\answer{$\left[\frac{1}{4} x^{4}-\frac{1}{3} x^{3}+\frac{1}{2} x^{2}- x \right]_{0}^{2} = \frac{4}{3}$}
\item $\displaystyle \int\limits_{-1}^{1} \left( \frac{x(x+1) }{ 2} \right)^2  \diff x$.

\answer{$\left[\frac{1}{20} x^{5}+\frac{1}{8} x^{4}+\frac{1}{12} x^{3} \right]_{-1}^{1}=\frac{4}{15}$}
\item $\displaystyle \int\limits_{0}^{1}(1+x^2)^3 dx$.

\answer{$\left[\frac{1}{7} x^{7}+\frac{3}{5} x^{5}+x^{3}+x \right]_{0}^{1}=\frac{96}{35}$}
\item $\displaystyle \int \limits_{1}^{2} \left(\frac{1}{x} - \frac{4}{x^2} \right)  \diff x$.

\answer{$\left[ 4 x^{-1}+\ln x\right]_{1}^{2}=\ln 2-2$}
\item $\displaystyle \int\limits_{1}^{4}\sqrt{x}(1+x) \diff x$.

\answer{$\left[\frac{2}{5} x^{\frac{5}{2}}+\frac{2}{3} x^{\frac{3}{2}} \right]_{1}^{4}=\frac{256}{15}$}
\item $\displaystyle \int\limits_{1}^{4}\sqrt{\frac{6}{x}} \diff x$.

\answer{$\left[ \right]_{}^{}=$}
\item $\displaystyle \int \limits_{1}^{4} \frac{\frac{1}{\sqrt{x}}+1+x}{ \sqrt{x}}  \diff x$.

\answer{$\left[ \right]_{}^{}=$}
\item $\displaystyle \int \limits_{1}^{4} \frac{1+x}{ \sqrt[3]{x}} \diff x$.

\answer{$\left[ \right]_{}^{}=$}
\item $\displaystyle \int\limits_{0}^1 \frac{\frac{1}{\sqrt[3]{x}}+\sqrt[3]{x}}{ \sqrt{x}}\diff x$.

\answer{$\left[ \right]_{}^{}=$}
\item $\displaystyle \int\limits_0^{1} (\sqrt[5]{x^6}+\sqrt[6]{x^5})\diff x $.

\answer{$\left[ \right]_{}^{}=$}
\item $\displaystyle \int\limits_{1}^{2} \left(x +\frac{1}{x} \right)^2 \diff x$.

\answer{$\left[ \right]_{}^{}=$}
\item $\displaystyle \int\limits_{1}^{2} \left(x +\frac{1}{x} \right)^3 \diff x$.

\answer{$\left[ \right]_{}^{}=$}
\item $\displaystyle \int\limits_{1}^{2} \left(\sqrt{x} +\frac{1}{\sqrt{x}} \right)^2 \diff x$.

\answer{$\left[ \right]_{}^{}=$}
\item $\displaystyle \int\limits_{1}^{2} \left(\sqrt{x} +\frac{1}{\sqrt{x}} \right)^3 \diff x$.

\answer{$\left[ \right]_{}^{}=$}
\item $\displaystyle \int\limits_{0}^{2}|x-1| \diff x$.

\answer{$\left[ \right]_{}^{}=$}
\item $\displaystyle \int\limits_{0}^{1} \left|x-\frac{1}{2}\right| \diff x$.

\answer{$\left[ \right]_{}^{}=$}
\item $\displaystyle \int\limits_{-1}^{1}(x-3|x|) \diff x$.

\answer{$\left[ \right]_{}^{}=$}
\item $\displaystyle \int\limits_{\frac{\pi}{4}}^{\frac{\pi}{2}} \csc^2\theta \diff \theta$.

\answer{$\left[ \right]_{}^{}=$}
\item $\displaystyle \int\limits_{0}^{\frac{\pi}{4}}\frac{1-\cos^2\theta}{\cos^2\theta} \diff \theta$.

\answer{$\left[ \right]_{}^{}=$}
\item $\displaystyle \int\limits_{0}^{\frac{\pi}{4}}\frac{\sin^2\theta}{\cos^2\theta} \diff \theta$.

\answer{$\left[ \right]_{}^{}=$}
\item $\displaystyle \int\limits_{0}^{\frac{\pi}{4}}\tan^2\theta \diff \theta$.

\answer{$\left[ \right]_{}^{}=$}
\item $\displaystyle \int\limits_{0}^{\frac{\pi}{3}} \frac{\sin \theta +\sin \theta \tan^2\theta}{\sec^2\theta}\diff \theta$.

\answer{$\left[ \right]_{}^{}=$}
\item $\displaystyle \int\limits_{0}^{\pi} (\sin \theta -\cos \theta) \diff \theta$.

\answer{$\left[ \right]_{}^{}=$}
\item $\displaystyle \int\limits_{0}^{\pi}|\sin x| \diff x$.
\end{enumerate}
\end{multicols}

