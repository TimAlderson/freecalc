\documentclass{article}
\usepackage{amsmath, amsfonts, amssymb, verbatim, hyperref}
\usepackage{enumitem}
\usepackage{pst-plot}
\usepackage{pstricks}
\addtolength{\hoffset}{-3.5cm}
\addtolength{\textwidth}{6.8cm}
\addtolength{\voffset}{-3.3cm}
\addtolength{\textheight}{6.3cm}
\title{Math 140 Calculus I \\ Fall 2013}
\begin{document}
%\color{green}
\maketitle
%\noindent\textbf{Time and place.}
%Monday, Wednesday, Friday 10-10:50, McCormack, Room 417, first floor. Monday 11:00-11:50,

\noindent \textbf{Instructor.} Todor Milev, \href{mailto:todor.milev@umb.edu}{\nolinkurl{todor.milev@umb.edu}} \quad \quad \quad .

\medskip
\noindent \textbf{Office hours. } MWF 14:00-15:00, room S-3-65, or by email appointment.

\medskip
\noindent \textbf{Online resources. }  \url{https://piazza.com/umb/fall2013/math140/home}  \quad \quad \quad .


\medskip\noindent \textbf{Textbook. }  James Stewart, Calculus, 7th edition, published by Brooks Cole, 2012. ISBN-13: 978-0-538-49781-7
ISBN-10: 0-538-49781-5.

\medskip \noindent \textbf{Lecture slides. } \url{https://piazza.com/umb/fall2013/math140/resources} \quad \quad \quad .

\medskip\noindent Lecture slides will become available as the course progresses.

%\medskip
%\noindent \textbf{Prerequisite. } A standard pre-calculus course or equivalent.


\medskip
\noindent \textbf{Grades.} Your grade will consist of three tests and a final exam, and a weekly quiz. 
\begin{itemize}
\item The quizzes will account for 10\% of your total grade.
\item The tests will account for 60\% (20\% each) of your total grade.
\item The final will account for 30\% of your total grade.
\end{itemize}
Please note that missed tests can not be made up, unless there is a valid medical reason accompanied with an official signed document from a medical doctor. Letter grades will be assigned as follows. 

\begin{center}
\begin{tabular}{cc|cc}
A & 85-100 & C & 65-69 \\
A-& 82-84 & C- & 62-64 \\
B+& 80-81 & D+ & 60-61 \\
B & 75-79& D & 55-59\\
B-& 72-74& D- & 50-54\\
C+& 70-71& F & below 50\\
\end{tabular}

\end{center}

No books, notes, calculators or any other electronic device (such as mobile phones) are allowed during any exam unless otherwise stated.

\medskip
\noindent \textbf{Homework.} You will be assigned weekly homework, which will be posted on

\url{https://piazza.com/umb/fall2013/math140/resources} \quad \quad \quad .

\noindent I will not collect homework, but I will expect you to complete it in a separate notebook/pieces of paper. I will proofread your homework only if you submit it to me.
 
 \medskip
\noindent \textbf{Weekly quizzes} You will be given 1 (or more if announced) weekly quiz. Your quiz problem will be one of your homework problems, verbatim. I am most likely to quiz you during the last meeting of the week.



\medskip
\noindent \textbf{Student conduct.} Students  are required to adhere the University Policy on Academic Standards and Cheating, to the University Statement of Plagiarism and the Documentation of Written Work, and to the Code of Student Conduct as described in the catalog of Undergraduate programs, pages 44-45 and 48-52. The code is available at the following web-page.

\noindent\url{http://www.umb.edu/life_on_campus/policies/code/}

\medskip
\noindent \textbf{List of topics from previous years.} The list of topics is a preliminary guideline, and will be subject to change.
\begin{enumerate}[label*=\arabic*.]
\item Functions and limits
\begin{enumerate}[label*=\arabic*.]
\item Ways to represent a function.
\item Some essential functions.
\item New functions from old.
\item Tangents; velocity problems.
\item Limit of a function.
\item Calculating limits with limit laws.
\item Continuity.
\end{enumerate}
\item Derivatives
\begin{enumerate}[label*=\arabic*.]
\item Derivatives and rates of change.
\item The derivative as a function.
\item Differentiation formulas.
\item Derivatives of trigonometric functions.
\item The chain rule.
\item Implicit differentiation.
\item Some applications of rates of change.
\item Related rates.
\item Linear approximations and differentials.
\end{enumerate}
\item Applications of differentiation.
\begin{enumerate}[label*=\arabic*.]
\item Maxima and minima.
\item The Mean Value Theorem.
\item Derivatives and shape of a graph.
\item Limits at $\infty$; asymptotes.
\item On curve sketching.
\item On optimization problems.
\item Newton's method.
\item Antiderivatives.
\end{enumerate}
\item Integrals
\begin{enumerate}[label*=\arabic*.]
\item Areas and distances.
\item Definite integrals.
\item The Fundamental Theorem of Calculus.
\item Indefinite integrals and the Net Change Theorem.
\item The substitution rule.
\end{enumerate}
\item First applications of integration.
\begin{enumerate}[label*=\arabic*.]
\item Area between curves.
\item Volumes of solids of revolution.
\end{enumerate}
\item Inverse functions.
\begin{enumerate}[label*=\arabic*.]
\item Inverse functions.
\item Exponential functions and their derivatives.
\item Logarithmic functions.
\item Derivatives of inverse functions.
\end{enumerate}
\end{enumerate}

\end{document}