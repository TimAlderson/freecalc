\documentclass%
%[handout]
{beamer}
% % % % % % % %
% % % % % % % %
% % % % % % % %
%IMPORTANT
%compiles with 
%pdflatex -shell-escape 
%IMPORTANT
% % % % % % % %
% % % % % % % %
% % % % % % % %
\mode<presentation>
{
\useinnertheme{rounded}
\useoutertheme{infolines}
\usecolortheme{orchid}
\usecolortheme{whale}
}

\usepackage[english]{babel}
\usepackage[latin1]{inputenc}
\usepackage[all,cmtip]{xy}
\usepackage{times}
\usepackage[T1]{fontenc}
\usepackage{../example-templates}
\usepackage{../pstricks-commands}
\usepackage{cancel}

\usepackage{auto-pst-pdf}
\usepackage{pst-plot}
%\usepackage{pstricks-add} 

% Or whatever. Note that the encoding and the font should match. If T1
% does not look nice, try deleting the line with the fontenc.

\graphicspath{{../../modules/}}

\newtheoremstyle{partialproof}{3pt}{3pt}{}{}{}{.}{.5em}{}
\theoremstyle{partialproof} \newtheorem{partialproof}[theorem]{Proof.}
%\DeclareMathOperator{\diff}{d}
\newcommand{\diff}{\text{d}}
\setbeamertemplate{navigation symbols}{}

\includeonlylecture{1}

\newcommand{\lect}[3]{
  \date{#1}
  \lecture[#1]{#2}{#3}
}

\setbeamertemplate{footline}
{
  \leavevmode%
  \hbox{%
  \begin{beamercolorbox}[wd=.333333\paperwidth,ht=2.25ex,dp=1ex,center]{author in head/foot}%
    \usebeamerfont{author in head/foot}\insertshortauthor
  \end{beamercolorbox}%
  \begin{beamercolorbox}[wd=.333333\paperwidth,ht=2.25ex,dp=1ex,center]{title in head/foot}%
    \usebeamerfont{title in head/foot}\insertshorttitle
  \end{beamercolorbox}%
  \begin{beamercolorbox}[wd=.333333\paperwidth,ht=2.25ex,dp=1ex,center]{date in head/foot}%
    \usebeamerfont{date in head/foot}\insertshortdate{}
  \end{beamercolorbox}}%
  \vskip0pt%
}

% If you have a file called "university-logo-filename.xxx", where xxx
% is a graphic format that can be processed by latex or pdflatex,
% resp., then you can add a logo as follows:

%\pgfdeclareimage[height=0.8cm]{logo}{bluelogo}
%\logo{\pgfuseimage{logo}}

\begin{document}

\AtBeginLecture{%

\title[\insertlecture]{FreeCalc}
\subtitle{\insertlecture}
\author[FreeCalc]{}
\institute[UMass Boston]{University of Massachusetts Boston}
\date{\insertshortlecture}
\begin{frame}
  \titlepage
\end{frame}
}%

% begin lecture
\lect{\today}{Sample}{1}
%% begin module derivatives-rules-summary
\begin{frame}
\frametitle{Rules of differentiation.}
We studied the basic rules of differentiation.
\begin{itemize}
\item<1->\alertNoH{12}{ $f (g(x))'=f'(g(x)) g'(x) $ (Chain rule). }
\item<2->\alertNoH{12}{ $(f*g)'=f'g+fg'$ (Product rule).}
\item<3->\alertNoH{12}{ $(f+g)'=f'+g'$ (Sum rule). }
\item<4->\alertNoH{12}{ $x'=1$. }
\item<5->\alertNoH{12}{ $(c)'=0$ if $c$ is a constant (Constant derivative rule).}
\end{itemize}
\uncover<6->{We studied additional differentiation rules.}
\begin{itemize}
\item<6->\alertNoH{13}{ $(e^x)'=e^x$ (Exponent derivative rule).}
\item<7->\alertNoH{13}{ $\left(\frac{f}{g}\right)'=\frac{f' g-f g' }{g^2}$ (Quotient rule).}
\item<8->\alertNoH{13}{ $(x^r)'=rx^{r-1} $, $r$-arbitrary real number (Power rule).}
\item<9->\alertNoH{13}{ $(\ln x)'=\frac{1}x$ (Logarithm derivative rule). }
\item<10->\alertNoH{13}{ $(\log_a x)'=\frac{1}{x\ln a}$.}
\item<11->\alertNoH{13}{ $(\sin x)'=\cos x$, $(\cos x)'=-\sin x$}
\end{itemize}

\uncover<12->{We derived \alertNoH{12}{the first set of rules}  by directly computing limits. } \uncover<13->{The \alertNoH{13}{second set of rules} can be derived from the first set \uncover<13>{algebraically}.}
\end{frame}
% end module derivatives-rules-summary

%% begin module power-rule-from-exponent
\begin{frame}
\begin{example}
Let $c$ be a constant. Derive the constant multiple rule 
\[
\alert<6>{(cf)'=c f'}
\]
\uncover<3->{using the \alert<3>{product rule $(fg)'=f'g+fg'$}} \uncover<4->{\alert<4>{and the constant derivative rule $(c)'=0$.}}

\[
\uncover<2->{\alert<3,6>{(cf)'}=} \uncover<3->{\alert<3>{\alert<4>{(c)'} f+ c f'}=}\uncover<4->{\alert<4>{0} f+cf'=}\uncover<5->{\alert<6>{cf'}}
\]
\uncover<6->{\alert<6>{as desired}.}
\end{example}

\end{frame}

%end module power-rule-from-exponent
%% begin module derivatives-rules-relations

\begin{frame}
\begin{example}
Let $n$-positive integer. Derive the positive integer power rules
\[
\alertNoH{5,8}{\left( x^2\right)' =2x }, \quad \quad
\alertNoH{9,12}{\left( x^3\right)' =3x^2}, \quad \quad \alertNoH{13}{\left( x^4\right)' =4x^3}, \quad \quad \alertNoH{19}{\dots}
\]
\uncover<2->{\alertNoH{2}{using the rule $(x)'=1$}} \uncover<4->{\alertNoH{4,7,11,16}{and the product rule}.}

\[
\begin{array}{rcl}
\uncover<2->{\alertNoH{2}{(x)'}&=&\alertNoH{2}{1}}\\
\uncover<3->{\alertNoH{5,8}{(x^2)'}&=& \alertNoH{4}{(x\cdot x)'} = \uncover<4->{\alertNoH{4}{x'x +x x'}=} \uncover<5->{x+x=\alertNoH{5,8}{2x} }} \\
\uncover<6->{\alertNoH{9,12}{(x^{3})'}&=& \alertNoH{7}{(x\cdot x^2)'}=} \uncover<7->{\alertNoH{7}{x' x^2+x\alertNoH{8}{(x^2)'}}=} \uncover<8->{x^2+ x (\alertNoH{8}{2x})=} \uncover<9->{x^2+2x^2= \alertNoH{9,12}{3x^2}}\\
\uncover<10->{\alertNoH{13}{(x^{4})'}&=& \alertNoH{11}{(x\cdot x^3)'}=\uncover<11->{\alertNoH{11}{x' x^3+x\alertNoH{12}{(x^3)'}}=} \uncover<12->{ x^3+ x (\alertNoH{12}{3x^2})=}\uncover<13->{x^3+3x^3=\alertNoH{13}{4x^3}}}\\
\uncover<14->{&\vdots&}\\
\uncover<15->{\alertNoH{17}{(x^{n})'}&=&\dots = \alertNoH{17}{n x^{n-1}}}\\
\uncover<15->{\alertNoH{18}{ (x^{n+1})'}&=&\alertNoH{16}{(x\cdot x^n)'}=}\uncover<16->{\alertNoH{16}{ x' x^n+ x \alertNoH{17}{(x^{n})'}}=} \uncover<17->{x^n+x (\alertNoH{17}{nx^{n-1}})=} \uncover<18->{\alertNoH{18}{(n+1)x^n}}\\
\uncover<19->{&\alertNoH{19}{\vdots}&}
\end{array}
\]
\end{example}


\end{frame}




%end module derivatives-rules-relations

%% begin module derivatives-rules-relations

\begin{frame}
\begin{example}
Let $n$ be a positive integer. Derive the negative integer power rule
\[
(x^{-n})'=\left(\frac{1}{x^n}\right)'= -n x^{-n-1} =-\frac{n}{x^{n+1}}
\]
\uncover<4->{using \alertNoH{4}{the product rule},} \uncover<5->{\alertNoH{5}{the constant derivative rule}} \uncover<6->{and \alertNoH{6}{the power rule for positive integers}.}

\[
\begin{array}{rcl p{2cm} |r}
\uncover<2->{ x^n x^{-n} &=& 1  &&} \uncover<3->{\frac{d}{dx}}\\
\uncover<3->{\alertNoH{4}{( x^n x^{-n})'} &=&\alertNoH{5}{(1)'}  }\\

\uncover<4->{ \alertNoH{4}{ \alertNoH{6}{(x^n)'} x^{-n} + x^n (x^{-n})'}&=&\alertNoH{5}{0}}\\
\uncover<6->{ \alertNoH{7}{\alertNoH{6}{n x^{n-1}}x^{-n}}+ x^n (x^{-n})'&=&0 }\\
\uncover<7->{\alertNoH{7}{\frac{n}{x}}+ x^n (x^{-n})'&=&0 }\\
\uncover<8->{x^n (x^{-n})'&=&-\frac{n}x && \frac{1}{x^n}}\\
\uncover<9->{(x^{-n})'&=&-\frac{n}{x^{n+1}}}
\end{array}
\]

\end{example}



\end{frame}




%end module derivatives-rules-relations

%% begin module power-rule-rationa-from-chain-rule
\begin{frame}
\begin{example}
Derive the power rule $\alertNoH{13}{\left(x^{\frac{1}{q}}\right)'=\frac{1}{q} x^{\frac{1}q-1}}$ \uncover<4->{using \alertNoH{4}{the rule $(x)'=1$},} \uncover<6->{\alertNoH{6}{the chain rule}} \uncover<7->{and the \alertNoH{7}{integer power rule $\frac{d}{du}(u^q)=qu^{q-1} $}.}

\[
\begin{array}{rclp{0.3cm}|l}
\uncover<2->{ \left(x^{\frac{1}{q}}\right)^q &=&x&& \uncover<3->{\alertNoH{3}{\frac{d}{dx}}}}\\
\uncover<3->{ \alertNoH{3}{\left(\left(\alertNoH{5}{x^{\frac{1}q}} \right)^q\right)'} &=& \uncover<4->{\alertNoH{4}{1}} \uncover<3>{\alertNoH{3}{(x)'}} &&\uncover<5->{\alertNoH{5,8}{\text{Set~}u=x^{\frac1q} }}}\\
\uncover<5->{ \alertNoH{6}{\left(\alertNoH{5}{u}^{q}\right)'}&=&1  }  \\
\uncover<6->{\alertNoH{6}{ \alertNoH{7}{\frac{d}{du} \left(u^q\right)} u'}&=&1}\\
\uncover<7->{\alertNoH{7}{q (\alertNoH{8}{u})^{q-1}} (\alertNoH{8}{u})'&=&1 }\\
\uncover<8->{ q (\alertNoH{8}{x^{\alertNoH{9}{\frac{1}q}}})^{\alertNoH{9}{q-1}} \left( \alertNoH{8}{x^{\frac1q}}\right)'&=&1 }\\
\uncover<9->{\alertNoH{10}{q x^{\alertNoH{9}{\frac{q-1}q}}}\left( x^{\frac1q}\right)'&=&1&&\uncover<10->{\alertNoH{10}{\text{divide~by~}q x^{\frac{q-1}q}}}}\\
\uncover<10->{\alertNoH{13}{\left( x^{\frac1q}\right)'}&=&\frac{ 1}{\alertNoH{10}{q \alertNoH{11}{x^{\frac{q-1}q}}}}= \uncover<11->{ \frac{\alertNoH{11}{x^{\alertNoH{12}{-\frac{q-1}{q}}}}}{q}=}\uncover<12->{\alertNoH{13}{\frac{1}q x^{\alertNoH{12}{\frac{1}q-1}}}} &&\uncover<13->{\alertNoH{13}{\text{as~desired}}}}
\end{array}
\]
\end{example}

\end{frame}

%end module power-rule-rationa-from-chain-rule

%% begin module quotient-rule-from-power-and-chain
\begin{frame}
\begin{example}
Derive the quotient rules
\[
\begin{array}{rcl}
\alertNoH{5,9}{\displaystyle\left(\frac{1}{g}\right)'}&=&\displaystyle\alertNoH{5,9}{-\frac{g'}{g^2}}
\\
\displaystyle\alertNoH{11}{\left(\frac{f}{g}\right)'}&=&\displaystyle\alertNoH{11}{\frac{f'g-fg'}{g^2}}
\end{array}
\]
\uncover<3->{using \alertNoH{3}{the chain rule},} \uncover<4->{\alertNoH{4}{the negative power rule}} \uncover<8->{and \alertNoH{8}{the product rule}.}
\[
\begin{array}{rclp{0.3cm}|l}
\displaystyle \uncover<2->{ \alertNoH{3,5}{\left(\frac{1}{g}\right)'} &=&\displaystyle \uncover<3->{\alertNoH{3}{\alertNoH{4}{\frac{d}{dg}\left(\frac1g\right)} g'}=}\uncover<4->{\alertNoH{5}{\alertNoH{4}{-\frac{1}{g^2}}g'}}&&\uncover<5->{\alertNoH{5}{\text{as~desired}}}}\\
\\
\displaystyle
\uncover<6->{\alertNoH{7,11}{\left(\frac{f}{g}\right)'}\uncover<7->{&=&\displaystyle\alertNoH{7,8}{\left( f \frac{1}g\right)'}=}\uncover<8->{\alertNoH{8}{f' \frac{1}{g} +f\alertNoH{9}{\left(\frac{1}g\right)'}}=}\uncover<9->{\alertNoH{10}{\frac{f'}{g}+f\alertNoH{9}{\left(-\frac{g'}{g^2}\right)}}}}\\
\\
\uncover<10->{&=&\displaystyle
%\frac{f'g}{g^2} - \frac{fg'}{g^2}=
\alertNoH{10,11}{\frac{f'g-fg'}{g^2}} &&\uncover<11->{\alertNoH{11}{\text{as~desired}}}}
\end{array}
\]
\end{example}
\end{frame}
%end module quotient-rule-from-power-and-chain

%% begin module power-rule-rationa-from-chain-rule
\begin{frame}

\begin{example}
Derive \alert<11>{the exponent rule $\left(e^x\right)'=e^x$} \uncover<2->{using the Calc II formula below,} \uncover<3->{ \alert<3>{the infinite} {\color{gray!50} (both sides uniformly convergent)} \alert<3>{sum rule $(f_1+f_2 +f_3 +\dots)' =f_1' +f_2'  +f_3' +\dots$}}
\uncover<4->{and \alert<4>{the power rule $(x^n)'=nx^{n-1}$}. } 
\uncover<2->{\[
\alert<2,10>{e^x}=\alert<2,10>{1+x+\frac{x^2}{2!}+\frac{x^3}{3!}+\dots},
\]
}
\uncover<2->{where $n!=1\cdot2\cdot3\cdot \dots\cdot n$.} \uncover<5->{We have that $\alert<5,9>{\frac{n}{\alert<6>{ n!}}} =\uncover<6->{ \frac{\alert<7>{n} }{ \alert<6>{ 1\cdot 2\cdot \dots\cdot (n-1) \cdot  \alert<7>{n}}}=}\uncover<7->{\frac{1}{\alert<8>{1\cdot 2\cdot \dots\cdot (n-1)}}=}\uncover<8->{\alert<9>{ \frac{1}{ \alert<8>{(n-1)!} }}} $.}
\[
\begin{array}{rcl}
\uncover<2->{\alert<11>{\left(\alert<2>{e^x}\right)'} &=&\alert<3>{\left(\alert<2>{1+x+\frac{x^2}{2!}+\frac{x^3}{3!}+\dots} \right)' }} \\
\uncover<3->{&=& \alert<3>{ \alert<4>{(1)'}+\alert<4>{(x)'}+\frac{\alert<4>{(x^2)'}}{2!} +\frac{\alert<4>{(x^3)'}}{3!}+\dots + \frac{\alert<4>{(x^n)'}}{n!}+\dots}}
\\ \uncover<4->{&=&\alert<4>{0}+ \alert<4>{1}+ \frac{\alert<4>{\alert<5-9>{2}x}}{\alert<5-9>{2!}}+ \frac{\alert<4>{\alert<5-9>{3}x^2}}{\alert<5-9>{3!}}+ \dots +\frac{\alert<4>{\alert<5-9>{n}x^{n-1}}}{\alert<5-9>{n!}}+\dots }\\
\uncover<9->{&=& \alert<10>{ \phantom{ 0~ + }  1 + \frac{\phantom{2}x}{\alert<9>{1!}}+\frac{\phantom{3}x^{2}}{\alert<9>{2!}}+\dots+\frac{\phantom{n}x^{n-1}}{\alert<9>{(n-1)!}}+\dots}=}\uncover<10->{\alert<10,11>{e^x}}\\
\end{array}
\]
\uncover<11->{\alert<11>{as desired.}}
\end{example}

\end{frame}

%end module power-rule-rationa-from-chain-rule
% begin module logarithm-rule-from-exponent
\begin{frame}
\begin{example}
Derive the logarithm derivative rules
\[
\begin{array}{rcl}
\alertNoH{11,15}{(\ln x)'}&=&\alertNoH{11,15}{\frac{1}{x}}\\
\alertNoH{16}{(\log_a x)'}&=&\alertNoH{16}{\frac{1}{x\ln a}}
\end{array}
\]
\uncover<5->{using \alertNoH{5}{the chain rule},} \uncover<6->{the \alertNoH{6}{exponent derivative rule $(e^{x})'=e^x$}, } \uncover<7->{\alertNoH{7}{the rule $(x)'=1$}} \uncover<14->{and the \alertNoH{14}{constant multiple rule $(cf)'=cf'$}.}
\[
\begin{array}{rclp{0.3cm}|l}
\uncover<2->{
e^{\alertNoH{3}{\ln x}}&=&x \uncover<3->{&&  \alertNoH{3,8}{ \text{set~~~}u =\ln x }}}\\
\uncover<3->{e^{\alertNoH{3}{u}}&=&x \uncover<4->{& & \alertNoH{4}{ \frac{\diff}{\diff x}}}}\\
\uncover<5->{ \alertNoH{5}{\alertNoH{6}{\frac{\diff}{\diff u}(e^u)} u'}&=&\alertNoH{7}{(x)'}}\\
\uncover<6->{ \alertNoH{6}{e^{\alertNoH{8}{u}}} \alertNoH{8}{u}'&=&\alertNoH{7}{1}}\\
\uncover<8->{\alertNoH{9}{e^{\alertNoH{8}{\ln x}}} (\alertNoH{8}{\ln x})'&=&1}\\
\uncover<9->{\alertNoH{9}{x}(\ln x)'&=&1\uncover<10->{& &\alertNoH{10}{ \cdot \frac{1}{x}}}}\\
\uncover<10->{ \alertNoH{11}{(\ln x)'} &=& \displaystyle\alertNoH{11}{\frac{1}x} \uncover<11->{ & &\alertNoH{11}{ \text{as~desired}}}} 
\uncover<12->{\\ \hline
\alertNoH{16}{(\alertNoH{13}{\log_a x})'} & =&  \uncover<13->{ \alertNoH{14}{\left(\alertNoH{13}{\frac{\ln x}{\ln a}} \right)' }=} \uncover<14->{\alertNoH{14}{ \frac{ \alertNoH{15}{ (\ln x)'}}{\ln a}} =} \uncover<14->{\alertNoH{16}{ \frac{ \alertNoH{15}{1} }{\alertNoH{15}{x}\ln a}}}\uncover<16>{ & &  \alertNoH{16}{ \text{as~desired~}}}}
\end{array}
\]
\end{example}
\end{frame}

%end module module logarithm-rule-from-exponent

%% begin module power-rule-from-exponent
\begin{frame}
\begin{example}

\end{example}

\end{frame}

%end module power-rule-from-exponent
%%begin module sin-cos-from-exponent
\begin{frame}
\begin{example}
Derive the sine and cosine  rules
\[
\begin{array}{rcl}
\alertNoH{12}{\left(\sin x\right)'}&=&\alertNoH{12}{\cos x}\\
\alertNoH{11}{\left(\cos x\right)'}&=&\alertNoH{11}{-\sin x}\\
\end{array}
\]
\uncover<2->{using \alertNoH{2,8}{Euler's formula},}\uncover<4->{\alertNoH{4,5}{ the exponent derivative rule},} \uncover<4->{\alertNoH{4,5}{the chain rule},} \uncover<6->{\alertNoH{6}{the sum rule}} \uncover<6->{and \alertNoH{6,7}{the constant multiple rule}.} \uncover<5->{\alertNoH{5}{Assume all rules are valid over the complex numbers $\mathbb C$.}}
\[
\begin{array}{rclp{0.5cm}|l}
\uncover<2->{\alertNoH{2,8}{e^{ix}}&=&\alertNoH{2,8}{\cos x+i \sin x }&& \uncover<3->{ \alertNoH{3}{\frac{\diff}{\diff x}}}}\\
\uncover<3->{\alertNoH{4,5}{ \alertNoH{3}{\frac{\diff}{\diff x}} \left(e^{ix} \right)}&=& \alertNoH{6}{\alertNoH{3}{\frac{\diff }{\diff x}}\left( \cos x+i\sin x\right)}}\\
\uncover<4->{\alertNoH{4,5}{e^{ix} \alertNoH{7}{(ix)'}} &=&}\uncover<6-> { \alertNoH{6}{ (\cos x)' +i(\sin x)'}}\\
\uncover<7->{\alertNoH{7}{i} \alertNoH{8}{e^{ix}} &=&(\cos x)'+i(\sin x)'}\\
\uncover<9->{\alertNoH{10}{i^2}\sin x+i\cos x =} \uncover<8->{i(\alertNoH{8}{\cos x+i\sin x})&=&(\cos x)'+i(\sin x)'}\\
\uncover<10->{\alertNoH{11}{ \alertNoH{10}{-}\sin x}+i\alertNoH{12}{\cos x}&=&\alertNoH{11}{(\cos x)'}+i\alertNoH{12}{(\sin x)'}}
\end{array}
\]
\uncover<11->{Compare real part} \uncover<12->{and \alertNoH{12}{coefficients of $i$}} \uncover<11->{ \alertNoH{11,12}{to get the desired equalities}.}
\end{example}
\end{frame}

%end module sin-cos-from-exponent

% end lecture

\end{document}
