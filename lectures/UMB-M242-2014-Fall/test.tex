\documentclass%
%[handout]
{beamer}
% % % % % % % %
% % % % % % % %
% % % % % % % %
%IMPORTANT
%compiles with
%pdflatex -shell-escape
%IMPORTANT
% % % % % % % %
% % % % % % % %
% % % % % % % %
\mode<presentation>
{
\useinnertheme{rounded}
\useoutertheme{infolines}
\usecolortheme{orchid}
\usecolortheme{whale}
}

\usepackage[english]{babel}
\usepackage[latin1]{inputenc}
\usepackage{times}
\usepackage[T1]{fontenc}
\usepackage{../example-templates}
\usepackage{../pstricks-commands}

\usepackage{auto-pst-pdf}
\usepackage{pst-plot}
%\usepackage{pstricks-add}

% Or whatever. Note that the encoding and the font should match. If T1
% does not look nice, try deleting the line with the fontenc.


\graphicspath{{../../modules/}}

\newtheoremstyle{partialproof}{3pt}{3pt}{}{}{}{.}{.5em}{}
\theoremstyle{partialproof} \newtheorem{partialproof}[theorem]{Proof.}
%\DeclareMathOperator{\diff}{d}
\setbeamertemplate{navigation symbols}{}

\includeonlylecture{1}

\newcommand{\lect}[3]{
  \date{#1}
  \lecture[#1]{#2}{#3}
}

\setbeamertemplate{footline}
{
  \leavevmode%
  \hbox{%
  \begin{beamercolorbox}[wd=.333333\paperwidth,ht=2.25ex,dp=1ex,center]{author in head/foot}%
    \usebeamerfont{author in head/foot}\insertshortauthor
  \end{beamercolorbox}%
  \begin{beamercolorbox}[wd=.333333\paperwidth,ht=2.25ex,dp=1ex,center]{title in head/foot}%
    \usebeamerfont{title in head/foot}\insertshorttitle
  \end{beamercolorbox}%
  \begin{beamercolorbox}[wd=.333333\paperwidth,ht=2.25ex,dp=1ex,center]{date in head/foot}%
    \usebeamerfont{date in head/foot}\insertshortdate{}
  \end{beamercolorbox}}%
  \vskip0pt%
}

% If you have a file called "university-logo-filename.xxx", where xxx
% is a graphic format that can be processed by latex or pdflatex,
% resp., then you can add a logo as follows:

%\pgfdeclareimage[height=0.8cm]{logo}{bluelogo}
%\logo{\pgfuseimage{logo}}
\renewcommand{\Arcsin}{\arcsin}
\renewcommand{\Arccos}{\arccos}
\renewcommand{\Arctan}{\arctan}
\renewcommand{\Arccot}{\text{arccot\hspace{0.03cm}}}
\renewcommand{\Arcsec}{\text{arcsec\hspace{0.03cm}}}
\renewcommand{\Arccsc}{\text{arccsc\hspace{0.03cm}}}



\begin{document}

\AtBeginLecture{%

\title[\insertlecture]{FreeCalc}
\subtitle{\insertlecture}
\author[FreeCalc]{}
\institute[UMass Boston]{University of Massachusetts Boston}
\date{\insertshortlecture}
\begin{frame}
  \titlepage
\end{frame}
}%

% begin lecture
\lect{\today}{Sample}{1}{
%\begin{frame}
\frametitle{Double Integral Properties}
\[
\iint_{\mathcal{R}} f(P) \; \diff A = \lim_{\max_k(\text{diam}D_k) \to 0} \sum_k f(P_k) \; \text{area}(D_k)
\]
\begin{itemize}
\item<2-> If $f$ is bounded and continuous, except maybe on a finite number of smooth curves, then the limit exists and is finite.
\item<3-> Linearity 
\[
\iint_{\mathcal{R}} [\lambda f(P) +\mu g(P)] \, \diff A = \lambda \, \iint_{\mathcal{R}} f(P) \, \diff A +\mu \, \iint_{\mathcal{R}} g(P) \, \diff A \; .
\]
\item<4-> Domain additivity: if $\mathcal{R}_1$ and $\mathcal{R}_2$ intersect only along boundaries:
\[
\iint_{\mathcal{R}_1\cup \mathcal{R}_2} f(P)\, \diff A = \iint_{\mathcal{R}_1} f(P)\, \diff A + \iint_{\mathcal{R}_2} f(P)\, \diff A 
\]
\item<5-> Monotonicity property: If $m \leq f(P) \leq M$ for all $P$ in $\mathcal{R}$, then
\[
m \, \text{area}(\mathcal{R}) \leq \iint_{\mathcal{R}} f(P)\, \diff A \leq M \, \text{area} (\mathcal{R})\; .
\]
\end{itemize}
\end{frame} 
%\begin{frame}
\frametitle{Applications}
\begin{itemize}
\item Average value of $f$ on $\mathcal{R}$.
\[
\begin{array}{r@{~}c@{~}l}
\displaystyle \iint _{ \mathcal{R}} f(P) \, \diff A &=&
\displaystyle \iint_{ \mathcal{R}} (\text{average value of }f \text{ on } \mathcal{R}) \, \diff A \\
&=&\displaystyle (\text{average value of }f \text{ on } \mathcal{R})
\iint_{ \mathcal R} \diff A \\
&=&\displaystyle   (\text{average value of }f \text{ on } \mathcal{R})
\cdot \text{area}(\mathcal{R}) \\
\displaystyle \text{average value of }f \text{ on } \mathcal{R} &=&\displaystyle  \frac{1}{ \text{area}( \mathcal{R} )} \iint_{  \mathcal{R}} f(P) \, \diff A\; .
\end{array}
\]
\end{itemize}
\end{frame} 
%\begin{frame}
\begin{theorem}[Mean Value Theorem]
If $f$ is continuous on $\mathcal{R}$, then there exists $P_0$ in $\mathcal{R}$ such that
\[
f(P_0) = \frac{1}{\text{area}(\mathcal{R})} \iint_{\mathcal R} f(Q) \,\diff A
\]
\end{theorem}
\begin{theorem}[Analog of Fundamental Theorem of Calculus]
If $f$ is continuous around $P$, then
\[
\lim_{D \to \{ P \} } \frac{1}{\text{area}(D)} \iint_D f(Q) \diff A = f(P)
\]
\end{theorem}
\end{frame} 
%\begin{frame}
\frametitle{Vectorial Integrals}
The double integral definition extends directly to f-ns with vector output.
\begin{definition}
\[
\iint_{\mathcal{R}} \fcv{F}(P) \; \diff A = \lim_{\text{maxdiam}(\mathcal{D}) \to 0} \sum_k \fcv{F}(P_k) \; \text{area}(D_k)
\]
\end{definition}

\end{frame} 
%\begin{frame}
\frametitle{Theoretical example: Electric force on a lamina}
\begin{itemize}
\item Given: 
\begin{itemize}
\item a charge $Q$, located at the origin;
\item charge $q$, uniformly distributed on a planar lamina $\mathcal{R}$.
\end{itemize}
\item What is the resulting (total) force $\fcv F$ on $Q$?
\item<5-> Recall that the attraction force exerted on a \alert<6>{charge $Q$} located at the origin by \alert<7>{a charge $c$ located at a point} with \alert<6>{position vector $\fcv r$} is $\alert<6>{ \varepsilon Q} \alert<7>{c} \alert<6>{\frac{ \fcv r }{|\fcv r|^3}}$.
\end{itemize}
\[
\renewcommand{\arraystretch}{1.8}
\begin{array}{rcl}
\uncover<2->{\alert<8>{\displaystyle \diff q}} &\uncover<2->{\alert<8>{=}} & \uncover<2->{(\text{density of charge})  \diff A} \uncover<3->{ =\displaystyle \alert<8>{ \frac{q}{A(\mathcal{R})} \diff A} }\\
\displaystyle \uncover<4->{ \alert<10>{\diff \fcv{F}}} &\uncover<4->{\alert<10>{=}}& \displaystyle \uncover<4->{\alert<6>{ \varepsilon Q \frac{ \fcv{r }}{ | \fcv{ r}|^3}} \alert<7,8>{\diff q}} \uncover<8->{ = \alert<10>{\varepsilon \frac{Q \alert<8>{q}}{\alert<8>{A(\mathcal{R})} } \frac{ \fcv{r }}{ | \fcv{ r}|^3} \alert<8>{\diff A}}} \\
\displaystyle \alert<12>{\uncover<9->{\fcv{F}}} &\uncover<9->{\alert<12>{=}} & \displaystyle \uncover<9->{\iint_{\mathcal{R}} \alert<10>{\diff \fcv{F}}} \uncover<10->{= \iint_{\mathcal{R} }  \alert<11>{\varepsilon \alert<10>{ \frac{ Q q}{A(\mathcal{R})} }  \frac{ \fcv{r}}{|\fcv{r}|^3}  \diff A}}\\
\uncover<11,12->{&=&\displaystyle  \alert<12>{ \alert<11>{\varepsilon \frac{Q q}{A(\mathcal{R})}}  \iint_{\mathcal{R} } \frac{\fcv r}{ |\fcv{r}|^3} \diff A}}
\end{array}
\]
\end{frame} 
%\begin{frame}
\frametitle{Iterated Integrals}
\begin{columns}
\column{0.25\textwidth}
\psset{xunit=0.6cm, yunit=0.6cm}
\begin{pspicture}(-2.2,-2.3)(4,4)%
\tiny%
\newcommand{\theFun}{x 2 sub dup mul y 2 sub dup mul add 2 div\space}%
\renewcommand{\fcScreen}{[-1 -1.1 -1] 0}%
\only<1>{\fcRectangularRiemannSum[colorUV=cyan, linecolor=blue]{0}{0}{2}{2} {25}{25}{ \theFun}}%
\fcStartIIIdScene%
\fcAxesIIIdFullInScene{-0.5}{-0.5}{-0.5}{3}{3}{4.5}%
\only<1>{%
\fcSurfaceInScene[arrows=(none), linecolor=cyan, colorUV=cyan, iterationsU=4, iterationsV=4]{0}{0}{2}{2}{[2 dict begin /x u def /y v def x y  \theFun end]}{}%
\fcSurfaceInScene[arrows=(none), linecolor=black, colorUV=cyan, iterationsU=1, iterationsV=1]{0}{0}{2}{1}{[2 dict begin /x 2 def /y u def x y  \theFun v mul end]}{}%
\fcSurfaceInScene[arrows=(none), linecolor=black, colorUV=cyan, iterationsU=1, iterationsV=1]{0}{0}{1}{2}{[2 dict begin /x v def /y 2 def x y  \theFun u mul end]}{}%
}%
\fcFinishIIIdScene[true]%
\only<2->{%
\fcRectangularRiemannSum[colorUV=cyan, linecolor=blue]{0}{0}{2}{0.4}{5}{1}{\theFun}%
\only<4,5,10>{%
\fcRectangularRiemannSum[colorUV=pink, linecolor=red]{0}{0}{2}{0.4}{5}{1}{\theFun}%
}%
\fcRectangularRiemannSum[colorUV=cyan, linecolor=blue]{0}{0.4}{2}{0.8}{5}{1}{\theFun}%
\only<6>{%
\fcRectangularRiemannSum[colorUV=pink, linecolor=red]{0}{0.4}{2}{0.8}{5}{1}{\theFun}%
}%
\fcRectangularRiemannSum[colorUV=cyan, linecolor=blue]{0}{0.8}{2}{1.2}{5}{1}{\theFun}%
\only<7>{%
\fcRectangularRiemannSum[colorUV=pink, linecolor=red]{0}{0.8}{2}{1.2}{5}{1}{\theFun}%
}%
\fcRectangularRiemannSum[colorUV=cyan, linecolor=blue]{0}{1.2}{2}{1.6}{5}{1}{\theFun}%
\only<8>{%
\fcRectangularRiemannSum[colorUV=pink, linecolor=red]{0}{1.2}{2}{1.6}{5}{1}{\theFun}%
}
\fcRectangularRiemannSum[colorUV=cyan, linecolor=blue]{0}{1.6}{2}{2}{5}{1}{\theFun}%
\only<9>{%
\fcRectangularRiemannSum[colorUV=pink, linecolor=red]{0}{1.6}{2}{2}{5}{1}{\theFun}%
}%
\fcPolyLineIIId{[2 0 2] [2 0 0] [2 2 0] [0 2 0] [0 2 2] }%
}%
\fcCurveIIId[arrows=(none), linecolor=black, forceForeground=true]{0}{2}{ [2 dict begin /x t def /y 0 def x y  \theFun end]}%
\fcCurveIIId[arrows=(none), linecolor=black, forceForeground=true]{0}{2}{ [2 dict begin /x t def /y 2 def x y  \theFun end]}%
\fcCurveIIId[arrows=(none), linecolor=black, forceForeground=true]{0}{2}{ [2 dict begin /x 0 def /y t def x y  \theFun end]}%
\fcCurveIIId[arrows=(none), linecolor=black, forceForeground=true]{0}{2}{ [2 dict begin /x 2 def /y t def x y  \theFun end]}%
\end{pspicture}
\column{0.75\textwidth}
\[
\begin{array}{rcl}
\displaystyle \iint\limits_{[a,b] \times [c,d]} f(x,y)  \diff x  \diff y &\uncover<2->{\alertNoH{2}{\approx}}&\displaystyle \uncover<2->{ \alertNoH{2}{ \alertNoH{3}{ \sum_{1 \leq i ,j \leq n}}  f ( x_i,y_j)   \Delta x  \Delta y }} \\
&\uncover<3->{=}&\displaystyle  \uncover<3->{\alertNoH{3,5-9}{ \sum \limits_{ j = 1 }^n } \left( \alertNoH{4,10,11}{ \alertNoH{3}{\sum_{i=1}^n} f(x_i,y_j) \Delta x  } \right) \Delta y}\; .
\end{array}
\]
\end{columns}
\uncover<10->{ The $j^{th}$ summand is a Riemann sum for $\displaystyle \alertNoH{10,11}{ g(y_j) = \int_{x=a}^{x=b} f(x,y_j) \diff x }\quad.$}
\[
\begin{array}{r@{~}c@{~}l}
\uncover<11->{\displaystyle \alertNoH{14}{\sum_{j=1}^n}\left( \alertNoH{11}{\alertNoH{14}{\sum_{i=1}^n} f(x_i,y_j) \alertNoH{14}{\Delta x}} \right) \alertNoH{14}{\Delta y}} &\uncover<11->{\approx}&\displaystyle  \uncover<11->{ \alertNoH{15}{\sum_{j=1}^n} \alertNoH{11}{ g(y_j)} \alertNoH{15}{\Delta y}} \uncover<11->{\alertNoH{11}{\approx \int_{y=c}^{y=d} \alertNoH{16}{ g(y)}  \diff y}}
\\
\uncover<13->{ \displaystyle \alertNoH{14}{ \iint\limits_{[a,b] \times [c,d]}} f(x,y)  \alertNoH{14}{\diff x \diff y }&=& \displaystyle  \alertNoH{15}{\int\limits_{y=c}^{y=d}} g(y) \alertNoH{15}{ \diff y }= \int\limits_{y=c}^{y=d} \left( \alertNoH{16}{\int\limits_{x=a}^{x=b} f(x,y)  \diff x} \right)  \diff y}
\end{array}
\]
\end{frame}
 
%\begin{frame}
\begin{theorem}
If  $f$ is continuous the double integral $\displaystyle \iint _{ [a,b]\times [c,d]} f(x,y)\diff x \diff y$ exists.
\end{theorem}
\uncover<2->{
\begin{theorem}[Fubini's Theorem]
Suppose the double integral of $f$ exists. Then, except at a \alert<4>{set of measure $0$}, the iterated integrals exist and
$\begin{array}{rcl}
\displaystyle 
\iint\limits_{[a,b]\times [c,d]} f(x,y) \; \diff x\diff y &=&\displaystyle \int \limits^{{\color{red}{y=d}}}_{{\color{red}{y=c}}} \left( \int \limits_{{\color{blue}{x=a}}}^{{\color{blue}{x=b}}} f(x,y) \; {\color{blue}{\diff x}} \right)  {\color{red}{\diff y}} \\
&=&\displaystyle \int\limits_{{\color{blue}{x=a }}}^{{\color{blue}{x=b}}} 
\left( \int \limits_{{\color{red}{y=c}}}^{{\color{red}{y=d}}} f(x,y)  {\color{red}{\diff y}} \right) {\color{blue}{\diff x}}\; .
\end{array}$
\end{theorem}
}
\uncover<3->{This theorem allows to integrate non-continuous functions.}
\uncover<4->{The term ``\alert<4>{set of measure 0}'' is too technical to define here; usually studied in the subject(s) ``Real Analysis/Measure Theory''.} 
\end{frame} 
%\begin{frame}
  \frametitle{Example}

Use iterated integrals to compute the following integral:
%
$$\int\!\!\!\int_{[1,2]\times [2,3]}\!\!\!\!\!\! (2x+3y^2) \; dxdy$$
%
\begin{itemize}
  \item \pause For $(x,y)$ in $[1,2]\times [2,3]$, $\textcolor[rgb]{0.98,0.00,0.00}{y}$ takes values \pause between $\textcolor[rgb]{0.98,0.00,0.00}{c=2}$ and $\textcolor[rgb]{0.98,0.00,0.00}{d=3}$.
  \item \pause For a fixed value $\textcolor[rgb]{0.98,0.00,0.00}{y=y_0}$, $\textcolor[rgb]{0.00,0.00,1.00}{x}$ takes values \pause between $\textcolor[rgb]{0.00,0.00,1.00}{a=1}$ and $\textcolor[rgb]{0.00,0.00,1.00}{b=2}$.
\end{itemize}
\pause
Then the double integral can be set up as the following iterated integral:
%
$$\int\!\!\!\int_{[1,2]\times [2,3]}\!\!\!\!\!\! (2x+3y^2) \; dxdy = \int_{\textcolor[rgb]{0.98,0.00,0.00}{y=2}}^{\textcolor[rgb]{0.98,0.00,0.00}{y=3}} \left( \int_{\textcolor[rgb]{0.00,0.00,1.00}{x=1}}^{\textcolor[rgb]{0.00,0.00,1.00}{x=2}} (2x+3y^2) \; \textcolor[rgb]{0.00,0.00,1.00}{dx} \right) \; \textcolor[rgb]{0.98,0.00,0.00}{dy} $$
\pause
Integral with respect to $x$ \pause $\Longrightarrow$ $y$ is a constant:
%
$$g(y) = \int_{x=1}^{x=2} (2x+3y^2) \; dx = \left. (x^2+3y^2x) \right|_{x=1}^{x=2} = (4+6y^2)-(1+3y^2) = 3+3y^2\; .$$
%
\pause Continuing the computation we get\pause
%
$$\int\!\!\!\int_{[1,2]\times [2,3]}\!\!\!\!\!\! (2x+3y^2) \; dxdy = \int_{y=2}^{y=3} (3+3y^2)\; dy = \left. (3y+y^3) \right|_{y=2}^{y=3} = 36-14=22\; .$$
\end{frame} 
\begin{frame}
\frametitle{More General Regions}
What makes iterated integrals work over rectangular regions? \uncover<2->{Slices with respect to one variable are intervals in the other.} \uncover<3->{If variable is $x$:}


\begin{columns}
\column{0.3\textwidth}
\begin{pspicture}(-0.5,-0.5)(3,2.7)
\tiny
\fcAxesStandard{-0.5}{-0.5}{3}{2.7}
\pstVerb{
/f {x 170 mul sin 0.4 mul 0.6 add} def
/g {x 190 mul cos 0.4 mul 2 add x 4 div sub} def
}
\pscustom*[linecolor=\fcColorAreaUnderGraph]{
\psplot{0.2 }{2}{f }
\psplot{2 }{0.2}{g}
}
\multido{\na=4+1}{4}{%
\pstVerb{1 dict begin /x \na\space 4 sub 0.5 mul 0.3 add def}%
\only<\na->{\psline[linecolor=red, linewidth=2pt](! x f)(! x g)}%
\only<\na->{%
\psline[linestyle=dotted](! x f)(! x 0)%
\rput[t](! x -0.1){$\alertNoH{8}{x_{\fcEvalToInt{\na-3}}}$}%
}%
\pstVerb{end}%
}%
\uncover<3>{\rput[t](! 0.3 -0.1){$x_1$}}
\psplot{0.2}{2}{f}
\rput[l](2.1, 2.1){$y=g(x)$}
\psplot{2}{0.2}{g}
\rput[l](2, 0.2){$y=f(x)$}
\end{pspicture}
\column{0.7\textwidth}
\begin{itemize}
\item<3-> fix $x$,
\item<4-> integrate with respect to $y$,
\item<8-> to obtain function that depends only on $x$,
\item<9-> then integrate the so obtained function in $x$.
\end{itemize}
\end{columns}
\uncover<10->{So far used rectangular regions;} \uncover<11->{this also works if \alertNoH{11}{slices are intervals whose endpoints depend continuously} on the location of the slice.}
\begin{itemize}
\item<12-> Regions of type I: vertical slices are segments.
\item<12-> Regions of type II: horizontal slices are segments.
\end{itemize}
\uncover<12->{We call such regions curvilinear trapezoids.}
\end{frame}
 
\begin{frame}
\frametitle{Strategy: Curvilinear Trapezoids (Type I)}
\begin{columns}
\column{0.3\textwidth}
\begin{pspicture}(-0.5,-0.5)(3,2.7)
\tiny
\fcAxesStandard{-0.5}{-0.5}{3}{2.7}
\pstVerb{ 10 dict begin
/f1 {x 170 mul sin 0.4 mul 0.6 add} def
/f2 {x 190 mul cos 0.4 mul 2 add x 4 div sub} def
}
\pscustom*[linecolor=\fcColorAreaUnderGraph]{%
\psplot{0.2 }{2}{f1 }%
\psplot{2 }{0.2}{f2}%
}%
\psplot{0.2}{2}{f1}%
\psplot{2}{0.2}{f2}%
\uncover<7>{%
\psplot[linewidth=2pt, linecolor=red]{0.2}{2}{f1}%
\psplot[linewidth=2pt, linecolor=red]{2}{0.2}{f2}%
}%
\pstVerb{/x 0.2 def}
\psline(! x f1)(! x f2)
\uncover<2>{\psline[linewidth=2pt, linecolor=red](! x f1)(! x f2)}
\uncover<2->{%
\psline[linestyle=dotted](! x f1)(! x 0)%
\rput[lt](! x -0.2 ){$~a$}
}%
\pstVerb{/x 2 def}
\psline(! x f1)(! x f2)
\uncover<3>{\psline[linewidth=2pt, linecolor=red](! x f1)(! x f2)}
\uncover<3->{%
\psline[linestyle=dotted](! x f1)(! x 0)%
\rput[lt](! x -0.2 ){$~b$}
}%
\pstVerb{/x 1 def}
\uncover<4>{\psline[linewidth=2pt, linecolor=red](! x f1)(! x f2)}
\uncover<5->{\psline(! x f1)(! x f2)}
\uncover<4->{%
\psline[linestyle=dotted](! x f1)(! x 0)%
\rput[lt](! x -0.2){$~x$}
}%
\uncover<6,7>{
\psline[linecolor=red, linewidth=2pt](0.2, -0.5)(0.2, 2.7)
\psline[linecolor=red, linewidth=2pt](2, -0.5)(2, 2.7)
}
\uncover<5->{%
\fcFullDot{x}{f1}%
\fcFullDot{x}{f2}%
\rput[lb](! x f1){$~~(x,f(x))$}%
\rput[lt](! x f2){$~~(x,g(x))$}%
}%
\rput[l](2.1, 2.1){$y=g(x)$}
\rput[l](2.1, 0.2){$y=f(x)$}
\pstVerb{end}%
\end{pspicture}
\column{0.7\textwidth}
\begin{itemize}
\item<2-> Identify the \alertNoH{2}{leftmost point(s), with $x$-coordinate $x=a$} and \alertNoH{3}{the rightmost point(s), $x=b$}.
\item<4-> Draw a vertical slice at a value $x$ between $a$ and $b$.
\item<5-> Find the lowest point on that slice, $(x,f(x))$ and the highest point, $(x,g(x))$.
\end{itemize}
\end{columns}
\uncover<6->{
The region is the region bounded by:
\begin{itemize}
\item<6-> the vertical lines $\alertNoH{6}{x=a}$ and $\alertNoH{6}{x=b}$;
\item<7-> the graphs of $\alertNoH{7}{y=f(x)}$ and $\alertNoH{7}{y=g(x)}$, with $f,g \colon [a,b] \to \mathbb{R}$.
 \end{itemize}
}
\uncover<8->{
\[
\mathcal{R} = \{(x,y)  | {\color{blue}{a \leq x \leq b}} ,  {\color{red}{f(x) \leq y \leq g(x)}} \}  .
\]
}
\uncover<9->{
\[
\iint_{\mathcal{R}} f(x,y) \; \diff x\diff y = \int_{{\color{blue}{x=a}}}^{{\color{blue}{x=b}}} \left( \int_{{\color{red}{y=f(x)}}}^{{\color{red}{y=g(x)}}} f({\color{blue}{x}},{\color{red}{y}})  {\color{red}{\diff y}} \right) \; \color{blue}{\diff x}
\]
}
\end{frame}
 
\begin{frame}
  \frametitle{Strategy: Curvilinear Trapezoids (Type II)}
%
\begin{itemize}
  \item Identify the lowest point, $(*,c)$, and the highest point, $(*,d)$.
  \item Draw a generic horizontal slice at some value $y$ between $c$ and $d$.
  \item Find the leftmost point on that slice, $(h_1(y),y)$ and the rightmost point, $(h_2(y),y)$.
\end{itemize}

The region is bounded by:
 \begin{itemize}
   \item horizontal lines $\textcolor[rgb]{0.98,0.00,0.00}{y=c}$ and $\textcolor[rgb]{0.98,0.00,0.00}{y=d}$
   \item graphs of $\textcolor[rgb]{0.00,0.00,1.00}{x=h_1(y)}$ and $\textcolor[rgb]{0.00,0.00,1.00}{x=h_2(y)}$, with  $h_1,h_2 \colon [c,d] \to \mathbb{R}$:
 \end{itemize}
%
$$\mathcal{R} = \{(x,y) \; | \; \textcolor[rgb]{0.98,0.00,0.00}{c \leqslant y \leqslant d}\; , \; \textcolor[rgb]{0.00,0.00,1.00}{h_1(y) \leqslant x \leqslant h_2(y)} \} \; .$$
%
$$\int\!\!\!\int_{\mathcal{R}} f(x,y) \; dxdy = \int_{\textcolor[rgb]{0.98,0.00,0.00}{y=c}}^{\textcolor[rgb]{0.98,0.00,0.00}{y=d}} \left( \int_{\textcolor[rgb]{0.00,0.00,1.00}{x=h_1(y)}}^{\textcolor[rgb]{0.00,0.00,1.00}{x=h_2(y)}} f(\textcolor[rgb]{0.00,0.00,1.00}{x},\textcolor[rgb]{0.98,0.00,0.00}{y}) \; \textcolor[rgb]{0.00,0.00,1.00}{dx}\right) \; \textcolor[rgb]{0.98,0.00,0.00}{dy}$$
\end{frame} 
\begin{frame}
\begin{example}
\begin{columns}
\column{0.4\textwidth}
\psset{xunit=0.6cm, yunit=0.6cm}
\begin{pspicture}(-0.1, -1.2)(4,4.5)%
\tiny%
\fcBoundingBox{-0.1}{-1.2}{4}{4.5}%
\renewcommand{\fcScreen}{[-1 2 -3] 0}%
\only<13->{%
\pscustom*[linecolor=\fcColorAreaUnderGraph]{%
\fcCurveIIId{0}{2}{[t t t mul 0]}%
\fcLineIIId{[2 4 0]}{[2 4 2.5]}%
\fcCurveIIId{2}{0}{[t 2 t mul t t mul 4 t t mul mul add 8 div]}%
}%
\fcLineIIId[linewidth=0.5pt, linecolor=black]{[2 4 0]}{[2 4 2.5]}%
\fcStartIIIdScene%
\fcSurfaceInScene[linewidth=0.5, iterationsU=4, iterationsV=4, colorUV={0.2 0.7 0.9}]{0.00001}{0}{2}{1}{[2 dict begin /x u def /y {v 2 u mul mul 1 v sub u u mul mul add } def x y x x mul y y mul add 0.125 mul end]}{}%
\fcFinishIIIdScene[fastsort=true]%
}%
\only<handout:0|6-12>{%
\pscustom*[linecolor=\fcColorAreaUnderGraph]{%
\fcCurveIIId[linewidth=0.5pt, linecolor=black]{0}{2}{[t t t mul 0]}
\fcCurveIIId[linewidth=0.5pt, linecolor=black, linestyle=dashed]{2}{0}{[t 2 t mul 0]}%
}%
}%
\uncover<5->{\fcCurveIIId[linewidth=0.5pt, linecolor=black]{0}{2}{[t t t mul 0]}}%
\uncover<3->{\fcCurveIIId[linewidth=0.5pt, linecolor=black, linestyle=dashed]{2}{0}{[t 2 t mul 0]}}%
\fcAxesIIId{3}{3}{3}%
\end{pspicture}
\psset{xunit=0.6cm, yunit=0.6cm}
\begin{pspicture}(-0.5,-0.5)(2.6,5.2)
\tiny
\fcBoundingBox{-0.5}{-0.5}{2.6}{5.2}
\fcAxesStandard{-0.5}{-0.5}{2.5}{5}
\fcLabels{2.5}{5}
\uncover<6->{
\pscustom*[linecolor=\fcColorAreaUnderGraph]{%
\psplot{0}{2}{x x mul}%
\psline(2, 4)(0,0)%
}%
}
\uncover<5->{
\psplot{-0.2}{2.2}{x x mul}
\rput[l](1.3, 1){$\alert<2>{y=x^2}$}
}
\uncover<3->{
\rput[r](1.2, 2.4){$\alert<2>{y=2x}$}
\psline(-0.2, -0.4)(2.4, 4.8)
}
\uncover<10->{
\fcFullDot{0}{0}
\rput[tl](0.1, -0.1){$\alert<10>{(0,0)}$}
}
\uncover<12->{
\rput[r](1.9, 4){$\alert<12>{(2,4)}$}
\fcFullDot{2}{4}
}
\uncover<15->{
\psline[linecolor=red, linewidth=1.5pt, arrows=<->](0,0)(2,0)
\fcFullDot{0}{0}
\fcFullDot{2}{0}
}
\uncover<16->{
\fcFullDot{1}{0}
}
\uncover<17->{
\psline[linestyle=dotted](1,1)(1,0)
\psline[linecolor=red, linewidth=1.5pt, arrows=<->](1,1)(1,2)
\fcFullDot{1}{1}
\fcFullDot{1}{2}
}
\end{pspicture}

\column{0.6\textwidth}
Let $\mathcal{R}$ be the region bounded by $y=2x$ and $y=x^2$. Compute
\[
\iint_{\mathcal{R}} \frac{1}{8}\left(x^2+y^2\right) \diff x\diff y
\]
\fcQuestion{2}{Plot $y=2x$.} \fcQuestion{4}{ Plot $y=x^2$.} \uncover<6->{\alert<6>{Identify the region. }} 
\end{columns}
\only<handout:1|1-18>{
\fcQuestion{7}{The two curves intersect when } \fcAnswer{8}{ $
\begin{array}{rcl}x^2&=&2x \\ x(x-2)&=&0\\ x&=& 0 \text{ or } 2.\end{array}
$ 
}

\uncover<9->{The \alert<9,10,11,12>{intersection points are} therefore $\alert<9,10>{(0, \fcAnswer{10}{0} )}$ and $\alert<11,12>{(2,\fcAnswer{12}{4})}$.} \uncover<13->{We can plot the function $\frac{1}{8}\left(x^2+y^2\right)$ as above.} \uncover<14->{Our integral is}
}

$
\begin{array}{r@{~}c@{~}l}
\uncover<14->{\alert<18,19>{\displaystyle \int\limits_{{ \fcQuestion{ 14}{x=} \fcAnswer{15}{0}}}^{{ \fcQuestion{ 14}{ x =}\fcAnswer{15}{2}}}\left( \int\limits_{{ \fcQuestion{16}{ y= }\fcAnswer{17}{x^2 }}}^{{\fcQuestion{ 16}{ y=} \fcAnswer{ 17}{2x}}} \frac{1}{8}\left(\alert<20,21>{x^2+y^2} \right) \diff y \right) \diff x}} \uncover<handout:2| 20->{&=&\displaystyle \frac{1 }{8} \int_{x=0 }^{x=2}  { \left[\fcAnswer{21}{\alert<22,23>{ x^2y +\frac{ y^3}{3}}} \right]}_{ \alert<23>{y=x^2}}^{ \alert<22>{y=2x} }\diff x} \\
\uncover<handout:2| 22->{&=&\displaystyle \frac{1}{8}{\alert<24,25>{ \int}}_{0}^2 \alert<24,25>{\left( \alert<22>{ 2x^3+ \frac{8}{3}x^3} \alert<23>{  -x^4 -\frac{ x^6 }{3}} \right)\diff x}} \\
\uncover<handout:2| 24->{&=& \displaystyle \frac{1}{8} {\left[\fcAnswer{25}{-\frac{ 1}{21} x^{7}-\frac{1}{5} x^{5}+\frac{7}{6} x^{4}}\right]}_{x=0}^{x=2}\\ \uncover<26->{&=&\frac{27}{35}}}
\end{array}
$


\end{example}

\vskip 10 cm

\end{frame} 
%

\begin{frame}
\frametitle{Midpoint Rule}
\small
\begin{columns}
\column{0.3\textwidth}
\begin{pspicture}(-2.2,-1.8)(2.18,3.3)
\tiny
\pstVerb{%
/theRFf {x 2 sub dup mul y 2 sub dup mul add 2 div} def%
}%
\fcBoundingBox{-2.2}{-1.8}{2.18}{3.3}
\renewcommand{\fcScreen}{[-1 -1.1 -1] 0}
\fcAxesIIId{3}{3}{3.3}
\uncover<3-11>{%
\fcLineIIId[linewidth =0.5pt]{[-0.1 0 0]}{[2.1 0 0]}%
\fcLineIIId[linewidth =0.5pt]{[-0.1 1 0]}{[2.1 1 0]}%
\fcLineIIId[linewidth =0.5pt]{[-0.1 2 0]}{[2.1 2 0]}%
\fcLineIIId[linewidth =0.5pt]{[0 -0.1 0]}{[0 2.1 0]}%
\fcLineIIId[linewidth =0.5pt]{[1 -0.1 0]}{[1 2.1 0]}%
\fcLineIIId[linewidth =0.5pt]{[2 -0.1 0]}{[2 2.1 0]}%
}%
\uncover<3->{%
\fcPutIIId[l]{[0 0 0]}{$~~(a,c)$}
\fcPutIIId[lb]{[1.9 2.2 0]}{$~~(b,d)$}
}%
\uncover<9>{%
\pscustom*[linecolor=cyan]{\fcPolyLineIIId{[1 0 0] [2 0 0] [2 1 0] [1 1 0] [1 0 0]}}%
}%
\uncover<5-9>{%
\fcPolyLineIIId[linecolor=cyan]{[1 0 0] [2 0 0] [2 1 0] [1 1 0] [1 0 0]}
\fcDotIIId{[1.5 0.5 0]}%
}%
\uncover<10>{%
\fcBoxIIIdFilledNew[linecolor=blue, colorUV=cyan]{[2 0 0]}{[2 1 0]}{[1 0 0]}{[2 0 2 dict begin /x 1.5 def /y 0.5 def theRFf end]}%
}%
\uncover<3-9>{%
\fcPutIIId[r]{[1.5 -0.1 0]}{\alertNoH{3}{$\Delta x$}}%
}%
\uncover<3-10>{%
\fcPutIIId[bl]{[-0.1 1.5 0]}{\alertNoH{4}{$\Delta y$}}%
}%
\uncover<6>{%
\fcDotIIId{[0.5 0.5 0]}%
\fcDotIIId{[0.5 1.5 0]}%
\fcDotIIId{[1.5 1.5 0]}%
}%
\uncover<7-10>{%
\fcPutIIId[r]{[2 dict begin /x 1.5 def /y 0.5 def x y theRFf end ]}{$f(P_{s,t})~~$}%
}%
\uncover<7-9>{%
\fcLineIIId[linecolor=red]{[1.5 0.5 0]}{[2 dict begin /x 1.5 def /y 0.5 def x y theRFf end]}
\fcDotIIId{[2 dict begin /x 1.5 def /y 0.5 def x y theRFf end]}
}%
\multido{\na=11+1}{10}{%
\only<\na>{%
\fcRectangularRiemannSum[colorUV=cyan, linecolor=blue]{0}{0}{2}{2}{\na \space 10 sub 3 mul 1 sub}{\na\space 10 sub 3 mul 1 sub}{theRFf}
}%
}%
\uncover<21->{%
\fcRectangularRiemannSum[colorUV=cyan, linecolor=blue]{0}{0}{2}{2}{10 3 mul 1 sub}{10 3 mul 1 sub}{theRFf}%
\fcStartIIIdScene%
\fcSurfaceInScene[linecolor=blue, colorUV=cyan, colorVU=cyan, arrows=none]{0}{0}{2}{2}{[2 dict begin /x u def /y v def x y theRFf end]}{}%
\fcFinishIIIdScene%
}
\end{pspicture}
\column{0.7\textwidth}
\begin{itemize}
\item Suppose region of integration $\mathcal R$ is rectangle, i.e., $\mathcal{R} = [a,b] \times [c,d]$, integration w.r.t. $\diff A = \diff x \diff y$.
$\iint_{\mathcal{R}} f(P) \diff A = \iint_{[a,b] \times [c,d]} f(x,y) \; \diff x\, \diff y$.
\item<2-> If integral exists: approximate by fine enough Riemann sum.
\item<3-> Simplest way: divide $\mathcal R$ into $n\times n$ equal pieces, sides $\alertNoH{3}{\Delta x = \frac{b-a}{n}}$, $\alertNoH{4}{\Delta y = \frac{d-c}{n}} $.
\item<5-> For $(s,t)^{th}$ rectangle $D_{st}$, sample at \alertNoH{5}{midpoint} $\alertNoH{5,6}{ P_{s,t}=} \fcAnswer{6}{\left( a+ \left( s - \frac{1}{2} \right) \Delta x, c+\left( t-\frac{1}{2}\right)\Delta y\right)} $.
\end{itemize}
\end{columns}
\uncover<7->{
\[
\begin{array}{r@{~}c@{~}l}
\displaystyle \alertNoH{21}{ \iint\limits_{\mathcal R} f(x,y)  \diff x \diff y }  &=&\displaystyle \alertNoH{12-20}{\lim\limits_{n \to \infty}  \alertNoH{11}{\sum_{1\leq s,t \leq n} \alertNoH{10}{ \alertNoH{8}{ f \left(P_{s,t} \right)} \alertNoH{9,22}{ \text{area}(D_{st})} }} }\\
\uncover<22->{ &\approx&\displaystyle \sum_{1\leq i,j \leq n} f\left(P_{s,t} \right)  \alertNoH{22}{ \Delta x  \Delta y } \quad .}
\end{array}
\]
}
\end{frame}
 

%\begin{frame}
\begin{example}
\begin{columns}
\column{0.3 \textwidth}
\psset{xunit=0.3cm, yunit=0.3cm}
\begin{pspicture}(-4,-4)(4,5.5)%
\tiny%
\renewcommand{\fcScreen}{[-1 -1.1 -3] 0}%
\fcBoundingBox{-4}{-4}{4}{5.5}%
\fcAxesIIId{5}{5}{11}%
\uncover<3->{%
\fcLineIIId[linewidth=0.6pt]{[4 -0.1 0]}{[4 2.1 0]}%
\fcLineIIId[linewidth=0.6pt]{[2 -0.1 0]}{[2 2.1 0]}%
\fcLineIIId[linewidth=0.6pt]{[0 -0.1 0]}{[0 2.1 0]}%
\fcLineIIId[linewidth=0.6pt]{[-0.1 0 0]}{[4.1 0 0]}%
\fcLineIIId[linewidth=0.6pt]{[-0.1 1 0]}{[4.1 1 0]}%
\fcLineIIId[linewidth=0.6pt]{[-0.1 2 0]}{[4.1 2 0]}%
}%
\uncover<5->{%
\fcDotIIId{[1 0.5 0]}%
\fcDotIIId{[1 1.5 0]}%
\fcDotIIId{[3 0.5 0]}%
\fcDotIIId{[3 1.5 0]}%
}%
\uncover<7->{%
\pstVerb{%
0.5 setalpha%
}%
\fcRectangularRiemannSum[colorUV=cyan, linecolor=blue]{0}{0}{4}{2}{2}{2}{x x mul y add}}%
\end{pspicture}
\column{0.7\textwidth}
Use the Midpoint Rule to approximate $\displaystyle \iint_{[0,4]\times [0,2]} x^2y \diff x\diff y,$ with each side divided into $n=2$ pieces.

\uncover<2->{\alertNoH{2,3}{The small rectangles have dimensions}} \fcAnswer{3}{ $\frac{4-0}{2} \cdot \frac{2-0}{2} = 2\cdot 1$ and area $2$.} \uncover<4->{\alertNoH{4,5}{The midpoints are}}
\end{columns}
\fcAnswer{5}{$\displaystyle
P_{11} = \left(1,\frac{1}{2}\right), \quad P_{12} = \left(1,\frac{3}{2}\right), \quad P_{21} = \left(3,\frac{1}{2}\right),  \quad P_{22} = \left(3,\frac{3}{2}\right)\; .
$}
\[
\begin{array}{r@{~}c@{~}l}
\displaystyle \fcQuestion{6}{\iint\limits_{[0,4]\times [0,2]} x^2y\; \diff x \diff y}  &\fcQuestion{6}{\approx} &  \displaystyle \fcAnswer{7}{ 2\left(f \left(1,\frac{1}{2 }\right)  +  f\left(3,\frac{1}{2}\right) + f\left(1,\frac{3}{2}\right)   + f\left(3,\frac{3}{2}\right) \right)} \\
\uncover<8->{& =&\displaystyle   1\cdot \frac{1}{2}\cdot 2 + 9 \cdot \frac{1}{2} \cdot 2 + 1\cdot \frac{3}{2} \cdot 2 + 9 \cdot \frac{3}{2} \cdot 2} \\
\uncover<9->{&=&\displaystyle  1+9+3+27 = 40\; .}
\end{array}
\]

\end{example}
\end{frame}
 

%\begin{frame}
\frametitle{Double Integral Properties}
\[
\iint_{\mathcal{R}} f(P) \; \diff A = \lim_{\max_k(\text{diam}D_k) \to 0} \sum_k f(P_k) \; \text{area}(D_k)
\]
\begin{itemize}
\item<2-> If $f$ is bounded and continuous, except maybe on a finite number of smooth curves, then the limit exists and is finite.
\item<3-> Linearity 
\[
\iint_{\mathcal{R}} [\lambda f(P) +\mu g(P)] \, \diff A = \lambda \, \iint_{\mathcal{R}} f(P) \, \diff A +\mu \, \iint_{\mathcal{R}} g(P) \, \diff A \; .
\]
\item<4-> Domain additivity: if $\mathcal{R}_1$ and $\mathcal{R}_2$ intersect only along boundaries:
\[
\iint_{\mathcal{R}_1\cup \mathcal{R}_2} f(P)\, \diff A = \iint_{\mathcal{R}_1} f(P)\, \diff A + \iint_{\mathcal{R}_2} f(P)\, \diff A 
\]
\item<5-> Monotonicity property: If $m \leq f(P) \leq M$ for all $P$ in $\mathcal{R}$, then
\[
m \, \text{area}(\mathcal{R}) \leq \iint_{\mathcal{R}} f(P)\, \diff A \leq M \, \text{area} (\mathcal{R})\; .
\]
\end{itemize}
\end{frame} 
%\begin{frame}
\frametitle{Applications}
\begin{itemize}
\item Average value of $f$ on $\mathcal{R}$.
\[
\begin{array}{r@{~}c@{~}l}
\displaystyle \iint _{ \mathcal{R}} f(P) \, \diff A &=&
\displaystyle \iint_{ \mathcal{R}} (\text{average value of }f \text{ on } \mathcal{R}) \, \diff A \\
&=&\displaystyle (\text{average value of }f \text{ on } \mathcal{R})
\iint_{ \mathcal R} \diff A \\
&=&\displaystyle   (\text{average value of }f \text{ on } \mathcal{R})
\cdot \text{area}(\mathcal{R}) \\
\displaystyle \text{average value of }f \text{ on } \mathcal{R} &=&\displaystyle  \frac{1}{ \text{area}( \mathcal{R} )} \iint_{  \mathcal{R}} f(P) \, \diff A\; .
\end{array}
\]
\end{itemize}
\end{frame} 
%\begin{frame}
\begin{theorem}[Mean Value Theorem]
If $f$ is continuous on $\mathcal{R}$, then there exists $P_0$ in $\mathcal{R}$ such that
\[
f(P_0) = \frac{1}{\text{area}(\mathcal{R})} \iint_{\mathcal R} f(Q) \,\diff A
\]
\end{theorem}
\begin{theorem}[Analog of Fundamental Theorem of Calculus]
If $f$ is continuous around $P$, then
\[
\lim_{D \to \{ P \} } \frac{1}{\text{area}(D)} \iint_D f(Q) \diff A = f(P)
\]
\end{theorem}
\end{frame} 
%\begin{frame}
\frametitle{Vectorial Integrals}
The double integral definition extends directly to f-ns with vector output.
\begin{definition}
\[
\iint_{\mathcal{R}} \fcv{F}(P) \; \diff A = \lim_{\text{maxdiam}(\mathcal{D}) \to 0} \sum_k \fcv{F}(P_k) \; \text{area}(D_k)
\]
\end{definition}

\end{frame} 
%\begin{frame}
\frametitle{Iterated Integrals}
\begin{columns}
\column{0.25\textwidth}
\psset{xunit=0.6cm, yunit=0.6cm}
\begin{pspicture}(-2.2,-2.3)(4,4)%
\tiny%
\newcommand{\theFun}{x 2 sub dup mul y 2 sub dup mul add 2 div\space}%
\renewcommand{\fcScreen}{[-1 -1.1 -1] 0}%
\only<1>{\fcRectangularRiemannSum[colorUV=cyan, linecolor=blue]{0}{0}{2}{2} {25}{25}{ \theFun}}%
\fcStartIIIdScene%
\fcAxesIIIdFullInScene{-0.5}{-0.5}{-0.5}{3}{3}{4.5}%
\only<1>{%
\fcSurfaceInScene[arrows=(none), linecolor=cyan, colorUV=cyan, iterationsU=4, iterationsV=4]{0}{0}{2}{2}{[2 dict begin /x u def /y v def x y  \theFun end]}{}%
\fcSurfaceInScene[arrows=(none), linecolor=black, colorUV=cyan, iterationsU=1, iterationsV=1]{0}{0}{2}{1}{[2 dict begin /x 2 def /y u def x y  \theFun v mul end]}{}%
\fcSurfaceInScene[arrows=(none), linecolor=black, colorUV=cyan, iterationsU=1, iterationsV=1]{0}{0}{1}{2}{[2 dict begin /x v def /y 2 def x y  \theFun u mul end]}{}%
}%
\fcFinishIIIdScene[true]%
\only<2->{%
\fcRectangularRiemannSum[colorUV=cyan, linecolor=blue]{0}{0}{2}{0.4}{5}{1}{\theFun}%
\only<4,5,10>{%
\fcRectangularRiemannSum[colorUV=pink, linecolor=red]{0}{0}{2}{0.4}{5}{1}{\theFun}%
}%
\fcRectangularRiemannSum[colorUV=cyan, linecolor=blue]{0}{0.4}{2}{0.8}{5}{1}{\theFun}%
\only<6>{%
\fcRectangularRiemannSum[colorUV=pink, linecolor=red]{0}{0.4}{2}{0.8}{5}{1}{\theFun}%
}%
\fcRectangularRiemannSum[colorUV=cyan, linecolor=blue]{0}{0.8}{2}{1.2}{5}{1}{\theFun}%
\only<7>{%
\fcRectangularRiemannSum[colorUV=pink, linecolor=red]{0}{0.8}{2}{1.2}{5}{1}{\theFun}%
}%
\fcRectangularRiemannSum[colorUV=cyan, linecolor=blue]{0}{1.2}{2}{1.6}{5}{1}{\theFun}%
\only<8>{%
\fcRectangularRiemannSum[colorUV=pink, linecolor=red]{0}{1.2}{2}{1.6}{5}{1}{\theFun}%
}
\fcRectangularRiemannSum[colorUV=cyan, linecolor=blue]{0}{1.6}{2}{2}{5}{1}{\theFun}%
\only<9>{%
\fcRectangularRiemannSum[colorUV=pink, linecolor=red]{0}{1.6}{2}{2}{5}{1}{\theFun}%
}%
\fcPolyLineIIId{[2 0 2] [2 0 0] [2 2 0] [0 2 0] [0 2 2] }%
}%
\fcCurveIIId[arrows=(none), linecolor=black, forceForeground=true]{0}{2}{ [2 dict begin /x t def /y 0 def x y  \theFun end]}%
\fcCurveIIId[arrows=(none), linecolor=black, forceForeground=true]{0}{2}{ [2 dict begin /x t def /y 2 def x y  \theFun end]}%
\fcCurveIIId[arrows=(none), linecolor=black, forceForeground=true]{0}{2}{ [2 dict begin /x 0 def /y t def x y  \theFun end]}%
\fcCurveIIId[arrows=(none), linecolor=black, forceForeground=true]{0}{2}{ [2 dict begin /x 2 def /y t def x y  \theFun end]}%
\end{pspicture}
\column{0.75\textwidth}
\[
\begin{array}{rcl}
\displaystyle \iint\limits_{[a,b] \times [c,d]} f(x,y)  \diff x  \diff y &\uncover<2->{\alertNoH{2}{\approx}}&\displaystyle \uncover<2->{ \alertNoH{2}{ \alertNoH{3}{ \sum_{1 \leq i ,j \leq n}}  f ( x_i,y_j)   \Delta x  \Delta y }} \\
&\uncover<3->{=}&\displaystyle  \uncover<3->{\alertNoH{3,5-9}{ \sum \limits_{ j = 1 }^n } \left( \alertNoH{4,10,11}{ \alertNoH{3}{\sum_{i=1}^n} f(x_i,y_j) \Delta x  } \right) \Delta y}\; .
\end{array}
\]
\end{columns}
\uncover<10->{ The $j^{th}$ summand is a Riemann sum for $\displaystyle \alertNoH{10,11}{ g(y_j) = \int_{x=a}^{x=b} f(x,y_j) \diff x }\quad.$}
\[
\begin{array}{r@{~}c@{~}l}
\uncover<11->{\displaystyle \alertNoH{14}{\sum_{j=1}^n}\left( \alertNoH{11}{\alertNoH{14}{\sum_{i=1}^n} f(x_i,y_j) \alertNoH{14}{\Delta x}} \right) \alertNoH{14}{\Delta y}} &\uncover<11->{\approx}&\displaystyle  \uncover<11->{ \alertNoH{15}{\sum_{j=1}^n} \alertNoH{11}{ g(y_j)} \alertNoH{15}{\Delta y}} \uncover<11->{\alertNoH{11}{\approx \int_{y=c}^{y=d} \alertNoH{16}{ g(y)}  \diff y}}
\\
\uncover<13->{ \displaystyle \alertNoH{14}{ \iint\limits_{[a,b] \times [c,d]}} f(x,y)  \alertNoH{14}{\diff x \diff y }&=& \displaystyle  \alertNoH{15}{\int\limits_{y=c}^{y=d}} g(y) \alertNoH{15}{ \diff y }= \int\limits_{y=c}^{y=d} \left( \alertNoH{16}{\int\limits_{x=a}^{x=b} f(x,y)  \diff x} \right)  \diff y}
\end{array}
\]
\end{frame}
 
%\begin{frame}
\begin{theorem}
If  $f$ is continuous the double integral $\displaystyle \iint _{ [a,b]\times [c,d]} f(x,y)\diff x \diff y$ exists.
\end{theorem}
\uncover<2->{
\begin{theorem}[Fubini's Theorem]
Suppose the double integral of $f$ exists. Then, except at a \alert<4>{set of measure $0$}, the iterated integrals exist and
$\begin{array}{rcl}
\displaystyle 
\iint\limits_{[a,b]\times [c,d]} f(x,y) \; \diff x\diff y &=&\displaystyle \int \limits^{{\color{red}{y=d}}}_{{\color{red}{y=c}}} \left( \int \limits_{{\color{blue}{x=a}}}^{{\color{blue}{x=b}}} f(x,y) \; {\color{blue}{\diff x}} \right)  {\color{red}{\diff y}} \\
&=&\displaystyle \int\limits_{{\color{blue}{x=a }}}^{{\color{blue}{x=b}}} 
\left( \int \limits_{{\color{red}{y=c}}}^{{\color{red}{y=d}}} f(x,y)  {\color{red}{\diff y}} \right) {\color{blue}{\diff x}}\; .
\end{array}$
\end{theorem}
}
\uncover<3->{This theorem allows to integrate non-continuous functions.}
\uncover<4->{The term ``\alert<4>{set of measure 0}'' is too technical to define here; usually studied in the subject(s) ``Real Analysis/Measure Theory''.} 
\end{frame} 
%\begin{frame}
  \frametitle{Example}

Use iterated integrals to compute the following integral:
%
$$\int\!\!\!\int_{[1,2]\times [2,3]}\!\!\!\!\!\! (2x+3y^2) \; dxdy$$
%
\begin{itemize}
  \item \pause For $(x,y)$ in $[1,2]\times [2,3]$, $\textcolor[rgb]{0.98,0.00,0.00}{y}$ takes values \pause between $\textcolor[rgb]{0.98,0.00,0.00}{c=2}$ and $\textcolor[rgb]{0.98,0.00,0.00}{d=3}$.
  \item \pause For a fixed value $\textcolor[rgb]{0.98,0.00,0.00}{y=y_0}$, $\textcolor[rgb]{0.00,0.00,1.00}{x}$ takes values \pause between $\textcolor[rgb]{0.00,0.00,1.00}{a=1}$ and $\textcolor[rgb]{0.00,0.00,1.00}{b=2}$.
\end{itemize}
\pause
Then the double integral can be set up as the following iterated integral:
%
$$\int\!\!\!\int_{[1,2]\times [2,3]}\!\!\!\!\!\! (2x+3y^2) \; dxdy = \int_{\textcolor[rgb]{0.98,0.00,0.00}{y=2}}^{\textcolor[rgb]{0.98,0.00,0.00}{y=3}} \left( \int_{\textcolor[rgb]{0.00,0.00,1.00}{x=1}}^{\textcolor[rgb]{0.00,0.00,1.00}{x=2}} (2x+3y^2) \; \textcolor[rgb]{0.00,0.00,1.00}{dx} \right) \; \textcolor[rgb]{0.98,0.00,0.00}{dy} $$
\pause
Integral with respect to $x$ \pause $\Longrightarrow$ $y$ is a constant:
%
$$g(y) = \int_{x=1}^{x=2} (2x+3y^2) \; dx = \left. (x^2+3y^2x) \right|_{x=1}^{x=2} = (4+6y^2)-(1+3y^2) = 3+3y^2\; .$$
%
\pause Continuing the computation we get\pause
%
$$\int\!\!\!\int_{[1,2]\times [2,3]}\!\!\!\!\!\! (2x+3y^2) \; dxdy = \int_{y=2}^{y=3} (3+3y^2)\; dy = \left. (3y+y^3) \right|_{y=2}^{y=3} = 36-14=22\; .$$
\end{frame} 
%\begin{frame}
\frametitle{More General Regions}
What makes iterated integrals work over rectangular regions? \uncover<2->{Slices with respect to one variable are intervals in the other.} \uncover<3->{If variable is $x$:}


\begin{columns}
\column{0.3\textwidth}
\begin{pspicture}(-0.5,-0.5)(3,2.7)
\tiny
\fcAxesStandard{-0.5}{-0.5}{3}{2.7}
\pstVerb{
/f {x 170 mul sin 0.4 mul 0.6 add} def
/g {x 190 mul cos 0.4 mul 2 add x 4 div sub} def
}
\pscustom*[linecolor=\fcColorAreaUnderGraph]{
\psplot{0.2 }{2}{f }
\psplot{2 }{0.2}{g}
}
\multido{\na=4+1}{4}{%
\pstVerb{1 dict begin /x \na\space 4 sub 0.5 mul 0.3 add def}%
\only<\na->{\psline[linecolor=red, linewidth=2pt](! x f)(! x g)}%
\only<\na->{%
\psline[linestyle=dotted](! x f)(! x 0)%
\rput[t](! x -0.1){$\alertNoH{8}{x_{\fcEvalToInt{\na-3}}}$}%
}%
\pstVerb{end}%
}%
\uncover<3>{\rput[t](! 0.3 -0.1){$x_1$}}
\psplot{0.2}{2}{f}
\rput[l](2.1, 2.1){$y=g(x)$}
\psplot{2}{0.2}{g}
\rput[l](2, 0.2){$y=f(x)$}
\end{pspicture}
\column{0.7\textwidth}
\begin{itemize}
\item<3-> fix $x$,
\item<4-> integrate with respect to $y$,
\item<8-> to obtain function that depends only on $x$,
\item<9-> then integrate the so obtained function in $x$.
\end{itemize}
\end{columns}
\uncover<10->{So far used rectangular regions;} \uncover<11->{this also works if \alertNoH{11}{slices are intervals whose endpoints depend continuously} on the location of the slice.}
\begin{itemize}
\item<12-> Regions of type I: vertical slices are segments.
\item<12-> Regions of type II: horizontal slices are segments.
\end{itemize}
\uncover<12->{We call such regions curvilinear trapezoids.}
\end{frame}
 
}
\end{document}
