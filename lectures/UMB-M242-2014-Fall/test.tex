\documentclass%
%[handout]
{beamer}
% % % % % % % %
% % % % % % % %
% % % % % % % %
%IMPORTANT
%compiles with
%pdflatex -shell-escape
%IMPORTANT
% % % % % % % %
% % % % % % % %
% % % % % % % %
\mode<presentation>
{
\useinnertheme{rounded}
\useoutertheme{infolines}
\usecolortheme{orchid}
\usecolortheme{whale}
}

\usepackage[english]{babel}
\usepackage[latin1]{inputenc}
\usepackage{times}
\usepackage[T1]{fontenc}
\usepackage{../example-templates}
\usepackage{../pstricks-commands}

\usepackage{auto-pst-pdf}
\usepackage{pst-plot}
%\usepackage{pstricks-add}

% Or whatever. Note that the encoding and the font should match. If T1
% does not look nice, try deleting the line with the fontenc.


\graphicspath{{../../modules/}}

\newtheoremstyle{partialproof}{3pt}{3pt}{}{}{}{.}{.5em}{}
\theoremstyle{partialproof} \newtheorem{partialproof}[theorem]{Proof.}
%\DeclareMathOperator{\diff}{d}
\setbeamertemplate{navigation symbols}{}

\includeonlylecture{1}

\newcommand{\lect}[3]{
  \date{#1}
  \lecture[#1]{#2}{#3}
}

\setbeamertemplate{footline}
{
  \leavevmode%
  \hbox{%
  \begin{beamercolorbox}[wd=.333333\paperwidth,ht=2.25ex,dp=1ex,center]{author in head/foot}%
    \usebeamerfont{author in head/foot}\insertshortauthor
  \end{beamercolorbox}%
  \begin{beamercolorbox}[wd=.333333\paperwidth,ht=2.25ex,dp=1ex,center]{title in head/foot}%
    \usebeamerfont{title in head/foot}\insertshorttitle
  \end{beamercolorbox}%
  \begin{beamercolorbox}[wd=.333333\paperwidth,ht=2.25ex,dp=1ex,center]{date in head/foot}%
    \usebeamerfont{date in head/foot}\insertshortdate{}
  \end{beamercolorbox}}%
  \vskip0pt%
}

% If you have a file called "university-logo-filename.xxx", where xxx
% is a graphic format that can be processed by latex or pdflatex,
% resp., then you can add a logo as follows:

%\pgfdeclareimage[height=0.8cm]{logo}{bluelogo}
%\logo{\pgfuseimage{logo}}
\renewcommand{\Arcsin}{\arcsin}
\renewcommand{\Arccos}{\arccos}
\renewcommand{\Arctan}{\arctan}
\renewcommand{\Arccot}{\text{arccot\hspace{0.03cm}}}
\renewcommand{\Arcsec}{\text{arcsec\hspace{0.03cm}}}
\renewcommand{\Arccsc}{\text{arccsc\hspace{0.03cm}}}



\begin{document}

\AtBeginLecture{%

\title[\insertlecture]{FreeCalc}
\subtitle{\insertlecture}
\author[FreeCalc]{}
\institute[UMass Boston]{University of Massachusetts Boston}
\date{\insertshortlecture}
\begin{frame}
  \titlepage
\end{frame}
}%

% begin lecture
\lect{\today}{Sample}{1}{
%\begin{frame}
\frametitle{A Cheaper ``Census''}
Imagine we want cheap procedure to estimate population in region $\mathcal{R}$.
\begin{itemize}
\item<2-> We decompose $\mathcal{R}$ into pairwise non-overlapping smaller regions $D_k$ (states, counties, finer division...).
\uncover<3->{%
\[
\text{population}(\mathcal{R}) = \sum_k \text{population}(D_k) = \sum_k \text{density}(D_k) \cdot \text{area}(D_k)
\]
}
\item<4-> To find the population density in $D_k$ we need to count everyone (what an actual census does).
\item<5-> Instead, we estimate the population density as follows.
\begin{itemize}
\item<6-> We pick a sample point $P_k$ in each region $D_k$.
\item<7-> We estimate the population density $\text{density}(D_k)$ by counting people in a small region around $P_k$ ($\text{density\_near}(P_k)$).
\end{itemize}
\item<8-> Our population estimate becomes
\[
\text{population}(\mathcal{R}) = \sum \text{pop.}(D_k) \simeq \sum \text{density\_near}(P_k) \text{area}(D_k).
\]
\end{itemize}
\end{frame}
%\begin{frame}
\frametitle{Riemann sum in two variables}
Let $\mathcal{R}$ be a compact (closed, bounded) region in the plane, and let $f \colon \mathcal{R} \to \mathbb{R}$ be a function on $\mathcal{R}$.
\uncover<2->{Let $\{D_k\}$ be finite set of regions covering $\mathcal{R}$} with the following properties.
\begin{itemize}
\item Each $D_k$ is a compact set.
\item The boundary of each $D_k$ is a collection of smooth curves.
\item Two regions $D_i$ and $D_j$ may overlap only on their boundaries.
\end{itemize}
Let $\mathcal{P} = (P_k)_k$ be a collection of sample points, with $P_k \in D_k$ for all $k$.

\begin{definition}[Riemann sum]
The \emph{Riemann sum} defined by such data is
$
%\Sigma_{f,\mathcal{D}, \mathcal{P}} = 
\displaystyle\sum\limits_k f(P_k) \; \text{area}(D_k)\; .
$
\end{definition}
\end{frame} 
\begin{frame}\frametitle{Riemann sum over rectangular regions}
\begin{pspicture}(-1,-1)(1,1)
\fcAxesIIId{2}{2}{2}
\fcRectangularRiemannSum{0}{0}{1}{1}{10}{10}{x}
\end{pspicture}
\end{frame}
 

%\begin{frame}\frametitle{Double Integrals}
$\mathcal R$-region covered by $D_k$, $D_k$ don't overlap except at boundaries.
\begin{definition}[Riemann sum]
The \emph{Riemann sum} defined by such data is
$
%\Sigma_{f,\mathcal{D}, \mathcal{P}} = 
\displaystyle\sum\limits_k f(P_k) \; \text{area}(D_k)\; .
$
\end{definition}

\begin{definition}

If the limit
\[
\lim\limits_{\max_k(\text{diam} D_k) \to 0} \sum_k f(P_k) \; \text{area}(D_k)
\]
exists and is finite, then its value is called the \emph{double integral of $f$ over $\mathcal{R}$ (with respect to area)}, and is denoted by
\[
\iint_{\mathcal{R}} f(P) \diff A\quad  .
\]
\end{definition}

\end{frame} 
%\begin{frame}\frametitle{Double Integrals}
$\mathcal R$-region covered by $D_k$, $D_k$ don't overlap except at boundaries.
\begin{definition}[Riemann sum]
The \emph{Riemann sum} defined by such data is
$
%\Sigma_{f,\mathcal{D}, \mathcal{P}} = 
\displaystyle\sum\limits_k f(P_k) \; \text{area}(D_k)\; .
$
\end{definition}

\begin{definition}

If the limit
\[
\lim\limits_{\max_k(\text{diam} D_k) \to 0} \sum_k f(P_k) \; \text{area}(D_k)
\]
exists and is finite, then its value is called the \emph{double integral of $f$ over $\mathcal{R}$ (with respect to area)}, and is denoted by
\[
\iint_{\mathcal{R}} f(P) \diff A\quad  .
\]
\end{definition}

\end{frame} 
%\begin{frame}
\frametitle{Examples}
\begin{itemize}
\item The total population over a region $\mathcal{R}$ is:
\[
\text{population}(\mathcal{R}) = \iint_{\mathcal{R}} \text{density}(P) \, \diff A \simeq \sum_k \text{density}(P_k) \, \text{area}(D_k) \; .
\]
\item<2-> Mass is the double integral of density with respect to area:
\[
\text{mass}(\mathcal{R}) = \iint_{\mathcal{R}} \text{density}(P) \, \diff A\; .
\]
\item<3-> Volume under the graph of $h\colon \mathcal{R} \to [0,\infty)$
\[
\text{Volume} = \iint_{\mathcal{R}} h(P) \, \diff A \; .
\]
\item<4-> Area of a region:
\[
\text{Area}(\mathcal{R}) = \iint_{\mathcal{R}} 1 \; \diff A\quad .
\]
\end{itemize}
\end{frame} 
%\begin{frame}
\frametitle{Double Integral Properties}
\[
\iint_{\mathcal{R}} f(P) \; \diff A = \lim_{\max_k(\text{diam}D_k) \to 0} \sum_k f(P_k) \; \text{area}(D_k)
\]
\begin{itemize}
\item<2-> If $f$ is bounded and continuous, except maybe on a finite number of smooth curves, then the limit exists and is finite.
\item<3-> Linearity 
\[
\iint_{\mathcal{R}} [\lambda f(P) +\mu g(P)] \, \diff A = \lambda \, \iint_{\mathcal{R}} f(P) \, \diff A +\mu \, \iint_{\mathcal{R}} g(P) \, \diff A \; .
\]
\item<4-> Domain additivity: if $\mathcal{R}_1$ and $\mathcal{R}_2$ intersect only along boundaries:
\[
\iint_{\mathcal{R}_1\cup \mathcal{R}_2} f(P)\, \diff A = \iint_{\mathcal{R}_1} f(P)\, \diff A + \iint_{\mathcal{R}_2} f(P)\, \diff A 
\]
\item<5-> Monotonicity property: If $m \leq f(P) \leq M$ for all $P$ in $\mathcal{R}$, then
\[
m \, \text{area}(\mathcal{R}) \leq \iint_{\mathcal{R}} f(P)\, \diff A \leq M \, \text{area} (\mathcal{R})\; .
\]
\end{itemize}
\end{frame} 

%\begin{frame}
\frametitle{Applications}
\begin{itemize}
\item Average value of $f$ on $\mathcal{R}$.
\[
\begin{array}{r@{~}c@{~}l}
\displaystyle \iint _{ \mathcal{R}} f(P) \, \diff A &=&
\displaystyle \iint_{ \mathcal{R}} (\text{average value of }f \text{ on } \mathcal{R}) \, \diff A \\
&=&\displaystyle (\text{average value of }f \text{ on } \mathcal{R})
\iint_{ \mathcal R} \diff A \\
&=&\displaystyle   (\text{average value of }f \text{ on } \mathcal{R})
\cdot \text{area}(\mathcal{R}) \\
\displaystyle \text{average value of }f \text{ on } \mathcal{R} &=&\displaystyle  \frac{1}{ \text{area}( \mathcal{R} )} \iint_{  \mathcal{R}} f(P) \, \diff A\; .
\end{array}
\]
\end{itemize}
\end{frame} 
%\begin{frame}
\begin{theorem}[Mean Value Theorem]
If $f$ is continuous on $\mathcal{R}$, then there exists $P_0$ in $\mathcal{R}$ such that
\[
f(P_0) = \frac{1}{\text{area}(\mathcal{R})} \iint_{\mathcal R} f(Q) \,\diff A
\]
\end{theorem}
\begin{theorem}[Analog of Fundamental Theorem of Calculus]
If $f$ is continuous around $P$, then
\[
\lim_{D \to \{ P \} } \frac{1}{\text{area}(D)} \iint_D f(Q) \diff A = f(P)
\]
\end{theorem}
\end{frame} 

\begin{frame}
\frametitle{Vectorial Integrals}
The double integral definition extends directly to f-ns with vector output.
\begin{definition}
\[
\iint_{\mathcal{R}} \fcv{F}(P) \; \diff A = \lim_{\text{maxdiam}(\mathcal{D}) \to 0} \sum_k \fcv{F}(P_k) \; \text{area}(D_k)
\]
\end{definition}

\end{frame} 


\begin{frame}
\frametitle{Midpoint Rule}
\small
\begin{columns}
\column{0.3\textwidth}
\begin{pspicture}(-2.2,-1.8)(2.18,3.3)
\tiny
\pstVerb{%
/theRFf {x 2 sub dup mul y 2 sub dup mul add 2 div} def%
}%
\fcBoundingBox{-2.2}{-1.8}{2.18}{3.3}
\renewcommand{\fcScreen}{[-1 -1.1 -1] 0}
\fcAxesIIId{3}{3}{3.3}
\uncover<3-11>{%
\fcLineIIId[linewidth =0.5pt]{[-0.1 0 0]}{[2.1 0 0]}%
\fcLineIIId[linewidth =0.5pt]{[-0.1 1 0]}{[2.1 1 0]}%
\fcLineIIId[linewidth =0.5pt]{[-0.1 2 0]}{[2.1 2 0]}%
\fcLineIIId[linewidth =0.5pt]{[0 -0.1 0]}{[0 2.1 0]}%
\fcLineIIId[linewidth =0.5pt]{[1 -0.1 0]}{[1 2.1 0]}%
\fcLineIIId[linewidth =0.5pt]{[2 -0.1 0]}{[2 2.1 0]}%
}%
\uncover<3->{%
\fcPutIIId[l]{[0 0 0]}{$~~(a,c)$}
\fcPutIIId[lb]{[1.9 2.2 0]}{$~~(b,d)$}
}%
\uncover<9>{%
\pscustom*[linecolor=cyan]{\fcPolyLineIIId{[1 0 0] [2 0 0] [2 1 0] [1 1 0] [1 0 0]}}%
}%
\uncover<5-9>{%
\fcPolyLineIIId[linecolor=cyan]{[1 0 0] [2 0 0] [2 1 0] [1 1 0] [1 0 0]}
\fcDotIIId{[1.5 0.5 0]}%
}%
\uncover<10>{%
\fcBoxIIIdFilledNew[linecolor=blue, colorUV=cyan]{[2 0 0]}{[2 1 0]}{[1 0 0]}{[2 0 2 dict begin /x 1.5 def /y 0.5 def theRFf end]}%
}%
\uncover<3-9>{%
\fcPutIIId[r]{[1.5 -0.1 0]}{\alertNoH{3}{$\Delta x$}}%
}%
\uncover<3-10>{%
\fcPutIIId[bl]{[-0.1 1.5 0]}{\alertNoH{4}{$\Delta y$}}%
}%
\uncover<6>{%
\fcDotIIId{[0.5 0.5 0]}%
\fcDotIIId{[0.5 1.5 0]}%
\fcDotIIId{[1.5 1.5 0]}%
}%
\uncover<7-10>{%
\fcPutIIId[r]{[2 dict begin /x 1.5 def /y 0.5 def x y theRFf end ]}{$f(P_{s,t})~~$}%
}%
\uncover<7-9>{%
\fcLineIIId[linecolor=red]{[1.5 0.5 0]}{[2 dict begin /x 1.5 def /y 0.5 def x y theRFf end]}
\fcDotIIId{[2 dict begin /x 1.5 def /y 0.5 def x y theRFf end]}
}%
\multido{\na=11+1}{10}{%
\only<\na>{%
\fcRectangularRiemannSum[colorUV=cyan, linecolor=blue]{0}{0}{2}{2}{\na \space 10 sub 3 mul 1 sub}{\na\space 10 sub 3 mul 1 sub}{theRFf}
}%
}%
\uncover<21->{%
\fcRectangularRiemannSum[colorUV=cyan, linecolor=blue]{0}{0}{2}{2}{10 3 mul 1 sub}{10 3 mul 1 sub}{theRFf}%
\fcStartIIIdScene%
\fcSurfaceInScene[linecolor=blue, colorUV=cyan, colorVU=cyan, arrows=none]{0}{0}{2}{2}{[2 dict begin /x u def /y v def x y theRFf end]}{}%
\fcFinishIIIdScene%
}
\end{pspicture}
\column{0.7\textwidth}
\begin{itemize}
\item Suppose region of integration $\mathcal R$ is rectangle, i.e., $\mathcal{R} = [a,b] \times [c,d]$, integration w.r.t. $\diff A = \diff x \diff y$.
$\iint_{\mathcal{R}} f(P) \diff A = \iint_{[a,b] \times [c,d]} f(x,y) \; \diff x\, \diff y$.
\item<2-> If integral exists: approximate by fine enough Riemann sum.
\item<3-> Simplest way: divide $\mathcal R$ into $n\times n$ equal pieces, sides $\alertNoH{3}{\Delta x = \frac{b-a}{n}}$, $\alertNoH{4}{\Delta y = \frac{d-c}{n}} $.
\item<5-> For $(s,t)^{th}$ rectangle $D_{st}$, sample at \alertNoH{5}{midpoint} $\alertNoH{5,6}{ P_{s,t}=} \fcAnswer{6}{\left( a+ \left( s - \frac{1}{2} \right) \Delta x, c+\left( t-\frac{1}{2}\right)\Delta y\right)} $.
\end{itemize}
\end{columns}
\uncover<7->{
\[
\begin{array}{r@{~}c@{~}l}
\displaystyle \alertNoH{21}{ \iint\limits_{\mathcal R} f(x,y)  \diff x \diff y }  &=&\displaystyle \alertNoH{12-20}{\lim\limits_{n \to \infty}  \alertNoH{11}{\sum_{1\leq s,t \leq n} \alertNoH{10}{ \alertNoH{8}{ f \left(P_{s,t} \right)} \alertNoH{9,22}{ \text{area}(D_{st})} }} }\\
\uncover<22->{ &\approx&\displaystyle \sum_{1\leq i,j \leq n} f\left(P_{s,t} \right)  \alertNoH{22}{ \Delta x  \Delta y } \quad .}
\end{array}
\]
}
\end{frame}
 
\begin{frame}
\begin{example}
\begin{columns}
\column{0.3 \textwidth}
\psset{xunit=0.3cm, yunit=0.3cm}
\begin{pspicture}(-4,-4)(4,5.5)%
\tiny%
\renewcommand{\fcScreen}{[-1 -1.1 -3] 0}%
\fcBoundingBox{-4}{-4}{4}{5.5}%
\fcAxesIIId{5}{5}{11}%
\uncover<3->{%
\fcLineIIId[linewidth=0.6pt]{[4 -0.1 0]}{[4 2.1 0]}%
\fcLineIIId[linewidth=0.6pt]{[2 -0.1 0]}{[2 2.1 0]}%
\fcLineIIId[linewidth=0.6pt]{[0 -0.1 0]}{[0 2.1 0]}%
\fcLineIIId[linewidth=0.6pt]{[-0.1 0 0]}{[4.1 0 0]}%
\fcLineIIId[linewidth=0.6pt]{[-0.1 1 0]}{[4.1 1 0]}%
\fcLineIIId[linewidth=0.6pt]{[-0.1 2 0]}{[4.1 2 0]}%
}%
\uncover<5->{%
\fcDotIIId{[1 0.5 0]}%
\fcDotIIId{[1 1.5 0]}%
\fcDotIIId{[3 0.5 0]}%
\fcDotIIId{[3 1.5 0]}%
}%
\uncover<7->{%
\pstVerb{%
0.5 setalpha%
}%
\fcRectangularRiemannSum[colorUV=cyan, linecolor=blue]{0}{0}{4}{2}{2}{2}{x x mul y add}}%
\end{pspicture}
\column{0.7\textwidth}
Use the Midpoint Rule to approximate $\displaystyle \iint_{[0,4]\times [0,2]} x^2y \diff x\diff y,$ with each side divided into $n=2$ pieces.

\uncover<2->{\alertNoH{2,3}{The small rectangles have dimensions}} \fcAnswer{3}{ $\frac{4-0}{2} \cdot \frac{2-0}{2} = 2\cdot 1$ and area $2$.} \uncover<4->{\alertNoH{4,5}{The midpoints are}}
\end{columns}
\fcAnswer{5}{$\displaystyle
P_{11} = \left(1,\frac{1}{2}\right), \quad P_{12} = \left(1,\frac{3}{2}\right), \quad P_{21} = \left(3,\frac{1}{2}\right),  \quad P_{22} = \left(3,\frac{3}{2}\right)\; .
$}
\[
\begin{array}{r@{~}c@{~}l}
\displaystyle \fcQuestion{6}{\iint\limits_{[0,4]\times [0,2]} x^2y\; \diff x \diff y}  &\fcQuestion{6}{\approx} &  \displaystyle \fcAnswer{7}{ 2\left(f \left(1,\frac{1}{2 }\right)  +  f\left(3,\frac{1}{2}\right) + f\left(1,\frac{3}{2}\right)   + f\left(3,\frac{3}{2}\right) \right)} \\
\uncover<8->{& =&\displaystyle   1\cdot \frac{1}{2}\cdot 2 + 9 \cdot \frac{1}{2} \cdot 2 + 1\cdot \frac{3}{2} \cdot 2 + 9 \cdot \frac{3}{2} \cdot 2} \\
\uncover<9->{&=&\displaystyle  1+9+3+27 = 40\; .}
\end{array}
\]

\end{example}
\end{frame}
 


}
\end{document}
