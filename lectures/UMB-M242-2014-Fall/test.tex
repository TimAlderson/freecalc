\documentclass%
%[handout]
{beamer}
% % % % % % % %
% % % % % % % %
% % % % % % % %
%IMPORTANT
%compiles with
%pdflatex -shell-escape
%IMPORTANT
% % % % % % % %
% % % % % % % %
% % % % % % % %
\mode<presentation>
{
\useinnertheme{rounded}
\useoutertheme{infolines}
\usecolortheme{orchid}
\usecolortheme{whale}
}

\usepackage[english]{babel}
\usepackage[latin1]{inputenc}
\usepackage{times}
\usepackage[T1]{fontenc}
\usepackage{../example-templates}
\usepackage{../pstricks-commands}

\usepackage{auto-pst-pdf}
\usepackage{pst-plot}
%\usepackage{pstricks-add}

% Or whatever. Note that the encoding and the font should match. If T1
% does not look nice, try deleting the line with the fontenc.


\graphicspath{{../../modules/}}

\newtheoremstyle{partialproof}{3pt}{3pt}{}{}{}{.}{.5em}{}
\theoremstyle{partialproof} \newtheorem{partialproof}[theorem]{Proof.}
%\DeclareMathOperator{\diff}{d}
\setbeamertemplate{navigation symbols}{}

\includeonlylecture{1}

\newcommand{\lect}[3]{
  \date{#1}
  \lecture[#1]{#2}{#3}
}

\setbeamertemplate{footline}
{
  \leavevmode%
  \hbox{%
  \begin{beamercolorbox}[wd=.333333\paperwidth,ht=2.25ex,dp=1ex,center]{author in head/foot}%
    \usebeamerfont{author in head/foot}\insertshortauthor
  \end{beamercolorbox}%
  \begin{beamercolorbox}[wd=.333333\paperwidth,ht=2.25ex,dp=1ex,center]{title in head/foot}%
    \usebeamerfont{title in head/foot}\insertshorttitle
  \end{beamercolorbox}%
  \begin{beamercolorbox}[wd=.333333\paperwidth,ht=2.25ex,dp=1ex,center]{date in head/foot}%
    \usebeamerfont{date in head/foot}\insertshortdate{}
  \end{beamercolorbox}}%
  \vskip0pt%
}

% If you have a file called "university-logo-filename.xxx", where xxx
% is a graphic format that can be processed by latex or pdflatex,
% resp., then you can add a logo as follows:

%\pgfdeclareimage[height=0.8cm]{logo}{bluelogo}
%\logo{\pgfuseimage{logo}}
\renewcommand{\Arcsin}{\arcsin}
\renewcommand{\Arccos}{\arccos}
\renewcommand{\Arctan}{\arctan}
\renewcommand{\Arccot}{\text{arccot\hspace{0.03cm}}}
\renewcommand{\Arcsec}{\text{arcsec\hspace{0.03cm}}}
\renewcommand{\Arccsc}{\text{arccsc\hspace{0.03cm}}}



\begin{document}

\AtBeginLecture{%

\title[\insertlecture]{FreeCalc}
\subtitle{\insertlecture}
\author[FreeCalc]{}
\institute[UMass Boston]{University of Massachusetts Boston}
\date{\insertshortlecture}
\begin{frame}
  \titlepage
\end{frame}
}%

% begin lecture
\lect{\today}{Sample}{1}{
%\begin{frame}
\frametitle{More General Regions}
What makes iterated integrals work over rectangular regions? \uncover<2->{Slices with respect to one variable are intervals in the other.} \uncover<3->{If variable is $x$:}


\begin{columns}
\column{0.3\textwidth}
\begin{pspicture}(-0.5,-0.5)(3,2.7)
\tiny
\fcAxesStandard{-0.5}{-0.5}{3}{2.7}
\pstVerb{
/f {x 170 mul sin 0.4 mul 0.6 add} def
/g {x 190 mul cos 0.4 mul 2 add x 4 div sub} def
}
\pscustom*[linecolor=\fcColorAreaUnderGraph]{
\psplot{0.2 }{2}{f }
\psplot{2 }{0.2}{g}
}
\multido{\na=4+1}{4}{%
\pstVerb{1 dict begin /x \na\space 4 sub 0.5 mul 0.3 add def}%
\only<\na->{\psline[linecolor=red, linewidth=2pt](! x f)(! x g)}%
\only<\na->{%
\psline[linestyle=dotted](! x f)(! x 0)%
\rput[t](! x -0.1){$\alertNoH{8}{x_{\fcEvalToInt{\na-3}}}$}%
}%
\pstVerb{end}%
}%
\uncover<3>{\rput[t](! 0.3 -0.1){$x_1$}}
\psplot{0.2}{2}{f}
\rput[l](2.1, 2.1){$y=g(x)$}
\psplot{2}{0.2}{g}
\rput[l](2, 0.2){$y=f(x)$}
\end{pspicture}
\column{0.7\textwidth}
\begin{itemize}
\item<3-> fix $x$,
\item<4-> integrate with respect to $y$,
\item<8-> to obtain function that depends only on $x$,
\item<9-> then integrate the so obtained function in $x$.
\end{itemize}
\end{columns}
\uncover<10->{So far used rectangular regions;} \uncover<11->{this also works if \alertNoH{11}{slices are intervals whose endpoints depend continuously} on the location of the slice.}
\begin{itemize}
\item<12-> Regions of type I: vertical slices are segments.
\item<12-> Regions of type II: horizontal slices are segments.
\end{itemize}
\uncover<12->{We call such regions curvilinear trapezoids.}
\end{frame}
 
%\begin{frame}
\frametitle{Strategy: Curvilinear Trapezoids (Type I)}
\begin{columns}
\column{0.3\textwidth}
\begin{pspicture}(-0.5,-0.5)(3,2.7)
\tiny
\fcAxesStandard{-0.5}{-0.5}{3}{2.7}
\pstVerb{ 10 dict begin
/f1 {x 170 mul sin 0.4 mul 0.6 add} def
/f2 {x 190 mul cos 0.4 mul 2 add x 4 div sub} def
}
\pscustom*[linecolor=\fcColorAreaUnderGraph]{%
\psplot{0.2 }{2}{f1 }%
\psplot{2 }{0.2}{f2}%
}%
\psplot{0.2}{2}{f1}%
\psplot{2}{0.2}{f2}%
\uncover<7>{%
\psplot[linewidth=2pt, linecolor=red]{0.2}{2}{f1}%
\psplot[linewidth=2pt, linecolor=red]{2}{0.2}{f2}%
}%
\pstVerb{/x 0.2 def}
\psline(! x f1)(! x f2)
\uncover<2>{\psline[linewidth=2pt, linecolor=red](! x f1)(! x f2)}
\uncover<2->{%
\psline[linestyle=dotted](! x f1)(! x 0)%
\rput[lt](! x -0.2 ){$~a$}
}%
\pstVerb{/x 2 def}
\psline(! x f1)(! x f2)
\uncover<3>{\psline[linewidth=2pt, linecolor=red](! x f1)(! x f2)}
\uncover<3->{%
\psline[linestyle=dotted](! x f1)(! x 0)%
\rput[lt](! x -0.2 ){$~b$}
}%
\pstVerb{/x 1 def}
\uncover<4>{\psline[linewidth=2pt, linecolor=red](! x f1)(! x f2)}
\uncover<5->{\psline(! x f1)(! x f2)}
\uncover<4->{%
\psline[linestyle=dotted](! x f1)(! x 0)%
\rput[lt](! x -0.2){$~x$}
}%
\uncover<6,7>{
\psline[linecolor=red, linewidth=2pt](0.2, -0.5)(0.2, 2.7)
\psline[linecolor=red, linewidth=2pt](2, -0.5)(2, 2.7)
}
\uncover<5->{%
\fcFullDot{x}{f1}%
\fcFullDot{x}{f2}%
\rput[lb](! x f1){$~~(x,f(x))$}%
\rput[lt](! x f2){$~~(x,g(x))$}%
}%
\rput[l](2.1, 2.1){$y=g(x)$}
\rput[l](2.1, 0.2){$y=f(x)$}
\pstVerb{end}%
\end{pspicture}
\column{0.7\textwidth}
\begin{itemize}
\item<2-> Identify the \alertNoH{2}{leftmost point(s), with $x$-coordinate $x=a$} and \alertNoH{3}{the rightmost point(s), $x=b$}.
\item<4-> Draw a vertical slice at a value $x$ between $a$ and $b$.
\item<5-> Find the lowest point on that slice, $(x,f(x))$ and the highest point, $(x,g(x))$.
\end{itemize}
\end{columns}
\uncover<6->{
The region is the region bounded by:
\begin{itemize}
\item<6-> the vertical lines $\alertNoH{6}{x=a}$ and $\alertNoH{6}{x=b}$;
\item<7-> the graphs of $\alertNoH{7}{y=f(x)}$ and $\alertNoH{7}{y=g(x)}$, with $f,g \colon [a,b] \to \mathbb{R}$.
 \end{itemize}
}
\uncover<8->{
\[
\mathcal{R} = \{(x,y)  | {\color{blue}{a \leq x \leq b}} ,  {\color{red}{f(x) \leq y \leq g(x)}} \}  .
\]
}
\uncover<9->{
\[
\iint_{\mathcal{R}} f(x,y) \; \diff x\diff y = \int_{{\color{blue}{x=a}}}^{{\color{blue}{x=b}}} \left( \int_{{\color{red}{y=f(x)}}}^{{\color{red}{y=g(x)}}} f({\color{blue}{x}},{\color{red}{y}})  {\color{red}{\diff y}} \right) \; \color{blue}{\diff x}
\]
}
\end{frame}
 
\begin{frame}
  \frametitle{Strategy: Curvilinear Trapezoids (Type II)}
%
\begin{itemize}
  \item Identify the lowest point, $(*,c)$, and the highest point, $(*,d)$.
  \item Draw a generic horizontal slice at some value $y$ between $c$ and $d$.
  \item Find the leftmost point on that slice, $(h_1(y),y)$ and the rightmost point, $(h_2(y),y)$.
\end{itemize}

The region is bounded by:
 \begin{itemize}
   \item horizontal lines $\textcolor[rgb]{0.98,0.00,0.00}{y=c}$ and $\textcolor[rgb]{0.98,0.00,0.00}{y=d}$
   \item graphs of $\textcolor[rgb]{0.00,0.00,1.00}{x=h_1(y)}$ and $\textcolor[rgb]{0.00,0.00,1.00}{x=h_2(y)}$, with  $h_1,h_2 \colon [c,d] \to \mathbb{R}$:
 \end{itemize}
%
$$\mathcal{R} = \{(x,y) \; | \; \textcolor[rgb]{0.98,0.00,0.00}{c \leqslant y \leqslant d}\; , \; \textcolor[rgb]{0.00,0.00,1.00}{h_1(y) \leqslant x \leqslant h_2(y)} \} \; .$$
%
$$\int\!\!\!\int_{\mathcal{R}} f(x,y) \; dxdy = \int_{\textcolor[rgb]{0.98,0.00,0.00}{y=c}}^{\textcolor[rgb]{0.98,0.00,0.00}{y=d}} \left( \int_{\textcolor[rgb]{0.00,0.00,1.00}{x=h_1(y)}}^{\textcolor[rgb]{0.00,0.00,1.00}{x=h_2(y)}} f(\textcolor[rgb]{0.00,0.00,1.00}{x},\textcolor[rgb]{0.98,0.00,0.00}{y}) \; \textcolor[rgb]{0.00,0.00,1.00}{dx}\right) \; \textcolor[rgb]{0.98,0.00,0.00}{dy}$$
\end{frame} 
\begin{frame}
\begin{example}
\begin{columns}
\column{0.4\textwidth}
\psset{xunit=0.6cm, yunit=0.6cm}
\begin{pspicture}(-0.1, -1.2)(4,4.5)%
\tiny%
\fcBoundingBox{-0.1}{-1.2}{4}{4.5}%
\renewcommand{\fcScreen}{[-1 2 -3] 0}%
\only<13->{%
\pscustom*[linecolor=\fcColorAreaUnderGraph]{%
\fcCurveIIId{0}{2}{[t t t mul 0]}%
\fcLineIIId{[2 4 0]}{[2 4 2.5]}%
\fcCurveIIId{2}{0}{[t 2 t mul t t mul 4 t t mul mul add 8 div]}%
}%
\fcLineIIId[linewidth=0.5pt, linecolor=black]{[2 4 0]}{[2 4 2.5]}%
\fcStartIIIdScene%
\fcSurfaceInScene[linewidth=0.5, iterationsU=4, iterationsV=4, colorUV={0.2 0.7 0.9}]{0.00001}{0}{2}{1}{[2 dict begin /x u def /y {v 2 u mul mul 1 v sub u u mul mul add } def x y x x mul y y mul add 0.125 mul end]}{}%
\fcFinishIIIdScene[fastsort=true]%
}%
\only<handout:0|6-12>{%
\pscustom*[linecolor=\fcColorAreaUnderGraph]{%
\fcCurveIIId[linewidth=0.5pt, linecolor=black]{0}{2}{[t t t mul 0]}
\fcCurveIIId[linewidth=0.5pt, linecolor=black, linestyle=dashed]{2}{0}{[t 2 t mul 0]}%
}%
}%
\uncover<5->{\fcCurveIIId[linewidth=0.5pt, linecolor=black]{0}{2}{[t t t mul 0]}}%
\uncover<3->{\fcCurveIIId[linewidth=0.5pt, linecolor=black, linestyle=dashed]{2}{0}{[t 2 t mul 0]}}%
\fcAxesIIId{3}{3}{3}%
\end{pspicture}
\psset{xunit=0.6cm, yunit=0.6cm}
\begin{pspicture}(-0.5,-0.5)(2.6,5.2)
\tiny
\fcBoundingBox{-0.5}{-0.5}{2.6}{5.2}
\fcAxesStandard{-0.5}{-0.5}{2.5}{5}
\fcLabels{2.5}{5}
\uncover<6->{
\pscustom*[linecolor=\fcColorAreaUnderGraph]{%
\psplot{0}{2}{x x mul}%
\psline(2, 4)(0,0)%
}%
}
\uncover<5->{
\psplot{-0.2}{2.2}{x x mul}
\rput[l](1.3, 1){$\alert<2>{y=x^2}$}
}
\uncover<3->{
\rput[r](1.2, 2.4){$\alert<2>{y=2x}$}
\psline(-0.2, -0.4)(2.4, 4.8)
}
\uncover<10->{
\fcFullDot{0}{0}
\rput[tl](0.1, -0.1){$\alert<10>{(0,0)}$}
}
\uncover<12->{
\rput[r](1.9, 4){$\alert<12>{(2,4)}$}
\fcFullDot{2}{4}
}
\uncover<15->{
\psline[linecolor=red, linewidth=1.5pt, arrows=<->](0,0)(2,0)
\fcFullDot{0}{0}
\fcFullDot{2}{0}
}
\uncover<16->{
\fcFullDot{1}{0}
}
\uncover<17->{
\psline[linestyle=dotted](1,1)(1,0)
\psline[linecolor=red, linewidth=1.5pt, arrows=<->](1,1)(1,2)
\fcFullDot{1}{1}
\fcFullDot{1}{2}
}
\end{pspicture}

\column{0.6\textwidth}
Let $\mathcal{R}$ be the region bounded by $y=2x$ and $y=x^2$. Compute
\[
\iint_{\mathcal{R}} \frac{1}{8}\left(x^2+y^2\right) \diff x\diff y
\]
\fcQuestion{2}{Plot $y=2x$.} \fcQuestion{4}{ Plot $y=x^2$.} \uncover<6->{\alert<6>{Identify the region. }} 
\end{columns}
\only<handout:1|1-18>{
\fcQuestion{7}{The two curves intersect when } \fcAnswer{8}{ $
\begin{array}{rcl}x^2&=&2x \\ x(x-2)&=&0\\ x&=& 0 \text{ or } 2.\end{array}
$ 
}

\uncover<9->{The \alert<9,10,11,12>{intersection points are} therefore $\alert<9,10>{(0, \fcAnswer{10}{0} )}$ and $\alert<11,12>{(2,\fcAnswer{12}{4})}$.} \uncover<13->{We can plot the function $\frac{1}{8}\left(x^2+y^2\right)$ as above.} \uncover<14->{Our integral is}
}

$
\begin{array}{r@{~}c@{~}l}
\uncover<14->{\alert<18,19>{\displaystyle \int\limits_{{ \fcQuestion{ 14}{x=} \fcAnswer{15}{0}}}^{{ \fcQuestion{ 14}{ x =}\fcAnswer{15}{2}}}\left( \int\limits_{{ \fcQuestion{16}{ y= }\fcAnswer{17}{x^2 }}}^{{\fcQuestion{ 16}{ y=} \fcAnswer{ 17}{2x}}} \frac{1}{8}\left(\alert<20,21>{x^2+y^2} \right) \diff y \right) \diff x}} \uncover<handout:2| 20->{&=&\displaystyle \frac{1 }{8} \int_{x=0 }^{x=2}  { \left[\fcAnswer{21}{\alert<22,23>{ x^2y +\frac{ y^3}{3}}} \right]}_{ \alert<23>{y=x^2}}^{ \alert<22>{y=2x} }\diff x} \\
\uncover<handout:2| 22->{&=&\displaystyle \frac{1}{8}{\alert<24,25>{ \int}}_{0}^2 \alert<24,25>{\left( \alert<22>{ 2x^3+ \frac{8}{3}x^3} \alert<23>{  -x^4 -\frac{ x^6 }{3}} \right)\diff x}} \\
\uncover<handout:2| 24->{&=& \displaystyle \frac{1}{8} {\left[\fcAnswer{25}{-\frac{ 1}{21} x^{7}-\frac{1}{5} x^{5}+\frac{7}{6} x^{4}}\right]}_{x=0}^{x=2}\\ \uncover<26->{&=&\frac{27}{35}}}
\end{array}
$


\end{example}

\vskip 10 cm

\end{frame} 
\section{Triple Integrals}
\begin{frame}
\frametitle{Density and Mass}
\begin{question}
Let $\mathcal R$ be a region in space. Suppose we know the density of $\mathcal R$ at every point. Can we find the mass of $\mathcal R$?
\end{question}
\begin{itemize}
\item<2-> Partition the region $\cR$ into regions with $D_1,\dots, D_k$ with small diameter. Denote by $\text{maxdiam}(\cD) = \max_{k} \text{diam}(D_k)$ the maximum of the diameters of the $D_i$'s. 
\item<3-> Choose a sample point $P_k$ inside each $D_k$. Then 
$
\text{mass}(D_k) \approx \rho(P_k) \text{vol}(D_k) .
$
\item<4-> Sum the above approximations to get an approximation for  $\text{mass}\mathcal R$: $\text{mass}(\cR) \approx \sum_{k=1}^{N} \rho(P_k)\text{vol}(D_k)\; .$
\item<5-> Take the limit as the diameter of the partitions tends to zero:
$$\text{mass}(\cR) = \lim_{\text{maxdiam}(\cD) \to 0}  \sum_{k=1}^{N} \rho(P_k)\text{vol}(D_k)\; .$$
\end{itemize}
\end{frame} 
\begin{frame}
\frametitle{Triple Integrals}
Let $f$ be a scalar or vector-valued function on region $\cR$.
\begin{definition}
If the limit
$$\lim_{\max_k\text{diam}(D_k) \to 0}  \sum_{k=1}^{N} f(P_k)\text{vol}(D_k)$$
exists and is finite, its value is called  \emph{the integral of $f$ on $\cR$ with respect to volume} and is denoted by
\[
\iiint_{\cR} f(P) \diff V\quad .
\]
\end{definition}
\begin{itemize}
\item<2-> If $f$ is a scalar function, then the value of the integral is a scalar;
\item<3-> If $f$ is a vector-valued function, then the value of the integral is a vector.
\end{itemize}
The limit does not always exist. It exists if the function $f$ is continuous.
\end{frame} 
\begin{frame}
\frametitle{Examples}

\begin{itemize}
  \item Volume of a region:
  %
  $$\text{vol}(\cR) = \int\!\!\!\!\int\!\!\!\!\int_{\cR} 1 \cdot dV$$
  %
  \item Mass of a body:
  %
  $$\text{mass}(\cR) = \int\!\!\!\!\int\!\!\!\!\int_{\cR} \text{density}(P) \cdot dV\; .$$
  %
  \item The average value of a function $f$ with respect to volume:
%
$$\text{average value of }f = \frac{1}{\text{vol}(\cR)} \int\!\!\!\!\int\!\!\!\!\int_{\cR} f(P) \cdot dV\; .$$
  %
  \item The average value of a function $f$ with respect to mass distribution:
%
$$\text{average value of }f = \frac{1}{\text{m}(\cR)} \int\!\!\!\!\int\!\!\!\!\int_{\cR} f(P)\,  dm = \frac{1}{\text{m}(\cR)} \int\!\!\!\!\int\!\!\!\!\int_{\cR} f(P)\rho(P)\, dV\; .$$
\end{itemize}

\end{frame} 
\begin{frame}
  \frametitle{Iterated Integrals}

  There are two ways of slicing a 3D region $\cR$:
  \begin{itemize}
    \item By slices:
    \begin{itemize}
      \item Project the body on an axis;
      \item Look at 2D slices perpendicular to that axis (CT-scan)
    \end{itemize}
      %
      $$\iiint_{\cR}f(P) \, dV = \int_{\text{location of slice}} \left(\iint_{\text{slice}} f(P) \, dA\right) \, dh$$
      %
      \item By rods:
      \begin{itemize}
        \item Project the body on a plane;
        \item Look at 1D slices perpendicular to that plane (rods)
      \end{itemize}
      %
      $$\iiint_{\cR}f(P) \, dV = \iint_{\text{location of rod}} \left(\int_{\text{rod}} f(P) \, dh\right) \, dA$$
      %
  \end{itemize}
  %
  \begin{itemize}
    \item Reduce the computation to computations of
    \begin{itemize}
      \item  a single integral and
      \item a double integral
    \end{itemize}
  \end{itemize}
\end{frame} 
\begin{frame}
\frametitle{Example: Moment of Inertia}
\begin{itemize}
\item Problem: compute the moment of inertia $I$
\begin{itemize}
\item \alertNoH{2}{of a rectangular box with sides $2a$, $2b$, and $2c$}
\item \alertNoH{3}{rotating about axis $L$ through center that is perpendicular to a face.}
\item \alertNoH{4}{The box has constant density $\rho$.} \uncover<11->{Therefore it's mass is $m=8\rho abc $.}
\end{itemize}
\item<5-> Coord. system: rotation axis = $z$-axis, $x,y$ axes along box sides.
\uncover<6->{
\[
I = \iiint_{\cR} \rho\, \text{dist}^2(P,L) \diff V = \iiint_{\cR} \rho (x^2+y^2)\, \diff x\diff y \diff z\; .
\]
}
\item<7-> Decompose into slices as follows.
\begin{itemize}
\item<7-> Project $\cR$ onto the $z-$axis \alertNoH{8}{to get segment from $z=-c$ to $z=c$.}
\[
\iiint_{\cR} \rho (x^2+y^2) \diff x \diff y \diff z = \alertNoH{8}{\int_{z=-c}^{z=c}} \left( \iint_{S_z} \rho (x^2+y^2)\, \diff x \diff y \right) \diff z
\]
\item<9-> For a fixed $z$, the slice $S_z$ is: $-a \leq x \leq a$, $-b \leq  y \leq  b$.
\[
I_L = \alertNoH{10,11}{ \int_{z=-c}^{z=c} \left(\int_{x=-a}^{x=a} \left( \int_{y=-b}^{y=b} \rho (x^2+y^2)  \diff y \right)  \diff x\right)  \diff z} \uncover<10->{\alertNoH{10,11}{=}} \fcAnswer{11}{\frac{m(a^2+b^2)}{3}} \uncover<11->{.}
\]
%

\end{itemize}
  \end{itemize}
\end{frame}
 
\begin{frame}
\begin{example}
\begin{columns}
\column{0.3\textwidth} 
\begin{pspicture}(-1,-1)(1,1)
\tiny
\renewcommand{\fcScreen}{[-1 3 -1] 0}
\fcBoundingBox{-0.4}{-1.8}{3.3}{2.4}
\fcStartIIIdScene%
\only<5->{\fcPatchInScene{[0 0 0]}{[0 1 0]}{[0 0 2]}}%
\only<4->{\fcPatchInScene{[0 0 0]}{[1 0 0]}{[0 1 0]}}%
\only<6->{\fcPatchInScene{[0 0 0]}{[1 0.5 0]}{[0 0 2]}}%
\only<7->{%
\fcTriangleInScene{[0 0 2]}{[1 0.5 0]}{[0 1 0]}%
\fcTriangleInScene{[1 1.5 -2]}{[1 0.5 0]}{[0 1 0]}%
\fcLineIIIdInScene{[0 1 0]}{[1 0.5 0]}%
}%
\fcAxesIIIdInScene{3}{3}{2.3}%
\fcFinishIIIdScene%
\uncover<8->{
\fcDotIIId{[0 0 0]}
\fcDotIIId{[0 1 0]}
\fcDotIIId{[1 0.5 0]}
\fcDotIIId{[0 0 2]}
}
\fcPutIIId[l]{[3 0 0]}{$x$}
\fcPutIIId[l]{[0 3 0]}{$y$}
\fcPutIIId[b]{[0 0 2.4]}{$z$}
\end{pspicture}
\column{0.7\textwidth}
Compute the volume of the \alert<3-7>{region $\cR$ bounded by} \alert<7>{$x+2y+z=2$}, \alert<6>{$x=2y$}, \alert<5>{$x=0$}, \alert<4>{$z=0$}.
\uncover<2->{
\[
\text{vol}(\cR) = \iiint_{\cR} 1\cdot \diff V\; .
\]
}
\uncover<3->{\alert<3-7>{$\cR$ is}} \uncover<3-7>{\alert<3-7>{\textbf{?}}} \uncover<8->{\alert<8>{a tetrahedron with vertices at $(0,0,0)$, $(0,1,0)$, $(0,0,2)$, and $\left(1, \frac{1}{2}, 0\right)$}.}
\end{columns}
\only<1-13>{ 
\uncover<9->{Project $\cR$ onto the $z-$axis to get segment from $z=0$ to $z=2$.} \fcQuestion{10}{Fix a value for $z$ to get slice $S_z$ equal to} \fcAnswer{11}{the triangle $(0,0,z)$, $(0,1-\frac{z}{2},z)$, $(1-z, \frac{1}{2}-\frac{z}{2},z)$}
}
\uncover<12->{
\alert<13,14>{
$\iiint_{\cR} 1\cdot \diff V = \int_{z=0}^{z=2} \left( \iint_{S_z} 1\cdot \diff x \diff y \right)  \diff z
$
}
}

\uncover<15->{
Project $S_z$ onto the $x-$axis to get segment from $x=0$ to $x=1-z$.} \uncover<16->{ Fix $x$ in that range. Then the vertical slice is a segment from $y=\frac{x}{2}$ to $y=1-\frac{z}{2} - \frac{x}{2}$.}
\uncover<17->{
\[
\text{vol}(\cR) = \iiint_{\cR} 1\cdot \diff V = \int_{z=0}^{z=2} \left(\int_{x=0}^{x=1-z} \left( \int_{y=\frac{x}{2}}^{y=1-\frac{z}{2}-\frac{x}{2}} 1 \cdot \diff y \right)  \diff x\right) \diff z .
\]
}
\end{example}
\end{frame} 
\begin{frame}
\underline{Decomposition by rods}:
\begin{itemize}
  \item Projection of the region onto the $xy-$plane:
  \begin{itemize}
    \item triangle $D$ with vertices $(0,0,0)$, $(0,1,0)$, and $(1,\frac{1}{2},0)$
  \end{itemize}
  \item For a generic location $(x,y)$ within this region, the vertical rod is
  \begin{itemize}
    \item segment with endpoints $z=0$ and $z=2-x-2y$
  \end{itemize}
 %
$$\iiint_{\cR} 1\cdot dV = \iint_D \left( \int_{z=0}^{z=2-x-2y} 1\cdot dz \right) \, dxdy = \iint_D (2-x-2y) \; dxdy$$
%
\item Projection of $D$ over the $x-$axis:
\begin{itemize}
  \item segment from $x=0$ to $x=1$;
\end{itemize}
%
\item For generic $x$ in that range, the slice is
\begin{itemize}
  \item segment from $y=\frac{x}{2}$ to $y=1- \frac{x}{2}$
\end{itemize}
%
$$\iint_{D} f(x,y) \, dxdy = \int_{x=0}^{x=1-z} \left( \int_{y=x/2}^{y=1-x/2} f(x,y)\, dy \right) \, dx$$
%
$$\text{vol}(\cR) = \iiint_{\cR} 1\cdot dV = \int_{x=0}^{x=1-z} \left( \int_{y=x/2}^{y=1-x/2} \left( \int_{z=0}^{z=2-x-2y} 1\cdot dz \right)dy \right) \, dx \; .$$
\end{itemize}

\end{frame}  
}
\end{document}
