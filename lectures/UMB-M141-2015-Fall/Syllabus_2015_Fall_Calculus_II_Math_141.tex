\documentclass{article}
\usepackage{amsmath, amsfonts, amssymb, verbatim, hyperref}
\usepackage{enumitem}
\usepackage{pst-plot}
\usepackage{pstricks}
\usepackage{lscape}

\addtolength{\hoffset}{-3.5cm}
\addtolength{\textwidth}{6.8cm}
\addtolength{\voffset}{-3.3cm}
\addtolength{\textheight}{6.3cm}

\newcounter{topicsCounter}
\newcounter{topicsSubCounter}[topicsCounter]
\newcounter{topicsSubSubCounter}[topicsSubCounter]

\usepackage{longtable}
\usepackage{xr}
\externaldocument{../../homework/UMB-All-Problems-By-Course/Calc-II-MasterProblemSheet}
%\externaldocument{./Calc-I-MasterProblemSheetOneFile}

\newcommand{\refBad}[1]{%
\ifthenelse{\equal{\ref{#1}}{??}}%
{(n/a)}%
{\ref{#1}}%
}%example usage: \refBad{\ref{eqMacLaurinDef}}{their definition}{their definition (Definition \ref{eqMacLaurinDef})}


\newcommand{\counterTopic}{ \refstepcounter{topicsCounter}\thetopicsCounter.&&& }
\newcommand{\counterSubTopic}{&\refstepcounter{topicsSubCounter}\thetopicsCounter.\thetopicsSubCounter. && }
\newcommand{\counterSubSubTopic}{&&\refstepcounter{topicsSubSubCounter}\thetopicsCounter.\thetopicsSubCounter.\thetopicsSubSubCounter.& }

\newcommand{\apex}{A\kern -1pt \lower -2pt\hbox{P}\kern -4pt \lower .7ex\hbox{E}\kern -1pt X}


\newcommand{\websitebase}{https://piazza.com/umb/fall2015/m141}

\usepackage{pdfpages}

\title{Math 141 Calculus II \\ Fall 2015}
\date{}
\begin{document}

%\color{green}
\maketitle
%\noindent\textbf{Time and place.}
%Monday, Wednesday, Friday 10-10:50, McCormack, Room 417, first floor. Monday 11:00-11:50,

\noindent \textbf{Instructor(s).} 
\begin{tabular}{ll}
Todor Milev & \href{mailto:todor.milev@gmail.com}{\nolinkurl{todor.milev@gmail.com}} 
\end{tabular}

\medskip
\noindent \textbf{Office hours. } \begin{tabular}{lp{12cm}}
Todor Milev & By appointment, or walk-in office hours: Tuesday 12:30-14:00, Thursday 15:30-17:00 (if I am not in the office I may be off to lunch).  Room: S-03-65.\\
\end{tabular}





\medskip \noindent \textbf{Lecture slides. }  \url{\websitebase/resources}

\medskip\noindent Lecture slides may be updated as the course progresses.


\medskip \noindent \textbf{Master Problem Sheet. }  \url{\websitebase/resources} 

\medskip\noindent The master problem sheet contains a collection of Calculus I problems. 

\medskip
\noindent \textbf{Homework.} You will be assigned homework, which will be posted on

\url{\websitebase/resources} \quad \quad \quad .

\noindent You will be expected to complete the homework in written form in a convenient for you format (notebook, folder, etc.). However, \textbf{the instructor(s) will not check/collect/proofread your homework.} 
 
\medskip
\noindent \textbf{Quizzes.} You will be given quizzes in class. \textbf{The time of the quiz will be announced in class}. Quizzes may be announced from one lecture day to the next. Your quiz problem will be one of your homework problems, verbatim (no number changes).





\medskip\noindent \textbf{Textbook. } There are two official textbooks for this course. It is  \textbf{mandatory} to have access to \textbf{at least one of the textbooks}. Having only one of the two textbooks is sufficient to complete the course. No mandatory homework will be assigned from either textbook. 

\begin{itemize}
\item Option I. The textbook \apex{} 3.0, Chapters 6-9. To download a free pdf file of the textbook, or to buy a physical copy of the textbook, visit the following site.

\url{http://www.apexcalculus.com/downloads/} 
\item Option II. James Stewart, Calculus, $7^{th}$ or $8^{th}$ edition (both editions are acceptable).
\end{itemize}

%\medskip
%\noindent \textbf{Prerequisite. } A standard pre-calculus course or equivalent.


\medskip
\noindent \textbf{Grades.} Your grade will consist of two tests, a comprehensive final exam, and a number of quizzes. 
\begin{itemize}
\item The quizzes will account for 20\% of your total grade.
\item The tests will account for 50\% (25\% each) of your total grade.
\item The final will account for 30\% of your total grade.
\end{itemize}
Please note that missed tests can not be made up, unless there is a valid medical reason accompanied with an official signed document from a medical doctor. Letter grades will be assigned as follows. 

\begin{center}
\begin{tabular}{lc|lc}
A & 85-100 & C & 65-69 \\
A-& 82-84 & C- & 62-64 \\
B+& 80-81 & D+ & 60-61 \\
B & 75-79& D & 55-59\\
B-& 72-74& D- & 50-54\\
C+& 70-71& F & below 50\\
\end{tabular}

\end{center}

No books, notes, calculators or any other electronic device (such as mobile phones) are allowed during any exam unless otherwise stated.

\medskip
\noindent \textbf{Student conduct.} Students  are required to adhere the University Policy on Academic Standards and Cheating, to the University Statement of Plagiarism and the Documentation of Written Work, and to the Code of Student Conduct as described in the catalog of Undergraduate programs, pages 44-45 and 48-52. The code is available at the following web-page.

\noindent\url{http://www.umb.edu/life_on_campus/policies/code/}
\begin{landscape}

\noindent \begin{longtable}{|@{}r@{}l@{}l@{~}l@{~}c cccc|}\hline
\multicolumn{9}{|c|}{\textbf{List of topics.} The list of topics is a preliminary guideline, and will be subject to change.
}\\\hline
&&& Topic & \apex{} 3.0 textbook & \begin{tabular}{l}Stewart, \\ Calculus, $7^{th}$ ed. \end{tabular} & \begin{tabular}{l} Relevant problems \\ Master Problem\\ Sheet\end{tabular} & \begin{tabular}{l}Lecture  \end{tabular} & \begin{tabular}{l} Expected \\ Week \\ (total 14) \end{tabular}  \\ \hline
\counterTopic Inverse trigonometric functions. & & & & & \\
\counterSubTopic     Review of trigonometry.  & &  &  &  &  \\ 
\counterSubTopic    Inverse trigonometric functions & & & & &\\
\counterSubSubTopic Definitions of inverse trigonometric functions. & & & & & \\
\counterSubSubTopic Derivatives of inverse trigonometric functions. & & & & & \\
\counterTopic Review of Integration. & & & & & \\
\counterTopic Integration, Review.. & & & & & \\
\counterSubTopic The Fundamental Theorem of Calculus.. & & & & & \\
\counterSubTopic Differential forms. & & & & & \\
\counterSubTopic Integration and logarithms. & & & & & \\
\counterTopic  Techniques of integration. & & & & & \\
\counterSubTopic Integration by parts. & & & & & \\
\counterSubTopic Integration of rational functions. & & & & & \\
\counterSubSubTopic Building block integrals. & & & & & \\
\counterSubSubTopic Partial fractions. & & & & & \\
\counterSubTopic Trigonometric integrals. & & & & & \\
\counterSubTopic Integrals of radicals of quadratics. & & & & & \\
\counterSubSubTopic Trigonometric substitutions. & & & & & \\
\counterSubSubTopic Euler substitutions corresponding to trig substitutions.&  &  & & & \\
\counterTopic L'Hospital's rule. & & & & & \\
\counterTopic Improper integrals. & & & & & \\
\counterTopic Polar coordinates. & & & & & \\
\counterTopic Curves & & & & & \\
\counterSubTopic Curve images and parametric curves. & & & & & \\
\counterSubTopic Curve (arc) length.& & & & &  \\
\counterSubTopic Curves in polar coordinates. & & & & & \\
\counterSubTopic Area locked by curves. & & & & & \\
\counterTopic Sequences. & & & & & \\
\counterTopic Series. & & & & & \\
\counterSubTopic Geometric and arithmetic sums.& & & & & \\
\counterSubTopic Telescoping series. & & & & & \\
\counterSubTopic Comparison test.& & & & & \\
\counterSubTopic Integral test.& & & & & \\
\counterSubTopic Integral test.& & & & & \\
\counterSubTopic Absolute convergence, alternating series.& & & & & \\
\counterSubTopic Ratio and root tests.& & & & & \\
\counterTopic  Power series. & & & & & \\
\counterSubTopic Radius and interval of convergence. & & & & & \\
\counterSubTopic Maclaurin and Taylor series. & & & & & \\
\counterSubTopic Integrating and differentiating Maclaurin and Taylor series. & & & & & \\
\counterSubTopic Maclaurin series of $\ln(1+x), e^x, \sin x, \cos x, \arctan x, \arcsin x$. & & & & & \\
\counterTopic  Differential equations.& & & & & \\
\counterSubTopic Direction fields. & & & & & \\
\counterSubTopic Separable equations. & & & & & \\
\counterSubTopic The logistic (population growth) equation.& & & & & \\
\counterTopic  Complex numbers. & & & & & \\
\end{longtable}
\end{landscape}
\end{document}