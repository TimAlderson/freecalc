\documentclass{article}
%\addtolength{\hoffset}{-3.5cm}
%\addtolength{\textwidth}{6.8cm}
%\addtolength{\voffset}{-3cm}
%\addtolength{\textheight}{6cm}
\ProvidesPackage{homework-problems}
\usepackage{amsmath, amsfonts, amssymb, verbatim, hyperref, ifthen}
\usepackage{auto-pst-pdf}
\usepackage{pst-plot}
\usepackage{multicol}
\renewcommand{\Re}{\mathrm{Re~}}
\renewcommand{\Im}{\mathrm{Im~}}
\newcommand{\doublebrace}[4]{\left\{\begin{array}{ll} #1 & #2 \\#3 & #4  \end{array} \right.}
\newcommand{\triplebrace}[6]{\left\{\begin{array}{ll} #1 & #2 \\#3 & #4  \\#5 & #6\end{array} \right.}
\newcommand{\bigFatWarning}{ %\textbf{This homework contains copyrighted material from  James Stewart, Calculus, 7th edition, 2012. You are not permitted to copy this file for any purpose other than completing your homework. You are not allowed to give a copy of this file to anyone outside of our course. }
}
\newenvironment{solution}%
{\begin{proof}[\bfseries\upshape Solution]\renewcommand{\qedsymbol}{}}%
{\end{proof}}%
\newcommand{\ans}[1]{\iftoggle{solutions}{\begin{solution}#1\end{solution}}{}}
\newcommand{\homeworkEnd}{\end{enumerate}\end{document}}
\newcommand{\homeworkStart}[2]{\title{\course \\ Homework \ #1}\date{%
\ifthenelse{\equal{#2}{}}{}{%
Due #2 at \deadline}}%
\begin{document}\maketitle\begin{enumerate}
}%
\newcommand{\points}[1]{\stepcounter{enumi}\item[ ({\bf #1 mark\ifthenelse{\equal{#1}{1}}{}{s}}) \arabic{enumi}.]}
\newcommand{\pointsii}[1]{\stepcounter{enumii}\item[ ({\bf #1 mark\ifthenelse{\equal{#1}{1}}{}{s}}) (\alph{enumii})]}
\newcommand{\answer}[1]{ \hfill{~} \rotatebox{180}{ answer: #1}}
 %warning folder paths are relative to the file that uses the includepackage

\renewcommand{\answer}[1]{\iftoggle{answers}{ \hfill{~} \rotatebox{180}{\tiny answer: #1}}{} }
\renewcommand{\hiddenanswer}{\answer}
\renewcommand{\points}[1]{\item}
\renewcommand{\pointsii}[1]{\item}
\renewcommand{\Arctan}{\arctan}
\renewcommand{\Arccos}{\arccos}
\renewcommand{\Arcsin}{\arcsin}
\renewcommand{\Arccot}{\operatorname{arccot}}


\toggletrue{solutions}
\toggletrue{answers}
\renewcommand{\fcProblemRef}{\theproblem.\theenumi}
\renewcommand{\fcSubProblemRef}{\theproblem.\theenumi.\theenumii}


\newcommand{\hide}[1]{}
\newtheorem{problem}{Problem}
\pagestyle{empty}
\begin{document}
\begin{center}
\Large
Review sheet Test 2 \\ Math 140 Calculus I \\ \normalsize Fall 2015 \\ Instructor(s): Todor Milev, Ping Xu
\end{center}
%\noindent \textbf{Name:\underline{~~~~~~~~~~~~~~~~~~~~~~~} } \hfill{~}



\noindent The exam is closed textbook. \textbf{No electronic devices are allowed during the exam. } The exam will contain a number of problems and sub-problems (the number of problems may depend on your section/instructor). \textbf{Expect each problem type from this review sheet to appear on the test.}%You are allowed one single formula sheet, handwritten by you. No template problem solutions are allowed. The sheet will be collected with the test. Photocopied formula sheets are not allowed. 

\begin{problem}Compute the limits. The answer key has not been fully proofread, use with caution.
\begin{multicols}{2}
\begin{enumerate}
\item $\displaystyle \lim\limits_{x\to 0} \frac{\sin x  }{x}$. 

\answer{$1$}
\item $\displaystyle \lim\limits_{x\to 0} \frac{x}{\ln (1+x)}$. 

\answer{$ 1$}
\item $\displaystyle \lim\limits_{x\to 0} \frac{x^2}{x-\ln (1+x)}$. 

\answer{$2$}
\item $\displaystyle \lim\limits_{x\to 0} \frac{x^2}{\sin x\ln (1+x)}$. 

\answer{$ 1$}
\item $\displaystyle \lim\limits_{x\to 0} \frac{\sin^2 x  }{\left(\ln (1+x)\right)^2}$.

\answer{$ 1$}
\item $\displaystyle \lim\limits_{x\to 0} \frac{\cos x- 1}{\sin x\ln (1+x)}$.

\answer{$- \frac{1}{2} $}
\item $\displaystyle \lim\limits_{x\to 0} \frac{\arctan x -x}{x^3} $.

\answer{$ -\frac{1}{3} $}
\item $\displaystyle \lim\limits_{x\to 0} \frac{\arcsin x -x}{x^3} $.

\answer{$ \frac{1}{6}$}
\item $\displaystyle \lim\limits_{x\to 1} \frac{x}{x-1}-\frac{1}{\ln x}$.

\answer{$ \frac{1}{2}$}
\item $\displaystyle \lim\limits_{x\to 0} \frac{\cos (nx) -\cos (mx)}{x^2 }$.

\answer{$\frac{m^2-n^2}2 $}
\item \label{eqProblemLimlixto0(arcsinx-x-x^3/6)/(sin^5 x)}  $\displaystyle \lim \limits_{x\to 0} \frac{\arcsin x-x-\frac{1}{6}x^3}{\sin^5 x} $. 

\answer{$\frac{3}{40}$}
\item \label{problemLHospital (sin (pi x) ln x )/ (cos pi x +1)}  $\displaystyle \lim\limits_{x\to 1} \frac{\sin \left(\pi x \right)\ln x }{\cos(\pi x)+1 } $.

\answer{$-\frac{2}{\pi}$}

\item $\displaystyle \lim\limits_{x\to 0} \frac{\sin x-x }{\arcsin x-x } $.

\answer{$ -1$}
\item \label{problemlim x to 0 (sin x - x)/(arctan x - x)} $\displaystyle \lim\limits_{x\to 0}\frac{\sin x- x}{\Arctan x -x}$.

\answer{$\frac{1}{2}$}

\item 
\label{problemlimxtoinftysin(2/x)}
$ {\displaystyle \lim_{x \to \infty} x \sin\left(\frac{2}{x}\right)}.$

\answer{$2$}

\end{enumerate}
\end{multicols}
Problem \ref{eqProblemLimlixto0(arcsinx-x-x^3/6)/(sin^5 x)} can be done easily using Maclaurin series, but we challenge the student to try it using L'Hospital's rule.
\end{problem}
\solution{\ref{problemLHospital (sin (pi x) ln x )/ (cos pi x +1)}
The limit is of the form ``$\frac{0}{0}$'' so we are allowed to use L'Hospital's rule.
\[
\begin{array}{rcll|l}
\displaystyle
\lim\limits_{x\to 1} \frac{\sin \left(\pi x\right)\ln x }{\cos(\pi x)+1 } &=&\displaystyle \lim\limits_{x\to 1}  \frac{\left(\sin \left(\pi x\right)\ln x\right)' }{\left(\cos(\pi x)+1\right)' }\\
&=&\displaystyle \lim\limits_{x\to 1}  \frac{\left(\pi \cos \left(\pi x\right)\ln x+ \sin\left(\pi x\right)\frac{1}{x}\right) }{\left(-\pi\sin(\pi x)\right) } &&\text{type ``$\frac{0}{0}$'',  L'Hospital's rule}\\
&=&\displaystyle \lim\limits_{x\to 1}  \frac{\left(\pi \cos \left(\pi x\right)\ln x+ \sin\left(\pi x\right)\frac{1}{x}\right)' }{\left(-\pi\sin(\pi x)\right)' } \\
&=&\displaystyle \lim\limits_{x\to 1}  \frac{- \pi^{2} \sin{}\left(\pi x\right) \ln{}\left(x\right)+2 \pi \cos{}\left(\pi x\right) x^{-1}- \sin{}\left(\pi x\right) x^{-2} }{\left(-\pi^2\cos(\pi x)\right) } \\
&=&\displaystyle  \frac{- \pi^{2} \sin{}\left(\pi \right) \ln(1)+2 \pi \cos{}\left(\pi \right) - \sin{}\left(\pi \right)  }{\left(-\pi^2\cos(\pi )\right) } \\
&=&\displaystyle -\frac{2}{\pi}\quad .
\end{array}
\]
}


\solution{
\ref{problemlim x to 0 (sin x - x)/(arctan x - x)}
\noindent \textbf{Solution I.} 
\[
\begin{array}{rcll|l}
\displaystyle \lim\limits_{x\to 0}\frac{\sin x- x}{\Arctan x -x}&=&\displaystyle \lim\limits_{x\to 0} \frac{\cos x -1 }{ \frac{1}{1+x^2}-1 } &&\text{L'Hospital rule}\\
&=&\displaystyle \lim\limits_{x\to 0}\frac{-\sin x }{\frac{ -2x}{(1+x^2)^2} }&& \text{L'Hospital rule again}\\
&=&\displaystyle \lim\limits_{x\to 0} \frac{(1+x^2)^2}{2}\frac{\sin x}{x}  \\
&=&\displaystyle \lim\limits_{x\to 0} \frac{(1+x^2)^2}{2}\lim\limits_{x\to 0}\frac{\sin x}{x} \\
&=&\displaystyle \frac{1}{2}\quad .
\end{array}
\] 

\noindent \textbf{Solution II.}
\[
\begin{array}{rcll|l}
\displaystyle \lim\limits_{x\to 0}\frac{\sin x- x}{\Arctan x -x}&=&\displaystyle \lim\limits_{x\to 0} \frac{\left( x-\frac{x^3}{3!}+\frac{x^5}{5!}-\dots\right) -x}{\left( x-\frac{x^3}{3}+\frac{x^5}{5}-\dots\right)-x} &&\text{use the Maclaurin series of }\sin, \Arctan \\
&=&\displaystyle \lim\limits_{x\to 0}\frac{- \frac{x^3 }{6} + x^5\left(\frac{1}{5!}-\dots\right) }{- \frac{ x^3}{3} + x^5 \left(\frac{1}{5}-\dots \right)  }&& \begin{array}{rcl}
\text{The expressions in parenthesis }\\
\text{are continous functions in x }
\end{array}\\
&=&\displaystyle \lim\limits_{x\to 0} \frac{-\frac{1}{6}+ x^2 \left(\frac{1}{5!}-\dots\right) }{- \frac{1}{3} +x^2 \left( \frac{1}{5}-\dots \right)  }\\
&=&\displaystyle \frac{-\frac{1}{6}+0}{\frac{1}{3}+0}\\
&=&\displaystyle \frac{1}{2}\quad .
\end{array}
\] 
}

\solution{\ref{problemlimxtoinftysin(2/x)}.
\[
\begin{array}{rcll|l}
\displaystyle \lim_{x\to\infty }x\sin\left( \frac{2}{x} \right)& =&\displaystyle \lim_{x \to \infty} \frac{\sin \left(\frac{2}{x}\right) }{ \frac{ 1}{x}}
 &&\begin{array}{l}
\text{indeterminate form }\\
\text{Use L'Hospital's rule}
\end{array}\\
&=&\displaystyle \lim_{x \to \infty} \frac{\cos\left(\frac{2}{x}\right) \left (-\frac{2}{x^2}\right)}{-\frac{1}{x^2}} \\ 
&=&\displaystyle \lim_{x \to \infty} 2 \cos\left(\frac{x}{2}\right) \\
&=&\displaystyle 2\quad .
\end{array}
\]
}



\begin{problem}
Determine whether the integral is convergent or divergent. Motivate your answer. The answer key has not been proofread, use with caution.

\begin{multicols}{2}
\begin{enumerate}[ref={\fcProblemRef}]
\item $\displaystyle \int\limits_{2}^{\infty}\frac{1}{(x-1)^{\frac32}} \diff x$.
\answer{convergent}
\item $\displaystyle \int\limits_{-1}^{1}\frac{1}{\sqrt[5]{1+x}} \diff x$.
\answer{convergent}
\item $\displaystyle \int\limits_{1}^{\infty }\frac{1}{\sqrt[5]{1+x}} \diff x$.
\answer{divergent}
\item $\displaystyle \int\limits_{-1}^{\infty }\frac{1}{\sqrt[5]{1+x}} \diff x$.
\answer{divergent}
\item $\displaystyle \int\limits_{-\infty }^{0}\frac{1}{2-3x} \diff x$.
\answer{divergent}
\item $\displaystyle \int\limits_{-\infty }^{0}\frac{1}{(2-3x)^{2}} \diff x$.
\answer{convergent}
\item $\displaystyle \int\limits_{-\infty }^{0}\frac{1}{(2-3x)^{1.00000001}} \diff x$.
\answer{convergent}
\item $\displaystyle \int\limits_{-2}^{ \frac{1}{ 2}} \frac{1}{2x-1} \diff x$.
\answer{divergent}
\item $\displaystyle \int\limits_{-5}^{\infty} e^{-3x} \diff x$.
\answer{convergent}
\item $\displaystyle \int\limits_{-\infty }^{5}  2^x \diff x$.
\answer{convergent}
\item $\displaystyle \int\limits_{-\infty }^{\infty}x^3 \diff x$.
\answer{divergent}
\item  $\displaystyle \int\limits_{-\infty}^{\infty} x e^{-x^2} \diff x$.
\answer{convergent}
\item \label{problemConvergencesqrt(x)e^-sqrt(x)zerotoinfty} $\displaystyle \int\limits_{0}^{\infty} \sqrt{x} e^{-\sqrt{x}} \diff x$.
\answer{convergent}
\item $\displaystyle \int\limits_{0}^{\infty}\sin^2 x \diff x$.
\answer{divergent}
\item $\displaystyle \int\limits_{0}^{5}\frac{1}{x^2+x-2} \diff x$.
\answer{divergent}
\item $\displaystyle \int\limits_{0}^{\infty}\frac{1}{x^2+x+1} \diff x$.
\answer{convergent}
\item $\displaystyle \int\limits_{2}^{\infty}\frac{1}{x^2-x-1} \diff x$.
\answer{convergent}
\item $\displaystyle \int\limits_{0}^{\infty}\frac{1}{x^2-x-1} \diff x$.
\answer{divergent}
\item \label{problemConvergencex^2/(x^4+2)from-inftyto+infty}

$\displaystyle \int\limits_{-\infty}^{\infty} \frac{x^2}{x^4+2} \diff x$.
\answer{convergent}
\end{enumerate}
\end{multicols}


\end{problem}
\solution{\ref{problemConvergencex^2/(x^4+2)from-inftyto+infty}
The integrand is a rational function and therefore we can solve this problem by finding the indefinite integral and then computing the limit. We would need to start by factoring $x^4+2$ into irreducible quadratic factors - that is already quite laborious:
\[
x^4+2= \left(x^2+\sqrt[4]{8}x+\sqrt{2} \right)\left(x^2-\sqrt[4]{8}x+\sqrt{2}\right)\quad.
\]
The problem asks us only to establish the convergence of the integral; it does not ask us to compute its actual numerical value. Therefore we can give a much simpler solution. The function is even and therefore it suffices to show that $\displaystyle \int\limits_0^\infty \frac{x^2}{x^4+2}\diff x$ is convergent.

We have that 
\[
\int\limits_{0}^\infty \frac{x^2}{x^4+2}\diff x= \int\limits_{0}^1 \frac{x^2}{ x^4 +2 }\diff x+\int\limits_{1}^\infty \frac{x^2}{x^4+2}\diff x\quad.
\]
The function $ \frac{x^2}{x^4+2}$ is continuous so $\int\limits_{0}^1 \frac{x^2}{x^4+2}\diff x$ integrates to a number, which does not affect the convergence of the above expression. Therefore the convergence of our integral is governed by the convergence of $\int\limits_{1}^\infty \frac{x^2}{x^4+2}\diff x$. To establish that that integral is convergent, we use the comparison theorem as follows.
\[
\begin{array}{rcll|l}
\displaystyle
\int\limits_{1}^\infty \frac{x^2}{x^4+2}\diff x&\leq & \displaystyle \int \limits_{ 1}^\infty \frac{x^2}{x^4}\diff x && \begin{array}{l} \text{we have that } x^4+2>x^4\\ \text{and therefore }\displaystyle \frac{x^2}{x^4+2}\leq \frac{x^2}{x^4}\end{array}\\
&=&\displaystyle \int_1^{\infty} x^{-2}\diff x\\
&=&\displaystyle \lim\limits_{t\to \infty} \left[ -\frac{1}{x}\right]_{1}^{t}\\
&=&\displaystyle  \lim\limits_{t\to\infty} 1- \frac{1}{t}\\
&=&\displaystyle  1
\end{array}
\]
We have that $\displaystyle \frac{x^2}{x^4+2}\geq 0 $ and by the preceding computation we have that $\displaystyle\int_{1}^\infty \frac{x^2}{x^4+2} \diff x \leq 1$. Therefore by the comparison theorem $\displaystyle\int_{1}^\infty \frac{ x^2}{x^4+2} \diff x $ is convergent.
}

\solution{\ref{problemConvergencesqrt(x)e^-sqrt(x)zerotoinfty}
It is possible to show that this integral is convergent by using the comparison theorem. However, we shall use direct integration instead. First, we solve the indefinite integral:

\[
\begin{array}{rcll|l}
\displaystyle \int \sqrt{x} e^{-\sqrt{x}} \diff x&=& \displaystyle \int \sqrt{x} e^{-\sqrt{x}} \frac{2\sqrt{x} \diff x}{2\sqrt{x}} &&\text{use }  \diff \sqrt{x} = \frac{\diff x}{2\sqrt{x}} \\
&=& \displaystyle \int \sqrt{x} e^{-\sqrt{x}} \left(2\sqrt{x} \diff \sqrt{x}\right) &&\text{Set } \sqrt{x}=u\\
&=&\displaystyle 2\int u^2 e^{-u}\diff u\\
&=&\displaystyle 2\left( -\int u^2 \diff \left(e^{-u}\right) \right) &&\text{integrate by parts} \\
&=&\displaystyle 2\left(- u^2e^{-u}+\int e^{-u}\diff \left(u^2\right) \right) \\
&=&\displaystyle 2\left(- u^2e^{-u}+\int 2 u e^{-u}\diff u \right)\\
&=&\displaystyle 2\left(- u^2e^{-u}-\int 2 u \diff e^{-u} \right) &&\text{integrate by parts again}\\
&=&\displaystyle 2\left(- u^2e^{-u}- 2 u  e^{-u}+ \int 2e^{-u}\diff u \right) \\
&=&\displaystyle 2\left( - u^2e^{-u} -2ue^{-u} -2e^{-u}\right)+C\\
&=&\displaystyle 2\left( - xe^{-\sqrt{x}} -2\sqrt{x}e^{-\sqrt{x}} -2e^{-\sqrt{x}}\right)+C
\end{array}
\]
Therefore 
\[
\begin{array}{rcll|l}
\displaystyle \int \sqrt{x} e^{-\sqrt{x}} \diff x&=&\displaystyle\lim\limits_{t\to \infty} 2\left[ - xe^{-\sqrt{x}} -2\sqrt{x}e^{-\sqrt{x}} -2e^{-\sqrt{x}}\right]_{0}^{\infty}\\
&=&\displaystyle 4+ \lim\limits_{t\to \infty} 4\left( -te^{-\sqrt{t}} -\sqrt{t}e^{-\sqrt{t}}- e^{-\sqrt{t}} \right) && \text{Set }u=\sqrt{t}\\
&=&\displaystyle 4- 4\lim\limits_{u\to \infty} \left(u^2e^{-u} + ue^{-u}+e^{-u}\right)\\
&=& \displaystyle 4- 4\lim\limits_{u\to \infty} \frac{u^2+u+1}{e^u} &&\text{use L'Hospital's rule for limit, see below}\\
&=& 4\quad ,
\end{array}
\]
and the integral converges to $4$. In the above computation we used the following limit computation
\[
\begin{array}{rcll|l}
\displaystyle \lim\limits_{u\to \infty}\frac{u^2+u+1}{e^u}&=&\displaystyle  \lim\limits_{u\to \infty} \frac{2u+1}{e^u}&&\text{Apply L'Hospital's rule}\\
&=&\displaystyle \lim\limits_{u\to \infty} \frac{2}{e^u}\\
&=&\displaystyle 0\quad .
\end{array}
\]

}
\begin{problem}
Plot the curve. Set up an integral that expresses its length. Find the length of the curve. 
\begin{enumerate}[ref={\fcProblemRef}]
\item $y=\sqrt{x}$, $x\in [1, 2]$.

\answer{$L=$}
\item $y=x^2$, $x\in [1, 2]$.

\answer{$L=$}
\item 
$\gamma:\left| 
\begin{array}{rcl}
x(t)&=&\frac{1}{t}+\frac{t^3}{3}\\
y(t)&=&2t\\
\end{array}\right., t\in [1,2]\quad . $
\item \label{problemlengthx=sqrt(t)-2t,y=8/3t^(3/4)} $\displaystyle x = \sqrt{t} - 2t$ and $\displaystyle y = \frac{8}{3}t^{\frac{3}{4}}$ from $t = 1$ to $t = 4$.

\answer{$L=7$}
\item $\gamma:\left| 
\begin{array}{rcl}
x(t)&=&\frac{1}{t}+t\\
y(t)&=&2\ln t\\
\end{array}\right., t\in [1,2]\quad . $

\answer{$L=$}
\end{enumerate}


\end{problem}
\solution{\ref{problemlengthy=sqrt(x)from1to2}
The length of the parametric curve is given by
\[
\begin{array}{rcll|l}
L&=&\displaystyle \int_{1}^{2}\sqrt{1+\left(\frac{\diff y}{\diff x}\right)^2 }\diff x\\
&=&\displaystyle \int_{1}^{2}\sqrt{1+\left(\frac{1}{2\sqrt{x}}\right)^2 }\diff x\\
&=&\displaystyle \int_{x=1}^{x=2} \sqrt{1 +\frac{1}{4x}}\diff x &&\begin{array}{rcl}\text{Substitute }4x&=&u\\\diff x&=&\frac{1}{4}\diff u\\ \end{array}\\
&=&\displaystyle \int_{u=4}^{u=8} \sqrt{1 +\frac{1}{u}}\left(\frac{1}{4}\diff u\right)\\
&=&\displaystyle \frac{1}{4}\int_{4}^{8}\sqrt{\frac{u+1}{u}}\diff u\\
&=&\displaystyle \frac{1}{4}\int_{4}^{8}\sqrt{\frac{u(u+1)}{u^2}}\diff u\\
&=&\displaystyle \frac{1}{4}\int_{4}^{8}\frac{\sqrt{u^2+u }}{u}\diff u\\
&=&\displaystyle \frac{1}{4}\int_{4}^{8}\frac{\sqrt{u^2+u+\frac{1}{4}-\frac{1}{4} }}{u}\diff u\\
&=&\displaystyle \frac{1}{4}\int_{4}^{8}\frac{\sqrt{\left(u+\frac{1}{2}\right)^2-\frac{1}{4} }}{u}\diff u\\
&=&\displaystyle \frac{1}{4}\int_{4}^{8}\frac{ \sqrt{\frac{1}{4}\left( \left(2u+1\right)^2-1\right) }}{u}\diff u\\
&=&\displaystyle \frac{1}{8}\int_{u=4}^{u=8}\frac{ \sqrt{ \left(2u+1\right)^2-1 }}{u}\diff u &&\begin{array}{rcl}\text{Substitute }2u+1&=&z\\ u&=&\frac{z-1}{2}\\\diff u&=&\frac{1}{2}\diff z \end{array}\\ 
&=&\displaystyle \frac{1}{8}\int_{z=9}^{z=17}\frac{ \sqrt{ z^2-1 }}{\frac{z-1}{2}}\frac{1}{2}\diff z \\
&=&\displaystyle \frac{1}{8}\int_{z=9}^{z=17}\frac{ \sqrt{ z^2-1 }}{z-1}\diff z&&  \begin{array}{rcl}
\text{Trig. subst.: }z&=&\sec\theta \\ 
\sqrt{z^2-1}&=&\tan \theta \\
\diff z &= &\tan \theta\sec\theta \diff \theta
\end{array}\\
&=&\displaystyle \frac{1}{8}\int_{\theta=\Arcsec(9) }^{\theta=\Arcsec(17)}\frac{\tan \theta}{\sec \theta - 1} \sec \theta \tan \theta \diff \theta  && 
\begin{array}{rcl}
\text{Set }\alpha&=&\Arcsec(9)\\ 
\text{Set }\beta&=&\Arcsec(17)\\ 
\end{array}
\\
&=&\displaystyle \frac{1}{8}\int_{\alpha }^{\beta}\frac{\tan^2 \theta}{\sec \theta - 1} \sec \theta \diff \theta && \text{Use} \tan^2\theta=\sec^2\theta-1\\
&=&\displaystyle \frac{1}{8}\int_{\alpha }^{\beta}\frac{\sec^2 \theta-1}{\sec \theta - 1} \sec \theta \diff \theta \\
&=&\displaystyle \frac{1}{8}\int_{\alpha }^{\beta}\frac{(\cancel{ \sec\theta-1})(\sec \theta+1)}{\cancel{\sec \theta - 1}} \sec \theta \diff \theta \\
&=&\displaystyle \frac{1}{8}\int_{\alpha }^{\beta}\left(\sec^2\theta+ \sec \theta \right)\diff \theta && \begin{array}{l}\displaystyle\int\sec \theta\diff \theta= \ln\left|\sec\theta+\tan\theta\right|+C\\ \text{previously studied}\end{array}\\
&=&\displaystyle \frac{1}{8}\left[\tan \theta +\ln \left|\sec\theta+\tan\theta\right| \right]_{\alpha}^{\beta} && \begin{array}{rlc}\tan \theta &=& \sqrt{\sec^2\theta-1}, \theta\in\left[0,\frac{\pi}{2}\right) \\ \tan \alpha&=&\sqrt{9^2-1}=4\sqrt{5}\\ 
\tan \beta&=& \sqrt{17^2-1}=12\sqrt{2} \end{array}  \\
&=& \frac{1}{8}\left( 12\sqrt{2} +\ln (17+12\sqrt{2})-4\sqrt{5}-\ln (9+4\sqrt{5})\right) \approx 1.083
\end{array}
\]

}

\solution{\ref{problemlengthy=x^2from1to2}
The length of the parametric curve is given by
\[
\begin{array}{rcll|l}
L&=&\displaystyle \int_{1}^{2}\sqrt{1+\left(\frac{\diff y}{\diff x}\right)^2 }\diff x\\
&=&\displaystyle \int_{x=1}^{x=2}\sqrt{1+4x^2 }\diff x &&\begin{array}{rcl}\text{Substitute }2x&=&u\\\diff x&=&\frac{1}{2}\diff u\\ \end{array}\\
&=&\displaystyle \int_{u=2}^{u=4} \sqrt{u^2+1}\left(\frac{1}{2}\diff u\right)\\
&=&\displaystyle \frac{1}{2}\int_{u=2}^{u=4} \sqrt{u^2+1}\diff u&& 
\begin{array}{l}\displaystyle
\int \sqrt{u^2+1}\diff u \\
= \displaystyle \frac{1}{2}\left(u\sqrt{u^2+1}+\ln\left(u+\sqrt{u^2+1}\right) \right)+C\\
\text{previously studied}
\end{array}
\\
&=&\frac{1}{4} \left[ u\sqrt{u^2+1}+\ln\left(u+\sqrt{u^2+1}\right)\right]_{2}^4\\
&=&\sqrt{17}+\frac{1}{4} \log{}\left(\sqrt{17}+4\right)-\frac{1}{4} \log{}\left(\sqrt{5}+2\right)-\frac{\sqrt{5}}{2} \\
&\approx& 3.167841
\end{array}
\]

}

\solution{ \ref{problemlengthx=sqrt(t)-2t,y=8/3t^(3/4)}.
The length of the parametric curve is given by
\[
L={\displaystyle \int_{1}^4 \sqrt{\left(\frac{\diff x}{\diff t}\right)^2 + \left( \frac{\diff y}{\diff t} \right)^2}  \diff t}\quad .
\]
We have that 
\[
\begin{array}{rclll}
\displaystyle \frac{\diff x}{\diff t} &=&\displaystyle  \frac{1}{2\sqrt{t}} - 2\\
\displaystyle \frac{\diff y}{\diff t} &=&\displaystyle  2t^{-\frac{1}4}\\
\displaystyle \left(\frac{\diff x}{\diff t}\right)^2 &=&\displaystyle  \frac{1}{4t} - \frac{2}{\sqrt{t}} + 4\\
\displaystyle \left(\frac{\diff y}{\diff t}\right)^2 &=&\displaystyle  4t^{-\frac{1}{2}} = \frac{4}{\sqrt{t}}\\
\displaystyle \left(\frac{\diff x}{\diff t}\right)^2+\left(\frac{\diff y}{\diff t}\right)^2 & =&\displaystyle  \frac{1}{4t} + 2\frac{1}{\sqrt{t}} + 4 = \left(\frac{1}{2\sqrt{t}} + 2\right)^2\quad .
\end{array}
\]

$\frac{1}{2\sqrt{t}} +2$ is positive and $\sqrt{\left(\frac{ 1}{2 \sqrt{t}} +2\right)^2} =\frac{1}{2\sqrt{t}} +2$. So the integral becomes 
\[\displaystyle 
L= \int_1^4 \left(\frac{1}{2\sqrt{t}} +2\right)  \diff t=\left[\sqrt{t} + 2t\right]_{t=1}^{t=4}=(2+8)-(1+2)=7\quad .
\]
}


\begin{problem}
\begin{enumerate}[ref={\fcProblemRef}]
\item \label{problem-Area-swept-by-r=1+sin2theta} The curve given in polar coordinates by $r=1+\sin 2\theta$ is plotted below by computer. Find the area lying outside of this curve and inside of the circle $x^2+y^2=1$.

\psset{xunit=1cm, yunit=1cm}
\begin{pspicture}(-2.016386, -2.016424)(2.016407,2.116335)
\tiny
\fcAxesStandard{-1.766386}{-1.766424}{1.766407}{1.766335}
%Calculator command: drawPolar{}(\sin{}(2 t)+1, 0, 2 \pi)
\parametricplot[linecolor=\fcColorGraph, plotpoints=1000, algebraic=false]{0}{6.28319}{ 1 t 2 mul 57.29578 mul sin add t 57.29578 mul cos mul 1 t 2 mul 57.29578 mul sin add t 57.29578 mul sin mul }
\end{pspicture}

\answer{$a=2-\frac{\pi}{4}$}
\item \label{problem-Area-swept-by-r=cos2theta} The curve given in polar coordinates by $r=\cos (2\theta)$ is plotted below by computer. Find the area lying inside the curve and outside of the circle $x^2+y^2=\frac14$.

\begin{pspicture}(-1.399902, -1.399975)(1.4,1.499975)
\tiny
\fcAxesStandard{-1.149902}{-1.149975}{1.15}{1.149975}
%Calculator command: drawPolar{}(\cos{}(2 t), 0, 2 \pi)
\parametricplot[linecolor=\fcColorGraph, plotpoints=1000, algebraic=false]{0}{6.28319}{t 2 mul 57.29578 mul cos t 57.29578 mul cos mul t 2 mul 57.29578 mul cos t 57.29578 mul sin mul }
\end{pspicture}


\answer{$\frac{\pi}{6}+\frac{\sqrt{3}}{4} $}
\item \label{problem-Area-swept-byr=sin2theta_outsider=1/2} Below is a computer generated plot of the curve $r=\sin(2\theta)$. Find the area locked inside one petal of the curve and outside of the circle $\displaystyle x^2+y^2=\frac{1}{4}$.

\psset{xunit=1cm, yunit=1cm, algebraic=false}
\begin{pspicture}(-1.2, -1.2)(1.2, 1.2)
\tiny
\psaxes[labels=none, ticks=none, arrows=->](0,0)(-1.5,-1.5)(1.5,1.5)
\pscustom*[linecolor=cyan]{
\parametricplot[linecolor=red, plotpoints=300]{15}{75}{t cos t 2 mul sin mul t sin t 2 mul sin mul}
\parametricplot[linecolor=\fcColorGraph, algebraic=false]{75}{15}{t cos 0.5 mul  t sin 0.5 mul}
}
\parametricplot[linecolor=red, plotpoints=300]{0}{360}{t cos t 2 mul sin mul t sin t 2 mul sin mul}
\parametricplot[linecolor=red, plotpoints=300]{0}{360}{t cos 0.5 mul  t sin 0.5 mul}
\end{pspicture}

\answer{$\frac{\pi}{24} +\frac{\sqrt{3}}{16}$}
%\item The curve given in polar coordinates by $r=1+\cos (3t)$ is plotted below. Find the area \textbf{outside} of the curve and \textbf{inside} the circle $x^2+y^2=\frac14$.

%\begin{pspicture}(-1.611991, -2.182513)(2.4,2.282513) \tiny \fcAxesStandard{-1.361991}{-1.932513}{2.15}{1.932513}
%Calculator command: drawPolar{}(\cos{}(3 t)+1, 0, 2 \pi)
%\parametricplot[linecolor=\fcColorGraph, plotpoints=1000, algebraic=false]{0}{6.28319}{ 1 t 3 mul 57.29578 mul cos add t 57.29578 mul cos mul 1 t 3 mul 57.29578 mul cos add t 57.29578 mul sin mul }
%\end{pspicture}

\end{enumerate}



\end{problem}
\solution{\ref{problem-Area-swept-by-r=1+sin2theta}. A computer generated plot of the two curves is included below. The circle $x^2+y^2=1$ has one-to-one polar representation given by $r=1, \theta\in [0,2\pi)$. Except the origin, which is traversed four times by the curve $r=1+\sin (2\theta)$, the second curve is in a one-to-one correspondence with points in the $r,\theta$-plane given by the equation $r=1+\sin (2\theta), \theta\in [0,2\pi)$. Since the two curves do not meet in the origin, we may conclude that the two curves may intersect only when their values for $r$ and $\theta$ coincide. Therefore we have an intersection when
\[\begin{array}{rcll|l}
1+\sin (2\theta)&=&1\\
\sin (2\theta)&=&0\\
\theta &=& 0,\frac{\pi}{2}, \pi, \frac{3\pi}{2}&&\text{because } \theta\in [0,2\pi) \\
\end{array}
\]
Therefore the two curves meet in the points $(0,1)(-1,0)$ and $(0,-1),(1,0)$.

Denote the investigated region by $A$. From the computer-generated plot, it is clear that when a point has polar coordinates $\theta\in [\frac{\pi}{2}, \pi] \cup[\frac{3\pi}{2}, 2\pi]$, $r\in [1+\sin(2\theta),1]$ it lies in $A$. Furthermore, the points $r,\theta$ lying in the above intervals are in one-to-one correspondence with the points in $A$.

Suppose we have a curve $r=f(\theta), \theta\in [a,b]$ for which no two points lie on the same ray from the origin. Recall from theory that the area swept by that curve is given by
\[
\int\limits_{a}^b\frac{1}{2} f^2(\theta)\diff \theta\quad .
\]

Therefore the area $a$ of $A$ is computed via the integrals
\[
\begin{array}{rcll|l}
a&=&\displaystyle \int\limits_{\frac{\pi}{2}}^{\pi} \frac{1}{2} \left( {\underbrace{ 1}_{\text{outer curve}}}^2- \left(\underbrace{1+\sin(2\theta)}_{\text{inner curve}}\right)^2 \right)\diff \theta + \int \limits_{ \frac{3\pi}{2}}^{ 2\pi} \frac{1}{2} \left(1^2- (1+\sin(2\theta) )^2 \right) \diff \theta &&\text{use the symmetry of } A\\
&=&\displaystyle  \int\limits_{\frac{\pi}{2}}^{\pi} \left(1^2-(1+\sin(2\theta))^2\right)\diff \theta= \int\limits_{\frac{\pi}{2}}^{\pi} \left( - 2\sin(2\theta) - \sin^2(2\theta)\right) \diff \theta  &&\text{use } \sin^2 z=\frac{1-\cos (2z)}{2} \\
&=&\displaystyle  \int\limits_{\frac{\pi}{2}}^{\pi}  \left( -2\sin(2\theta) -\frac{1}{2} +\frac{1}{2}\cos (4\theta)\right)\diff \theta = \left[\cos (2\theta) -\frac{1}{2}\theta -\frac{1}{8}\sin (4\theta) \right]_{\frac{\pi}{2}}^{\pi} \\
&=&2-\frac{\pi}{4}\quad .
\end{array}
\]

\psset{xunit=1cm, yunit=1cm}
\begin{pspicture}(-2.016386, -2.016424)(2.016407,2.116335)
\tiny
\fcAxesStandard{-1.766386}{-1.766424}{1.766407}{1.766335}
\pscustom*[linecolor=\fcColorAreaUnderGraph]{
%Calculator command: drawPolar{}(1, 1/2 \pi, \pi)
\parametricplot[linecolor=\fcColorGraph, plotpoints=1000, algebraic=false]{1.5708}{3.14159}{ 1 t 57.29578 mul cos mul 1 t 57.29578 mul sin mul }
%Calculator command: drawPolar{}(\sin{}(2 t)+1, 1/2 \pi, \pi)
\parametricplot[linecolor=\fcColorGraph, plotpoints=1000, algebraic=false]{1.5708}{3.14159}{ 1 t 2 mul 57.29578 mul sin add t 57.29578 mul cos mul 1 t 2 mul 57.29578 mul sin add t 57.29578 mul sin mul }
} %pscustom
\pscustom*[linecolor=\fcColorAreaUnderGraph]{
%Calculator command: drawPolar{}(\sin{}(2 t)+1, -1/2 \pi, 0)
\parametricplot[linecolor=\fcColorGraph, plotpoints=1000, algebraic=false]{-1.5708}{0}{ 1 t 2 mul 57.29578 mul sin add t 57.29578 mul cos mul 1 t 2 mul 57.29578 mul sin add t 57.29578 mul sin mul }
%Calculator command: drawPolar{}(1, -1/2 \pi, 0)
\parametricplot[linecolor=\fcColorGraph, plotpoints=1000, algebraic=false]{-1.5708}{0}{ 1 t 57.29578 mul cos mul 1 t 57.29578 mul sin mul }
} %pscustom

%Calculator command: drawPolar{}(1, 0, 2 \pi)
\parametricplot[linecolor=\fcColorGraph, plotpoints=1000, algebraic=false]{0}{6.28319}{ 1 t 57.29578 mul cos mul 1 t 57.29578 mul sin mul }
%Calculator command: drawPolar{}(\sin{}(2 t)+1, 0, 2 \pi)
\parametricplot[linecolor=\fcColorGraph, plotpoints=1000, algebraic=false]{0}{6.28319}{ 1 t 2 mul 57.29578 mul sin add t 57.29578 mul cos mul 1 t 2 mul 57.29578 mul sin add t 57.29578 mul sin mul }
\end{pspicture}
}
\solution{\ref{problem-Area-swept-by-r=cos2theta} A computer generated plot of the figure is included below. The circle $x^2+y^2=\frac{1}{4} $ is centered at $0$ and of radius $\frac{1}{2}$ and therefore can be parametrized in polar coordinates via $r=\frac{1}{2}, \theta\in [0, 2\pi]$.

Points with polar coordinates $(r_1, \theta_1) $ and $(r_2,\theta_2)$ coincide if one of the three holds:
\begin{itemize}
\item[$\bullet$] $r_1=r_2\neq 0$ and $\theta_1=\theta_2+2k\pi, k\in \mathbb Z $,
\item[$\bullet$] $r_1=-r_2\neq 0$ and $\theta_1=\theta_2+(2k+1)\pi, k\in \mathbb Z$,
\item[$\bullet$] $r_1=r_2=0 $ and $\theta$ is arbitrary.
\end{itemize}
To find the intersection points of the two curves we have to explore each of the cases above. The third case is not possible as the circle does not pass through the origin. Suppose we are in the first case. Then the value of $r$ (as a function of $\theta$)  is equal for the two curves. Thus the two curves intersect if
\[
\begin{array}{rcll|l}
r=\cos (2\theta)&=&\frac12\\
2\theta&=& \pm\frac{\pi}{3}+2k\pi&&\text{where }k\in \mathbb Z\\
\theta &=& \pm\frac{\pi}{6}+k\pi &&\text{where }k\in \mathbb Z\\
\theta &=& \frac{\pi}{6}, \frac{\pi}{6}+\pi, -\frac{\pi }{6}+\pi, -\frac{\pi }{6}+2\pi &&\text{all other values discarded as }\theta\in [0,2\pi]\\
\theta&=&\frac{\pi}{6}, \frac{7\pi}{6}, \frac{5\pi}{6}, \frac{11\pi}{6}
\end{array}
\]
This gives us only four intersection points, and the computer-generated plot shows eight. Therefore the second case must yield new intersection points: the two curves intersect also when
\[
\begin{array}{rcll|l}
r=\cos (2\theta)&=&-\frac{1}{2}\\
2\theta &=& \pm \frac{2\pi}{3} +2k\pi &&\text{where } k\in \mathbb Z\\
\theta &=& \pm \frac{\pi}{3} +k\pi &&\text{where } k\in \mathbb Z\\
\theta&=& \frac{\pi }{3}, \frac{\pi}{3}+\pi, \frac{-\pi}{3} +\pi, \frac{-\pi}{3}+2\pi &&\text{all other values are discarded as }\theta \in [0,2\pi]\\
\theta&=&\frac{\pi}{3}, \frac{4\pi}3, \frac{2\pi}{3}, \frac{5\pi}{3}  \quad .
\end{array}
\]
From the computer-generated plot below, we can see that the area we are looking for is 4 times the area locked between the two curves for $\theta\in \left[\frac{-\pi}{6}, \frac{\pi}{6}\right] $. Therefore the area we are looking for is given by
\[
4\int\limits_{-\frac{\pi}{6}}^{\frac{\pi}{6}} \frac{1}{2}\left(\cos^2(2\theta)-\left(\frac{1}{2}\right)^2 \right)\diff \theta\quad .
\]
We leave the above integral to the reader.
\psset{xunit=2cm, yunit=2cm}
\begin{pspicture}(-1.399902, -1.399975)(1.4,1.499975)
\tiny
\pscustom*[linecolor=\fcColorAreaUnderGraph]{
%Calculator command: drawPolar{}(1/2, 1/6 \pi, -1/6 \pi)
\parametricplot[linecolor=\fcColorGraph, plotpoints=1000, algebraic=false]{0.523599}{-0.523599}{ 0.5 t 57.29578 mul cos mul 0.5 t 57.29578 mul sin mul }
%Calculator command: drawPolar{}(\cos{}(2 t), -1/6 \pi, 1/6 \pi)
\parametricplot[linecolor=\fcColorGraph, plotpoints=1000, algebraic=false]{-0.523599}{0.523599}{t 2 mul 57.29578 mul cos t 57.29578 mul cos mul t 2 mul 57.29578 mul cos t 57.29578 mul sin mul }
}
\pscustom*[linecolor=\fcColorAreaUnderGraph]{
%Calculator command: drawPolar{}(1/2, 5/3 \pi, 4/3 \pi)
\parametricplot[linecolor=\fcColorGraph, plotpoints=1000, algebraic=false]{5.23599}{4.18879}{ 0.5 t 57.29578 mul cos mul 0.5 t 57.29578 mul sin mul }
%Calculator command: drawPolar{}(\cos{}(2 t), 1/3 \pi, 2/3 \pi)
\parametricplot[linecolor=\fcColorGraph, plotpoints=1000, algebraic=false]{1.0472}{2.0944}{t 2 mul 57.29578 mul cos t 57.29578 mul cos mul t 2 mul 57.29578 mul cos t 57.29578 mul sin mul }
}
\pscustom*[linecolor=\fcColorAreaUnderGraph]{
%Calculator command: drawPolar{}(1/2, 7/6 \pi, 5/6 \pi)
\parametricplot[linecolor=\fcColorGraph, plotpoints=1000, algebraic=false]{3.66519}{2.61799}{ 0.5 t 57.29578 mul cos mul 0.5 t 57.29578 mul sin mul }
%Calculator command: drawPolar{}(\cos{}(2 t), 5/6 \pi, 7/6 \pi)
\parametricplot[linecolor=\fcColorGraph, plotpoints=1000, algebraic=false]{2.61799}{3.66519}{t 2 mul 57.29578 mul cos t 57.29578 mul cos mul t 2 mul 57.29578 mul cos t 57.29578 mul sin mul }
}
\pscustom*[linecolor=\fcColorAreaUnderGraph]{
%Calculator command: drawPolar{}(1/2, 2/3 \pi, 1/3 \pi)
\parametricplot[linecolor=\fcColorGraph, plotpoints=1000, algebraic=false]{2.0944}{1.0472}{ 0.5 t 57.29578 mul cos mul 0.5 t 57.29578 mul sin mul }
%Calculator command: drawPolar{}(\cos{}(2 t), 4/3 \pi, 5/3 \pi)
\parametricplot[linecolor=\fcColorGraph, plotpoints=1000, algebraic=false]{4.18879}{5.23599}{t 2 mul 57.29578 mul cos t 57.29578 mul cos mul t 2 mul 57.29578 mul cos t 57.29578 mul sin mul }
}
\parametricplot[linecolor=\fcColorGraph, plotpoints=1000, algebraic=false]{0}{6.28319}{ 0.5 t 57.29578 mul cos mul 0.5 t 57.29578 mul sin mul }
%Calculator command: drawPolar{}(\cos{}(2 t), 0, 2 \pi)
\parametricplot[linecolor=\fcColorGraph, plotpoints=1000, algebraic=false]{0}{6.28319}{t 2 mul 57.29578 mul cos t 57.29578 mul cos mul t 2 mul 57.29578 mul cos t 57.29578 mul sin mul }
\psaxes[ticks=none, labels=none, arrows = <->](0,0)(-1.149902,-1.149975)(1.15,1.149975)
\fcLabels{1.15}{1.149975}
\end{pspicture}
%Calculator command: drawPolar{}(1/2, 0, 2 \pi)
}

\noindent \solution{\ref{problem-Area-swept-byr=sin2theta_outsider=1/2}. The circle $x^2+y^2=\frac{1}{4} $ is centered at $0$ and of radius $\frac{1}{2}$ and therefore can be parametrized in polar coordinates via $r=\frac{1}{2}, \theta\in [0, 2\pi)$.

Points with polar coordinates $(r_1, \theta_1) $ and $(r_2,\theta_2)$ coincide if one of the three holds:
\begin{itemize}
\item $r_1=r_2\neq 0$ and $\theta_1=\theta_2+2k\pi, k\in \mathbb Z $,
\item $r_1=-r_2\neq 0$ and $\theta_1=\theta_2+(2k+1)\pi, k\in \mathbb Z$,
\item $r_1=r_2=0 $ and $\theta$ is arbitrary.
\end{itemize}


To find the intersection points of the two curves we have to explore each of the cases above. The third case is not possible as the circle does not pass through the origin. Suppose we are in the first case. Then the value of $r$ (as a function of $\theta$)  is equal for the two curves. Thus the two curves intersect if
\[
\begin{array}{rcll|l}
r=\sin (2\theta)&=&\frac12\\
2\theta&=& \frac{\pi}{6}+2k\pi\text{ or } \frac{5\pi}{6}+2k\pi&&\text{where }k\in \mathbb Z\\
\theta &=& \frac{\pi}{12}+k\pi \text{ or } \frac{5\pi}{12} &&\text{where }k\in \mathbb Z\\
\theta &=& \frac{\pi}{12}, \frac{13\pi}{12}, \frac{5\pi}{12}, \frac{17\pi}{12}&&
\begin{array}{l}
\text{other values discarded as}\\
\theta\in [0,2\pi]
\end{array}
\end{array}
\]
This gives us only four intersection points, and the computer-generated plot shows eight. Therefore the second case must yield 4 new intersection points. However, from the figure we see there are only two intersection points that participate in the boundary of our area, and both of those were found above. Therefore we shall not find the remaining 4 intersections.

Both the areas locked by the petal and the area locked by the section of the circle are found by the formula for the area locked by a polar curve. Subtracting the two we get that the area we are looking for is: 
\[
\begin{array}{rcl}
\text{Area}&=&\displaystyle \int\limits_{\theta=\frac{\pi}{ 12}}^{\theta= \frac{5\pi}{12}} \frac{1}{2}\left( \sin^2(2\theta)- \left( \frac{1}{2} \right)^2 \right)\diff \theta\quad .\\
&=&\displaystyle \frac{1}{2} \int\limits_{\theta=\frac{\pi}{ 12}}^{\theta= \frac{5\pi}{12}} \left(\frac{1-\cos(4\theta)}{2}-\frac{1}{4} \right)\diff \theta\\
&=&\displaystyle\frac{1}{2}\left[\frac{1}{4}\theta -\frac{\sin(4\theta)}{8} \right]_{\theta=\frac{\pi}{ 12}}^{\theta= \frac{5\pi}{12}}\\
&=&\frac{1}{24} \pi+\frac{\sqrt{3}}{16} \quad .
\end{array}
\]
}




\begin{problem}
Determine if the sequence is convergent or divergent. If convergent, find the limit of the sequence.
\begin{multicols}{2}
\begin{enumerate}
\item $\displaystyle a_n=n$.

\answer{divergent}
\item \label{problemlim_n->infty2^n} $\displaystyle a_n=2^n$.

\answer{divergent}
\item $\displaystyle a_n=1.0001^n$.

\answer{divergent}
\item $\displaystyle a_n=0.999999^n$.

\answer{convergent, $\lim\limits_{n\to \infty} a_n=0$}
\item $\displaystyle a_n=n-\sqrt{n+1}\sqrt{n+2}$

\answer{convergent, $\lim_{n\to \infty} a_n=-\frac{3}{2}$}
\item $\displaystyle a_n=\frac{\ln n}{n}$.

\answer{convergent, $\lim_{n\to \infty} a_n=0$}
\item $\displaystyle a_n=\frac{\ln n}{\sqrt[10]{n}}$.

\answer{convergent, $\lim_{n\to \infty} a_n=0$}
\item $\displaystyle a_n=\frac{1}{n}$.

\answer{convergent, $\lim_{n\to \infty} a_n=0$}
\item $\displaystyle a_n=\frac{1}{n!}$.

\answer{convergent, $\lim_{n\to \infty} a_n=0$}
\item $\displaystyle a_n=\frac{n^n}{n!}$.

\answer{divergent}
\item $\displaystyle a_n=\cos n$.

\answer{divergent}
\item $\displaystyle a_n=\cos\left(\frac{1}{n}\right)$

\answer{convergent, $\lim_{n\to \infty} a_n=1$}
\item \label{problemlim_n->infty((n+1)/n)^n} $\displaystyle a_n= \left(\frac{n+1}{n}\right)^{n}$.

\answer{convergent, $\lim_{n\to \infty} a_n=e$}
\item \label{problemlim_n->infty((2n+1)/n)^n} $\displaystyle a_n= \left(\frac{2n+1}{n}\right)^{n}$.

\answer{divergent}
\item \label{problemlim_n->infty((n+1)/n)^(2n)} $\displaystyle a_n= \left(\frac{n+1}{n}\right)^{2n}$.

\answer{convergent, $\lim_{n\to \infty} a_n=e^2$}
\item \label{problemlim_n->infty((n+1)/(2n))^n} $\displaystyle a_n= \left(\frac{n+1}{2n}\right)^{n}$.

\answer{convergent, $\lim_{n\to \infty} a_n=0$}
\end{enumerate}
\end{multicols}
\end{problem}
\solution{\ref{problemlim_n->infty((n+1)/n)^n}. 

Consider $f(x)= \left( \frac{x+1}{x} \right)^x$, where $x$ is a positive number. We will now show that $\lim\limits_{x\to \infty}f(x)$ exists. Since the limit is of the form $1^{\infty}$, we will start by finding the limit of the logarithm $\ln (f(x))$. We will then exponentiate that limit to find the limit of $f(x)$.

\[
\begin{array}{rcll|l}
\displaystyle \lim\limits_{x\to \infty} \ln \left(\left(\frac{x+1}{x}\right)^x\right)&=& \displaystyle \lim\limits_{x\to \infty} x\ln \left(\frac{x+1}{x}\right)\\
&=&\displaystyle \lim\limits_{x\to \infty} \frac{\ln \left(\frac{x+1}{x}\right)}{\frac{1}{x}}\\
&=& \displaystyle \lim\limits_{x\to \infty} \frac{\ln \left(1+ \frac{1}{x}\right)}{\frac{1}{x}} && \begin{array}{l} \text{Form ``}\frac{0}{0}\text{''}\\\text{L'Hospital rule}\end{array}\\
&=&\displaystyle  \lim\limits_{x\to \infty}\frac{ \frac{1}{1+ \frac{1}{x}} \left(1+\frac{1}{x}\right)'}{-\frac{1}{x^2}} \\
&=&\displaystyle \lim\limits_{x\to \infty} \frac{\frac{1}{ \left(1+\frac{1}{x}\right) } \cancel{\left(-\frac{1}{x^2}\right)} }{ \cancel{ \left(-\frac{1}{x^2}\right)} } \\
&=&\displaystyle  \lim\limits_{x\to \infty}\frac{1}{1+\frac{1}{x}}\\
&=&1\\
\displaystyle \lim\limits_{x\to \infty} \left(\frac{x+1}{x}\right)^x&=&\displaystyle \lim\limits_{x\to \infty} e^{\ln \left(\left(\frac{x+1}{x}\right)^x\right)}&&\text{The exponent is continuous}\\
&=&\displaystyle  e^{\lim\limits_{x\to \infty} \ln \left(\left(\frac{x+1}{x}\right)^x\right)}\\
&=&\displaystyle e^{1} &&\text{use preceding}\\
&=&\displaystyle e\quad .
\end{array}
\] 
Therefore $\displaystyle \lim\limits_{\substack{n\to \infty\\ n-\text{ integer}}} \left(\frac{n+1}{n}\right)^n = \lim\limits_{\substack{x\to \infty\\ x-\text{ real}}} \left(\frac{x+1}{x}\right)^x=e$ and the sequence converges (to $e$).

}

\solution{ \ref{problemlim_n->infty((2n+1)/n)^n}. 

This problem can be solved in fashion similar to Problem \ref{problemlim_n->infty((n+1)/n)^n}. However there is a much simpler solution:

\[
\begin{array}{rcll|l}
\displaystyle \frac{2n+1}{n}&\geq& 2 &&\text{for }n>0\\
\displaystyle \lim\limits_{n\to\infty}\left(\frac{2n+1}{n}\right)^n&\geq& \displaystyle  \lim\limits_{n\to\infty}2^n &&\begin{array}{l}\text{limits respect non-strict inequalities}\\ \lim\limits_{n\to\infty}2^n \text{ computed in Problem \ref{problemlim_n->infty2^n}} \end{array}\\
\displaystyle \lim\limits_{n\to\infty}\left(\frac{2n+1}{n}\right)^n &=&\infty \quad .
\end{array}
\]

}


\begin{problem}
 Express the sum of the series as a rational number.
\begin{multicols}{2}
\begin{enumerate}[ref={\fcProblemRef}]
\item 
\label{problemSum(2^n+3^n)/(5^n)}
$
\displaystyle \sum\limits_{n=1}^{\infty} \frac{2^n+3^n}{5^n}
$

\answer{$\frac{13}{6}$}

\item \label{problemsumn=0^infty(2^n+5^n)/10^n}
$\displaystyle\sum_{n=0}^{\infty} \frac{2^n+5^n}{10^n}$

\answer{$\frac{13}{4}$}
 
\item \label{problemSum(3^n+5^n)/(7^n)}
$\displaystyle
\sum\limits_{n=1}^{\infty} \frac{5^n-3^n}{7^n}
$

\answer{$\frac{7}{4}$}
\item \label{sum_n=1^infty(3^(n+1)+7^(n-1))/21^n}
$\displaystyle
\sum_{n=1}^\infty \frac{3^{n+1}+7^{n-1}}{21^n}
$
\answer{$ \frac{4}{7}$}

\item \label{sum_n=0^infty(2^(n+1)+(-3)^(n-1))/5^n}
$\displaystyle
\sum_{n=0}^\infty \frac{2^{n+1}+(-3)^{n-1}}{5^n}
$

\answer{$ \frac{25}{8} $}
 
\end{enumerate}
\end{multicols} 
\end{problem}
\solution{\ref{problemSum(2^n+3^n)/(5^n)}.

\[
\begin{array}{rcll|l}
\displaystyle \sum\limits_{n=1}^{\infty} \frac{2^n+3^n}{5^n}&=&\displaystyle \sum\limits_{n=1}^{\infty} \left(\frac{2}{5}\right)^n
+\sum\limits_{n=1}^{\infty} \left(\frac{3}{5}\right)^n\\
&=&\displaystyle  \frac{2}{5}\sum\limits_{n=0}^{\infty} \left(\frac{2}{5} \right)^n+\frac{3}{5} \sum\limits_{n=0}^{ \infty} \left(\frac{3}{5}\right)^n&&
\begin{array}{l}
\text{Use geometric series sum f-la: }\\
\sum\limits_{n=0}^{\infty}r^n=\frac{1}{1-r},\\
\text{provided } |r|<1
\end{array}\\
&=&\displaystyle  \frac{2}{5}\cdot  \frac{1}{\left(1-\frac{2 }{5} \right)} +\frac{3}{5}\cdot  \frac{1}{ \left(1- \frac{3 }{5} \right)}\\
&=&\displaystyle \frac{13}{6}\quad .
\end{array}
\]
}

\solution{\ref{problemsumn=0^infty(2^n+5^n)/10^n}.
\[
\begin{array}{rcll|l}
\displaystyle \sum\limits_{n=0}^{\infty}\frac{2^n+5^n}{10^n}&=&\displaystyle  \sum \limits_{ n=0}^{\infty}\left(\frac{1}{5^n}+\frac{1}{2^n}\right)&&\text{use } \sum\limits_{ n= 0}^{\infty} r^n=\frac{1}{1-r}, \text{ for } |r|<1\\
&=&\displaystyle \frac{1}{1-\frac{1}{2}} +\frac{1}{1-\frac{1}{5}}\\
&=&\displaystyle \frac{13}{4}\quad .
\end{array}
\]
}

\solution{\ref{sum_n=1^infty(3^(n+1)+7^(n-1))/21^n}.
\[
\begin{array}{rcll|l}
\displaystyle \sum\limits_{n=1}^{\infty} \frac{3^{n+1}+ 7^{ n-1}}{21^n} &=& \displaystyle \sum \limits_{n=1}^{ \infty}\left(3 \cdot \frac{3^{n}}{21^n}+ \frac{ 1 }{7}\cdot \frac{ 7^n }{21^n}\right)\\
&=&\displaystyle 3\sum_{n =1}^{\infty} \left(\frac{1}{7} \right)^n + \frac{1 }{7} \sum_{n=1}^{\infty}  \left(\frac{1}{3} \right)^n\\
&=&\displaystyle \frac{3}{7} \sum_{n=0}^{\infty} \left( \frac{1}{7 } \right)^n +\frac{1 }{21} \sum_{ n=0}^{ \infty}\left(\frac{1}{3 } \right) ^n &&\text{use }\sum_{n= 0 }^\infty r^n=\frac{1}{1-r}, |r|<1\\
&=&\displaystyle \frac{3}{7}\cdot  \frac{1}{ \left(1 -\frac{ 1}{7} \right)}+ \frac{ 1}{21}\cdot  \frac{1 }{ \left(1-\frac{1 }{ 3} \right)}\\
&=&\displaystyle \frac{4}{7}\quad .
\end{array}
\]
}
\solution{\ref{sum_n=0^infty(2^(n+1)+(-3)^(n-1))/5^n}.
\[
\begin{array}{rcll|l}
\displaystyle \sum\limits_{n=0}^{\infty} \frac{2^{n+1}+ (-3)^{ n-1}}{5^n} &=& \displaystyle \sum \limits_{n=0}^{ \infty} \left(2\cdot \frac{2^{n}}{5^n}-\frac{ 1 }{3}\cdot  \frac{ (-3)^n }{5^n}\right)\\
&=&\displaystyle 2 \sum_{n=0}^{ \infty}\left(\frac{2}{5 } \right)^n -\frac{1}{3} \sum_{ n=0}^{\infty}\left(-\frac{3}{5 } \right) ^n &&\text{use }\sum_{n= 0 }^\infty r^n=\frac{1}{1-r}, |r|<1\\
&=&\displaystyle 2\cdot  \frac{1}{ \left(1 -\frac{ 2}{5} \right)}- \frac{ 1}{3}\cdot  \frac{1 }{ \left(1-\left(-\frac{3}{ 5}\right) \right)} \\
&=&\displaystyle \frac{25}{8} \quad .
\end{array}
\]

}
\begin{problem}
Find whether the series is convergent or divergent using an
appropriate test. Some of the problems require the alternating series test. The test states the following.

\frame{\textbf{Alternating series test.} Suppose $b_n \searrow 0$. Then $\sum (-1)^n b_n$ is convergent.} 

Here, $b_n\searrow 0$ means the following.
\begin{itemize}
\item The sequence of numbers $b_n$ is decreasing.
\item The sequence decreases to $0$, that is, 
\[\lim\limits_{n\to \infty} b_n=0\quad .
\]
\end{itemize}
%
\begin{multicols}{2}
\begin{enumerate}[ref={\fcProblemRef}]
\item \label{problemConvergencesumn=1^infty(-1)^nlnn} $\displaystyle\sum_{n=1}^{\infty} (-1)^n\ln n  . $

\answer{diverges, basic divergence test}
\item \label{problemConvergencesumn=2^infty(-1)^n/lnn} $\displaystyle \sum_{n=2}^{\infty} \frac{(-1)^n }{\ln n}  .$

\answer{converges, alternating series test}
\item \label{problemConvergencesum_n=2^infty(-1)^nn/ln(n)} $\displaystyle \sum\limits_{n=2}^{\infty}\frac{n}{\ln n}$

\answer{diverges, basic divergence test}
\item \label{problemConvergencesum_n=2^infty(-1)^nln(n)/n} $\displaystyle \sum\limits_{n=2}^{\infty}\frac{\ln n}{n}$

\answer{converges, alternating series test}
\end{enumerate}
\end{multicols}
\end{problem}
\solution{\ref{problemConvergencesumn=1^infty(-1)^nlnn}. $\lim\limits_{n\to\infty }(-1)^n \ln n$ does not exist and therefore the sum is not convergent.
}

\noindent \solution{\ref{problemConvergencesumn=2^infty(-1)^n/lnn}. For $n>2$, we have that $\ln n$ is a positive increasing function and therefore $\frac{1}{\ln n}$ is a decreasing positive function. Furthermore $\displaystyle \lim_{n\to \infty}\frac{1}{\ln n} =0 $. Therefore the series is convergent by the alternating series test.
}

\begin{problem}
Use integral test, the comparison test or the limit comparison test to determine whether the series is convergent or divergent. Justify your answer.
\begin{multicols}{2}
\begin{enumerate}
\item 
$\displaystyle \sum\limits_{n=1}^{\infty} \frac{1}{2n+1}$.

\answer{divergent}
\item 
$\displaystyle \sum\limits_{n=1}^{\infty} \frac{1}{2n^2+n^3}$.

\answer{convergent}
\item $\displaystyle \sum\limits_{n=1}^{\infty}\frac{n^2+3}{3n^5+n}$

\answer{convergent, can use limit comparison test}
\item \label{problemConvergencesum_2^infty1/(xlnx)dx}
$\displaystyle \sum_{2}^\infty \frac{1}{n \ln n}$

\item  \label{problemConvergencesum_2^infty1/((2n+1)ln(n)}
$\displaystyle \sum\limits_{n=2}^{\infty} \frac{1}{(2n+1)\ln (n)}$.

\answer{divergent}
\item $\displaystyle \sum\limits_{n=2}^{\infty}\frac{1}{n(\ln n)^2}$

\answer{convergent, can use integral test}
\item 
$\displaystyle \sum\limits_{n=2}^{\infty} \frac{1}{(2n+1)(\ln (n))^2}$.

\answer{convergent}
\item 
Determine all values of $p$, $q$ $r$ for which the series 
\[
\displaystyle \sum_{n=30}^{\infty} \frac{1}{n^p(\ln n)^q(\ln (\ln n))^r}
\]
is convergent.


\end{enumerate}
\end{multicols}


\end{problem}
\solution{\ref{problemConvergencesum_2^infty1/(xlnx)dx}.
\[
\begin{array}{rcl}
\displaystyle \int_{2}^{\infty}\frac{1}{x\ln x}\diff x&=&\displaystyle \lim_{t\to\infty} \int_{2}^{t}\frac{1}{x\ln x}\diff x\\
&=&\displaystyle \lim_{t\to\infty} \int_{2}^{t}\frac{1}{\ln x}\diff(\ln x)\\
&=&\displaystyle \lim_{t\to\infty} \int_{2}^{t}\diff(\ln(\ln x))\\
&=&\displaystyle \lim_{t\to \infty}\left[ \ln(\ln x)\right]_{x=2}^{x=t}\\
&=&\displaystyle \lim_{t\to \infty}\left(\ln(\ln t)-\ln (\ln 2)\right)\\
&=&\displaystyle \infty,
\end{array}
\]
therefore the integral is divergent (and diverges to $+\infty$).

The function $\frac{1}{x\ln x}$ is decreasing, as for $x>2$, it is the quotient of $1$ by increasing positive functions. $\frac{ 1}{ x\ln x}$ tends to $0$ as $x\to \infty$, and therefore the integral criterion implies that $\sum\limits_{2}^{ \infty} \frac{1 }{n \ln n}$ is divergent.
}



%\vskip 18cm
%\hfill \begin{tabular}{c|c|c|c|c|c|c||c}
%Problem&1 &2&3&4&5&6& $\sum$\\ \hline
%Score&&&&&&&\\ \hline
%Max&17&17&17&17&17&17&102
%\end{tabular} 


\end{document}