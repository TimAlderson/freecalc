\documentclass[12pt]{article}
\usepackage{amsmath}
%\usepackage{cancel}

%compiles with pdflatex -shell-escape
%\addtolength{\hoffset}{-3.5cm}
%\addtolength{\textwidth}{6.8cm}
%\addtolength{\voffset}{-3cm}
%\addtolength{\textheight}{6cm}
\ProvidesPackage{homework-problems}
\usepackage{amsmath, amsfonts, amssymb, verbatim, hyperref, ifthen}
\usepackage{auto-pst-pdf}
\usepackage{pst-plot}
\usepackage{multicol}
\renewcommand{\Re}{\mathrm{Re~}}
\renewcommand{\Im}{\mathrm{Im~}}
\newcommand{\doublebrace}[4]{\left\{\begin{array}{ll} #1 & #2 \\#3 & #4  \end{array} \right.}
\newcommand{\triplebrace}[6]{\left\{\begin{array}{ll} #1 & #2 \\#3 & #4  \\#5 & #6\end{array} \right.}
\newcommand{\bigFatWarning}{ %\textbf{This homework contains copyrighted material from  James Stewart, Calculus, 7th edition, 2012. You are not permitted to copy this file for any purpose other than completing your homework. You are not allowed to give a copy of this file to anyone outside of our course. }
}
\newenvironment{solution}%
{\begin{proof}[\bfseries\upshape Solution]\renewcommand{\qedsymbol}{}}%
{\end{proof}}%
\newcommand{\ans}[1]{\iftoggle{solutions}{\begin{solution}#1\end{solution}}{}}
\newcommand{\homeworkEnd}{\end{enumerate}\end{document}}
\newcommand{\homeworkStart}[2]{\title{\course \\ Homework \ #1}\date{%
\ifthenelse{\equal{#2}{}}{}{%
Due #2 at \deadline}}%
\begin{document}\maketitle\begin{enumerate}
}%
\newcommand{\points}[1]{\stepcounter{enumi}\item[ ({\bf #1 mark\ifthenelse{\equal{#1}{1}}{}{s}}) \arabic{enumi}.]}
\newcommand{\pointsii}[1]{\stepcounter{enumii}\item[ ({\bf #1 mark\ifthenelse{\equal{#1}{1}}{}{s}}) (\alph{enumii})]}
\newcommand{\answer}[1]{ \hfill{~} \rotatebox{180}{ answer: #1}}
 %warning folder paths are relative to the file that uses the includepackage

\renewcommand{\answer}[1]{\iftoggle{answers}{ \hfill{~} \rotatebox{180}{\tiny answer: #1}}{} }
\renewcommand{\hiddenanswer}{\answer}
\renewcommand{\points}[1]{\item}
\renewcommand{\pointsii}[1]{\item}
\renewcommand{\Arctan}{\arctan}
\renewcommand{\Arccos}{\arccos}
\renewcommand{\Arcsin}{\arcsin}
\renewcommand{\Arccot}{\operatorname{arccot}}


\toggletrue{solutions}
\toggletrue{answers}

\newtheorem{problem}{Problem}
\renewcommand{\fcProblemRef}{\theproblem.\theenumi}
\renewcommand{\fcSubProblemRef}{\theenumi.\theenumii}

\newcommand{\finishUMBExamPage}{
\vfill
%\begin{tabular}{l}
%****CONFIDENTIAL **** \\ **** PLEASE DO NOT PRINT  **** 
%\end{tabular}
\hfill \begin{tabular}{|c|}\hline Score\\\hline  \\\hline \end{tabular}
}
\newcommand{\startUMBExamPage}{\newpage
\noindent Name: \underline{~~~~~~~~~~~~~~~~~~~~~~~~~~~~~~~~~~~~~}\hfill ID:\underline{~~~~~~~~~~~~~~~}}

\pagestyle{empty}
\begin{document}
\begin{center}
\Huge 
\Huge
\textsc{Math 140 Final Exam}\\
\textsc{Solutions}\\
\Large Spring 2016\\
University of Massachusetts Boston
\end{center}
\textbf{Instructions.}
\begin{enumerate}
\item Write your name on every page.
\item This is a closed book exam.
\item Show your answers and arguments for your answers in the space provided.
\item Answers without proper justification will \textbf{not receive full credit}.
\item \textbf{Calculators, cell phones and other electronic devices are not allowed.}
\item Grading table. 

{\Large
\begin{tabular}{|c|c|c|}\hline
Problem & Points & Score\\\hline\hline
1& 15 &\\\hline
2& 10 &\\\hline
3& 25 &\\\hline
4& 15 &\\\hline
5& 10 &\\\hline
6& 10 &\\\hline
7& 10 &\\\hline
8&  15 &\\\hline
9&  10 &\\\hline
10& 10 &\\\hline
11& 10 &\\\hline
$\sum$ & 140 &\\\hline
\end{tabular}
}
\end{enumerate}
\begin{problem}
Evaluate the limit or show that it does not exist.
\begin{enumerate}[ref={\fcProblemRef}]
\item 
\label{problemlimxto-2(2x^2+x-6)/(x^2-4)}

$\displaystyle \lim_{x\to -2} \frac{2x^2+x-6}{x^2-4}$
\answer{$\frac{7}{4}$}
\item \label{problemlimhto0(1/(2+h)^2-1/4)/h}
$ \displaystyle \lim_{h\to 0} \frac{\frac{1}{(2+h)^2}-\frac{1}{4}}{h}  $.

\answer{$-\frac{1}{4}$}
\item \label{problemlimxtoinfty(sqrt(3x^2+2x+1)/(x+1))}

$ \displaystyle \lim_{x\to \infty} \frac{\sqrt{3x^2+2x+1}}{x+1} $. 

\answer{$\sqrt{3}$}
\end{enumerate}
\solution{\ref{problemlimxto-2(2x^2+x-6)/(x^2-4)}

$\begin{array}{rcll|l}
\displaystyle \lim\limits_{x\to -2} \frac{2x^2+x-6}{x^2-4}&=&\displaystyle \lim\limits_{x\to -2}\frac{(2x-3)\cancel{(x+2)}}{(x-2)\cancel{(x+2)}}&&\text{factor and cancel}\\ 
&=&\displaystyle  \frac{(2(-2)-3)}{-2-2} &&\text{substitute}\\
&=&\displaystyle \frac{7}{4}
\end{array}
$

}
\solution{\ref{problemlimhto0(1/(2+h)^2-1/4)/h}.

\textbf{Variant I.}

$\begin{array}{rcll|l}
\displaystyle \lim_{h\to 0} \frac{\frac{1}{(2+h)^2}-\frac{1}{4}}{h}&=&\displaystyle \lim_{h\to 0}\frac{\frac{4-(2+h)^2}{4(2+h)^2}}{h}\\
&=&\displaystyle \lim_{h\to 0} \frac{4- (4+4h+h^2)}{4h(2+h)^2}\\
&=&\displaystyle \lim_{h\to 0} \frac{-4h-h^2}{4h(2+h)^2}\\
&=&\displaystyle \lim_{h\to 0} \frac{\cancel{h}(-4-h) }{4\cancel{h}(2+h)^2}&&\text{substitute }h=0\\
&=&\displaystyle \frac{-4-0}{4(2+0)^2}\\
&=&\displaystyle -\frac{1}{4}
\end{array}
$

\textbf{Variant II.}

$\begin{array}{rcll|l}
\displaystyle \lim_{h\to 0} \frac{\frac{1}{(2+h)^2}-\frac{1}{4}}{h}&=&\displaystyle \frac{\diff }{\diff x}\left(\frac{1}{x^2}\right)_{|x=2}\\
&=&\displaystyle \left(\frac{-2}{x^3}\right)_{|x=2}\\
&=&\displaystyle -\frac{1}{4}
\end{array}
$

}
\solution{\ref{problemlimxtoinfty(sqrt(3x^2+2x+1)/(x+1))}

$\begin{array}{rcll|l}
\displaystyle \lim_{x\to \infty} \frac{\sqrt{3x^2+2x+1}}{x+1} &=&\displaystyle \lim_{x\to \infty} \frac{\frac{1}{x} \sqrt{3x^2+2x+1}}{\frac{1}{x}\left(x+1\right) } \\
&=&\displaystyle\lim_{x\to \infty} \frac{ \sqrt{\frac{3x^2+2x+1}{x^2 }}}{\left(1+\frac{1}{x}\right) }\\
&=&\displaystyle \lim_{x\to \infty} \frac{ \sqrt{3+\frac{2}{x}+\frac{1}{x^2 }}}{\left(1+\frac{1}{x}\right) }\\
&=&\displaystyle\frac{\sqrt{3+0+0}}{1+0}\\
&=&\displaystyle \sqrt{3}
\end{array}
$
}


\end{problem}
\begin{problem}
Find the vertical and horizontal asymptotes of the function.

\label{problemAsymptotesy=x/(sqrt(x^2+3) -2x)}
$\displaystyle 
f(x)= \frac{x}{\sqrt{x^2+3} -2x}
$

\answer{vertical: $x=1$, horizontal: $y=-\frac{1}{3}$, $y=-1$}
\end{problem}
\solution{\ref{problemAsymptotesy=x/(sqrt(x^2+3) -2x)}

\textbf{Vertical asymptotes.} A function $f(x)$ has a vertical asymptote at $x=a$ if $\lim\limits_{x\to a} f(x)=\pm \infty$. 

The function is algebraic, and therefore has a finite limit at every point it is defined (i.e., no asymptote). Therefore the function can have vertical asymptotes only for those $x$ for which $f(x)$ is not defined. The function is not defined for 

\[
\begin{array}{rcll|l}
\sqrt{x^2+3}-2x&=&0\\
\sqrt{x^2+3}&=&2x&&\begin{array}{l} \text{square both sides}\\\text{may introduce extraneous solutions} \end{array}\\
x^2+3&=&4x^2\\
3x^2-3&=&0\\
3(x-1)(x+1)&=&0\\
x=1 \quad &\text{or}& \cancel{ x=-1}\\
&&x=-1 \text{ is extraneous:}\\
&& \sqrt{(-1)^2+3}-(-1)2=4\neq 0
\end{array}
\]

$x=-1$ is indeed a vertical asymptote:
\[
\lim\limits_{x\to 1^+}  \frac{x}{\sqrt{x^2+3} -2x}=\infty \quad \quad \quad 
\lim\limits_{x\to 1^-}  \frac{x}{\sqrt{x^2+3} -2x}=-\infty .
\]
\textbf{Horizontal asymptotes.} 
\[
\begin{array}{rcll|l}
\displaystyle \lim\limits_{x\to -\infty}  \frac{x}{\sqrt{x^2+3} -2x}&=&\displaystyle \lim\limits_{x\to - \infty} \frac{1}{\frac{\sqrt{x^2+3}}{x} -2} \\
&=& \displaystyle \lim\limits_{x\to - \infty} \frac{1}{-\sqrt{\frac{x^2+3}{x^2}} -2}   && \frac{1}{x} =- \sqrt{\frac{1}{x^2} } \text{ when } x<0\\
&=&\displaystyle \lim\limits_{x\to - \infty} \frac{1}{-\sqrt{1+\frac{3}{x^2}} -2}  \\
&=& \displaystyle \frac{1}{-\sqrt{1+0}-2}\\
&=&\displaystyle -\frac{1}{3}.\\
\displaystyle \lim\limits_{x\to -\infty}  \frac{x}{\sqrt{x^2+3} -2x}&=&\displaystyle \lim\limits_{x\to  \infty} \frac{1}{\frac{\sqrt{x^2+3}}{x} -2} \\
&=& \displaystyle \lim\limits_{x\to  \infty} \frac{1}{\sqrt{\frac{x^2+3}{x^2}} -2}   && \frac{1}{x} = \sqrt{\frac{1}{x^2} } \text{ when } x>0\\
&=&\displaystyle \lim\limits_{x\to  \infty} \frac{1}{\sqrt{1+\frac{3}{x^2}} -2}  \\
&=& \displaystyle \frac{1}{\sqrt{1+0}-2}\\
&=&\displaystyle -1.\\
\end{array}
\]
Therefore $y=-\frac{1}{3}$ and $y=-1$ are the two horizontal asymptotes. 


A computer generated graph confirms our computations.

\psset{xunit=0.2cm, yunit=0.2cm}
\begin{pspicture}(-16, -20)(16,17)
\tiny
\fcAxesStandard{-15}{-19.32133}{15}{16.190354}
\fcXTick{10}
\rput[t](10, -0.6){$10$}
\newcommand{\theFun}{x x x mul 3 add sqrt -2 x mul add div}
%Function formula: \frac{2 x}{\sqrt{x^{2}+x+3}-3}
\psplot[linecolor=\fcColorGraph, plotpoints=1000]{1.036}{15}{\theFun }
%Function formula: \frac{2 x}{\sqrt{x^{2}+x+3}-3}
\psplot[linecolor=\fcColorGraph, plotpoints=1000]{-15}{0.961}{\theFun }
\psline[linestyle=dotted](1,-19.3)(1,16.1)
\psline[linestyle=dashed, linecolor=blue](! -15 -1  3 div)(!15 -1 3 div)
\psline[linestyle=dashed, linecolor=blue](-15, -1)(15, -1)
\rput[b](-8, 0.2){$y=-\frac{1}{3}$}
\rput[t](8, -2){$y=-1$}
\rput[bl](5,5){$y=\frac{x}{\sqrt{x^2+3} -2x}$}
\rput[l](1.6,-8){$x=1$}
\end{pspicture}


}

\begin{problem}
Differentiate with respect to $x$.
\begin{enumerate}[ref={\fcProblemRef}]
\item \label{problemd/dx((secx)e^x)}
$\displaystyle f(x)=(\sec x )e^{x}$.

\answer{$\sec x\tan x e^x + \sec x e^x=(\tan x +1)(\sec x) e^x$}
\item \label{problemd/dx((x+1)/(x^3+1))}

$\displaystyle f(x)= \frac{x+1}{x^3+1}$.
\item \label{problemd/dx(e^(-1/x))}

$\displaystyle f(x)=e^{-\frac{1}{x}}$.

\answer{$ \frac{e^{-\frac{1}{x}}}{x^2}$}

\item \label{problemd/dxsqrt(1-sqrt(x))}

$f(x)=\sqrt{1-\sqrt{x}}$.

\answer{$-\frac{1}{4} x^{-\frac{1}{2}} \left(- \sqrt{x}+1\right)^{-\frac{1}{2}} $}

\item \label{problemd/dx(ln(sec x)+ln(cot x))}

$f(x)=\ln (\sec x) +\ln (\cot x)$.

\answer{$-\cot x$}
\end{enumerate}
\end{problem}
\solution{\ref{problemd/dx((secx)e^x)}
\[
\begin{array}{rcll|l}
\displaystyle \frac{\diff}{\diff x}\left((\sec x)e^x\right)&=&\displaystyle
\left(\frac{\diff}{\diff x}\left(\sec x\right)\right)e^x+ \left(\sec x\right) \frac{\diff}{\diff x}\left(e^x\right)&&\text{product rule}\\
&=&\displaystyle \sec x \tan x e^x+ \sec x e^x\\
&=&(\tan x+1) \sec x e^x
\end{array}
\]
}
\solution{\ref{problemd/dx((x+1)/(x^3+1))}

\[\begin{array}{rcl}
\displaystyle \frac{\diff }{\diff x}\left(\frac{x+1}{x^3+1}\right)&=&\displaystyle \frac{\diff }{\diff x}\left(\frac{\cancel{x+1}}{\cancel{(x+1)}(x^2-x+1)}\right)\\
&=&\displaystyle \frac{\diff }{\diff x}\left(\frac{1}{x^2-x+1}\right)\\
\multicolumn{3}{l}{\textbf{Variant I: use quotient rule.}}\\
&=&\displaystyle \frac{ \frac{\diff }{\diff x}(1)\cdot (x^2-x+1)-1\cdot\frac{\diff }{\diff x}\left(x^2-x+1\right)}{\left(x^2-x+1\right)^2}\\
&=&\displaystyle \frac{-2x+1}{\left(x^2-x+1\right)^2}\\
\multicolumn{3}{l}{\textbf{Variant I: use chain rule.}}\\
&=&\displaystyle \frac{\diff }{\diff x}\left(\left(x^2-x+1\right)^{-1}\right)\\
&=&\displaystyle -(x^2-x+1)^{-2}\frac{\diff}{\diff x}(x^2-x+1)\\
&=&\displaystyle -(x^2-x+1)^{-2}(2x-1)\\
&=&\displaystyle \frac{-2x+1}{\left(x^2-x+1\right)^2}.
\end{array}
\]
}
\solution{\ref{problemd/dx(e^(-1/x))}
\[
\begin{array}{rcll|l}
\frac{\diff }{\diff x}\left(e^{-\frac{1}{x}}\right)&=&e^{-\frac{1}{x}} \frac{\diff }{\diff x}\left(-\frac{1}{x}\right)&&\text{chain rule}\\
&=&\displaystyle -e^{-\frac{1}{x}} \frac{\diff }{\diff x} \left(x^{-1}\right)\\
&=&\displaystyle x^{-2}e^{-\frac{1}{x}}\\
&=&\displaystyle \frac{e^{-\frac{1}{x}}}{x^2}
\end{array}
\]
}
\solution{\ref{problemd/dxsqrt(1-sqrt(x))}

\[
\begin{array}{rcll|l}
\displaystyle \frac{\diff }{\diff x}\left(\sqrt{1-\sqrt{x}}\right)&= &\displaystyle \frac{\diff }{\diff x}\left(\left(1-x^{\frac{1}{2}} \right)^{ \frac{ 1}{2}}\right)&&\text{chain rule}\\
&=&\displaystyle \frac{1}{2}\left(1-x^{\frac{1}{ 2}}\right)^{- \frac{1}{ 2}}\frac{\diff }{\diff x}\left(1-x^{\frac{1}{2}}\right)\\
&=&\displaystyle -\frac{1}{4}x^{-\frac{1}{2}} \left(1-x^{\frac{1}{ 2}}\right)^{- \frac{1}{ 2}}
\end{array}
\]
}
\solution{\ref{problemd/dx(ln(sec x)+ln(cot x))}

\[\begin{array}{rcll|l}
\displaystyle \frac{\diff }{\diff x}(\ln(\sec x)+\ln(\cot x))&=&\displaystyle  \frac{\diff }{\diff x}\left(\ln\left(\frac{1}{\cos x}\right)+\ln\left(\frac{\cos x}{\sin x}\right)\right)\\
&=&\displaystyle  \frac{\diff }{\diff x}\left( \cancel{- \ln\left(\cos x\right)}+\cancel{\ln\left(\cos x\right)}-\ln (\sin x) \right)\\
&=&\displaystyle -\frac{\diff }{\diff x}(\ln (\sin x))&&\text{chain rule}\\
~\\
&=&\displaystyle -\frac{1}{\sin x}\frac{\diff }{\diff x}(\sin x)\\
~\\
&=&\displaystyle -\frac{\cos x}{\sin x}\\
~\\
&=&\displaystyle -\cot x
\end{array}\]
}


\begin{problem}
Verify that the coordinates of the given point satisfy the given equation. Use implicit differentiation to find an equation of the tangent line to the curve at the given point. 

\label{problemImplicitTangenty^3+x^3+4xy=3/4}
$\displaystyle y^{3}+x^{3}+4 x y=\frac{3}{4}$, $\left(-\frac{1}{2},-\frac{1}{2}\right)$

\psset{xunit=0.7cm, yunit=0.7cm}
\begin{pspicture}(-2.4,-2.4)(2.4,2.4)
\fcAxesStandard{-2}{-2}{2}{2}
\fcImplicitIId[linecolor=red, linestyle=dashed, dashes={[1 1] 0}, showGridImplicitIId=false, useMidpointImplicitPlots=false]{-3}{-3}{120}{120}{0.05}{0.05}{y y y mul mul x x x mul mul add 4 x y mul mul add 0.75 sub} 
\fcFullDot{-0.5}{-0.5}
\end{pspicture}


\answer{$y=-x-1$}
\end{problem}
\solution{\ref{problemImplicitTangenty^3+x^3+4xy=3/4}
\[
\begin{array}{rcll|l}
\displaystyle y^{3}+x^3+4x y&=& \displaystyle \frac{3}{4} &&\text{apply }\frac{\diff }{\diff x}\\
\displaystyle 3y^2 \frac{\diff y}{\diff x}+3x^2+4\left(y+x\frac{\diff y}{\diff x}\right)&=&
\displaystyle \frac{\diff y}{\diff x}\left(3y^2+4x\right)&=&\displaystyle -3x^2-4y\\
\displaystyle \frac{\diff y}{\diff x}&=&\displaystyle  \frac{-3x^2-4y}{ 3y^2+4x}&&\text{substitute }x=-\frac{1}{2}, y=-\frac{1}{2}\\
\displaystyle \frac{\diff y}{\diff x}_{|x=-\frac{1}{2}, y=-\frac{1}{2}}&=& \frac{-3\left(-\frac{1}{2}\right)^2-4\left(-\frac{1}{2}\right)}{ 3\left(-\frac{1}{2}\right)^2+4\left(-\frac{1}{2}\right)}\\
\displaystyle \frac{\diff y}{\diff x}_{|x=-\frac{1}{2}, y=-\frac{1}{2}}&=&-1
\end{array}
\]
Therefore the equation of the tangent is $y-\left(-\frac{1}{2}\right)=-(x-\left(-\frac{1}{2}\right))$ which simplifies to $y=-x-1$. A computer-generated plot visually confirms our computations.

\psset{xunit=0.7cm, yunit=0.7cm}
\begin{pspicture}(-2.4,-2.4)(2.4,2.4)
\fcAxesStandard{-2}{-2}{2}{2}
\fcImplicitIId[linecolor=red, linestyle=dashed, dashes={[1 1] 0}, showGridImplicitIId=false, useMidpointImplicitPlots=false]{-3}{-3}{120}{120}{0.05}{0.05}{y y y mul mul x x x mul mul add 4 x y mul mul add 0.75 sub} 
\psplot[linecolor=\fcColorTangent]{-3}{2}{x -1 mul -1 add}
\fcFullDot{-0.5}{-0.5}
\end{pspicture}
}


\begin{problem}Find the maximum and the minimum of the function over the indicated closed interval. Indicate for which values of $x$ is the maximum/minimum attained.
\begin{enumerate}[ref={\fcProblemRef}]
\item  \label{problemmaxminx^3-x+1over[-2,1]}
$f(x)=x^3-x+1$,  $x\in[-2,1]$.

\answer{$f_{max}=f\left(-\frac{\sqrt{3}}{3}\right)= \frac{2}{9}\sqrt{3}+1, \quad f_{min}=f\left(-2\right)= -5 $}
\item \label{problemmaxminxe^(2x)over[-2,1/2]}
$f(x)=x e^{2x}$, $x\in\left[ -2,\frac{1}{2}\right]$.

\answer{$ f_{max}=f\left(\frac{1}{2}\right)= \frac{e}{2}$, $f_{min}=f\left(-\frac{1}{2}\right)=-\frac{1}{2e}$}
\end{enumerate}
\end{problem}
\solution{\ref{problemmaxminx^3-x+1over[-2,1]}
By the closed interval method the maximuma/minima over $[-2,1]$ are obtained either at the endpoints or at the critical points of $f(x)$. Since $f'(x)$ is defined over the entire interval, the only critical points are the ones for which  $f'(x)=0$.
\[
\begin{array}{rcl}
\displaystyle f'(x)&=&0\\
\displaystyle 3x^2-1&=&0\\
\displaystyle x^2&=&\displaystyle \frac{1}{3}\\
\displaystyle x&=&\displaystyle \pm\sqrt{\frac{1}{3}}=\pm\frac{\sqrt{3}}{3}
\end{array}
\]
The maximum and minimum of $f$ over $[-2,1]$ is attained at one of the points $x=-2, -\frac{\sqrt{3}}{3}, \frac{\sqrt{3}}{3}, 1$.

$\begin{array}{c|c}
x& f(x)\\\hline
-2&f(-2)=(-2)^3-(-2)+1=-5\\
-\frac{\sqrt{3}}{3}&f\left( -\frac{\sqrt{3}}{3}\right)= \left( -\frac{\sqrt{3 }}{3}\right)^3-\left( -\frac{\sqrt{3}}{3}\right)+1=\frac{2}{9}\sqrt{3}+1 \\
\frac{\sqrt{3}}{3}&f\left( \frac{\sqrt{3}}{3}\right)= \left( \frac{\sqrt{3}}{3}\right)^3-\left( \frac{\sqrt{3}}{3}\right)+1 =-\frac{2}{9}\sqrt{3}+1 \\
1&f(1)= \left( 1\right)^3-\left( 1\right)+1=1 \\
\end{array}
$

Observation of the table above shows that the maximum of $f$ over $[-2,1]$ is $f\left( -\frac{\sqrt{3}}{3}\right) =\frac{2}{9}\sqrt{3}+1$ attained for $x=-\frac{\sqrt{3}}{3}$ and the minimum is $f(-2)=-5$ (attained for $x=-2$).
}
\solution{\ref{problemmaxminxe^(2x)over[-2,1/2]}


By the closed interval method the maximuma/minima over $[-2,\frac{1}{2}]$ are obtained either at the endpoints or at the critical points of $f(x)$. Since $f'(x)$ is defined over the entire interval, the only critical points are the ones for which  $f'(x)=0$.
\[
\begin{array}{rcll|l}
\displaystyle f'(x)&=&0\\
\displaystyle \frac{\diff}{\diff x}\left(x e^{2x}\right)&=&0\\

\displaystyle  2 x e^{2 x}+e^{2 x}&=&0\\
\displaystyle e^{2x}(2x+1)&=&0&&e^{2x}\neq 0\\
\displaystyle 2x+1&=&\displaystyle 0 \\
x&=&\displaystyle -\frac{1}{2}
\end{array}
\]
The maximum and minimum of $f$ over $\left[-2,\frac{1}{2}\right]$ is attained at one of the points $x=-2, -\frac{1}{2}, \frac{1}{2}$.

$\begin{array}{c|c}
x& f(x)\\\hline
-2&f(-2)=-2e^{-4}\\
-\frac{1}{2}& f\left(-\frac{1}{2}\right)= -\frac{e^{-1}}{2}\\
\frac{1}{2}&f\left(\frac{1}{2}\right)= \frac{e}{2}\\
\end{array}
$

We have $2e^{-4}= \frac{e^{-1}}{2} \left(4e^{-3}\right)<\frac{e^{-1}}{2}$ and therefore $-\frac{e^{-1}}{2}<-2e^{-4}$. Therefore the maximum of $f$ over $\left[-2,\frac{1}{2}\right]$ is $f\left( \frac{1}{2}\right) =\frac{e}{2}$ (attained for $x=\frac{1}{2}$) and the minimum is $f\left(-\frac{1}{2}\right)=-\frac{e^{-1}}{2}=-\frac{1}{2e}$ (attained for $x=-\frac{1}{2}$).
}
\begin{problem}
\label{problemSketch(4x^2+10x+5)/(2x+1)}

Consider the function $\displaystyle f(x)=\frac{4 x^{2}+10 x+5}{2 x+1} $. Computation shows that $\displaystyle f'(x)=\frac{8 x^{2}+8 x}{\left(2 x+1\right)^{2}}$ and $\displaystyle f''(x)=\frac{8}{\left(2 x+1\right)^{3}} $.

\begin{itemize}
\item Find the intervals of increase and intervals of decrease of $f$.

\item Find the local maxima and minima of $f$. 
\item Find where the function is concave up and where it is concave down.
\item Sketch the function $f(x)$ roughly by hand. Make sure that your plot matches your computations from the preceding parts of the problem.

You may use the provided grid and coordinate system. From the previous page, we recall that $\displaystyle f(x)=\frac{4 x^{2}+10 x+5}{2 x+1} $, $\displaystyle f'(x)=\frac{8 x^{2}+8 x}{\left(2 x+1\right)^{2}}$ and $\displaystyle f''(x)=\frac{8}{\left(2 x+1\right)^{3}} $.

The 4 points plotted on the grid are known to lie on the curve.

\psset{xunit=1cm, yunit=1cm}
\begin{pspicture}(-3.1,-3)(3.1,12.1)
\fcAxesStandard{-3}{-3}{3}{12}
\fcGrid[linestyle=dashed, linewidth=0.5, linecolor=gray]{-3}{-3}{6}{15}{1}{1}{}
\rput[t](0.9,-0.2){$1$}
\fcLabels{3}{12}
\fcFullDot{-3}{1 dict begin /x -3 def x x mul 4 mul 10 x mul 5 add add 2 x mul 1 add div end}
\fcFullDot{-0.59}{1 dict begin /x -0.59 def x x mul 4 mul 10 x mul 5 add add 2 x mul 1 add div end}
\fcFullDot{-0.44}{1 dict begin /x -0.44 def x x mul 4 mul 10 x mul 5 add add 2 x mul 1 add div end}
\fcFullDot{3}{1 dict begin /x 3 def x x mul 4 mul 10 x mul 5 add add 2 x mul 1 add div end}

%\psplot{-3}{-0.59}{x x mul 4 mul 10 x mul 5 add add 2 x mul 1 add div}
%\psplot{-0.44}{3}{x x mul 4 mul 10 x mul 5 add add 2 x mul 1 add div}
\end{pspicture}
\end{itemize}
\end{problem}
\solution{\ref{problemSketch(4x^2+10x+5)/(2x+1)}

\textbf{Intervals of increase and decrease.} The intervals of increase and decrease of $f(x)$ are determined by the intervals where $f'(x)$ does not change sign. The candidates for the endpoints of these intervals are the critical points of $f(x)$, i.e., the points for which $f'(x)=0$ and the points for which $f'(x)$ is not defined. Since $f'(x)=\frac{8x(x+1)}{(2x+1)^2}$, it follows that $f'(x)$ may change sign the critical points $x=0$, $x=-1$ and $x=-\frac{1}{2}$. However $f'(x)$ does not change sign near $x=-\frac{1}{2}$ as the term $(2x+1)$ is raised to an even power. Therefore the intervals of increase and decrease are given in the following table.
\begin{tabular}{c|c|c|c}
&$(-\infty, -1)$& $(-1,0)$& $(0,\infty)$\\\hline
$f'(x)$& $+$&$-$&$+$\\\hline
 $f(x)$& $\nearrow $&$\searrow$& $\nearrow$
\end{tabular}

\textbf{Local maxima and minima. } A local maximum occurs where $f$ changes from increasing to decreasing or the other way around. Therefore the preceding point implies that $f$ has local maximum at $x=-1$ equal to $f(-1)=1$ and local minimum at $x=0$ equal to 5.

\textbf{Intervals of concavity.} The intervals of concavity are determined by the points where $f''(x)=\frac{8}{(2x+1)^3}$ changes sign. The denominator of $f''(x)$ changes sign near $x=-\frac{1}{2}$, so the intervals of concavity are $\left(-\infty, -\frac{1}{2}\right)$ and $\left(-\frac{1}{2}, \infty \right)$. The concavity of $f(x)$ is then determined by the following table.
\begin{tabular}{c|c|c}
&$\left(-\infty, -\frac{1}{2}\right)$& $\left(-\frac{1}{2},\infty\right)$ \\\hline
$f'(x)$& $-$&$+$\\\hline
 $f(x)$& $\cap $&$\cup$
\end{tabular}

\textbf{Curve sketching.}Please note that $x=-\frac{1}{2}$ is a vertical asymptote of $f(x)$. Together with the data computed above this makes it relatively easy to quickly produce a relatively accurate plot of $f(x)$ by hand.  A computer generated plot is included below.
\psset{xunit=1cm, yunit=1cm}
\begin{pspicture}(-3.1,-3)(3.1,12.1)
\fcAxesStandard{-3}{-3}{3}{12}
\fcGrid[linestyle=dashed, linewidth=0.5, linecolor=gray]{-3}{-3}{6}{15}{1}{1}{}
\rput[t](0.9,-0.2){$1$}
\fcLabels{3}{12}
\fcFullDot{-3}{1 dict begin /x -3 def x x mul 4 mul 10 x mul 5 add add 2 x mul 1 add div end}
\fcFullDot{-0.59}{1 dict begin /x -0.59 def x x mul 4 mul 10 x mul 5 add add 2 x mul 1 add div end}
\fcFullDot{-0.44}{1 dict begin /x -0.44 def x x mul 4 mul 10 x mul 5 add add 2 x mul 1 add div end}
\fcFullDot{3}{1 dict begin /x 3 def x x mul 4 mul 10 x mul 5 add add 2 x mul 1 add div end}

\psplot[linecolor=red]{-3}{-0.59}{x x mul 4 mul 10 x mul 5 add add 2 x mul 1 add div}
\psplot[linecolor=red]{-0.44}{3}{x x mul 4 mul 10 x mul 5 add add 2 x mul 1 add div}
\psline[linestyle=dotted](-0.5,-3)(-0.5,12)
\end{pspicture}
}

\begin{problem}
{\sc Do not solve the integral. } Compute a \textbf{Riemann sum} for the indicated integral using the indicated number of intervals with the indicated sampling point. 
\label{problemRiemannSum1/(3x^2+1)from-1to0with3intervalsLeftEndpt}
$\displaystyle\int_{-1}^0\frac{1}{3 {{x}}^{2}+1}\diff x
$. Use \textbf{$3$ intervals} of equal width, choose the sampling points to be the \textbf{left endpoints} of each interval. 
Simplify your answer to a rational number (single fraction of two integers).

\answer{ $\Delta x = \frac{1}{3}$ and $f(x) =\frac{1}{3 {{x}}^{2}+1}$. Thus $\displaystyle \int\limits_{-1}^0 f(x) \diff x$  is approximated by $\Delta x \left(f{}\left(-1\right)+f{}\left( -\frac{2}{3}\right)+f{}\left( -\frac{1}{3}\right)\right)=\frac{10}{21}$.}

\end{problem}
\solution{\ref{problemRiemannSum1/(3x^2+1)from-1to0with3intervalsLeftEndpt}

$\Delta x = \frac{1}{3}$ and $f(x) =\frac{1}{3 {{x}}^{2}+1}$. Thus $\displaystyle \int\limits_{-1}^0 f(x) \diff x$  is approximated by $\Delta x \left(f{}\left(-1\right)+f{}\left(-\frac{2}{3}\right)+f{}\left(-\frac{1}{3}\right)\right)=\frac{10}{21}$.

}


\begin{problem}
Integrate.
\begin{enumerate}[ref={\fcProblemRef}]
\item \label{problemIntegrate(sqrt[3]x-x^(1/2)+1)/xdx}
$\displaystyle\int \frac{\sqrt[3]{x}-x^{\frac{1}{2}}+1 }{x}\diff x$.

\answer{$ -2 \sqrt{x}+3 x^{\frac{1}{3}}+\ln \left|x\right| +C$}
\item  \label{problemIntegratefrom-3to2_x/(1-x^2)dx}

$\displaystyle\int_{-3}^{-2} \frac{x}{1-x^2}\diff x$.

\answer{$\left[-\frac{1}{2}\ln \left|1-x^2\right| \right]_{-3}^{-2}=\frac{1}{2} \ln{}\left(\frac{8}{3}\right) $}
\item \label{problemIntegrate(sin(lnx))/xdx}

$\displaystyle\int \frac{\sin (\ln x)}{x} \diff x$.

\answer{$-\cos(\ln x)+C  $}
\end{enumerate}
\end{problem}
\solution{\ref{problemIntegrate(sqrt[3]x-x^(1/2)+1)/xdx}

$\begin{array}{rcl}
\displaystyle \int \frac{\sqrt[3]x-x^{\frac{1}{2}}+1}{x}\diff x&=&\displaystyle \int \left(x^{-\frac{2}{3}} -x^{-\frac{1}{2}}+\frac{1}{x}\right)\diff x\\
&=& +3 x^{\frac{1}{3}}-2 \sqrt{x}+\ln \left|x\right| +C.
\end{array}
$
}
\solution{\ref{problemIntegratefrom-3to2_x/(1-x^2)dx}

\[\begin{array}{rcll|l}
\displaystyle \int_{-3}^{-2} \frac{x}{1-x^2}\diff x&=&\displaystyle \int\limits_{\tiny \begin{array}{rcl}x&=&-3\\ u&=&-8\end{array}}^{\tiny  \begin{array}{rcl}x&=&-2\\ u&=&-3\end{array}} \frac{1}{u}\left(-\frac{1}{2}\diff u\right)&& \begin{array}{rcl}
u&=&1-x^2\\
\diff u&=&-2x\diff x\\
x\diff x&=&-\frac{1}{2}\diff u
\end{array}\\
&=&\displaystyle -\frac{1}{2}\left[\ln |u|\right]_{-8}^{-3}\\~\\
&=&\displaystyle -\frac{1}{2} \left(\ln|3|-\ln|8| \right)\\~\\
&=&\displaystyle \frac{\ln\left|\frac{8}{3}\right|}{2}
\end{array}
\]

}
\solution{\ref{problemIntegrate(sin(lnx))/xdx}
\[
\begin{array}{rcll|l}
\displaystyle \int\frac{\sin(\ln x)}{x}\diff x &=&\displaystyle \int \sin(\ln x)\diff \left(\ln x\right)&&u=\ln x\\
&=&\displaystyle \int \sin u\diff u\\
&=&-\cos u+C\\
&=&-\cos(\ln x)+C
\end{array}
\]
}
\begin{problem}
Find the  derivative of $f(x)$ using the Fundamental Theorem of Calculus, Part I.
\begin{enumerate}[ref={\fcProblemRef}]
\item \label{problemd/dxint_1^x(sqrt(t)-t^(1/3))dt}
$\displaystyle f(x)=\int_{1}^x\left(\sqrt{t}- \sqrt[3]{t}\right)\diff t$.

\answer{$ \sqrt{x}-\sqrt[3]{x}$}
\item 
\label{problemd/dxint_1^(1/(x+1))sin(t^2)dt}

$\displaystyle f(x)=\int_{1}^{\frac{1}{x+1} }\sin \left(  t^2\right) \diff t$.

\answer{}
\end{enumerate}
\end{problem}
\solution{\ref{problemd/dxint_1^x(sqrt(t)-t^(1/3))dt}

\[\begin{array}{rcll|l}
\displaystyle \frac{\diff }{\diff x}\int_1^x\left(\sqrt{t}-\sqrt[3]{t }\right)\diff t&=&\sqrt{x}-\sqrt[3]{x}&&\text{FTC part I} \\
\end{array}
\]
}
\solution{\ref{problemd/dxint_1^(1/(x+1))sin(t^2)dt}
\[
\begin{array}{rcll|l}
\displaystyle \frac{\diff }{\diff x}\int_{1}^{\frac{1}{x+1}} \sin \left(t^2\right)\diff t&=&\displaystyle \frac{\diff}{\diff x}\int_1^{u}\sin (t^2)\diff t   &&u=\frac{1}{x+1}, \text{ use FTC part I, chain rule} \\
&=&\displaystyle \sin\left(u^2\right)\frac{\diff u}{\diff x}\\
&=&\displaystyle \sin\left(\frac{1}{(x+1)^2} \right)\frac{\diff }{\diff x}\left(\frac{1}{x+1} \right)\\
&=&\displaystyle \sin\left(\frac{1}{(x+1)^2} \right)\left(-\frac{1}{(x+1)^2} \right)\\
&=&\displaystyle -\frac{1}{(x+1)^2}\sin\left(\frac{1}{(x+1)^2} \right)\\
\end{array}
\]
}
\begin{problem}~
\label{problemareabetweeny=x^2andy=2x^2+x-2}
\begin{itemize}
\item Sketch the region bounded by the curves $y=x^2$ and $y=2x^2+x-2$.

\psset{xunit=0.5cm, yunit=0.5cm}
\begin{pspicture}(-3.4,-3.6)(3,5.7)
\fcAxesStandardNoFrame{-3.5}{-3.5}{2.5}{5.5}
\fcGrid[linestyle=dashed, linewidth=0.5, linecolor=gray]{-3}{-3}{5}{8}{1}{1}{}
\rput[t](0.9,-0.2){$1$}
\fcLabels{3.5}{5.5}
%\psplot{-3}{2}{x x mul}
%\psplot{-3}{2}{x x mul 2 mul x -2 add add}
\end{pspicture}

\item Find the area of the region.

\answer{$\frac{9}{2}$}
\end{itemize}
\end{problem}
\solution{\ref{problemareabetweeny=x^2andy=2x^2+x-2}

\textbf{Region plot.}

\psset{xunit=0.5cm, yunit=0.5cm}
\begin{pspicture}(-3.4,-3.4)(3,5.7)
\pscustom*[linecolor=\fcColorAreaUnderGraph]{%
\psplot{-2}{1}{x x mul}%
\psplot{1}{-2}{x x mul 2 mul x -2 add add}%
}%
\fcAxesStandardNoFrame{-3.5}{-3.5}{2.5}{5.5}
\psplot{-3}{2}{x x mul}
\psplot{-3}{2}{x x mul 2 mul x -2 add add}
\fcGrid[linestyle=dashed, linewidth=0.5, linecolor=gray]{-3}{-3}{5}{8}{1}{1}{}
\rput[t](0.9,-0.2){$1$}
\fcLabels{3.5}{5.5}
\end{pspicture}

The intersection between the two parabolas are found via
\[
\begin{array}{rcl}
x^2&=&2x^2+x-2\\
x^2+x-2&=&0\\
(x-1)(x+2)&=&0\\
x=1&& x=-2\\
y=1&&y=4.
\end{array}
\]

\textbf{Area of the region.} 
\[
\begin{array}{rcll|l}
A&=&\displaystyle\int_{1}^{-2}\left|x^2-(2x^2+x-2) \right|\diff x&&x^2>(2x^2+x-2) \text{ for }x\in [-2,1] \text{ (from plot)}\\
&=&\displaystyle\int_{1}^{-2}\left(x^2-(2x^2+x-2) \right)\diff x\\
&=&\displaystyle \left[-\frac{1}{3} x^{3}-\frac{1}{2} x^{2}+2 x \right]_{-2}^1\\
&=&\displaystyle \frac{9}{2}.
\end{array}
\]
}

\begin{problem}
\label{problemVolumeAreay=-x^2+2andy=0rotatedAroundy=0andy=-3}
Set up \textsc{but do not evaluate} an integral to calculate the volume of the solid obtained by rotating the region bounded by $y=-x^2+2$ and $y=0$ about the given line. 

\begin{itemize}
\item The $x$ axis.
\item The line $y=-3$.
\end{itemize}

\end{problem}
\solution{\ref{problemVolumeAreay=-x^2+2andy=0rotatedAroundy=0andy=-3}
First, we plot the 2d region. The two curves intersect when $-x^2+2=0$, i.e., when $x=\pm \sqrt{2}$


\hfil \hfil \begin{pspicture}(-6.2,-3.2)(6.2,3.2)\tiny
\pscustom*[linecolor=\fcColorAreaUnderGraph]{
\psplot[linecolor=\fcColorGraph]{2 sqrt -1 mul}{2 sqrt}{x x mul -1 mul 2 add}
}
\newcommand{\theFuN}{x x mul -1 mul 2 add\space}%
\psplot[linecolor=\fcColorGraph]{-2}{2}{x x mul -1 mul 2 add}
\psline[linecolor=\fcColorGraph](-2,0)(2,0)
\fcAxesStandardNoFrame{-2}{-3.1}{2}{3}
\rput[t](! 2 sqrt -0.1){$\sqrt{2}$}
\rput[t](! 2 sqrt -1 mul -0.1){$-\sqrt{2}$}
\psline[arrows=<->](1, 0)(! 1 1 dict begin /x 1 def \theFuN end)
\rput[l](1.1,0.3){cross-section rad., $y=0$}
\psline[arrows=<->](-1, -3)(! -1 1 dict begin /x -1 def \theFuN end)
\rput[r](-1.1,-1){cross-section rad., $y=-3$}
\psline[linecolor=green](-2,-3)(2,-3)
\end{pspicture}

\textbf{Rotation about $y=0$. }

Unless explicitly stated in the problem, a 3d plot of the solid is not required in the solution. Nevertheless generating such a plot helps to understand the problem. 

To generate a 3d plot of the solid, we draw the circular cross-sections of the solid of revolution. By hand, this can be done roughly by drawing ovals (circles look like ovals when observed at an angle) centered at the axis about which we revolve the 2d-region. We include a computer-generated plot below; the plot's precision is above what is expected on an exam.

\hfil \hfil \begin{pspicture}(-3,-3)(4.2,3.2)%
\newcommand{\theFun}{u u mul -1 mul 2 add\space}%
\renewcommand{\fcScreenStyle}{x}
\renewcommand{\fcScreen}{[-1 -0.2 -0.75] -1}
\fcStartIIIdScene%
\fcAxesIIIdFullInScene{-3}{-3}{-3}{3}{3}{3}%
\fcSurfaceInScene[arrows=(none), iterationsV=15, iterationsU=8, colorVU={1 0.5 0.5}]{2 sqrt -1 mul 0.001 add}{0}{2 sqrt -0.001 add}{360}{[u v cos \theFun mul v sin \theFun mul]}{}%
\fcSurfaceInScene[arrows=(none), iterationsV=4, iterationsU=3, colorUV={0.3 0.7 1}, forceForeground=true]{2 sqrt -1 mul }{0}{2 sqrt}{1}{[u \theFun v mul  0]}{}%
\fcFinishIIIdScene[true]%
\fcPutIIId{[3 0 0]}{$x$}
\fcPutIIId{[0 3 0]}{$y$}
\fcPutIIId{[0 0 3]}{$z$}
\end{pspicture}

The volume of a solid (and in particular, of a solid of revolution) is computed by integrating the area $A(x)=\pi(\text{radius cross-section})= \pi (-x^2+2)^2 $ of the cross-section of the solid. Therefore the volume $V$ equals
\[
\begin{array}{rcll|l}
V&=&\displaystyle \int_{a}^bA(x)\diff x\\
&=&\displaystyle\int_{-\sqrt{2}}^{\sqrt{2}}\pi (-x^2+2)^2  \diff x\\
&=&\displaystyle\pi\left[\frac{1}{5} x^{5}-\frac{4}{3} x^{3}+4 x\right]_{-\sqrt{2}}^{\sqrt 2}&&\text{step not required by problem}\\
&=&\displaystyle \pi \frac{64}{15}\sqrt{2}&&\text{step not required by problem.}
\end{array}
\]


\textbf{Rotation about $y=-3$. } The cross-section of this solid of revolution is a washer with inner radius $ 3$ and outer radius $-x^2+2-(-3)=5-x^2$. Therefore the area of the cross-section is $\pi (5-x^2)^2-\pi 3^2$ and the volume is computed via

\[
\begin{array}{rcll|l}
V&=&\displaystyle \int_{a}^bA(x)\diff x\\
&=&\displaystyle\int_{-\sqrt{2}}^{\sqrt{2}} \pi \left( (5-x^2)^2- 3^2\right)  \diff x\\
&=&\displaystyle\pi\left[\frac{1}{5} x^{5}-\frac{10}{3} x^{3}+16 x  \right]_{-\sqrt{2}}^{\sqrt 2}&&\text{step not required by problem}\\
&=&\displaystyle \pi  \frac{304}{15}\sqrt{2}&&\text{step not required by problem.}
\end{array}
\]
\hfil \hfil \begin{pspicture}(-4,-9)(5,4.2)%
\newcommand{\theFuN}{u u mul -1 mul 2 add\space}%
\renewcommand{\fcScreenStyle}{x}
\fcStartIIIdScene%
\fcAxesIIIdFullInScene{-3}{-9}{-4}{3}{3}{4}%
\renewcommand{\fcScreen}{[-1 -0.2 -0.75] -1}
\fcSurfaceInScene[arrows=(none), iterationsV=15, iterationsU=8, colorVU={0.7 0.2 0.2}, colorUV={0.7 0.2 0.2}]{2 sqrt -1 mul }{0}{2 sqrt -0.001 add}{360}{[u v cos 3 mul -3 add v sin 3 mul]}{}%
\fcSurfaceInScene[arrows=(none), iterationsV=15, iterationsU=8, linecolor=black,colorUV={1 0.5 0.5}, colorVU={1 0.5 0.5}]{2 sqrt -1 mul 0.01 add}{0}{2 sqrt -0.01 add}{360}{[u v cos \theFuN 3 add mul -3 add v sin \theFuN 3 add mul]}{}%
\fcSurfaceInScene[arrows=(none), iterationsV=1, iterationsU=8, colorUV={0.3 0.7 1}, forceForeground=true]{2 sqrt -1 mul }{0}{2 sqrt}{1}{[u \theFuN v mul  0]}{}%
\fcLineIIIdInScene[linecolor=green, linewidth=2]{[-6 -3 0]}{[6 -3 0]}
\fcCurveIIIdInScene[linecolor=red, arrows=(none), linewidth=2]{2 sqrt -1 mul}{2 sqrt}{[1 dict begin /u t def u \theFuN 0 end]}
\fcFinishIIIdScene[true]%
\fcPutIIId{[3 0 0]}{$x$}
\fcPutIIId{[0 3 0]}{$y$}
\fcPutIIId{[0 0 3]}{$z$}
\end{pspicture}
}


\end{document}
