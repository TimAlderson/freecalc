\documentclass{article}
\usepackage{../homework-problems-UMB}

\toggletrue{solutions}
\toggletrue{answers}
%\togglefalse{solutions}

\renewcommand{\course}{Math 140}
\newcommand{\tooeasy}{ ~[The problem is too easy to appear on a quiz or test.]}

\begin{comment}
\homeworkStart{on Lectures 1 and 2 \\Will be quizzed Monday February 3}{}
\item (Stewart, 7th ed., page 21, 31-37) Find the implied domain of the function
\begin{multicols}{2}
\begin{enumerate}
\item $f(x)=\frac{x+4}{x^2-9}$. 
\answer{\begin{tabular}{l} $x\neq \pm 3$, \\alternatively:\\ $x\in (-\infty, -3)\cup (-3,3)\cup (3,\infty)$\end{tabular} }
\item $f(x)=\frac{2x^3-5}{x^2+x-6}$.
\answer{\begin{tabular}{l} $x\neq 2,-3$, \\alternatively:\\ $x\in (-\infty, -3)\cup (-3,2)\cup (2,\infty)$\end{tabular} }
\item $f(t)=\sqrt[3]{2t-1}$.
\answer{$x\in \mathbb R$ (the domain is all real numbers) }
\item $g(t)=\sqrt{3-t}-\sqrt{2+t}$.
\item $h(x)=\frac{1}{\sqrt[4]{x^2-5x}}$.
\item $f(u)=\frac{u+1}{1+\frac{1}{u+1}}$.
\item $F(p)=\sqrt{2-{\sqrt{p}}}$.
\end{enumerate}
\end{multicols}
\item Find the functions $f\circ g$, $g\circ f$, $f\circ f$ and $g\circ g$ and their implied domains.

\begin{enumerate}
\item $f(x)=x^2+1$, $g(x)=x+1$. 
\answer{\begin{tabular}{l} Domain, all 4 cases: $x\in \mathbb R$ (all reals)\\ in some order: $(1+x)^{2}+1, (x)^{2}+2, ((x)^{2}+1)^{2}+1, 2+x$\end{tabular} }
\item $f(x)=\sqrt{x+1}$, $g(x)=x+1$. 
\answer{\begin{tabular}{l} Domain of $f\circ g$ is $x\geq -2$. Domain of $g\circ f$ is $x\geq -1$ \\Domain of $f\circ f$ is $ x\geq -1$. Domain of $g\circ g$ is all reals ($x\in \mathbb R$). \\ in some order:$\sqrt{2+x}, 1+\sqrt{1+x}, \sqrt{1+\sqrt{1+x}}, 2+x$\end{tabular}}
\item $f(x)= 2x$, $g(x)= \tan x$. 

You are not required to find the domain.
\answer{\begin{tabular}{l}
Domain  $f\circ f$: all reals ($x\in \mathbb R$). Domain $g\circ f$: $x\neq (2k+1)\frac{\pi}{2}$ for all $k\in \mathbb Z$\\ Domain $ f\circ g$: $x\neq (4k+1)\frac{\pi}{4}$, $x\neq (4k+3)\frac{\pi}{4}$ for all $k\in \mathbb Z$\\
Domain $g\circ g$: $x\neq (2k+1)\frac{\pi}{2}$ and $x\neq k\pi+ \arctan \left(\frac{\pi}{2}\right)$ for all $k\in \mathbb Z$
\\
in some order:$2 \tan{}x, \tan{}(2 x), 4 x, \tan{}(\tan{}x) $
\end{tabular}
}
\item $f(x)=\frac{x+1}{x-1}$, $g(x)=\frac{x-1}{x+1}$.
\answer{ 
\begin{tabular}{l}
Domain $ f\circ f$: $x\neq 1$. Domain $g\circ g$: $x\neq 0$, $x\neq -1$\\
Domain $f\circ g$: $x \neq -1$. Domain $g\circ f$: $x\neq 0$, $x\neq 1$\\
in some order: $- x, \frac{1}{x}, x, -\frac{1}{x} $
\end{tabular}
}
\end{enumerate}

\item Evaluate the difference and simplify your answer.
\begin{multicols}{2}
\begin{enumerate}
\item $\frac{f(2+h)-f(2)}{h}$, where $f(x)=x^2-x-1$.

\answer{$ h+5 $}
\item $\frac{f(a+h)-f(a)}{h}$, where $f(x)= x^2$.

\answer{$ h+2 a  $}
\item $\frac{f(a+h)-f(a)}{h}$, where $f(x)= x^3$.

\answer{$h^{2}+3 a^{2}+3 a h  $}
\item $\frac{f(a+h)-f(a)}{h}$, where $f(x)= x^4$.

\answer{$ 6 a^{2} h+4 a h^{2}+h^{3}+4 a^{3}$}
\item $\frac{f(x)-f(a)}{x-a}$, where $f(x)=\frac{1}{x}$.

\answer{$-\frac{1}{ax}$.}
\item $\frac{f(x)-f(1)}{x-1}$, where $f(x)=\frac{x-1}{x+1}$.
\answer{$\frac{1}{x+1}$.}
\end{enumerate}
\end{multicols}

\item  Compute the expressions $(f\circ g)(x)$, $(g\circ f)(x)$ and simplify to a single fraction. 

\begin{enumerate}
\item $\displaystyle f{}({{x}})=\frac{x+2}{x-2},
g{}({{x}})=\frac{x-1}{x+2}$.

\answer{$(f\circ g)(x)= \frac{3+3 x}{-5- x}$, $(g\circ f)(x)=\frac{4}{-2+3 x}$  }
\item 
$\displaystyle f{}({{x}})=\frac{x+1}{3x-2},
g{}({{x}})=\frac{x-2}{x-1}
$.

\answer{
$(f\circ g)(x)= \frac{-3+2 x}{-4+x}
$, 
$(g\circ f)(x)=\frac{5-5 x}{3-2 x}
$  }
\item 
$\displaystyle f{}({{x}})=\frac{2x+1}{3x-1},
g{}({{x}})=\frac{x-2}{2x-1}
$.

\answer{
$(f\circ g)(x)=\frac{-5+4 x}{-5+x}
$, 
$(g\circ f)(x)=\frac{3-4 x}{3+x}
$  }
\item 
$\displaystyle f{}({{x}})=\frac{x+1}{x-2},
g{}({{x}})=\frac{x+2}{2x-1}
$.

\answer{
$(f\circ g)(x)= \frac{1+3 x}{4-3 x}
$, 
$(g\circ f)(x)=\frac{-3+3 x}{4+x}
$  }
\item 
$\displaystyle f{}({{x}})=\frac{5x+1}{4x-1},
g{}({{x}})=\frac{4x-1}{3x+1}
$.

\answer{
$(f\circ g)(x)= \frac{-4+23 x}{-5+13 x}
$, 
$(g\circ f)(x)=\frac{5+16 x}{2+19 x}
$  }
\item $\displaystyle  f(x)= \frac{3x-5}{x-2}$, $\displaystyle g(y)=\frac{y-2 }{y-4} $. 

\answer{ $(f\circ g)(x)=\frac{-2 x+14}{- x+6}$, $(g\circ f)(x)=\frac{x-1}{- x+3}$}
\item $\displaystyle  f(x)= \frac{x-3}{x+2}$, $\displaystyle g(y)=\frac{y+3 }{y-4} $. 

\answer{ $(f\circ g)(x)=\frac{-2 (x)+15}{3 x-5}$, $(g\circ f)(x)=\frac{4 x+3}{-3 x-11}$}
\end{enumerate}

\item  (Stewart, 7ed., page 21, problems 45, 46, 49)
Plot the piecewise defined functions.
\begin{multicols}{2}
\begin{enumerate}
\item $G(x)=\frac{3x+|x|}x$.
\item $g(x)=|x|-x$.
\item $f(x)=\doublebrace{x+2}{x\leq -1}{x^2}{x\geq -1}$.
\end{enumerate}
\end{multicols}

\item (Stewart, 7ed, page 21, problems 55-56)
Write down formulas for function whose graphs are as follows. The graphs are up to scale. The arc is a part of a circle.
\begin{multicols}{2}
\begin{enumerate}
\psset{xunit=0.4cm, yunit=0.4cm}
\item
\tiny
\begin{pspicture}(-1,-1)(6,5)
\psaxes{->}(0,0)(-1,-1)(6,5)
\psline[linecolor=red](0,3)(3, 0)(5, 4)
\fcFullDot{5}{4}
\rput[r](4.9, 4){$(5, 4)$}
\rput[b](6,0.1 ){$x$}
\rput[l](0.1,5 ){$y$}
\end{pspicture}
\normalsize
\item
\tiny
\psset{xunit=0.4cm, yunit=0.4cm}
\begin{pspicture}(-4,-1)(4,5)
\psaxes{->}(0,0)(-4.5,-1)(4.5,4)
\psplot[linecolor=red]{-2}{2}{4 x x mul sub sqrt }
\psline[linecolor=red](2,0)(4,3)
\psline[linecolor=red](-2,0)(-4,3)
\rput[b](4.5,0.1 ){$x$}
\rput[l](0.1,5 ){$y$}
\fcFullDot{4}{3}
\rput[r](3.9, 3){$(4, 3)$}
\fcFullDot{-4}{3}
\rput[l](-3.9, 3){$(-4, 3)$}

\end{pspicture}
\normalsize
\end{enumerate}
\end{multicols}

%Graph the functions roughly by hand, by applying consecutively the transformations learned in class. Each consequtive graph is a transformations of the preceding one. Compare your answer with the graph produced by a graphic calculator.
\begin{multicols}{2}
\begin{enumerate}
\item $y=\frac{1}{x}$.
\item $y=\frac{1}{x+1}$.
\item $y=\frac{1}{2x+1}$.
\item $y=\frac{3}{2x+1}$.
\item $y=\frac{3+x}{2x+1}$.
\item $y=\left|\frac{3+x}{2x+1}\right|$.
\end{enumerate}
\end{multicols}

\homeworkEnd
\end{comment}
\begin{comment}
\homeworkStart{on Lectures 3 \\Will be quizzed: date to be announced}{}
\item \tooeasy Convert from degrees to radians.
\begin{multicols}{3}
\begin{enumerate}
\item $15^\circ$.
\item $30^\circ$.
\item $36^\circ$.
\item $45^\circ$.
\item $60^\circ$.
\item $75^\circ$.
\item $90^\circ$.
\item $120^\circ$.
\item $135^\circ$.
\item $150^\circ$.
\item $180^\circ$.
\item $225^\circ$.
\item $270^\circ$.
\item $305^\circ$.
\item $360^\circ$.
\item $405^\circ$.
\item $1200^\circ$.
\item $-900^\circ$.
\item $-2014^\circ$.
\end{enumerate}
\end{multicols}

\item \tooeasy (Textbook page A32, problems 7-12). 
Convert from radians to degrees.
\begin{multicols}{3}
\begin{enumerate}
\item $4\pi$.
\item $-7/2\pi$.
\item $5/12\pi$.
\item $8/3\pi$.
\item $-3/8\pi$.
\item $5$.
\end{enumerate}
\end{multicols}
\item Derive the trigonometry identities.
\begin{multicols}{3}
\begin{enumerate}
\item $\sin \theta\cot \theta =\cos \theta$.
\item $(\sin \theta +\cos \theta)^2=1+\sin(2\theta)$.
\item $\sec \theta - \cos \theta= \tan \theta \sin \theta$.
\item $\tan^2 \theta-\sin^2 \theta=\tan^2\theta\sin^2\theta$.
\item $\cot^2\theta+\sec^2\theta=\tan^2\theta+\csc^2\theta$.
\item $2\csc 2\theta= \sec \theta \csc \theta$.
\item $\tan (2\theta) =\frac{2\tan \theta}{1-\tan^2\theta} $.
\item $\frac{1}{1-\sin \theta}+ \frac{1}{1+\sin \theta}=2\sec^2\theta$.
\item $\tan \alpha + \tan \beta = \frac{\sin (\alpha+\beta)}{\cos \alpha \cos \beta}$.
\item $\tan (\alpha+\beta)= \frac{\tan \alpha +\tan \beta}{1-\tan \alpha\tan \beta}$.
\item $\sin (3\theta) +\sin \theta = 2 \sin 2\theta \cos \theta $.
\item $\cos (3\theta) = 4\cos^3\theta-3\cos \theta $.
\end{enumerate} 
\end{multicols}

\item (Textbook page A33, problems 65-72).
Find all values of $x$ in the interval $[0,2\pi]$ that satisfy the 
equation.
\begin{multicols}{2}
\begin{enumerate}
\item $2\cos x - 1=0$. \\ \answer{$x=\frac{\pi}{3}$ or $x=\frac{5\pi}3$} 
\item $3\cot^2 x= 1$. \\ \answer{$x=\frac{\pi}{3}$, $x=\frac{2\pi}3$, $x=\frac{4\pi}3$, or $x=\frac{5\pi}3$}
\item $2\sin^2 x= 1$. \\ \answer{$x=\frac{\pi}{4}$, $x=\frac{3\pi}4$, $x=\frac{5\pi}4$, or $x=\frac{7\pi}4$}
\item $|\tan x|=1 $. \\ \answer{$x=\frac{\pi}{4}$, $x=\frac{3\pi}4$, $x=\frac{5\pi}4$, or $x=\frac{7\pi}4$}
\item $\sin 2x = \cos x $.  \\ \answer{$x=\frac{\pi}{2}$, $x=\frac{3\pi}2$, $x=\frac{\pi}6$, or $x=\frac{5\pi}6$}
\item $2\cos x +\sin 2x=0$.\\ \answer{$x=\frac{\pi}{2}$, $x=\frac{3\pi}2$}
\item $\sin x =\tan x$. \\ \answer{$x=0$, $x=\pi$, or $x=2\pi$}
\item $2+\cos 2x = 3 \cos x$. \\ \answer{$x=0$, $x=2\pi$, $x=\frac{\pi}{3}$, or $x=\frac{5\pi}{3}$}
\end{enumerate}
\end{multicols}

\homeworkEnd
\end{comment}
\begin{comment}
\homeworkStart{on Lectures 4\\Will be quizzed: Wednesday February 19}{}

\item  \tooeasy (Textbook page 69, problems 3-9). 
Evaluate the limits. Justify your computations.
\begin{multicols}{3}
\begin{enumerate}
\item $\displaystyle\lim\limits_{x\to 3} 5x^3-3x^2+x-6$.
\answer{$105$}
\item $\displaystyle\lim\limits_{x\to -1} (x^4-3x)(x^2+5x+3)$.
\answer{$-4$}
\item $\displaystyle\lim\limits_{t\to -2} \frac{t^4- 2}{2t^2 -3t +2} $.
\answer{$\frac78$}
\item $\displaystyle\lim\limits_{u\to -2}\sqrt{u^4+3u +6}$.
\answer{$4$}
\item $\displaystyle\lim\limits_{x \to 8}(1+ \sqrt[3]{x} )(2- 6x^2 + x^3)$.
\answer{$390$}
\item $\displaystyle\lim\limits_{t \to 2}\left( \frac{t^2 - 2 }{ t^3-3t+5} \right)^2$.
\answer{$\frac{4}{49}$}
\item $\displaystyle\lim\limits_{x\to 2}\sqrt{ \frac{2x^2 + 1}{ 3x-2}}$.
\answer{$\frac32$}
\end{enumerate}
\end{multicols}

\item (Textbook page 70, problems 11-32). 
Evaluate the limit if it exists.
\begin{multicols}{3}
\begin{enumerate}
\item $\displaystyle\lim\limits_{x\to 5}\frac{x^2-6x+5}{x-5} $. 
\answer{4}
\item $\displaystyle\lim\limits_{x\to 4}\frac{x^2-4x}{x^2-3x-4} $.
\answer{$\frac{4}5$}
\item $\displaystyle\lim\limits_{x\to 5}\frac{x^2-5x+6}{x-5} $.
\answer{DNE}
\item $\displaystyle\lim\limits_{x\to -1}\frac{x^2-4x}{x^{2}-3x-4} $.
\answer{DNE}
\item $\displaystyle\lim\limits_{t\to -3}\frac{t^2-9}{2t^2+7t+3} $.
\answer{$\frac{6}{5}$}
\item $\displaystyle\lim\limits_{x\to -1}\frac{2x^2+3x+1}{x^2-2x-3} $.
\answer{$\frac{1}{4}$}
\item $\displaystyle\lim\limits_{h\to 0}\frac{(-5+h)^2-25}{h} $.
\answer{$-10$}
\item $\displaystyle\lim\limits_{h\to 0}\frac{(2+h)^3-8}{h} $.
\answer{$12$}
\item $\displaystyle\lim\limits_{x\to -2}\frac{x+2}{x^3+8} $.
\answer{$\frac{1}{12}$}
\item $\displaystyle\lim\limits_{t\to 1}\frac{t^4-1}{t^3-1} $.
\answer{$\frac{4}{3}$}
\item $\displaystyle\lim\limits_{h\to 0}\frac{\sqrt{9+h}-3}{h} $.
\answer{$\frac{1}{6}$}
\item $\displaystyle\lim\limits_{u\to 2} \frac{\sqrt{4u+1}-3}{u-2}$.
\answer{$\frac{2}{3}$}
\item $\displaystyle\lim\limits_{x\to -4} \frac{\frac{1}{4}+ \frac{1}{x}} {4+x}$.
\answer{$-\frac{1}{16}$}
\item $\displaystyle\lim\limits_{x\to -1} \frac{x^2+2x+1}{x^4-1}$.
\answer{$0$}
\item $\displaystyle\lim\limits_{t\to 0} \frac{\sqrt{1+t}- \sqrt{1-t}}{t}$.
\answer{$1$}
\item $\displaystyle\lim\limits_{t\to 0}\left(\frac{1}t -\frac{1}{t^2+t}\right)$.
\answer{$1$}
\item $\displaystyle\lim\limits_{x\to 16} \frac{4-\sqrt{x}}{16x-x^2}$.
\answer{$\frac{1}{128}$}
\item $\displaystyle\lim\limits_{h \to 0}\frac{(3+h)^{-1}-3^{-1}}{h} $.
\answer{$-\frac{1}{9}$}
\item $\displaystyle\lim\limits_{t\to 0} \left(\frac{1}{t\sqrt{1+t}}-\frac{1}{t} \right)$.
\answer{$-\frac{1}{2}$}
\item $\displaystyle\lim\limits_{x\to -4} \frac{\sqrt{x^2+9}-5}{x+4}$.
\answer{$-\frac{4}{5}$}
\item $\displaystyle\lim\limits_{h\to 0}\frac{(x+h)^3-x^3}{h} $.
\answer{$3x^2$}
\item $\displaystyle\lim\limits_{h\to 0}\frac{\frac{1}{(x+h)^2}-\frac{1}{x^2}}{h} $.
\answer{$-\frac{2}{x^3}$}
\end{enumerate}
\end{multicols}

\homeworkEnd
\end{comment}
\begin{comment}
\homeworkStart{on Lectures 5 and 6. \\ Problem types will be included on the Test}{}
\item Show the following limits do not exist and compute whether they evaluate to $\infty $, $-\infty$, or neither. 
\begin{multicols}{3}
\begin{enumerate}
\item $\displaystyle\lim_{x\to 3^+} \frac{x^{2}+x-1}{x^2-2x-3} $.
\answer{$\infty$.}
\item $\displaystyle\lim_{x\to 3^-} \frac{x^{2}+x-1}{x^2-2x-3} $.
\answer{$-\infty$.}
\item $\displaystyle\lim_{x\to 1^+} \frac{x^2+1}{\sqrt{x^2+3 }-2} $.
\answer{$\infty$.}
\item $\displaystyle\lim_{x\to 1^-} \frac{x^2+1}{\sqrt{x^2+3 }-2} $.
\answer{$-\infty$.}
\item $\displaystyle\lim_{x\to 2^+} \frac{\sqrt{x^3-8}}{ -x^2+x+2} $.
\answer{$-\infty$.}
\item $\displaystyle\lim_{x\to -1^+} \frac{\sqrt[3]{x^2+2x+1}}{ x^2-2x-3} $.
\answer{$-\infty$.}

\end{enumerate}
\end{multicols}

\item \begin{problem}(Textbook, page 235, problems 33-38).
Find the horizontal and vertical asymptotes of each curve. Check your work by plotting the function using the internet.
\begin{multicols}{3}
\begin{enumerate}
\item $y=\frac{2x+1}{x-2}$.
\item $y=\frac{x^2+1}{2x^2-3x-2}$.
\item $y=\frac{2x^2+x-1}{x^2+x-2}$.
\item $y=\frac{1+x^4}{x^2-x^4}$.
\item $y=\frac{x^3-x}{x^2-6x+5}$.
\item $y=\frac{x-9}{\sqrt{4x^2+3x+2}}$.
\end{enumerate}
\end{multicols}
\end{problem}
\item Find the limit or show that it does not exist. If the limit does not exist, indicate whether it is $\pm\infty$, or neither. The answer key has not been proofread, use with caution.
\begin{multicols}{3}
\begin{enumerate}
\item $\lim\limits_{x\to\infty }\frac{x-2}{2x+1}$.

\answer{$\frac12$}
\item $\lim\limits_{x\to\infty }\frac{1-x^2}{x^3-x-1}$.

\answer{$ 0$}
\item $\lim\limits_{x\to-\infty }\frac{x-2}{x^2+5}$.

\answer{$ 0$}
\item $\lim\limits_{x\to-\infty }\frac{3x^3+2}{2x^3-4x+5}$.

\answer{$ \frac{3}{2}$}
\item $\lim\limits_{x\to\infty }\frac{\sqrt{x}+x^2}{\sqrt{x}-x^2}$.

\answer{$-1$}
\item $\lim\limits_{x\to\infty }\frac{3-x\sqrt{t}}{2x^{\frac{3}{2}}-2}$.

\answer{$-\frac12$}
\item $\lim\limits_{x\to\infty }\frac{(2x^2+3)^2}{(x-1)^2(x^2+1)}$.

\answer{$ 4$}
\item $\lim\limits_{x\to\infty }\frac{x^2-3}{\sqrt{x^4+3}}$.

\answer{$1$}
\item $\lim\limits_{x\to\infty }\frac{\sqrt{16x^6-3x}}{x^3+2}$.

\answer{$3$}
\item $\lim\limits_{x\to-\infty }\frac{\sqrt{16x^6-3x}}{x^3+2}$.

\answer{$-3$}
\item $\lim\limits_{x\to\infty}\sqrt{4x^2+x}-2x$.

\answer{$\frac{1}{4}$}
\item $\lim\limits_{x\to-\infty} x+\sqrt{x^2+3x} $.

\answer{$-\frac{3}{2} $}
\item $\lim\limits_{x\to\infty}\sqrt{x^2+ax}-\sqrt{x^2+bx}$.

\answer{$\frac{a-b}2$}
\item $\lim\limits_{x\to\infty}\cos x$.

\answer{DNE}
\item $\lim\limits_{x\to\infty}\frac{x^4+x}{x^3-x+2}$.

\answer{$\infty$}
\item $\lim\limits_{x\to\infty}\sqrt{x^2+1}$.

\answer{$\infty$}
\item $\lim\limits_{x\to-\infty}(x^4+x^5)$.

\answer{$-\infty$}
\item $\lim\limits_{x\to-\infty}\frac{\sqrt{1+x^6}}{1+x^2}$.

\answer{$\infty$}
\item $\lim\limits_{x\to\infty}(x-\sqrt{x})$.

\answer{$\infty$}
\item $\lim\limits_{x\to\infty}(x^2-x^3)$.

\answer{$-\infty$}
\item $\lim\limits_{x\to\infty}x\sin x$.

\answer{DNE}
\item $\lim\limits_{x\to\infty}\sqrt{x}\sin x$.

\answer{DNE}
\end{enumerate}
\end{multicols}
\item (Textbook, page 92, problem 51-56) Use the Intermediate Value Theorem to show that there is a real number solution of the given equation in the specified interval. 
\begin{multicols}{2}
\begin{itemize}
\item $x^4+x-3=0$ where $x\in (1,2)$.
\item $\sqrt[3]{x}=1-x$ where $x\in (0,2) $.
\item $\cos x=x$, where $x\in (0,1)$.
\item $\sin x=x^2-x$, where $x\in (1,2)$.
\item $\cos x=x^3$, where $x\in \mathbb R$ (i.e., $x$ is an arbitrary real number).
\item $x^5-x^2+2x+3$, where $x\in \mathbb R$.
\end{itemize}
\end{multicols}

\homeworkEnd
\end{comment}
\begin{comment}
\homeworkStart{Homework Math 140, Lectures 6 and 7. \\ Will be quizzed Wednesday September 25}{}
(Textbook, page 91, problem 23-24). Find the (implied) domain of $f(x)$. Extend the definition of $f$ at $x=2$ to make $f$ continuous at $2$.
\begin{multicols}{2}
\begin{enumerate}
\item $f(x)=\frac{x^2-x-2}{x-2}$.
\item $f(x)=\frac{x^3-8}{x^2-4}$.
\end{enumerate}
\end{multicols}
For which values of $x$ is $f$ continuous?
\begin{itemize}
\item $f(x)=\doublebrace{0}{\mathrm{if~} x\mathrm{~is~rational}}{1}{\mathrm{if~}x~\mathrm{is~irrational}}$
\item $f(x)=\doublebrace{0}{\mathrm{if~} x\mathrm{~is~rational}}{x}{\mathrm{if~}x~\mathrm{is~irrational}}$
\end{itemize}
~\\
\psset{xunit=0.5cm, yunit=0.5cm}
\begin{tabular}{cccc}
\multicolumn{4}{l}{Match each of the following function plots:}\\
$1.$&$2.$&$3.$&$4.$\\
\begin{pspicture}(-3.1,-4.3)(3.1,4.1)
\fcAxesStandard{-3}{-4}{3}{4}
\fcGrid[linestyle=dashed, linewidth=0.5, linecolor=gray]{-3}{-4}{6}{8}{1}{1}{}
%Function formula: -8 ((x) ((x) (x)))+2 (x)
\psplot[linecolor=red, plotpoints=1000]{-0.9}{0.9}{x 2 mul x x mul x mul -8 mul add }
\end{pspicture}
&
\begin{pspicture}(-3.1,-4.3)(3.1,4.1)
\fcAxesStandard{-3}{-4}{3}{4}
\fcGrid[linestyle=dashed, linewidth=0.5, linecolor=gray]{-3}{-4}{6}{8}{1}{1}{}
%Function formula: -2+x
\psplot[linecolor=red, plotpoints=1000]{1}{3}{x -2 add } %Function formula: - (x)
\psplot[linecolor=red, plotpoints=1000]{-1}{1}{x -1 mul } %Function formula: 2+x
\psplot[linecolor=red, plotpoints=1000]{-3}{-1}{x 2 add }
\end{pspicture}
&
\begin{pspicture}(-3.1,-4.3)(3.1,4.1)
\fcAxesStandard{-3}{-4}{3}{4}
\fcGrid[linestyle=dashed, linewidth=0.5, linecolor=gray]{-3}{-4}{6}{8}{1}{1}{}
%Function formula: - ((1)/((x)^{2}+1))
\psplot[linecolor=red, plotpoints=1000]{-3}{3}{1 1 x 2 exp add div -1 mul }
\end{pspicture}
&
\begin{pspicture}(-3.1,-4.3)(3.1,4.1)
\fcAxesStandard{-3}{-4}{3}{4}
\fcGrid[linestyle=dashed, linewidth=0.5, linecolor=gray]{-3}{-4}{6}{8}{1}{1}{}
%Function formula: - (((x)^{2}) ((x) (x)))+(x)^{2}
\psplot[linecolor=red, plotpoints=1000]{-1.59}{1.59}{x 2 exp x x mul x 2 exp mul -1 mul add }
\end{pspicture}
\\
\multicolumn{4}{l}{to their derivative plots:}\\
$(a)$&$(b)$&$(c)$&$(d)$\\
\begin{pspicture}(-3.1,-4.3)(3.1,4.1)
\fcAxesStandard{-3}{-4}{3}{4}
\fcGrid[linestyle=dashed, linewidth=0.5, linecolor=gray]{-3}{-4}{6}{8}{1}{1}{}
%Function formula: (x)/(((x)^{2}+1)^{2})
\psplot[linecolor=blue, plotpoints=1000]{-3}{3}{x 1 x 2 exp add 2 exp div }
\end{pspicture}
&
\begin{pspicture}(-3.1,-4.3)(3.1,4.1)
\fcAxesStandard{-3}{-4}{3}{4}
\fcGrid[linestyle=dashed, linewidth=0.5, linecolor=gray]{-3}{-4}{6}{8}{1}{1}{}
%Function formula: -24 ((x) (x))+2
\psplot[linecolor=blue, plotpoints=1000]{-0.485}{0.485}{2 x x mul -24 mul add }
\end{pspicture}
&
\begin{pspicture}(-3.1,-4.3)(3.1,4.1)
\fcAxesStandard{-3}{-4}{3}{4}
\fcGrid[linestyle=dashed, linewidth=0.5, linecolor=gray]{-3}{-4}{6}{8}{1}{1}{}
%Function formula: -4 ((x)^{3})+2 (x)
\psplot[linecolor=blue, plotpoints=1000]{-1.15}{1.15}{x 2 mul x 3 exp -4 mul add }
\end{pspicture}
&
\begin{pspicture}(-3.1,-4.3)(3.1,4.1)
\fcAxesStandard{-3}{-4}{3}{4}
\fcGrid[linestyle=dashed, linewidth=0.5, linecolor=gray]{-3}{-4}{6}{8}{1}{1}{}
%Function formula: 1
\psplot[linecolor=blue, plotpoints=1000]{1}{3}{1}
\fcHollowDotBlue{-1}{1}
%Function formula: -1
\fcHollowDotBlue{-1}{-1}
\psplot[linecolor=blue, plotpoints=1000]{-1}{1}{-1}
\fcHollowDotBlue{1}{-1}
%Function formula: 1
\fcHollowDotBlue{1}{1}
\psplot[linecolor=blue, plotpoints=1000]{-3}{-1}{1}
\end{pspicture}
\end{tabular}

Give reasons for your choices. Can you guess formulas that would give a similar (or precisely the same) graph, and confirm visually your guess using a graphing device?

\homeworkEnd
\end{comment}
\begin{comment}
\homeworkStart{on Lecture 7. \\ Will be quizzed Wednesday March 5 \\ Problem 2(d) uses material from Lecture 8 and \\will not appear on the quiz}{}
\item % begin homework inverse-functions3
Find the inverse function. You are asked to do the algebra only; you are not asked to determine the domain or range of the function or its inverse. 
\begin{enumerate}[ref={\fcProblemRef}]
\item $f(x)= 3x^2+4x-7$, where $x\geq -\frac{2}{3}$.
\answer{$f^{-1}(x)= -\frac{2}3+\frac{\sqrt{25+3x}}{3}, \quad x\geq -\frac{25}{3}$}

\item $f(x)= 2x^2+3x-5$, where $x\geq -\frac{3}{4}$.
\answer{$f^{-1}(x)=-\frac{3}{4}+\frac{\sqrt{49+8x}}{4}, \quad x\geq -\frac{49}{8}$}

\item $f(x)= \frac{2x+5}{x-4}$, where $x\neq 4$.
\answer{$f^{-1}(x)=\frac{4x+5}{x-2}, \quad x\neq 2$}

\pointsii{3} \label{problemFindInversef=(3x+5)/(2x-4)} $f(x)= \frac{3x+5}{2x-4}$, where $x\neq 2$.
\hiddenanswer{$f^{-1}(x)=\frac{4x+5}{2x-3}, \quad x\neq \frac{3}{2}$}



\item $f(x)=2^{2x}+2^{x}-2$.
\answer{$f^{-1}(x) =\log_2\frac{-1+\sqrt{9+4x}}{2}, \quad x\geq -2$}

\end{enumerate}
% end homework inverse-functions3

\homeworkEnd
\end{comment}

\begin{comment}
\homeworkStart{on Lecture 8. \\ Will be quizzed Friday March 7}{}
\item \tooeasy 
\begin{problem}(Textbook page 408, problems 3-8). Find the exact value of each expression.
\begin{multicols}{3}
\begin{enumerate}
\item $\log_5 125$.
\item $\log_3 \frac{1}{27}$.
\item $\ln \left(\frac{1}{e}\right) $.
\item $\log_{10}\sqrt{10}$.
\item $e^{\ln 4.5}$.
\item $\log_{10} 0.0001 $.
\item $\log_{1.5}2.25$.
\item $\log_5 4- \log_5 500$.
\item $\log_2 6 - \log_2 15 +\log_2 20$.
\item $\log_3 100- \log_3 18 - \log _3 50 $.
\item $e^{-2\ln 5}$.
\item $\ln \left(\ln e^{e^{10}}\right)$.
\end{enumerate}
\end{multicols}
\end{problem}
\item \tooeasy % begin homework logarithms-basic2
Use the definition of a logarithm to evaluate each of the following without using a calculator.  

\begin{enumerate}
\item   $\log_2 16$

\item   $\log_3 (1/9)$

\item   $\log_{10} 1000$

\item   $\log_{6} 36^{-2/3}$

\item   $\log_{2} (8\sqrt{2})$

\item $\log_7(49^x/343^y)$

%\solution{%
%\begin{align*}
%\log_7(49^x/343^y) & = \log_749^x - \log_7343^y \\
% & = x\log_749 - y\log_7343 \\
%\intertext{But $49 = 7^2$ and $343=7^3$, therefore}
%\log_7(49^x/343^y) & = 2x-3y.
%\end{align*}
%}%


\end{enumerate}
% end homework logarithms-basic2

\item \tooeasy % begin homework logarithms-combine
Express each of the following as a single logarithm.  

\begin{enumerate}
\item   $\ln 4 + \ln 6 - \ln 5$

\pointsii{2} $2\ln 2 - 3\ln 3 + 4\ln 4$

\solution{%
\begin{align*}
2\ln 2 - 3\ln 3 + 4\ln 4 & = \ln 2^2 - \ln 3^3 + \ln 4^4 \\
 & = \ln 4 - \ln 27 + \ln 256 \\
 & = \ln \Big( \frac{4}{27}\Big) + \ln 256 \\
 & = \ln \Big( \frac{4\cdot 256}{27}\Big) \\
 & = \ln \Big( \frac{1024}{27}\Big).
\end{align*}
}%

\item   $\ln 36 - 2\ln 3 - 3\ln 2$

\end{enumerate}
% end homework logarithms-combine

\item % begin homework inverse-functions2
Find the inverse function and its domain. 
\begin{enumerate}
\item  $y=\ln (x+3)$.
\answer{$f^{-1}(x)=e^x-3$}

\solution{%
\begin{align*}
y & = \ln (x+3) \\
e^y & = e^{\ln (x+3)} \\
e^y & = x + 3 \\
e^y - 3 & = x \\
\text{Therefore} \quad f^{-1}(y) & = e^y - 3.
\end{align*}
The domain of $e^y$ is all real numbers, so the domain of $f^{-1}$ is all real numbers.  
}%

\item $f(x)=e^{x^3}$.
\answer{$f^{-1}(x)=\sqrt[3]{\ln x}, \quad x>0$}

\item $y=(\ln x)^2$, $x\geq 1$.
\answer{$f^{-1}(x)=e^{\sqrt{x}}, \quad x\geq 0 $}

\pointsii{5}  $y=\frac{e^x}{1+2e^x}$.
\hiddenanswer{$f^{-1}(x)= \ln \left(\frac{x}{1-2x}\right) $, \quad $x\in (0, \frac12) $}

\solution{%
\begin{align*}
y & = \frac{e^x}{1+2e^x} \\
y(1+2e^x) & = e^x \\
y & = e^x(1-2y) \\
\frac{y}{1-2y} & = e^x \\
\ln\frac{y}{1-2y} & = \ln e^x \\
\ln\frac{y}{1-2y} & = x \\
\text{Therefore} \quad f^{-1}(y) & = \ln\frac{y}{1-2y}.
\end{align*}
The natural logarithm function is only defined for positive input values.  
Therefore the domain is the set of all $y$ for which 
\begin{align*}
\frac{y}{1-2y} & > 0.
\end{align*}
This inequality holds if the numerator and denominator are both positive or both negative.  
This happens if either
\begin{enumerate}
\item  $y > 0$ and $y < 1/2$, or 
\item  $y < 0$ and $y > 1/2$.
\end{enumerate}
The latter option is impossible, so the domain is $\{ y \in \mathbb{R} \ | \ 0 < y < 1/2\}$.  
}%

\end{enumerate}
% end homework inverse-functions2

\item Solve each equation for $x$. Using a calculator give an ($\approx$) answer in decimal notation. Using calculator verify your approximate solutions.
\begin{multicols}{2}
\begin{enumerate}[ref={\fcProblemRef}]
\item $e^{7-4x}=7$.

\answer{$\frac{7-\ln 7 }{4}\approx 1.263522 $}
\item $\ln (2x-9)=2$.

\answer{$\frac{e^2+9}{2}\approx 8.194528 $}
\item $\ln (x^2-2)=3$.

\answer{$\pm \sqrt{e^3+2}\approx \pm 4.699525 $}
\item $2^{x-3}=5$.

\answer{$\log_2 5+3= \frac{\ln 5}{\ln 2}+3 \approx 5.321928 $}
\item \label{problemlnx+ln(x-1)=1} $\ln x+\ln (x-1)=1$.

\answer{$\frac{1}{2}\left(1+\sqrt{1+4e}\right)\approx 2.223$}
\item $e^{2x+1}=t$.

\answer{$\frac{\ln t-1}{2}$}
\item $\log_2(m x)=c$.

\answer{$\frac{2^c}{m}$}
\item $e- e^{-2x}=1$.

\answer{$-\frac12\ln (e-1)\approx -0.271$}
\item $8(1+e^{-x})^{-1}=3$.

\answer{$-\ln \frac53 =\ln \frac35 \approx -0.510826 $}
\item $\ln (\ln x)=1$.

\answer{$e^e\approx 15.154$}
\item $e^{e^x}=10$.

\answer{$\ln (\ln 10)\approx 0.834$}
\item $\ln(2x+1)=3-\ln x$.

\answer{$\frac{-1+\sqrt{1+8e^3}}{4}\approx 2.928878 $}
\item $e^{2x}-4e^x+3=0$.

\answer{$x=\ln 3\approx 1.098612, ~~~, x=0$}

\item $e^{4x}+3e^{2x}-4=0$. 

\answer{$x=0$}
\item $e^{2x}-e^x-6=0$.

\answer{$x=\ln 3$}
\item $4^{3x}-2^{3x+2}-5=0$. 

\answer{$x=\frac{\log_{2}5}{3}$}
\end{enumerate}
\end{multicols}


\item Differentiate.
\begin{multicols}{2}
\begin{enumerate}
\item $10^{x^3}$. \answer{$3(\ln 10) x^{2} (10)^{x^{3}}$}
\item $2^{\tan x}$. \answer{ $(\ln 2) 2^{\tan x}  \sec^2 x $  }
\item $x^x $. \answer{$x^x(\log{}(x) +1)$}
\item $x^{x^x}$. \answer{$(\ln(x))^{2}  x^{x^{x}+x}+x^{x^{x}+x-1}+(\ln x) x^{x^{x}+x}$}
\item $(\sin x)^{\cos x}$. \answer{$\frac{- \ln(\sin{}x)  (\sin{}x)^{\cos{}x+2} +(\sin{}x)^{\cos{}x} \cos^{2}{}x}{\sin{}x}$}
\item $(\ln x)^{\ln x}$. \answer{$\ln{}(\ln{}(x)) x^{-1} (\ln{}(x))^{\ln{}(x)}+x^{-1} (\ln{}(x))^{\ln{}(x)}$}
\end{enumerate}
\end{multicols}
\item Find the limit.
\begin{multicols}{2}
\begin{enumerate}
\item $\displaystyle \lim\limits_{x\to \infty} \left(1-\frac{2}{x} \right)^x$. \answer{$e^{-2}$}
\item $\displaystyle \lim\limits_{x\to 0} \left(1-x\right)^{\frac{1}{x}}$.
\answer{ $e^{-1}$}
\item $\displaystyle \lim\limits_{x\to \infty} \left(\frac{x}{x-5}\right)^{x}$.
\answer{$e^5$}
\item $\displaystyle \lim\limits_{x\to \infty} \left(\frac{x}{x-2}\right)^{3x+2}$.
\answer{$e^6$}
\end{enumerate}
\end{multicols}
\homeworkEnd
\end{comment}

\begin{comment}
\homeworkStart{on Lectures 9,10. \\ Will be quizzed Wednesday March 12.}{}

\item \tooeasy (Textbook, page 136, 1-44).
Compute the derivative.
\begin{multicols}{2}
\begin{enumerate}
\item $f(x)=2^{40}$.

\answer{$0$}
\item $f(x)=\pi^2$.

\answer{$0$}
\item $f(t)=2-\frac{2}{3}t$.

\answer{$-\frac{2}{3}$}
\item $F(x)=\frac{3}{4}x^8$.

\answer{$6 x^{7}$}
\item $f(x)=x^3-4x+6$.

\answer{$-4+3 x^{2} $}
\item $f(t)=\frac{1}{2}t^6-3t^4+t$.

\answer{$ 3 t^{5}-12 t^{3}+1$}
\item $g(x)=x^2(1-2x)$. 

\answer{$ 2 x-6 x^{2}$}
\item $h(x)=(x-2)(2x+3)$.

\answer{$ 4x-1$}
\item $g(t)=2t^{-3/4}$.

\answer{$-\frac{3}{2} t^{-\frac{7}{4}} $}
\item $B(y)=c y^{-6}$.

\answer{$-6 c y^{-7} $}
\item $A(s)=-\frac{12}{s^5}$.
\answer{$60 s^{-6}$}

\end{enumerate}
\end{multicols}

\item  (Textbook, page 136, 1-44). Compute the derivative.
\begin{multicols}{2}
\begin{enumerate}
\item $y=x^{\frac53}-x^{\frac23}$.

\answer{$ \frac53 x^{\frac23}-2/3 x^{-\frac13}$}
\item $S(p)=\sqrt{p}-p$.

\answer{$-1+\frac{1}{2} p^{-\frac{1}{2}} $}
\item $y=\sqrt{x}(x-1)$.

\answer{$ \frac{3}{2} x^{\frac{1}{2}}- \frac{1}{2}  x^{-\frac{1}{2}}$}
\item $R(a)=(3a+1)^2$.

\answer{$6+18 a $}
\item $S(R)=4\pi R^2$.

\answer{$8 \pi R$}
\item $y=\frac{ x^2+4x+3}{\sqrt{x}}$.

\answer{$ 2 x^{-\frac{1}{2}}+\frac{3}{2} x^{\frac{1}{2}}-\frac{3}{2} x^{-\frac{3}{2}}$}
\item $y=\frac{\sqrt{x}+x}{x^2}$.

\answer{$- x^{-2}-\frac{3}{2} x^{-\frac{5}{2}} $}
\item $H(x)=(x+x^{-1})^3$.

\answer{$3x^{2}+3-3x^{-2}-3x^{-4} $}
\item $g(u)=\sqrt 2 u +\sqrt{3u}$.

\answer{$ \sqrt{2}+\frac{\sqrt3}{2}  u^{-\frac{1}{2}}$}
\item $u=\sqrt[5]t+4\sqrt{t^5}$.

\answer{$10 t^{\frac{3}{2}}+\frac{1}{5} t^{-\frac{4}{5}} $}
\item $v=\left(\sqrt{x}+\frac{1}{\sqrt[3]{x}}\right)^2$.


\answer{$1+\frac{1}{3} x^{-\frac{5}{6}}-\frac{2}{3} x^{-\frac{5}{3}} $}
\item $f(x)=(1+2x^2)(x-x^2)$.

\answer{$1-2 x+6 x^{2}-8 x^{3}$}
\item $f(x)=\frac{x^4-5x^3+\sqrt{x}}{x^2}$.

\answer{$-5+2 x-\frac{3}{2} x^{-\frac{5}{2}} $}
\item $V(x)=(2x^3+3)(x^4-2x)$.

\answer{$-6-4 x^{3}+14 x^{6}$}
\item $L(x)=(1+x+x^2)(2-x^4)$.

\answer{$ 2+4 x-4 x^{3}-5 x^{4}-6 x^{5}$}
\item $F(y)=\left(\frac{1}{y^2}-\frac{3}{y^4} \right)(y+5y^3)$.

\answer{$5+9 y^{-4}+14 y^{-2} $}
\item $J(v)=(v^3-2v)(v^{-4}+v^{-2})$.

\answer{$1+6 v^{-4}+v^{-2}$}
\item $g(x)=\frac{1+2x}{3-4x}$.

\answer{$ 10 (3-4 x)^{-2}$}
\end{enumerate}
\end{multicols}
\item (Textbook, page 136, 1-44). Compute the derivative.
\begin{multicols}{2}
\begin{enumerate}

\item $f(x)=\frac{x-3}{x+3}$.

\answer{$6 (3+x)^{-2} $}
\item $y=\frac{x^3}{1-x^2}$.

\answer{$ \frac{3 x^{2}- x^{4}}{(1- x^{2})^{2}}$}
\item $y=\frac{x+1}{x^3+x-2}$.

\answer{$\frac{-3-3 x^{2}-2 x^{3}}{(-2+x+x^{3})^{2}} $}
\item $y=\frac{v^3-2v\sqrt{v}}{v}$.

\answer{$2 v- v^{-\frac{1}{2}}$}
\item $y=\frac{t}{(t-1)^2}$.

\answer{$-\frac{t +1}{(t-1)^3} $}
\item $y=\frac{t^2+2}{t^4-3t^2+1}$.

\answer{$\frac{14 t-8 t^{3}-2 t^{5}}{(1-3 t^{2}+t^{4})^{2}} $}
\item $g(t)=\frac{t-\sqrt{t}}{t^{1/3}}$.

\answer{$-\frac{1}{6} t^{-\frac{5}{6}}+\frac{2}{3} t^{-\frac{1}{3}} $}
\item $y=a x^2+b x + c$.

\answer{$ b+2 a x$}
\item $y=A+\frac{B}x +\frac{C}{x^2}$.

\answer{$\frac{- B-2 C x^{-1}}{x^{2}}$}
\item $f(t)=\frac{2t}{2+\sqrt{t}}$.

\answer{$\frac{4+t^{\frac{1}{2}}}{(2+t^{\frac{1}{2}})^{2}} $}
\item $y=\frac{c x}{1+c x}$.

\answer{$ c (1+c x)^{-2}$}
\item $y=\sqrt[3]{t}(t^2+t+t^{-1}) $.

\answer{$-\frac{2}{3} t^{-\frac{5}{3}}+\frac{4}{3} t^{\frac{1}{3}}+\frac{7}{3} t^{\frac{4}{3}} $}
\item $y=\frac{u^6-2u^3+5}{u^2}$.

\answer{$-2-10 u^{-3}+4 u^{3} $}
\item $f(x)=\frac{x}{x+\frac{c}{x}}$.

\answer{$ \frac{2 x c}{(c+x^{2})^{2}}$}
\item $f(x)=\frac{a x+b}{c x+ d}$.

\answer{$\frac{a d- b c}{(d+c x)^{2}}$}
\end{enumerate}
\end{multicols}


\homeworkEnd
\end{comment}

\begin{comment}
\homeworkStart{on Lecture 11. \\ Will be quizzed the Monday after spring break.}{}
\item 
(Textbook, page 146, problems 1-16).
Compute the derivative.
\begin{multicols}{2}
\begin{enumerate}
\item $\displaystyle f(x)= 3x^2 -2 \cos x$.

\answer{$ 6 x+2 \sin x$}
\item $\displaystyle f(x)=\sqrt{x}\sin x$.

\answer{$ x^{\frac{1}{2}}\cos x +\frac{1}{2}x^{-\frac{1}{2}} \sin x $}
\item $\displaystyle f(x)=\sin x +\frac{1}{2}\cot x$.

\answer{$\frac{-\frac{1}{2}+\cos x \sin^2x}{\sin^2x} $}
\item $\displaystyle y=2\sec x - \csc x$.

\answer{$ \frac{\cos^3 x+2 \sin^3x}{(\cos x \sin x)^{2}}$}
\item $\displaystyle y=\frac{1+\sin^2\theta}{\cos^3\theta}$.

\answer{$ \frac{1+\sin^2 \theta}{\cos^3 \theta}$}
\item $\displaystyle g(t)=4 \sec t + \tan t$.

\answer{$\frac{1+4 \sin t}{\cos^2 t} $}
\item $\displaystyle y= c\cos t + t^2\sin t$.

\answer{$ - c \sin t+2 t \sin t+ t^{2}\cos t$}
\item $\displaystyle y=u(a\cos u + b \cot u)$.

\answer{$ \frac{- a u \sin^3 u+a \cos u \sin^2u- b u +b \cos u \sin u}{\sin^2u}$}
\item $\displaystyle y=\frac{x}{2-\tan x}$.

\answer{$ \frac{x - \cos x \sin x+2 \cos^2 x}{(2 \cos x- \sin x)^{2}}$}
\item $\displaystyle y=\sin \theta \cos \theta$.

\answer{$\cos (2\theta)= \cos^2\theta- \sin^2\theta$}
\item $\displaystyle f(\theta)=\frac{\sec \theta}{1+\sec \theta}$.

\answer{$\frac{\sin\theta}{(1+\cos\theta)^{2}} $}
\item $\displaystyle y=\frac{\cos x}{1-\sin x}$.

\answer{$\frac{1}{1- \sin x} $}
\item $\displaystyle y=\frac{t\sin t}{1+t}$.

\answer{$ \frac{\sin t+t \cos t+t^{2}\cos t }{(1+t)^{2}}$}
\item $\displaystyle y=\frac{1-\sec x}{\tan x}$.

\answer{$\frac{\cos x- 1}{\sin^2x} $}
\item $\displaystyle h(\theta)=\theta \csc \theta -\cot \theta$.

\answer{$\frac{1+\sin\theta- \theta \cos\theta}{\sin^2\theta}$}
\item $\displaystyle y=x^2\sin x\tan x$.

\answer{$\frac{2 x \cos{}x \sin^2{}x+2 x^{2} \sin{}x  \cos^2x+x^{2} \sin^3{}(x)}{\cos^2{}x} $}
\end{enumerate}
\end{multicols}

\item Differentiate.

\begin{multicols}{2}
\begin{enumerate}
\item $\tan x$.
\answer{$\sec^2 x$}
\item $\cot x$.
\answer{$-\csc^2 x$}
\item $\sec x$.
\answer{$\sec x \tan x= \frac{\sin x}{\cos^2 x}$}
\item $\csc x$.
\answer{$-\csc x \cot x= -\frac{\cos x }{\sin^2x} $}
\item $\sec x\tan x$.
\answer{$\sec x \tan^2 x+\sec^3 x$}
\item $\sec x+\tan x$.
\answer{$\sec x(\tan x +\sec x) $}
\item $\sec^2 x$.
\answer{$2\tan x\sec^2 x$}
\item $\csc^2 x$.
\answer{$ -2\cot x\csc^2 x$}
\item $\frac{\sin x}{x}$.
\answer{$\frac{x \cos{}x- \sin{}x}{x^{2}}$}
\end{enumerate}

\end{multicols}
\homeworkEnd
\end{comment}

\begin{comment}
\homeworkStart{on Lecture 12. \\ Quiz date to be announced.}{}
\item % begin homework chain-rule1
In each of the following cases find a simple function $u$ of $x$ such that the given function is a simple function of $u$.  
Use the Chain Rule to differentiate the given function with respect to $x$.   

\begin{enumerate}
\item   $y = \sqrt{1+x^2}$


\pointsii{3}  $y = (\cos x)^{1/2}$
\solution{%
\begin{align*}
\text{Let } \quad u & = \cos x. \\
\text{Then } \quad y & = u^{1/2}. \\
\text{Chain Rule: } \quad \frac{\diff y}{\diff x} & = \frac{\diff y}{\diff u}\frac{\diff u}{\diff x} \\
 & = \big(\frac{1}{2}u^{-1/2}\big) (-\sin x) \\
 & = -\frac{1}{2} \sin x (\cos x)^{-1/2}.
\end{align*}
}%

\item   $y = \sin^3 x$

\pointsii{3}  $y = (1+\cos x)^2$
\solution{%
\begin{align*}
\text{Let } \quad u & = 1+\cos x. \\
\text{Then } \quad y & = u^{2}. \\
\text{Chain Rule: } \quad \frac{\diff y}{\diff x} & = \frac{\diff y}{\diff u}\frac{\diff u}{\diff x} \\
 & = (2u) (-\sin x) \\
 & = -2 \sin x \cos x \\
 & = - \sin 2x. \quad \text{(This last step is optional.)}
\end{align*}
}%

\end{enumerate}
% end homework chain-rule1

\item % begin homework chain-rule2
Use the Chain Rule to differentiate the given function with respect to $x$.   

\begin{enumerate}
\item   $y = \frac{1}{\sin^3x}$

\item  $y = \sqrt[3]{4+3\tan x}$

\item  $y = (\cos x + 3\sin x)^4$

\pointsii{4}  $y = \sin\sqrt{x}$

\ans{%
\begin{align*}
\text{Let } \quad u & = \sqrt{x}. \\
\text{Then } \quad y & = \sin u. \\
\text{Chain Rule: } \quad \frac{\diff y}{\diff x} & = \frac{\diff y}{\diff u}\frac{\diff u}{\diff x} \\
 & = (\cos u) \big(\frac{1}{2}u^{-1/2}\big) \\
 & = \frac{\cos\sqrt{x}}{2\sqrt{x}}.
\end{align*}
}%


\item  $y = \cos4x$
\end{enumerate}
% end homework chain-rule2

\item Compute the derivative.
\begin{multicols}{2}
\begin{enumerate}[ref={\fcProblemRef}]
\item $\displaystyle f(x)= (x^4+3x^2-2)^5$.

\answer{$ \left(30 x +20 x^{3}\right) \left(-2+3 x^{2}+x^{4}\right)^{4}$}
\item $\displaystyle f(x)= (4x-x^2)^{100}$.

\answer{$(-200 x+400) \left(4 x- x^{2}\right)^{99}$}
\item $\displaystyle f(x)= (2x - 3)^4 (x^2 + x + 1)^5$.

\answer{$ \left(-7-12 x+28 x^{2}\right)\left(-3+2 x\right)^{3} \left(1+x +x^{2} \right)^{4}$}
\item $\displaystyle f(x)= (x^2+1)^3(x^2+2)^6$.

\answer{$\left(24 x+18 x^{3}\right)\left(1+x^{2}\right)^{2} \left(2+x^{2}\right)^{5} $}
\item $\displaystyle f(x)= (3x-1)^4(2x+1)^{-3}$.

\answer{$(3 x-1)^{3}\frac{6 x+18}{(2 x+1)^{4}}$}
\item $\displaystyle f(x)=\frac{1}{1+x^2} $.

\answer{$\frac{-2 x}{(1+x^{2})^{2}} $}
\item $\displaystyle f(x)=\left(\frac{x^2+1}{x^2-1} \right)^3 $.

\answer{$\frac{-12 x}{\left(x^{2}-1\right)^{2}} \left(\frac{x^{2}+1}{x^{2}-1}\right)^{2} $}
\item $\displaystyle f(x)= (x+1)^{\frac{2}{3}}(2x^2-1)^3$.

\answer{$ \left(\frac{40}{3} x^{2}+12 x-\frac{2}{3}\right)\left(2 x^{2}-1\right)^{2}\left(x+1\right)^{-\frac{1}{3}}$}
\item   $\displaystyle f(x)=\sqrt{1+x^2}$

\answer{$x (x^{2}+1)^{-\frac{1}{2}}  $}
\item $\displaystyle f(x)= \sqrt{1-2x}$.

\answer{$- (1-2 x)^{-\frac{1}{2}}$}
\item $\displaystyle f(x)= \sqrt{\frac{x^2+1}{x^2+4}}$.

\answer{$\frac{3 x}{\left(x^{2}+4\right)^{2}} \left(\frac{x^{2}+1}{x^{2}+4}\right)^{-\frac{1}{2}} $}
\item $\displaystyle f(x)= 3\cot (2x)$.

\answer{$ \frac{-6}{(\sin{}(2 x))^{2}}$}
\item \label{problemd/dx(cos(x))^(1/2)}  $\displaystyle f(x)=(\cos x)^{\frac{1}{ 2}}$

\answer{$-\frac{1}{2} \sin{}x (\cos{}x)^{-\frac{1}{2}}  $}
\item \label{problemd/dx(1+cos(x))^2} $\displaystyle f(x)=(1+\cos x)^2$

\answer{$ -2 \cos{}x \sin{}x-2 \sin{}x =-\sin(2x)-2\sin x$}
\item $\displaystyle f(x)=\sin^3 x$

\answer{$ 3 \cos{}x \sin^{2}{}x$}
\item   $\displaystyle f(x)=\frac{1}{\sin^3x}$

\answer{$  -\frac{3 \cos{}x}{\sin^{4}{}x} $}
\item  $\displaystyle f(x)= \sqrt[3]{4+3\tan x}$

\answer{$  (4+3\tan x)^{-\frac{2}{3}}\sec^2x $}
\item  $f(x)=(\cos x + 3\sin x)^4$

\answer{$4(\cos x + 3\sin x)^3 (3\cos x-\sin x) $}
\item \label{problemd/dx(sin(sqrt(x)))}  $f(x)=\sin\left(\sqrt{x}\right)$

\answer{$\frac{1}{2} x^{-\frac{1}{2}} \cos{}\left(\sqrt{x}\right)  = \frac{\cos \left(\sqrt{x}\right)}{2\sqrt{x}}$}
\item  $f(x)=\cos(4x)$

\answer{$-4 \sin{}\left(4 x\right)  $}
\item $\displaystyle f(x)= \frac{1}{(1+\sec x)^2}$.

\answer{$\frac{-2 \cos{}(x) \sin{}(x)}{(1+\cos{}(x))^{3}} =\frac{- \sin{}(2x)} {(1+\cos{}(x))^{3}} $}
\item $\displaystyle f(x)= \sqrt[3]{1+\tan x}$.

\answer{$ 
\frac{1}{3}\left(1+\tan x \right)^{-\frac{2}{3}} \sec^2 x
$}
\item $\displaystyle f(x)=\cos (2+x^3) $.

\answer{$ -3 x^{2}\sin{}\left(2+x^{3}\right) $}
\item $\displaystyle f(x)=\cos \left(\frac{1}{x}\right) \sin (x^2)$.

\answer{$x^{-2} \sin{}\left(x^{-1}\right) \sin{}\left(x^{2}\right)+2 x \cos{}\left(x^{-1}\right) \cos{}\left(x^{2}\right)  $}
\item $\displaystyle f(x)= x\sec (k x) $.

\answer{$\frac{\cos{}(k x)+k x \sin{}(k x) }{(\cos{}(k x))^{2}} $}

\end{enumerate}
\end{multicols}


\item Differentiate. 
\begin{multicols}{2}
\begin{enumerate}
\item $\displaystyle f(x)=\sin (\tan (2x)) $.

\answer{$2\sec^2(2x) \cos (\tan 2x) $}
\item $\displaystyle f(x)=\sec^2(m x) $.


\answer{$ \frac{2 m \sin{}(m x) }{(\cos{}(m x))^{3}} $}
\item $\displaystyle f(x)= \sec^2 x+\tan^2 x$.

\answer{$\frac{4 \sin{}x}{\cos^{3}{}x} $}
\item $\displaystyle f(x)=x\sin\left( \frac{1}{x}\right) $.

\answer{$- x^{-1}\cos{}(x^{-1}) +\sin{}(x^{-1})$}
\item $\displaystyle f(x)= \left(\frac{1-\cos (2x)}{1+\cos (2x)}\right)^4$.

\answer{$\frac{16 \sin{}(2 x)}{(\cos{}(2 x)+1)^{2}} \left(\frac{- \cos{}(2 x)+1}{\cos{}(2 x)+1}\right)^{3} $}
\item $\displaystyle f(x)=\sqrt{\frac{x}{x^2+4}} $.

\answer{$ \frac{-\frac{1}{2} x^{2}+2}{(x^{2}+4)^{2}} \left(\frac{x}{x^{2}+4}\right)^{-\frac{1}{2}}$}
\item $\displaystyle f(t)= \cot^2(\sin t)$.

\answer{$ \frac{-2 \cos{}t \cos{}(\sin{}t)}{\sin^{3}{}(\sin{}t)}$}
\item $\displaystyle f(x)= \left(a x+\sqrt{x^2+b^2}\right)^{-2}$.

\answer{$\frac{-2 x\left(x^{2}+b^{2} \right)^{-\frac{ 1 }{2}} -2 a}{\left(\left(x^{ 2}+b^{2}\right)^{ \frac{1}{2}} +a x\right)^{3}} $}
\item $\displaystyle f(x)= \left(x^2+(1-3x)^5 \right)^3$.

\answer{
\begin{tabular}{l}
$\left(-45 (-3 x+1)^{4} +6 x \right) \left((-3 x+1)^{5}+x^{2}\right)^{2}$
\\
Using computer algebra:
\\
$(-3645x^{4}+4860x^{3}-2430x^{2}+546x -45)\left((-3 x+1)^{5}+x^{2}\right)^{2}$ \\
Using computer algebra full expansion:
\\
$\begin{array}{l}
-215233605x^{14}+1004423490x^{13}-2176250895x^{12}\\
+2903793624x^{11} -2666357595x^{10}+1782098820x^{9}\\
-893713176x^{8} +341444160x^{7} -99805041x^{6} \\ +22199676x^{5}-3697470x^{4}  +447132x^{3}\\
-37125x^{2}+1896x -45
\end{array}
$
\end{tabular} 
}
\item $\displaystyle f(x)=\sin (\sin (\sin x))$.

\answer{$\cos{}x \cos{}(\sin{}x) \cos{}(\sin{}(\sin{}x)) $}
\item $\displaystyle f(x)= \sqrt{x+\sqrt{x}}$.

\answer{$ \left(\frac{1}{2} +\frac{1}{4} x^{-\frac{1}{2}}\right) \left(x^{\frac{1}{2}}+x\right)^{-\frac{1}{2}}$}
\item $\displaystyle f(x)= \sqrt{x+\sqrt{x+\sqrt{x}}}$.

\answer{$\frac{1}{2} \left(\left(x^{\frac{1}{2}}+x\right)^{\frac{1}{2}}+x\right)^{-\frac{1}{2}} \left(\frac{1}{2} \left(x^{\frac{1}{2}}+x\right)^{-\frac{1}{2}} \left(\frac{1}{2} x^{-\frac{1}{2}}+1\right)+1\right) $}
\item $\displaystyle f(x)=(2r \sin (r x)+n)^p $.

\answer{$ p r(2 r \sin{}(r x)+n)^{p-1} \cos{}(r x) $}
\item $\displaystyle f(x)=\cos^4(\sin^3 x) $.

\answer{$-12 \cos{}x \sin^{2}{}x \sin{}(\sin^{3}{}x) \cos^{3}{}(\sin^{3}{}x) $}
\item $\displaystyle f(x)=\cos \sqrt{\sin (\tan (\pi x))} $.

\answer{$ \frac{-\frac{1}{2} \pi \cos{}(\tan{}(\pi x))  \sin{}\left(\sqrt{\sin (\tan{}(\pi x) )} \right)}{\sqrt{\sin{}(\tan{}(\pi x))} \cos^{2}{}(\pi x) }$}
\item $\displaystyle f(x)=\left(x+(x+\sin^2 x)^3 \right)^4 $.

\answer{$4 ((\sin^{2}{}x+x)^{3}+x)^{3} (3 (\sin^{2}{}x+x)^{2} (2 \sin{}x \cos{}x+1)+1) $}
\end{enumerate}
\end{multicols}
\item Compute the second derivative.
\begin{multicols}{3}
\begin{enumerate}
\item $\displaystyle f(x)=\sin (-5x)$. 

\answer{$ 25 \sin{}(5 x)$}
\item $\displaystyle f(x)=e^{-3x}$. 

\answer{$9 e^{-3 x} $}
\item $\displaystyle f(x)=e^{\frac{1}x}$. 

\answer{$ 2 e^{x^{-1}} x^{-3}+e^{x^{-1}} x^{-4}$}
\item $\displaystyle f(x)=e^{\sqrt{x}}$. 

\answer{$ e^{x^{\frac{1}{2}}} x^{-\frac{3}{2}}+\frac{1}{4} e^{x^{\frac{1}{2}}} x^{-1}$}
\item $\displaystyle f(x)=\frac{e^{x}-e^{-x}}{e^x+e^{-x}} $

\answer{$\frac{-8 \left(- e^{- x}+e^{x}\right)}{\left(e^{- x}+e^{x}\right)^{3}} $}
\item $\displaystyle f(x)=\frac{1}2\ln \left(\frac{1+x}{1-x}\right) $

\answer{$\frac12\left(-\frac{1}{(x+1)^{2}}+\frac{1}{(- x+1)^{2}}\right)= \frac{ 2x}{\left(1-x^2\right)^2} $}
\end{enumerate}
\end{multicols}

\homeworkEnd
\end{comment}
\begin{comment}
\homeworkStart{on Lecture 14. \\ Will be quizzed Wednesday April 2}{}

\item (Textbook, page 161, problems 5-20) Express $\frac{\diff y}{\diff x}$ as a function of $x$ and $y$ by implicit differentiation. The answer key has not been proofread, use with caution.
\begin{multicols}{2}
\begin{enumerate}
\item $x^3+y^3=1$.
\answer{$\frac{\diff y}{\diff x}=-\frac{x^2}{y^2}$}
\item $ 2\sqrt x+\sqrt y=3$.
\answer{$\frac{\diff y}{\diff x}=-2\sqrt{\frac{ y}{x}}$}
\item $ x^2+x y-y^2=4$.
\answer{$\frac{\diff y}{\diff x}=\frac{-2x-y}{x-2y}$}
\item $ 2x^3+x^2y-x y^3=2$.
\answer{$\frac{\diff y}{\diff x}=-\frac{6x^2+2xy+y^3}{-3xy^2+x^2}$}
\item $ x^4(x+y)=y^2(3x-y)$.
\answer{$\frac{\diff y}{\diff x}= \frac{ -5x^4 -4x^3y +3y^2}{x^4- 6xy - 3y^2}$}
\item $ y^5+x^2y^3=1+x^4y $.
\answer{$\frac{\diff y}{\diff x}=\frac{ 4x^3y-2xy^3}{5y^4+3x^2y^2- x^4 }$}
\item $ y\cos x=x^2+y^2 $.
\answer{$\frac{\diff y}{\diff x}= \frac{ y\sin x+2x}{\cos x-2y}$}
\item $ \cos (x y)=1+\sin y$.
\answer{$\frac{\diff y}{\diff x}= -\frac{y\sin (xy)}{\cos y}+x\sin(xy) $}
\item $ 4\cos x\sin y=1$.
\answer{$\frac{\diff y}{\diff x}=\tan x \tan y$}
\item $ y\sin (x^2)=x\sin (y^2)$.
\answer{$\frac{\diff y}{\diff x}=\frac{-2xy\cos(x^2)+\sin (y^2)}{ - 2 x y \cos( y^2 )+\sin (x^2)}$}
\item $ \tan \left(\frac{x}{y}\right)=x+y$.
\answer{$\frac{\diff y}{\diff x}=\frac{-y^2+y\sec^2 \left(\frac{y}{x}\right) }{y^2 + x\sec^2\left(\frac{x}{y}\right) }$}
\item $ \sqrt{x+y}=1+x^2y^2$.
\answer{$\frac{\diff y}{\diff x}= \frac{-\frac{1}{2} (y+x)^{-\frac{1}{2}}+8 x y^{2}}{\frac{1}{2} (y+x)^{-\frac{1}{2}}-8 x^{2} y} $}
\item $ \sqrt{xy}=1+x^2 y$.
\answer{$\frac{\diff y}{\diff x}=\frac{-\frac{1}{2} x^{-\frac{1}{2}} y^{\frac{1}{2}} -2 x y}{\frac{1}{2} x^{\frac12} y^{-\frac{1}{2}} +x^{2}} $}
\item $ x\sin y+y\sin x=1$.
\answer{$\frac{\diff y}{\diff x}=\frac{- y \cos{}x- \sin{}y}{x \cos{}y+\sin{}x}$}
\item $ y\cos x=1+\sin (x y)$.
\answer{$\frac{\diff y}{\diff x}=\frac{\cos{}(x y) y+y \sin{}x}{- \cos{}(x y) x+\cos{}x}$}
\item $ \tan (x-y)=\frac{y}{1+x^2}$.

\answer{$\frac{\diff y}{\diff x}=\frac{- (\sec{}(- y+x))^{2} x^{4}-2 (\sec{}(- y+x))^{2} x^{2}- (\sec{}(- y+x))^{2}-2 y x}{- (\sec{}(- y+x))^{2} x^{4}-2 (\sec{}(- y+x))^{2} x^{2}- (\sec{}(- y+x))^{2}- x^{2}-1}$}
\end{enumerate}
\end{multicols}
\item (Textbook, page 162, problems 25-32) Use implicit differentiation to find an equation of the tangent line to the curve a the given point. The answer key has not been proofread, use with caution.
\begin{multicols}{3}
\begin{enumerate}
\item $y\sin (2x)=x\cos (2y) $, $\left(\frac{\pi}{2}, \frac{\pi}{4}\right)$. 
\answer{$y=\pi^{-1} x+\frac{1}{4} \pi-\frac{1}{4}$}
\item $ \sin (x+y)=2x-2y$, $(\pi,\pi)$ . 
\answer{$\frac{1}{3} x+\frac{2}{3} \pi $}
\item $x^2+x y+y^2=3 $, $(1,2)$ (ellipse). 
\answer{$y=-\frac{4}{5} x+\frac{14}{5} $}
\item $x^2+2x y-y^2+x=2 $, $(1,2)$ (hyperbola). 
\answer{$y= \frac{7}{2} x-\frac{3}{2}$}
\item $x^2+y^2=(2x^2+2y^2-x)^2 $, $(0,\frac{1}{2})$. 
\answer{$y= x+\frac{1}{2}$}
\item $x^{\frac{2}{3}}+y^{\frac{2}{3}}=4$, $(-3\sqrt{3},1)$. 
\answer{$y=\frac{1}{\sqrt{3}}x+4 $}
\item $2(x^2+y^2)^2 =25(x^2-y^2)$, $(3,1)$. 
\answer{$y= -\frac{9}{13} x+\frac{40}{13}$}
\item $y^2(y^2-4)=x^2(x^2-5) $, $(0,-2)$. 
\answer{$y=-2 $}
\end{enumerate}
\end{multicols}

\homeworkEnd

\end{comment}
\begin{comment}
\homeworkStart{on Lecture 14, related rates \\Quiz time to be announced}{}

\item Stewart, pages 180-183
\begin{enumerate}
\item If $V$ is the volume of a cube with edge length $x$ and the cube expands as time passes, find $\frac{dV}{dt}$ in terms of $\frac{dx}{dt}$.
\item Each side of a square is increasing at a rate of 6cm/s. At what rate is the area of the square increasing when the area of the square is 16 $cm^2$?
\item The radius of a ball is increasing at a rate of 4 mm/s. How fast is the volume increasing when the diameter is 80mm?
\item 
A street light is mounted at the top of a 4.5m tall pole. A man 180 cm tall walks away from the pole at a speed of 5km/h along a straight path. How fast is the tip of his shadow moving when he is 12m from the pole?


\item 
A boat is pulled into a dock by a rope attached to the bow of the boat and passing through a pulley on the dock that is 1m higher than the bow of the boat. If the rope is pulled in at a rate of 1m/s, how fast is the boat approaching the dock when it is 8m from the dock?

\item 
A Ferris wheel with a radius of 10m is rotating at a rate of one revolution every 2 minutes. How fast is a riding rising when his seat is 16 m above ground level?

\item 
The minute hand on a watch is 8mm long and the hour hand is 4mm long. How fast is the distance between the tips of the hands changing at one' clock?
\end{enumerate}




\homeworkEnd

\end{comment}

\begin{comment}
\homeworkStart{on Lecture 15. \\ Problem types will appear on the Test but will not be quizzed}{}
\item (Textbook page 205)
Find the absolute maximum and absolute minimum values of $f$ on the given interval.
\begin{multicols}{3}
\begin{enumerate}
\item $\displaystyle f(x)=12+4x-x^2$, $x\in [0,5]$.
\item $\displaystyle f(x)=5+54x-2x^3$, $x\in[0,4] $.
\item $\displaystyle f(x)=2x^3-3x^2-12x+1$, $x\in [-2,3]$.
\item $\displaystyle f(x)=x^3-6x^2+5$, $x\in [-3, 5]$.
\item $\displaystyle f(x)=3x^4-4x^3-12x^2+1$, $x\in [-2, 3]$.
\item $\displaystyle f(x)=(x^2-1)^3$, $x\in [-1, 2]$.
\item $\displaystyle f(x)=x+\frac{1}{x}$, $x\in [0.2,4 ]$.
\item $\displaystyle f(x)=\frac{x}{x^2-x+1}$, $x\in [0,3 ]$.
\item $\displaystyle f(t)=t\sqrt{4-t^2}$, $x\in [-1,2 ]$.
\item $\displaystyle f(t)=\sqrt[3]{t}(8-t) $, $x\in [0,8 ]$.
\item $\displaystyle f(t)=2\cos t+\sin (2t)$, $x\in [0,\frac{\pi}{2} ]$.
\item $\displaystyle f(t)=t+\cot (t/2) $, $x\in [\frac{\pi}{4},\frac{7\pi}{4} ]$.
\end{enumerate}
\end{multicols}

\item 
\begin{enumerate}
\item (page 257)
Find the dimensions of a rectangle with area 1000 $m^2$ whose perimeter is as small as possible.
\item (pages 256-259)
A box with an open top is to be constructed from a square piece of cardboard, 1m wide, by cutting out a square from each of the four corners and bending up the sides. Find the largest volume that such a box can have.
\item (pages 256-259)
A right circular cylinder is inscribed in a sphere of radius $r$. Find the largest possible volume of such a cylinder.
\item (pages 256-259)
A wedge of radius $2$ (depicted below) is folded into a cone cup. The volume varies depending on the angle of the wedge. Find the maximal possible volume of the cone cup and the angle of the wedge for which this maximal volume is achieved.
\psset{xunit=0.5cm, yunit=0.5cm}
\begin{pspicture}(-1.5, -1.5)(1.5,1.5) 
\tiny 
\pscustom*[linecolor=cyan!30]{ \psparametricplot[algebraic] {2.35619}{7.06858} {0+1*cos(t)| 0+1*sin(t)} \psline(0.707107, 0.707107)(0, 0)(-0.707107, 0.707107)}

\psparametricplot[algebraic,linecolor=blue]{2.35619}{7.06858}{cos(t)| sin(t)} 
\psline[linecolor=red](0.707107, 0.707107)(0, 0)(-0.707107, 0.707107)

\rput[t](0.4, 0.2){$r$}
\rput[lb](0.8,0.8){$B$}
\rput[rb](-0.8,0.8){$A$}
\rput[b](0,0.3){$O$}
\end{pspicture} 

\end{enumerate}

\homeworkEnd
\end{comment}
%\begin{comment}
\homeworkStart{on Lecture 17. \\ will be quizzed on Monday April 21}{}
\item 
Find the
\begin{multicols}{2}
\begin{itemize}
\item the implied domain of $f$.
\item $x$ and $y$ intercepts of $f$.
\item horizontal and vertical asymptotes.
\item intervals of increase and decrease
\item local and global minima, maxima,
\item intervals of concavity
\item points of inflection
\end{itemize}
\end{multicols}
Label all relevant points on the graph. Show all of your computations.
\begin{enumerate}
\item $\displaystyle f(x)=\frac{x+\frac 1 2}{x^{2}+x+1}$
\psset{xunit=1cm, yunit=1cm}
\begin{pspicture}(-5, -5)(5,5)
\psframe*[linecolor=white](-5,-5)(5,5)
\tiny
\psaxes[ticks=none, labels=none]{<->}(0,0)(-5,-0.5)(5,1.5)
\fcLabels{5}{1.5}
%Function formula: \frac{x+1/2}{x^{2}+x+1}
\psplot[linecolor=\fcColorGraph, plotpoints=1000]{-5}{5}{0.5 x add 1 x add x 2 exp add div }
\end{pspicture}

\answer{
\begin{tabular}{l}
$y$-intercept: $\frac12$. $x$-intercept: $-\frac12$\\
Horizontal asymptote: $y=0$, vertical: none \\
local and global min at $x=\frac{ -1-\sqrt{3}}{2}$, local and global max at $x=\frac{ -1+\sqrt{3}}{2}$\\
Intervals of decrease: $ \left(-\infty, \frac{-1 -\sqrt{3} }{2}\right)\cup \left(\frac{-1 +\sqrt{3} }{2}, \infty\right) $, intervals of decrease $\left( \frac{ -1-\sqrt{3}}{2}, \frac{-1+ \sqrt{3}}{2}\right)$ \\
Concave down on $(-\infty, -2)\cup \left(-\frac12, 1\right)$, concave up on $\left(-2, -\frac12\right)\cup (1,\infty)$\\
Inflection points at: $x=-2$, $x= -\frac12$, $x=1 $ \\
\end{tabular}
}

\item \label{problemSketchCurve(2x^2-5x+9/2)/(x^2-3 x+3)} $\displaystyle f(x)=\frac{2 x^{2}-5 x+\frac{9}{2}}{x^{2}-3 x+3}$
\psset{xunit=1cm, yunit=1cm}
\begin{pspicture}(-5, -5)(5,5)
\psframe*[linecolor=white](-5,-5)(5,5)
\tiny
\psaxes[ticks=none, labels=none]{<->}(0,0) (-5,-0.5) (5, 3.5)
\fcLabels{5}{3.5}
%Function formula: \frac{2 x^{2}-5 x+9/2}{x^{2}-3 x+3}
\psplot[linecolor=\fcColorGraph, plotpoints=1000]{-5}{5 } {4.5 x -5 mul add x 2 exp 2 mul add 3 x -3 mul add x 2 exp add div }
\end{pspicture}

\answer{
\begin{tabular}{l}
$y$-intercept: $\frac32$\\
horizontal asymptote: $y=2$, vertical: none\\
increasing on
$\left(\frac{3-\sqrt{3}}2, \frac{3+ \sqrt{3}}2 \right) $, decreasing on $\left(-\infty, \frac{3-\sqrt{3}}2\right)\cup \left(\frac{3+\sqrt{3}}2, \infty\right) $\\
local and global min at $x=\frac{3-\sqrt{3}}2$, local and global max at $x=\frac{3+\sqrt{3}}2$\\
concave up on $\left(0, \frac32\right)cup \left(3, \infty \right)$, concave down $\left(-\infty, 0\right)\cup \left(\frac32, 3\right)$\\
inflection points at $x=0,x=\frac32, x=3$
\end{tabular}
}
\item $\displaystyle f(x)=\frac{2 \sqrt{- x^{2}+1}+ 1} {\sqrt{- x^{2}+1}+1}$,  $f(x)=\frac{1}{\sqrt{- x^{2} +1}+1}$
\psset{xunit=1cm, yunit=1cm}
\begin{pspicture}(-1, -5)(1,5)
\psframe*[linecolor=white](-1,-5)(1,5)
\tiny
\psaxes[ticks=none, labels=none]{<->}(0,0)(-1,-0.5)(1,2.5)
\fcLabels{1}{2.5}
%Function formula: \frac{2 (- x^{2}+1)^{1/2}+1}{(- x^{2}+1)^{1/2}+1}
\rput(1,3){}
\psplot[linecolor=brown, plotpoints=1000]{-1}{1}{1 1 x 2 exp -1 mul add 0.5 exp 2 mul add 1 1 x 2 exp -1 mul add 0.5 exp add div }
%Function formula: \frac{1}{(- x^{2}+1)^{1/2}+1}
\rput(1,3){}
\psplot[linecolor=\fcColorGraph, plotpoints=1000]{-1}{1}{1 1 1 x 2 exp -1 mul add 0.5 exp add div }
\end{pspicture}
The two functions are plotted simultaneously in the $x,y$-plane. Indicate which part of the graph is the graph of which function.

\answer{
\begin{tabular}{l}
For $f(x)=\frac{2 \sqrt{- x^{2}+1}+1}{ \sqrt{- x^{2}+1}+1}$: \\
$y$-intercept: $x=\frac{3}2$, no $x$ intercept\\
no asymptotes\\
increasing on $[-1, 0]$, decreasing on $[0, 1]$ \\
global and local max at $x=0$, global and local min at $x=\pm 1$.\\
concave down on $[-1,1]$\\
no inflection points
\end{tabular}
}
\answer{
\begin{tabular}{l}
For $f(x)=\frac{1}{\sqrt{- x^{2}+1}+1}$: \\
$y$-intercept: $x=\frac{1}2$, no $x$ intercept\\
no asymptotes\\
decreasing on $[-1, 0]$, increasing on $[0, 1]$ \\
global and local min at $x=0$, global and local max at $x=\pm 1$.\\
concave up on $[-1,1]$\\
no inflection points
\end{tabular}
}
\item $\displaystyle f(x)=\frac{e^x+e^{-x}}{e^x-e^{-x}}$
\psset{xunit=0.5cm, yunit=0.5cm}
\begin{pspicture}(-4, -5)(4,5)
\psframe*[linecolor=white](-4,-5)(4,5)
\tiny
\psaxes[ticks=none, labels=none]{<->}(0,0)(-4,-4.5)(4,4.5)
\fcLabels{4}{5}
%Function formula: \frac{e^{- x}+e^{x}}{- e^{- x}+e^{x}}
\psplot[linecolor=\fcColorGraph, plotpoints=1000]{0.2}{4}{2.718281828 x exp 2.718281828 x -1 mul exp add 2.718281828 x exp 2.718281828 x -1 mul exp -1 mul add div }
%Function formula: \frac{e^{- x}+e^{x}}{- e^{- x}+e^{x}}
\psplot[linecolor=\fcColorGraph, plotpoints=1000]{-4}{-0.2}{2.718281828 x exp 2.718281828 x -1 mul exp add 2.718281828 x exp 2.718281828 x -1 mul exp -1 mul add div }
\end{pspicture}
\item $\displaystyle f(x)=\frac{- e^{- x}+e^{x}}{e^{- x}+e^{x}}$
\psset{xunit=1cm, yunit=1cm}
\begin{pspicture}(-4, -5)(4,5)
\psframe*[linecolor=white](-4,-5)(4,5)
\tiny
\psaxes[ticks=none, labels=none]{<->}(0,0)(-4,-1.1)(4,1.1)
\fcLabels{4}{1.1}
%Function formula: \frac{- e^{- x}+e^{x}}{e^{- x}+e^{x}}
\psplot[linecolor=\fcColorGraph, plotpoints=1000]{-4}{4}{2.718281828 x exp 2.718281828 x -1 mul exp -1 mul add 2.718281828 x exp 2.718281828 x -1 mul exp add div }
\end{pspicture}
\item $\displaystyle f(x)=\ln{}\left(\frac{{{x}}+1}{- {{x}}+1}\right)$
\psset{xunit=1cm, yunit=1cm}
\begin{pspicture}(-0.9, -5)(1,5)
\psframe*[linecolor=white](-0.9,-5)(1,5)
\tiny
\psaxes[ticks=none, labels=none]{<->}(0,0)(-1.3,-4)(1.3,4)
\fcLabels{1.3}{4}
%Function formula: \log{}(\frac{x+1}{- x+1})
\psplot[linecolor=\fcColorGraph, plotpoints=1000]{-0.94}{0.94}{1 x add 1 x -1 mul add div ln }

\end{pspicture}
\item $f(x)=\frac{x^{2}+3 x+1}{x^{2}+2 x}$
\psset{xunit=0.7cm, yunit=0.7cm}
\begin{pspicture}(-5, -5)(5,5)
\psframe*[linecolor=white](-5,-5)(5,5)
\tiny
\psaxes[ticks=none, labels=none]{<->}(0,0)(-5,-4.5)(5,4.5)
\fcLabels{5}{5}
%Function formula: \frac{x^{2}+3 x+1}{x^{2}+2 x}
\psplot[linecolor=\fcColorGraph, plotpoints=1000]{0.1}{5}{1 x 3 mul add x 2 exp add x 2 mul x 2 exp add div }
%Function formula: \frac{x^{2}+3 x+1}{x^{2}+2 x}
\psplot[linecolor=\fcColorGraph, plotpoints=1000]{-1.9}{-0.1}{1 x 3 mul add x 2 exp add x 2 mul x 2 exp add div }
%Function formula: \frac{x^{2}+3 x+1}{x^{2}+2 x}
\psplot[linecolor=\fcColorGraph, plotpoints=1000]{-5}{-2.1}{1 x 3 mul add x 2 exp add x 2 mul x 2 exp add div }

\end{pspicture}

\answer{
\begin{tabular}{l}
$y$-intercept: none, $x$-intercepts: $\frac{-3\mp\sqrt{5}}2$ \\
horizontal asymptote: $y=1$, vertical: $x=-2$ and $x=0$\\
always decreasing\\
no local/global minima/maxima\\
concave down on $\left(-\infty,-2\right)cup \left(-1,0 \right)$, concave up on $\left(-2, -1\right)\cup \left(0, \infty\right)$\\
inflection point at $x=-1$
\end{tabular}
}
\item \item $\displaystyle f(x)=\frac{x+1}{x^2+2x+4}$
\psset{xunit=2cm, yunit=2cm}
\begin{pspicture}(-4.500000, -5)(4.500000,5)
\psframe*[linecolor=white](-4.500000,-1)(4.500000,1)
\tiny
\fcAxesStandard{-5.000000}{-1}{3.000000}{1} %Function formula: \frac{x+1}{x^{2}+2 x+4}
\psplot[linecolor=\fcColorGraph, plotpoints=1000]{-5.000000}{3.000000}{1 x add 4 x 2 mul add x 2 exp add div }
\end{pspicture}

\answer{
\begin{tabular}{l}
$y$-intercept: $\frac14$, $x$-intercept: $-1$\\
horizontal asymptote: $y=0$, vertical: none\\
increasing on
$\left(-1-\sqrt{3}, -1+\sqrt{3}  \right) $, decreasing on $\left(-\infty, -1-\sqrt{3}\right)\cup \left(-1+\sqrt{3}, \infty\right) $\\
local and global min at $x=-1-\sqrt{3}$, local and global max at $x=-1+\sqrt{3}$\\
concave up on $\left(-4, -1\right)cup \left(2, \infty \right)$, concave down $\left(-\infty, -4\right)\cup \left(-1, 2\right)$\\
inflection points at $x=-4,x=-1, x=2$
\end{tabular}
}

\end{enumerate}

\solution{\ref{problemSketchCurve(2x^2-5x+9/2)/(x^2-3 x+3)}
We have that $f$ is not defined only when we have division by zero, i.e.,  if $x^2-3x+3$ equals zero. However, the roots of $x^{2}-3x+3$ are not real numbers: they are $\frac{3\pm \sqrt{3^2-4\cdot 3 }}{2}= \frac{3\pm \sqrt{-3}}{2}$, and therefore $x^2-3x+3$ can never equal zero. Alternatively, completing the square shows that the denominator is always positive:
\[
x^2-3x+3=x^2-2\cdot \frac{3}{2} x+\frac{9}{4}-\frac{9}{4}+3=\left(x-\frac{3}{2}\right)^2+\frac{3}{4} >0 
\]
Therefore the domain of $f$ is all real numbers.
}
\homeworkEnd
%\end{comment}
\begin{comment}
\homeworkStart{on Lecture 19. \\ will be quizzed on Monday April 28}{}
\item \begin{enumerate}
\item Find the linearization of the function $f(x) = \sqrt{x}$ at $a = 100$, and then use your new function to approximate
$\sqrt{99.8}$.

\answer{$L(x) = 10 + 0.05(x-100)$. Therefore $\sqrt{99.8} \approx L(99.8) = 9.99$.}

\item Use a linear approximation to estimate $(1.001)^9$. 

\answer{$(1.001)^9 \approx 1.009$.}

\item $f(x)=\sqrt{8+x}$ at $a=1$ and use it to approximate $\sqrt{9.02}$.

\answer{ $f(x)\approx 3+ \frac16 (x-1)=1/6 x+17/6$. Therefore $\sqrt{9.02}\approx 901/300\approx 3.003333$}
\item $f(x)=\sqrt[3]{8+x}$ at $a=0$ and use it to approximate $\sqrt[3] {7.97}$.

\answer{ $\sqrt[3]{8+x}\approx \frac{1}{12}x+2$. Therefore $\sqrt[3]{7.97}\simeq 799/400=1.9975$}

\end{enumerate}

\homeworkEnd
\end{comment}
\begin{comment}
\homeworkStart{Homework Math 140, Lectures 20-21. \\ Will be quizzed Monday December 2}{}
Find all antiderivatives of the functions.
\begin{multicols}{3}
\begin{enumerate}
\item $\displaystyle f(x)=\sqrt {3}+\pi^2$.

\answer{$\displaystyle x\left(\pi^{2} +\sqrt{3}\right)+C$}
\item $\displaystyle f(x)=x-5$.

\answer{$\displaystyle \frac{x}{2}-5x+C$}

\item $\displaystyle f(x)= x^2-2x+6$.

\answer{ $\frac{x^3}{3} -x^2+6x+C$ }

\item $\displaystyle f(x)=\frac{x(x+1)}{2} $.

\answer{$\frac{1}{6} x^{3}+\frac{1}{4} x^{2}+C$}
\item $\displaystyle f(x)=x(x+1)(2x+1)$.

\answer{$\frac{1}{2} x^{4}+x^{3}+\frac{1}{2} x^{2}+C$}
\item $\displaystyle f(x)=7x^{\frac{2}{7}}+x^{-\frac{4}{7}}$.

\answer{$\frac{49}{9} x^{\frac{9}{7}}+\frac{7}{3} x^{\frac{3}{7}}+ C$}
\item $\displaystyle f(x)=x^{2.4}-2x^{\sqrt{3}-1}$.

\answer{$\frac{5}{17} x^{\frac{17}{5}}-\frac{2\sqrt{3} x^{\sqrt{3}}}{3} +C$}
\item $\displaystyle f(x)=\frac{8}{x^7}$.

\answer{$-\frac{4}{3} x^{-6}+C$}
\item $\displaystyle f(x)=\frac{x+1}{x^3}$.

\answer{$- x^{-1}-\frac{1}{2} x^{-2}+C$}
\item $\displaystyle f(x)=\frac{1}{x}$.

\answer{$\ln |x|+C$}
\item $\displaystyle f(x)=\frac{x^2+1}{x}$.

\answer{$\frac{1}{2} x^{2}+\ln|x|+C $}
\item $\displaystyle f(x)=\frac{5-4x^3+2x^6}{x^4}$.

\answer{$\frac{2}{3} x^{3}-\frac{5}{3} x^{-3}-4 \ln|x|+C $}
\item $\displaystyle g(x)=\frac{1+\sqrt{x}+x}{\sqrt{x^3}}$.

\answer{$2 x^{\frac{1}{2}}-2 x^{-\frac{1}{2}}+\ln|x|+C $}
\item $\displaystyle f(x)=3\sin t-4\cos t$.

\answer{$-3\cos t -4\sin t +C $}
\item $\displaystyle f(\theta)=\sec^2\theta$.

\answer{$\tan \theta +C $}
\item $\displaystyle f(\theta)=\csc^2\theta$.

\answer{$\tan \theta +C $}

\item $\displaystyle f(t)=\sec t \tan t +\csc t \cot t$.

\answer{$\sec t-\csc t +C $}
\item $\displaystyle f(x)=\frac{2+x\cos x}{x}$.

\answer{$2\ln |x|+\sin x $}
\end{enumerate}
\end{multicols}
Evaluate the definite integral.
\begin{multicols}{3}
\begin{enumerate}
\item $\displaystyle \int\limits_{-2}^{3} \left(x^2-1 \right)  \diff x$.

\answer{$\left[ \frac{1}{3} x^{3}- x \right]_{-2}^3=\frac{20}{3}$}

\item $\displaystyle \int\limits_{1}^{2} \left(4x^3+3x^2+2x+1\right)  \diff x$.

\answer{$\left[ x^{4}+x^{3}+x^{2}+x\right]_{1}^{2}=26$}
\item $\displaystyle \int\limits_{0}^{2}(x-1)(x^2+1)  \diff x$.

\answer{$\left[\frac{1}{4} x^{4}-\frac{1}{3} x^{3}+\frac{1}{2} x^{2}- x \right]_{0}^{2} = \frac{4}{3}$}
\item $\displaystyle \int\limits_{-1}^{1} \left( \frac{x(x+1) }{ 2} \right)^2  \diff x$.

\answer{$\left[\frac{1}{20} x^{5}+\frac{1}{8} x^{4}+\frac{1}{12} x^{3} \right]_{-1}^{1}=\frac{4}{15}$}
\item $\displaystyle \int\limits_{0}^{1}(1+x^2)^3 dx$.

\answer{$\left[\frac{1}{7} x^{7}+\frac{3}{5} x^{5}+x^{3}+x \right]_{0}^{1}=\frac{96}{35}$}
\item $\displaystyle \int \limits_{1}^{2} \left(\frac{1}{x} - \frac{4}{x^2} \right)  \diff x$.

\answer{$\left[ 4 x^{-1}+\ln x\right]_{1}^{2}=\ln 2-2$}
\item $\displaystyle \int\limits_{1}^{4}\sqrt{x}(1+x) \diff x$.

\answer{$\left[\frac{2}{5} x^{\frac{5}{2}}+\frac{2}{3} x^{\frac{3}{2}} \right]_{1}^{4}=\frac{256}{15}$}
\item $\displaystyle \int\limits_{1}^{4}\sqrt{\frac{6}{x}} \diff x$.

\answer{$\left[ \right]_{}^{}=$}
\item $\displaystyle \int \limits_{1}^{4} \frac{\frac{1}{\sqrt{x}}+1+x}{ \sqrt{x}}  \diff x$.

\answer{$\left[ \right]_{}^{}=$}
\item $\displaystyle \int \limits_{1}^{4} \frac{1+x}{ \sqrt[3]{x}} \diff x$.

\answer{$\left[ \right]_{}^{}=$}
\item $\displaystyle \int\limits_{0}^1 \frac{\frac{1}{\sqrt[3]{x}}+\sqrt[3]{x}}{ \sqrt{x}}\diff x$.

\answer{$\left[ \right]_{}^{}=$}
\item $\displaystyle \int\limits_0^{1} (\sqrt[5]{x^6}+\sqrt[6]{x^5})\diff x $.

\answer{$\left[ \right]_{}^{}=$}
\item $\displaystyle \int\limits_{1}^{2} \left(x +\frac{1}{x} \right)^2 \diff x$.

\answer{$\left[ \right]_{}^{}=$}
\item $\displaystyle \int\limits_{1}^{2} \left(x +\frac{1}{x} \right)^3 \diff x$.

\answer{$\left[ \right]_{}^{}=$}
\item $\displaystyle \int\limits_{1}^{2} \left(\sqrt{x} +\frac{1}{\sqrt{x}} \right)^2 \diff x$.

\answer{$\left[ \right]_{}^{}=$}
\item $\displaystyle \int\limits_{1}^{2} \left(\sqrt{x} +\frac{1}{\sqrt{x}} \right)^3 \diff x$.

\answer{$\left[ \right]_{}^{}=$}
\item $\displaystyle \int\limits_{0}^{2}|x-1| \diff x$.

\answer{$\left[ \right]_{}^{}=$}
\item $\displaystyle \int\limits_{0}^{1} \left|x-\frac{1}{2}\right| \diff x$.

\answer{$\left[ \right]_{}^{}=$}
\item $\displaystyle \int\limits_{-1}^{1}(x-3|x|) \diff x$.

\answer{$\left[ \right]_{}^{}=$}
\item $\displaystyle \int\limits_{\frac{\pi}{4}}^{\frac{\pi}{2}} \csc^2\theta \diff \theta$.

\answer{$\left[ \right]_{}^{}=$}
\item $\displaystyle \int\limits_{0}^{\frac{\pi}{4}}\frac{1-\cos^2\theta}{\cos^2\theta} \diff \theta$.

\answer{$\left[ \right]_{}^{}=$}
\item $\displaystyle \int\limits_{0}^{\frac{\pi}{4}}\frac{\sin^2\theta}{\cos^2\theta} \diff \theta$.

\answer{$\left[ \right]_{}^{}=$}
\item $\displaystyle \int\limits_{0}^{\frac{\pi}{4}}\tan^2\theta \diff \theta$.

\answer{$\left[ \right]_{}^{}=$}
\item $\displaystyle \int\limits_{0}^{\frac{\pi}{3}} \frac{\sin \theta +\sin \theta \tan^2\theta}{\sec^2\theta}\diff \theta$.

\answer{$\left[ \right]_{}^{}=$}
\item $\displaystyle \int\limits_{0}^{\pi} (\sin \theta -\cos \theta) \diff \theta$.

\answer{$\left[ \right]_{}^{}=$}
\item $\displaystyle \int\limits_{0}^{\pi}|\sin x| \diff x$.
\end{enumerate}
\end{multicols}

\homeworkEnd
\end{comment}
\begin{comment}
\homeworkStart{Homework Math 140, Lecture 22. \\ Will be quizzed Monday December 9}{}
\begin{problem}
Evaluate the indefinite integral.
\begin{multicols}{2}
\begin{enumerate}
\item $\displaystyle\int x \sin(x^2)~dx $.
\item $\displaystyle\int x^2\cos(x^3)~dx $.
\item $\displaystyle\int (1-2x)^9 ~dx $.
\item $\displaystyle\int (3t+2)^{2.4}~dt $.
\item $\displaystyle\int (x+1)\sqrt{2x+x^2} ~dx $.
\item $\displaystyle\int \sec^2(2\theta) ~d\theta $.
\item $\displaystyle\int \sec (3t) \tan (3t)~dt $.
\item $\displaystyle\int u\sqrt{1-u^2} ~du $.
\item $\displaystyle\int \frac{a+bx^2}{\sqrt{3ax+bx^3}}~dx $.
\item $\displaystyle\int \frac{\sin \sqrt{x}}{\sqrt{x}} ~dx $.
\item $\displaystyle\int \sec^2\theta \tan^3\theta  ~d\theta $.
\item $\displaystyle\int \cos^4\theta\sin \theta ~d\theta $.
\item $\displaystyle\int (x^2+1)(x^3+3x)^4 ~dx $.
\item $\displaystyle\int \sqrt{x}\sin (1+x^{\frac{3}{2}}) ~dx $.
\item $\displaystyle\int \frac{\cos x}{\sin^2 x} ~dx $.
\item $\displaystyle\int \frac{\cos\left(\frac{\pi}{x}\right)}{x^2} ~dx $.
\item $\displaystyle\int \frac{z^2}{\sqrt[3]{1+z^3}} ~dz $.
\item $\displaystyle\int \frac{dt}{\cos^2 t\sqrt{1+\tan t}} $.
\item $\displaystyle\int \sqrt{\cot x} \csc^2 x~dx $.
\item $\displaystyle\int \sin t \sec^2(\cos t)~dt $.
\item $\displaystyle\int \sec^3x \tan x~dx $.
\item $\displaystyle\int x^2\sqrt{2+x}~dx $.
\item $\displaystyle\int x(2x+5)^8 ~dx $.
\item $\displaystyle\int x^3\sqrt{x^2+1} ~dx $.
\end{enumerate}
\end{multicols}
\end{problem}
\begin{problem}
Evaluate the integral. You may use the formula $\int \frac{1}{1+x^2}dx=\arctan x+C $. The function $\arctan x$ is the arctangent function (sometimes written as $\tan^{-1}x$).
\begin{multicols}{2}
\begin{enumerate}
\item $\displaystyle\int \frac{dx}{5-3x}$.
\item $\displaystyle\int e^x\sin (e^x) dx$.
\item $\displaystyle\int \frac{(\ln x)^2}{x} dx$.
\item $\displaystyle\int \frac{dx}{ax+b} dx$, $a\neq 0$.
\item $\displaystyle\int e^x\sqrt{1+e^x} dx$.
\item $\displaystyle\int e^{\cos t }\sin t dt$.
\item $\displaystyle\int e^{\tan x}\sec^2x dx$.
\item $\displaystyle\int \frac{\arctan^{-1}x}{1+x^2} dx$. 
\item $\displaystyle\int \frac{1+x}{1+x^2} dx$. 
\item $\displaystyle\int \frac{\sin \ln x}{x} dx$.
\item $\displaystyle\int \frac{\sin (2x)}{1+\cos^2x}dx$.
\item $\displaystyle\int \frac{\sin x}{1+\cos^2 x} dx$.
\item $\displaystyle\int \cot x dx$.
\item $\displaystyle\int \frac{x}{1+x^4}dx$.
\item $\displaystyle\int\limits_{e}^{e^4}\frac{dx}{x\sqrt{\ln x}} dx$.
\item $\displaystyle\int\limits_{0}^{1}xe^{-x^2} dx$.
\item $\displaystyle\int\limits_{0}^{1}\frac{e^z+1}{e^z+z} dz$.
%\item $\displaystyle\int\limits_{0}^{1/2}\frac{\arcsin x}{\sqrt{1-x^2}} dx$.
\end{enumerate}
\end{multicols}
\end{problem}
\homeworkEnd
\end{comment}