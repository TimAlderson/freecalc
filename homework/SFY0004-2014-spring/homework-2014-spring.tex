\documentclass[12pt]{article}
\usepackage{../homework-problems}
\usepackage{amsthm}
%\usepackage{etex} %avoiding error: too many packages. This is a LaTeX bug (``feature'')

\usepackage[english]{babel}
%\usepackage[latin1]{inputenc}
%\usepackage[all,cmtip]{xy}
\usepackage{times}
\usepackage{pstricks-add}
\usepackage{pst-math}
\usepackage{pst-node}
\usepackage{cancel}
\usepackage{tikz}


%%%%%%%%%%%%%%%%%%%%%%%%%%%%%%%%%%%%%%%%%%%%%%%%%%%%%%%%%
%
%  The following should be set to true if you want 
%  the solutions shown, and false otherwise. 
%  For the time being, this same tag is used 
%  for answer keys which are not solutions.
%
%%%%%%%%%%%%%%%%%%%%%%%%%%%%%%%%%%%%%%%%%%%%%%%%%%%%%%%%%

 %\toggletrue{solutions}
 \togglefalse{solutions}

 %\toggletrue{answers}
 \togglefalse{answers}


% The name of your course goes here. Default course name is Freecalc.
\renewcommand{\course}{SFY0004: Foundation Mathematics 2}
% The time when homework assignments are due goes here. Default due time is: (to be announced)
\renewcommand{\deadline}{10:00\\ (at the end of the lecture)}

\begin{comment}
\homeworkStart{1}{Friday February 7, 2014}
\item % begin homework chain-rule1
In each of the following cases find a simple function $u$ of $x$ such that the given function is a simple function of $u$.  
Use the Chain Rule to differentiate the given function with respect to $x$.   

\begin{enumerate}
\item   $y = \sqrt{1+x^2}$


\pointsii{3}  $y = (\cos x)^{1/2}$
\solution{%
\begin{align*}
\text{Let } \quad u & = \cos x. \\
\text{Then } \quad y & = u^{1/2}. \\
\text{Chain Rule: } \quad \frac{\diff y}{\diff x} & = \frac{\diff y}{\diff u}\frac{\diff u}{\diff x} \\
 & = \big(\frac{1}{2}u^{-1/2}\big) (-\sin x) \\
 & = -\frac{1}{2} \sin x (\cos x)^{-1/2}.
\end{align*}
}%

\item   $y = \sin^3 x$

\pointsii{3}  $y = (1+\cos x)^2$
\solution{%
\begin{align*}
\text{Let } \quad u & = 1+\cos x. \\
\text{Then } \quad y & = u^{2}. \\
\text{Chain Rule: } \quad \frac{\diff y}{\diff x} & = \frac{\diff y}{\diff u}\frac{\diff u}{\diff x} \\
 & = (2u) (-\sin x) \\
 & = -2 \sin x \cos x \\
 & = - \sin 2x. \quad \text{(This last step is optional.)}
\end{align*}
}%

\end{enumerate}
% end homework chain-rule1

\item % begin homework chain-rule2
Use the Chain Rule to differentiate the given function with respect to $x$.   

\begin{enumerate}
\item   $y = \frac{1}{\sin^3x}$

\item  $y = \sqrt[3]{4+3\tan x}$

\item  $y = (\cos x + 3\sin x)^4$

\pointsii{4}  $y = \sin\sqrt{x}$

\ans{%
\begin{align*}
\text{Let } \quad u & = \sqrt{x}. \\
\text{Then } \quad y & = \sin u. \\
\text{Chain Rule: } \quad \frac{\diff y}{\diff x} & = \frac{\diff y}{\diff u}\frac{\diff u}{\diff x} \\
 & = (\cos u) \big(\frac{1}{2}u^{-1/2}\big) \\
 & = \frac{\cos\sqrt{x}}{2\sqrt{x}}.
\end{align*}
}%


\item  $y = \cos4x$
\end{enumerate}
% end homework chain-rule2

\item Compute the derivative.
\begin{multicols}{2}
\begin{enumerate}[ref={\fcProblemRef}]
\item $\displaystyle f(x)= (x^4+3x^2-2)^5$.

\answer{$ \left(30 x +20 x^{3}\right) \left(-2+3 x^{2}+x^{4}\right)^{4}$}
\item $\displaystyle f(x)= (4x-x^2)^{100}$.

\answer{$(-200 x+400) \left(4 x- x^{2}\right)^{99}$}
\item $\displaystyle f(x)= (2x - 3)^4 (x^2 + x + 1)^5$.

\answer{$ \left(-7-12 x+28 x^{2}\right)\left(-3+2 x\right)^{3} \left(1+x +x^{2} \right)^{4}$}
\item $\displaystyle f(x)= (x^2+1)^3(x^2+2)^6$.

\answer{$\left(24 x+18 x^{3}\right)\left(1+x^{2}\right)^{2} \left(2+x^{2}\right)^{5} $}
\item $\displaystyle f(x)= (3x-1)^4(2x+1)^{-3}$.

\answer{$(3 x-1)^{3}\frac{6 x+18}{(2 x+1)^{4}}$}
\item $\displaystyle f(x)=\frac{1}{1+x^2} $.

\answer{$\frac{-2 x}{(1+x^{2})^{2}} $}
\item $\displaystyle f(x)=\left(\frac{x^2+1}{x^2-1} \right)^3 $.

\answer{$\frac{-12 x}{\left(x^{2}-1\right)^{2}} \left(\frac{x^{2}+1}{x^{2}-1}\right)^{2} $}
\item $\displaystyle f(x)= (x+1)^{\frac{2}{3}}(2x^2-1)^3$.

\answer{$ \left(\frac{40}{3} x^{2}+12 x-\frac{2}{3}\right)\left(2 x^{2}-1\right)^{2}\left(x+1\right)^{-\frac{1}{3}}$}
\item   $\displaystyle f(x)=\sqrt{1+x^2}$

\answer{$x (x^{2}+1)^{-\frac{1}{2}}  $}
\item $\displaystyle f(x)= \sqrt{1-2x}$.

\answer{$- (1-2 x)^{-\frac{1}{2}}$}
\item $\displaystyle f(x)= \sqrt{\frac{x^2+1}{x^2+4}}$.

\answer{$\frac{3 x}{\left(x^{2}+4\right)^{2}} \left(\frac{x^{2}+1}{x^{2}+4}\right)^{-\frac{1}{2}} $}
\item $\displaystyle f(x)= 3\cot (2x)$.

\answer{$ \frac{-6}{(\sin{}(2 x))^{2}}$}
\item \label{problemd/dx(cos(x))^(1/2)}  $\displaystyle f(x)=(\cos x)^{\frac{1}{ 2}}$

\answer{$-\frac{1}{2} \sin{}x (\cos{}x)^{-\frac{1}{2}}  $}
\item \label{problemd/dx(1+cos(x))^2} $\displaystyle f(x)=(1+\cos x)^2$

\answer{$ -2 \cos{}x \sin{}x-2 \sin{}x =-\sin(2x)-2\sin x$}
\item $\displaystyle f(x)=\sin^3 x$

\answer{$ 3 \cos{}x \sin^{2}{}x$}
\item   $\displaystyle f(x)=\frac{1}{\sin^3x}$

\answer{$  -\frac{3 \cos{}x}{\sin^{4}{}x} $}
\item  $\displaystyle f(x)= \sqrt[3]{4+3\tan x}$

\answer{$  (4+3\tan x)^{-\frac{2}{3}}\sec^2x $}
\item  $f(x)=(\cos x + 3\sin x)^4$

\answer{$4(\cos x + 3\sin x)^3 (3\cos x-\sin x) $}
\item \label{problemd/dx(sin(sqrt(x)))}  $f(x)=\sin\left(\sqrt{x}\right)$

\answer{$\frac{1}{2} x^{-\frac{1}{2}} \cos{}\left(\sqrt{x}\right)  = \frac{\cos \left(\sqrt{x}\right)}{2\sqrt{x}}$}
\item  $f(x)=\cos(4x)$

\answer{$-4 \sin{}\left(4 x\right)  $}
\item $\displaystyle f(x)= \frac{1}{(1+\sec x)^2}$.

\answer{$\frac{-2 \cos{}(x) \sin{}(x)}{(1+\cos{}(x))^{3}} =\frac{- \sin{}(2x)} {(1+\cos{}(x))^{3}} $}
\item $\displaystyle f(x)= \sqrt[3]{1+\tan x}$.

\answer{$ 
\frac{1}{3}\left(1+\tan x \right)^{-\frac{2}{3}} \sec^2 x
$}
\item $\displaystyle f(x)=\cos (2+x^3) $.

\answer{$ -3 x^{2}\sin{}\left(2+x^{3}\right) $}
\item $\displaystyle f(x)=\cos \left(\frac{1}{x}\right) \sin (x^2)$.

\answer{$x^{-2} \sin{}\left(x^{-1}\right) \sin{}\left(x^{2}\right)+2 x \cos{}\left(x^{-1}\right) \cos{}\left(x^{2}\right)  $}
\item $\displaystyle f(x)= x\sec (k x) $.

\answer{$\frac{\cos{}(k x)+k x \sin{}(k x) }{(\cos{}(k x))^{2}} $}

\end{enumerate}
\end{multicols}


\homeworkEnd

\homeworkStart{2}{Friday February 28, 2014}
\item % begin homework exponent-simplfy
Express each of the following as a single power.  

\begin{enumerate}
\item   $\displaystyle\frac{2^5\cdot 2^7}{2\sqrt{2}}$

\pointsii{2} $\displaystyle\frac{3^2\cdot 3^{-1}}{3^3\cdot \sqrt{3^3}}$

\solution{%
\begin{align*}
\frac{3^2\cdot 3^{-1}}{3^3\cdot\sqrt{3^3}} & = \frac{3^2\cdot 3^{-1}}{3^3\cdot (3^3)^{\frac{1}{2}}} \\
 & = \frac{3^2\cdot 3^{-1}}{3^3\cdot 3^{\frac{3}{2}}} \\
 & = \frac{3^{2-1}}{3^{3+\frac{3}{2}}} \\
 & = \frac{3^{1}}{3^{\frac{9}{2}}} \\
 & = 3^{1-\frac{9}{2}} \\
 & = 3^{-\frac{7}{2}}.
\end{align*}
}%

\item   $\displaystyle \frac{\pi^3}{\pi^{-1}\sqrt{\pi^5}}$

\end{enumerate}
% end homework exponent-simplify

\item % begin homework logarithms-combine
Express each of the following as a single logarithm.  

\begin{enumerate}
\item   $\ln 4 + \ln 6 - \ln 5$

\pointsii{2} $2\ln 2 - 3\ln 3 + 4\ln 4$

\solution{%
\begin{align*}
2\ln 2 - 3\ln 3 + 4\ln 4 & = \ln 2^2 - \ln 3^3 + \ln 4^4 \\
 & = \ln 4 - \ln 27 + \ln 256 \\
 & = \ln \Big( \frac{4}{27}\Big) + \ln 256 \\
 & = \ln \Big( \frac{4\cdot 256}{27}\Big) \\
 & = \ln \Big( \frac{1024}{27}\Big).
\end{align*}
}%

\item   $\ln 36 - 2\ln 3 - 3\ln 2$

\end{enumerate}
% end homework logarithms-combine

\item % begin homework logarithms-basic2
Use the definition of a logarithm to evaluate each of the following without using a calculator.  

\begin{enumerate}
\item   $\log_2 16$

\item   $\log_3 (1/9)$

\item   $\log_{10} 1000$

\item   $\log_{6} 36^{-2/3}$

\item   $\log_{2} (8\sqrt{2})$

\item $\log_7(49^x/343^y)$

%\solution{%
%\begin{align*}
%\log_7(49^x/343^y) & = \log_749^x - \log_7343^y \\
% & = x\log_749 - y\log_7343 \\
%\intertext{But $49 = 7^2$ and $343=7^3$, therefore}
%\log_7(49^x/343^y) & = 2x-3y.
%\end{align*}
%}%


\end{enumerate}
% end homework logarithms-basic2

\item % begin homework logarithms-equations2
Solve each equation for $x$.  
Then use a calculator to give an approximate answer in decimal notation.  
\begin{enumerate}
\item $\ln (3x-10)=2$.
\answer{$\frac{e^2+10}{3}\approx 5.796$}

\item $\ln (x^2-1)=3$.
\answer{$\pm \sqrt{e^3+1}\approx \pm 4.592$}

\item $e^{2x}-3e^x+2=0$.
\answer{$x=\ln 2\approx 0.693, ~~~, x=0$}

\item $2^{x-5}=3$.
\answer{$\log_2 3+5= \frac{\ln 3}{\ln 2}+5 \approx 6.585$}

\pointsii{5}  $\ln x+\ln (x-1)=1$.
\hiddenanswer{$\frac{1}{2}\left(1+\sqrt{1+4e}\right)\approx 2.223$}

\solution{%
\begin{align*}
\ln x + \ln (x-1) & = 1 \\
\ln (x^2-x) & = 1 \\
e^{\ln (x^2-x)} & = e^1 \\
x^2-x & = e \\
x^2-x-e & = 0 \\
\text{Quadratic formula:}\quad x & = \frac{-(-1)\pm \sqrt{(-1)^2-4(1)(-e)}}{2(1)} \\
 & = \frac{1\pm \sqrt{1+4e}}{2}.
\end{align*}
But $\frac{1-\sqrt{1+4e}}{2}$ is negative, so $\ln\frac{1-\sqrt{1+4e}}{2}$ is undefined.  
Hence the only solution is $x = \frac{1+\sqrt{1+4e}}{2}\approx 2.2229$.  
}%

\item $e^{3x+1}=k$.
\answer{$\frac{\ln k-1}{3}$}

\item $e- e^{-2x}=1$.
\answer{$-\frac12\ln (e-1)\approx -0.271$}

\item $10(1+e^{-x})^{-1}=3$.
\answer{$-\ln \frac73 =\ln \frac37 \approx -0.847$}

\item $\ln (\ln x)=1$.
\answer{$e^e\approx 15.154$}

\item $e^{2x}-e^x-6=0$.
\answer{$x=\ln 3$}


\end{enumerate}
% end homework logarithms-equations2

\item % begin homework inverse-functions2
Find the inverse function and its domain. 
\begin{enumerate}
\item  $y=\ln (x+3)$.
\answer{$f^{-1}(x)=e^x-3$}

\solution{%
\begin{align*}
y & = \ln (x+3) \\
e^y & = e^{\ln (x+3)} \\
e^y & = x + 3 \\
e^y - 3 & = x \\
\text{Therefore} \quad f^{-1}(y) & = e^y - 3.
\end{align*}
The domain of $e^y$ is all real numbers, so the domain of $f^{-1}$ is all real numbers.  
}%

\item $f(x)=e^{x^3}$.
\answer{$f^{-1}(x)=\sqrt[3]{\ln x}, \quad x>0$}

\item $y=(\ln x)^2$, $x\geq 1$.
\answer{$f^{-1}(x)=e^{\sqrt{x}}, \quad x\geq 0 $}

\pointsii{5}  $y=\frac{e^x}{1+2e^x}$.
\hiddenanswer{$f^{-1}(x)= \ln \left(\frac{x}{1-2x}\right) $, \quad $x\in (0, \frac12) $}

\solution{%
\begin{align*}
y & = \frac{e^x}{1+2e^x} \\
y(1+2e^x) & = e^x \\
y & = e^x(1-2y) \\
\frac{y}{1-2y} & = e^x \\
\ln\frac{y}{1-2y} & = \ln e^x \\
\ln\frac{y}{1-2y} & = x \\
\text{Therefore} \quad f^{-1}(y) & = \ln\frac{y}{1-2y}.
\end{align*}
The natural logarithm function is only defined for positive input values.  
Therefore the domain is the set of all $y$ for which 
\begin{align*}
\frac{y}{1-2y} & > 0.
\end{align*}
This inequality holds if the numerator and denominator are both positive or both negative.  
This happens if either
\begin{enumerate}
\item  $y > 0$ and $y < 1/2$, or 
\item  $y < 0$ and $y > 1/2$.
\end{enumerate}
The latter option is impossible, so the domain is $\{ y \in \mathbb{R} \ | \ 0 < y < 1/2\}$.  
}%

\end{enumerate}
% end homework inverse-functions2

\item % begin homework inverse-functions3
Find the inverse function. You are asked to do the algebra only; you are not asked to determine the domain or range of the function or its inverse. 
\begin{enumerate}[ref={\fcProblemRef}]
\item $f(x)= 3x^2+4x-7$, where $x\geq -\frac{2}{3}$.
\answer{$f^{-1}(x)= -\frac{2}3+\frac{\sqrt{25+3x}}{3}, \quad x\geq -\frac{25}{3}$}

\item $f(x)= 2x^2+3x-5$, where $x\geq -\frac{3}{4}$.
\answer{$f^{-1}(x)=-\frac{3}{4}+\frac{\sqrt{49+8x}}{4}, \quad x\geq -\frac{49}{8}$}

\item $f(x)= \frac{2x+5}{x-4}$, where $x\neq 4$.
\answer{$f^{-1}(x)=\frac{4x+5}{x-2}, \quad x\neq 2$}

\pointsii{3} \label{problemFindInversef=(3x+5)/(2x-4)} $f(x)= \frac{3x+5}{2x-4}$, where $x\neq 2$.
\hiddenanswer{$f^{-1}(x)=\frac{4x+5}{2x-3}, \quad x\neq \frac{3}{2}$}



\item $f(x)=2^{2x}+2^{x}-2$.
\answer{$f^{-1}(x) =\log_2\frac{-1+\sqrt{9+4x}}{2}, \quad x\geq -2$}

\end{enumerate}
% end homework inverse-functions3

\item % begin homework exponent-derivative
Differentiate each function.  

\begin{enumerate}
\item $\displaystyle f(x) = \frac{e^x}{1+2e^x}$.  

\pointsii{3}  $r(t) = Ae^{-kt^2}$, where $A$ and $k$ are unknown constants.  

\solution{%
\begin{align*}
r & = Ae^{-kt^2}. \\
\text{Let}\quad u & = -kt^2. \\
\text{Then}\quad r & = Ae^u. \\
\text{Chain Rule}\quad \frac{\diff r}{\diff t} & = \frac{\diff r}{\diff u}\frac{\diff u}{\diff t} \\
 & = (Ae^u)(-2kt) \\
 & = -2Akte^{-kt^2}.
\end{align*}
}%

\item $\displaystyle y = \frac{e^x}{2}(\sin x + \cos x)$.  
\end{enumerate}
% end homework exponent-derivative

\homeworkEnd
\end{comment}

\homeworkStart{3}{Friday March 21, 2014}
\item  % begin homework inverse-trig-evaluated-on-trig
Find each of the following values.  Express your answers precisely, not as decimals.  
\begin{enumerate}
\item $\arcsin(\sin 4)$  
\item $\arccos(\cos (-2))$  
\pointsii{2}  $\arctan(\tan 5)$  
\solution{
$3\pi/2\approx 4.71$ and $2\pi\approx 6.28$, so 
\begin{align*}
\frac{3\pi}{2} & < 5 < 2\pi \\
\text{Therefore } \quad -\frac{\pi}{2} & < 5-2\pi < 0 < \frac{\pi}{2}.
\end{align*}
Therefore $5-2\pi$ is in the restricted domain of the tangent function.  
Moreover, the tangent function is $\pi$-periodic, so $\tan 5 = \tan (5-2\pi)$.  
Therefore $\arctan (\tan 5)) = 5 - 2\pi$.  
}
\end{enumerate}
% end homework inverse-trig-evaluated-on-trig

\item  Let $x\in (0,1)$. Express the following using $x$ and $\sqrt{1-x^2}$.  
\begin{multicols}{2}
\begin{enumerate}
\item $\sin(\Arcsin (x))$. \answer{$x$}
\pointsii{3} $\sin(2\Arcsin (x))$. \answer{$2x\sqrt{1-x^2} $}
\item $\sin(3\Arcsin (x))$. \answer{$ -4x^3+3x $}
\item $\sin(\Arccos (x))$. \answer{$\sqrt{1-x^2} $}
\item $\sin(2\Arccos (x))$. \answer{$2x \sqrt{1-x^2 }$}
\item $\sin(3\Arccos (x))$. 
\answer{
\begin{tabular}{l}
$\left(4x^2-1\right)\sqrt{1-x^2}$ \\= $-4\left(\sqrt{1-x^2}\right)^3+3\sqrt{1-x^2} $
\end{tabular}
}
\item $\cos(2\Arcsin (x))$. \answer{$ 1-2x^2$}
\item $\cos(3\Arccos (x))$. \answer{$4x^3-3x $}
\end{enumerate}
\end{multicols}

\begin{enumerate}
\setcounter{enumii}{1}
\item  
\solution{%
Let $y = \Arcsin x$.  Then $\sin x = y$, and we can draw a right triangle with opposite side length $x$ and hypotenuse length $1$ to find the other trigonometric ratios of $y$.  

\begin{center}
\psset{xunit=1.0cm,yunit=1.0cm,algebraic=true,dotstyle=o,dotsize=3pt 0,linewidth=0.8pt,arrowsize=3pt 2,arrowinset=0.25}
\begin{pspicture*}(-3.33,-6.11)(14.05,6.58)
\psline(0,0)(4,0)
\psline(0,0)(4,3)
\psline(4,3)(4,0)
\psline(4,0.2)(3.8,0.2)
\psline(3.8,0.2)(3.8,0)
\rput[tl](0.83,0.5){$y$}
\rput[tl](1.56,1.82){$1$}
\rput[tl](4.1,1.4){$x$}
\rput[tl](1.7,-0.05){$\sqrt{1-x^2}$}
\parametricplot{0.0}{0.6435011087932844}{1*0.66*cos(t)+0*0.66*sin(t)+0|0*0.66*cos(t)+1*0.66*sin(t)+0}
\end{pspicture*}
\end{center}

Then $\cos y = \sqrt{1-x^2}/1 = \sqrt{1-x^2}$.  
Now we use the double angle formula to find $\sin(2\Arcsin x)$.  

\begin{align*}
\sin (2 \Arcsin x) & = \sin (2y) \\
& = 2\sin y\cos y \\
& = 2x\sqrt{1-x^2}.
\end{align*}
}%
\item
\solution{%
Use the result of the previous question.  
This also requires the addition formula for sine: 
\[
\sin(A+B) = \sin A \cos B + \sin B\cos A,
\]
and the double angle formula for cosine:
\[
\cos (2y) = \cos^2 y - \sin^2 y.
\]  
\begin{align*}
\sin(3\Arcsin x) & = \sin(3y) \\
& = \sin (2y + y) \\
\text{Addition formula: } \quad & = \sin(2y)\cos y + \sin y \cos (2y) \\
\text{Double angle formulas: } \quad & = (2\sin y \cos y)\cos y + \sin y (\cos^2 y - \sin^2 y) \\
& = 2\sin y \cos^2 y + \sin y \cos^2 y - \sin^3 y \\
& = 3\sin y \cos^2 y - \sin^3 y \\
& = 3x(1-x^2) - x^3 \\
& = 3x - 4x^3.
\end{align*}
}%
\end{enumerate}

\points{5} % begin homework logarithm-physics
A particle moves in such a way that, after $t$ seconds, it is $s(t) = \ln \left(2-t+t^2\right)$ m to the right of the origin.  
\begin{enumerate}
\item  What is the closest it comes to the origin?  

\solution{%
\begin{align*}
s'(t) & = \frac{-1+2t}{2-t+t^2}. \\
\text{Set } \quad s'(t) & = 0. \\
\frac{-1+2t}{2-t+t^2} & = 0 \\
-1+2t & = 0 \\
t & = \frac{1}{2}.
\end{align*}

Therefore the position function has a critical number at $t = \frac{1}{2}$.  
The parabola $2-t+t^2$ has a global minimum at $t = \frac{1}{2}$, and the natural logarithm function is an increasing function, so $\ln\left(2-t+t^2\right)$ also has a global minimum at $t=\frac{1}{2}$.  
The minimum value is $s(\frac{1}{2}) = \ln\left(2- \frac{1}{2} + \left( \frac{ 1}{2} \right)^2\right) = \ln\left( \frac{7}{4} \right)\approx 0.5596$ m.  
}%

\item  What is its acceleration when it is closest to the origin?  

\solution{%
\begin{align*}
v(t) = s'(t) & = \frac{-1+2t}{2-t+t^2}. \\
a(t) = s''(t) & = \frac{(2-t+t^2)(2) - (-1+2t)(-1+2t)}{(2-t+t^2)^2} \\
 & = \frac{(4-2t+2t^2) - (1-4t+4t^2)}{(2-t+t^2)^2} \\
 & = \frac{3+2t-2t^2}{(2-t+t^2)^2}. \\
\text{Plug in $t = \frac{1}{2}$:} \quad a\left(\frac{1}{2}\right) & = \frac{3+2(\frac{1}{2})-2(\frac{1}{2})^2}{\left(\frac{7}{4}\right)^2} \\
& = \frac{3+1-\frac{1}{2}}{\frac{49}{16}} \\
& = \frac{\frac{7}{2}}{\frac{49}{16}} \\
& = \frac{8}{7}.
\end{align*}
Therefore the particle is accelerating at a rate of $\frac{8}{7}\approx 1.1429$ m/s$^2$ when it is closest to the origin.  
}%

\item  For which values of $t$ is the position function $s(t)$ defined?  

\solution{%
The natural logarithm function is defined for all positive input values.  
The formula $y = 2-t+t^2$ is always positive.  
To see this, note that its graph is a parabola with discriminant $b^2-4ac = (-1)^2-4(1)(2) = -7$, which is negative.  
This means that the graph of $y = 2-t+t^2$ never touches the $x$-axis, and hence it is either always positive or always negative.  
Since $y(0) = 2$ is positive, this means all values are positive.  
Therefore $\ln(2-t+t^2)$ is defined for all input values $t$.  
}%

\end{enumerate}
% end homework logarithm-physics

\newpage
\points{5} % begin homework implicit-inverse-trig1
The variables $x$ and $y$ are related by
\[
x\Arctan y + y\Arctan x = \frac{\pi}{2}.
\]

\begin{enumerate}
\item   Show that $(1,1)$ is on the graph of this relation.  

\solution{%
\begin{align*}
\text{LS} & = 1\cdot \Arctan 1 + 1\cdot \Arctan 1 & \text{RS} & = \frac{\pi}{2}. \\
 & = 1\cdot \frac{\pi}{4} + 1\cdot \frac{\pi}{4}  & & \\
 & = \frac{\pi}{2}.  & & 
\end{align*}
The fact that the left side equals the right side when we plug in $x = 1$ and $y = 1$ means that the point $(1,1)$ is on the graph of the relation.  
}%

\item   Find $\frac{\diff y}{\diff x}$ in terms of $x$ and $y$.  

\solution{%
Differentiate implicitly.
\begin{align*}
\left((x)\frac{\diff}{\diff x}(\Arctan y) + (\Arctan y)\frac{\diff}{\diff x}(x)\right)  + \left((y)\frac{\diff}{\diff x}(\Arctan x) + (\Arctan x)\frac{\diff}{\diff x}(y)\right)  & = 0 \\
x\cdot \frac{1}{1+y^2}\cdot \frac{\diff y}{\diff x} + (\Arctan y)1 + y\cdot \frac{1}{1+x^2} + (\Arctan x)\frac{\diff y}{\diff x} & = 0 \\
\frac{x}{1+y^2}\frac{\diff y}{\diff x} + \Arctan y + \frac{y}{1+x^2} + \frac{\diff y}{\diff x}\Arctan x & = 0.
\end{align*}
Rearrange to isolate $\frac{\diff y}{\diff x}$ on one side.  
\begin{align*}
\frac{x}{1+y^2}\frac{\diff y}{\diff x} +  \frac{\diff y}{\diff x}\Arctan x & = - \frac{y}{1+x^2} - \Arctan y \\
\left(\frac{x}{1+y^2} + \Arctan x\right)\frac{\diff y}{\diff x} & = - \left(\frac{y}{1+x^2} + \Arctan y\right) \\
\frac{\diff y}{\diff x} & = -\frac{\frac{y}{1+x^2}+\Arctan y}{\frac{x}{1+y^2}+\Arctan x}. 
\end{align*}

}%

\item   Find the equation of the tangent to the graph at $(1,1)$.  

\solution{%
To find the slope of the tangent, plug in $x=1,y=1$ to the formula for $\frac{\diff y}{\diff x}$.  

\begin{align*}
\frac{\diff y}{\diff x} & = -\frac{\frac{1}{1+1^2}+\Arctan 1}{\frac{1}{1+1^2}+\Arctan 1} \\
& = -1.
\end{align*}

Now use the point $(1,1)$ to find an equation for the tangent line.  
\begin{align*}
y - 1 & = (-1)(x-1) \\
y & = -x +2.
\end{align*}
}%

\end{enumerate}
% end homework implicit-inverse-trig1

\points{5} % begin homework implicit-inverse-trig2
The variables $x$ and $y$ are related by
\[
x^2y+xy^2+\arcsin x = \frac{\pi}{6}.
\]

\begin{enumerate}
\item   Find all points on the graph of this relation for which $x = \frac{1}{2}$.  

\solution{%
Set $x = \frac{1}{2}$ and solve for $y$.  
\begin{align*}
\left(\frac{1}{2}\right)^2y+\frac{1}{2}y^2 + \arcsin \frac{1}{2} & = \frac{\pi}{6} \\
\frac{1}{4}y + \frac{1}{2}y^2 + \frac{\pi}{6} & = \frac{\pi}{6} \\
\frac{1}{4}y + \frac{1}{2}y^2  & = 0 \\
\frac{1}{4}y(1 + 2y)  & = 0,
\end{align*}
so $y = 0$ or $y = -\frac{1}{2}$.  
Therefore $(\frac{1}{2},0)$ and $(\frac{1}{2},-\frac{1}{2})$ are the points on the graph of the relation for which $x = \frac{1}{2}$.  
}%

\item   Find $\frac{\diff y}{\diff x}$ in terms of $x$ and $y$.  

\solution{%
Differentiate implicitly.
\begin{align*}
\left((x^2)\frac{\diff}{\diff x}(y) + (y)\frac{\diff}{\diff x}(x^2)\right) + \left( (x)\frac{\diff}{\diff x}(y^2) + (y^2)\frac{\diff}{\diff x}(x)\right) + \frac{1}{\sqrt{1-x^2}} & = 0 \\
x^2\frac{\diff y}{\diff x} + y(2x) + x(2y)\frac{\diff y}{\diff x} + y^2 + \frac{1}{\sqrt{1-x^2}} & = 0 \\
x^2\frac{\diff y}{\diff x} + 2xy + 2xy\frac{\diff y}{\diff x} + y^2 + \frac{1}{\sqrt{1-x^2}} & = 0.
\end{align*}
Rearrange to isolate $\frac{\diff y}{\diff x}$ on one side.  
\begin{align*}
x^2\frac{\diff y}{\diff x} + 2xy\frac{\diff y}{\diff x} & = -y^2-2xy-\frac{1}{\sqrt{1-x^2}} \\
(x^2+2xy)\frac{\diff y}{\diff x} & = -\left(y^2+2xy+\frac{1}{\sqrt{1-x^2}}\right) \\
\frac{\diff y}{\diff x} & = -\frac{y^2+2xy+\frac{1}{\sqrt{1-x^2}}}{x^2+2xy}.
\end{align*}

}%

\item   Find the equation of the tangent to the graph at each of the points you found in the first part.  

\solution{%
To find the slope of the tangent at $(\frac{1}{2},0)$, plug in $x=\frac{1}{2},y=0$ to the formula for $\frac{\diff y}{\diff x}$.  

\begin{align*}
\frac{\diff y}{\diff x} & = -\frac{(0)^2+2(\frac{1}{2})(0) + \frac{1}{\sqrt{1-(\frac{1}{2})^2}}}{(\frac{1}{2})^2+2(\frac{1}{2})(0)} \\
& = -\frac{0+0+\frac{1}{\sqrt{\frac{3}{4}}}}{\frac{1}{4} + 0} \\
& = -\frac{\frac{1}{\frac{\sqrt{3}}{2}}}{\frac{1}{4}} \\
& = -\frac{2}{\sqrt{3}}\cdot \frac{4}{1} \\
& = -\frac{8}{\sqrt{3}}.
\end{align*}

Now use the point $(\frac{1}{2},0)$ to find an equation for the tangent line.  
\begin{align*}
y - 0 & = -\frac{8}{\sqrt{3}}\left(x-\frac{1}{2}\right) \\
y & = -\frac{8}{\sqrt{3}}x +\frac{4}{\sqrt{3}}.
\end{align*}

This is the equation for one of the tangent lines.  

To find the slope of the tangent at $\left(\frac{1}{2},-\frac{1}{2}\right)$, plug in $x=\frac{1}{2},y=-\frac{1}{2}$ to the formula for $\frac{\diff y}{\diff x}$.  

\begin{align*}
\frac{\diff y}{\diff x} & = -\frac{(-\frac{1}{2})^2+2(\frac{1}{2})(-\frac{1}{2}) + \frac{1}{\sqrt{1-(\frac{1}{2})^2}}}{(\frac{1}{2})^2+2(\frac{1}{2})(-\frac{1}{2})} \\
& = -\frac{\frac{1}{4}-\frac{1}{2}+\frac{1}{\sqrt{\frac{3}{4}}}}{\frac{1}{4} -\frac{1}{2}} \\
& = -\frac{-\frac{1}{4}+\frac{2}{\sqrt{3}}}{-\frac{1}{4}} \\
& = \frac{-\frac{1}{4}+\frac{2}{\sqrt{3}}}{\frac{1}{4}} \\
& = 4\left(-\frac{1}{4}+\frac{2}{\sqrt{3}}\right) \\
& = -1 + \frac{8}{\sqrt{3}}.
\end{align*}

Now use the point $(\frac{1}{2},-\frac{1}{2})$ to find an equation for the tangent line.  
\begin{align*}
y - \left(-\frac{1}{2}\right) & = \left(-1 + \frac{8}{\sqrt{3}}\right) \left(x - \frac{ 1}{2}\right) \\
y  & = \left(-1 + \frac{8}{\sqrt{3}}\right)x +\frac{1}{2} - \frac{4}{\sqrt{3}} - \frac{1}{2} \\
y  & = \left(-1 + \frac{8}{\sqrt{3}}\right)x - \frac{4}{\sqrt{3}},
\end{align*}
and this is the equation of the other tangent line.  
}%


\end{enumerate}
% end homework implicit-inverse-trig2

\homeworkEnd

\begin{comment}
\end{comment}

