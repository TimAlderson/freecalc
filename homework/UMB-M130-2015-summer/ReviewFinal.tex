\documentclass{article}
%\addtolength{\hoffset}{-3.5cm}
%\addtolength{\textwidth}{6.8cm}
%\addtolength{\voffset}{-3cm}
%\addtolength{\textheight}{6cm}
\ProvidesPackage{homework-problems}
\usepackage{amsmath, amsfonts, amssymb, verbatim, hyperref, ifthen}
\usepackage{auto-pst-pdf}
\usepackage{pst-plot}
\usepackage{multicol}
\renewcommand{\Re}{\mathrm{Re~}}
\renewcommand{\Im}{\mathrm{Im~}}
\newcommand{\doublebrace}[4]{\left\{\begin{array}{ll} #1 & #2 \\#3 & #4  \end{array} \right.}
\newcommand{\triplebrace}[6]{\left\{\begin{array}{ll} #1 & #2 \\#3 & #4  \\#5 & #6\end{array} \right.}
\newcommand{\bigFatWarning}{ %\textbf{This homework contains copyrighted material from  James Stewart, Calculus, 7th edition, 2012. You are not permitted to copy this file for any purpose other than completing your homework. You are not allowed to give a copy of this file to anyone outside of our course. }
}
\newenvironment{solution}%
{\begin{proof}[\bfseries\upshape Solution]\renewcommand{\qedsymbol}{}}%
{\end{proof}}%
\newcommand{\ans}[1]{\iftoggle{solutions}{\begin{solution}#1\end{solution}}{}}
\newcommand{\homeworkEnd}{\end{enumerate}\end{document}}
\newcommand{\homeworkStart}[2]{\title{\course \\ Homework \ #1}\date{%
\ifthenelse{\equal{#2}{}}{}{%
Due #2 at \deadline}}%
\begin{document}\maketitle\begin{enumerate}
}%
\newcommand{\points}[1]{\stepcounter{enumi}\item[ ({\bf #1 mark\ifthenelse{\equal{#1}{1}}{}{s}}) \arabic{enumi}.]}
\newcommand{\pointsii}[1]{\stepcounter{enumii}\item[ ({\bf #1 mark\ifthenelse{\equal{#1}{1}}{}{s}}) (\alph{enumii})]}
\newcommand{\answer}[1]{ \hfill{~} \rotatebox{180}{ answer: #1}}
 %warning folder paths are relative to the file that uses the includepackage

\renewcommand{\answer}[1]{\iftoggle{answers}{ \hfill{~} \rotatebox{180}{\tiny answer: #1}}{} }
\renewcommand{\hiddenanswer}{\answer}
\renewcommand{\points}[1]{\item}
\renewcommand{\pointsii}[1]{\item}
\renewcommand{\Arctan}{\arctan}
\renewcommand{\Arccos}{\arccos}
\renewcommand{\Arcsin}{\arcsin}
\renewcommand{\Arccot}{\operatorname{arccot}}


\toggletrue{solutions}
\toggletrue{answers}
\renewcommand{\fcProblemRef}{\theproblem.\theenumi}
\renewcommand{\fcSubProblemRef}{\theenumi.\theenumii}


\newcommand{\hide}[1]{}
\newtheorem{problem}{Problem}
\pagestyle{empty}
\begin{document}
\begin{center}
\Large
Final Review sheet \\ Math 130 Precalculus \\ \normalsize Summer 2015 \\ Instructor: Todor Milev
\end{center}
%\noindent \textbf{Name:\underline{~~~~~~~~~~~~~~~~~~~~~~~} } \hfill{~}



\noindent The exam is closed textbook. \textbf{No electronic devices are allowed during the exam. } The exam will contain 9 problems, of the 9 problem types given in this review sheet. The material on the test includes the material from Lecture 1 to Lecture 17. The time for the final will be 150  minutes (2 hours and a half): from 10:00 to 12:30. %You are allowed one single formula sheet, handwritten by you. No template problem solutions are allowed. The sheet will be collected with the test. Photocopied formula sheets are not allowed. 

\begin{problem}
A set of lines is given by a pair of points. The answer key has not been proofread, use with caution.

\begin{tabular}{ccc}
Line name& Point on the line& Second point \\
$L_1$& $(1,2)$ & $(2,-1)$\\
$L_2$& $(1,1)$ & $(2,-2)$\\
$L_3$& $(0,1)$ & $(1,0)$\\
$L_4$& $(3,5)$ & $(7,-11)$\\
\end{tabular}

Find the intersection of the indicated pair of lines, or show that no such intersection exists (i.e., the lines are parallel).

\begin{enumerate}[ref={\fcProblemRef}]
\item $L_1$ and $L_2$.

\answer{The lines are parallel and do not intersect.}
\item $L_1$ and $L_3$.

\answer{Intersection: $(2,-1)$.}
\item \label{problemFindIntersectionLinesThrough((1,2),(2,-1))and((3,5),(7,-11))} $L_1$ and $L_4$.

\answer{point $(12,-31)$}
\item $L_2$ and $L_3$. 

\answer{Intersection: $\left(\frac{3}{2},-\frac{1}{2}\right)$.}
\item $L_2$ and $L_4$.

\answer{Intersection: $(13,-35)$.}
\item $L_3$ and $L_4$.

\answer{Intersection: $\left( \frac{16}{3},-\frac{13}{3}\right)$.}
\end{enumerate}
\end{problem}
\solution{\ref{problemFindIntersectionLinesThrough((1,2),(2,-1))and((3,5),(7,-11))}

An equation of a line passing through the points $(x_1, y_1)$ and $(x_2, y_2)$ is given by:
\[
\renewcommand{\arraystretch}{1.4}
\begin{array}{rcl}
\displaystyle(x_2-x_1)(y-y_1)&=&\displaystyle(y_2-y_1)(x-x_1)\\
\multicolumn{3}{c}{\text{ or alternatively if } x_1\neq x_2:}\\
\displaystyle y-y_1&=&\displaystyle \frac{y_2-y_1}{x_2-x_1}(x-x_1)\\
\multicolumn{3}{c}{\text{ or also }}\\
\displaystyle y-y_2&=&\displaystyle \frac{y_2-y_1}{x_2-x_1}(x-x_2).\\
\end{array}
\]
We recall that if $x_1\neq x_2$, the number $m=\frac{y_2-y_1}{x_2-x_1}$ is called the slope of the line. Therefore an equation of $L_1$ is given by:

\[
\begin{array}{rcl}
\displaystyle m_1&=&\displaystyle \frac{ -1-2}{2-1}=-3\\
y-2&=&m_1(x-1)=-3(x-1)\\
y+3x-5&=&0.
\end{array}
\]
Similarly, an equation of $L_4$ is given by:
\[
\begin{array}{rcl}
\displaystyle m_4&=&\displaystyle \frac{ -11-5}{7-3}=-4\\
y-5&=&m_4(x-3)=-4(x-3)\\
y+4 x-17 &=&0 
\end{array}
\]

To find the intersection of the two lines, we need to solve the system 
\[
\left| \begin{array}{rcl}
y+4 x-17 &=&0 \\
y+3x-5&=&0.\\
\end{array}\right.
\]
A standard method for solving such a system of equation is studied in the subject of Linear algebra. Alternatively, we can solve this system as follows. Observe that the second equation gives $y=-3x+5$. Substitute that into the first equation to get:
\[
\begin{array}{rcll|l}
y+4 x-17 &=&0&&\text{Substitute } y=-3x+5 \\
-3x+5 +4x-17&=&0\\
x&=&12.\\
\end{array}
\]
Therefore $y=-3x+5=-3(12)+5=-36+5=31$. Therefore two lines intersect at the point with coordinates $(12, 31)$. 

To check our work, we can substitute $x=12, y=-31$ in the equation $y+4x-17=0$ of $L_1$ to get that $ -31+4\cdot 12-17=-31+48-17=0$, as expected. Similarly, we can use the equation $y+3x-5=0$ of $L_4$ to check our work: $-31+3\cdot 12-5=-31+36-5=0$, as expected. Finally, a computer-generated plot gives a visual confirmation of our computations.

\psset{xunit=0.1cm, yunit=0.1cm}
\begin{pspicture}(-1,-1)(1,1)
\tiny
\fcAxesStandard{-32}{-32}{32}{32}
\fcGrid[linestyle=dashed, linewidth=0.2]{-32}{-32}{16}{16}{4}{4}{}
\psline[linecolor=blue](-9, 32)(! 37 3 div -32)
\psline[linecolor=green](! -15 4 div 32)(! 12.25 -32)
\fcFullDot{12}{-31}
\fcFullDot{1}{2}
\fcFullDot{2}{-1}
\fcFullDot{3}{5}
\fcFullDot{7}{-11}
\rput[l](12, -31){$~~~(12, -31)$}
\rput[l](7, -11){$~~~(7, -11)$}
\rput[l](3, 5){$~~~(3, 5)$}
\rput[r](1, 2){$(1, 2)~$}
\rput[r](2, -1.3){$(2, -1)~$}
\end{pspicture}
}
\begin{problem}
\begin{itemize}
\item Compute the vertex of the parabola and the $x$- and $y$- intercepts.
\item  Plot the quadratic function roughly by hand.
\end{itemize}

\begin{enumerate}[ref={\fcProblemRef}]
\item $x^2-x+1.$

\item $x^2+x-1.$
\item $x^2-6x+9.$
\item $\frac{1}{2}x^2+2x-1.$
\item $2x^2+3x-5.$
\end{enumerate}
\end{problem}
\begin{problem}
% begin homework inverse-functions3
Find the inverse function. You are asked to do the algebra only; you are not asked to determine the domain or range of the function or its inverse. 
\begin{enumerate}[ref={\fcProblemRef}]
\item $f(x)= 3x^2+4x-7$, where $x\geq -\frac{2}{3}$.
\answer{$f^{-1}(x)= -\frac{2}3+\frac{\sqrt{25+3x}}{3}, \quad x\geq -\frac{25}{3}$}

\item $f(x)= 2x^2+3x-5$, where $x\geq -\frac{3}{4}$.
\answer{$f^{-1}(x)=-\frac{3}{4}+\frac{\sqrt{49+8x}}{4}, \quad x\geq -\frac{49}{8}$}

\item $f(x)= \frac{2x+5}{x-4}$, where $x\neq 4$.
\answer{$f^{-1}(x)=\frac{4x+5}{x-2}, \quad x\neq 2$}

\pointsii{3} \label{problemFindInversef=(3x+5)/(2x-4)} $f(x)= \frac{3x+5}{2x-4}$, where $x\neq 2$.
\hiddenanswer{$f^{-1}(x)=\frac{4x+5}{2x-3}, \quad x\neq \frac{3}{2}$}



\item $f(x)=2^{2x}+2^{x}-2$.
\answer{$f^{-1}(x) =\log_2\frac{-1+\sqrt{9+4x}}{2}, \quad x\geq -2$}

\end{enumerate}
% end homework inverse-functions3

\end{problem}
\solution{
\ref{problemFindIversef=(5x+6)/(4x+5)}. Set $f(x)=y$. Then
\[
\begin{array}{rcl}
y&=&\displaystyle \frac{5x+6}{4x+5}\\
y(4x+5)&=&5x+6\\
x(4y-5)&=&-5y+6\\
x&=&\displaystyle \frac{-5y+6}{4y-5} .
\end{array}
\]
Therefore the function $\displaystyle x=g(y)=\frac{-5y+6}{4y-5}$ is the inverse of $f(x)$. We write $g=f^{-1}$. The function $g=f^{-1}$ is defined for $\displaystyle y\neq \frac{5}{4}$. For our final answer we relabel the argument of $g$ to $x$:

\[
g(x)=f^{-1}(x)= \frac{-5x+6}{4x-5}\quad .
\]

Let us check our work. In order for $f$ and $g$ to be inverses, we need that $g(f(x))$ be equal to $x$.
\[
g(f(x))=  \frac{-5f(x) +6}{4f(x)-5}=  \frac{-5\frac{(5x+6)}{4x+5} +6}{4\frac{(5x+6)}{4x+5}-5}= \frac{-5(5x+6) +6(4x+5)}{4(5x+6)-5(4x+5)}=\frac{-x}{-1}=x\quad ,
\]
as expected.
}

\solution{\ref{problemFindInversef=(3x+5)/(2x-4)}
This is a concise solution written in form suitable for test taking.
\[
\begin{array}{rcl}
y & =& \displaystyle \frac{3x+5}{2x-4} \\
y(2x-4) & =& 3x+5 \\
2xy-4y & =& 3x+5 \\
2xy-3x & =& 4y+5 \\
x(2y-3) & =& 4y+5 \\
x & =& \displaystyle  \frac{4y+5}{2y-3} \\
\text{Therefore}\quad \displaystyle  f^{-1}(y) & =& \displaystyle  \frac{5+4y}{2y -3 } \\
\displaystyle f^{-1}(x) & =& \displaystyle \frac{5+4x}{2x-3}.
\end{array}
\]
}%

\begin{problem}
Solve each equation for $x$. Using a calculator give an ($\approx$) answer in decimal notation. Using calculator verify your approximate solutions.
\begin{multicols}{2}
\begin{enumerate}[ref={\fcProblemRef}]
\item $e^{7-4x}=7$.

\answer{$\frac{7-\ln 7 }{4}\approx 1.263522 $}
\item $\ln (2x-9)=2$.

\answer{$\frac{e^2+9}{2}\approx 8.194528 $}
\item $\ln (x^2-2)=3$.

\answer{$\pm \sqrt{e^3+2}\approx \pm 4.699525 $}
\item $2^{x-3}=5$.

\answer{$\log_2 5+3= \frac{\ln 5}{\ln 2}+3 \approx 5.321928 $}
\item \label{problemlnx+ln(x-1)=1} $\ln x+\ln (x-1)=1$.

\answer{$\frac{1}{2}\left(1+\sqrt{1+4e}\right)\approx 2.223$}
\item $e^{2x+1}=t$.

\answer{$\frac{\ln t-1}{2}$}
\item $\log_2(m x)=c$.

\answer{$\frac{2^c}{m}$}
\item $e- e^{-2x}=1$.

\answer{$-\frac12\ln (e-1)\approx -0.271$}
\item $8(1+e^{-x})^{-1}=3$.

\answer{$-\ln \frac53 =\ln \frac35 \approx -0.510826 $}
\item $\ln (\ln x)=1$.

\answer{$e^e\approx 15.154$}
\item $e^{e^x}=10$.

\answer{$\ln (\ln 10)\approx 0.834$}
\item $\ln(2x+1)=3-\ln x$.

\answer{$\frac{-1+\sqrt{1+8e^3}}{4}\approx 2.928878 $}
\item $e^{2x}-4e^x+3=0$.

\answer{$x=\ln 3\approx 1.098612, ~~~, x=0$}

\item $e^{4x}+3e^{2x}-4=0$. 

\answer{$x=0$}
\item $e^{2x}-e^x-6=0$.

\answer{$x=\ln 3$}
\item $4^{3x}-2^{3x+2}-5=0$. 

\answer{$x=\frac{\log_{2}5}{3}$}
\end{enumerate}
\end{multicols}


\end{problem}

\solution{\ref{problem2^(x-3)=5}
\[\begin{array}{rcll|l}
\displaystyle 2^{x-3} &=& 5 &&\displaystyle  \text{take } \log_2 \\
x-3&=&\displaystyle  \log_2(5) &&\text{add } 3 \text{ to both sides}\\
x&=&\displaystyle \log_2(5)+3 &&\text{answer is complete} \\
&=&\displaystyle \frac{\ln 5}{\ln 2}+3 && \text{optional step: convert to }\ln\\
&\approx &5.321928095 &&\text{calculator}
\end{array}
\]
}

\solution{ \ref{probleme-e^(-2x)=1}

\[
\begin{array}{rcll|l}
\displaystyle e-e^{-2x}&=&1\\
\displaystyle e^{-2x}&=&e-1&& \text{apply }\ln\\
\displaystyle \ln e^{-2x}&=&\displaystyle \ln(e-1)\\
-2x&=&\displaystyle\ln(e-1)\\
x&=&\displaystyle-\frac{1}{2}\ln(e-1)\\
&\approx& -0.270662427&&\text{calculator}
\end{array}
\]

}

\solution{\ref{problemlnx+ln(x-1)=1} %
\begin{align*}
\ln x + \ln (x-1) & = 1 \\
\ln (x^2-x) & = 1 \\
e^{\ln (x^2-x)} & = e^1 \\
x^2-x & = e \\
x^2-x-e & = 0 \\
\text{Quadratic formula:}\quad x & = \frac{-(-1)\pm \sqrt{(-1)^2-4(1)(-e)}}{2(1)} \\
 & = \frac{1\pm \sqrt{1+4e}}{2}.
\end{align*}
However $\frac{1-\sqrt{1+4e}}{2}$ is negative, so $\ln\left( \frac{1-\sqrt{1 + 4e}}{2} \right)$ is undefined.  
Hence the only solution is $x = \frac{1+\sqrt{1+4e}}{2}\approx 2.2229$.  
}%



\begin{problem}
Find the $6$ trigonometric functions of the indicated angle in the indicated right triangle.
\begin{enumerate}[ref={\fcProblemRef}]
\item 
\begin{pspicture}(-1,-1)(1,1)
\pstVerb{20 dict begin
/theX 2 def
/theY 3 def
/adjacentAngle theY theX atan def
/oppositeAngle 90 adjacentAngle sub def
/arcRad 0.3 def
}%
\psline(0,0)(! theX 0)(! theX theY)(0,0)%
\fcPerpendicular{[theX theY]}{[1 0]}{0.2}%
\rput[t](! theX  2 div -0.1){$2$}%
\rput[l](! theX 0.1 add theY 2 div ){$3$}%
%\rput[br](! theX 2 div theY 2 div 0.1 add){$\sqrt{13}$}%
\parametricplot[linecolor=red]{0}{adjacentAngle}{t cos arcRad mul t sin arcRad mul}%
\rput(! adjacentAngle 2 div cos arcRad 0.2 add mul adjacentAngle 2 div sin arcRad 0.2 add mul){$\theta$}%
%\parametricplot[linecolor=red]{-90}{-90 oppositeAngle sub}{t cos arcRad mul theX add t sin arcRad mul theY add}%
%\rput(! -90 oppositeAngle  2 div sub cos arcRad 0.2 add mul theX add -90 oppositeAngle  2 div sub sin arcRad 0.2 add mul theY add){$\theta$}%
\pstVerb{end}%
\end{pspicture}

\answer{$\sin \theta = $, $\cos \theta=$, $\tan \theta= $, $\cot \theta=$, $\sec \theta=$, $\csc \theta=$  }
\item \begin{pspicture}(-1,-1)(1,1)
\pstVerb{20 dict begin
/theX 2 def
/theY 1 def
/adjacentAngle theY theX atan def
/oppositeAngle 90 adjacentAngle sub def
/arcRad 0.6 def
}%
\psline(0,0)(! theX 0)(! theX theY)(0,0)%
\fcPerpendicular{[theX theY]}{[1 0]}{0.2}%
%\rput[t](! theX  2 div -0.1){$1$}%
\rput[l](! theX 0.1 add theY 2 div ){$1$}%
\rput[br](! theX 2 div theY 2 div 0.1 add){$\sqrt{5}$}%
\parametricplot[linecolor=red]{0}{adjacentAngle}{t cos arcRad mul t sin arcRad mul}%
\rput(! adjacentAngle 2 div cos arcRad 0.2 add mul adjacentAngle 2 div sin arcRad 0.2 add mul){$\theta$}%
%\parametricplot[linecolor=red]{-90}{-90 oppositeAngle sub}{t cos arcRad mul theX add t sin arcRad mul theY add}%
%\rput(! -90 oppositeAngle  2 div sub cos arcRad 0.2 add mul theX add -90 oppositeAngle  2 div sub sin arcRad 0.2 add mul theY add){$\theta$}%
\pstVerb{end}%
\end{pspicture}

\answer{$\sin \theta = $, $\cos \theta=$, $\tan \theta= $, $\cot \theta=$, $\sec \theta=$, $\csc \theta=$  }
\item \psset{xunit=0.5cm, yunit=0.5cm} \begin{pspicture}(-1,-1)(1,1)
\pstVerb{20 dict begin
/theX 5 def
/theY 2 def
/adjacentAngle theY theX atan def
/oppositeAngle 90 adjacentAngle sub def
/arcRad 0.6 def
}%
\psline(0,0)(! theX 0)(! theX theY)(0,0)%
\fcPerpendicular{[theX theY]}{[1 0]}{0.2}%
\rput[t](! theX  2 div -0.1){$5$}%
\rput[l](! theX 0.1 add theY 2 div ){$2$}%
%\rput[br](! theX 2 div theY 2 div 0.1 add){$6$}%
%\parametricplot[linecolor=red]{0}{adjacentAngle}{t cos arcRad mul t sin arcRad mul}%
%\rput(! adjacentAngle 2 div cos arcRad 0.2 add mul adjacentAngle 2 div sin arcRad 0.2 add mul){$\theta$}%
\parametricplot[linecolor=red]{-90}{-90 oppositeAngle sub}{t cos arcRad mul theX add t sin arcRad mul theY add}%
\rput(! -90 oppositeAngle  2 div sub cos arcRad 0.2 add mul theX add -90 oppositeAngle  2 div sub sin arcRad 0.2 add mul theY add){$\theta$}%
\pstVerb{end}%
\end{pspicture}

\answer{$\sin \theta = $, $\cos \theta=$, $\tan \theta= $, $\cot \theta=$, $\sec \theta=$, $\csc \theta=$  }
\item \psset{xunit=0.5cm, yunit=0.5cm} \begin{pspicture}(-1,-1)(1,1)
\pstVerb{20 dict begin
/theX 11 sqrt def
/theY 5 def
/adjacentAngle theY theX atan def
/oppositeAngle 90 adjacentAngle sub def
/arcRad 0.6 def
}%
\psline(0,0)(! theX 0)(! theX theY)(0,0)%
\fcPerpendicular{[theX theY]}{[1 0]}{0.2}%
%\rput[t](! theX  2 div -0.1){$1$}%
\rput[l](! theX 0.1 add theY 2 div ){$5$}%
\rput[br](! theX 2 div theY 2 div 0.1 add){$6$}%
%\parametricplot[linecolor=red]{0}{adjacentAngle}{t cos arcRad mul t sin arcRad mul}%
%\rput(! adjacentAngle 2 div cos arcRad 0.2 add mul adjacentAngle 2 div sin arcRad 0.2 add mul){$\theta$}%
\parametricplot[linecolor=red]{-90}{-90 oppositeAngle sub}{t cos arcRad mul theX add t sin arcRad mul theY add}%
\rput(! -90 oppositeAngle  2 div sub cos arcRad 0.2 add mul theX add -90 oppositeAngle  2 div sub sin arcRad 0.2 add mul theY add){$\theta$}%
\pstVerb{end}%
\end{pspicture}

\answer{$\sin \theta = $, $\cos \theta=$, $\tan \theta= $, $\cot \theta=$, $\sec \theta=$, $\csc \theta=$  }

\end{enumerate}
\end{problem}

\begin{problem}
Use the sum-to-product formulas to find all solutions of the trigonometric equation in the interval $[0,2\pi)$. 

Please note that typing a query such as ``solve( sin(x)+sin (3x)=0)'' at \url{www.wolframalpha.com} will provide you with a correct answer and a function plot.

\begin{enumerate}[ref={\fcProblemRef}]
\item $\sin(x)+\sin (3x)=0$.

\answer{$ x= 0, \frac{\pi}{2},\pi, \frac{3\pi}{2} $ }
\item $\cos(x)+\cos (-3x)=0$.

\answer{$ x=\frac{\pi}{4}, \frac{\pi}{2}, \frac{3\pi}{4}, \pi, \frac{5\pi}{4},\frac{3\pi}{2},  \frac{7\pi}{4}$ }
\item $\sin(x)-\sin (3x)=0$.

\answer{$ x= 0, \frac{3\pi}{4} ,\pi, \frac{7\pi}{4}$ }
\item $\cos(2x)-\cos (3x)=0$.

\answer{$ x=0,\frac{2}{5}\pi,\frac{4\pi}{5},  \frac{6\pi}{5}$ }

\end{enumerate}
\end{problem}

\begin{problem}
Derive the trigonometry identities.
\begin{multicols}{3}
\begin{enumerate}
\item $\sin \theta\cot \theta =\cos \theta$.
\item $(\sin \theta +\cos \theta)^2=1+\sin(2\theta)$.
\item $\sec \theta - \cos \theta= \tan \theta \sin \theta$.
\item $\tan^2 \theta-\sin^2 \theta=\tan^2\theta\sin^2\theta$.
\item $\cot^2\theta+\sec^2\theta=\tan^2\theta+\csc^2\theta$.
\item $2\csc 2\theta= \sec \theta \csc \theta$.
\item $\tan (2\theta) =\frac{2\tan \theta}{1-\tan^2\theta} $.
\item $\frac{1}{1-\sin \theta}+ \frac{1}{1+\sin \theta}=2\sec^2\theta$.
\item $\tan \alpha + \tan \beta = \frac{\sin (\alpha+\beta)}{\cos \alpha \cos \beta}$.
\item $\tan (\alpha+\beta)= \frac{\tan \alpha +\tan \beta}{1-\tan \alpha\tan \beta}$.
\item $\sin (3\theta) +\sin \theta = 2 \sin 2\theta \cos \theta $.
\item $\cos (3\theta) = 4\cos^3\theta-3\cos \theta $.
\end{enumerate} 
\end{multicols}

\end{problem}

\begin{problem}
% begin homework inverse-trig-evaluated-on-trig
Find each of the following values.  Express your answers precisely, not as decimals.  
\begin{enumerate}
\item $\arcsin(\sin 4)$.
\item $\arcsin(\sin 0.5)$.
\item $\arcsin(\cos 120^\circ)$.
\item $\arccos(\cos (3))$.
\item $\arccos(\cos (-2))$.
\item $\arccos(\sin (-4))$.
\item \label{problemarctan(tan5)}  $\arctan(\tan 5)$. 
\end{enumerate}
% end homework inverse-trig-evaluated-on-trig

\end{problem}
\solution{\ref{problemarctan(tan5)}
$\frac{3\pi}{2}\approx 4.71$ and $2\pi\approx 6.28$, so 
\begin{align*}
\frac{3\pi}{2} & < 5 < 2\pi \\
\text{Therefore } \quad -\frac{\pi}{2} & < 5-2\pi < 0 < \frac{\pi}{2}.
\end{align*}
Therefore $5-2\pi$ is in the restricted domain of the tangent function.  
Moreover, the tangent function is $\pi$-periodic, so $\tan 5 = \tan (5-2\pi)$.  
Therefore $\arctan (\tan 5) = 5 - 2\pi$.  
}



\begin{problem}
Let $x\in (0,1)$. Express the following using $x$ and $\sqrt{1-x^2}$.  
\begin{multicols}{2}
\begin{enumerate}
\item $\sin(\Arcsin (x))$. \answer{$x$}
\pointsii{3} $\sin(2\Arcsin (x))$. \answer{$2x\sqrt{1-x^2} $}
\item $\sin(3\Arcsin (x))$. \answer{$ -4x^3+3x $}
\item $\sin(\Arccos (x))$. \answer{$\sqrt{1-x^2} $}
\item $\sin(2\Arccos (x))$. \answer{$2x \sqrt{1-x^2 }$}
\item $\sin(3\Arccos (x))$. 
\answer{
\begin{tabular}{l}
$\left(4x^2-1\right)\sqrt{1-x^2}$ \\= $-4\left(\sqrt{1-x^2}\right)^3+3\sqrt{1-x^2} $
\end{tabular}
}
\item $\cos(2\Arcsin (x))$. \answer{$ 1-2x^2$}
\item $\cos(3\Arccos (x))$. \answer{$4x^3-3x $}
\end{enumerate}
\end{multicols}

\begin{enumerate}
\setcounter{enumii}{1}
\item  
\solution{%
Let $y = \Arcsin x$.  Then $\sin x = y$, and we can draw a right triangle with opposite side length $x$ and hypotenuse length $1$ to find the other trigonometric ratios of $y$.  

\begin{center}
\psset{xunit=1.0cm,yunit=1.0cm,algebraic=true,dotstyle=o,dotsize=3pt 0,linewidth=0.8pt,arrowsize=3pt 2,arrowinset=0.25}
\begin{pspicture*}(-3.33,-6.11)(14.05,6.58)
\psline(0,0)(4,0)
\psline(0,0)(4,3)
\psline(4,3)(4,0)
\psline(4,0.2)(3.8,0.2)
\psline(3.8,0.2)(3.8,0)
\rput[tl](0.83,0.5){$y$}
\rput[tl](1.56,1.82){$1$}
\rput[tl](4.1,1.4){$x$}
\rput[tl](1.7,-0.05){$\sqrt{1-x^2}$}
\parametricplot{0.0}{0.6435011087932844}{1*0.66*cos(t)+0*0.66*sin(t)+0|0*0.66*cos(t)+1*0.66*sin(t)+0}
\end{pspicture*}
\end{center}

Then $\cos y = \sqrt{1-x^2}/1 = \sqrt{1-x^2}$.  
Now we use the double angle formula to find $\sin(2\Arcsin x)$.  

\begin{align*}
\sin (2 \Arcsin x) & = \sin (2y) \\
& = 2\sin y\cos y \\
& = 2x\sqrt{1-x^2}.
\end{align*}
}%
\item
\solution{%
Use the result of the previous question.  
This also requires the addition formula for sine: 
\[
\sin(A+B) = \sin A \cos B + \sin B\cos A,
\]
and the double angle formula for cosine:
\[
\cos (2y) = \cos^2 y - \sin^2 y.
\]  
\begin{align*}
\sin(3\Arcsin x) & = \sin(3y) \\
& = \sin (2y + y) \\
\text{Addition formula: } \quad & = \sin(2y)\cos y + \sin y \cos (2y) \\
\text{Double angle formulas: } \quad & = (2\sin y \cos y)\cos y + \sin y (\cos^2 y - \sin^2 y) \\
& = 2\sin y \cos^2 y + \sin y \cos^2 y - \sin^3 y \\
& = 3\sin y \cos^2 y - \sin^3 y \\
& = 3x(1-x^2) - x^3 \\
& = 3x - 4x^3.
\end{align*}
}%
\end{enumerate}

\end{problem}
\solution{\ref{problemsin(2arcsin x)}.
Let $y = \Arcsin x$.  Then $\sin x = y$, and we can draw a right triangle with opposite side length $x$ and hypotenuse length $1$ to find the other trigonometric ratios of $y$.  

\begin{center}
\psset{xunit=1.0cm,yunit=1.0cm,algebraic=true,dotstyle=o,dotsize=3pt 0,linewidth=0.8pt,arrowsize=3pt 2,arrowinset=0.25}
\begin{pspicture*}(-3.33,-6.11)(14.05,6.58)
\psline(0,0)(4,0)
\psline(0,0)(4,3)
\psline(4,3)(4,0)
\psline(4,0.2)(3.8,0.2)
\psline(3.8,0.2)(3.8,0)
\rput[tl](0.83,0.5){$y$}
\rput[tl](1.56,1.82){$1$}
\rput[tl](4.1,1.4){$x$}
\rput[tl](1.7,-0.05){$\sqrt{1-x^2}$}
\parametricplot{0.0}{0.6435011087932844}{1*0.66*cos(t)+0*0.66*sin(t)+0|0*0.66*cos(t)+1*0.66*sin(t)+0}
\end{pspicture*}
\end{center}

Then $\cos y = \sqrt{1-x^2}/1 = \sqrt{1-x^2}$.  
Now we use the double angle formula to find $\sin(2\Arcsin x)$.  

\begin{align*}
\sin (2 \Arcsin x) & = \sin (2y) \\
& = 2\sin y\cos y \\
& = 2x\sqrt{1-x^2}.
\end{align*}
}

\solution{\ref{problemsin(3arcsin x)}. Use the result of problem \ref{problemsin(2arcsin x)}. This also requires the addition formula for sine: 
\[
\sin(A+B) = \sin A \cos B + \sin B\cos A,
\]
and the double angle formula for cosine:
\[
\cos (2y) = \cos^2 y - \sin^2 y.
\]  
\[
\begin{array}{r@{~}c@{~}ll|l}
\sin(3\Arcsin x) & =& \sin(3y) \\
& =& \sin (2y + y) \\
& =& \sin(2y)\cos y + \sin y \cos (2y) &&\text{Use addition formula }\\
& =& (2\sin y \cos y)\cos y + \sin y (\cos^2 y - \sin^2 y)&&\text{Use double angle formulas} \\
& =& 2\sin y \cos^2 y + \sin y \cos^2 y - \sin^3 y \\
& =& 3\sin y \cos^2 y - \sin^3 y \\
& =& 3x(1-x^2) - x^3 \\
& =& 3x - 4x^3.
\end{array}
\]
}
%\vskip 18cm
%\hfill \begin{tabular}{c|c|c|c|c|c|c||c}
%Problem&1 &2&3&4&5&6& $\sum$\\ \hline
%Score&&&&&&&\\ \hline
%Max&17&17&17&17&17&17&102
%\end{tabular} 


\end{document}