\documentclass{article}
%\addtolength{\hoffset}{-3.5cm}
%\addtolength{\textwidth}{6.8cm}
%\addtolength{\voffset}{-3cm}
%\addtolength{\textheight}{6cm}
\ProvidesPackage{homework-problems}
\usepackage{amsmath, amsfonts, amssymb, verbatim, hyperref, ifthen}
\usepackage{auto-pst-pdf}
\usepackage{pst-plot}
\usepackage{multicol}
\renewcommand{\Re}{\mathrm{Re~}}
\renewcommand{\Im}{\mathrm{Im~}}
\newcommand{\doublebrace}[4]{\left\{\begin{array}{ll} #1 & #2 \\#3 & #4  \end{array} \right.}
\newcommand{\triplebrace}[6]{\left\{\begin{array}{ll} #1 & #2 \\#3 & #4  \\#5 & #6\end{array} \right.}
\newcommand{\bigFatWarning}{ %\textbf{This homework contains copyrighted material from  James Stewart, Calculus, 7th edition, 2012. You are not permitted to copy this file for any purpose other than completing your homework. You are not allowed to give a copy of this file to anyone outside of our course. }
}
\newenvironment{solution}%
{\begin{proof}[\bfseries\upshape Solution]\renewcommand{\qedsymbol}{}}%
{\end{proof}}%
\newcommand{\ans}[1]{\iftoggle{solutions}{\begin{solution}#1\end{solution}}{}}
\newcommand{\homeworkEnd}{\end{enumerate}\end{document}}
\newcommand{\homeworkStart}[2]{\title{\course \\ Homework \ #1}\date{%
\ifthenelse{\equal{#2}{}}{}{%
Due #2 at \deadline}}%
\begin{document}\maketitle\begin{enumerate}
}%
\newcommand{\points}[1]{\stepcounter{enumi}\item[ ({\bf #1 mark\ifthenelse{\equal{#1}{1}}{}{s}}) \arabic{enumi}.]}
\newcommand{\pointsii}[1]{\stepcounter{enumii}\item[ ({\bf #1 mark\ifthenelse{\equal{#1}{1}}{}{s}}) (\alph{enumii})]}
\newcommand{\answer}[1]{ \hfill{~} \rotatebox{180}{ answer: #1}}
 %warning folder paths are relative to the file that uses the includepackage

\renewcommand{\answer}[1]{\iftoggle{answers}{ \hfill{~} \rotatebox{180}{\tiny answer: #1}}{} }
\renewcommand{\hiddenanswer}{\answer}
\renewcommand{\points}[1]{\item}
\renewcommand{\pointsii}[1]{\item}
\renewcommand{\Arctan}{\arctan}
\renewcommand{\Arccos}{\arccos}
\renewcommand{\Arcsin}{\arcsin}
\renewcommand{\Arccot}{\operatorname{arccot}}


\toggletrue{solutions}
\toggletrue{answers}
\renewcommand{\fcProblemRef}{\theproblem.\theenumi}
\renewcommand{\fcSubProblemRef}{\theproblem.\theenumi.\theenumii}


\newcommand{\hide}[1]{}
\newtheorem{problem}{Problem}
\pagestyle{empty}
\begin{document}
\begin{center}
\Large
Review sheet Final Exam \\ Math 140 Calculus I \\ \normalsize Summer 2015 \\ Instructor: Todor Milev
\end{center}
%\noindent \textbf{Name:\underline{~~~~~~~~~~~~~~~~~~~~~~~} } \hfill{~}



\noindent The exam is closed textbook. The final exam will contain problems of types similar to the problem types given here. Each problem type given in this review sheet will be represented by at least one exam question. The final will contain 12 problems (in turn consisting in a number of sub-problems). 

\begin{problem}
(Textbook page 70, problems 11-32). 
Evaluate the limit if it exists.
\begin{multicols}{3}
\begin{enumerate}
\item $\displaystyle\lim\limits_{x\to 5}\frac{x^2-6x+5}{x-5} $. 
\answer{4}
\item $\displaystyle\lim\limits_{x\to 4}\frac{x^2-4x}{x^2-3x-4} $.
\answer{$\frac{4}5$}
\item $\displaystyle\lim\limits_{x\to 5}\frac{x^2-5x+6}{x-5} $.
\answer{DNE}
\item $\displaystyle\lim\limits_{x\to -1}\frac{x^2-4x}{x^{2}-3x-4} $.
\answer{DNE}
\item $\displaystyle\lim\limits_{t\to -3}\frac{t^2-9}{2t^2+7t+3} $.
\answer{$\frac{6}{5}$}
\item $\displaystyle\lim\limits_{x\to -1}\frac{2x^2+3x+1}{x^2-2x-3} $.
\answer{$\frac{1}{4}$}
\item $\displaystyle\lim\limits_{h\to 0}\frac{(-5+h)^2-25}{h} $.
\answer{$-10$}
\item $\displaystyle\lim\limits_{h\to 0}\frac{(2+h)^3-8}{h} $.
\answer{$12$}
\item $\displaystyle\lim\limits_{x\to -2}\frac{x+2}{x^3+8} $.
\answer{$\frac{1}{12}$}
\item $\displaystyle\lim\limits_{t\to 1}\frac{t^4-1}{t^3-1} $.
\answer{$\frac{4}{3}$}
\item $\displaystyle\lim\limits_{h\to 0}\frac{\sqrt{9+h}-3}{h} $.
\answer{$\frac{1}{6}$}
\item $\displaystyle\lim\limits_{u\to 2} \frac{\sqrt{4u+1}-3}{u-2}$.
\answer{$\frac{2}{3}$}
\item $\displaystyle\lim\limits_{x\to -4} \frac{\frac{1}{4}+ \frac{1}{x}} {4+x}$.
\answer{$-\frac{1}{16}$}
\item $\displaystyle\lim\limits_{x\to -1} \frac{x^2+2x+1}{x^4-1}$.
\answer{$0$}
\item $\displaystyle\lim\limits_{t\to 0} \frac{\sqrt{1+t}- \sqrt{1-t}}{t}$.
\answer{$1$}
\item $\displaystyle\lim\limits_{t\to 0}\left(\frac{1}t -\frac{1}{t^2+t}\right)$.
\answer{$1$}
\item $\displaystyle\lim\limits_{x\to 16} \frac{4-\sqrt{x}}{16x-x^2}$.
\answer{$\frac{1}{128}$}
\item $\displaystyle\lim\limits_{h \to 0}\frac{(3+h)^{-1}-3^{-1}}{h} $.
\answer{$-\frac{1}{9}$}
\item $\displaystyle\lim\limits_{t\to 0} \left(\frac{1}{t\sqrt{1+t}}-\frac{1}{t} \right)$.
\answer{$-\frac{1}{2}$}
\item $\displaystyle\lim\limits_{x\to -4} \frac{\sqrt{x^2+9}-5}{x+4}$.
\answer{$-\frac{4}{5}$}
\item $\displaystyle\lim\limits_{h\to 0}\frac{(x+h)^3-x^3}{h} $.
\answer{$3x^2$}
\item $\displaystyle\lim\limits_{h\to 0}\frac{\frac{1}{(x+h)^2}-\frac{1}{x^2}}{h} $.
\answer{$-\frac{2}{x^3}$}
\end{enumerate}
\end{multicols}

\end{problem}
\solution{\ref{problemlim(xto2)(x^2-5x+6)/(x-2)}

$
\begin{array}{rcll|l}
\displaystyle 
\displaystyle \lim\limits_{x\to 2}\frac{x^2-5x+6}{x-2} &=&\displaystyle \lim\limits_{x\to 2}\frac{(x-3)\cancel{(x-2)}}{\cancel{x-2}} &&\text{factor and cancel}\\
&=&\displaystyle 2-3=-1
\end{array}
$
}
\solution{\ref{problemlimxto-2(2x^2+x-6)/(x^2-4)}

$\begin{array}{rcll|l}
\displaystyle \lim\limits_{x\to -2} \frac{2x^2+x-6}{x^2-4}&=&\displaystyle \lim\limits_{x\to -2}\frac{(2x-3)\cancel{(x+2)}}{(x-2)\cancel{(x+2)}}&&\text{factor and cancel}\\ 
&=&\displaystyle  \frac{(2(-2)-3)}{-2-2} &&\text{substitute}\\
&=&\displaystyle \frac{7}{4}
\end{array}
$

}
\solution{\ref{limproblem(xto-2)(x^2-4)/(2x^2+5x+2)}

$
\begin{array}{rcll|l}
\displaystyle 
\displaystyle \lim\limits_{x\to 2}\frac{x^2-4}{2x^2+5x+2} &=&\displaystyle \lim\limits_{x\to -2} \frac{(x-2)\cancel{(x+2)}}{(2x+1) \cancel{(x+2)}} &&\text{factor and cancel}\\
&=&\displaystyle \frac{(-2)-2}{2(-2)+1}=\frac{4}{3}
\end{array}
$
}
\solution{
\ref{problemlim(xto-1)(2x^2+3x+1)/(3x^2-2x-5)}

$
\begin{array}{rcll|l}
\displaystyle \lim\limits_{x\to-1}\frac{2x^2+3x+1}{3x^2-2x-5} &=&\displaystyle \lim\limits_{x\to -1}\frac{(2x+1)\cancel{(x+1)}}{(3x-5)\cancel{(x+1)}}&&\text{factor and cancel}\\
&=&\displaystyle \frac{2(-1)+1}{3(-1)-5} =\frac{1}{8}.
\end{array}
$
}

\solution{ \ref{problemlim_hto0_(1/(x+h)^2-1/x^2)/h}

$
\begin{array}{rcl}
\displaystyle \lim\limits_{h\to 0}\frac{\frac{1}{(x+h)^2}-\frac{1}{x^2}}{h} &=&\displaystyle \lim\limits_{h\to 0}\frac{x^2-(x+h)^2}{hx^2(x+h)^2}=\lim\limits_{h\to 0} \frac{x^2-(x^2+2xh+h^2)}{hx^2(x+h)^2}\\
&=&\displaystyle \lim\limits_{h\to 0}\frac{\cancel{h}(-2x+h)}{\cancel{h}x^2(x+h)^2}= \frac{-2x+0}{x^2(x+0)^2}=-\frac{2}{x^3}.
\end{array}
$
}


\solution{\ref{problemlimhto0(1/(2+h)^2-1/4)/h}.

\textbf{Variant I.}

$\begin{array}{rcll|l}
\displaystyle \lim_{h\to 0} \frac{\frac{1}{(2+h)^2}-\frac{1}{4}}{h}&=&\displaystyle \lim_{h\to 0}\frac{\frac{4-(2+h)^2}{4(2+h)^2}}{h}\\
&=&\displaystyle \lim_{h\to 0} \frac{4- (4+4h+h^2)}{4h(2+h)^2}\\
&=&\displaystyle \lim_{h\to 0} \frac{-4h-h^2}{4h(2+h)^2}\\
&=&\displaystyle \lim_{h\to 0} \frac{\cancel{h}(-4-h) }{4\cancel{h}(2+h)^2}&&\text{substitute }h=0\\
&=&\displaystyle \frac{-4-0}{4(2+0)^2}\\
&=&\displaystyle -\frac{1}{4}
\end{array}
$

\textbf{Variant II.}

$\begin{array}{rcll|l}
\displaystyle \lim_{h\to 0} \frac{\frac{1}{(2+h)^2}-\frac{1}{4}}{h}&=&\displaystyle \frac{\diff }{\diff x}\left(\frac{1}{x^2}\right)_{|x=2}\\
&=&\displaystyle \left(\frac{-2}{x^3}\right)_{|x=2}\\
&=&\displaystyle -\frac{1}{4}
\end{array}
$

}


\begin{problem}
Find the limit or show that it does not exist. If the limit does not exist, indicate whether it is $\pm\infty$, or neither. The answer key has not been proofread, use with caution.
\begin{multicols}{3}
\begin{enumerate}[ref={\fcProblemRef}]
\item $\displaystyle \lim\limits_{x\to\infty }\frac{x-2}{2x+1}$.

\answer{$\frac12$}
\item $\displaystyle \lim\limits_{x\to\infty }\frac{1-x^2}{x^3-x-1}$.

\answer{$ 0$}
\item $\displaystyle \lim\limits_{x\to-\infty }\frac{x-2}{x^2+5}$.

\answer{$ 0$}
\item \label{problemlimxtominusinfty(3x^3+2)/(2x^3-4x+5)} $\displaystyle \lim\limits_{x\to-\infty }\frac{3x^3+2}{2x^3-4x+5}$.

\answer{$ \frac{3}{2}$}
\item $\displaystyle \lim\limits_{x\to\infty }\frac{\sqrt{x}+x^2}{\sqrt{x}-x^2}$.

\answer{$-1$}
\item $\displaystyle \lim\limits_{x\to\infty }\frac{3-x\sqrt{x}}{2x^{\frac{3}{2}}-2}$.

\answer{$-\frac12$}
\item $\displaystyle \lim\limits_{x\to\infty }\frac{(2x^2+3)^2}{(x-1)^2(x^2+1)}$.

\answer{$ 4$}
\item $\displaystyle \lim\limits_{x\to\infty }\frac{x^2-3}{\sqrt{x^4+3}}$.

\answer{$1$}
\item $\displaystyle \lim\limits_{x\to\infty }\frac{\sqrt{16x^6-3x}}{x^3+2}$.

\answer{$4$}
\item $\displaystyle \lim\limits_{x\to-\infty }\frac{\sqrt{16x^6-3x}}{x^3+2}$.

\answer{$-4$}
\item $\displaystyle \lim\limits_{x\to\infty}\sqrt{4x^2+x}-2x$.

\answer{$\frac{1}{4}$}
\item $\displaystyle \lim\limits_{x\to-\infty} x+\sqrt{x^2+3x} $.

\answer{$-\frac{3}{2} $}
\item $\lim\limits_{x \to\infty}\sqrt{x^2+2x}- \sqrt{ x^2 -2x} $. 

\answer{ $2 $.}
\item \label{problemlim(xto-infty)sqrt(x^2+x)-sqrt(x^2-x)} $\lim\limits_{x\to -\infty}\sqrt{x^2+x}- \sqrt{ x^2-x}$. 

\answer{$-1 $.}
\item $\displaystyle \lim\limits_{x\to \infty}\sqrt{ x^2+ax}- \sqrt{x^2+bx}$.

\answer{$\frac{a-b}2$}
\item $\displaystyle \lim\limits_{x\to\infty}\cos x$.

\answer{DNE}
\item $\displaystyle \lim\limits_{x\to\infty}\frac{x^4+x}{x^3-x+2}$.

\answer{$\infty$}
\item $\displaystyle \lim\limits_{x\to\infty}\sqrt{x^2+1}$.

\answer{$\infty$}
\item $\displaystyle \lim\limits_{x\to-\infty}(x^4+x^5)$.

\answer{$-\infty$}
\item $\displaystyle \lim\limits_{x\to-\infty}\frac{\sqrt{1+x^6}}{1+x^2}$.

\answer{$\infty$}
\item $\displaystyle \lim\limits_{x\to\infty}(x-\sqrt{x})$.

\answer{$\infty$}
\item $\displaystyle \lim\limits_{x\to\infty}(x^2-x^3)$.

\answer{$-\infty$}
\item $\displaystyle \lim\limits_{x\to\infty}x\sin x$.

\answer{DNE}
\item $\displaystyle \lim\limits_{x\to\infty}\sqrt{x}\sin x$.

\answer{DNE}
\end{enumerate}
\end{multicols}
\end{problem}
\solution{\ref{problemlimxtominusinfty(3x^3+2)/(2x^3-4x+5)}.


\[
\begin{array}{rcll|l}
\displaystyle \lim_{x\to-\infty} \frac{3x^3+2}{2x^3-4x+5}&=&\displaystyle \lim_{x\to-\infty} \frac{\left(3x^3+2\right) \frac{1}{x^3}}{ \left(2x^3-4x +5\right)\frac{1}{ x^3}}&& \begin{array}{l} \text{Divide top}\\ \text{and bottom}\end{array}\\
&=&\displaystyle \lim_{x\to-\infty} \frac{3+\frac{2}{x^3}}{2-\frac{4}{x^2}+\frac{5}{x^3}} && \begin{array}{l} \text{by highest term}\\ \text{in denominator}\end{array}\\
&=&\displaystyle\frac{3+0}{2-0+0}=\frac{3}{2}.
\end{array}
\]
}

\solution{\ref{problemlimxtoinftysqrt(x^2+1)/(x+1)}

$ 
\begin{array}{rcll|l}
\displaystyle \lim_{x\to -\infty} \frac{\sqrt{x^2+1}}{x+1} &=&\displaystyle \lim_{x\to- \infty} \frac{\frac{1}{x}\sqrt{x^2+1}}{\frac{1}{x}(x+1)}=\displaystyle \lim_{x\to- \infty} \frac{-\frac{1}{\sqrt{x^2}}\sqrt{x^2+1}}{\frac{1}{x}(x+1)}&& x=-\sqrt{x^2}, \text{ whenever } x<0\\
&=&\displaystyle \lim_{x\to- \infty} \frac{-\sqrt{\frac{x^2+1}{x^2}}}{1+\frac{1}{x}}= \lim_{x\to- \infty} \frac{-\sqrt{1+\underbrace{\frac{1}{x^2}}_{\to 0}}}{1+\underbrace{\frac{1}{x}}_{x\to 0}}\\
&=&1.
\end{array}
$
}
\solution{\ref{problemLimitxtominusinftysqrt(16x^6-3x)/(x^3+2)}.

$
\begin{array}{rcll|l}
\displaystyle 
\lim\limits_{x\to-\infty }\frac{\sqrt{16x^6-3x}}{x^3+2}&=&\displaystyle \lim\limits_{x\to-\infty } \frac{\sqrt{x^6\left(16-\frac{3}{x^5}\right)}}{x^3+2}\\
&=&\displaystyle \lim\limits_{x\to-\infty } \frac{\sqrt{x^6}\sqrt{\left(16-\frac{3}{x^5}\right)}}{x^3 +2} &&\displaystyle\sqrt{x^6}={\color{red}{-}} x^3\text{ because }x<0 \text{ as }x\to-\infty\\
&=&\displaystyle \lim\limits_{x\to-\infty } \frac{{\color{red}{-}}x^3 \sqrt{\left(16-\frac{3}{x^5}\right)}}{x^3 +2} &&\text{Divide by highest order term in denominator} \\
&=&\displaystyle \lim\limits_{x\to-\infty } \frac{-x^3 \sqrt{\left(16-\frac{3}{x^5}\right)}}{x^3 +2} \\
&=&\displaystyle \lim\limits_{x\to-\infty } \frac{-\cancel{x^3} \sqrt{\left(16-\frac{3}{ x^5}\right)}\frac{1}{\cancel{x^3}}}{\left(x^3 +2\right)\frac{1}{x^3}} \\
&=&\displaystyle \lim\limits_{x\to-\infty } \frac{-\sqrt{\left(16-\underbrace{\frac{3}{ x^5}}_{\to 0}\right)} }{1+\underbrace{\frac{2}{x^3}}_{\to 0}} \\
&=&\displaystyle \frac{-\sqrt{16}}{1}=-4.
\end{array}
$
}
\solution{\ref{problemlimxtoinfty(sqrt(3x^2+2x+1)/(x+1))}

$\begin{array}{rcll|l}
\displaystyle \lim_{x\to \infty} \frac{\sqrt{3x^2+2x+1}}{x+1} &=&\displaystyle \lim_{x\to \infty} \frac{\frac{1}{x} \sqrt{3x^2+2x+1}}{\frac{1}{x}\left(x+1\right) } \\
&=&\displaystyle\lim_{x\to \infty} \frac{ \sqrt{\frac{3x^2+2x+1}{x^2 }}}{\left(1+\frac{1}{x}\right) }\\
&=&\displaystyle \lim_{x\to \infty} \frac{ \sqrt{3+\frac{2}{x}+\frac{1}{x^2 }}}{\left(1+\frac{1}{x}\right) }\\
&=&\displaystyle\frac{\sqrt{3+0+0}}{1+0}\\
&=&\displaystyle \sqrt{3}
\end{array}
$
}

\solution{\ref{problemlim(xto-infty)sqrt(x^2+x)-sqrt(x^2-x)}.
\[ \begin{array}{rcl}
\displaystyle\lim_{x\to -\infty} \sqrt{x^2+x}-\sqrt{x^2-x} &=&\displaystyle\lim_{x\to -\infty} \left(\sqrt{x^2+x}-\sqrt{x^2-x}\right) \frac{ \left(\sqrt{x^2+x}+\sqrt{x^2-x}\right) }{\left( \sqrt{x^2+x}+\sqrt{x^2-x}\right)}
\\
&=&\displaystyle \lim_{x\to -\infty} \frac{x^2+x-(x^2-x) }{\sqrt{x^2+x}+\sqrt{x^2-x} } = \lim_{x\to -\infty} \frac{2x \frac{1}{x} }{\left(\sqrt{x^2+x}+\sqrt{x^2-x} \right)\frac{1}{x}} 
\\&=&\displaystyle \lim_{x\to -\infty} \frac{2}{\frac{\sqrt{x^2+x}}x+\frac{\sqrt{x^2-x}}x }= \lim_{x\to -\infty} \frac{2}{ {\color{red}-} \sqrt{\frac{x^2+x}{x^2}} {\color{red}-} \sqrt{\frac{x^2-x}{x^2}} }\\
&=&\displaystyle \lim_{x\to -\infty} \frac{2}{ - \sqrt{1+\frac1x} - \sqrt{1-\frac1x} }=\frac{2}{-\sqrt{1+0}-\sqrt{1-0}}=-1
\end{array}
\]
The sign highlighted in red arises from the fact that, for negative $x$, we have that $ x={\color{red}-}\sqrt{x^2}$.
}



\begin{problem}
(Textbook page 70, problems 11-32). 
Evaluate the limit if it exists.
\begin{multicols}{3}
\begin{enumerate}
\item $\displaystyle\lim\limits_{x\to 5}\frac{x^2-6x+5}{x-5} $. 
\answer{4}
\item $\displaystyle\lim\limits_{x\to 4}\frac{x^2-4x}{x^2-3x-4} $.
\answer{$\frac{4}5$}
\item $\displaystyle\lim\limits_{x\to 5}\frac{x^2-5x+6}{x-5} $.
\answer{DNE}
\item $\displaystyle\lim\limits_{x\to -1}\frac{x^2-4x}{x^{2}-3x-4} $.
\answer{DNE}
\item $\displaystyle\lim\limits_{t\to -3}\frac{t^2-9}{2t^2+7t+3} $.
\answer{$\frac{6}{5}$}
\item $\displaystyle\lim\limits_{x\to -1}\frac{2x^2+3x+1}{x^2-2x-3} $.
\answer{$\frac{1}{4}$}
\item $\displaystyle\lim\limits_{h\to 0}\frac{(-5+h)^2-25}{h} $.
\answer{$-10$}
\item $\displaystyle\lim\limits_{h\to 0}\frac{(2+h)^3-8}{h} $.
\answer{$12$}
\item $\displaystyle\lim\limits_{x\to -2}\frac{x+2}{x^3+8} $.
\answer{$\frac{1}{12}$}
\item $\displaystyle\lim\limits_{t\to 1}\frac{t^4-1}{t^3-1} $.
\answer{$\frac{4}{3}$}
\item $\displaystyle\lim\limits_{h\to 0}\frac{\sqrt{9+h}-3}{h} $.
\answer{$\frac{1}{6}$}
\item $\displaystyle\lim\limits_{u\to 2} \frac{\sqrt{4u+1}-3}{u-2}$.
\answer{$\frac{2}{3}$}
\item $\displaystyle\lim\limits_{x\to -4} \frac{\frac{1}{4}+ \frac{1}{x}} {4+x}$.
\answer{$-\frac{1}{16}$}
\item $\displaystyle\lim\limits_{x\to -1} \frac{x^2+2x+1}{x^4-1}$.
\answer{$0$}
\item $\displaystyle\lim\limits_{t\to 0} \frac{\sqrt{1+t}- \sqrt{1-t}}{t}$.
\answer{$1$}
\item $\displaystyle\lim\limits_{t\to 0}\left(\frac{1}t -\frac{1}{t^2+t}\right)$.
\answer{$1$}
\item $\displaystyle\lim\limits_{x\to 16} \frac{4-\sqrt{x}}{16x-x^2}$.
\answer{$\frac{1}{128}$}
\item $\displaystyle\lim\limits_{h \to 0}\frac{(3+h)^{-1}-3^{-1}}{h} $.
\answer{$-\frac{1}{9}$}
\item $\displaystyle\lim\limits_{t\to 0} \left(\frac{1}{t\sqrt{1+t}}-\frac{1}{t} \right)$.
\answer{$-\frac{1}{2}$}
\item $\displaystyle\lim\limits_{x\to -4} \frac{\sqrt{x^2+9}-5}{x+4}$.
\answer{$-\frac{4}{5}$}
\item $\displaystyle\lim\limits_{h\to 0}\frac{(x+h)^3-x^3}{h} $.
\answer{$3x^2$}
\item $\displaystyle\lim\limits_{h\to 0}\frac{\frac{1}{(x+h)^2}-\frac{1}{x^2}}{h} $.
\answer{$-\frac{2}{x^3}$}
\end{enumerate}
\end{multicols}

\end{problem}
\solution{\ref{problemlim(xto2)(x^2-5x+6)/(x-2)}

$
\begin{array}{rcll|l}
\displaystyle 
\displaystyle \lim\limits_{x\to 2}\frac{x^2-5x+6}{x-2} &=&\displaystyle \lim\limits_{x\to 2}\frac{(x-3)\cancel{(x-2)}}{\cancel{x-2}} &&\text{factor and cancel}\\
&=&\displaystyle 2-3=-1
\end{array}
$
}
\solution{\ref{problemlimxto-2(2x^2+x-6)/(x^2-4)}

$\begin{array}{rcll|l}
\displaystyle \lim\limits_{x\to -2} \frac{2x^2+x-6}{x^2-4}&=&\displaystyle \lim\limits_{x\to -2}\frac{(2x-3)\cancel{(x+2)}}{(x-2)\cancel{(x+2)}}&&\text{factor and cancel}\\ 
&=&\displaystyle  \frac{(2(-2)-3)}{-2-2} &&\text{substitute}\\
&=&\displaystyle \frac{7}{4}
\end{array}
$

}
\solution{\ref{limproblem(xto-2)(x^2-4)/(2x^2+5x+2)}

$
\begin{array}{rcll|l}
\displaystyle 
\displaystyle \lim\limits_{x\to 2}\frac{x^2-4}{2x^2+5x+2} &=&\displaystyle \lim\limits_{x\to -2} \frac{(x-2)\cancel{(x+2)}}{(2x+1) \cancel{(x+2)}} &&\text{factor and cancel}\\
&=&\displaystyle \frac{(-2)-2}{2(-2)+1}=\frac{4}{3}
\end{array}
$
}
\solution{
\ref{problemlim(xto-1)(2x^2+3x+1)/(3x^2-2x-5)}

$
\begin{array}{rcll|l}
\displaystyle \lim\limits_{x\to-1}\frac{2x^2+3x+1}{3x^2-2x-5} &=&\displaystyle \lim\limits_{x\to -1}\frac{(2x+1)\cancel{(x+1)}}{(3x-5)\cancel{(x+1)}}&&\text{factor and cancel}\\
&=&\displaystyle \frac{2(-1)+1}{3(-1)-5} =\frac{1}{8}.
\end{array}
$
}

\solution{ \ref{problemlim_hto0_(1/(x+h)^2-1/x^2)/h}

$
\begin{array}{rcl}
\displaystyle \lim\limits_{h\to 0}\frac{\frac{1}{(x+h)^2}-\frac{1}{x^2}}{h} &=&\displaystyle \lim\limits_{h\to 0}\frac{x^2-(x+h)^2}{hx^2(x+h)^2}=\lim\limits_{h\to 0} \frac{x^2-(x^2+2xh+h^2)}{hx^2(x+h)^2}\\
&=&\displaystyle \lim\limits_{h\to 0}\frac{\cancel{h}(-2x+h)}{\cancel{h}x^2(x+h)^2}= \frac{-2x+0}{x^2(x+0)^2}=-\frac{2}{x^3}.
\end{array}
$
}


\solution{\ref{problemlimhto0(1/(2+h)^2-1/4)/h}.

\textbf{Variant I.}

$\begin{array}{rcll|l}
\displaystyle \lim_{h\to 0} \frac{\frac{1}{(2+h)^2}-\frac{1}{4}}{h}&=&\displaystyle \lim_{h\to 0}\frac{\frac{4-(2+h)^2}{4(2+h)^2}}{h}\\
&=&\displaystyle \lim_{h\to 0} \frac{4- (4+4h+h^2)}{4h(2+h)^2}\\
&=&\displaystyle \lim_{h\to 0} \frac{-4h-h^2}{4h(2+h)^2}\\
&=&\displaystyle \lim_{h\to 0} \frac{\cancel{h}(-4-h) }{4\cancel{h}(2+h)^2}&&\text{substitute }h=0\\
&=&\displaystyle \frac{-4-0}{4(2+0)^2}\\
&=&\displaystyle -\frac{1}{4}
\end{array}
$

\textbf{Variant II.}

$\begin{array}{rcll|l}
\displaystyle \lim_{h\to 0} \frac{\frac{1}{(2+h)^2}-\frac{1}{4}}{h}&=&\displaystyle \frac{\diff }{\diff x}\left(\frac{1}{x^2}\right)_{|x=2}\\
&=&\displaystyle \left(\frac{-2}{x^3}\right)_{|x=2}\\
&=&\displaystyle -\frac{1}{4}
\end{array}
$

}



\begin{problem}
\begin{problem}(Textbook, page 235, problems 33-38).
Find the horizontal and vertical asymptotes of each curve. Check your work by plotting the function using the internet.
\begin{multicols}{3}
\begin{enumerate}
\item $y=\frac{2x+1}{x-2}$.
\item $y=\frac{x^2+1}{2x^2-3x-2}$.
\item $y=\frac{2x^2+x-1}{x^2+x-2}$.
\item $y=\frac{1+x^4}{x^2-x^4}$.
\item $y=\frac{x^3-x}{x^2-6x+5}$.
\item $y=\frac{x-9}{\sqrt{4x^2+3x+2}}$.
\end{enumerate}
\end{multicols}
\end{problem}
\end{problem}
\solution{\ref{problemAsymptotesy=(2x/(sqrt(x^2+x+3)-3))}
\textbf{Vertical asymptotes.} A function $f(x)$ has a vertical asymptote at $x=a$ if $\lim\limits_{x\to a} f(x)=\pm \infty$. 

The function is algebraic, and therefore has a finite limit at every point it is defined (i.e., no asymptote). Therefore the function can have vertical asymptotes only for those $x$ for which $f(x)$ is not defined. The function is not defined for $\sqrt{x^2+x+3}-3=0$, which has two solutions, $x=2$ and $x=-3$. These are precisely the vertical asymptotes: indeed, 
\[
\lim\limits_{x\to 2^+} \frac{2x}{\sqrt{x^2+x+3}-3}=\infty \quad \quad \quad 
\lim\limits_{x\to 2^-} \frac{2x}{\sqrt{x^2+x+3}-3}=-\infty 
\]
and
\[
\lim\limits_{x\to -3^+} \frac{2x}{\sqrt{x^2+x+3}-3}=\infty \quad \quad \quad 
\lim\limits_{x\to -3^-} \frac{2x}{\sqrt{x^2+x+3}-3}=-\infty 
\]

\textbf{Horizontal asymptotes.} A function $f(x)$ has a horizontal asymptote if $\lim\limits_{x\to \pm\infty} f(x)$ exists. If that limit exists, and is some number, say, $N$, then $y=N$ is the equation of the corresponding asymptote.

Consider the limit $x\to -\infty$. We have that 
\[
\begin{array}{rcll|l}
\displaystyle \lim\limits_{x\to -\infty} \frac{2x}{ \sqrt{x^2+3x+3} - 3}&=&\displaystyle \lim\limits_{x\to - \infty} \frac{ 2}{ \frac{ \sqrt{ x^2 + x+3}}x-\frac3x}\\
&=&\displaystyle  \lim\limits_{x\to - \infty} \frac{2}{-\sqrt{\frac{ x^2 +3x+3}{x^2}}-\frac3x}  && \frac{1}{x} =- \sqrt{\frac{1}{x^2} } \text{ when } x<0\\
&=&\displaystyle \lim\limits_{x\to - \infty} \frac{2}{-\sqrt{1+ \frac{3}{x} + \frac{3}{x^2}}-\frac3x}\\
&=& \displaystyle \frac{\lim\limits_{x\to - \infty} 2}{-\sqrt{ \lim \limits_{x \to - \infty} 1+\lim\limits_{x\to - \infty} \frac{3}{x} + \lim\limits_{x \to - \infty} \frac{3}{x^2}}-\lim\limits_{x\to - \infty} \frac3x}\\
&=&\displaystyle \frac{2}{-\sqrt{1+0+0}-0}\\
&=&\displaystyle -2\quad . 
\end{array}
\]
Therefore $y=-2$ is a horizontal asymptote. 

The case $x\to \infty$, is handled similarly and yields that $y=2$ is a horizontal asymptote.

A computer generated graph confirms our computations.

\psset{xunit=0.2cm, yunit=0.2cm}
\begin{pspicture}(-15, -19.57133)(15,16.540354)
\tiny
\fcAxesStandard{-15}{-19.32133}{15}{16.190354}
\fcXTick{10}
\rput[t](10, -0.6){$10$}
%Function formula: \frac{2 x}{\sqrt{x^{2}+x+3}-3}
\psplot[linecolor=\fcColorGraph, plotpoints=1000]{2.344}{15}{x  2 mul    -3  3 x add   x  2 exp   add    0.5 exp   add   div  }
%Function formula: \frac{2 x}{\sqrt{x^{2}+x+3}-3}
\psplot[linecolor=\fcColorGraph, plotpoints=1000]{-2.6}{1.78}{x  2 mul    -3  3 x add   x  2 exp   add    0.5 exp   add   div  }
%Function formula: \frac{2 x}{\sqrt{x^{2}+x+3}-3}
\psplot[linecolor=\fcColorGraph, plotpoints=1000]{-15}{-3.4}{x  2 mul    -3  3 x add   x  2 exp   add    0.5 exp   add   div  }
\psline[linestyle=dotted](-3,-19.3)(-3,16.1)
\psline[linestyle=dotted](2,-19.3)(2,16.1)
\psline[linestyle=dashed, linecolor=blue](-15, 2)(15, 2)
\psline[linestyle=dashed, linecolor=blue](-15, -2)(15, -2)
\rput[b](-8, 2.6){$y=2$}
\rput[t](8, -2.6){$y=-2$}
\rput[bl](5,5){$y=\frac{2x}{\sqrt{x^2+x+3}-3}$}
\rput[l](2.6,-8){$x=2$}
\rput[r](-3.6,8){$x=-3$}
\end{pspicture}

}


\begin{problem}
Differentiate.

\begin{multicols}{2}
\begin{enumerate}
\item $\tan x$.
\answer{$\sec^2 x$}
\item $\cot x$.
\answer{$-\csc^2 x$}
\item $\sec x$.
\answer{$\sec x \tan x= \frac{\sin x}{\cos^2 x}$}
\item $\csc x$.
\answer{$-\csc x \cot x= -\frac{\cos x }{\sin^2x} $}
\item $\sec x\tan x$.
\answer{$\sec x \tan^2 x+\sec^3 x$}
\item $\sec x+\tan x$.
\answer{$\sec x(\tan x +\sec x) $}
\item $\sec^2 x$.
\answer{$2\tan x\sec^2 x$}
\item $\csc^2 x$.
\answer{$ -2\cot x\csc^2 x$}
\item $\frac{\sin x}{x}$.
\answer{$\frac{x \cos{}x- \sin{}x}{x^{2}}$}
\end{enumerate}

\end{multicols}
\end{problem}
\begin{problem}
Differentiate.

\begin{multicols}{2}
\begin{enumerate}
\item $\tan x$.
\answer{$\sec^2 x$}
\item $\cot x$.
\answer{$-\csc^2 x$}
\item $\sec x$.
\answer{$\sec x \tan x= \frac{\sin x}{\cos^2 x}$}
\item $\csc x$.
\answer{$-\csc x \cot x= -\frac{\cos x }{\sin^2x} $}
\item $\sec x\tan x$.
\answer{$\sec x \tan^2 x+\sec^3 x$}
\item $\sec x+\tan x$.
\answer{$\sec x(\tan x +\sec x) $}
\item $\sec^2 x$.
\answer{$2\tan x\sec^2 x$}
\item $\csc^2 x$.
\answer{$ -2\cot x\csc^2 x$}
\item $\frac{\sin x}{x}$.
\answer{$\frac{x \cos{}x- \sin{}x}{x^{2}}$}
\end{enumerate}

\end{multicols}
\end{problem}
\begin{problem}
(Textbook, page 136, 1-44). Compute the derivative.
\begin{multicols}{2}
\begin{enumerate}

\item $f(x)=\frac{x-3}{x+3}$.

\answer{$6 (3+x)^{-2} $}
\item $y=\frac{x^3}{1-x^2}$.

\answer{$ \frac{3 x^{2}- x^{4}}{(1- x^{2})^{2}}$}
\item $y=\frac{x+1}{x^3+x-2}$.

\answer{$\frac{-3-3 x^{2}-2 x^{3}}{(-2+x+x^{3})^{2}} $}
\item $y=\frac{v^3-2v\sqrt{v}}{v}$.

\answer{$2 v- v^{-\frac{1}{2}}$}
\item $y=\frac{t}{(t-1)^2}$.

\answer{$-\frac{t +1}{(t-1)^3} $}
\item $y=\frac{t^2+2}{t^4-3t^2+1}$.

\answer{$\frac{14 t-8 t^{3}-2 t^{5}}{(1-3 t^{2}+t^{4})^{2}} $}
\item $g(t)=\frac{t-\sqrt{t}}{t^{1/3}}$.

\answer{$-\frac{1}{6} t^{-\frac{5}{6}}+\frac{2}{3} t^{-\frac{1}{3}} $}
\item $y=a x^2+b x + c$.

\answer{$ b+2 a x$}
\item $y=A+\frac{B}x +\frac{C}{x^2}$.

\answer{$\frac{- B-2 C x^{-1}}{x^{2}}$}
\item $f(t)=\frac{2t}{2+\sqrt{t}}$.

\answer{$\frac{4+t^{\frac{1}{2}}}{(2+t^{\frac{1}{2}})^{2}} $}
\item $y=\frac{c x}{1+c x}$.

\answer{$ c (1+c x)^{-2}$}
\item $y=\sqrt[3]{t}(t^2+t+t^{-1}) $.

\answer{$-\frac{2}{3} t^{-\frac{5}{3}}+\frac{4}{3} t^{\frac{1}{3}}+\frac{7}{3} t^{\frac{4}{3}} $}
\item $y=\frac{u^6-2u^3+5}{u^2}$.

\answer{$-2-10 u^{-3}+4 u^{3} $}
\item $f(x)=\frac{x}{x+\frac{c}{x}}$.

\answer{$ \frac{2 x c}{(c+x^{2})^{2}}$}
\item $f(x)=\frac{a x+b}{c x+ d}$.

\answer{$\frac{a d- b c}{(d+c x)^{2}}$}
\end{enumerate}
\end{multicols}

\end{problem}
\solution{\ref{problemd/dt(t/(t-1)^2)}
This can be differentiated more efficiently using the chain rule, however let us show how the problem can be solved directly using the quotient rule.
\[
\begin{array}{rcl}
\displaystyle  \left(\frac{t}{(t-1)^2}\right)'&=&\displaystyle \frac{(t)' (t-1)^2-t\left((t-1)^2\right)' }{(t-1)^4}\\
&=&\displaystyle \frac{(t-1)^2 - t \left(t^2-2t+1\right)' }{(t-1)^4}\\
&=&\displaystyle \frac{(t-1)^2 - t \left(2t-2\right) }{(t-1)^4}\\
&=&\displaystyle \frac{\cancel{(t-1)} \left((t-1) - 2t \right)}{(t-1)^{\cancel{4} ~3}}\\
&=&\displaystyle \frac{-t -1}{(t-1)^3}\\
&=&\displaystyle-\frac{t+1}{(t-1)^3}
\end{array}
\]


}


\solution{\ref{problemd/dx((x+1)/(x^3+1))}

\[\begin{array}{rcl}
\displaystyle \frac{\diff }{\diff x}\left(\frac{x+1}{x^3+1}\right)&=&\displaystyle \frac{\diff }{\diff x}\left(\frac{\cancel{x+1}}{\cancel{(x+1)}(x^2-x+1)}\right)\\
&=&\displaystyle \frac{\diff }{\diff x}\left(\frac{1}{x^2-x+1}\right)\\
\multicolumn{3}{l}{\textbf{Variant I: use quotient rule.}}\\
&=&\displaystyle \frac{ \frac{\diff }{\diff x}(1)\cdot (x^2-x+1)-1\cdot\frac{\diff }{\diff x}\left(x^2-x+1\right)}{\left(x^2-x+1\right)^2}\\
&=&\displaystyle \frac{-2x+1}{\left(x^2-x+1\right)^2}\\
\multicolumn{3}{l}{\textbf{Variant I: use chain rule.}}\\
&=&\displaystyle \frac{\diff }{\diff x}\left(\left(x^2-x+1\right)^{-1}\right)\\
&=&\displaystyle -(x^2-x+1)^{-2}\frac{\diff}{\diff x}(x^2-x+1)\\
&=&\displaystyle -(x^2-x+1)^{-2}(2x-1)\\
&=&\displaystyle \frac{-2x+1}{\left(x^2-x+1\right)^2}.
\end{array}
\]
}


\begin{problem}
% begin homework chain-rule1
In each of the following cases find a simple function $u$ of $x$ such that the given function is a simple function of $u$.  
Use the Chain Rule to differentiate the given function with respect to $x$.   

\begin{enumerate}
\item   $y = \sqrt{1+x^2}$


\pointsii{3}  $y = (\cos x)^{1/2}$
\solution{%
\begin{align*}
\text{Let } \quad u & = \cos x. \\
\text{Then } \quad y & = u^{1/2}. \\
\text{Chain Rule: } \quad \frac{\diff y}{\diff x} & = \frac{\diff y}{\diff u}\frac{\diff u}{\diff x} \\
 & = \big(\frac{1}{2}u^{-1/2}\big) (-\sin x) \\
 & = -\frac{1}{2} \sin x (\cos x)^{-1/2}.
\end{align*}
}%

\item   $y = \sin^3 x$

\pointsii{3}  $y = (1+\cos x)^2$
\solution{%
\begin{align*}
\text{Let } \quad u & = 1+\cos x. \\
\text{Then } \quad y & = u^{2}. \\
\text{Chain Rule: } \quad \frac{\diff y}{\diff x} & = \frac{\diff y}{\diff u}\frac{\diff u}{\diff x} \\
 & = (2u) (-\sin x) \\
 & = -2 \sin x \cos x \\
 & = - \sin 2x. \quad \text{(This last step is optional.)}
\end{align*}
}%

\end{enumerate}
% end homework chain-rule1

\end{problem}
\solution{\ref{problemDifferentialtexDivsqrt(1+2divx^2)}
\[
\begin{array}{rclr|r}
\displaystyle\left(\frac{x }{\sqrt{1+\frac{2}{x^2}}}\right)'&=&\displaystyle\frac{\sqrt{1+\frac{2}{x^2}}- x\left(\sqrt{1+\frac{2}{x^2}}\right)'}{1+\frac{2}{x^2}} =\frac{\sqrt{1+\frac{2}{x^2}}-  x\frac{\frac12}{\sqrt{1 +\frac{ 2}{ x^2 }}}  \left(\frac{2}{x^2}\right)'}{1+\frac{2}{x^2}}\\
&=& \displaystyle\frac{\sqrt{1+\frac{2}{x^2}}+  \frac{2}{x^2\sqrt{1 +\frac{ 2}{ x^2 }}} }{1+\frac{2}{x^2}} = \frac{x^2\left(1+\frac{2}{x^2}\right)+  2 }{x^2\left(1+\frac{2}{x^2}\right)^{\frac32}}= \frac{x^2+4}{x^2\left(1+\frac{2}{x^2}\right)^{\frac32}}
\end{array}
\]
Please note that this problem can be solved also by applying the transformation 
\[
\displaystyle  \frac{x}{\sqrt{1+\frac{2}{x^2}}}= \frac{x}{\sqrt{\frac{x^2+2}{x^2}}} =\frac{x}{\frac{1}{\pm x}\sqrt{x^2+2}} = \frac{\pm x^2}{\sqrt{x^2+2}}
\]
before differentiating, however one must not forget the $\pm $ sign arising from $\sqrt{x^2}=\pm x$. Our original approach resulted in more algebra, but did not have the disadvantage of dealing with the $\pm$ sign.
}

\solution{ \ref{problemd/dx((cosx)^(1/2))}%
\begin{align*}
\text{Let } \quad u & = \cos x. \\
\text{Then } \quad y & = u^{\frac{1}{2}}. \\
\text{Chain Rule: } \quad \frac{\diff y}{\diff x} & = \frac{\diff y}{\diff u}\frac{\diff u}{\diff x} \\
 & = \left(\frac{1}{2}u^{-\frac{1}{2}}\right) (-\sin x) \\
 & = -\frac{1}{2} \sin x (\cos x)^{-\frac{1}{2}}.
\end{align*}
}%

\solution{\ref{problemd/dx((1+cosx)^2)} %
\begin{align*}
\text{Let } \quad u & = 1+\cos x. \\
\text{Then } \quad y & = u^{2}. \\
\text{Chain Rule: } \quad \frac{\diff y}{\diff x} & = \frac{\diff y}{\diff u}\frac{\diff u}{\diff x} \\
 & = (2u) (-\sin x) \\
 & = -2 \sin x (1+\cos x) \\
 & = -2\sin x -2 \sin x \cos x \\
 & = -2\sin x -\sin (2x). \quad \text{(This last step is optional.)}
\end{align*}
}%

\solution{\ref{problemd/dx(sin(sqrt(x)))} %
\begin{align*}
\text{Let } \quad u & = \sqrt{x}. \\
\text{Then } \quad y & = \sin u. \\
\text{Chain Rule: } \quad \frac{\diff y}{\diff x} & = \frac{\diff y}{\diff u}\frac{\diff u}{\diff x} \\
 & = (\cos u) \left(\frac{1}{2}u^{-\frac{1}{2}}\right) \\
 & = \frac{\cos\left(\sqrt{x}\right) }{2\sqrt{x}}.
\end{align*}
}%

\solution{\ref{problemd/dx(sqrt(sec(4x)))} %
\begin{align*}
\text{Chain Rule: } \quad \frac{\diff y}{\diff x} & = \left( \frac{1}{2}(\sec (4x))^{-\frac{1}{2}} \right) \frac{\diff}{\diff x}(\sec (4x)) \\
\text{Chain Rule: } \quad \frac{\diff y}{\diff x} & = \left( \frac{1}{2\sqrt{\sec (4x)}} \right) (\sec (4x) \tan (4x))\frac{\diff}{\diff x}(4x) \\
 & = \left( \frac{1}{2\sqrt{\sec (4x)}}\right) (\sec (4x) \tan (4x))(4) \\
 & =  \frac{2\sec (4x)\tan (4x)}{\sqrt{\sec (4x)}} \\
\intertext{There are many ways to simplify this answer, including both of the following.}
 & =  2\sqrt{\sec (4x)}\tan (4x). \\
 & =  2(\sec (4x))^{\frac{3}{2}}\sin (4x). 
\end{align*}
}%

\solution{\ref{problemd/dx(x^2tan(5x))} %
\begin{align*}
\text{Product Rule: } \quad \frac{\diff y}{\diff x} & = (x^2)\frac{\diff}{\diff x}(\tan (5x)) + (\tan (5x))\frac{\diff}{\diff x}(x^2) \\
\intertext{Use the Chain Rule to differentiate $\tan (5x)$ in the first term.}
\frac{\diff y}{\diff x} & = (x^2)(-5\sec^2 (5x) + (\tan (5x))(2x) \\
 & = 2x\tan (5x) - 5x^2\sec^2 (5x).
\end{align*}
}%


\solution{\ref{problemd/dx((1+sin(x^2))/(1+cos(x^2)))} %
\begin{align*}
\text{Quotient Rule: } \quad \frac{\diff y}{\diff x} & = \frac{\left(1+ \cos \left( x^2 \right)\right)\frac{\diff}{\diff y}(1+\sin \left(x^2\right) ) - (1+\sin \left(x^2\right))\frac{\diff}{\diff x}(1+\cos \left(x^2\right))}{(1+\cos \left(x^2\right))^2} \\
\intertext{By the Chain Rule, $\frac{\diff}{\diff x}(1+\cos \left(x^2\right)) = -2x\sin \left(x^2\right)$ and $\frac{\diff}{\diff x}(1+\sin \left(x^2\right)) = 2x\cos \left(x^2\right)$.}
\frac{\diff y}{\diff x} & = \frac{(1+\cos \left(x^2\right))(2x\cos \left(x^2\right)) - (1+\sin \left(x^2\right))(-2x\sin \left(x^2\right))}{(1+\cos \left(x^2\right))^2} \\
 & = \frac{2x\cos \left(x^2\right) + 2x\cos^2 \left(x^2\right) + 2x\sin \left(x^2\right) + 2x\sin^2 \left(x^2\right)}{(1+\cos \left(x^2\right))^2} \\
 & = \frac{2x(\cos^2 \left(x^2\right) + \sin^2 \left(x^2\right)) + 2x(\cos \left(x^2\right) + \sin \left(x^2\right))}{(1+\cos \left(x^2\right))^2} \\
\intertext{By the Pythagorean Identity, $\cos^2 \left(x^2\right) + \sin^2 \left(x^2\right) = 1$.}
\frac{\diff y}{\diff x} & = \frac{2x + 2x(\cos \left(x^2\right) + \sin \left(x^2\right))}{(1+\cos \left(x^2\right))^2} \\
 & = \frac{2x(1 + \cos \left(x^2\right) + \sin \left(x^2\right))}{(1+\cos \left(x^2\right))^2}.
\end{align*}
}%

\begin{problem}
(Textbook, page 162, problems 25-32) Use implicit differentiation to find an equation of the tangent line to the curve a the given point. The answer key has not been proofread, use with caution.
\begin{multicols}{3}
\begin{enumerate}
\item $y\sin (2x)=x\cos (2y) $, $\left(\frac{\pi}{2}, \frac{\pi}{4}\right)$. 
\answer{$y=\pi^{-1} x+\frac{1}{4} \pi-\frac{1}{4}$}
\item $ \sin (x+y)=2x-2y$, $(\pi,\pi)$ . 
\answer{$\frac{1}{3} x+\frac{2}{3} \pi $}
\item $x^2+x y+y^2=3 $, $(1,2)$ (ellipse). 
\answer{$y=-\frac{4}{5} x+\frac{14}{5} $}
\item $x^2+2x y-y^2+x=2 $, $(1,2)$ (hyperbola). 
\answer{$y= \frac{7}{2} x-\frac{3}{2}$}
\item $x^2+y^2=(2x^2+2y^2-x)^2 $, $(0,\frac{1}{2})$. 
\answer{$y= x+\frac{1}{2}$}
\item $x^{\frac{2}{3}}+y^{\frac{2}{3}}=4$, $(-3\sqrt{3},1)$. 
\answer{$y=\frac{1}{\sqrt{3}}x+4 $}
\item $2(x^2+y^2)^2 =25(x^2-y^2)$, $(3,1)$. 
\answer{$y= -\frac{9}{13} x+\frac{40}{13}$}
\item $y^2(y^2-4)=x^2(x^2-5) $, $(0,-2)$. 
\answer{$y=-2 $}
\end{enumerate}
\end{multicols}

\end{problem}
\solution{\ref{problemImplicitTangentysin(2x)=xcos(2y)point(pi/2,pi/4)}

\psset{xunit=0.5cm, yunit=0.5cm}
\begin{pspicture}(-5.8,-5.8)(5.8,5.8)
\fcAxesStandard{-5.5}{-5.5}{5.5}{5.5}
\fcLabels{5.5}{5.5}
\fcXTickWithLabel{1}{$1$}
\fcImplicitIId[linestyle=solid, linecolor=red, linewidth-=0.05, showGridImplicitIId=false]{-5}{-5}{1000}{1000}{0.01}{0.01}{2 x mul 180 mul 3.141592654 div sin y mul 2 y mul 180 mul 3.141592654 div cos x mul sub} 

\psline[linecolor=blue](-5,-2.5)(5,2.5)
\fcFullDot[linecolor=blue]{3.141592654 2 div}{3.141592654 4 div}
\end{pspicture}


First we verify that the point $\displaystyle (x,y)=\left(\frac{\pi}{2}, \frac{\pi}{4}\right)$ indeed satisfies the given equation:

\[
\begin{array}{rcll|l}
\displaystyle y \sin (2x)_{|x=\frac{\pi}{2}, y=\frac{\pi}{4}}= \frac{\pi}{4}\sin \pi &=& \displaystyle 0 && \text{left hand side}\\
\displaystyle x \cos (2y)_{|x=\frac{\pi}{2}, y=\frac{\pi}{4}}= \frac{\pi}{2}\cos \left(\frac{\pi}{2} \right)&=&\displaystyle 0 && \text{right hand side}\\
\end{array}
\]
so the two sides of the equation are equal (both to $0$) when $x=\frac{\pi}{2}$ and $y=\frac{\pi}{4}$.

Since we are looking an equation of the tangent line, we need to find  $\frac{\diff y}{\diff x}_{|x=\frac{\pi}{2}, y= \frac{\pi}{4}}$ - that is, the derivative of $y$ at the point $x=\frac{\pi}{2}$, $y= \frac{\pi}{4}$. To do so we use implicit differentiation.
\[
\begin{array}{rcll|l}
\displaystyle y \sin (2x)&=&\displaystyle x\cos (2y)&&\frac{\diff }{\diff x}\\
\displaystyle \frac{\diff y}{\diff x} \sin (2x) +y \frac{\diff }{\diff x}\left(\sin (2x)\right)&=&\displaystyle  \cos (2y)+x\frac{\diff }{\diff x}(\cos (2y))\\
\displaystyle \frac{\diff y}{\diff x}\sin (2x)+2y\cos (2x)&= & \displaystyle \cos (2y)-2x \sin (2y) \frac{\diff y}{\diff x}\\
\displaystyle \frac{\diff y}{\diff x}(\sin (2x)+2x\sin (2y))&=&\displaystyle \cos (2y)-2y\cos (2x)&& \text{Set }x=\frac{ \pi}{2}, y=\frac{\pi}{4}\\
\displaystyle \frac{\diff y}{\diff x}_{|x=\frac{\pi}{2}, y=\frac{\pi}{4}} \left(\sin \pi+\pi \sin \left(\frac{\pi}{2}\right)\right)&=& \displaystyle \cos \left(\frac{\pi}{2}\right)-\frac{\pi}{2}\cos \pi\\
\displaystyle \pi \frac{\diff y}{\diff x}_{|x=\frac{\pi}{2}, y=\frac{\pi}{4}} &=& -\frac{\pi}{2}\cos \pi\\
\displaystyle \frac{\diff y}{\diff x}_{|x=\frac{\pi}{2}, y=\frac{\pi}{4}}&=& \displaystyle \frac{1}{2}.
\end{array}
\]
Therefore the equation of the line through $x=\frac{\pi}{2}, y=\frac{\pi}{4}$ is 
\[
\begin{array}{rcl}
\displaystyle y-\frac{\pi}{4}&=&\displaystyle \frac{1}{2}\left( x-\frac{\pi}{2} \right)\\
y&=&\displaystyle \frac{1}{2} x .
\end{array}
\]
}

\begin{problem}
\begin{enumerate}
% Optimization
\item \label{problemponthyperbolax^2-4y^2closestTo1,1} What is the $x$-coordinate of the point on the hyperbola $x^2 - 4y^2 = 16$ that is closest to the point $(1, 0)$?

\psset{xunit=0.3cm, yunit=0.3cm}
\begin{pspicture}(-10.500000, -5)(10.500000,5)
\psframe*[linecolor=white](-10.500000,-5)(10.500000,5)
\tiny
\fcAxesStandard{-10.000000}{-4.5}{10.000000}{4.5} %Function formula: - (1/4 x^{2}-4)^{1/2}
\psplot[linecolor=\fcColorGraph, plotpoints=1000]{-10.000000}{-4.000000}{-4 x 2 exp 0.25 mul add 0.5 exp -1 mul }
%Function formula: (1/4 x^{2}-4)^{1/2}
\psplot[linecolor=\fcColorGraph, plotpoints=1000]{-10.000000}{-4.000000}{-4 x 2 exp 0.25 mul add 0.5 exp }
%Function formula: - (1/4 x^{2}-4)^{1/2}
\psplot[linecolor=\fcColorGraph, plotpoints=1000]{4.000000}{10.000000}{-4 x 2 exp 0.25 mul add 0.5 exp -1 mul }
%Function formula: (1/4 x^{2}-4)^{1/2}
\psplot[linecolor=\fcColorGraph, plotpoints=1000]{4.000000}{10.000000}{-4 x 2 exp 0.25 mul add 0.5 exp }
\fcFullDot{1}{0}
\fcFullDot{4}{0}
\pscircle[linestyle=dotted](1,0){0.9}
\end{pspicture}

\answer{$x = 4$}

\item What is the $x$-coordinate of the point on the ellipse $x^2+4y^2=16$ closest to the point $(1,0)$?

\psset{xunit=0.3cm, yunit=0.3cm}
\begin{pspicture}(-4.500000, -5)(4.500000,5)
\psframe*[linecolor=white](-4.500000,-5)(4.500000,5)
\tiny
\psline[linecolor=red!1](-10,0)(-9.99,0)
\fcAxesStandard{-4.000000}{-4.5}{4.000000}{4.5} %Function formula: - (-1/4 x^{2}+4)^{1/2}
\psplot[linecolor=\fcColorGraph, plotpoints=1000]{-4.000000}{4.000000}{4 x 2 exp -0.25 mul add 0.5 exp -1 mul }
%Function formula: (-1/4 x^{2}+4)^{1/2}
\psplot[linecolor=\fcColorGraph, plotpoints=1000]{-4.000000}{4.000000}{4 x 2 exp -0.25 mul add 0.5 exp }
\pscircle[linestyle=dotted](1,0){0.574456}
\fcFullDot{1}{0}
\fcFullDot{1.333333}{1.885618}
\fcFullDot{1.333333}{-1.885618}
\end{pspicture}
\answer{$x=\frac43$}
\item \label{problemMaxVolumeBoxFixedAreaDoubleBottomNoLid} You want to build a rectangular box with a square base out of sheet metal. You are going to use 2 pieces of sheet metal for the bottom of the box to reinforce it, and only a single piece of sheet metal for all of the sides and the top. If you want to use no more than $36$ sq. ft. of material, what is the largest possible volume you can enclose?

\answer{12 cubic feet.}
\end{enumerate}

\end{problem}
\solution{\ref{problemponthyperbolax^2-4y^2closestTo1,1}

The distance function between an arbitrary point $(x,y)$ and the point $(1,0)$ is $d=\sqrt{(x-1)^2+(y-0)^2}$. On the other hand, when the point $(x,y)$ lies on the hyperbola we have $y^2= \frac{x^2 -16 }{4 }$. In this way, the problem becomes that of minimizing the distance function

\[
dist(x)=\sqrt{(x-1)^2+y^2}=\sqrt{(x-1)^2+\frac{x^2-16}{4}} \quad .
\]
This is a standard optimization problem: we need to find the critical endpoints, i.e., the points where $dist'=0$. As the square root function is an increasing function, the function $\displaystyle \sqrt{(x-1)^2+\frac{x^2-16}{4}}$ achieves its minimum when the function 
\[
l=dist^2=(x-1)^2+\frac{x^2-16}{4}\quad 
\]
does. $l$ is a quadratic function of $x$ and we can directly determine its minimimum via elementary methods. Alternatively, we find the critical points of $l$:
\[
\begin{array}{rcl}
\displaystyle l'&=&\displaystyle 0\\
\displaystyle 2(x-1)+\frac{x}{2} &=&\displaystyle 0\\
\displaystyle \frac{5}{2}x-2&=&0\\
\displaystyle x&=&\displaystyle \frac{4}{5}\quad .
\end{array}
\]
On the other hand, $x^2=16+4y^2$ and therefore $|x|\geq \sqrt{16} = 4$. Therefore $x\in (-\infty, -4]\cup [4,\infty)$. As $x= \frac{4}{5 }$ is outside of the allowed range, it follows that our function either attains its minimum at one of the endpoints $\pm 4$ or the function has no minimum at all. It is clear however that as $x$ tends to  $\infty$, so does $dist$. Therefore $dist$ attains its minimum for $x=4$ or $-4$ and $y=\pm\sqrt{(\pm4)^2-16}=0$. Direct check shows that $dist_{|x=4} =\sqrt{(4-1)^2 +\frac{4^2- 16}{4 }}=3$ and $dist_{|x=-4}=\sqrt{(-4-1)^2+\frac{4^2-16}{4}}=5$  so our function $dist$ has a minimal value of $3$ achieved when $x=4$, which is our final answer. Notice that this answer can be immediately given without computation by looking at the figure drawn for \ref{problemponthyperbolax^2-4y^2closestTo1,1}. Indeed, it is clear that there are no points from the hyperbola lying inside the dotted circle centered at $(1,0)$. Therefore the point where this circle touches the hyperbola must have the shortest distance to the center of the circle.
}

\solution{\ref{problemMaxVolumeBoxFixedAreaDoubleBottomNoLid}
Let $B$ denote the area of the base of the box, equal to the area of the top. Let $W$ denote the area of the four walls of the box (the four walls are all equal because the base of the box is a square). Then the surface area $S$ of material used will be 
\[
S=\underbrace{2B}_{\text{two pieces for the bottom}}+\underbrace{4W}_{4 \text{ walls}} +\underbrace{B}_{\text{the box lid}}=3B+4W\quad.
\] 
Let $x$ denote the length of the side of the square base and let $y$ denote the height of the box.  Then  
\[
B=x^2
\]
and 
\[
W=xy\quad .
\]
As the surface area $S$ is fixed to be $36$ square feet, we have that
\[
S=3B+4W=36= 3x^2 + 4xy\quad .
\]
As $y$ is positive, the above formula shows that $3x^2\leq 36$ and so $x\leq \sqrt{12}$. Let us now express $y$ in terms of $x$:
\[
\begin{array}{rcl}
3x^2+4xy&=&36\\
4xy&=&36-x^2\\
y&=&\displaystyle\frac{36-x^2}{4x}\quad .
\end{array}
\]
The problem asks us to maximize the volume $V$ of the box. The volume of the box equals the area of the base times the height of the box: 
\[
V=B\cdot y=yx^2 = \frac{(36-3x^2)}{4x}x^2=\frac{36x-3x^3}{4}\quad .
\]
As $x$ is non-negative, it follows that the domain for $x$ is:
\[
x\in [0, \sqrt{12}]\quad .
\]
To maximize the volume we find the critical points, i.e., the values of $x$ for which  $V'$ vanishes:

\[\begin{array}{rcl}
0&=&V' = \displaystyle\left(\frac{36x-3x^3}{4}\right)'\\
0&=&\displaystyle \frac{36- 9x^2}{4}\\
9x^2&=&36\\
x^2&=&4\\
x&=&\pm 2
\end{array}
\]
As $x$ measures length, $x=-2$ is not possible (outside of the domain for $x$). Therefore the only critical point is $x=2$. Direct check shows that at the endpoints $x=0$ and $x=\sqrt{12}$, we have that $V=0$. Therefore the maximal volume is achieved when $x=2$:
 
\[
V_{max}=V_{|x=2}= \frac{36(2)-3(2)^3}{4} =12\quad .
\]
 
}


\begin{problem}

Find the
\begin{multicols}{2}
\begin{itemize}
\item the implied domain of $f$.
\item $x$ and $y$ intercepts of $f$.
\item horizontal and vertical asymptotes.
\item intervals of increase and decrease
\item local and global minima, maxima,
\item intervals of concavity
\item points of inflection
\end{itemize}
\end{multicols}
Label all relevant points on the graph. Show all of your computations.
\begin{enumerate}
\item $\displaystyle f(x)=\frac{x+\frac 1 2}{x^{2}+x+1}$
\psset{xunit=1cm, yunit=1cm}
\begin{pspicture}(-5, -5)(5,5)
\psframe*[linecolor=white](-5,-5)(5,5)
\tiny
\psaxes[ticks=none, labels=none]{<->}(0,0)(-5,-0.5)(5,1.5)
\fcLabels{5}{1.5}
%Function formula: \frac{x+1/2}{x^{2}+x+1}
\psplot[linecolor=\fcColorGraph, plotpoints=1000]{-5}{5}{0.5 x add 1 x add x 2 exp add div }
\end{pspicture}

\answer{
\begin{tabular}{l}
$y$-intercept: $\frac12$. $x$-intercept: $-\frac12$\\
Horizontal asymptote: $y=0$, vertical: none \\
local and global min at $x=\frac{ -1-\sqrt{3}}{2}$, local and global max at $x=\frac{ -1+\sqrt{3}}{2}$\\
Intervals of decrease: $ \left(-\infty, \frac{-1 -\sqrt{3} }{2}\right)\cup \left(\frac{-1 +\sqrt{3} }{2}, \infty\right) $, intervals of decrease $\left( \frac{ -1-\sqrt{3}}{2}, \frac{-1+ \sqrt{3}}{2}\right)$ \\
Concave down on $(-\infty, -2)\cup \left(-\frac12, 1\right)$, concave up on $\left(-2, -\frac12\right)\cup (1,\infty)$\\
Inflection points at: $x=-2$, $x= -\frac12$, $x=1 $ \\
\end{tabular}
}

\item \label{problemSketchCurve(2x^2-5x+9/2)/(x^2-3 x+3)} $\displaystyle f(x)=\frac{2 x^{2}-5 x+\frac{9}{2}}{x^{2}-3 x+3}$
\psset{xunit=1cm, yunit=1cm}
\begin{pspicture}(-5, -5)(5,5)
\psframe*[linecolor=white](-5,-5)(5,5)
\tiny
\psaxes[ticks=none, labels=none]{<->}(0,0) (-5,-0.5) (5, 3.5)
\fcLabels{5}{3.5}
%Function formula: \frac{2 x^{2}-5 x+9/2}{x^{2}-3 x+3}
\psplot[linecolor=\fcColorGraph, plotpoints=1000]{-5}{5 } {4.5 x -5 mul add x 2 exp 2 mul add 3 x -3 mul add x 2 exp add div }
\end{pspicture}

\answer{
\begin{tabular}{l}
$y$-intercept: $\frac32$\\
horizontal asymptote: $y=2$, vertical: none\\
increasing on
$\left(\frac{3-\sqrt{3}}2, \frac{3+ \sqrt{3}}2 \right) $, decreasing on $\left(-\infty, \frac{3-\sqrt{3}}2\right)\cup \left(\frac{3+\sqrt{3}}2, \infty\right) $\\
local and global min at $x=\frac{3-\sqrt{3}}2$, local and global max at $x=\frac{3+\sqrt{3}}2$\\
concave up on $\left(0, \frac32\right)cup \left(3, \infty \right)$, concave down $\left(-\infty, 0\right)\cup \left(\frac32, 3\right)$\\
inflection points at $x=0,x=\frac32, x=3$
\end{tabular}
}
\item $\displaystyle f(x)=\frac{2 \sqrt{- x^{2}+1}+ 1} {\sqrt{- x^{2}+1}+1}$,  $f(x)=\frac{1}{\sqrt{- x^{2} +1}+1}$
\psset{xunit=1cm, yunit=1cm}
\begin{pspicture}(-1, -5)(1,5)
\psframe*[linecolor=white](-1,-5)(1,5)
\tiny
\psaxes[ticks=none, labels=none]{<->}(0,0)(-1,-0.5)(1,2.5)
\fcLabels{1}{2.5}
%Function formula: \frac{2 (- x^{2}+1)^{1/2}+1}{(- x^{2}+1)^{1/2}+1}
\rput(1,3){}
\psplot[linecolor=brown, plotpoints=1000]{-1}{1}{1 1 x 2 exp -1 mul add 0.5 exp 2 mul add 1 1 x 2 exp -1 mul add 0.5 exp add div }
%Function formula: \frac{1}{(- x^{2}+1)^{1/2}+1}
\rput(1,3){}
\psplot[linecolor=\fcColorGraph, plotpoints=1000]{-1}{1}{1 1 1 x 2 exp -1 mul add 0.5 exp add div }
\end{pspicture}
The two functions are plotted simultaneously in the $x,y$-plane. Indicate which part of the graph is the graph of which function.

\answer{
\begin{tabular}{l}
For $f(x)=\frac{2 \sqrt{- x^{2}+1}+1}{ \sqrt{- x^{2}+1}+1}$: \\
$y$-intercept: $x=\frac{3}2$, no $x$ intercept\\
no asymptotes\\
increasing on $[-1, 0]$, decreasing on $[0, 1]$ \\
global and local max at $x=0$, global and local min at $x=\pm 1$.\\
concave down on $[-1,1]$\\
no inflection points
\end{tabular}
}
\answer{
\begin{tabular}{l}
For $f(x)=\frac{1}{\sqrt{- x^{2}+1}+1}$: \\
$y$-intercept: $x=\frac{1}2$, no $x$ intercept\\
no asymptotes\\
decreasing on $[-1, 0]$, increasing on $[0, 1]$ \\
global and local min at $x=0$, global and local max at $x=\pm 1$.\\
concave up on $[-1,1]$\\
no inflection points
\end{tabular}
}
\item $\displaystyle f(x)=\frac{e^x+e^{-x}}{e^x-e^{-x}}$
\psset{xunit=0.5cm, yunit=0.5cm}
\begin{pspicture}(-4, -5)(4,5)
\psframe*[linecolor=white](-4,-5)(4,5)
\tiny
\psaxes[ticks=none, labels=none]{<->}(0,0)(-4,-4.5)(4,4.5)
\fcLabels{4}{5}
%Function formula: \frac{e^{- x}+e^{x}}{- e^{- x}+e^{x}}
\psplot[linecolor=\fcColorGraph, plotpoints=1000]{0.2}{4}{2.718281828 x exp 2.718281828 x -1 mul exp add 2.718281828 x exp 2.718281828 x -1 mul exp -1 mul add div }
%Function formula: \frac{e^{- x}+e^{x}}{- e^{- x}+e^{x}}
\psplot[linecolor=\fcColorGraph, plotpoints=1000]{-4}{-0.2}{2.718281828 x exp 2.718281828 x -1 mul exp add 2.718281828 x exp 2.718281828 x -1 mul exp -1 mul add div }
\end{pspicture}
\item $\displaystyle f(x)=\frac{- e^{- x}+e^{x}}{e^{- x}+e^{x}}$
\psset{xunit=1cm, yunit=1cm}
\begin{pspicture}(-4, -5)(4,5)
\psframe*[linecolor=white](-4,-5)(4,5)
\tiny
\psaxes[ticks=none, labels=none]{<->}(0,0)(-4,-1.1)(4,1.1)
\fcLabels{4}{1.1}
%Function formula: \frac{- e^{- x}+e^{x}}{e^{- x}+e^{x}}
\psplot[linecolor=\fcColorGraph, plotpoints=1000]{-4}{4}{2.718281828 x exp 2.718281828 x -1 mul exp -1 mul add 2.718281828 x exp 2.718281828 x -1 mul exp add div }
\end{pspicture}
\item $\displaystyle f(x)=\ln{}\left(\frac{{{x}}+1}{- {{x}}+1}\right)$
\psset{xunit=1cm, yunit=1cm}
\begin{pspicture}(-0.9, -5)(1,5)
\psframe*[linecolor=white](-0.9,-5)(1,5)
\tiny
\psaxes[ticks=none, labels=none]{<->}(0,0)(-1.3,-4)(1.3,4)
\fcLabels{1.3}{4}
%Function formula: \log{}(\frac{x+1}{- x+1})
\psplot[linecolor=\fcColorGraph, plotpoints=1000]{-0.94}{0.94}{1 x add 1 x -1 mul add div ln }

\end{pspicture}
\item $f(x)=\frac{x^{2}+3 x+1}{x^{2}+2 x}$
\psset{xunit=0.7cm, yunit=0.7cm}
\begin{pspicture}(-5, -5)(5,5)
\psframe*[linecolor=white](-5,-5)(5,5)
\tiny
\psaxes[ticks=none, labels=none]{<->}(0,0)(-5,-4.5)(5,4.5)
\fcLabels{5}{5}
%Function formula: \frac{x^{2}+3 x+1}{x^{2}+2 x}
\psplot[linecolor=\fcColorGraph, plotpoints=1000]{0.1}{5}{1 x 3 mul add x 2 exp add x 2 mul x 2 exp add div }
%Function formula: \frac{x^{2}+3 x+1}{x^{2}+2 x}
\psplot[linecolor=\fcColorGraph, plotpoints=1000]{-1.9}{-0.1}{1 x 3 mul add x 2 exp add x 2 mul x 2 exp add div }
%Function formula: \frac{x^{2}+3 x+1}{x^{2}+2 x}
\psplot[linecolor=\fcColorGraph, plotpoints=1000]{-5}{-2.1}{1 x 3 mul add x 2 exp add x 2 mul x 2 exp add div }

\end{pspicture}

\answer{
\begin{tabular}{l}
$y$-intercept: none, $x$-intercepts: $\frac{-3\mp\sqrt{5}}2$ \\
horizontal asymptote: $y=1$, vertical: $x=-2$ and $x=0$\\
always decreasing\\
no local/global minima/maxima\\
concave down on $\left(-\infty,-2\right)cup \left(-1,0 \right)$, concave up on $\left(-2, -1\right)\cup \left(0, \infty\right)$\\
inflection point at $x=-1$
\end{tabular}
}
\item \item $\displaystyle f(x)=\frac{x+1}{x^2+2x+4}$
\psset{xunit=2cm, yunit=2cm}
\begin{pspicture}(-4.500000, -5)(4.500000,5)
\psframe*[linecolor=white](-4.500000,-1)(4.500000,1)
\tiny
\fcAxesStandard{-5.000000}{-1}{3.000000}{1} %Function formula: \frac{x+1}{x^{2}+2 x+4}
\psplot[linecolor=\fcColorGraph, plotpoints=1000]{-5.000000}{3.000000}{1 x add 4 x 2 mul add x 2 exp add div }
\end{pspicture}

\answer{
\begin{tabular}{l}
$y$-intercept: $\frac14$, $x$-intercept: $-1$\\
horizontal asymptote: $y=0$, vertical: none\\
increasing on
$\left(-1-\sqrt{3}, -1+\sqrt{3}  \right) $, decreasing on $\left(-\infty, -1-\sqrt{3}\right)\cup \left(-1+\sqrt{3}, \infty\right) $\\
local and global min at $x=-1-\sqrt{3}$, local and global max at $x=-1+\sqrt{3}$\\
concave up on $\left(-4, -1\right)cup \left(2, \infty \right)$, concave down $\left(-\infty, -4\right)\cup \left(-1, 2\right)$\\
inflection points at $x=-4,x=-1, x=2$
\end{tabular}
}

\end{enumerate}

\end{problem}
\solution{\ref{problemSketchCurve(2x^2-5x+9/2)/(x^2-3 x+3)}

\textbf{Domain.} We have that $f$ is not defined only when we have division by zero, i.e.,  if $x^2-3x+3$ equals zero. However, the roots of $x^{2}-3x+3$ are not real numbers: they are $\frac{3\pm \sqrt{3^2-4\cdot 3 }}{2}= \frac{3\pm \sqrt{-3}}{2}$, and therefore $x^2-3x+3$ can never equal zero. Alternatively, completing the square shows that the denominator is always positive:
\[
x^2-3x+3=x^2-2\cdot \frac{3}{2} x+\frac{9}{4}-\frac{9}{4}+3=\left(x-\frac{3}{2}\right)^2+\frac{3}{4} >0 
\]
Therefore the domain of $f$ is all real numbers.

\textbf{$x$, $y$-intercepts.}  The $y$-intercept of $f$ equals by definition $\displaystyle f(0)= \frac{ 2\cdot 0^2-5\cdot 0+ \frac{9}{2}}{0^2-3\cdot 0 + 3}=\frac{\frac{9}{2}}{3}= \frac{3}{2}$. The $x$ intercept of $f$ is those values of $x$ for which $f(x)=0$. The graph of $f$ shows no such $x$, and that is confirmed by solving the equation $f(x)=0$:

\[
\begin{array}{rcll|l}
f(x)&=&0\\
\displaystyle \frac{2x^2-5x+\frac{9}{2}}{x^2-3 x+3}&=&0&&\text{Mult. by }x^2-3 x+3\\
\displaystyle 2x^2-5x+\frac{9}{2}&=&0\\
\displaystyle x_1, x_2&=&\displaystyle \frac{5 \pm \sqrt{25- 4\cdot 2\cdot \frac{9}{2}}}{4}=\frac{5\pm \sqrt{-9}}{4}\quad ,
\end{array}
\]
so there are no real solutions (the number $\sqrt{-9}$ is not real).

\textbf{Asymptotes.} Since $f$ is defined for all real numbers, its graph has no vertical asymptotes. To find the horizontal asymptote(s), we need to compute the limits $\lim\limits_{x\to \infty } f(x)$ and $\lim\limits_{x\to -\infty} f (x)$. The two limits are equal, as the direct computation below shows:
\[
\begin{array}{rcll|l}
\displaystyle \lim_{x\to \pm\infty} \frac{2x^2-5x+\frac{9}{2}}{x^2-3 x+3}&=& \displaystyle  \lim_{x\to \pm\infty}\frac{\left(2x^2-5x+ \frac{9}{2}\right)\frac{1}{x^2}}{\left(x^2-3 x+3\right)\frac{1}{x^2}} &&\begin{array}{l}\text{Divide by leading}\\ \text{monomial in denominator}\end{array}\\
&=&\displaystyle\lim_{x\to \pm \infty}\frac{2-\frac{5}{x} +\frac{9}{2x^2}}{1-\frac{3}{x}+\frac{3}{x^2}}\\
&=&\displaystyle \frac{2-0+0}{1-0+0}\\
&=& 2
\end{array}
\]
Therefore the graph of $f(x)$ has a single horizontal asymptote at $y=2$.

\textbf{Intervals of increase and decrease.}
The intervals of increase and decrease of $f$ are governed by the sign of $f'$. We compute:

\[
\begin{array}{rcl}
f'(x)&=&\displaystyle \left(\frac{2x^2-5x+\frac{9}{2} }{x^2- 3 x+3} \right)' \\
&=&\displaystyle \frac{\left(2x^2-5x+\frac{9}{2}\right)'\left(x^2- 3 x+3\right)-\left(2x^2-5x+\frac{9}{2}\right)\left(x^2- 3 x+3\right)' }{ \left(x^2- 3 x+3\right)^2}\\
&=&\displaystyle \frac{- x^{2}+3 x-\frac{3}{2} }{ \left(x^2- 3 x+3\right)^2}
\end{array}
\]
As the denominator is a square, the sign of $f'$ is governed by the sign of $- x^{2}+3 x-\frac{3}{2}$. To find where $- x^{2}+3 x-\frac{3}{2}$ changes sign, we compute the zeroes of this expression:

\[
\begin{array}{rcll|l}
\displaystyle - x^{2}+3 x-\frac{3}{2}&=&0&& \text{Mult. by }-2\\
\displaystyle  2x^{2}-6 x+3&=&0\\
x_1, x_2&=&\displaystyle \frac{ 6\pm \sqrt{36-24 }}{4}=\frac{6\pm \sqrt{12}}{4}\\
x_1, x_2&=&\displaystyle \frac{3\pm \sqrt{3}}{2} 
\end{array}
\]
Therefore the quadratic $- x^{2}+3 x-\frac{3}{2}$ factors as 
\begin{equation}
\label{eq1problemSketch(2x^2-5x+9/2)/(x^2-3 x+3)}
-(x-x_1)(x-x_2)=-\left(x-\left(\frac{3- \sqrt{3}}{2} \right)\right)\left(x-\left(\frac{3+ \sqrt{3}}{2}\right)\right)
\end{equation} 

The points $x_1, x_2$ split the real line into three intervals: $\left(-\infty, \frac{3- \sqrt{3}}{2}\right)$, $\left(\frac{3- \sqrt{3}}{2}, \frac{3+ \sqrt{3}}{2} \right)$ and $\left(\frac{3+ \sqrt{3}}{2}, \infty \right)$, and each of the factors of \eqref{eq1problemSketch(2x^2-5x+9/2)/(x^2-3 x+3)} has constant sign inside each of the intervals. If we choose $x$ to be a very negative number, it follows that $-(x-x_1)(x-x_2)$ is a negative, and therefore $ f'(x)$ is negative for $x\in(-\infty, \frac{3- \sqrt{3}}{2})$. For $x\in (\frac{3- \sqrt{3}}{2}, \frac{3+ \sqrt{3}}{2})$, exactly one factor of $f'$ changes sign and therefore $f'(x)$ is positive in that interval; finally only one factor of $f'(x)$ changes sign in the last interval so $f'(x)$ is negative on $(\frac{3+ \sqrt{3}}{2}, \infty )$.

Our computations can be summarized in the following table. 

\begin{tabular}{|lll|}\hline
Interval & $f'(x)$ & $f(x)$   \\\hline
$\left(-\infty, \frac{3- \sqrt{3}}{2}\right)$ & $-$& $\searrow $ \\\hline
$\left(\frac{3- \sqrt{3}}{2}, \frac{3+ \sqrt{3}}{2} \right)$ &$+$&$\nearrow$\\\hline
$\left( \frac{3+ \sqrt{3}}{2}, \infty\right)$&$-$&$\searrow$ \\\hline
\end{tabular}

\textbf{Local and global minima and maxima. } The table above shows that $f(x)$ changes from decreasing to increasing at $x=x_1=\frac{3- \sqrt{3}}{2}$ and therefore $f$ has a local minimum at that point. The table also shows that $f(x)$ changes from increasing to decreasing at $ x=x_2=\frac{3+ \sqrt{3}}{2}$ and therefore $f$ has a local maximum at that point. The so found local maximum and local minimum turn out to be global: there are two things to consider here. First, no other finite point is critical and thus cannot be maximum or minimum - however this leaves out the possibility of a maximum/minimum ``at infinity''. This possibility can be quickly ruled out by looking at the graph of $f$. To do so via algebra, compute first $f(x_1)$ and $f(x_2)$:

\[
\begin{array}{rcl}
\displaystyle f(x_1)= f\left(\frac{3- \sqrt{3}}{2} \right)&=& \displaystyle \frac{2\left(\frac{3- \sqrt{3}}{2} \right)^2-5\left(\frac{3- \sqrt{3}}{2} \right)+\frac{9}{2} }{\left(\frac{3- \sqrt{3}}{2} \right)^2- 3 \left(\frac{3- \sqrt{3}}{2} \right)+3}=2-\frac{\sqrt{3}}{3} \\

\displaystyle f(x_2)= f\left(\frac{3+ \sqrt{3}}{2} \right)&=& \displaystyle \frac{2\left(\frac{3+ \sqrt{3}}{2} \right)^2-5\left(\frac{3+ \sqrt{3}}{2} \right)+\frac{9}{2} }{\left(\frac{3+ \sqrt{3}}{2} \right)^2- 3 \left(\frac{3+ \sqrt{3}}{2} \right)+3}=2+\frac{\sqrt{3}}{3}\quad . 
\end{array}
\]
On the other hand, while computing the horizontal asymptotes, we established that $\lim\limits_{x\to\pm \infty}f(x)=2$. This implies that all $x$ sufficiently far away from $x=0$, we have that $f(x)$ is close to $2 $. Therefore $f(x)$ is larger than $f(x_1)$ and smaller than $f(x_2)$ for all sufficiently far away from $x=0$. This rules out the possibility for a maximum or a minimum ``at infinity'', as claimed above.

\textbf{Intervals of concavity. } 
The intervals of concavity of $f$ are governed by the sign of $f''$. The second derivative of $f$ is:
\[
\begin{array}{rcll|@{}l}
f''(x)&=&\displaystyle (f'(x))'= \left( \frac{- x^{2}+3 x-\frac{3}{2} }{ \left(x^2- 3 x+3\right)^2 } \right)'\\
&=&\displaystyle \left(- x^{2}+3 x-\frac{3}{2} \right)' \left(\frac{1 }{\left(x^2- 3 x+3\right)^2}\right)+ \left(- x^{2}+3 x-\frac{3}{2} \right)\left(\frac{1}{\left(x^2- 3 x+3\right)^2}\right)' &&\begin{array}{@{}l}\text{second differentiation:}\\\text{chain rule }\end{array}\\
&=&\displaystyle (-2x+3)\left(\frac{1}{\left(x^2- 3 x+3\right)^2} \right)+\left(- x^{2}+3 x-\frac{3}{2} \right)(-2)\frac{\left(x^2- 3 x+3\right)'}{\left(x^2- 3 x+3\right)^{3}}\\
&=&\displaystyle (-2x+3)\left(\frac{1}{\left(x^2- 3 x+3\right)^2}\right) +\left(2x^{2}-6 x+3 \right) \frac{(2x-3)}{\left(x^2- 3 x+3\right)^{3}}&&\text{factor out }\frac{(2x-3)}{\left(x^2- 3 x+3\right)^2}\\
&=&\displaystyle \frac{(2x-3)}{\left(x^2- 3 x+3\right)^2}\left(-1+\frac{(2x^{2}-6 x+3)}{\left(x^2- 3 x+3\right)}\right)\\
&=&\displaystyle \frac{(2x-3)}{\left(x^2- 3 x+3\right)^2}\left(\frac{-\left(x^2- 3 x+3\right)+(2x^{2}-6 x+3)}{\left(x^2- 3 x+3\right)} \right)\\
&=&\displaystyle \frac{(2x-3)(x^{2}-3 x )}{\left(x^2- 3 x+3\right)^3}\\
&=&\displaystyle \frac{(2x-3)x(x-3)}{\left(x^2- 3 x+3\right)^3}
\end{array}
\]
When computing the domain of $f$, we established that the denominator of the above expression is always positive. Therefore $f''(x)$ changes sign when the terms in the numerator change sign, namely, at $x=0$, $x=\frac{3}{2}$ and $x=3$. 

Our computations can be summarized in the following table. In the table, we use the $\cup$ symbol to denote that the function is concave up in the indicated interval, and $\cap$ to denote that the function is concave down.

\begin{tabular}{|lll|}\hline
Interval & $f''(x)$ & $f(x)$   \\\hline
$(-\infty, 0)$ & $-$& $\cap$ \\\hline
$(0, \frac{3}{2})$ &$+$&$\cup$\\\hline
$(\frac{3}{2}, 3)$&$-$&$\cap$ \\\hline
$(3, \infty)$&$+$&$\cup$ \\\hline
\end{tabular}

\textbf{Points of inflection.} The preceding table shows that $f''(x)$ changes sign at $0, \frac{3}{2}, 3$ and therefore the points of inflection are located at $x=0, x=\frac{3}{2}$ and $x=3$, i.e., the points of inflection are $\left(0, f(0)\right)= \left(0, \frac{3}{2} \right) $, $\left(\frac {3}{2}, f\left(\frac{3}{2}\right)\right) =\left(\frac{3}{2}, 2\right)$, $\left(3, f(3)\right)=\left(3, \frac{5}{2}\right)$.

We can command our graphing device to use the so computed information to label the graph of the function. Finally, we can confirm visually that our function does indeed behave in accordance with our computations.

\psset{xunit=0.6cm, yunit=0.6cm}
\begin{pspicture}(-5, -5)(5,5)
\psframe*[linecolor=white](-5,-5)(5,5)
\tiny
\psaxes[ticks=none, labels=none]{<->}(0,0) (-5,-0.5) (5, 3.5)
\fcLabels{5}{3.5}
%Function formula: \frac{2 x^{2}-5 x+9/2}{x^{2}-3 x+3}
\psplot[linecolor=\fcColorGraph, plotpoints=1000]{-5}{5 } {4.5 x -5 mul add x 2 exp 2 mul add 3 x -3 mul add x 2 exp add div }
\fcFullDot[linecolor=green]{0}{3 2 div}
\rput[r](-2, 0.2){infl.: $\left(0, \frac{3}{2}\right)$}
\psline[linestyle=dotted, arrows=->](-2, 0.2)(-0.05, 1.45)
\fcFullDot[linecolor=green]{3 2 div }{2}
\rput[r](1.2, 0.2){infl.: $\left(\frac{3}{2},2 \right)$}
\psline[linestyle=dotted, arrows=->](1.2, 0.2)(1.45, 1.95)
\fcFullDot[linecolor=green]{3 }{5 2 div}
\rput[l](2, 0.2){infl.: $\left(3, \frac{5}{2}\right)$}
\psline[linestyle=dotted, arrows=->](2, 0.2)(2.95, 2.45)
\fcFullDot[linecolor=blue]{3 3 sqrt add 2 div}{2 3 sqrt 3 div add}
\rput(2, 3){$\left(\frac{3+\sqrt{3}}{2}, 2+\frac{\sqrt{3}}{3} \right)$}
\fcFullDot[linecolor=blue]{3 3 sqrt sub 2 div}{2 3 sqrt 3 div sub}
\rput[r](-2, 3){$\left(\frac{3-\sqrt{3}}{2}, 2-\frac{\sqrt{3}}{3} \right)$}
\psline[linestyle=dotted, arrows=->](-2, 3)(! 3 3 sqrt sub 2 div 0.05 add 2 3 sqrt 3 div sub 0.05 add)
\end{pspicture}

}

\solution{\ref{problemSketch(x+1)/(x^2+2x+4)} 

\textbf{This problem is very similar to Problem \ref{problemSketchCurve(2x^2-5x+9/2)/(x^2-3 x+3)}. We recommend to the student to solve the problem first ``with closed textbook'' and only then to compare with the present solution.}

\textbf{Domain.} As $f$ is a quotient of two polynomials (rational function), its implied domain is all $x$ except those for which we get division by zero for $f$. Consequently the domain of $f$ is all $x$ for which $x ^2+2x+4=0$. However, the polynomial $x^2+2x+4$ has no real roots - its roots are $\displaystyle \frac{-2\pm \sqrt{4-16} }{2}=-1\pm \sqrt{-3}$, and therefore the domain of $f$ is all real numbers. Alternatively, we can complete the square: $x^2+2x+4=(x+1)^2+3$ and so $x^2+2x+4$ is positive for all values of $x$. 

\textbf{$x$, $y$-intercepts.} The $y$-intercept of $f$ equals by definition $\displaystyle f(0)= \frac{ 0+ 1}{0^2+2\cdot 0 + 4}=\frac{1}{4}$. The $x$ intercept of $f$ is those values of $x$ for which $f(x)=0$. We compute

\[
\begin{array}{rcl}
\displaystyle f(x)&=&0\\
\displaystyle \frac{x+1}{x^2+2x+4}&=&0\\
x+1&=&0\\
x&=&-1\quad ,
\end{array}
\]
and the $x$-intercept of $f$ is $x=-1$. 

\textbf{Asymptotes.} The line $x=a$ is a vertical asymptote when $\lim\limits_{x\to a^{\pm}}f(x)=\pm \infty$; as $f$ is defined for all real numbers, this implies that there are no vertical asymptotes. 
 
The line $y=L$ is a horizontal asymptote if $\lim\limits_{x\to\pm \infty}f(x)$ exists and equals $L$. We compute:
\[
\lim\limits_{x\to \infty} f(x)=\lim\limits_{x\to \infty} \frac{(x+1)\frac{1}{x^2}}{ (x^2+ 2x +4)\frac{1}{x^2}} = \lim\limits_{x\to \infty}\frac{\frac{1}{x}+\frac{1}{x^2}}{ 1+ \frac{2}{x} +\frac{4}{x^2}}=\frac{0+0}{1+0+0}=0
\]
Therefore $y=0$ is a horizontal asymptote for $f$. An analogous computation shows that $\lim\limits_{x\to\pm \infty}f(x)=0$ and so $y=0$ is the only horizontal asymptote of $f$.

\textbf{Intervals of increase and decrease.} 
The intervals of increase and decrease of $f$ are governed by the sign of $f'$. We compute:
\[
\begin{array}{rcll|l}
f'(x)&=&\displaystyle \left(\frac{x+1}{x^2+2x+4}\right)' &&\text{qutotient rule}\\
&=&\displaystyle \frac{(x+1)'\left(x^2+ 2x+4\right)- (x+1)\left( x^2 +2 x+4\right)'}{\left(x^2+2x+4 \right)^2}\\
&=&\displaystyle\frac{ x^2+2x+4-(x+1)(2x+2)}{\left(x^2+2x+4 \right)^2}\\
&=&\displaystyle \frac{x^2+2x+4-\left( 2x^2+ 4x+ 2 \right)}{ \left( x^2 +2x+4 \right)^2}\\
&=&\displaystyle \frac{-x^2-2x+2}{\left(x^2+2x+4 \right)^2}
\end{array}
\] 
As $x^2+2x+4$ is positive, the sign of $f'$ is governed by the sign of $-x^2+2x+2$. To find out where $-x^2+2x+2$ changes sign, we compute the zeroes of this expression:
\[\begin{array}{rcll|l}
-x^2-2x+2&=&0\\
x^2+2x-2&=&0 &&\text{use the quadratic formula}\\
x_1, x_2&=& -1\pm \sqrt{3}\quad .
\end{array}
\]
Therefore the quadratic $-x^2+2x+2$ factors as 
\begin{equation}
\label{eq1problemSketch(x+1)/(x^2+2x+4)}
-(x-x_1)(x-x_2)=-\left(x-\left(-1-\sqrt{3}\right)\right)\left(x-\left(-1+\sqrt{3}\right)\right)
\end{equation} 
The points $x_1, x_2$ split the real line into three intervals: $(-\infty, -1-\sqrt{3})$, $(-1-\sqrt{3}, -1+\sqrt{3})$ and $(-1+ \sqrt{3}, \infty )$, and each of the factors of \eqref{eq1problemSketch(x+1)/(x^2+2x+4)} has constant sign inside each of the intervals. If we choose $x$ to be a very negative number, it follows that $-(x-x_1)(x-x_2)$ is a negative, and therefore $ f'(x)$ is negative for $x\in(-\infty, -1-\sqrt{3})$. For $x\in (-1-\sqrt{3}, -1+\sqrt{3})$, exactly one factor of $f'$ changes sign and therefore $f'(x)$ is positive in that interval; finally only one factor of $f'(x)$ changes sign in the last interval so $f'(x)$ is negative on $(-1+ \sqrt{3}, \infty )$.

Our computations can be summarized in the following table. 

\begin{tabular}{|lll|}\hline
Interval & $f'(x)$ & $f(x)$   \\\hline
$(-\infty, -1-\sqrt{3})$ & $-$& $\searrow $ \\\hline
$(-1-\sqrt{3}, -1+\sqrt{3})$ &$+$&$\nearrow$\\\hline
$( -1+\sqrt{3}, \infty)$&$-$&$\searrow$ \\\hline
\end{tabular}

\textbf{Local and global minima and maxima. } The table above shows that $f(x)$ changes from decreasing to increasing at $x=x_1=-1-\sqrt{3}$ and therefore $f$ has a local minimum at that point. The table also shows that $f(x)$ changes from increasing to decreasing at $ x=x_2=-1+\sqrt{3}$ and therefore $f$ has a local maximum at that point. The so found local maximum and local minimum turn out to be global: indeed, no other finite point is critical and thus cannot be maximum or minimum; on the other hand $\lim\limits_{x\to\pm \infty}f(x)=1$ and this implies that all $x$ sufficiently far away from $x=0$ have that $f(x)$ is close to $0$, and therefore $f(x)$ is larger than $f(x_1)$ and smaller than $f(x_2)$ for all $x$.

\textbf{Intervals of concavity. } 
The intervals of concavity of $f$ are governed by the sign of $f''$. The second derivative of $f$ is:
\[
\begin{array}{rcll|l}
f''(x)&=&\displaystyle (f'(x))'= \left(\frac{-x^2-2x+2}{\left(x^2+2x+4 \right)^2}\right)'\\
&=&\displaystyle (-x^2-2x+2)'\left(\frac{1}{(x^2+2x+4)^2}\right)+(-x^2-2x+2)\left(\frac{1}{(x^2+2x+4)^2}\right)' &&\begin{array}{l}\text{use chain rule }\\\text{for second differentiation}\end{array}\\
&=&\displaystyle (-2x-2)\left(\frac{1}{(x^2+2x+4)^2}\right)+(-x^2-2x+2)(-2)\frac{(x^2+2x+4)'}{(x^2+2x+4)^{3}}\\
&=&\displaystyle -(2x+2)\left(\frac{1}{(x^2+2x+4)^2}\right) +(2x^2+4x-4)\frac{(2x+2)}{(x^2+2x+4)^{3}}&&\text{factor out }\frac{(2x+2)}{(x^2+2x+4)^2}\\
&=&\displaystyle \frac{(2x+2)}{(x^2+2x+4)^2}\left(-1+\frac{(2x^2+4x-4)}{(x^2+2x+4)}\right)\\
&=&\displaystyle \frac{(2x+2)}{(x^2+2x+4)^2}\left(\frac{-(x^2+2x+4)+(2x^2+4x-4)}{(x^2+2x+4)} \right)\\
&=&\displaystyle \frac{(2x+2)(x^{2}+2 x-8)}{(x^2+2x+4)^3}&& \text{factor } (x^2+2x-8)\\
&=&\displaystyle \frac{(2x+2)(x+4)(x-2)}{(x^2+2x+4)^3}
\end{array}
\]
As we previously established, the denominator of the above expression is always positive. Therefore the expression above changes sign when the terms in the numerator change sign, namely, at $x=-1$, $x=-4$ and $x=2$. 

Our computations can be summarized in the following table. 

\begin{tabular}{|lll|}\hline
Interval & $f''(x)$ & $f(x)$   \\\hline
$(-\infty, -4)$ & $-$& $\cap$ \\\hline
$(-4, -1)$ &$+$&$\cup$\\\hline
$(-1, 2)$&$-$&$\cap$ \\\hline
$(2, \infty)$&$+$&$\cup$ \\\hline
\end{tabular}

\textbf{Points of inflection.} The preceding table shows that $f''(x)$ changes sign at $-4, -1, 2$ and therefore the points of inflection are located at $x=-4, x=-1$ and $x=2$, i.e., the points of inflection are $\left(-4, -\frac{1}{4}\right)$, $\left(-1, 0\right)$, $\left(2, \frac{1}{4}\right)$.

}


\begin{problem}
Use the Intermediate Value theorem and the Mean Value Theorem/Rolle's Theorem to prove that the function has \textbf{exactly one} real root.
\begin{enumerate}
\item \label{problemIVTandMVTx^3+4x+7} $f(x)=x^3+4x+7$.
\item $f(x)= x^3 +x^2+x+1$.
\item \label{problemIVTandMVTcos3xdiv3+sinx-3x} $f(x)=\cos^3 \left({\frac{x}{3}}\right) +\sin x-  3x$.
\end{enumerate}

\end{problem}
\textbf{Solution \ref{problemIVTandMVTx^3+4x+7}.}  $f(0) = -7$ and $f(2) = 9$. Since $f(x)$ is continuous and has both negative and positive outputs, it must have a zero. In other words, for some $c$ between $0$ and $2$, $f(c) = 0$. If there were solutions $x = a$ and $x = b$,  then we would have $f(a) = f(b)$, and Rolle's Theorem would guarantee that for some $x$-value, $f'(x) = 0$. However, $f'(x) = 3x^2 + 4$, which is never 0. Therefore there cannot be 2 or more solutions. 

The above can be stated informally as follows. Note that $f'(x) = 3x^2 + 4$, which is always positive. Therefore, the graph of $f$ is increasing from left to right. So once the graph crosses the $x$-axis, it can never turn around and cross again, so there can only be a single zero (that is, a single solution to $f(x) = 0$).


\textbf{Solution \ref{problemIVTandMVTcos3xdiv3+sinx-3x}.} $f(5)= \cos^3 \left(\frac{5}{3}\right) +\sin 5-15 \leq 2-15=-13<0 $ (because $\cos a, \sin b\in [-1,1]$ for arbitrary $a,b$). Similarly $f(-5)=\cos^3\left(-\frac{5}{3}\right) +\sin (-5)+15 \geq 15-2>0$. Therefore by the Intermediate Value Theorem $f(x)=0$ has at least one solution in the interval $[-5,5]$.

Suppose on the contrary to what we are trying to prove, $f(x)=0$ has two or more solutions; call the first 2 solutions $a,b$. That means that $f(a)=f(b)=0$, so by the Mean value theorem, there exists a $c\in (a,b)$ such that $f'(c)=(f(a)-f(b))/(a-b)=(0-0)/(a-b)=0$. On the other hand we may compute:
\[ 
f'(x)=-3+\cos x-\cos^{2}\left(\frac{x}3\right)\sin\left(\frac{x}{3}\right) \leq -1<0,
\] 
where the first inequality follows from the fact that $\sin x,\cos x\in [-1,1]$. So we got that $f'(c)=0$ for some $c$ but at the same time $f'(x)<0$ for all $x$, which is a contradiction. Therefore $f(x)=0$ has exactly one solution. 

\begin{problem}
Estimate the integral using a Riemann sum using the indicated sample points and interval length.
\begin{enumerate}
% Riemann sums
\item $\displaystyle \int_0^4 \left(\sqrt{8x+1}\right)\diff x$. Use four intervals of equal width, choose the sample point to be the left endpoint of each interval. 

\answer{ $\Delta x = 1$ and $f(x) = \sqrt{8x+1}$. ${\displaystyle \int_0^4 f(x) \diff x \approx 9 + \sqrt{17}}$.}

\item $\displaystyle \int_0^6 \frac{1}{x^2+1} \diff x$. Use three intervals of equal width, choose the sample point to be the left endpoint. 

\answer{ $\Delta x = 2$ and $f(x) = \frac{1}{x^2+1}$. ${\displaystyle \int_0^6 f(x) \diff x \approx \frac{214}{85}}$.}

\item $\displaystyle\int_{0}^2 \frac{\diff x}{1+x+x^3}$. Use $\Delta x=\frac{1}2 $ and right endpoint sampling points.

\answer{$ \frac{1}{2}\left(\frac{8}{13}+\frac{1}{3}+\frac{8}{47}+\frac{1}{11}\right)=\frac{12197}{20163}\approx 0.604920$}
\item $\displaystyle\int_{-2}^{0} \frac{\diff x}{1+x+x^2}$. Use $\Delta x=\frac23 $ and left endpoint sampling points.

\answer{$\frac23\left(\frac{1}{3}+\frac{9}{13}+\frac{9}{7}\right)=\frac{1262}{819}\approx 1.540904$}

\end{enumerate}

\end{problem}
\solution{\ref{problemRiemannSum-sqrt(8x+1)}. The interval $[0,4]$ is subdivided into $n=4$ intervals, therefore the length of each is $\Delta x=1$. The intervals are therefore 
\[
[0,1], [1,2], [2,3], [3,4]\quad .
\]
The problem asks us to use the left endpoints of each interval as sampling points. Therefore our sampling points are $0,1,2,3$. Therefore the Riemann sum we are looking for is 
\[
\Delta x\left(f(0)+f(1)+f(2)+f(3) \right)=1\cdot \left(\sqrt{8\cdot 0+1}+\sqrt{8\cdot 1+1}+\sqrt{8\cdot 2+1}+\sqrt{8\cdot 3+1}\right)= 9+\sqrt{17}\approx 13.1231
\]
\psset{xunit=1cm, yunit=1cm}
\begin{pspicture}(-0.9, -0.9)(4.4,6.233433) 
\tiny 
\psline*[linecolor=\psColorAreaUnderGraph, linewidth=0.1pt](0.000000, 0.000000)(0.000000, 1.000000)(1.000000, 1.000000)(1.000000, 0.000000)(0.000000, 0.000000)
\psline*[linecolor=\psColorAreaUnderGraph, linewidth=0.1pt](1.000000, 0.000000)(1.000000, 3.000000)(2.000000, 3.000000)(2.000000, 0.000000)(1.000000, 0.000000)
\psline*[linecolor=\psColorAreaUnderGraph, linewidth=0.1pt](2.000000, 0.000000)(2.000000, 4.123106)(3.000000, 4.123106)(3.000000, 0.000000)(2.000000, 0.000000)
\psline*[linecolor=\psColorAreaUnderGraph, linewidth=0.1pt](3.000000, 0.000000)(3.000000, 5.000000)(4.000000, 5.000000)(4.000000, 0.000000)(3.000000, 0.000000)
\psline[linecolor=blue, linewidth=0.1pt](0.000000, 0.000000)(0.000000, 1.000000)(1.000000, 1.000000)(1.000000, 0.000000)(0.000000, 0.000000)
\psline[linecolor=blue, linewidth=0.1pt](1.000000, 0.000000)(1.000000, 3.000000)(2.000000, 3.000000)(2.000000, 0.000000)(1.000000, 0.000000)
\psline[linecolor=blue, linewidth=0.1pt](2.000000, 0.000000)(2.000000, 4.123106)(3.000000, 4.123106)(3.000000, 0.000000)(2.000000, 0.000000)
\psline[linecolor=blue, linewidth=0.1pt](3.000000, 0.000000)(3.000000, 5.000000)(4.000000, 5.000000)(4.000000, 0.000000)(3.000000, 0.000000)
%Function formula: (8 x+1)^{1/2} 
\psplot[linecolor=\psColorGraph, plotpoints=1000]{0}{4}{ 1 x 8 mul add 0.5 exp }
\psaxes(0,0)(-0.65,-0.65)(4.15,5.883433)
\end{pspicture} 
}


\begin{problem}
Evaluate the definite integral.
\begin{multicols}{3}
\begin{enumerate}
\item $\displaystyle \int\limits_{-2}^{3} \left(x^2-1 \right)  \diff x$.

\answer{$\left[ \frac{1}{3} x^{3}- x \right]_{-2}^3=\frac{20}{3}$}

\item $\displaystyle \int\limits_{1}^{2} \left(4x^3+3x^2+2x+1\right)  \diff x$.

\answer{$\left[ x^{4}+x^{3}+x^{2}+x\right]_{1}^{2}=26$}
\item $\displaystyle \int\limits_{0}^{2}(x-1)(x^2+1)  \diff x$.

\answer{$\left[\frac{1}{4} x^{4}-\frac{1}{3} x^{3}+\frac{1}{2} x^{2}- x \right]_{0}^{2} = \frac{4}{3}$}
\item $\displaystyle \int\limits_{-1}^{1} \left( \frac{x(x+1) }{ 2} \right)^2  \diff x$.

\answer{$\left[\frac{1}{20} x^{5}+\frac{1}{8} x^{4}+\frac{1}{12} x^{3} \right]_{-1}^{1}=\frac{4}{15}$}
\item $\displaystyle \int\limits_{0}^{1}(1+x^2)^3 dx$.

\answer{$\left[\frac{1}{7} x^{7}+\frac{3}{5} x^{5} + x^{3} + x \right]_{0}^{1}=\frac{96}{35}$}
\item $\displaystyle \int \limits_{1}^{2} \left(\frac{1}{x} - \frac{4}{x^2} \right)  \diff x$.

\answer{$\left[ 4 x^{-1}+\ln x\right]_{1}^{2}=\ln 2-2$}
\item $\displaystyle \int\limits_{1}^{4}\sqrt{x}(1+x) \diff x$.

\answer{$\left[\frac{2}{5} x^{\frac{5}{2}}+\frac{2}{3} x^{\frac{3}{2}} \right]_{1}^{4}=\frac{256}{15}$}
\item $\displaystyle \int\limits_{1}^{4} \sqrt{ \frac{6 }{x }} \diff x$.

\answer{$\left[2 \sqrt{6} \sqrt{x} \right]_{1}^{4}=-2 \sqrt{6} + (2)^{3/2}$}
\item $\displaystyle \int \limits_{1}^{4} \frac{ \frac{ 1}{ \sqrt{x}}+1+x}{ \sqrt{x}}  \diff x$.

\answer{$\left[ \right]_{}^{}=$}
\item $\displaystyle \int \limits_{1}^{4} \frac{1+x}{ \sqrt[3]{x}} \diff x$.

\answer{$\left[ \right]_{}^{}=$}
\item $\displaystyle \int\limits_{0}^1 \frac{\frac{1}{ \sqrt[3]{x}} +\sqrt[3]{x}}{ \sqrt{x}}\diff x$.

\answer{$\left[ \right]_{}^{}=$}
\item $\displaystyle \int\limits_0^{1} (\sqrt[5]{x^6} + \sqrt[6]{x^5})\diff x $.

\answer{$\left[ \right]_{}^{}=$}
\item $\displaystyle \int\limits_{1}^{2} \left(x + \frac{1}{x} \right)^2 \diff x$.

\answer{$\left[ \right]_{}^{}=$}
\item $\displaystyle \int\limits_{1}^{2} \left(x + \frac{1}{x} \right)^3 \diff x$.

\answer{$\left[ \right]_{}^{}=$}
\item $\displaystyle \int\limits_{1}^{2} \left(\sqrt{x} + \frac{1}{\sqrt{x}} \right)^2 \diff x$.

\answer{$\left[ \right]_{}^{}=$}
\item $\displaystyle \int\limits_{1}^{2} \left(\sqrt{x} +\frac{1}{\sqrt{x}} \right)^3 \diff x$.

\answer{$\left[ \right]_{}^{}=$}
\item $\displaystyle \int\limits_{0}^{2}|x-1| \diff x$.

\answer{$\left[ \right]_{}^{}=$}
\item $\displaystyle \int\limits_{0}^{1} \left|x-\frac{1}{2}\right| \diff x$.

\answer{$\left[ \right]_{}^{}=$}
\item $\displaystyle \int\limits_{-1}^{1}(x-3|x|) \diff x$.

\answer{$\left[ \right]_{}^{}=$}
\item $\displaystyle \int\limits_{\frac{\pi}{4}}^{\frac{\pi}{2}} \csc^2\theta \diff \theta$.

\answer{$\left[ \right]_{}^{}=$}
\item $\displaystyle \int\limits_{0}^{\frac{\pi}{4}}\frac{1-\cos^2\theta}{\cos^2\theta} \diff \theta$.

\answer{$\left[ \right]_{}^{}=$}
\item $\displaystyle \int\limits_{0}^{\frac{\pi}{4}}\frac{\sin^2\theta}{\cos^2\theta} \diff \theta$.

\answer{$\left[ \right]_{}^{}=$}
\item $\displaystyle \int\limits_{0}^{\frac{\pi}{4}}\tan^2\theta \diff \theta$.

\answer{$\left[ \right]_{}^{}=$}
\item $\displaystyle \int\limits_{0}^{\frac{\pi}{3}} \frac{\sin \theta +\sin \theta \tan^2\theta}{\sec^2\theta}\diff \theta$.

\answer{$\left[ \right]_{}^{}=$}
\item $\displaystyle \int\limits_{0}^{\pi} (\sin \theta -\cos \theta) \diff \theta$.

\answer{$\left[ \right]_{}^{}=$}
\item $\displaystyle \int\limits_{0}^{\pi}|\sin x| \diff x$.
\end{enumerate}
\end{multicols}

\end{problem}
\solution{\ref{problemIntegralAbsoluteValuexminushalf}
\[
\begin{array}{rcll|l}
\displaystyle \int\limits_{0}^{1} \left|x-\frac{1}{2}\right| \diff x
&=&\displaystyle  \int\limits_{0}^{\frac{1}{2}}  \left|x-\frac{1}{2}\right| \diff x+\int\limits_{\frac{1}{2}}^1  \left|x-\frac{1}{2}\right| \diff x&&\begin{array}{l}\left|x-\frac{1}{2}\right| =\frac{1}{2}-x \text{ when }x\leq \frac{1}{2} \\
\left|x-\frac{1}{2}\right| =x-\frac{1}{2} \text{ when }x\geq \frac{1}{2}
\end{array}\\
&=&\displaystyle  \int\limits_{0}^{\frac{1}{2}} \left(\frac{1}{2}-x\right) \diff x+\int \limits_{\frac{1}{2}}^1 \left(x-\frac{1}{2}\right) \diff x\\
&=&\displaystyle \left[-\frac{x^2}{2}+\frac{x}{2}\right]_{0}^{\frac{1}{2}} +\left[\frac{x^2}{2}-\frac{x}{2}\right]_{\frac{1}{2}}^{1}\\
&=&\displaystyle \left(-\frac{1}{8}+\frac{1}{4}\right) +\left(\frac{1}{2} -\frac{1}{2}-\left(\frac{1}{8}-\frac{1}{4} \right)\right)\\
&=&\displaystyle \frac{1}{4}
\end{array}
\]
}


\begin{problem}
Evaluate the definite integral. The answer key has not been proofread, use with caution.
\begin{enumerate}[ref={\fcProblemRef}]
\item $\displaystyle\int\limits_{e}^{e^3}\frac{\diff x}{x \sqrt[3]{\ln x}} $.

\answer{$ \frac{3}{2}\left( \sqrt[3]{9} -1\right)$}

\item $\displaystyle\int\limits_{0}^{1}xe^{-x^2} \diff x$.

\answer{$ \frac{1-e^{-1}}{2}$}
\item $\displaystyle\int\limits_{0}^{1}\frac{e^x+1}{e^x+x} \diff x$.

\answer{$\ln(e+1) $}
\item \label{problemIntx/(2x^2+1)} $\displaystyle\int\limits_{1}^{2} \frac{x}{2x^2+1 }  \diff x$.

\answer{$\frac14 \ln 3$}

\item $\displaystyle\int\limits_{0}^{\frac{1}4}\frac{x }{\sqrt{1-3x^2}}\diff x$.

\answer{$\frac{1}3\left(1-\sqrt{\frac{13}{16}} \right)$}

\end{enumerate}

\end{problem}
\solution{\ref{problemIntx/(2x^2+1)}
\[
\begin{array}{rcll|l}
\displaystyle\int\limits_{1}^{2} \frac{x}{2x^2+1 }  \diff x&=&\displaystyle \int \limits_{x=1}^{x=2} \frac{\frac{1}{4}\diff (2x^2)}{2x^2+1 }  =
\frac{1}{4}\int \limits_{x=1}^{x=2} \frac{\diff (2x^2+1)}{2x^2+1 } &&\text{Set }u=2x^2+1\\
&=&\displaystyle \frac{1}{4}\int\limits_{\substack{x=1 \\u=3} }^{\substack{ x=2\\u=9}} \frac{\diff u}{u} = \frac{1}{4}\left[ \ln u\right]_{3}^9=\frac{1}{4}\left(\ln 9-\ln 3\right)=\frac{\ln 3}{4}.
\end{array}
\]
}
\solution{\ref{problemIntegratefrom-3to2_x/(1-x^2)dx}

\[\begin{array}{rcll|l}
\displaystyle \int_{-3}^{-2} \frac{x}{1-x^2}\diff x&=&\displaystyle \int\limits_{\tiny \begin{array}{rcl}x&=&-3\\ u&=&-8\end{array}}^{\tiny  \begin{array}{rcl}x&=&-2\\ u&=&-3\end{array}} \frac{1}{u}\left(-\frac{1}{2}\diff u\right)&& \begin{array}{rcl}
u&=&1-x^2\\
\diff u&=&-2x\diff x\\
x\diff x&=&-\frac{1}{2}\diff u
\end{array}\\
&=&\displaystyle -\frac{1}{2}\left[\ln |u|\right]_{-8}^{-3}\\~\\
&=&\displaystyle -\frac{1}{2} \left(\ln|3|-\ln|8| \right)\\~\\
&=&\displaystyle \frac{\ln\left|\frac{8}{3}\right|}{2}
\end{array}
\]

}
\solution{\ref{problemintfrom-3to-2of3x/(2-x^2)dx}

\[
\begin{array}{rcll|l}
\displaystyle \int_{-3}^{-2} \frac{3x}{2-x^2}\diff x&=&\displaystyle \int_{-3}^{-2} \frac{3  \frac{\diff\left(x^2\right)}{2}}{2-x^2} \diff x \\
&=&\displaystyle \frac{3}{2} \int_{-3}^{-2} \frac{  -\diff\left(-x^2\right)}{2-x^2}  \\
&=&\displaystyle -\frac{3}{2} \int_{-3}^{-2} \frac{  \diff\left(-x^2\right)}{2-x^2} \\
&=&\displaystyle -\frac{3}{2} \int_{x=-3}^{x=-2} \frac{  \diff\left(2-x^2\right)}{2-x^2} &&\text{Set } 2-x^2=u \\
&=&\displaystyle -\frac{3}{2} \int_{x=-3, u=-7}^{x=-2, u=-2} \frac{\diff u}{u}\\
&=&\displaystyle -\frac{3}{2}\left[\ln |u|\right]_{-7}^{-2}\\
&=&\displaystyle -\frac{3}{2}(\ln 2-\ln 7 )\\
&=&\displaystyle \frac{3}{2}\ln\left(\frac{7}{2}\right).\\
\end{array}
\]
}



\begin{problem}
Differentiate $f(x)$ using the Fundamental Theorem of Calculus part 1.
\begin{multicols}{2}
\begin{enumerate}[ref={\fcProblemRef}]
\item $\displaystyle f(x) = \int\limits_1^x \sin \left(t^2\right)  \diff t$

\answer{$\displaystyle \sin \left(x^2\right)$} 
\item \label{problemd/dxint_1^x(t-sqrt(t))dt}

$\displaystyle f(x)=\int_{1}^x\left(t-\sqrt{t}\right)\diff t $.

\answer{$x-\sqrt{t}$}
\item \label{problemDifferentiateFTC1int_x^1(2+t^4)^5dt}  ${\displaystyle f(x) = \int\limits_x^1 (2+t^4)^5 \; \diff t}$

\answer{${\displaystyle -\left(2+x^4\right)^5}$} 

\item $\displaystyle f(x)=\int\limits_{0}^{x^2} t^2\diff t $.

\answer{$f'(x)=2x^5$ }

\item \label{problemd/dx(int_(ln x)^(e^x)t^3dt)} $\displaystyle f(x)=\int\limits_{\ln x}^{e^x} t^3\diff t $.

\answer{$f'(x)=e^{4x}-\frac{(\ln x)^3}{x}$ }

\item \label{problemd/dxint_1^x(sqrt(t)-t^(1/3))dt}
$\displaystyle f(x)=\int_{1}^x\left(\sqrt{t}- \sqrt[3]{t}\right)\diff t$.

\answer{$ \sqrt{x}-\sqrt[3]{x}$}
\item \label{problemd/dxint_1^(1/(x+1))sin(t^2)dt}

$\displaystyle f(x)=\int_{1}^{\frac{1}{x+1} }\sin \left(  t^2\right) \diff t$.

\answer{}
\item \label{problemd/dxint_1^(1/(1+x))cos(t^2)dt}

$\displaystyle f(x)= \int_{1}^{\frac{1}{x+1} }\cos  \left(  t^2\right) \diff t$.

\answer{$-\frac{1}{(x+1)^2}\cos \left( \frac{1}{(x+1)^2}\right) $}
\item ${\displaystyle f(x) = \int_{0}^{x^3} \cos^2 t \; \diff t}$

\answer{${\displaystyle 3x^2 \cos^2\left(x^3\right)}$} 

\end{enumerate}
\end{multicols}
\end{problem}
\solution{\ref{problemd/dx(int_0^(x^2)t^2dt)}


}

\begin{problem}
\begin{enumerate}
% Area problems
\item \label{problemAreaBetweeny=2x^2,y=4+x^2} Find the area of the region bounded by the curves $y = 2x^2$ and $y = 4 + x^2$.

\answer{$\frac{32}{3}$}
\item \label{problemAreaBetweeny=2-x,x=4-y^2} Find the area of the region bounded by the curves $x = 4 - y^2$ and $y = 2 - x$

\answer{$\frac92$}
\end{enumerate}

\end{problem}
\textbf{Solution. 
\ref{problemAreaBetweeny=2x^2,y=4+x^2}.} $x=4-y^2$ is a parabola (here we consider $x$ as a function of $y$). $y=-x+2$ implies that $x=2-y$ and so the two curves intersect when 
\[
\begin{array}{rcl}
4-y^2&=&2-y\\
-y^2+y+2&=&0\\
-(y+1)(y-2)&=&0\\
y&=& -1\text{~or~}2\quad \quad .
\end{array}
\]
As $x=2-y$, this implies that $x=0$ when $y=2$ and $x=3$ when $y=-1$, or in other words the points of intersection are $(0,2)$ and $(3, -1)$. Therefore we may identify the region whose area we are looking for as the area plotted below. Integrating with respect to $y$, we get that the area is 

\[
\begin{array}{rcl}
A&=&\displaystyle \int_{-1}^{2} |4-x^2-(-x+2) | \diff y = \int_{-1}^2 (-y^2+y+2)\diff y \\
&=& \displaystyle \left[- \frac{y^3}3 +\frac{y^2}{2}+ 2y\right]_{-1}^2
=-\frac{8}{3}+2+4 -\left(-\frac{(-1)^3}{3} +\frac{ (-1)^2}{2}-2 \right)\\
&=&\displaystyle \frac{9}{2}
\end{array}
\]


\psset{xunit=0.5cm, yunit=0.5cm}
\begin{pspicture}(-3.500000, -5)(4.500000,5) 
\psframe*[linecolor=white](-3.500000,-5)(4.500000,5) 
\tiny 
\rput(-1.5,5){$y=- x+2$} 
\psplot[linecolor=\psColorGraph, plotpoints=1000]{-3.000000}{4.000000}{2 x -1 mul add }
%Function formula: - (- x+4)^{1/2} 
\psplot[linecolor=\psColorGraph, plotpoints=1000]{-3.000000}{4.000000}{4 x -1 mul add 0.5 exp -1 mul }
%Function formula: (- x+4)^{1/2} 
\rput(2,2){$x=4-y^2$} 
\psplot[linecolor=\psColorGraph, plotpoints=1000]{-3.000000}{4.000000}{4 x -1 mul add 0.5 exp }
\pscustom*[linecolor=cyan]{
\psplot[linecolor=\psColorGraph, plotpoints=1000]{0}{4}{4 x -1 mul add 0.5 exp }
\psplot[linecolor=\psColorGraph, plotpoints=1000]{4}{3}{4 x -1 mul add 0.5 exp -1 mul }
}
\psaxes[arrows=<->, ticks=none, labels=none](0,0) (-3.000000,-4.5)(4.5,4.5) %Function formula: - x+2 
\end{pspicture}


\begin{problem}
\begin{enumerate}[ref={\fcProblemRef}]~
% Volume problems
\item  \label{problemVolumeRegionBoundedByy=2x^2-x+1,y=x^2+1rotatedAroundx=0} Consider the region bounded by the curves $y = 2x^2-x+1$ and $y =x^2+1$. What is the volume of the solid obtained by rotating this region about the line $x = 0$?

\item Consider the region bounded by the curves $y = 1-x^2$ and $y =0$. What is the volume of the solid obtained by
rotating this region about the line $y = 0$?

\answer{$\frac{16 \pi}{15}$}
\item Consider the region bounded by the curves $y = x^2$ and $x = y^2$. What is the volume of the solid obtained by
rotating this region about the line $x = 2$?

\answer{ $\frac{31 \pi}{30}$}


\end{enumerate}

\end{problem}
\solution{\ref{problemVolumeRegionBoundedByy=2x^2-x+1,y=x^2+1rotatedAroundx=0}



}

%\vskip 18cm
%\hfill \begin{tabular}{c|c|c|c|c|c|c||c}
%Problem&1 &2&3&4&5&6& $\sum$\\ \hline
%Score&&&&&&&\\ \hline
%Max&17&17&17&17&17&17&102
%\end{tabular} 


\end{document}