\documentclass{article}
%\addtolength{\hoffset}{-3.5cm}
%\addtolength{\textwidth}{6.8cm}
%\addtolength{\voffset}{-3cm}
%\addtolength{\textheight}{6cm}
\ProvidesPackage{homework-problems}
\usepackage{amsmath, amsfonts, amssymb, verbatim, hyperref, ifthen}
\usepackage{auto-pst-pdf}
\usepackage{pst-plot}
\usepackage{multicol}
\renewcommand{\Re}{\mathrm{Re~}}
\renewcommand{\Im}{\mathrm{Im~}}
\newcommand{\doublebrace}[4]{\left\{\begin{array}{ll} #1 & #2 \\#3 & #4  \end{array} \right.}
\newcommand{\triplebrace}[6]{\left\{\begin{array}{ll} #1 & #2 \\#3 & #4  \\#5 & #6\end{array} \right.}
\newcommand{\bigFatWarning}{ %\textbf{This homework contains copyrighted material from  James Stewart, Calculus, 7th edition, 2012. You are not permitted to copy this file for any purpose other than completing your homework. You are not allowed to give a copy of this file to anyone outside of our course. }
}
\newenvironment{solution}%
{\begin{proof}[\bfseries\upshape Solution]\renewcommand{\qedsymbol}{}}%
{\end{proof}}%
\newcommand{\ans}[1]{\iftoggle{solutions}{\begin{solution}#1\end{solution}}{}}
\newcommand{\homeworkEnd}{\end{enumerate}\end{document}}
\newcommand{\homeworkStart}[2]{\title{\course \\ Homework \ #1}\date{%
\ifthenelse{\equal{#2}{}}{}{%
Due #2 at \deadline}}%
\begin{document}\maketitle\begin{enumerate}
}%
\newcommand{\points}[1]{\stepcounter{enumi}\item[ ({\bf #1 mark\ifthenelse{\equal{#1}{1}}{}{s}}) \arabic{enumi}.]}
\newcommand{\pointsii}[1]{\stepcounter{enumii}\item[ ({\bf #1 mark\ifthenelse{\equal{#1}{1}}{}{s}}) (\alph{enumii})]}
\newcommand{\answer}[1]{ \hfill{~} \rotatebox{180}{ answer: #1}}
 %warning folder paths are relative to the file that uses the includepackage

\renewcommand{\answer}[1]{\iftoggle{answers}{ \hfill{~} \rotatebox{180}{\tiny answer: #1}}{} }
\renewcommand{\hiddenanswer}{\answer}
\renewcommand{\points}[1]{\item}
\renewcommand{\pointsii}[1]{\item}
\renewcommand{\Arctan}{\arctan}
\renewcommand{\Arccos}{\arccos}
\renewcommand{\Arcsin}{\arcsin}
\renewcommand{\Arccot}{\operatorname{arccot}}


\toggletrue{solutions}
\toggletrue{answers}
\renewcommand{\fcProblemRef}{\theproblem.\theenumi}
\renewcommand{\fcSubProblemRef}{\theenumi.\theenumii}


\newcommand{\hide}[1]{}
\newtheorem{problem}{Problem}
\pagestyle{empty}
\begin{document}
\begin{center}
\Large
Review sheet Test 1 \\ Math 140 Calculus I \\ \normalsize Summer 2015 \\ Instructor: Todor Milev
\end{center}
%\noindent \textbf{Name:\underline{~~~~~~~~~~~~~~~~~~~~~~~} } \hfill{~}



\noindent The exam is closed textbook. \textbf{No electronic devices are allowed during the exam. } The exam will contain 6 problems, of the 6 problem types given in this review sheet. %You are allowed one single formula sheet, handwritten by you. No template problem solutions are allowed. The sheet will be collected with the test. Photocopied formula sheets are not allowed. 

\begin{problem}
Compute the composite functions $(f\circ g)(x)$, $(g\circ f)(x)$. Simplify your answer to a single fraction. Find the domain of the composite function. The answer key has not been fully proofread, use with caution. 

\begin{enumerate}[ref={\fcProblemRef}]
\item $\displaystyle f{}({{x}})=\frac{x+2}{x-2},
g{}({{x}})=\frac{x-1}{x+2}$.

\answer{
\begin{tabular}{rl}
$(f\circ g)(x)= \frac{3+3 x}{-5- x}$& $x\neq -2, -5$\\ 
$(g\circ f)(x)=\frac{4}{-2+3 x}$& $x\neq 2, \frac{2}{3}$ 
\end{tabular}
}
\item 
$\displaystyle f{}({{x}})=\frac{x+1}{3x-2}, g{}({{x}})= \frac{x-2}{x-1}
$.

\answer{
\begin{tabular}{rl}
$(f\circ g)(x)= \frac{-3+2 x}{-4+x}
$ & $x\neq 4, 1$ \\
$(g\circ f)(x)=\frac{5-5 x}{3-2 x}$
& $x\neq \frac{2}{3}, \frac{3}{2}$
\end{tabular}
}
\item 
$\displaystyle f{}({{x}})=\frac{2x+1}{3x-1},
g{}({{x}})=\frac{x-2}{2x-1}
$.

\answer{
\begin{tabular}{rl}
$(f\circ g)(x)=\frac{-5+4 x}{-5+x}
$ &$x\neq 5, \frac{1}{2}$ \\
$(g\circ f)(x)=\frac{3-4 x}{3+x}
$ &$x\neq -3, \frac{1}{3}$
\end{tabular}
}
\item 
$\displaystyle f{}({{x}})=\frac{x+1}{x-2},
g{}({{x}})=\frac{x+2}{2x-1}
$.

\answer{
\begin{tabular}{rl}
$(f\circ g)(x)= \frac{1+3 x}{4-3 x}
$&$x\neq \frac{4}{3}, \frac{1}{2}$\\ 
$(g\circ f)(x)=\frac{-3+3 x}{4+x}
$&$x\neq -4, 2$
\end{tabular}
}
\item 
$\displaystyle f{}({{x}})=\frac{5x+1}{4x-1},
g{}({{x}})=\frac{4x-1}{3x+1}
$.

\answer{
\begin{tabular}{rl}
$(f\circ g)(x)= \frac{-4+23 x}{-5+13 x}
$&$x\neq \frac{5}{13}, -\frac{1}{3 }$\\
$(g\circ f)(x)=\frac{5+16 x}{2+19 x}
$&$x\neq -\frac{2}{19}, \frac{ 1}{4}$
\end{tabular}
}


\item 
$\displaystyle  f(x)= \frac{3x-5}{x-2}$, $\displaystyle g(x)=\frac{x-2 }{x-4} $. 

\answer{ 
\begin{tabular}{rl}
$(f\circ g)(x)=\frac{-2 x+14}{- x+6}$&$x\neq 6, 4$\\
$(g\circ f)(x)=\frac{x-1}{- x+3}$&$x\neq 3,2$
\end{tabular}
}

\item 
$\displaystyle  f(x)= \frac{x-3}{x+2}$, $\displaystyle g(y)=\frac{y+3 }{y-4} $. 

\answer{ 
\begin{tabular}{rl}
$(f\circ g)(x)=\frac{-2x+15}{3 x-5}$&$x\neq \frac{5}{3}, 4$\\ 
$(g\circ f)(x)=\frac{4 x+3}{-3 x-11}$&$x\neq -\frac{11}{3}, -2$
\end{tabular}
}

\end{enumerate}

\end{problem}
\input{../../modules/functions-basics/homework/functions-composing-fractional-linear-1-solutions}
\begin{problem}
(Textbook page 70, problems 11-32). 
Evaluate the limit if it exists.
\begin{multicols}{3}
\begin{enumerate}
\item $\displaystyle\lim\limits_{x\to 5}\frac{x^2-6x+5}{x-5} $. 
\answer{4}
\item $\displaystyle\lim\limits_{x\to 4}\frac{x^2-4x}{x^2-3x-4} $.
\answer{$\frac{4}5$}
\item $\displaystyle\lim\limits_{x\to 5}\frac{x^2-5x+6}{x-5} $.
\answer{DNE}
\item $\displaystyle\lim\limits_{x\to -1}\frac{x^2-4x}{x^{2}-3x-4} $.
\answer{DNE}
\item $\displaystyle\lim\limits_{t\to -3}\frac{t^2-9}{2t^2+7t+3} $.
\answer{$\frac{6}{5}$}
\item $\displaystyle\lim\limits_{x\to -1}\frac{2x^2+3x+1}{x^2-2x-3} $.
\answer{$\frac{1}{4}$}
\item $\displaystyle\lim\limits_{h\to 0}\frac{(-5+h)^2-25}{h} $.
\answer{$-10$}
\item $\displaystyle\lim\limits_{h\to 0}\frac{(2+h)^3-8}{h} $.
\answer{$12$}
\item $\displaystyle\lim\limits_{x\to -2}\frac{x+2}{x^3+8} $.
\answer{$\frac{1}{12}$}
\item $\displaystyle\lim\limits_{t\to 1}\frac{t^4-1}{t^3-1} $.
\answer{$\frac{4}{3}$}
\item $\displaystyle\lim\limits_{h\to 0}\frac{\sqrt{9+h}-3}{h} $.
\answer{$\frac{1}{6}$}
\item $\displaystyle\lim\limits_{u\to 2} \frac{\sqrt{4u+1}-3}{u-2}$.
\answer{$\frac{2}{3}$}
\item $\displaystyle\lim\limits_{x\to -4} \frac{\frac{1}{4}+ \frac{1}{x}} {4+x}$.
\answer{$-\frac{1}{16}$}
\item $\displaystyle\lim\limits_{x\to -1} \frac{x^2+2x+1}{x^4-1}$.
\answer{$0$}
\item $\displaystyle\lim\limits_{t\to 0} \frac{\sqrt{1+t}- \sqrt{1-t}}{t}$.
\answer{$1$}
\item $\displaystyle\lim\limits_{t\to 0}\left(\frac{1}t -\frac{1}{t^2+t}\right)$.
\answer{$1$}
\item $\displaystyle\lim\limits_{x\to 16} \frac{4-\sqrt{x}}{16x-x^2}$.
\answer{$\frac{1}{128}$}
\item $\displaystyle\lim\limits_{h \to 0}\frac{(3+h)^{-1}-3^{-1}}{h} $.
\answer{$-\frac{1}{9}$}
\item $\displaystyle\lim\limits_{t\to 0} \left(\frac{1}{t\sqrt{1+t}}-\frac{1}{t} \right)$.
\answer{$-\frac{1}{2}$}
\item $\displaystyle\lim\limits_{x\to -4} \frac{\sqrt{x^2+9}-5}{x+4}$.
\answer{$-\frac{4}{5}$}
\item $\displaystyle\lim\limits_{h\to 0}\frac{(x+h)^3-x^3}{h} $.
\answer{$3x^2$}
\item $\displaystyle\lim\limits_{h\to 0}\frac{\frac{1}{(x+h)^2}-\frac{1}{x^2}}{h} $.
\answer{$-\frac{2}{x^3}$}
\end{enumerate}
\end{multicols}

\end{problem}
\begin{problem}
\begin{enumerate}
\item (1) Solve the equation $x^2+13x+41=1$.  (2) Use the intermediate value theorem to prove that the equation $x^2+13x+41=\sin  x$ has at least two solutions, lying between the two numbers found in (1).
\solution{
\noindent (1) 
\begin{eqnarray*}
x^2+13x+41&=&1\\
x^2+13x+40&=&0\\
(x+5)(x+8)&=&0\quad .
\end{eqnarray*}
Therefore the two solutions are $x_1=-5$ and $x_2=-8$.

\noindent (2) Consider the function
\[
f(x)=x^2+13x+41-\sin x\quad. 
\]
Our strategy for proving $f(x)=0$ has a solution consists in finding a number $a$ such that $f(a)<0$ and a number $b$ such that $f(b)>0$, and then using the Intermediate Value Theorem (IVT) with $N=0$. 

Let 
\[
g(x)=x^2+13x+41,
\]
and so $f(x)=g(x)-\sin x$. We have no techniques for evaluating $\sin x$ without calculator, but we do have all knowledge necessary to evaluate $g(x)$. Indeed, from high school we know that the lowest point of the parabola $g(x)$ is located at $x=-\frac{13}2=-6.5$. Then $g(-6.5)= -1.25$. Therefore 
\[
f(-6.5)=g(-6.5)-\sin(-6.5)= g(-6.5)+\sin (6.5)=-1.25+\sin 6.5 \leq -0.25, 
\]
where for the very last inequality we use the fact that $\sin 6.5< 1 $ (remember $\sin t\leq 1$ for all real values of $t$).

On the other hand, 
\[f(-5)= g(-5)-\sin (-5) = 1+\sin 5> 0\]
as $\sin 5 >-1$ (remember $\sin t\geq -1$ for all real values of $t$). Therefore $f(-5)>0$ and $f(-6.5)<0$ and by the Intermediate Value Theorem (IVT) $f(x)=0$ has a solution in the interval $x\in (-6.5, -5)$. 
 
Proving $f(x)=0$ has a solution in the interval $x\in (-8, -6.5)$ is similar and we leave it to the student. 

Below is a computer generated plot of the function with the use of which we can visually verify our answer.

\psset{xunit=1cm, yunit=1cm}
\begin{pspicture}(-9, -5)(1,5) 
\psframe*[linecolor=white](-9,-5)(1,5) 
\tiny 
\psaxes[ticks=none, labels=none]{<->}(0,0)(-9,-4.5)(1,4.5)
\psLabels{1}{5}
\psXTickWithLabel{-6.5}{$-6.5$}
\psXTickWithLabel{-8}{$-8$}
\psXTickWithLabel{-5}{$-5$}
%Function formula: (x)^{2}+40+13 (x) 
\rput(-4.5,-2){$y=x^{2}+13 x+40$} 
\psplot[linecolor=grey!30, plotpoints=1000]{-9}{-4}{x 13 mul 40 x 2 exp add add }
\rput(-6.5,3){$y=x^{2}+41+13 x- \sin x$} 
\psplot[linecolor=\psColorGraph, plotpoints=1000]{-9}{-4}{x 57.29578 mul sin -1 mul x 13 mul 41 x 2 exp add add add }

\end{pspicture}
}

\item (1) Solve the equation $x^2-15x+55=1$.  (2) Use the intermediate value theorem to prove that the equation $x^2-15x+55=\cos  x$ has at least two solutions, lying between the two numbers found in (1).
\end{enumerate}

\end{problem}
\solution{\ref{problemIVTtoshowx^2+13x+14=sinx-has-solutionsPart1}.
\[
\begin{array}{rcl}
x^2+13x+41&=&1\\
x^2+13x+40&=&0\\
(x+5)(x+8)&=&0\quad .
\end{array}
\]eqnarray
Therefore the two solutions are $x_1=-5$ and $x_2=-8$.


\noindent \ref{problemIVTtoshowx^2+13x+14=sinx-has-solutionsPart2}. Consider the function
\[
f(x)=x^2+13x+41-\sin x\quad.
\]
Our strategy for proving $f(x)=0$ has a solution consists in finding a number $a$ such that $f(a)<0$ and a number $b$ such that $f(b)>0$, and then using the Intermediate Value Theorem (IVT) with $N=0$.

Let
\[
g(x)=x^2+13x+41,
\]
and so $f(x)=g(x)-\sin x$. We have no techniques for evaluating $\sin x$ without calculator, but we do have all knowledge necessary to evaluate $g(x)$. Indeed, from high school we know that the lowest point of the parabola $g(x)$ is located at $x=-\frac{13}2=-6.5$. Then $g(-6.5)= -1.25$. Therefore
\[
f(-6.5)=g(-6.5)-\sin(-6.5)= g(-6.5)+\sin (6.5)=-1.25+\sin 6.5 \leq -0.25,
\]
where for the very last inequality we use the fact that $\sin 6.5< 1 $ (remember $\sin t\leq 1$ for all real values of $t$).

On the other hand,
\[f(-5)= g(-5)-\sin (-5) = 1+\sin 5> 0\]
as $\sin 5 >-1$ (remember $\sin t\geq -1$ for all real values of $t$). Therefore $f(-5)>0$ and $f(-6.5)<0$ and by the Intermediate Value Theorem (IVT) $f(x)=0$ has a solution in the interval $x\in (-6.5, -5)$.

Proving $f(x)=0$ has a solution in the interval $x\in (-8, -6.5)$ is similar and we leave it to the student.

Below is a computer generated plot of the function with the use of which we can visually verify our answer.

\psset{xunit=1cm, yunit=1cm}
\begin{pspicture}(-9, -5)(1,5)
\psframe*[linecolor=white](-9,-5)(1,5)
\tiny
\psaxes[ticks=none, labels=none]{<->}(0,0)(-9,-4.5)(1,4.5)
\fcLabels{1}{5}
\fcXTickWithLabel{-6.5}{$-6.5$}
\fcXTickWithLabel{-8}{$-8$}
\fcXTickWithLabel{-5}{$-5$}
%Function formula: (x)^{2}+40+13 (x)
\rput(-4.5,-2){$y=x^{2}+13 x+40$}
\psplot[linecolor=grey!30, plotpoints=1000]{-9}{-4}{x 13 mul 40 x 2 exp add add }
\rput(-6.5,3){$y=x^{2}+41+13 x- \sin x$}
\psplot[linecolor=\fcColorGraph, plotpoints=1000]{-9}{-4}{x 57.29578 mul sin -1 mul x 13 mul 41 x 2 exp add add add }

\end{pspicture}
}
\begin{problem}
\begin{problem}(Textbook, page 235, problems 33-38).
Find the horizontal and vertical asymptotes of each curve. Check your work by plotting the function using the internet.
\begin{multicols}{3}
\begin{enumerate}
\item $y=\frac{2x+1}{x-2}$.
\item $y=\frac{x^2+1}{2x^2-3x-2}$.
\item $y=\frac{2x^2+x-1}{x^2+x-2}$.
\item $y=\frac{1+x^4}{x^2-x^4}$.
\item $y=\frac{x^3-x}{x^2-6x+5}$.
\item $y=\frac{x-9}{\sqrt{4x^2+3x+2}}$.
\end{enumerate}
\end{multicols}
\end{problem}
\end{problem}
\solution{\ref{problemAsymptotesy=(2x/(sqrt(x^2+x+3)-3))}
\textbf{Vertical asymptotes.} A function $f(x)$ has a vertical asymptote at $x=a$ if $\lim\limits_{x\to a} f(x)=\pm \infty$. 

The function is algebraic, and therefore has a finite limit at every point it is defined (i.e., no asymptote). Therefore the function can have vertical asymptotes only for those $x$ for which $f(x)$ is not defined. The function is not defined for $\sqrt{x^2+x+3}-3=0$, which has two solutions, $x=2$ and $x=-3$. These are precisely the vertical asymptotes: indeed, 
\[
\lim\limits_{x\to 2^+} \frac{2x}{\sqrt{x^2+x+3}-3}=\infty \quad \quad \quad 
\lim\limits_{x\to 2^-} \frac{2x}{\sqrt{x^2+x+3}-3}=-\infty 
\]
and
\[
\lim\limits_{x\to -3^+} \frac{2x}{\sqrt{x^2+x+3}-3}=\infty \quad \quad \quad 
\lim\limits_{x\to -3^-} \frac{2x}{\sqrt{x^2+x+3}-3}=-\infty 
\]

\textbf{Horizontal asymptotes.} A function $f(x)$ has a horizontal asymptote if $\lim\limits_{x\to \pm\infty} f(x)$ exists. If that limit exists, and is some number, say, $N$, then $y=N$ is the equation of the corresponding asymptote.

Consider the limit $x\to -\infty$. We have that 
\[
\begin{array}{rcll|l}
\displaystyle \lim\limits_{x\to -\infty} \frac{2x}{ \sqrt{x^2+3x+3} - 3}&=&\displaystyle \lim\limits_{x\to - \infty} \frac{ 2}{ \frac{ \sqrt{ x^2 + x+3}}x-\frac3x}\\
&=&\displaystyle  \lim\limits_{x\to - \infty} \frac{2}{-\sqrt{\frac{ x^2 +3x+3}{x^2}}-\frac3x}  && \frac{1}{x} =- \sqrt{\frac{1}{x^2} } \text{ when } x<0\\
&=&\displaystyle \lim\limits_{x\to - \infty} \frac{2}{-\sqrt{1+ \frac{3}{x} + \frac{3}{x^2}}-\frac3x}\\
&=& \displaystyle \frac{\lim\limits_{x\to - \infty} 2}{-\sqrt{ \lim \limits_{x \to - \infty} 1+\lim\limits_{x\to - \infty} \frac{3}{x} + \lim\limits_{x \to - \infty} \frac{3}{x^2}}-\lim\limits_{x\to - \infty} \frac3x}\\
&=&\displaystyle \frac{2}{-\sqrt{1+0+0}-0}\\
&=&\displaystyle -2\quad . 
\end{array}
\]
Therefore $y=-2$ is a horizontal asymptote. 

The case $x\to \infty$, is handled similarly and yields that $y=2$ is a horizontal asymptote.

A computer generated graph confirms our computations.

\psset{xunit=0.2cm, yunit=0.2cm}
\begin{pspicture}(-15, -19.57133)(15,16.540354)
\tiny
\fcAxesStandard{-15}{-19.32133}{15}{16.190354}
\fcXTick{10}
\rput[t](10, -0.6){$10$}
%Function formula: \frac{2 x}{\sqrt{x^{2}+x+3}-3}
\psplot[linecolor=\fcColorGraph, plotpoints=1000]{2.344}{15}{x  2 mul    -3  3 x add   x  2 exp   add    0.5 exp   add   div  }
%Function formula: \frac{2 x}{\sqrt{x^{2}+x+3}-3}
\psplot[linecolor=\fcColorGraph, plotpoints=1000]{-2.6}{1.78}{x  2 mul    -3  3 x add   x  2 exp   add    0.5 exp   add   div  }
%Function formula: \frac{2 x}{\sqrt{x^{2}+x+3}-3}
\psplot[linecolor=\fcColorGraph, plotpoints=1000]{-15}{-3.4}{x  2 mul    -3  3 x add   x  2 exp   add    0.5 exp   add   div  }
\psline[linestyle=dotted](-3,-19.3)(-3,16.1)
\psline[linestyle=dotted](2,-19.3)(2,16.1)
\psline[linestyle=dashed, linecolor=blue](-15, 2)(15, 2)
\psline[linestyle=dashed, linecolor=blue](-15, -2)(15, -2)
\rput[b](-8, 2.6){$y=2$}
\rput[t](8, -2.6){$y=-2$}
\rput[bl](5,5){$y=\frac{2x}{\sqrt{x^2+x+3}-3}$}
\rput[l](2.6,-8){$x=2$}
\rput[r](-3.6,8){$x=-3$}
\end{pspicture}

}


\begin{problem}
Solve each equation for $x$. Using a calculator give an ($\approx$) answer in decimal notation. Using calculator verify your approximate solutions.
\begin{multicols}{2}
\begin{enumerate}[ref={\fcProblemRef}]
\item $e^{7-4x}=7$.

\answer{$\frac{7-\ln 7 }{4}\approx 1.263522 $}
\item $\ln (2x-9)=2$.

\answer{$\frac{e^2+9}{2}\approx 8.194528 $}
\item $\ln (x^2-2)=3$.

\answer{$\pm \sqrt{e^3+2}\approx \pm 4.699525 $}
\item $2^{x-3}=5$.

\answer{$\log_2 5+3= \frac{\ln 5}{\ln 2}+3 \approx 5.321928 $}
\item \label{problemlnx+ln(x-1)=1} $\ln x+\ln (x-1)=1$.

\answer{$\frac{1}{2}\left(1+\sqrt{1+4e}\right)\approx 2.223$}
\item $e^{2x+1}=t$.

\answer{$\frac{\ln t-1}{2}$}
\item $\log_2(m x)=c$.

\answer{$\frac{2^c}{m}$}
\item $e- e^{-2x}=1$.

\answer{$-\frac12\ln (e-1)\approx -0.271$}
\item $8(1+e^{-x})^{-1}=3$.

\answer{$-\ln \frac53 =\ln \frac35 \approx -0.510826 $}
\item $\ln (\ln x)=1$.

\answer{$e^e\approx 15.154$}
\item $e^{e^x}=10$.

\answer{$\ln (\ln 10)\approx 0.834$}
\item $\ln(2x+1)=3-\ln x$.

\answer{$\frac{-1+\sqrt{1+8e^3}}{4}\approx 2.928878 $}
\item $e^{2x}-4e^x+3=0$.

\answer{$x=\ln 3\approx 1.098612, ~~~, x=0$}

\item $e^{4x}+3e^{2x}-4=0$. 

\answer{$x=0$}
\item $e^{2x}-e^x-6=0$.

\answer{$x=\ln 3$}
\item $4^{3x}-2^{3x+2}-5=0$. 

\answer{$x=\frac{\log_{2}5}{3}$}
\end{enumerate}
\end{multicols}


\end{problem}

\solution{\ref{problem2^(x-3)=5}
\[\begin{array}{rcll|l}
\displaystyle 2^{x-3} &=& 5 &&\displaystyle  \text{take } \log_2 \\
x-3&=&\displaystyle  \log_2(5) &&\text{add } 3 \text{ to both sides}\\
x&=&\displaystyle \log_2(5)+3 &&\text{answer is complete} \\
&=&\displaystyle \frac{\ln 5}{\ln 2}+3 && \text{optional step: convert to }\ln\\
&\approx &5.321928095 &&\text{calculator}
\end{array}
\]
}

\solution{ \ref{probleme-e^(-2x)=1}

\[
\begin{array}{rcll|l}
\displaystyle e-e^{-2x}&=&1\\
\displaystyle e^{-2x}&=&e-1&& \text{apply }\ln\\
\displaystyle \ln e^{-2x}&=&\displaystyle \ln(e-1)\\
-2x&=&\displaystyle\ln(e-1)\\
x&=&\displaystyle-\frac{1}{2}\ln(e-1)\\
&\approx& -0.270662427&&\text{calculator}
\end{array}
\]

}

\solution{\ref{problemlnx+ln(x-1)=1} %
\begin{align*}
\ln x + \ln (x-1) & = 1 \\
\ln (x^2-x) & = 1 \\
e^{\ln (x^2-x)} & = e^1 \\
x^2-x & = e \\
x^2-x-e & = 0 \\
\text{Quadratic formula:}\quad x & = \frac{-(-1)\pm \sqrt{(-1)^2-4(1)(-e)}}{2(1)} \\
 & = \frac{1\pm \sqrt{1+4e}}{2}.
\end{align*}
However $\frac{1-\sqrt{1+4e}}{2}$ is negative, so $\ln\left( \frac{1-\sqrt{1 + 4e}}{2} \right)$ is undefined.  
Hence the only solution is $x = \frac{1+\sqrt{1+4e}}{2}\approx 2.2229$.  
}%


\begin{problem}
% begin homework inverse-functions2
Find the inverse function and its domain. 
\begin{enumerate}[ref={\fcProblemRef}]
\item  \label{problemFindInversey=ln(x+3)} $\displaystyle y=\ln (x+3)$.

\answer{$f^{-1}(x)=e^x-3$}
\item  $f(x)=e^{x^3}$.

\answer{$f^{-1}(x)=\sqrt[3]{\ln x}, \quad x>0$}

\item \label{problemFindInversey=(lnx)^2} $\displaystyle y=(\ln x)^2$, $x\geq 1$.

\answer{$f^{-1}(x)=e^{\sqrt{x}}, \quad x\geq 0 $}

\pointsii{5} \label{problemFindInversey=e^x/(1+2e^x)}  $\displaystyle y=\frac{e^x}{1+2e^x}$.

\hiddenanswer{$f^{-1}(x)= \ln \left(\frac{x}{1-2x}\right) $, \quad $x\in \left(0, \frac12\right) $}

\item \label{problemFindInversef=2^(2x)+2^x-2} $f(x)=2^{2x}+2^{x}-2$.

\answer{$f^{-1}(x) =\log_2\frac{-1+\sqrt{9+4x}}{2}, \quad x\geq -2$}
\end{enumerate}
% end homework inverse-functions2

\end{problem}
\solution{\ref{problemFindInversey=ln(x+3)}
\begin{align*}
y & = \ln (x+3) \\
e^y & = e^{\ln (x+3)} \\
e^y & = x + 3 \\
e^y - 3 & = x \\
\text{Therefore} \quad f^{-1}(y) & = e^y - 3.
\end{align*}
The domain of $e^y$ is all real numbers, so the domain of $f^{-1}$ is all real numbers.  
}%

\solution{ \ref{problemFindInversey=(lnx)^2}
\[ 
\begin{array}{rcll|l}y&=&(\ln x)^2 &&\mathrm{take~ \sqrt{~} ~on~both~sides,~ } y\geq 0 \\ \sqrt{y}&=&\ln x&&\mathrm{ ~exponentiate} \\ e^{\sqrt{y}}&=&e^{\ln x}=x \\ f^{-1}(y)&=&e^{\sqrt{y}} \\f^{-1}(x)&=&e^{\sqrt{x}} \end{array}
\]
}

\solution{\ref{problemFindInversey=e^x/(1+2e^x)}
\begin{align*}
y & = \frac{e^x}{1+2e^x} \\
y(1+2e^x) & = e^x \\
y & = e^x(1-2y) \\
\frac{y}{1-2y} & = e^x \\
\ln\frac{y}{1-2y} & = \ln e^x \\
\ln\frac{y}{1-2y} & = x \\
\text{Therefore} \quad f^{-1}(y) & = \ln\frac{y}{1-2y}.
\end{align*}
The natural logarithm function is only defined for positive input values.  
Therefore the domain is the set of all $y$ for which 
\begin{align*}
\frac{y}{1-2y} & > 0.
\end{align*}
This inequality holds if the numerator and denominator are both positive or both negative.  
This happens if either
\begin{enumerate}
\item  $y > 0$ and $y < \frac{1}{2}$, or 
\item  $y < 0$ and $y > \frac{1}{2}$.
\end{enumerate}
The latter option is impossible, so the domain is $\{ y \in \mathbb{R} \ | \ 0 < y < \frac{1}{2}\}$.  
}%

%\vskip 18cm
%\hfill \begin{tabular}{c|c|c|c|c|c|c||c}
%Problem&1 &2&3&4&5&6& $\sum$\\ \hline
%Score&&&&&&&\\ \hline
%Max&17&17&17&17&17&17&102
%\end{tabular} 


\end{document}