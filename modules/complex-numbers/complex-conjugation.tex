\begin{frame}

\begin{columns}
\column{0.2\textwidth}
\begin{pspicture}(-1,-1)(1,1)
\tiny
\fcAxesStandard{-0.4}{-1.5}{1.8}{1.5}%
\fcLabels[$\Re$][$\Im$]{1.8}{1.5}%
\fcFullDot{1}{0.5}%
\rput[lb](1,0.5){$~x+iy$}%
\psline[arrows=->](0,0)(1,0.5)%
\uncover<3->{%
\fcFullDot{1}{-0.5}%
\psline[arrows=->](0,0)(1,-0.5)%
\rput[lt](1,-0.5){$~x-iy$}%
\fcPerpendicular{[1 0.5]}{[0 0] [1 0]}{0.07}%
\fcPerpendicular{[1 -0.5]}{[0 0] [1 0]}{0.07}%
}%
\uncover<13->{\rput[br](0.5, 0.25){$\alertNoH{13}{|z|~}$}}
\uncover<handout:0|13->{%
\psline[linecolor=red, linewidth=2pt](0,0)(1,0.5)
}%
\uncover<handout:0|12->{%
\rput[t](0.5, -0.1){$\alertNoH{12,13}{x}$}
\psline[linecolor=red, linewidth=2pt](0,0)(1,0)
\rput[l](1, 0.25){$~~\alertNoH{12,13}{y}$}
\psline[linecolor=red, linewidth=2pt](1,0)(1,0.5)
}%
\end{pspicture}
\column{0.8\textwidth}
Let $z=x+iy$ be a complex number.
\begin{definition}[Complex conjugation]
We say that $\alert<handout:0|1>{\bar z = \overline{(x+iy)}=x-iy}$ is the \alert<handout:0|1>{\emph{complex conjugate of $z$}}.
\uncover<2->{
The transformation that maps $z$ to $\bar z$ is called \emph{complex conjugation}.
}
\end{definition}
\end{columns}
\uncover<3->{In the complex plane, complex conj. = reflection across real axis.}
\uncover<4->{%
\begin{theorem}
$z \bar z$ is a non-negative real number. $z\bar z$ equals $0$ if and only if $z=0$.
\end{theorem}
}%
\uncover<5->{
\begin{proof}
$z \bar z= \uncover<6->{(x+iy)(x-iy)=}\uncover<7->{x^2\alertNoH{8}{-}(\alertNoH{8}{i}y)^{\alertNoH{8}{2}}=} \uncover<8->{x^2\alertNoH{8}{+}y^2}$ \uncover<8->{is real and non-negative.} \uncover<9->{ $z \bar z=0$ implies $x^2+y^2=0$,} \uncover<10->{which implies $x=y=0$.}
\end{proof}
}
\uncover<11->{
\begin{definition}
The quantity $\alertNoH{11,13}{|z|}=\sqrt{z\bar z}= \sqrt{ {\alertNoH{12}{x}}^2+ { \alertNoH{12}{y }}^2}$ is called the \alertNoH{11}{\emph{absolute} value of $z$}.
\end{definition}
}
\end{frame}
