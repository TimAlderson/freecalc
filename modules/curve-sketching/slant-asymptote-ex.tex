% begin module slant-asymptote-ex
\begin{frame}
\begin{example}
Find the slant asymptote of $f(x) = \frac{2x^3 + x^2 - 1}{x^2 - 2}$.
\begin{columns}[t]
\column{.5\textwidth}
\uncover<2->{%
\[
\begin{array}{r@{}r@{}c@{}r@{}c@{}c@{}c@{}r@{}r@{}}
& & & %
\uncover<4->{\alert<handout:0| 4-6,17>{2x}} & %
\uncover<11->{\alert<handout:0| 11-13,17>{+}} & %
\uncover<11->{\alert<handout:0| 11-13,17>{ 1}} & %
 & %
 & \\%
\cline{3-9}
\alert<handout:0| 3-6,10-13>{x^2} & %
\alert<handout:0| 5-6,12-13>{-2} & %
\Big) & %
\alert<handout:0| 3-4,7-8>{2x^3} & %
\alert<handout:0| 7-8>{+} & %
\alert<handout:0| 7-8>{ x^2} & %
 & %
 & %
\alert<handout:0| 9>{-1} \\%
& & & %
\uncover<6->{\alert<handout:0| 6-8>{2x^3}} & %
&  & %
\uncover<6->{\alert<handout:0| 6-8>{-}} & %
\uncover<6->{\alert<handout:0| 6-8>{4x}} & \\%
%\uncover<7->{\alert<handout:0| 7-8>{%
\cline{4-8}%
%}}%
& & & %
&  & %
\uncover<8->{\alert<handout:0| 8,10-11,14-15>{x^2}} & %
\uncover<8->{\alert<handout:0| 8,14-15>{+}} & %
\uncover<8->{\alert<handout:0| 8,14-15>{4x}} & 
\uncover<9->{\alert<handout:0| 9,14-15>{-1}} \\%
& & & %
&  & %
\uncover<13->{\alert<handout:0| 13-15>{x^2}} & %
 & & %
\uncover<13->{\alert<handout:0| 13-15>{-2}} \\%
%\uncover<14->{\alert<handout:0| 14-15>}%
& & & %
 & %
 & %
 & %
 & %
\uncover<15->{\alert<handout:0| 15,18>{4x}} & %
\uncover<15->{\alert<handout:0| 15,18>{+1}} \\%
\end{array}
\]
}%

\only<handout:0| -4,10-11>{\uncover<3->{%
Divide %
}}%
\only<handout:0| 5-6,12-13>{%
Multiply %
}%
\only<handout:0| 7-8,14-15>{%
Subtract %
}%
\only<handout:0| 3-4>{%
$2x^3$ %
}%
\only<handout:0| 10-11>{%
$x^2$ %
}%
\only<handout:0| 5-6>{%
$2x$ %
}%
\only<handout:0| 12-13>{%
$1$ %
}%
\only<handout:0| 7-8>{%
$2x^3-4x$ %
}%
\only<handout:0| 14-15>{%
$x^2-2$ %
}%
\only<handout:0| -4,10-11>{\uncover<3->{%
by %
}}%
\only<handout:0| 5-6,12-13>{%
by %
}%
\only<handout:0| 7-8,14-15>{%
from %
}%
\only<handout:0| 3-4,10-11>{%
$x^2$ %
}%
\only<handout:0| 5-6,12-13>{%
$x^2-2$ %
}%
\only<handout:0| 7-8>{%
$2x^3+x^2$ %
}%
\only<handout:0| 14-15>{%
$x^2+4x-1$ %
}%
\only<handout:0| 9>{%
Bring down the $-1$%
}%
\invisible<1->{%
y%
}%
\column{.5\textwidth}
%\begin{eqnarray*}
\[
\begin{array}{r@{\ }c@{\ }l}
\uncover<16->{%
\displaystyle \frac{2x^3 + x^2-1}{x^2 - 2} %
}%
& \uncover<16->{ = } & %
\uncover<16->{%
\displaystyle  \alert<handout:0| 17>{2x+1} + \frac{\alert<handout:0| 18>{4x+1}}{x^2 - 2}%
}\\%
\end{array}
\]
\uncover<19->{Therefore}
\[
\begin{array}{r@{\ }c@{\ }l}
&&%
\uncover<19->{%
\displaystyle \lim_{x\to\infty} \left( f(x) - (2x+1)\right)%
}\\%
& \uncover<19->{ = } & %
\uncover<19->{%
\displaystyle \lim_{x\to\infty} \frac{4x+1}{\alert<handout:0| 20>{x^2}-2}\uncover<20->{\alert<handout:0| 20>{\cdot \frac{\frac{1}{x^2}}{\frac{1}{x^2}}}}%
}\\%
& \uncover<21->{ = } & %
\uncover<21->{%
\displaystyle \lim_{x\to\infty} \frac{\frac{4}{x}+\frac{1}{x^2}}{1-\frac{2}{x^2}}%
} \uncover<21->{ = } \uncover<22->{%
\displaystyle \frac{0+0}{1-0} = 0%
}%
\end{array}
\]
%\end{eqnarray*}
\uncover<23->{%
Therefore $y = 2x+1$ is a slant asymptote of $f$.
}%
\end{columns}
\end{example}
\end{frame}
% end module slant-asymptote-ex
