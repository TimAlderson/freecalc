\solution{\ref{problemSketch(x+1)/(x^2+2x+4)}
\textbf{Domain.} As $f$ is a quotient of two polynomials (rational function), its implied domain is all $x$ except those for which we get division by zero for $f$. Consequently the domain of $f$ is all $x$ for which $x ^2+2x+4=0$. However, the polynomial $x^2+2x+4$ has no real roots - its roots are $\displaystyle \frac{-2\pm \sqrt{4-16} }{2}=-1\pm \sqrt{-3}$, and therefore the domain of $f$ is all real numbers. Alternatively, we can complete the square: $x^2+2x+4=(x+1)^2+3$ and so $x^2+2x+4$ is positive for all values of $x$. 

The $y$-intercept of $f$ equals by definition $\displaystyle f(0)= \frac{ 0+ 1}{0^2+2\cdot 0 + 4}=\frac{1}{4}$. The $x$ intercept of $f$ is those values of $x$ for which $f(x)=0$. We compute

\[
\begin{array}{rcl}
\displaystyle f(x)&=&0\\
\displaystyle \frac{x+1}{x^2+2x+4}&=&0\\
x+1&=&0\\
x&=&-1\quad ,
\end{array}
\]
and the $x$-intercept of $f$ is $x=-1$. 

\textbf{Asymptotes.} The line $x=a$ is a vertical asymptote when $\lim\limits_{x\to a^{\pm}}f(x)=\pm \infty$; as $f$ is defined for all real numbers, this implies that there are no vertical asymptotes. 
 
The line $y=L$ is a horizontal asymptote if $\lim\limits_{x\to\pm \infty}f(x)$ exists and equals $L$. We compute:
\[
\lim\limits_{x\to \infty} f(x)=\lim\limits_{x\to \infty} \frac{(x+1)\frac{1}{x^2}}{ (x^2+ 2x +4)\frac{1}{x^2}} = \lim\limits_{x\to \infty}\frac{\frac{1}{x}+\frac{1}{x^2}}{ 1+ \frac{2}{x} +\frac{4}{x^2}}=\frac{0+0}{1+0+0}=0
\]
Therefore $y=0$ is a horizontal asymptote for $f$. An analogous computation shows that $\lim\limits_{x\to\pm \infty}f(x)=0$ and so $y=0$ is the only horizontal asymptote of $f$.

\textbf{Intervals of increase and decrease.} 
The intervals of increase and decrease of $f$ are governed by the sign of $f'$. We compute:
\[
\begin{array}{rcll|l}
f'(x)&=&\displaystyle \left(\frac{x+1}{x^2+2x+4}\right)' &&\text{qutotient rule}\\
&=&\displaystyle \frac{(x+1)'\left(x^2+ 2x+4\right)- (x+1)\left( x^2 +2 x+4\right)'}{\left(x^2+2x+4 \right)^2}\\
&=&\displaystyle\frac{ x^2+2x+4-(x+1)(2x+2)}{\left(x^2+2x+4 \right)^2}\\
&=&\displaystyle \frac{x^2+2x+4-\left( 2x^2+ 4x+ 2 \right)}{ \left( x^2 +2x+4 \right)^2}\\
&=&\displaystyle \frac{-x^2-2x+2}{\left(x^2+2x+4 \right)^2}
\end{array}
\] 
As $x^2+2x+4$ is positive, the sign of $f'$ is governed by the sign of $-x^2+2x+2$. To find out where $-x^2+2x+2$ changes sign, we compute the zeroes of this expression:
\[\begin{array}{rcll|l}
-x^2-2x+2&=&0\\
x^2+2x-2&=&0 &&\text{use the quadratic formula}\\
x_1, x_2&=& -1\pm \sqrt{3}\quad .
\end{array}
\]
Therefore the quadratic $-x^2+2x+2$ factors as 
\begin{equation}
\label{eq1problemSketch(x+1)/(x^2+2x+4)}
-(x-x_1)(x-x_2)=-\left(x-\left(-1-\sqrt{3}\right)\right)\left(x-\left(-1+\sqrt{3}\right)\right)
\end{equation} 
The points $x_1, x_2$ split the real line into three intervals: $(-\infty, -1-\sqrt{3})$, $(-1-\sqrt{3}, -1+\sqrt{3})$ and $(-1+ \sqrt{3}, \infty )$, and each of the factors of \eqref{eq1problemSketch(x+1)/(x^2+2x+4)} has constant sign inside each of the intervals. If we chooose $x$ to be a very negative number, it follows that $-(x-x_1)(x-x_2)$ is a negative, and therefore $ f'(x)$ is negative for $x\in(-\infty, -1-\sqrt{3})$. For $x\in (-1-\sqrt{3}, -1+\sqrt{3})$, exactly one factor of $f'$ changes sign and therefore $f'(x)$ is positive in that interval; finally only one factor of $f'(x)$ changes sign in the last interval so $f'(x)$ is negative on $(-1+ \sqrt{3}, \infty )$.

Our computations can be summarized in the following table. 

\begin{tabular}{|lll|}\hline
Interval & $f'(x)$ & $f(x)$   \\\hline
$(-\infty, -1-\sqrt{3})$ & $-$& $\searrow $ \\\hline
$(-1-\sqrt{3}, -1+\sqrt{3})$ &$+$&$\nearrow$\\\hline
$( -1+\sqrt{3}, \infty)$&$-$&$\searrow$ \\\hline
\end{tabular}

\textbf{Local and global minima and maxima. } The table above shows that $f(x)$ changes from decreasing to increasing at $x=x_1=-1-\sqrt{3}$ and therefore $f$ has a local minimum at that point. The table also shows that $f(x)$ changes from increasing to decreasing at $ x=x_2=-1+\sqrt{3}$ and therefore $f$ has a local maximum at that point. The so found local maximum and local minimum turn out to be global: indeed, no other finite point is critical and thus cannot be maximum or minimum; on the other hand $\lim\limits_{x\to\pm \infty}f(x)=1$ and this implies that all $x$ sufficiently far away from $x=0$ have that $f(x)$ is close to $0$, and therefore $f(x)$ is larger than $f(x_1)$ and smaller than $f(x_2)$ for all $x$.

\textbf{Intervals of concavity. } 
The intervals of concavity of $f$ are governed by the sign of $f''$. The second derivative of $f$ is:
\[
\begin{array}{rcll|l}
f''(x)&=&\displaystyle (f'(x))'= \left(\frac{-x^2-2x+2}{\left(x^2+2x+4 \right)^2}\right)'\\
&=&\displaystyle (-x^2-2x+2)'\left(\frac{1}{(x^2+2x+4)^2}\right)+(-x^2-2x+2)\left(\frac{1}{(x^2+2x+4)^2}\right)' &&\begin{array}{l}\text{use chain rule }\\\text{for second differentiation}\end{array}\\
&=&\displaystyle (-2x-2)\left(\frac{1}{(x^2+2x+4)^2}\right)+(-x^2-2x+2)(-2)\frac{(x^2+2x+4)'}{(x^2+2x+4)^{3}}\\
&=&\displaystyle -(2x+2)\left(\frac{1}{(x^2+2x+4)^2}\right) +(2x^2+4x-4)\frac{(2x+2)}{(x^2+2x+4)^{3}}&&\text{factor out }\frac{(2x+2)}{(x^2+2x+4)^2}\\
&=&\displaystyle \frac{(2x+2)}{(x^2+2x+4)^2}\left(-1+\frac{(2x^2+4x-4)}{(x^2+2x+4)}\right)\\
&=&\displaystyle \frac{(2x+2)}{(x^2+2x+4)^2}\left(\frac{-(x^2+2x+4)+(2x^2+4x-4)}{(x^2+2x+4)} \right)\\
&=&\displaystyle \frac{(2x+2)(x^{2}+2 x-8)}{(x^2+2x+4)^3}&& \text{factor } (x^2+2x-8)\\
&=&\displaystyle \frac{(2x+2)(x+4)(x-2)}{(x^2+2x+4)^3}
\end{array}
\]
As we previously established, the denominator of the above expression is always positive. Therefore the expression above changes sign when the terms in the numerator change sign, namely, at $x=-1$, $x=-4$ and $x=2$. 

Our computations can be summarized in the following table. 

\begin{tabular}{|lll|}\hline
Interval & $f''(x)$ & $f(x)$   \\\hline
$(-\infty, -4)$ & $-$& $\cap$ \\\hline
$(-4, -1)$ &$+$&$\cup$\\\hline
$(-1, 2)$&$-$&$\cap$ \\\hline
$(2, \infty)$&$+$&$\cup$ \\\hline
\end{tabular}

\textbf{Points of inflection.} The preceding table shows that $f''(x)$ changes sign at $-4, -1, 2$ and therefore the points of inflection are located at $x=-4, x=-1$ and $x=2$, i.e., the points of inflection are $\left(-4, -\frac{1}{4}\right)$, $\left(-1, 0\right)$, $\left(2, \frac{1}{4}\right)$.

}