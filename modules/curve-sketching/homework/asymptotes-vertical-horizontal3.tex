Find the horizontal and vertical asymptotes of the curve
\begin{enumerate}
\item $y=\frac{2x}{\sqrt{x^2+x+3}-3}$. \answer{Vertical: $x=2, x=-3$, horizontal: $y=2, y=-2$}
\solution{
\textbf{Vertical asymptotes.} A function $f(x)$ has a vertical asymptote at $x=a$ if $\lim_{x\to a} f(x)=\pm \infty$. 

The function is algebraic, and therefore, if it is defined, has a finite limit (i.e., no asymptote). Therefore the function can have vertical asymptotes only for those $x$ for which $f(x)$ is not defined. The function is not defined for $\sqrt{x^2+x+3}-3=0$, which has two solutions, $x=2$ and $x=-3$. These are precisely the vertical asymptotes: indeed, 
\[
\lim_{x\to 2^+} \frac{2x}{\sqrt{x^2+x+3}-3}=\infty \quad \quad \quad 
\lim_{x\to 2^-} \frac{2x}{\sqrt{x^2+x+3}-3}=-\infty 
\]
and
\[
\lim_{x\to -3^+} \frac{2x}{\sqrt{x^2+x+3}-3}=\infty \quad \quad \quad 
\lim_{x\to -3^-} \frac{2x}{\sqrt{x^2+x+3}-3}=-\infty 
\]

\textbf{Horizontal asymptotes.} A function $f(x)$ has a horizontal asymptote if $\lim_{x\to \pm\infty} f(x)$ exists. If that limit exists, and is some number, say, $N$, then $y=N$ is the equation of the corresponding asymptote.

Consider the limit $$x\to -\infty$$. We have that 
\[
\lim_{x\to -\infty} \frac{2x}{\sqrt{x^2+3x+3}-3}= \lim_{x\to \pm \infty} \frac{2}{\frac{\sqrt{x^2+x+3}}x-\frac3x}=\lim_{x\to \pm \infty} \frac{2}{-\sqrt{\frac{x^2+3x+3}{x^2}}-\frac3x} =-2\quad . 
\]
Therefore $y=-2$ is a horizontal asymptote. 

The case $x\to \infty$, is handled similarly and yields that $y=2$ is a horizontal asymptote.
}

\item $y=\frac{3x^2}{\sqrt{x^2+2x+10}-5}$. \answer{Vertical: $x=3, x=-5$, horizontal: none.}
\end{enumerate}