\solution{\ref{problemAsymptotesy=(2x/(sqrt(x^2+x+3)-3))}
\textbf{Vertical asymptotes.} A function $f(x)$ has a vertical asymptote at $x=a$ if $\lim\limits_{x\to a} f(x)=\pm \infty$. 

The function is algebraic, and therefore has a finite limit at every point it is defined (i.e., no asymptote). Therefore the function can have vertical asymptotes only for those $x$ for which $f(x)$ is not defined. The function is not defined for $\sqrt{x^2+x+3}-3=0$, which has two solutions, $x=2$ and $x=-3$. These are precisely the vertical asymptotes: indeed, 
\[
\lim\limits_{x\to 2^+} \frac{2x}{\sqrt{x^2+x+3}-3}=\infty \quad \quad \quad 
\lim\limits_{x\to 2^-} \frac{2x}{\sqrt{x^2+x+3}-3}=-\infty 
\]
and
\[
\lim\limits_{x\to -3^+} \frac{2x}{\sqrt{x^2+x+3}-3}=\infty \quad \quad \quad 
\lim\limits_{x\to -3^-} \frac{2x}{\sqrt{x^2+x+3}-3}=-\infty 
\]

\textbf{Horizontal asymptotes.} A function $f(x)$ has a horizontal asymptote if $\lim\limits_{x\to \pm\infty} f(x)$ exists. If that limit exists, and is some number, say, $N$, then $y=N$ is the equation of the corresponding asymptote.

Consider the limit $x\to -\infty$. We have that 
\[
\begin{array}{rcll|l}
\displaystyle \lim\limits_{x\to -\infty} \frac{2x}{ \sqrt{x^2+3x+3} - 3}&=&\displaystyle \lim\limits_{x\to - \infty} \frac{ 2}{ \frac{ \sqrt{ x^2 + x+3}}x-\frac3x}\\
&=&\displaystyle  \lim\limits_{x\to - \infty} \frac{2}{-\sqrt{\frac{ x^2 +3x+3}{x^2}}-\frac3x}  && \frac{1}{x} =- \sqrt{\frac{1}{x^2} } \text{ when } x<0\\
&=&\displaystyle \lim\limits_{x\to - \infty} \frac{2}{-\sqrt{1+ \frac{3}{x} + \frac{3}{x^2}}-\frac3x}\\
&=& \displaystyle \frac{\lim\limits_{x\to - \infty} 2}{-\sqrt{ \lim \limits_{x \to - \infty} 1+\lim\limits_{x\to - \infty} \frac{3}{x} + \lim\limits_{x \to - \infty} \frac{3}{x^2}}-\lim\limits_{x\to - \infty} \frac3x}\\
&=&\displaystyle \frac{2}{-\sqrt{1+0+0}-0}\\
&=&\displaystyle -2\quad . 
\end{array}
\]
Therefore $y=-2$ is a horizontal asymptote. 

The case $x\to \infty$, is handled similarly and yields that $y=2$ is a horizontal asymptote.

A computer generated graph confirms our computations.

\psset{xunit=0.2cm, yunit=0.2cm}
\begin{pspicture}(-16, -20)(16,17)
\tiny
\fcAxesStandard{-15}{-19.32133}{15}{16.190354}
\fcXTick{10}
\rput[t](10, -0.6){$10$}
%Function formula: \frac{2 x}{\sqrt{x^{2}+x+3}-3}
\psplot[linecolor=\fcColorGraph, plotpoints=1000]{2.344}{15}{x  2 mul    -3  3 x add   x  2 exp   add    0.5 exp   add   div  }
%Function formula: \frac{2 x}{\sqrt{x^{2}+x+3}-3}
\psplot[linecolor=\fcColorGraph, plotpoints=1000]{-2.6}{1.78}{x  2 mul    -3  3 x add   x  2 exp   add    0.5 exp   add   div  }
%Function formula: \frac{2 x}{\sqrt{x^{2}+x+3}-3}
\psplot[linecolor=\fcColorGraph, plotpoints=1000]{-15}{-3.4}{x  2 mul    -3  3 x add   x  2 exp   add    0.5 exp   add   div  }
\psline[linestyle=dotted](-3,-19.3)(-3,16.1)
\psline[linestyle=dotted](2,-19.3)(2,16.1)
\psline[linestyle=dashed, linecolor=blue](-15, 2)(15, 2)
\psline[linestyle=dashed, linecolor=blue](-15, -2)(15, -2)
\rput[b](-8, 2.6){$y=2$}
\rput[t](8, -2.6){$y=-2$}
\rput[bl](5,5){$y=\frac{2x}{\sqrt{x^2+x+3}-3}$}
\rput[l](2.6,-8){$x=2$}
\rput[r](-3.6,8){$x=-3$}
\end{pspicture}

}

\solution{\ref{problemAsymptotesy=x/(sqrt(x^2+3) -2x)}

\textbf{Vertical asymptotes.} A function $f(x)$ has a vertical asymptote at $x=a$ if $\lim\limits_{x\to a} f(x)=\pm \infty$. 

The function is algebraic, and therefore has a finite limit at every point it is defined (i.e., no asymptote). Therefore the function can have vertical asymptotes only for those $x$ for which $f(x)$ is not defined. The function is not defined for 

\[
\begin{array}{rcll|l}
\sqrt{x^2+3}-2x&=&0\\
\sqrt{x^2+3}&=&2x&&\begin{array}{l} \text{square both sides}\\\text{may introduce extraneous solutions} \end{array}\\
x^2+3&=&4x^2\\
3x^2-3&=&0\\
3(x-1)(x+1)&=&0\\
x=1 \quad &\text{or}& \cancel{ x=-1}\\
&&x=-1 \text{ is extraneous:}\\
&& \sqrt{(-1)^2+3}-(-1)2=4\neq 0
\end{array}
\]

$x=-1$ is indeed a vertical asymptote:
\[
\lim\limits_{x\to 1^+}  \frac{x}{\sqrt{x^2+3} -2x}=\infty \quad \quad \quad 
\lim\limits_{x\to 1^-}  \frac{x}{\sqrt{x^2+3} -2x}=-\infty .
\]
\textbf{Horizontal asymptotes.} 
\[
\begin{array}{rcll|l}
\displaystyle \lim\limits_{x\to -\infty}  \frac{x}{\sqrt{x^2+3} -2x}&=&\displaystyle \lim\limits_{x\to - \infty} \frac{1}{\frac{\sqrt{x^2+3}}{x} -2} \\
&=& \displaystyle \lim\limits_{x\to - \infty} \frac{1}{-\sqrt{\frac{x^2+3}{x^2}} -2}   && \frac{1}{x} =- \sqrt{\frac{1}{x^2} } \text{ when } x<0\\
&=&\displaystyle \lim\limits_{x\to - \infty} \frac{1}{-\sqrt{1+\frac{3}{x^2}} -2}  \\
&=& \displaystyle \frac{1}{-\sqrt{1+0}-2}\\
&=&\displaystyle -\frac{1}{3}.\\
\displaystyle \lim\limits_{x\to -\infty}  \frac{x}{\sqrt{x^2+3} -2x}&=&\displaystyle \lim\limits_{x\to  \infty} \frac{1}{\frac{\sqrt{x^2+3}}{x} -2} \\
&=& \displaystyle \lim\limits_{x\to  \infty} \frac{1}{\sqrt{\frac{x^2+3}{x^2}} -2}   && \frac{1}{x} = \sqrt{\frac{1}{x^2} } \text{ when } x>0\\
&=&\displaystyle \lim\limits_{x\to  \infty} \frac{1}{\sqrt{1+\frac{3}{x^2}} -2}  \\
&=& \displaystyle \frac{1}{\sqrt{1+0}-2}\\
&=&\displaystyle -1.\\
\end{array}
\]
Therefore $y=-\frac{1}{3}$ and $y=-1$ are the two horizontal asymptotes. 


A computer generated graph confirms our computations.

\psset{xunit=0.2cm, yunit=0.2cm}
\begin{pspicture}(-16, -20)(16,17)
\tiny
\fcAxesStandard{-15}{-19.32133}{15}{16.190354}
\fcXTick{10}
\rput[t](10, -0.6){$10$}
\newcommand{\theFun}{x x x mul 3 add sqrt -2 x mul add div}
%Function formula: \frac{2 x}{\sqrt{x^{2}+x+3}-3}
\psplot[linecolor=\fcColorGraph, plotpoints=1000]{1.036}{15}{\theFun }
%Function formula: \frac{2 x}{\sqrt{x^{2}+x+3}-3}
\psplot[linecolor=\fcColorGraph, plotpoints=1000]{-15}{0.961}{\theFun }
\psline[linestyle=dotted](1,-19.3)(1,16.1)
\psline[linestyle=dashed, linecolor=blue](! -15 -1  3 div)(!15 -1 3 div)
\psline[linestyle=dashed, linecolor=blue](-15, -1)(15, -1)
\rput[b](-8, 0.2){$y=-\frac{1}{3}$}
\rput[t](8, -2){$y=-1$}
\rput[bl](5,5){$y=\frac{x}{\sqrt{x^2+3} -2x}$}
\rput[l](1.6,-8){$x=1$}
\end{pspicture}


}

