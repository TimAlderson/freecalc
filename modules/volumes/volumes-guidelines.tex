% begin module volumes-guidelines
\begin{frame}
\begin{center}
\begin{tabular}{|l||c|c|}
\hline
&  \multicolumn{2}{|c|}{Rotate about $\ldots$}\\
\cline{2-3}
& $\ldots$ a horizontal line & $\ldots$ a vertical line \\
\hline
\hline
$y$ is a & Cross-sections & Cylindrical shells \\
function of $x$ & $\displaystyle \int \ \cdot \ \diff x$ & $\displaystyle \int \ \cdot \ \diff x$ \\
\hline
$x$ is a & Cylindrical shells & Cross-sections \\
function of $y$ & $\displaystyle \int \ \cdot \ \diff y$ & $\displaystyle \int \ \cdot \ \diff y$ \\
\hline
\end{tabular}
\end{center}

\begin{itemize}
\item  $\displaystyle \int \ \cdot \ \diff x$ means integrate with respect to $x$.
\item  $\displaystyle \int \ \cdot \ \diff y$ means integrate with respect to $y$.
\item  Some equations express $y$ as a function of $x$ and $x$ as a function of $y$.  In such cases, you may use either method.
%\item  If you are rotating the region bounded by two functions, first find the volume of the region obtained by rotating the inner function (the one closer to the axis of rotation), then subtract it from the volume obtained by rotating the outer function (the one further from the axis of rotation).
\end{itemize}
\end{frame}
% end module volumes-guidelines
