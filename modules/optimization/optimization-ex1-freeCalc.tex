\begin{frame}
\begin{example}
A cone is folded from a wedge-shaped profile of radius $r$. Find the maximal possible volume $V$ of such a cone.

\begin{columns}[c]
\column{.27\textwidth}
%Please note: the below pictures are up to scale. The cone is an isogonal projection onto the tangent plane to the sphere centered at the origin and passing through point (0, 2, 0.4). The function used to draw the cone is 
%the vector partition calculator function plotConeUsualProjection(3/4, \sqrt(1-DoubleValue{}(3/4)^2), 2, 0.4)
\psset{xunit=1.5cm, yunit=1.5cm}
\begin{pspicture}(-1.1, -1.05)(1.1,1) 
\psframe*[linecolor=white](-1.1, -1.05)(1.1,1) 
\tiny 
\pscustom*[linecolor=cyan!30]{ \psparametricplot[algebraic] {2.35619}{7.06858} {0+1*cos(t)| 0+1*sin(t)} \psline(0.707107, 0.707107)(0, 0)(-0.707107, 0.707107)}

\psparametricplot[algebraic,linecolor=blue]{2.35619}{7.06858}{cos(t)| sin(t)} 
\psline[linecolor=red](0.707107, 0.707107)(0, 0)(-0.707107, 0.707107)

\rput[t](0.35, 0.30){$r$}
\rput[lb](0.8,0.8){$B$}
\rput[rb](-0.8,0.8){$A$}
\rput[b](0,0.1){$O$}

\uncover<7->{
\psparametricplot[algebraic,linecolor=purple]{2.35619}{7.06858}{0.07*cos(t)| 0.07*sin(t)} 
\rput[t](0, -0.1){$\alert<7>{\theta}$}
}
\uncover<8->{
\psparametricplot[linewidth=1pt, algebraic,arrows=<->, linecolor =blue] {4.41238898} {5.01238898}{cos(t)| sin(t)} 
}
\uncover<9->{
\rput[b](0, -0.9){\alert<9>{$r\theta$}}
}
\psline[linecolor=red!1](-1.03,-1.03)(-1.03,-1.02)
\end{pspicture} 

\psset{xunit=1.5cm, yunit=1.5cm}
\begin{pspicture}(-1.1, -0.2)(1.1,1)
\psframe*[linecolor=white](-1.1, -0.2)(1.1,1) 
\tiny 
\rput[b](0,0.7){$O$}
\rput[l](0.8, 0.0333562){$A$}

\psline[linecolor=\psColorGraph](0.73046, 0.0333562)(0, 0.648593)
\psline[linecolor=\psColorGraph](-0.73046, 0.0333562)(0, 0.648593)
\psparametricplot[algebraic,linecolor=blue]{-3.37036}{0.228769}{0.75*cos(t) |0.147087*sin(t)}
\psparametricplot[algebraic, linestyle=dashed, linecolor=blue] {0.228769}{2.91282}{0.75*cos(t) |0.147087*sin(t)}
\rput[bl](0.38,0.34 ){$r$}

\uncover<10,11>{
\psparametricplot[algebraic,linecolor=red]{-3.37036}{0.228769}{0.75*cos(t) |0.147087*sin(t)}
\psparametricplot[algebraic, linestyle=dashed, linecolor=red] {0.228769}{2.91282}{0.75*cos(t) |0.147087*sin(t)}
\rput[bl](0.38,0.34 ){$r$}
}

\psline[linecolor=red!1](-1, 0)(-0.99,0)
\psline[linecolor=red!1](0.99, 0)(1,0)
\uncover<2->{
\psline[linecolor=black](0,0)(0,0.648593)
\rput[r](-0.07,0.32){$h$}
}
\uncover<3->{
\psline[linecolor=black](0,0)(0.73046, 0.0333562)
\psline[linecolor=black](0.073046, 0.00333562) (0.073046, 0.0764577) (0, 0.0731221)
\rput[t](0.35, -0.02){$t$}
}
\end{pspicture} 

\psset{xunit=1.5cm, yunit=1.5cm}
\begin{pspicture}(-1.1, -0.1)(1.1,1) 
\psframe*[linecolor=white](-1.1, -0.2)(1.1,1) 
\tiny 
\uncover<13->{
\rput[b](0,0.7){$O$}
\rput[l](0.8, 0){$A$}
\rput[r](-0.05,0.32){\alert<13,14>{$h$}}
\rput[t](0.38,-0.05){$\alert<14>{t}$}
\rput[bl](0.38,0.34 ){$\alert<14>{r}$}
\psline(0,0)(0.75,0)
\psline[linecolor=\psColorGraph](0.75,0)(0, 0.648593)
\psline(0, 0.648593)(0,0)
\psline(0.075, 0)(0.075, 0.075)(0, 0.075)
}
\psline[linecolor=red!1](-1, -0.147087)(-0.99,-0.147087)
\psline[linecolor=red!1](0.99, 0.8)(1,0.8)
\end{pspicture} 

\vspace{1cm}
\column{.73\textwidth}
\only<1-20>{
\uncover<2->{
Set $h$ - cone height,} \uncover<3->{$t$ - cone radius.} \uncover<4->{Then $\alert<4,17>{V=} \uncover<5->{\alert<5>{\frac{1}3 (\alert<6>{\text{area cone base}})h} }\uncover<6->{=\alert<17>{ \frac13 \alert<6>{\pi   t^2} h }}$.} \uncover<7->{ Let $\alert<7>{\theta}$ - angle of the wedge.} \uncover<8->{Then $ \alert<8,9>{\text{arc}{AB}=} \uncover<9->{\alert<9,12>{r\theta}}$ \uncover<10->{\alert<10,11>{= perimeter cone base =}} $\uncover<11->{\alert<11,12>{2\pi t}.}$} \uncover<12->{Therefore $\alert<12,15,18>{t=\frac{r\theta}{2\pi}}$.} \uncover<13->{Then 

$\displaystyle
\alert<13,14,19>{h=}  \uncover<14->{ \alert<14>{ \sqrt{ r^2- \alert<15>{t}^2 }}} \uncover<15->{= \sqrt{r^2- \left(\alert<15>{ \frac{r\theta}{ 2\pi}}\right)^2}}\uncover<16->{=\alert<19>{\frac{r}{2\pi }\sqrt{ 4\pi^2-\theta^2 }},}
$
}%13 

\uncover<17->{
and therefore 

$
\begin{array}{rcl}
\alert<17>{V}&\alert<17>{=}&\displaystyle\alert<17>{ \frac13\pi \alert<18>{t}^2 \alert<19>{h}}= \uncover<18->{\frac13\pi \left(\alert<18>{\frac{r\theta}{2\pi}}\right)^2\alert<19>{\frac{r}{2\pi}\sqrt{4\pi^2-\theta^2}}}\\ 
\uncover<20->{&=& \displaystyle \frac{r^3}{24\pi^2} \theta^2\sqrt{4\pi^2-\theta^2}\quad . }
\end{array}
$
}
}
\only<21-24>{
\noindent We reduced the problem to: find the maximum of

$
V=\displaystyle \frac{r^3}{24\pi^2} \theta^2\sqrt{4\pi^2-\theta^2},\quad \quad  \uncover<22->{\alert<22,23>{\uncover<23->{0} \leq \theta \leq \uncover<23->{2\pi}}} 
$

as function of $\theta$ (using the \alert<22, 23>{closed interval} method).

\uncover<24->{We need to find the critical points of $V$, i.e., the values of $\theta$ for which $\frac{dV}{d\theta}=0$ and the values of $\theta$ for which  $\frac{dV}{d\theta}$ is not defined.}
}
\only<25-35>{
\noindent $
\begin{array}{rcl}
V&=&\displaystyle \frac{r^3}{24\pi^2} \theta^2\sqrt{4\pi^2-\theta^2}, \quad \quad \quad 0\leq \theta \leq2\pi \\
\displaystyle \uncover<26->{\displaystyle\frac{dV}{d\theta} &=&} \displaystyle \uncover<27->{\phantom{+}\left(\frac{r^3}{24\pi^2}\right)   \alert<28,29>{\frac{d}{d\theta}\left(\theta^2\right)} \sqrt{4\pi^2-\theta^2}}\\
&&\uncover<27->{\displaystyle+\left(\frac{r^3}{24\pi^2}\right)\theta^2\alert<30,31>{\frac{d}{ d\theta}\left(\sqrt{4\pi^2-\theta^2} \right)}} \\
\uncover<28->{&=& \displaystyle \phantom{+}\left(\frac{r^3}{24\pi^2}\right) \alert<28,29>{( \uncover<29->{\alert<29,33>{2\theta}})} \alert<33>{\sqrt{4\pi^2-\theta^2}}}\\
&&\displaystyle\uncover<28->{ +\left(\frac{r^3}{24\pi^2}\right) \alert<34>{\theta^2} \alert<30,31,34>{\left(\uncover<31->{ \frac{1}{2} \frac{ \frac{d}{d\theta} (-\theta^2)}{\sqrt{4\pi^2-\theta^2}}}\right)}  } \\
\uncover<32->{&=&\displaystyle\left(\frac{r^3}{24\pi^2}\right)\frac{ \alert<33>{2\theta(4\pi^2-\theta^2)}\alert<34>{-\theta^3} }{\alert<33,34>{ \sqrt{4\pi^2-\theta^2}}}}\\
\uncover<35->{&=&\displaystyle\left(\frac{r^3}{24\pi^2}\right)\frac{8 \theta \pi^2-3\theta^3 }{\sqrt{4\pi^2-\theta^2}}}
\end{array}
$
}
\only<36-44>{
$
\begin{array}{rcl}
V&=&\displaystyle \frac{r^3}{24\pi^2} \theta^2\sqrt{4\pi^2-\theta^2}, \quad \quad \quad \alert<44>{0\leq \theta \leq2\pi} \\
\displaystyle \frac{dV}{d\theta}&=& \displaystyle\left(\frac{r^3 } {24\pi^2} \right)\frac{\alert<38>{8\theta\pi^2-3\theta^3} }{\sqrt{\alert<43>{4\pi^2-\theta^2}}}
\end{array}
$

\uncover<37->{We have that $\frac{dV}{d\theta}=0$ when }
$
\begin{array}{rcl}
\uncover<38->{ \alert<38>{8\theta\pi^2-3\theta^3}&=&0}\\
\uncover<39->{\theta(8\pi^2-3\theta^2)&=&0}\\
\uncover<40->{-3\alert<41>{\theta}\alert<42>{\left(\theta-\sqrt{\frac{8}{3}}\pi \right)} \alert<44>{\left(\theta+\sqrt{\frac{8}{3}}\pi \right)}&=&0.}
\end{array}
$

\uncover<41->{Therefore $\theta$ is critical point for $V$ if $\alert<41>{\theta= 0}$, $\alert<42>{\theta=\sqrt{\frac83}\pi }$, or \alert<43>{$\theta=2\pi$}} \uncover<44->{(note $\alert<44>{\theta=-\sqrt{\frac83}\pi}$ is outside of the domain of $V$).} \uncover<45->{For $\theta=0,2\pi$ the volume $V$ is $0$, so the maximum 
volume is attained at $\theta=\sqrt{\frac83}\pi$.}
} %frame36
\only<46->{
\[
V(\alert<47>{\theta} )=\displaystyle \frac{r^3}{24\pi^2} \alert<47>{\theta}^2 \sqrt{4 \pi^2 - \alert<47>{\theta}^2} 
\]
Finally, the answer to the problem is
$
\begin{array}{rcl}
V_{max}&=&\displaystyle V\left(\alert<47>{ \sqrt{\frac83} \pi} \right)\\
\uncover<47->{ &=&\displaystyle \frac{r^3}{\alert<48>{24}\alert<49>{\pi^2}} \left( \alert<47>{ \alert<48>{\sqrt{\frac83}} \alert<49>{\pi}} \right)^{\alert<48,49>{2}} \sqrt{4 \alert<49>{\pi^2} -\left( \alert<47>{ \sqrt{\frac83} \alert<49>{\pi}} \right)^{\alert<49>{2}}}}\\
\uncover<48->{ &=&\displaystyle \frac{r^3}{\alert<48>{9}} \alert<49>{\pi} \sqrt{\alert<50>{4-\frac{8}3}}}\\
\uncover<50->{&=& \displaystyle \pi\frac{r^3}9 \sqrt{ \alert<50>{\frac43}}} \uncover<51>{=\frac{2\pi r^3}{9\sqrt3}}
\end{array}
$
}

\end{columns}
\uncover<3>{}
\end{example}
\end{frame}