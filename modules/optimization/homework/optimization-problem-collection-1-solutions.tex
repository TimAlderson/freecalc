\solution{\ref{problemponthyperbolax^2-4y^2closestTo1,1}

The distance function between an arbitrary point $(x,y)$ and the point $(1,0)$ is $d=\sqrt{(x-1)^2+(y-0)^2}$. On the other hand, when the point $(x,y)$ lies on the hyperbola we have $y^2= \frac{x^2 -16 }{4 }$. In this way, the problem becomes that of minimizing the distance function

\[
dist(x)=\sqrt{(x-1)^2+y^2}=\sqrt{(x-1)^2+\frac{x^2-16}{4}} \quad .
\]
This is a standard optimization problem: we need to find the critical endpoints, i.e., the points where $dist'=0$. As the square root function is an increasing function, the function $\displaystyle \sqrt{(x-1)^2+\frac{x^2-16}{4}}$ achieves its minimum when the function 
\[
l=dist^2=(x-1)^2+\frac{x^2-16}{4}\quad 
\]
does. $l$ is a quadratic function of $x$ and we can directly determine its minimimum via elementary methods. Alternatively, we find the critical points of $l$:
\[
\begin{array}{rcl}
\displaystyle l'&=&\displaystyle 0\\
\displaystyle 2(x-1)+\frac{x}{2} &=&\displaystyle 0\\
\displaystyle \frac{5}{2}x-2&=&0\\
\displaystyle x&=&\displaystyle \frac{4}{5}\quad .
\end{array}
\]
On the other hand, $x^2=16+4y^2$ and therefore $|x|\geq \sqrt{16} = 4$. Therefore $x\in (-\infty, -4]\cup [4,\infty)$. As $x= \frac{4}{5 }$ is outside of the allowed range, it follows that our either function attains its minimum at one of the endpoints $\pm 4$ or the function has no minimum at all. It is clear however that as $x$ tends to  $\infty$, so does $dist$. Therefore $dist$ attains its maximum for $x=4$ or $-4$ and $y=\pm\sqrt{(\pm4)^2-16}=0$. Direct check shows that $dist_{|x=4} =\sqrt{(4-1)^2 +\frac{4^2- 16}{4 }}=3$ and $dist_{|x=-4}=\sqrt{(-4-1)^2+\frac{4^2-16}{4}}=5$  so our function $dist$ has a minimal value of $3$ achieved when $x=4$, which is our final answer. Notice that this answer can be immediately given without computation by looking at the figure drawn for \ref{problemponthyperbolax^2-4y^2closestTo1,1}. Indeed, it is clear that there are no points from the hyperbola lying inside the dotted circle centered at $(1,0)$. Therefore the point where this circle touches the hyperbola must have the shortest distance to the center of the circle.
}

\solution{\ref{problemMaxVolumeBoxFixedAreaDoubleBottomNoLid}
Let $B$ denote the area of the base of the box, equal to the area of the top. Let $W$ denote the area of the four walls of the box (the four walls are all equal because the base of the box is a square). Then the surface area $S$ of material used will be 
\[
S=\underbrace{2B}_{\text{two pieces for the bottom}}+\underbrace{4W}_{4 \text{ walls}} +\underbrace{B}_{\text{the box lid}}=3B+4W\quad.
\] 
Let $x$ denote the length of the side of the square base and let $y$ denote the height of the box.  Then  
\[
B=x^2
\]
and 
\[
W=xy\quad .
\]
As the surface area $S$ is fixed to be $36$ square feet, we have that
\[
S=3B+4W=36= 3x^2 + 4xy\quad .
\]
As $y$ is positive, the above formula shows that $3x^2\leq 36$ and so $x\leq \sqrt{12}$. Let us now express $y$ in terms of $x$:
\[
\begin{array}{rcl}
3x^2+4xy&=&36\\
4xy&=&36-x^2\\
y&=&\displaystyle\frac{36-x^2}{4x}\quad .
\end{array}
\]
The problem asks us to maximize the volume $V$ of the box. The volume of the box equals the area of the base times the height of the box: 
\[
V=B\cdot y=yx^2 = \frac{(36-3x^2)}{4x}x^2=\frac{36x-3x^3}{4}\quad .
\]
As $x$ is non-negative, it follows that the domain for $x$ is:
\[
x\in [0, \sqrt{12}]\quad .
\]
To maximize the volume we find the critical points, i.e., the values of $x$ for which  $V'$ vanishes:

\[\begin{array}{rcl}
0&=&V' = \displaystyle\left(\frac{36x-3x^3}{4}\right)'\\
0&=&\displaystyle \frac{36- 9x^2}{4}\\
9x^2&=&36\\
x^2&=&4\\
x&=&\pm 2
\end{array}
\]
As $x$ measures length, $x=-2$ is not possible (outside of the domain for $x$). Therefore the only critical point is $x=2$. Direct check shows that at the endpoints $x=0$ and $x=\sqrt{12}$, we have that $V=0$. Therefore the maximal volume is achieved when $x=2$:
 
\[
V_{max}=V_{|x=2}= \frac{36(2)-3(2)^3}{4} =12\quad .
\]
 
}

