% begin module higher-derivatives-ex6
\begin{frame}
\begin{example} %[Example 7, p. 153]
If $f(x) = x^3-x$, find $f''(x)$.
\begin{columns}[c]
\column{.4\textwidth}

\psset{xunit=1.2cm, yunit=1.2cm}
\begin{pspicture}(-5, -5)(5,5) 
\psframe*[linecolor=white](-5,-5)(5,5) 
\tiny
\psaxes[ticks=none, labels=none]{<->}(0,0)(-2,-2)(2,2)
\psLabelXOne
\psLabelYOne
\uncover<8->{
%Function formula: 6 (x) 
\psplot[linecolor=green, plotpoints=1000]{-0.333333333}{0.333333333}{x 6 mul } 
\rput[br](-0.4,-2){$f''(x)$}
}
\uncover<2->{
%Function formula: 3 ((x)^{2})-1 
\psplot[linecolor=blue, plotpoints=1000]{-1}{1}{-1 x 2 exp 3 mul add } 
\rput[r](-1, 1){$f'(x)$}
}
%Function formula: - (x)+(x)^{3} 
\psplot[linecolor=red, plotpoints=1000]{-1.521379707}{1.521379707}{x 3 exp x -1 mul add }
\rput[bl](1.2, 0.1){$f(x)$}
\end{pspicture} 
\column{.6\textwidth}
\uncover<2->{%
In a previous exercise we found that the first derivative is $f'(x) = 3x^2 - 1$.
}%
\abovedisplayskip=0pt
\belowdisplayskip=0pt
\begin{align*}
&  \uncover<3->{f''(x)}\\%
 & \uncover<3->{ = }  %
\uncover<3->{\lim_{h\rightarrow 0}\frac{f'(x+h)-f'(x)}{h}}\\%
 & \uncover<4->{ = }  %
\uncover<4->{\lim_{h\rightarrow 0}\frac{[3(x+h)^2-1]-[3x^2-1]}{h}}\\%
 & \uncover<5->{ = }  %
\uncover<5->{\lim_{h\rightarrow 0}\frac{3x^2 + 6xh + 3h^2 -1 - 3x^2 +1}{h}}\\%
 & \uncover<6->{ = }  %
\uncover<6->{\lim_{h\rightarrow 0}\frac{6xh + 3h^2}{h}}\\%
 & \uncover<7->{ = }  %
\uncover<7->{\lim_{h\rightarrow 0}(6x + 3h)}\uncover<8->{ = 6x}%
\end{align*}
\end{columns}
\end{example}
\end{frame}
% end module higher-derivatives-ex6
