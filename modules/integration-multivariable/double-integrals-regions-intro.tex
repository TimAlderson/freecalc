\begin{frame}
  \frametitle{More General Regions}

  What makes the iterated integral method work over rectangular regions?\pause

\begin{quote}
  For all values of one variable, all slices with respect to that variable are intervals in the other variable.
\end{quote}
\pause
If the variable is $y$, then:
\begin{itemize}
  \item we can integrate over each slice with respect to $x$,
  \item obtaining a function that depends only on the location of the slice,
  \item given by the $y-$value,
  \item then we integrate the result again with respect to $y$.
\end{itemize}
\pause
We can apply the same procedure if the slices are intervals, with endpoints depending continuously on the location of the slice.\pause
%
\begin{itemize}
  \item Regions of type I: vertical slices are segments.
  \item Regions of type II: horizontal slices are segments.
\end{itemize}
\end{frame}