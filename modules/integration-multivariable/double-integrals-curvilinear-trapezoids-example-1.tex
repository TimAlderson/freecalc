\begin{frame}
\begin{example}
\begin{columns}
\column{0.4\textwidth}
\psset{xunit=0.6cm, yunit=0.6cm}
\begin{pspicture}(-2, -2)(2,2)%
\tiny%
\fcBoundingBox{-0.1}{-1.2}{4}{4.5}%
\renewcommand{\fcScreen}{[-1 2 -3] 0}%
\only<13->{%
\pscustom*[linecolor=\fcColorAreaUnderGraph]{%
\fcCurveIIId{0}{2}{[t t t mul 0]}%
\fcLineIIId{[2 4 0]}{[2 4 2.5]}%
\fcCurveIIId{2}{0}{[t 2 t mul t t mul 4 t t mul mul add 8 div]}%
}%
\fcLineIIId[linewidth=0.5pt, linecolor=black]{[2 4 0]}{[2 4 2.5]}%
\fcStartIIIdScene%
\fcSurfaceInScene[linewidth=0.5, iterationsU=4, iterationsV=4, colorUV={0.2 0.7 0.9}]{0.00001}{0}{2}{1}{[2 dict begin /x u def /y {v 2 u mul mul 1 v sub u u mul mul add } def x y x x mul y y mul add 0.125 mul end]}{}%
\fcFinishIIIdScene[fastsort=true]%
}%
\only<handout:0|6-12>{%
\pscustom*[linecolor=\fcColorAreaUnderGraph]{%
\fcCurveIIId[linewidth=0.5pt, linecolor=black]{0}{2}{[t t t mul 0]}
\fcCurveIIId[linewidth=0.5pt, linecolor=black, linestyle=dashed]{2}{0}{[t 2 t mul 0]}%
}%
}%
\uncover<5->{\fcCurveIIId[linewidth=0.5pt, linecolor=black]{0}{2}{[t t t mul 0]}}%
\uncover<3->{\fcCurveIIId[linewidth=0.5pt, linecolor=black, linestyle=dashed]{2}{0}{[t 2 t mul 0]}}%
\fcAxesIIId{3}{3}{3}%
\end{pspicture}
\psset{xunit=0.6cm, yunit=0.6cm}
\begin{pspicture}(-1,-1)(1,1)
\tiny
\fcBoundingBox{-0.5}{-0.5}{2.6}{5.2}
\fcAxesStandard{-0.5}{-0.5}{2.5}{5}
\fcLabels{2.5}{5}
\uncover<6->{
\pscustom*[linecolor=\fcColorAreaUnderGraph]{%
\psplot{0}{2}{x x mul}%
\psline(2, 4)(0,0)%
}%
}
\uncover<5->{
\psplot{-0.2}{2.2}{x x mul}
\rput[l](1.3, 1){$\alert<2>{y=x^2}$}
}
\uncover<3->{
\rput[r](1.2, 2.4){$\alert<2>{y=2x}$}
\psline(-0.2, -0.4)(2.4, 4.8)
}
\uncover<10->{
\fcFullDot{0}{0}
\rput[tl](0.1, -0.1){$\alert<10>{(0,0)}$}
}
\uncover<12->{
\rput[r](1.9, 4){$\alert<12>{(2,4)}$}
\fcFullDot{2}{4}
}
\uncover<15->{
\psline[arrows=<->, linecolor=red, linewidth=1.5pt](0,0)(2,0)
}
\uncover<17->{
\psline[arrows=<->, linecolor=red, linewidth=1.5pt](1,1)(1,2)
}
\end{pspicture}

\column{0.6\textwidth}
Let $\mathcal{R}$ be the region bounded by $y=2x$ and $y=x^2$. Compute
\[
\iint_{\mathcal{R}} \frac{1}{8}\left(x^2+y^2\right) \diff x\diff y
\]
\fcQuestion{2}{Plot $y=2x$.} \fcQuestion{4}{ Plot $y=x^2$.} \uncover<6->{\alert<6>{Identify the region. }} 
\end{columns}
\only<handout:1|1-18>{
\fcQuestion{7}{The two curves intersect when } \fcAnswer{8}{ $
\begin{array}{rcl}x^2&=&2x \\ x(x-2)&=&0\\ x&=& 0 \text{ or } 2.\end{array}
$ 
}

\uncover<9->{The \alert<9,10,11,12>{intersection points are} therefore $\alert<9,10>{(0, \fcAnswer{10}{0} )}$ and $\alert<11,12>{(2,\fcAnswer{12}{4})}$.} \uncover<13->{We can plot the function $\frac{1}{8}\left(x^2+y^2\right)$ as above.} \uncover<14->{Our integral is}
}

$
\begin{array}{r@{~}c@{~}l}
\uncover<14->{\alert<18,19>{\displaystyle \int\limits_{{ \fcQuestion{ 14}{x=} \fcAnswer{15}{0}}}^{{ \fcQuestion{ 14}{ x =}\fcAnswer{15}{2}}}\left( \int\limits_{{ \fcQuestion{16}{ y= }\fcAnswer{17}{x^2 }}}^{{\fcQuestion{ 16}{ y=} \fcAnswer{ 17}{2x}}} \frac{1}{8}\left(\alert<20,21>{x^2+y^2} \right) \diff y \right) \diff x}} \uncover<handout:2| 20->{&=&\displaystyle \frac{1 }{8} \int_{x=0 }^{x=2}  { \left[\fcAnswer{21}{\alert<22,23>{ x^2y +\frac{ y^3}{3}}} \right]}_{ \alert<23>{y=x^2}}^{ \alert<22>{y=2x} }\diff x} \\
\uncover<handout:2| 22->{&=&\displaystyle \frac{1}{8}{\alert<24,25>{ \int}}_{0}^2 \alert<24,25>{\left( \alert<22>{4x^3+\frac{8}{3}x^3} \alert<23>{  -x^4 -\frac{ x^6 }{3}} \right)\diff x}} \\
\uncover<handout:2| 24->{&=& \displaystyle \frac{1}{8} {\left[\fcAnswer{25}{-\frac{ 1}{21} x^{7}-\frac{1}{5} x^{5}+\frac{5}{3} x^{4}}\right]}_{x=0}^{x=2}\\ \uncover<26->{&=&\frac{62}{35}}}
\end{array}
$


\end{example}

\vskip 10 cm

\end{frame}