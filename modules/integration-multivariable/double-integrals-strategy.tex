\begin{frame}
\frametitle{Strategy for Computing a Double Integral}
\begin{columns}
\column{0.4\textwidth}
\begin{pspicture}(-0.5,-0.5)(3.1,2.3)
\tiny
\fcAxesStandard{-0.5}{-0.5}{3.1}{2.3}%
\fcLabels{3.1}{2.3}
\uncover<5->{%
\pscustom*[linecolor=\fcColorAreaUnderGraph]{%
\psccurve(0.3,0.3)(0.5, 0.5)(2, 0.1)(3, 0.5)(2.7,1)(2.9,2)(2,2)(0.2, 1.1)(2, 1.1)(2, 0.7)(0.05,0.7)%
}%
}%
\uncover<4->{\psccurve(0.3,0.3)(0.5, 0.5)(2, 0.1)(3, 0.5) (2.7,1) (2.9,2) (2,2)(0.2, 1.1)(2, 1.1)(2, 0.7)(0.05,0.7)% 
}%
\uncover<6->{
\psccurve[linecolor=red, linewidth=1pt](0.3,0.3)(0.5, 0.5)(2, 0.1)(3, 0.5) (2.7,1) (2.9,2) (2,2)(0.2, 1.1)(2, 1.1)(2, 0.7)(0.05,0.7)%
\psline[linecolor=red, linewidth=1pt](2.0735, 0.105)(2.0735, 2.03)
\psline[linecolor=red, linewidth=1pt](2.0735, 2.0248)(2.89, 2.0248)
\fcFullDot[scale=0.7, linecolor=red]{0.035}{0.63}%
\fcFullDot[scale=0.7, linecolor=red]{2.889}{2.025}
\fcFullDot[scale=0.7, linecolor=red]{2.0735}{2.025}
\fcFullDot[scale=0.7, linecolor=red]{0.19}{1.15}
\fcFullDot[scale=0.7, linecolor=red]{2.0735}{0.105}
\fcFullDot[scale=0.7, linecolor=red]{2.0735}{0.92}
}
\uncover<7->{
\fcFullDot[scale=0.7, linecolor=red]{0.3}{0.3}
\psline[linecolor=red, linewidth=1pt](0.3,0.3)(0.3, 0.867)
\fcFullDot[scale=0.7, linecolor=red]{0.3}{0.867}
}
\end{pspicture}
\column{0.6\textwidth}
\begin{problem}
Find the integral $\iint_{\mathcal R} f(x,y)\diff x \diff y$ over a region $\mathcal R$ enclosed by a set of smooth curves.
\end{problem}
\end{columns}
\begin{itemize}
\item<2-> We present a strategy for approaching the above problem. 
\item<3-> The tractability of this strategy depends on the concrete description of $f$ and the enclosing curves.
\begin{itemize}
\item<4-> Plot the curve(s) enclosing $\mathcal R$.
\item<5-> Identify the region $\mathcal R$.
\item<6-> Chop $\mathcal R$ into curvilinear trapezoids; the trapezoids are allowed to intersect only on their boundaries.
\item<7-> By possible subdivision ensure trapezoids have smooth boundaries.
\item<8-> Integrate $f$ over the obtained curvilinear trapezoids \& collect terms.
\end{itemize}
\item<9-> Our strategy will be augmented/combined later with variable changes (via the multivariable substitution rule).
\end{itemize}
\end{frame}