\begin{frame}
\begin{example}
\begin{columns}
\column{0.4\textwidth}
\begin{pspicture}(-1,-1)(1,1)
\tiny
\fcBoundingBox{-2.5}{-1.4}{2.1}{2.1}%
\uncover<11->{%
\pscustom*[linecolor=\fcColorAreaUnderGraph]{%
\psline(-1, -1)(0, -1)%
\parametricplot{-1}{1}{t t 3 exp sub t}%
\parametricplot{0}{-1}{t t 1 add dup mul}%
\psline(-1, 0)(-1, -1)%
}%
}%
\uncover<14->{%
\pscustom*[linecolor=green]{%
\psline(-1, -1)(0, -1)%
\parametricplot{-1}{0}{t t 3 exp sub t}%
\psline(0,0)(-1, 0)%
\psline(-1,0)(-1, -1)%
}%
\rput[r](-0.5,-0.5){$\mathcal R_2$}
\pscustom*[linecolor=cyan]{%
\parametricplot{0}{1}{t t 3 exp sub t}%
\parametricplot{0}{-1}{t t 1 add dup mul}%
\psline(-1,0)(0,0)%
}%
\rput[b](-0.2,0.1){$\mathcal R_1$}
}%
\uncover<9->{\parametricplot[linecolor=red]{-1.4 }{1.6}{t t 3 exp sub t}}
\uncover<7->{\parametricplot[linecolor=red]{-2.4}{2 sqrt 1 sub}{t t 1 add dup mul}}
\uncover<3->{\psline[linecolor=red](-1, -1.4)(-1, 2)}
\uncover<5->{\psline[linecolor=red](-2.4, -1)(2, -1)}
\fcAxesStandardNoFrame{-2.5}{-1.4}{2}{2}%
\uncover<13->{%
\fcFullDot{0}{-1}%
\fcFullDot{-1}{-1}%
\fcFullDot{-1}{0}%
\fcFullDot{0}{1}%
\fcLabels{2}{2}%
}%
\uncover<18->{%
\psline[linecolor=red, linewidth=1.5pt, arrows=<->](0,0)(0,1)%
\fcFullDot{0}{0}%
\fcFullDot{0}{1}%
\psline[linecolor=red, linewidth=1.5pt, arrows=<->](0,0)(0,-1)%
\fcFullDot{0}{0}%
\fcFullDot{0}{-1}%
}%
\uncover<20->{%
\psline[linecolor=red, linewidth=1.5pt, arrows=<->](! 0.5 sqrt 1 sub 0.5)(0.375,0.5)%
\fcFullDot{0.5 sqrt 1 sub}{0.5}%
\fcFullDot{0.375}{0.5}%
\psline[linecolor=red, linewidth=1.5pt, arrows=<->](-1,-0.5)(-0.375,-0.5)%
\fcFullDot{-1}{-0.5}%
\fcFullDot{-0.375}{-0.5}%
}%
\end{pspicture}
\column{0.6\textwidth}
Let $\mathcal{R}$ be region bounded by $y=(x+1)^2$, $x=y-y^3$, the line $x=-1$ and the line $y=-1$. Set-up iterated integrals for 
\[
\iint_{\mathcal{R}} f \diff A .
\]
\end{columns}
\fcQuestion{2}{Plot $x=-1$.} \fcQuestion{4}{Plot $y=-1$.} \fcQuestion{6}{Plot $y=(x+1)^2$.} \fcQuestion{8}{Plot $x=y-y^3$.} \fcQuestion{10}{Identify the region.} \fcQuestion{12}{Compute the intersection points: the four points lying on the boundary of our region have coordinates:} \fcAnswer{13}{$(-1,-1), (0,-1), (-1,0), (0,1)$. } \uncover<14->{Split into two curvilinear trapezoids: $\mathcal R=\mathcal R_1\cup \mathcal R_2$, where $\mathcal R_1,\mathcal R_2$ are as indicated. }  \fcQuestion{15}{\alert<15-20>{The integral becomes:}}

$\displaystyle
\uncover<15->{\iint\limits_{\mathcal R_1}f \diff A+\iint\limits_{\mathcal R_2}f\diff A =  \int\limits_{{ \fcQuestion{17}{ y=}\fcAnswer{18}{0}}}^{ {\fcQuestion{17}{y=}\fcAnswer{18}{1}}} \int\limits_{{ \fcQuestion{19}{x=}\fcAnswer{20}{\sqrt{y}-1}} }^{ {\fcQuestion{19}{x=}\fcAnswer{20}{y-y^3}} } f  \fcAnswer{16}{\diff x\diff y}+ 
\int\limits_{{ \fcQuestion{17}{ y= }\fcAnswer{18}{-1} }}^{{\fcQuestion{17}{ y=}\fcAnswer{18}{0} } } \int\limits_{{\fcQuestion{19}{x=}\fcAnswer{20}{-1} }}^{{\fcQuestion{19}{x=}\fcAnswer{20}{y-y^3}}} f  \fcAnswer{16}{\diff x\diff y} 
}
$
\end{example}


\end{frame}