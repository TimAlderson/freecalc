% begin module maclaurin-series-ex4
\begin{frame}
\begin{example}
Find the Maclaurin series of $f(x) = \sin x$ and its radius of convergence.
\abovedisplayskip=2pt
\belowdisplayskip=2pt
\[
\begin{array}{rcl@{\qquad}rcl}
\uncover<2->{%
f(x)%
}%
&%
\uncover<2->{%
=%
}%
&%
\uncover<2->{%
\sin x%
}%

&%
\uncover<2->{%
\alertNoH{ 2-3,21}{%
f(0)%
}%
}%
&%
\uncover<2->{%
\alertNoH{ 2-3,21}{%
=%
}%
}%
&%
\alertNoH{ 2-3,21}{%
\uncover<3->{%
0%
}%
}\\%

\uncover<2->{%
\alertNoH{ 4-5}{%
f'(x)%
}%
}%
&%
\uncover<2->{%
\alertNoH{ 4-5}{%
=%
}%
}%
&%
\alertNoH{ 4-5}{%
\uncover<5->{%
\cos x%
}%
}%

&%
\uncover<2->{%
\alertNoH{ 6-7,22-23}{%
f'(0)%
}%
}%
&%
\uncover<2->{%
\alertNoH{ 6-7,22-23}{%
=%
}%
}%
&%
\alertNoH{ 6-7,22-23}{%
\uncover<7->{%
1%
}%
}\\%

\uncover<2->{%
\alertNoH{ 8-9}{%
f''(x)%
}%
}%
&%
\uncover<2->{%
\alertNoH{ 8-9}{%
=%
}%
}%
&%
\alertNoH{ 8-9}{%
\uncover<9->{%
-\sin x%
}%
}%

&%
\uncover<2->{%
\alertNoH{ 10-11,24}{%
f''(0)%
}%
}%
&%
\uncover<2->{%
\alertNoH{ 10-11,24}{%
=%
}%
}%
&%
\alertNoH{ 10-11,24}{%
\uncover<11->{%
0%
}%
}\\%

\uncover<2->{%
\alertNoH{ 12-13}{%
f'''(x)%
}%
}%
&%
\uncover<2->{%
\alertNoH{ 12-13}{%
=%
}%
}%
&%
\alertNoH{ 12-13}{%
\uncover<13->{%
-\cos x%
}%
}%

&%
\uncover<2->{%
\alertNoH{ 14-15,25-26}{%
f'''(0)%
}%
}%
&%
\uncover<2->{%
\alertNoH{ 14-15,25-26}{%
=%
}%
}%
&%
\alertNoH{ 14-15,25-26}{%
\uncover<15->{%
-1%
}%
}\\%

\uncover<2->{%
\alertNoH{ 16-17}{%
f^{(4)}(x)%
}%
}%
&%
\uncover<2->{%
\alertNoH{ 16-17}{%
=%
}%
}%
&%
\alertNoH{ 16-17}{%
\uncover<17->{%
\sin x%
}%
}%

&%
\uncover<2->{%
\alertNoH{ 18-19,27}{%
f^{(4)}(0)%
}%
}%
&%
\uncover<2->{%
\alertNoH{ 18-19,27}{%
=%
}%
}%
&%
\alertNoH{ 18-19,27}{%
\uncover<19->{%
0%
}%
}%
\end{array}
\]
\uncover<20->{%
The Maclaurin series is
\abovedisplayskip=2pt
\belowdisplayskip=2pt
\[
\sum_{n=0}^\infty \frac{f^{(n)}(0)}{n!}x^n = %
\uncover<23->{%
\alertNoH{ 23}{%
\alertNoH{ 33-34}{x}%
}%
}%
\uncover<26->{%
\alertNoH{ 26}{%
\alertNoH{ 31-32}{-} \frac{x^{\alertNoH{ 33-34}{3}}}{\alertNoH{ 33-34}{3}!}%
}%
}%
\uncover<29->{%
\alertNoH{ 29}{%
\alertNoH{ 31-32}{+} \frac{x^{\alertNoH{ 33-34}{5}}}{\alertNoH{ 33-34}{5}!}%
}%
}%
\uncover<30->{%
\alertNoH{ 30}{%
\alertNoH{ 31-32}{-} \frac{x^{\alertNoH{ 33-34}{7}}}{\alertNoH{ 33-34}{7}!} \alertNoH{ 31-32}{+} \cdots %
}%
}%
\uncover<31->{%
 = \sum_{n=0}^\infty \uncover<32->{\alertNoH{ 32}{(-1)^n}}\frac{\uncover<34->{x^{\alertNoH{ 34}{2n+1}}}}{\uncover<34->{(\alertNoH{ 34}{2n+1})!}}%
}%
\]
}%
\uncover<35->{%
Use the Ratio Test to find $R$.%
}%
\abovedisplayskip=2pt
\belowdisplayskip=2pt
\begin{eqnarray*}
\uncover<35->{%
\lim_{n\to\infty} \left| \frac{a_{n+1}}{a_n}\right|%
}%
& \uncover<35->{ = } &%
\uncover<35->{%
\lim_{n\to\infty} \left| \frac{(-1)^{n+1}\alertNoH{ 36-37}{x^{2n+3}}}{\alertNoH{ 38-39}{(2n+3)!}}\cdot \frac{\alertNoH{ 38-39}{(2n+1)!}}{(-1)^n\alertNoH{ 36-37}{x^{2n+1}}}\right|%
}\\%
& \uncover<36->{ = } &%
\uncover<36->{%
\alertNoH{ 40-41}{\lim_{n\to\infty} \frac{\uncover<37->{\alertNoH{ 37}{x^2}}}{\uncover<39->{\alertNoH{ 39}{(2n+2)(2n+3)}}}}%
}%
 \uncover<40->{\alertNoH{ 40-41}{ = }} %
\uncover<41->{%
\alertNoH{ 41}{0}%
}%
\end{eqnarray*}
\uncover<42->{Therefore $\alertNoH{ 42-43}{R = \uncover<43->{\infty}}$.  }%
\uncover<44->{It can be shown that this series sums to $\sin x$.}%
\end{example}
\end{frame}
% end module maclaurin-series-ex4
