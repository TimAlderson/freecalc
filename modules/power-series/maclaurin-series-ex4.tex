% begin module maclaurin-series-ex4
\begin{frame}
\begin{example}
Find the Maclaurin series of $f(x) = \sin x$ and its radius of convergence.
\abovedisplayskip=2pt
\belowdisplayskip=2pt
\[
\begin{array}{rcl@{\qquad}rcl}
\uncover<2->{%
f(x)%
}%
&%
\uncover<2->{%
=%
}%
&%
\uncover<2->{%
\sin x%
}%

&%
\uncover<2->{%
\alert<handout:0| 2-3,21>{%
f(0)%
}%
}%
&%
\uncover<2->{%
\alert<handout:0| 2-3,21>{%
=%
}%
}%
&%
\alert<handout:0| 2-3,21>{%
\uncover<3->{%
0%
}%
}\\%

\uncover<2->{%
\alert<handout:0| 4-5>{%
f'(x)%
}%
}%
&%
\uncover<2->{%
\alert<handout:0| 4-5>{%
=%
}%
}%
&%
\alert<handout:0| 4-5>{%
\uncover<5->{%
\cos x%
}%
}%

&%
\uncover<2->{%
\alert<handout:0| 6-7,22-23>{%
f'(0)%
}%
}%
&%
\uncover<2->{%
\alert<handout:0| 6-7,22-23>{%
=%
}%
}%
&%
\alert<handout:0| 6-7,22-23>{%
\uncover<7->{%
1%
}%
}\\%

\uncover<2->{%
\alert<handout:0| 8-9>{%
f''(x)%
}%
}%
&%
\uncover<2->{%
\alert<handout:0| 8-9>{%
=%
}%
}%
&%
\alert<handout:0| 8-9>{%
\uncover<9->{%
-\sin x%
}%
}%

&%
\uncover<2->{%
\alert<handout:0| 10-11,24>{%
f''(0)%
}%
}%
&%
\uncover<2->{%
\alert<handout:0| 10-11,24>{%
=%
}%
}%
&%
\alert<handout:0| 10-11,24>{%
\uncover<11->{%
0%
}%
}\\%

\uncover<2->{%
\alert<handout:0| 12-13>{%
f'''(x)%
}%
}%
&%
\uncover<2->{%
\alert<handout:0| 12-13>{%
=%
}%
}%
&%
\alert<handout:0| 12-13>{%
\uncover<13->{%
-\cos x%
}%
}%

&%
\uncover<2->{%
\alert<handout:0| 14-15,25-26>{%
f'''(0)%
}%
}%
&%
\uncover<2->{%
\alert<handout:0| 14-15,25-26>{%
=%
}%
}%
&%
\alert<handout:0| 14-15,25-26>{%
\uncover<15->{%
-1%
}%
}\\%

\uncover<2->{%
\alert<handout:0| 16-17>{%
f^{(4)}(x)%
}%
}%
&%
\uncover<2->{%
\alert<handout:0| 16-17>{%
=%
}%
}%
&%
\alert<handout:0| 16-17>{%
\uncover<17->{%
\sin x%
}%
}%

&%
\uncover<2->{%
\alert<handout:0| 18-19,27>{%
f^{(4)}(0)%
}%
}%
&%
\uncover<2->{%
\alert<handout:0| 18-19,27>{%
=%
}%
}%
&%
\alert<handout:0| 18-19,27>{%
\uncover<19->{%
0%
}%
}%
\end{array}
\]
\uncover<20->{%
The Maclaurin series is
\abovedisplayskip=2pt
\belowdisplayskip=2pt
\[
\sum_{n=0}^\infty \frac{f^{(n)}(0)}{n!}x^n = %
\uncover<23->{%
\alert<handout:0| 23>{%
\alert<handout:0| 33-34>{x}%
}%
}%
\uncover<26->{%
\alert<handout:0| 26>{%
\alert<handout:0| 31-32>{-} \frac{x^{\alert<handout:0| 33-34>{3}}}{\alert<handout:0| 33-34>{3}!}%
}%
}%
\uncover<29->{%
\alert<handout:0| 29>{%
\alert<handout:0| 31-32>{+} \frac{x^{\alert<handout:0| 33-34>{5}}}{\alert<handout:0| 33-34>{5}!}%
}%
}%
\uncover<30->{%
\alert<handout:0| 30>{%
\alert<handout:0| 31-32>{-} \frac{x^{\alert<handout:0| 33-34>{7}}}{\alert<handout:0| 33-34>{7}!} \alert<handout:0| 31-32>{+} \cdots %
}%
}%
\uncover<31->{%
 = \sum_{n=0}^\infty \uncover<32->{\alert<handout:0| 32>{(-1)^n}}\frac{\uncover<34->{x^{\alert<handout:0| 34>{2n+1}}}}{\uncover<34->{(\alert<handout:0| 34>{2n+1})!}}%
}%
\]
}%
\uncover<35->{%
Use the Ratio Test to find $R$.%
}%
\abovedisplayskip=2pt
\belowdisplayskip=2pt
\begin{eqnarray*}
\uncover<35->{%
\lim_{n\to\infty} \left| \frac{a_{n+1}}{a_n}\right|%
}%
& \uncover<35->{ = } &%
\uncover<35->{%
\lim_{n\to\infty} \left| \frac{(-1)^{n+1}\alert<handout:0| 36-37>{x^{2n+3}}}{\alert<handout:0| 38-39>{(2n+3)!}}\cdot \frac{\alert<handout:0| 38-39>{(2n+1)!}}{(-1)^n\alert<handout:0| 36-37>{x^{2n+1}}}\right|%
}\\%
& \uncover<36->{ = } &%
\uncover<36->{%
\alert<handout:0| 40-41>{\lim_{n\to\infty} \frac{\uncover<37->{\alert<handout:0| 37>{x^2}}}{\uncover<39->{\alert<handout:0| 39>{(2n+2)(2n+3)}}}}%
}%
 \uncover<40->{\alert<handout:0| 40-41>{ = }} %
\uncover<41->{%
\alert<handout:0| 41>{0}%
}%
\end{eqnarray*}
\uncover<42->{Therefore $\alert<handout:0| 42-43>{R = \uncover<43->{\infty}}$.  }%
\uncover<44->{It can be shown that this series sums to $\sin x$.}%
\end{example}
\end{frame}
% end module maclaurin-series-ex4
