% begin module power-series-def
\begin{frame}
\frametitle{(12.8)  Power Series}
\begin{definition}[Power Series]
A power series is a series of the form
\abovedisplayskip=0pt
\belowdisplayskip=0pt
\[
\sum_{n=0}^\infty c_n x^n = c_0 + c_1 x + c_2 x^2 + c_3x^3 + \cdots%
\]
where $x$ is a variable and the $c_n$'s are constants called the coefficients of the series.
\end{definition}
\begin{itemize}
\item<2->  For each fixed $x$, this is a series of constants which either converges or diverges.
\item<3->  A power series might converge for some values of $x$ and diverge for others.
\item<4->  The sum of the series is a function
\abovedisplayskip=0pt
\belowdisplayskip=0pt
\[
\uncover<4->{%
f(x) = c_0 + c_1x + c_2x^2 + c_3x^3 + \cdots %
}%
\]
\uncover<4->{whose domain is the set of all $x$ for which the series converges.}
\item<5->  $f$ resembles a polynomial, except it has infinitely many terms.
\end{itemize}
\end{frame}

\begin{frame}
\begin{definition}[Power Series Centered at $a$]
A series of the form
\abovedisplayskip=0pt
\belowdisplayskip=0pt
\[
\sum_{n=0}^\infty c_n (x-a)^n = c_0 + c_1 (x-a) + c_2 (x-a)^2 + c_3(x-a)^3 + \cdots%
\]
is called a power series centered at $a$ or a power series about $a$ or a power series in $(x-a)$.
\end{definition}
\begin{itemize}
\item<2->  We use the convention that $(x-a)^0 = 1$, even if $x = a$.
\item<3->  If $x = a$, then all terms are $0$ for $n \geq 1$, so the series always converges when $x = a$.
\end{itemize}
\end{frame}
% end module power-series-def
