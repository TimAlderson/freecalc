% begin module taylor-series-ex3
\begin{frame}
\begin{example}[Example 3, p. 775]
Find the Taylor series for $f(x) = e^x$ at $a = 2$.
\begin{itemize}
\item<2-| alert@2-3>  $f^{(n)}(x) = $ \uncover<3->{$e^x$.}
\item<4-| alert@4-6>  $f^{(n)}(2) = $ \uncover<5->{$e^2$.}
\item<6->  Therefore the Taylor series is
\end{itemize}
\uncover<6->{%
\abovedisplayskip=0pt
\belowdisplayskip=0pt
\[
\sum_{n=0}^\infty \frac{\alert<handout:0| 6>{f^{(n)}(2)}}{n!}(x-2)^n = \sum_{n=0}^\infty \frac{\alert<handout:0| 6>{e^2}}{n!}(x-2)^n %= e^2 + \frac{e^2}{1!}(x-2) + \frac{e^2}{2!}(x-2)^2 + \frac{e^3}{3!}(x-2)^3 + \cdots%
\]
}%
\begin{itemize}
\item<7->  To find the radius of convergence, let $a_n = \frac{e^2}{n!}(x-2)^n$.
\end{itemize}
\abovedisplayskip=0pt
\belowdisplayskip=0pt
%\begin{eqnarray*}
\[
\uncover<8->{%
\lim_{n\to\infty} \left| \frac{a_{n+1}}{a_n}\right|%
}%
 \uncover<8->{ = } %
\uncover<8->{%
\lim_{n\to\infty} \left| \frac{e^2\alert<handout:0| 9-10>{(x-2)^{n+1}}}{\alert<handout:0| 11-12>{(n+1)!}}\cdot \frac{\alert<handout:0| 11-12>{n!}}{e^2\alert<handout:0| 9-10>{(x-2)^n}}\right|%
}%
 \uncover<9->{ = } %
\uncover<9->{%
\alert<handout:0| 13-14>{\lim_{n\to\infty} \frac{\uncover<10->{\alert<handout:0| 10>{|x-2|}}}{\uncover<12->{\alert<handout:0| 12>{n+1}}}}%
}%
 \uncover<13->{\alert<handout:0| 13-14>{ = }} %
\uncover<14->{%
\alert<handout:0| 14>{0}%
}%
% \uncover<15->{\alert<handout:0| 15>{ < 1}} %
%\end{eqnarray*}
\]
\begin{itemize}
\item<15->  Therefore by the Ratio Test the series converges for all $x$.
\item<16->  Therefore $R = \infty$.
\item<17->  Just like the Maclaurin series, this series also represents $e^x$.
\end{itemize}
\end{example}
\end{frame}
% end module taylor-series-ex3
