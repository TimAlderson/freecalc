% begin module radius-of-convergence
\begin{frame}[t]
\begin{theorem}[Convergence of Power Series]
For a power series $\displaystyle \sum c_n (x-a)^n$, there are three possibilities:
\begin{enumerate}
\item<1-| alert@3,6>  The series converges only when $x = a$.
\item<1-| alert@4,7>  The series converges for all $x$.
\item<1-| alert@2,8>  There is a positive number $R$ such that the series converges if $|x-a| < R$ and diverges if $|x-a|>R$.
\end{enumerate}
\end{theorem}

\uncover<2->{%
\only<handout:1| 1-4>{%
\begin{definition}[Radius of Convergence]
The number $R$ in case three of the theorem is called the radius of convergence of the power series.
\end{definition}
\begin{enumerate}
\item<3-| alert@3>  In the first case, we say $R = 0$.
\item<4-| alert@4>  In the second case, we say $R = \infty$.
\end{enumerate}
}%
\only<handout:2| 5->{%
\begin{definition}[Interval of Convergence]
The interval of convergence of a power series is the interval consisting of all numbers $x$ for which the series converges.
\end{definition}
\begin{enumerate}
\item<6-| alert@6>  In the first case, the interval contains the single point $ a$.
\item<7-| alert@7>  In the second case, the interval is $( - \infty , \infty )$.
\item<8-| alert@8>  In the third case, the inequality $|x - a| < R$ can be rewritten $a - R < x < a + R$.
\end{enumerate}
}%
}%
\end{frame}
% end module radius-of-convergence
