\begin{frame}
\begin{columns}
\column{0.25\textwidth}
\begin{pspicture}(-1.2, -0.5)(2.2,3.2)
\tiny
\fcBoundingBox{-1.2}{-0.5}{2.2}{3.2}
\renewcommand{\fcScreenStyle}{x}%
\pstVerb{5 dict begin /ver1 [1 0 0.2] def /ver2 [0 1 0.2] def /ver3 [1.4 1 2] def /perpFoot ver3 1 dict begin /normal ver1 ver2 \fcVectorCrossVector def normal normal ver3 \fcVectorScalarVector normal normal \fcVectorScalarVector div \fcVectorTimesScalar end \fcVectorMinusVector def}%
\fcStartIIIdScene
\only<5->{\fcLineIIIdInScene[linewidth=2, linecolor=brown]{ver3}{perpFoot}%
\fcPerpendicularIIId[linestyle=none, linecolor=brown]{ver3}{ [0 0 0] perpFoot}{0.2}%
\fcLineIIIdInScene[linewidth=2]{[0 0 0]}{perpFoot 1.3 \fcVectorTimesScalar}%
}%
\only<2->{%
\fcBoxIIIdInScene[linewidth=1, linecolor=black, colorUV=cyan, dashes={[0.5 3] 0}, forceForeground=false]{[0 0 0]}{ver1}{ver2}{ver3}%
}%
\only<4->{\fcPatchInScene[colorUV=blue, colorVU=blue, forceForeground=true]{[0 0 0]}{ver1}{ver2}}%
\fcAxesIIIdInScene[linewidth=1, arrows=->, linecolor=black]{2}{4}{2}
\fcFinishIIIdScene[true]
\only<1>{%
\fcLineIIId[linewidth=1.5pt, arrows=->, linecolor=red]{[0 0 0]}{ver1}%
\fcLineIIId[linewidth=1.5pt, arrows=->, linecolor=red]{[0 0 0]}{ver2}%
\fcLineIIId[linewidth=1.5pt, arrows=->, linecolor=red]{[0 0 0]}{ver3}%
}%
\pstVerb{end}
\end{pspicture}

\column{0.75\textwidth}

\begin{itemize}
\item Let $ \fcv v_1=(v_{11}, \dots, v_{1n})$, $\dots$, $\fcv v_n=(v_{n1},\dots, v_{nn} )$ be $n$-vectors in $n$-dimensional space.
\item<2-> Let $\mathcal R_k$ be parallelotope spanned by $\fcv v_1, \dots, \fcv v_k$.
\item<3-> $\mathcal R_k$ can be regarded as ``prism'' with \alert<4>{base $ \mathcal R_{k-1}$}.
\item<5-> Let $h_k$ be the height from $\fcv v_k$ to the base $\mathcal R_{k-1}$. 
\end{itemize}
\end{columns}
\uncover<6->{
\begin{definition}[$k$-volume of a parallelotope]
Define $\text{Vol}_1(\mathcal R_1)=|\fcv v_1|$. \uncover<7->{For $k>1$, define $\text{Vol}_{k}(\mathcal R_k)= h_k \text{Vol}_{k-1}( \mathcal R_{k-1})$.}
\end{definition}
}


\begin{itemize}
\item<8-> Let the height vector $\fcv h_k$ be the vector of the form $\fcv h_k =\fcv v_k+a_1\fcv v_1 +\dots a_{k-1}\fcv v_{k-1}$ for which $\fcv h_k\cdot \fcv v_1 =0,\dots, \fcv h_k\cdot \fcv v_{k-1}=0$.
\item<9-> Then $h_k$ is computed as the length of $\fcv h_k$.
\item<10-> For the largest parallelotope $\mathcal R_n$, we already have definition of volume: \uncover<11->{the integral of $1$ over $\mathcal R_n$.} 
\item<12-> We will see that $ \text{Vol}_n (\mathcal R_n) $ equals that integral.
\end{itemize}
\end{frame}
