% begin module improper-integral-def
\begin{frame}
\frametitle{(8.8) Improper Integrals}
\begin{itemize}
\item  The definition of $\int_a^b f(x) \diff x$, where $f$ is defined on $[a,b]$,  has two requirements:
\begin{enumerate}
\item  $[a,b]$ is a finite interval.
%\item  $f$ is defined on $[a,b]$.
\item  $f$ has no infinite discontinuities in $[a,b]$.
\end{enumerate}
\item<2->  We are now going to relax these requirements.
\begin{enumerate}
\item<2->  We allow infinite intervals, such as $(a,\infty), (-\infty , b)$, and $(-\infty , \infty )$.
\item<2->  $f$ might have infinite discontinuities in $[a,b]$.
\end{enumerate}
\item<3->  Such integrals are called improper integrals.
\end{itemize}
\uncover<4->{%
\begin{definition}[Improper Integral]
The integral
\[
\int_a^b f(x)\diff x
\]
is called improper if one or more of the endpoints $a$ and $b$ is infinite, or if $f$ has an infinite discontinuity on $[a,b]$.
\end{definition}
}%
\end{frame}
% end module improper-integral-def
