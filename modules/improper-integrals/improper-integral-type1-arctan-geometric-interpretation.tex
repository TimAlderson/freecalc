%begin module improper-integral-type1-arctan-geometric-interpretation
%\begin{comment}
\begin{frame}[t]

\psset{xunit=2cm, yunit=2cm}
\begin{pspicture}(-3.1, -0.1)(3.1,1.2) 
\psframe*[linecolor=white](-3.1,-0.1)(3.1,1.2) 
\tiny 
\psline[linecolor=red!1](0,1.2)(0.001,1.2) %bounding boxes don't always work right
\psaxes[arrows=<->, ticks=none, labels=none] (0,0) (-3.05,-0.1) (3.05,1.1)
\rput[t](3, -0.1){$x$}

%Function formula: - (- x^{2}+1)^{1/2} 
%\psplot[linecolor=\psColorGraph, plotpoints=1000]{-1}{1}{1 x 2 exp -1 mul add 0.5 exp -1 mul }
%Function formula: (- x^{2}+1)^{1/2} 
\psplot[linecolor=blue, plotpoints=1000] {-1} {1} {1 x 2 exp -1 mul add 0.5 exp }
\psline(-3,1)(3,1) 

\uncover<3->{
\psline{<-}(-0.05, 0)(-0.05,0.4)
\psline{->}(-0.05, 0.6)(-0.05,1)
\rput(-0.05, 0.5){\alert<3>{$1$}}
}
\uncover<2>{
\psline[linecolor=red](0.5,1)(1,1)
\psFullDot{1}{1}
}
\uncover<2>{
\psFullDot{0.5}{1}
}

\rput[br](-0.05, 1.05){ $\phantom{Q_0=P_0}A$}
\uncover<2->{
\rput[lb](0.5, 1.05){$P_1\phantom{(x_1 ,1)}$}
\rput[lb](1, 1.05){$P_2(\alert<2>{x_2}, 1)$}
}
\uncover<2->{
\rput[t](0.75,0.95){\alert<2>{$\Delta$}}
}
\uncover<5->{
\rput[lt](0.520000, 0.4500000){$\alert<5>{Q_2}$}
\rput[l](0.42000, 0.8){$\alert<5>{Q_1}$}
}
\rput[tl](0.05, -0.05){$O$}
%calculator commands:
%\Delta:=0.5;
%xInv{}{{x}}:=DoubleValue{}x/(1+x^2);
%yInv{}{{x}}:=DoubleValue{} 1/(1+x^2);
%xDir{}{{x}}:=-xInv{}x;
%yDir{}{{x}}:=1-yInv{}x;
%lengthN{}{{x}}:=((xDir{}x)^2+(yDir{}x)^2)^{1/2};
%xDirN{}{{x}}:=0.05 xDir{}x /lengthN{}x ;
%yDirN{}{{x}}:=0.05 yDir{}x/lengthN{}x;
%xT{}{{x}}:=-yDirN{}x;
%yT{}{{x}}:=xDirN{}x;

%inversePoint{}{{x}}:=(xInv{}x,yInv{}x );
%f{}{{x}}:= \psline(DoubleValue{}x,1 )(0,0) \psline inversePoint{}x (0,1)  \psline inversePoint{}x inversePoint{}(x+\Delta) \psline (xInv{}x +xDirN{}x, yInv{}x+yDirN{}x ) (xInv{}x +xDirN{}x+xT{}x, yInv{}x+yDirN{}x +yT{}x)(xInv{}x +xT{}x, yInv{}x +yT{}x);
%f{}(\Delta) f{}(2\Delta)f{}(3\Delta)f{}(4\Delta)f{}(5\Delta)f{}(6\Delta)

\uncover<5->{
\psline (0.5, 1) (0, 0) 
}
\uncover<5->{
\psline (0.4, 0.8) (0, 1) 
}
\uncover<15->{
\psline (0.4, 0.8) (0.5, 0.5) 
}
\uncover<5->{\psline (0.355279, 0.822361) (0.332918, 0.777639) (0.377639, 0.755279) }

\uncover<4->{\psline (1, 1) (0, 0) }
\uncover<5->{\psline (0.5, 0.5) (0, 1)} 
%\psline (0.5, 0.5) (0.461538, 0.307692) 
\uncover<5->{\psline (0.464645, 0.535355) (0.429289, 0.5) (0.464645, 0.464645) }


%\psline (1.500000, 1) (0, 0) 
%\psline (0.461538, 0.307692) (0, 1) 
%\psline (0.461538, 0.307692) (0.4, 0.2) 
%\psline (0.433803, 0.349295) (0.392201, 0.321560) (0.419936, 0.279957) 

%\psline (2, 1) (0, 0) 
%\psline (0.4, 0.2) (0, 1) 
%\psline (0.4, 0.2) (0.344828, 0.137931) 
%\psline (0.377639, 0.244721) (0.332918, 0.222361) (0.355279, 0.177639) 

%\psline (2.5, 1) (0, 0) 
%\psline (0.344828, 0.137931) (0, 1) 
%\psline (0.344828, 0.137931) (0.300000, 0.100000) 
%\psline (0.326258, 0.184355) (0.279834, 0.165785) (0.298404, 0.119362) 

%\psline (3, 1) (0, 0) 
%\psline (0.300000, 0.100000) (0, 1) 
%\psline (0.300000, 0.100000) (0.264151, 0.075472) 
%\psline (0.284189, 0.147434) (0.236754, 0.131623) (0.252566, 0.084189)

%\psline (-0.5, 1) (0, 0) 
%\psline (-0.4, 0.8) (0, 1) 
%\psline (-0.4, 0.8) (-0.5, 0.5) \psline (-0.355279, 0.822361) (-0.377639, 0.867082) (-0.422361, 0.844721) 

%\psline (-1, 1) (0, 0) 
%\psline (-0.5, 0.5) (0, 1) 
%\psline (-0.5, 0.5) (-0.461538, 0.307692) 
%\psline (-0.464645, 0.535355) (-0.5, 0.570711) (-0.535355, 0.535355) 

%\psline (-1.500000, 1) (0, 0) 
%\psline (-0.461538, 0.307692) (0, 1) 
%\psline (-0.461538, 0.307692) (-0.4, 0.2) 
%\psline (-0.433803, 0.349295) (-0.475406, 0.377030) (-0.503141, 0.335427) 

%\psline (-2, 1) (0, 0) 
%\psline (-0.4, 0.2) (0, 1) 
%\psline (-0.4, 0.2) (-0.344828, 0.137931) 
%\psline (-0.377639, 0.244721) (-0.422361, 0.267082) (-0.444721, 0.222361) 

%\psline (-2.5, 1) (0, 0) 
%\psline (-0.344828, 0.137931) (0, 1) 
%\psline (-0.344828, 0.137931) (-0.300000, 0.100000) 
%\psline (-0.326258, 0.184355) (-0.372682, 0.202924) (-0.391251, 0.156501) 

%\psline (-3, 1) (0, 0) 
%\psline (-0.300000, 0.100000) (0, 1) 
%\psline (-0.300000, 0.100000) (-0.264151, 0.075472) 
%\psline (-0.284189, 0.147434) (-0.331623, 0.163246) (-0.347434, 0.115811)


%Function formula: - (- x^{2}+1/4)^{1/2}+1/2 
%\psplot[linecolor=blue, plotpoints=1000]{-0.5}{0.5}{0.5 0.2500000 x 2 exp -1 mul add 0.5 exp -1 mul add }

%Function formula: (- x^{2}+1/4)^{1/2}+1/2 
%\psplot[linecolor=blue, plotpoints=1000]{-0.5}{0.5}{0.5 0.2500000 x 2 exp -1 mul add 0.5 exp add }

\uncover<3,6>{
\psline[linecolor=red](1, 1) (0, 0)(0,1)(1,1) 
}
\uncover<4>{
\psline[linecolor=red](0.5, 1) (0, 0)(0,1)(0.5,1) 
}

\uncover<7>{
\psline[linecolor=red](0.5, 0.5) (0, 0)(0,1)(0.5,0.5) 
}
\uncover<8>{
\psline[linecolor=green](0,0)(0,1)
\psline[linecolor=red](0,0)(1,1)
}
\uncover<9>{
\psline[linecolor=green](0,0)(0.5,0.5)
\psline[linecolor=red](0,0)(0,1)
}
\uncover<15>{
\psline[linecolor=red](0,0)(0.5, 0.5)(0.4,0.8)(0,0)
}
\uncover<16>{
\psline[linecolor=red](0,0)(1, 1)(0.5,1)(0,0)
}
\uncover<17,18>{
\psline[linecolor=green](0.5, 0.5)(0.4, 0.8) 
\psline[linecolor=red](0.5, 1)(1, 1) 
} 
\uncover<18>{
\psline[linecolor=green](0.5, 0.5)(0.0, 0.0) 
\psline[linecolor=red](0, 0)(0.5, 1) 
}
\end{pspicture}
\only<1-26>{
Draw a unit circle as above, let $O, A$ be as indicated. \uncover<2->{Let $P_2$ be the point $(\alert<2>{x_2},1)$, $P_1$ be the point $(x_2-\alert<2>{\Delta},1)$.} \uncover<3->{By the Pythagorean theorem, $\alert<25>{|OP_2|^2= 1 + x_2^2}$} \uncover<4->{and similarly $ |OP_1 |^2=1+(x_2-\Delta )^2$.} \uncover<5->{Let \alert<5>{$Q_1$}, \alert<5>{$Q_2$} be as indicated.} \uncover<6->{Then $\alert<6>{ \triangle OP_2A} $ is similar to $\alert<7>{\triangle OAQ_2} $.} \uncover<8->{By Euclidean geometry, $\alert<8>{ \frac{ \only<1-8>{{\color{green} |OA|}} \only<9->{|OA|} }{ |OP_2| }} =\alert<9>{ \frac{\only<9>{{\color{green}|OQ_2|}} \only<8,10->{|OQ_2|} }{|OA|}}$} \uncover<10->{ and so $\alert<12,24>{|OQ_2| |OP_2| }= |OA|^2 \alert<12,24>{ =1}$ \uncover<23->{ and therefore $\alert<23>{ \frac{|OQ_2|}{|OP_2|}} = \frac{ \alert<24>{|OQ_2| \alert<23>{| OP_2 |}} }{|OP_2|^{\alert<23>{2} }} \uncover<24->{= \frac{ \alert<24>{ 1} }{ \alert<25>{|OP_2|^2}} } \uncover<25->{ = \frac{1}{\alert<25>{ 1+x_2^2}}.}$}
} 
%The points $Q_2$, $P_2$ are often called ``inverse points w.r.t. the unit circle''.
\uncover<11->{Similarly conclude $\alert<13>{ |OQ_1|} \alert<14>{|OP_1|} =|OA|^2= \alert<12>{1} \uncover<12->{ \alert<12>{\alert<13,14>{=} \alert<14>{ |OQ_2|}\alert<13>{ |OP_2| }}.}$} 
\uncover<13->{Therefore $\alert<13>{ \frac{|OQ_1|}{  |OP_2|} } \alert<13,14>{=} \alert<14>{\frac{|OQ_2| }{ |OP_1|}} $} \uncover<15->{and so $ \alert<15>{\triangle OQ_2Q_1} $ is similar to $\alert<16>{\triangle OP_1P_2} $.} \uncover<17->{Therefore $\frac{ \only<17>{{\color{green}|Q_1Q_2|}} \only<18->{\alert<19>{ |Q_1Q_2|}} } { \alert<17,20>{|P_1P_2|} }= \alert<20>{ \frac{ \only<18>{\color{green}|OQ_2|} \only<1-17,19->{ |OQ_2|} }{ \alert<18>{|OP_1|} }}$} \uncover<19->{and so }
}
\uncover<19->{\alert<26,27>{\noindent $\alert<19>{|Q_1Q_2|} =\alert<20>{\frac{|P_1P_2| |OQ_2|} {|OP_1|}} \uncover<21->{=\left(\frac{\alert<21>{|OP_2|} }{|OP_1|}\right) \alert<23,24,25>{\frac{|OQ_2|} {\alert<21>{ |OP_2|}}} \alert<22>{ |P_1P_2|} } \uncover<22->{= \frac{|OP_2| } {|OP_1|} \frac{\alert<22>{ \Delta} }{\alert<23,24,25>{ 1+x_2^2}}.}$}}
\end{frame}
%\end{comment}
%\begin{comment}
\begin{frame}[t]
\psset{xunit=2cm, yunit=2cm}
\begin{pspicture}(-3.1, -0.1)(3.1,1.2) 
\psframe*[linecolor=white](-3.1,-0.1)(3.1,1.2) 
\tiny 
\psline[linecolor=red!1](0,1.2)(0.001,1.2) %bounding boxes don't always work right
\uncover<48->{
\pscustom*[linecolor=\psColorAreaUnderGraph]{
%Function formula: \frac{1}{x^{2}+1} 
\psplot[linecolor=\psColorGraph, plotpoints=1000]{-3.000000}{3.000000}{1.0000000 1.0000000 x 2.0000000 exp add div }
\psline(3.000000, 0)(-3.000000, 0)
}

%Function formula: \frac{1}{x^{2}+1} 
\psplot[linecolor=\psColorGraph, plotpoints=1000]{-3.000000} {3.000000} {1.0000000 1.0000000 x 2.0000000 exp add div }
}

\psaxes[arrows=<->, ticks=none, labels=none](0,0)(-3.05,-0.1)(3.05,1.1)
%Function formula: - (- x^{2}+1)^{1/2} 
%\psplot[linecolor=\psColorGraph, plotpoints=1000]{-1}{1}{1 x 2 exp -1 mul add 0.5 exp -1 mul }
%Function formula: (- x^{2}+1)^{1/2} 
\psplot[linecolor=blue, plotpoints=1000]{-1}{1}{1 x 2 exp -1 mul add 0.5 exp }
\psline(-3,1)(3,1) 
%calculator commands:
%\Delta:=0.5;
%xInv{}{{x}}:=DoubleValue{}x/(1+x^2);
%yInv{}{{x}}:=DoubleValue{} 1/(1+x^2);
%xDir{}{{x}}:=-xInv{}x;
%yDir{}{{x}}:=1-yInv{}x;
%lengthN{}{{x}}:=((xDir{}x)^2+(yDir{}x)^2)^{1/2};
%xDirN{}{{x}}:=0.05 xDir{}x /lengthN{}x ;
%yDirN{}{{x}}:=0.05 yDir{}x/lengthN{}x;
%xT{}{{x}}:=-yDirN{}x;
%yT{}{{x}}:=xDirN{}x;

%inversePoint{}{{x}}:=(xInv{}x,yInv{}x );
%f{}{{x}}:= \psline(DoubleValue{}x,1 )(0,0) \psline inversePoint{}x (0,1)  \psline inversePoint{}x inversePoint{}(x+\Delta) \psline (xInv{}x +xDirN{}x, yInv{}x+yDirN{}x ) (xInv{}x +xDirN{}x+xT{}x, yInv{}x+yDirN{}x +yT{}x)(xInv{}x +xT{}x, yInv{}x +yT{}x);
%f{}(\Delta) f{}(2\Delta)f{}(3\Delta)f{}(4\Delta)f{}(5\Delta)f{}(6\Delta)

\uncover<1-33>{
\psline(0.5, 1)(0, 0)
}
\uncover<34-41>{
\psline(0.4, 0.8)(0, 0)
} 
\psline(0.4, 0.8)(0, 1) 
\uncover<28,34>{
\psline[linecolor=red](0.4, 0.8)(0, 1) 
}
\uncover<1-41>{
\psline(0.355279, 0.822361)(0.332918, 0.777639)(0.377639, 0.755279) 
}
\uncover<22>{
\psline[linecolor=red](0.4, 0.8)(0, 1) 
\psline[linecolor=red](0.355279, 0.822361)(0.332918, 0.777639)(0.377639, 0.755279) 
}
\uncover<7-12>{
\psline(0.600000, 1)(0, 0) 
\psline(0.441176, 0.735294)(0, 1) 
\psline(0.441176, 0.735294)(0.4, 0.8) 
\psline(0.398302, 0.761019)(0.372577, 0.718144)(0.415452, 0.692419) 
}
\rput[lt](0.46, 0.71){\uncover<7-12>{$Q_2$}}
\uncover<6>{
\psline(0.700000, 1)(0, 0) 
\psline(0.469799, 0.671141)(0, 1) 
\psline(0.469799, 0.671141)(0.4, 0.8)
\psline(0.428837, 0.699814)(0.400164, 0.658852)(0.441126, 0.630179)
}
\rput[lt](0.48, 0.65){\uncover<6>{$Q_2$}}
\uncover<5>{
\psline(0.8, 1)(0, 0) 
\psline(0.487805, 0.609756)(0, 1) 
\psline(0.487805, 0.609756)(0.4, 0.8)
\psline(0.448761, 0.640991)(0.417527, 0.601947)(0.456570, 0.570713) 
}
\rput[lt](0.5, 0.58){\uncover<5>{$Q_2$}}
\uncover<4>{
\psline(0.900000, 1) (0, 0) 
\psline(0.497238, 0.552486)(0, 1) 
\psline(0.497238, 0.552486)(0.4, 0.8)
\psline(0.460073, 0.585934)(0.426625, 0.548770)(0.463789, 0.515321)
}
\rput[lt](0.5, 0.54){\uncover<4>{$Q_2$}}

\uncover<1-3>{ %
\psline(1, 1)(0, 0) 
\psline(0.4, 0.8)(0.5, 0.5) 
} %
\rput[lt](0.520000, 0.4500000){\uncover<1-3>{$Q_2$}}
\uncover<13->{ %
\psline(0.4, 0.8)(0.5, 0.5) 
} %
\rput[lt](0.520000, 0.4500000){\uncover<13->{$Q_2$}}
\uncover<13-33>{ %
\psline(1, 1)(0, 0)
} 
\uncover<34-41>{
\psline(0.5, 0.5)(0, 0)
}
\uncover<30,34>{
\psline[linecolor=red](0.4, 0.8)(0.5, 0.5) 
}
\uncover<32->{
\psline(0.5, 0.5)(0.461538, 0.307692)
}
\uncover<32,34>{
\psline[linecolor=red](0.5, 0.5)(0.461538, 0.307692)
}
\uncover<1-3, 13-41>{
\psline(0.5, 0.5)(0, 1) 
\psline(0.464645, 0.535355)(0.429289, 0.5)(0.464645, 0.464645) 
}
\uncover<23>{
\psline[linecolor=red](0.5, 0.5)(0, 1) 
\psline[linecolor=red](0.464645, 0.535355)(0.429289, 0.5)(0.464645, 0.464645) 
}
\uncover<14>{ %
\psline[linecolor=red](0,0)(0,1)
} %
\uncover<15>{
\psline[linecolor=red](0,0)(0.5,1)
}
\uncover<16>{
\psline[linecolor=red](0,0)(1,1)
}
\uncover<17-33>{
\psline(1.5, 1)(0, 0) 
}
\uncover<34-41>{
\psline(0.461538, 0.307692)(0, 0) 
}
\uncover<17>{
\psline[linecolor=red](1.500000, 1)(0, 0) 
}
\uncover<32->{
\psline(0.461538, 0.307692)(0.4, 0.2) 
}
\uncover<32,34>{
\psline[linecolor=red](0.461538, 0.307692)(0.4, 0.2) 
}
\uncover<24-41>{
\psline(0.461538, 0.307692)(0, 1) 
\psline(0.433803, 0.349295)(0.392201, 0.321560)(0.419936, 0.279957) 
}
\uncover<24>{
\psline[linecolor=red](0.461538, 0.307692)(0, 1) 
\psline[linecolor=red](0.433803, 0.349295)(0.392201, 0.321560)(0.419936, 0.279957) 
}
\uncover<18-33>{ 
\psline(2, 1)(0, 0) 
}
\uncover<34-41>{
\psline(0.4, 0.2)(0, 0) 
}
\uncover<18>{ 
\psline[linecolor=red](2, 1)(0, 0) 
}
\uncover<32->{
\psline(0.4, 0.2)(0.344828, 0.137931) 
}
\uncover<32,34>{
\psline[linecolor=red](0.4, 0.2)(0.344828, 0.137931) 
}
\uncover<25-41>{
\psline(0.4, 0.2)(0, 1) 
\psline(0.377639, 0.244721)(0.332918, 0.222361)(0.355279, 0.177639) 
}
\uncover<25>{
\psline[linecolor=red](0.4, 0.2)(0, 1) 
\psline[linecolor=red](0.377639, 0.244721)(0.332918, 0.222361)(0.355279, 0.177639) 
}
\uncover<19-33>{ 
\psline(2.5, 1)(0, 0) 
}
\uncover<34-41>{ 
\psline(0.344828, 0.137931)(0, 0) 
}
\uncover<19>{ 
\psline[linecolor=red](2.5, 1)(0, 0)
}
\uncover<32->{
\psline(0.344828, 0.137931)(0.300000, 0.100000) 
}
\uncover<32,33,34>{
\psline[linecolor=red](0.344828, 0.137931)(0.300000, 0.100000) 
}
\uncover<26-41>{
\psline(0.344828, 0.137931)(0, 1) 
\psline(0.326258, 0.184355)(0.279834, 0.165785)(0.298404, 0.119362) 
}
\uncover<26>{
\psline[linecolor=red](0.344828, 0.137931)(0, 1) 
\psline[linecolor=red](0.326258, 0.184355)(0.279834, 0.165785)(0.298404, 0.119362) 
}
\uncover<20-33>{ 
\psline(3, 1)(0, 0) 
}
\uncover<34-41>{
\psline(0.300000, 0.100000)(0, 0) 
}
\uncover<20>{ 
\psline[linecolor=red](3, 1)(0, 0) 
}
%\psline(0.300000, 0.100000)(0.264151, 0.075472) 
\uncover<27-41>{
\psline(0.300000, 0.100000)(0, 1) 
\psline(0.284189, 0.147434)(0.236754, 0.131623)(0.252566, 0.084189)
}
\uncover<27->{
\rput[lt](0.3,0.095){$Q_n$}
}
\uncover<27>{
\psline[linecolor=red](0.300000, 0.100000)(0, 1) 
\psline[linecolor=red](0.284189, 0.147434)(0.236754, 0.131623)(0.252566, 0.084189)
}

\uncover<44>{
\psline(-0.5, 1)(0, 0) 
\psline(-0.4, 0.8)(0, 1) 
\psline(-0.355279, 0.822361)(-0.377639, 0.867082)(-0.422361, 0.844721) 

\psline(-1, 1)(0, 0) 
\psline(-0.5, 0.5)(0, 1) 
\psline(-0.464645, 0.535355)(-0.5, 0.570711)(-0.535355, 0.535355) 

\psline(-1.500000, 1) (0, 0) 
\psline(-0.461538, 0.307692)(0, 1) 
\psline(-0.433803, 0.349295)(-0.475406, 0.377030)(-0.503141, 0.335427) 

\psline(-2, 1) (0, 0) 
\psline(-0.4, 0.2)(0, 1) 
\psline(-0.377639, 0.244721)(-0.422361, 0.267082)(-0.444721, 0.222361) 

\psline(-2.5, 1) (0, 0) 
\psline(-0.344828, 0.137931)(0, 1) 
\psline(-0.326258, 0.184355)(-0.372682, 0.202924)(-0.391251, 0.156501) 

\psline(-3, 1)(0, 0) 
\psline(-0.300000, 0.100000)(0, 1) 
%\psline (-0.300000, 0.100000) (-0.264151, 0.075472) 
\psline(-0.284189, 0.147434)(-0.331623, 0.163246)(-0.347434, 0.115811)
}
\uncover<44->{
\psline(-0.4, 0.8)(-0.5, 0.5) 
\psline(-0.5, 0.5)(-0.461538, 0.307692) 
\psline(-0.461538, 0.307692)(-0.4, 0.2) 
\psline(-0.4, 0.2)(-0.344828, 0.137931) 
\psline(-0.344828, 0.137931)(-0.300000, 0.100000) 
}

%Function formula: - (- x^{2}+1/4)^{1/2}+1/2 
\uncover<42->{\psplot[linecolor=red, plotpoints=1000]{-0.5}{0.5}{0.5 0.2500000 x 2 exp -1 mul add 0.5 exp -1 mul add }

%Function formula: (- x^{2}+1/4)^{1/2}+1/2 
\psplot[linecolor=red, plotpoints=1000]{-0.5}{0.5}{0.5 0.2500000 x 2 exp -1 mul add 0.5 exp add }
}
\uncover<8-12>{
\psline[linecolor=red](0,0)(0.6,1)
\psline[linecolor=green](0,0)(0.5,1)
}
\uncover<4>{
\rput[lb](0.9, 1.05){$P_2$}
\rput[t](0.7,0.95){$\Delta$}
}
\uncover<5>{
\rput[lb](0.8, 1.05){$P_2$}
\rput[t](0.65,0.95){$\Delta$}
}
\uncover<6>{
\rput[lb](0.7, 1.05){$P_2$}
\rput[t](0.6,0.95){$\Delta$}
}
\uncover<7-12>{
\rput[lb](0.6, 1.05){$P_2$}
\rput[t](0.55,0.95){$\Delta$}
}
\uncover<1-3,13->{
\rput[lb](1, 1.05){\uncover<1-33>{$P_2\alert<16>{(x_2, 1)}$}}
\rput[t](0.75,0.95){$\Delta$}
}
\rput[rb](3, 1.05){$\uncover<20-33>{\alert<20>{P_n}}$}
\rput[lb](2, 1.05){
$\uncover<17-33>{\alert<17>{\dots}}$
}

\rput[lb](0.5, 1.05){\uncover<1-33>{$P_1\uncover<15->{\alert<15>{(x_1 ,1)}}$}}
\rput[t](3, -0.1){$x$}
\rput[br](-0.05, 1.05){$\uncover<21->{\alert<21>{Q_0= }}\uncover<14->{ \alert<14>{ P_0=}}A$}
\rput[l](0.42000, 0.8){$Q_1$}
\rput[tl](0.05, -0.05){$O$}
\end{pspicture}
\alert<1>{ \noindent $\alert<28>{|Q_1Q_2| =} \alert<29>{\frac{|OP_2| } {|OP_1|}} \alert<28>{\frac{ \Delta} {1+x_2^2}}. \vphantom{={\frac{|P_1P_2| |OQ_2|} {|OP_1|}} {=\left(\frac{|OP_2|}{|OP_1|}\right) \frac{|OQ_2|} { |OP_2|} |P_1P_2| } }$} \uncover<13->{For any $\varepsilon >0$, can choose $\Delta$: $\alert<29>{1< \frac{|OP_2| } {|OP_1|} < 1 + \varepsilon}$.}

\only<1-12>{ \uncover<2->{If we let $P_1\to P_2$}\uncover<3->{, i.e., $\Delta \to 0$,} \uncover<8->{we get $\alert<8>{ \frac{|OP_2| }{ |OP_1|}\to 1}$.} \uncover<9->{In strict mathematical language: for every $\varepsilon>0$ there exists $\delta >0$ such that when $\Delta  < \delta$ we have that $1>\frac{ |OP_2| }{|OP_1|}>1-\varepsilon $.} \uncover<10->{Furthermore, the choice of $\delta$ can be made independent of the value of $x_2$: } \uncover<11->{to prove that one analyzes the expression $\frac{|OP_2|}{|OP_1|} = \sqrt{ \frac{ 1+x_2^2} { 1 + (x_2-\Delta )^2}}$.} \uncover<12->{We leave the tedious but otherwise easy details to the interested student. } 
}

\only<13-27>{
\uncover<13->{Fix a large number $N$ and let $\Delta$ be such that $n= \frac{N}{\Delta} $ is integer.} \uncover<14->{ Let  $\alert<14>{P_0=(0,1)} $, $\alert<15>{P_1=(\Delta, 1 )}$, $\alert<16>{P_2=(2\Delta,1)}, \dots, \alert<20>{P_n= (n \Delta,1)}$}\uncover<21->{, and let $\alert<21>{Q_0}, \alert<22>{ Q_1},\alert<23>{Q_2}, \dots, \alert<27>{Q_n}$ be as indicated.} 
}

\uncover<28->{
$
\begin{array}{rcrcl}
\only<1-39>{
\alert<28>{ \frac{\Delta}{1+x_1^2 } }&<\phantom{=}&\alert<28>{ |Q_0Q_1|} &< \phantom{=} & \alert<29>{ (1+ \varepsilon )} \alert<28>{\frac{ \Delta}{1+x_1^2}}  \\
\uncover<30->{\alert<30>{ \frac{\Delta}{1+x_2^2 }} &<\phantom{=} &\alert<30>{|Q_1Q_2|} &<\phantom{=} & \alert<31>{ (1 + \varepsilon)} \alert<30>{\frac{ \Delta }{1+ x_2^2}}}  \\
\uncover<32->{ &\alert<32>{\vdots}} \\
\uncover<33->{\frac{\Delta}{1+x_n^2 } &<\phantom{=} &\alert<33>{ |Q_{n-1}Q_n|} &< \phantom{=} &(1+\varepsilon) \frac{\Delta}{1+x_n^2}} \uncover<34->{\\ \hline} 
\uncover<34->{ \sum_{i=1}^n\frac{\Delta}{1+x_i^2 } &< \phantom{=}&\alert<34>{ \sum_{i=1}^n |Q_{i-1} Q_i|} & <\phantom{=}& (1 +\varepsilon)\sum_{i=1}^n \frac{\Delta}{1+x_i^2} }\\
\uncover<35->{ \downarrow  &&\downarrow&&\downarrow}\\
}
\uncover<35->{ \alert<39,40>{\int_{0}^{\uncover<37->{\alert<37>{\infty}} \uncover<35,36>{N}} \frac{\diff x}{1+x^2}} & \uncover<1-38>{<} \uncover<39->{\alert<39,40>{=}}& \alert<39,40>{  \lim\limits_{\Delta\uncover<37->{, \alert<37>{N} }\uncover<39->{, \alert<39>{\varepsilon}}} \sum| Q_{i-1} Q_i|} \uncover<1-39>{ &\uncover<1-38>{<} \uncover<39->{\alert<39>{=}}& \uncover<1-38>{(1 + \varepsilon )} \int_0^{\uncover<37->{\alert<37>{\infty}} \uncover<35,36>{ N}} \frac{ \diff x}{1+x^2}}}
\end{array}
$
}

\only<1-39>{
\uncover<35->{Let $\Delta\to 0$.} \uncover<36->{Next take $\alert<36,37>{N\to \infty}$.} \uncover<38->{Finally take \alert<38,39>{$\varepsilon\to 0$}\uncover<39->{, use squeeze thm.}} 
}

\uncover<41->{
The points $Q_1, Q_2, \dots$ see the segment $OA$ from an angle of $\frac{\pi}{2}$. }\uncover<42->{Therefore, by Euclidean geometry, the points $Q_1, Q_2,\dots $ lie on the circle $C$ with radius $\frac{1}{2}$ and center $(0,\frac{1}{2}) $. } \uncover<43->{Therefore $ \sum |Q_{i-1}Q_{i}| $ approximates half of the circumference of the circle $C$.} 
\uncover<44->{\alert<44,45>{By symmetry,}
\uncover<46->{
\[
\alert<47>{\int_{-\infty}^{\infty} \frac{\diff x}{1+x^2} }= \text{ circumference of }C  \uncover<47->{=2\pi \left(\frac{1}{2}\right) \alert<47>{=\pi},}
\] 
\uncover<47,48->{as desired.}
}
}
\end{frame}
%\end{comment}
\begin{comment}
\begin{frame}[t]
\psset{xunit=2cm, yunit=2cm}
\begin{pspicture}(-3.1, -0.1)(3.1,1.2) 
\psframe*[linecolor=white](-3.1,-0.1)(3.1,1.2) 
\tiny 
%\psline[linecolor=red!1](0,1.2)(0.001,1.2) %bounding boxes don't always work right
%\pscustom*[linecolor=\psColorAreaUnderGraph]{
%Function formula: \frac{1}{x^{2}+1} 
%\psplot[linecolor=\psColorGraph, plotpoints=1000]{-3.000000}{3.000000}{1.0000000 1.0000000 x 2.0000000 exp add div }
%\psline(3.000000, 0)(-3.000000, 0)
%}

%Function formula: \frac{1}{x^{2}+1} 
%\psplot[linecolor=\psColorGraph, plotpoints=1000]{-3.000000} {3.000000} {1.0000000 1.0000000 x 2.0000000 exp add div }

\psaxes[arrows=<->, ticks=none, labels=none](0,0)(-3.05,-0.1)(3.05,1.1)
%Function formula: - (- x^{2}+1)^{1/2} 
%\psplot[linecolor=\psColorGraph, plotpoints=1000]{-1}{1}{1 x 2 exp -1 mul add 0.5 exp -1 mul }
%Function formula: (- x^{2}+1)^{1/2} 
\psplot[linecolor=blue, plotpoints=1000]{-1}{1}{1 x 2 exp -1 mul add 0.5 exp}
\psline(-3,1)(3,1) 

%Function formula: - (- x^{2}+1/4)^{1/2}+1/2 
\psplot[linecolor=red, plotpoints=1000] {-0.5}{0.5}{0.5 0.2500000 x 2 exp -1 mul add 0.5 exp -1 mul add }
%Function formula: (- x^{2}+1/4)^{1/2}+1/2 
\psplot[linecolor=red, plotpoints=1000]{-0.5}{0.5}{0.5 0.2500000 x 2 exp -1 mul add 0.5 exp add }


%calculator commands:
%\Delta:=0.5;
%xInv{}{{x}}:=DoubleValue{}x/(1+x^2);
%yInv{}{{x}}:=DoubleValue{} 1/(1+x^2);
%xDir{}{{x}}:=-xInv{}x;
%yDir{}{{x}}:=1-yInv{}x;
%lengthN{}{{x}}:=((xDir{}x)^2+(yDir{}x)^2)^{1/2};
%xDirN{}{{x}}:=0.05 xDir{}x /lengthN{}x ;
%yDirN{}{{x}}:=0.05 yDir{}x/lengthN{}x;
%xT{}{{x}}:=-yDirN{}x;
%yT{}{{x}}:=xDirN{}x;

%inversePoint{}{{x}}:=(xInv{}x,yInv{}x );
%f{}{{x}}:= \psline(DoubleValue{}x,1 )(0,0) \psline inversePoint{}x (0,1)  \psline inversePoint{}x inversePoint{}(x+\Delta) \psline (xInv{}x +xDirN{}x, yInv{}x+yDirN{}x ) (xInv{}x +xDirN{}x+xT{}x, yInv{}x+yDirN{}x +yT{}x)(xInv{}x +xT{}x, yInv{}x +yT{}x);
%f{}(\Delta) f{}(2\Delta)f{}(3\Delta)f{}(4\Delta)f{}(5\Delta)f{}(6\Delta)

\psline(0.5, 1)(0, 0)
\psline(0.4, 0.8)(0, 0)

\psline(1, 1)(0, 0) 
\psline(0.4, 0.8)(0.5, 0.5)
\psline[linecolor=red](0.4, 0.8)(0.5, 0.5) 
\psline(0.5, 0.5)(0, 0)

\psline(0.5,0)(0.5,0.316228)(1, 0.316228)(1,0)
\psline[linecolor=red](0.5,0 )(0.5,0.316228)


\rput[lb](1, 1.05){$P_2(x_2, 1)$}
\rput[t](0.75,0.95){$\Delta$}

\rput[lb](0.5, 1.05) {  $P_1  (x_1 ,1)$}
\rput[t](3, -0.1){$x$}
\rput[lt](0.520000, 0.4500000){$Q_2$}
\rput[l](0.42000, 0.8){$Q_1$}
\rput[tl](0.05, -0.05){$O$}
\end{pspicture}

We finish with illustration of the integral $\int_{-\infty}^{\infty} \frac{1}{1+x^2}\diff x $. Recall $|Q_1Q_2|\approx \frac{\Delta}{1+x_2^2} $. 

\end{frame}
\end{comment}
%end module improper-integral-type1-arctan-geometric-interpretation
