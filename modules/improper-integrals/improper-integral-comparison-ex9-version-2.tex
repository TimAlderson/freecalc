% begin module improper-integral-comparison-ex9
\begin{frame}
\vskip -0.15cm
\begin{example}[Show $\ds \alertNoH{3}{ \int_0^\infty \alertNoH{2}{ e^{-x^2}} \diff x}$ is convergent.] 

\begin{itemize}
\item<4->  The antiderivative of $e^{-x^2}$ isn't an elementary function.
\item<5-> If integral were $ \alertNoH{7}{\int_{0}^{ \infty } \alertNoH{6}{e^{-x} }\diff x}$, we'd have no problem integrating. 
\item<8-> Notice that $ \alertNoH{27}{ 0\leq \alertNoH{8,18}{ \alertNoH{9}{ \alertNoH{ 12}{ e}^{\alertNoH{11}{-x^2}}} \leq \alertNoH{10 }{ \alertNoH{12}{ e}^{\alertNoH{11}{-x}}} }} $ \alertNoH{27}{for $x\geq 1$} \uncover<11->{(because \alertNoH{11}{$-x^2<-x$} for $x>1$ and \alertNoH{12}{the exponential is an increasing function}).}
\item<13-> Split: $\ds \alertNoH{28}{ \int_{\alertNoH{13}{0}}^{ \alertNoH{13}{ \infty}} e^{-x^2}\diff x = \alertNoH{14,16}{ \int_{\alertNoH{13}{0} }^{ \alertNoH{13}{1} } e^{-x^2}\diff x} + \alertNoH{15,17}{\int_{ \alertNoH{ 13}{1} }^{\alertNoH{13}{\infty}} e^{-x^2}\diff x}}$.
\item<16->  On the RHS, \alertNoH{16,28}{first integral is proper} - no effect on convergence.
\end{itemize}
\vskip -0.15cm
\begin{columns}[c]



\column{.55\textwidth}
$\begin{array}{@{\!\!\!}r@{}c@{}l}
\displaystyle \uncover<17->{\alertNoH{17}{ \int_{1}^{\infty}  \alertNoH{18}{e^{-x^2}}\diff x}} &\uncover<18->{ \alertNoH{18 }{\leq}} &\displaystyle  \uncover<18->{\alertNoH{19,27}{  \int_1^{\alertNoH{20}{\infty}} \alertNoH{18}{ e^{-x}}\diff x}}  \uncover<20->{ = } \displaystyle  \uncover<20->{\lim\limits_{\alertNoH{20}{t\rightarrow \infty}} {\alertNoH{21,22}{\int}}_{\!\!\!1}^{ \alertNoH{ 20}{t}} \alertNoH{21,22}{e^{-x}\diff x}}\\[3mm]%
& \uncover<21->{ = } &\displaystyle  \uncover<21->{ \lim_{ t \to \infty} \left[ \fcAnswer{22}{ -e^{- \alertNoH{22, 23} { x}}} \right]_{ \alertNoH{ 24}{1} }^{ \alertNoH{23}{t}}}\\%
& \uncover<23->{ = } &\displaystyle  \uncover<23->{ \alertNoH{25}{\lim_{t \to\infty}}  \alertNoH{25}{ \alertNoH{26}{} \left( \alertNoH{26}{ \alertNoH{24}{\frac{1}{e}}} {-\alertNoH{23}{\frac{1}{e^t}}}  \right)}} %
  \uncover<25->{ \alertNoH{27}{=} } \displaystyle  \uncover<25->{ \alertNoH{ 27 }{ \frac{1}{e}} }%
\end{array}$
\column{.43\textwidth}
\psset{xunit=1.5cm, yunit=1.5cm}
\begin{pspicture}(-0.5,-0.5)(3.2,1.3)
\psframe*[linecolor=white](-0.5,-0.5)(3.200000,1.3)
\tiny
\uncover<3->{%
\pscustom*[linecolor=orange]{%
\psplot[linecolor=\fcColorGraph, plotpoints=1000] {1} {3}{2.718281828 x 2.0000000 exp -1.0000000 mul exp }%
\psline( 3,0)(1,0)%
}%
}%
\uncover<3->{%
\pscustom*[linecolor=orange]{%
\psplot[linecolor=\fcColorGraph, plotpoints=1000] {0} {1}{2.718281828 x 2 exp -1 mul exp }%
\psline( 1,0)(0,0)%
}%
}%
\uncover<handout:0|14,16>{%
\pscustom*[linecolor=purple]{%
\psplot[linecolor=\fcColorGraph, plotpoints=1000] {0} {1}{2.718281828 x 2 exp -1 mul exp }%
\psline( 1,0)(0,0)%
}%
}%

%Function formula: e^{- x^{2}}
\uncover<2->{
\psplot[linecolor=\fcColorGraph, plotpoints=1000] {0} {3}{2.718281828 x 2 exp -1 mul exp }%
\rput[lb](0.5, 0.9){$y=e^{-x^2}$}%
}
\uncover<7,19->{
\pscustom*[linecolor=cyan]{
%Function formula: e^{- x}
\psplot {1}{3}{2.718281828 x -1.0000000 mul exp }
\psline(3, 0 )(1,0)
}
}
\uncover<handout:1|15,17->{%
\pscustom*[linecolor=purple]{%
\psplot[linecolor=\fcColorGraph, plotpoints=1000] {1} {3}{2.718281828 x 2.0000000 exp -1.0000000 mul exp }%
\psline( 3,0)(1,0)%
}%
}%
\uncover<handout:0|7>{
\pscustom*[linecolor=cyan]{
%Function formula: e^{- x}
\psplot {0.000000}{1}{2.718281828 x -1.0000000 mul exp }
\psline(1, 0 )(0,0)
}
}
\uncover<6->{
%Function formula: e^{- x}
\psplot[linecolor=blue, plotpoints=1000] {0.000000}{3.000000}{2.718281828 x -1.0000000 mul exp }
\rput[t](0.5, 0.45){$y=e^{-x}$}
}
\uncover<9-12,18>{
\psplot[linecolor=\fcColorGraph, plotpoints=1000, linewidth=2pt] {1} {3}{2.718281828 x 2 exp -1 mul exp }%
%\rput[lb](0.5, 0.9){$y=e^{-x^2}$}%
}
\uncover<10-12,18>{
%Function formula: e^{- x}
\psplot[linecolor=blue, linewidth=2pt, plotpoints=1000] {1}{3}{2.718281828 x -1 mul exp }
}
\fcAxesStandardNoFrame{-0.3}{-0.3}{3}{1.15}
\fcXTickWithLabel{1}{$1$}
\end{pspicture}
\end{columns}
\uncover<27->{%
\alertNoH{27}{By the Comparison Theorem, $\ds \int_1^\infty e^{-x^2}\diff x$ converges}.\\
 \alertNoH{28}{It follows that  $\ds \int_0^\infty e^{-x^2}\diff x$ also converges.}
}%

\end{example}
\end{frame}
% end module improper-integral-comparison-ex9
