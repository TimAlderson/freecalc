\begin{frame}
  \frametitle{Gravity and Gradient}

  Object at $(x_0,y_0,z_0 = f(x_0,y_0))$, moves along surface $z=f(x,y)$
  \pause

  Force: Component of $\textbf{G}= -mg\; \textbf{k}$ tangent to the surface

  \pause
  Normal to surface:
  %
  $$\textbf{n} = \langle -f_x(x_0,y_0), -f_y(x_0,y_0), 1\rangle = -\nabla f + \textbf{k}$$
  \pause
  $$\textbf{F} = \textbf{orth}_{\bm{n}} \textbf{G} =
  -mg \; \textbf{orth}_{\bm{n}} \textbf{k}$$
  \pause
  $$\textbf{orth}_{\bm{n}} \textbf{k} =
  \textbf{k} - \textbf{proj}_{\bm{n}} \textbf{k} =
  \textbf{k} - \frac{\textbf{k}\cdot \textbf{n}}{|\textbf{n}|^2} \, \textbf{n} =
  \textbf{k} - \frac{1}{|\textbf{n}|^2} (-\nabla f + \textbf{k})
  $$
\pause
Horizontal component of $\textbf{F}$:
%
$$\frac{mg}{1+|\nabla f|^2} (-\nabla f)$$
\pause
Gravity pulls object in the direction of fastest descent.

\pause
\underline{Question}: Is the object moving in that direction?

\end{frame}