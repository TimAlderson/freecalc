\begin{frame}
  \frametitle{Back to Example}

In the example of $f(x,y) = x^4+y^4-4xy$ we have
%
\begin{itemize}
  \item $f_{xx} = 12x^2$;
  \item $f_{xy} = -4$;
  \item $f_{yy}=12y^2$;
  \item $D= f_{xx}f_{yy}-f_{xy}^2 = 144x^2y^2-16$
\end{itemize}

\pause
At the critical points:

\begin{tabular}{|c|c|c|c|c|c|}
    \hline
    % after \\: \hline or \cline{col1-col2} \cline{col3-col4} ...
    $(x_0,y_0)$ & $f_{xx}(x_0,y_0)$ & $f_{yy}(x_0,y_0)$ & $f_{xy}(x_0,y_0)$ & $D(x_0,y_0)$ &  Conclusion \\
    \hline
    (0,0) & 0 & 0 & -4 & $-16 <0$ & Saddle point  \\
    \hline
    (1,1) & 12 & 12 & -4 & $144-16 > 0$ & Local min  \\
    \hline
    (-1,-1) & 12 & 12 & -4 & $144-16 > 0$ & Local min \\
    \hline
  \end{tabular}

\medskip
\pause
In this case it turns out that the two local minimum points are actually global minimum points, because
%
$$f(x,y) = x^4+y^4-4xy = (x^2-1)^2+(y^2-1)^2+2(x-y)^2 -2  \geqslant -2\; .$$

\pause
But in general, global extreme points may not exist.
\end{frame}