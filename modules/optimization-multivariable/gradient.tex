\begin{frame}
\frametitle{Gradient}
\begin{itemize}
\item Let $f$ be a differentiable function.
\item At a given point $P$, in which direction does $f$ increase the fastest?
\item<2-> What is that maximal rate of increase?
\item<3-> It can be shown that if the maximal rate of increase is strictly positive, then it is achieved in exactly one direction. 
\uncover<4->{
\begin{definition}
The \emph{gradient vector} of $f$ at $P$ is the unique vector that has
\begin{itemize}
\item magnitude equal to >the maximal rate of increase of $f$ at $P$.
\item if the magnitude is not zero, then the direction is the one in which $f$ increases the fastest.
\end{itemize}
\end{definition}
}
\item<5-> Notation: $(\nabla f)(P) = (\nabla f)_P = (\textbf{grad} \;f)(P)$
\item<6-> $\nabla f$: gradient vector field \hspace{2cm}
$P \rightsquigarrow (\nabla f)(P)$
\item<7-> $\nabla$: \emph{del} operator \hspace{2cm}
function $f$ $\rightsquigarrow$ vector field $\nabla f$
\end{itemize}

\end{frame}