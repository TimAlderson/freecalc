\solution{\ref{problemFindExtremae^(x+y)UnderRestrictionx^3+y^3=2}
The restriction is $g(x,y)= x^3+y^3-2=0$. We use the method of Lagrange multipliers. We have that $\nabla f= (e^{x+y},e^{x+y})$ and $\nabla g = \left(3x^2, 3y^2 \right)$. We have a local extremum when $\lambda\nabla f= \nabla g$, i.e., when
\[
\begin{array}{rcl}
\lambda e^{x+y}&=& 3x^2\\
\lambda e^{x+y}&=& 3y^2\\
x^3+y^3&=&2
\end{array}
\]
The first two equations imply $y^2=x^2$ which implies $y=\pm x$. 

Case 1. Suppose $y=-x$. Then the last equation $x^3+y^3=2$ reduces to $0=2 $, which has no solutions; this case yields no candidates for maxima and minima.

Case 2. Suppose $y=x$. We substitute into the third equation and solve:

\[
\begin{array}{rcll|l}
2x^3&=&2\\
x^3-1&=&0\\
(x-1)(x^2+x+1)&=&0&&x^2+x+1\neq 0\text{ for all real }x \\
x&=&1
\end{array}
\] 

Therefore $x=1$, $y=1$ is the only critical point obtained by the method of Lagrange multipliers. To find out whether the critical point is a maximum or minimum, we can rewrite our restriction as $y(x)=\sqrt[3]{2-x^3}$ and so $f(x,y(x))= e^{x+\sqrt[3]{2-x^3}}$. Since the exponent is an increasing function, $e^{x+\sqrt[3]{2-x^3}}$ has extrema if and only if the function $x+\sqrt[3]{2-x^3}$ has the same type of extrema. $x+\sqrt[3]{2-x^3}$ has second derivative 
$ -2 x^{4} (- x^{3}+2)^{-\frac{5}{3}}-2 x (- x^{3}+2)^{-\frac{2}{3}} $, which evaluates to $ -4$ when $x=1$. Therefore by the single-variable second derivative criterion $f(x,y(x))= e^{x+\sqrt[3]{2-x^3}}$ has a local maximum and so the critical point is a local maximum. 

We point out that via the equality $f(x,y(x))= e^{x+\sqrt[3]{2-x^3}}$ this problem can be solved without using Lagrange multipliers, however the computations would be longer.

}

\solution{\ref{problemFindExtremay+xUnderRestrictiony^2+y+x^2+x=1}
The restriction is $g(x,y)= y^2+y+x^2+x-1=0$. We use the method of Lagrange multipliers. We have that $\nabla f= (1,1)$ and $\nabla g = \left(2y+1,2x+1 \right)$. We have a local extremum when $\lambda\nabla f= \nabla g$, i.e., when
\[
\begin{array}{rcl}
\lambda &=& (2y+1)\\
\lambda &=& (2x+1)\\
y^2+y+x^2+x-1&=&0
\end{array}
\]
The first two equations imply $y=x$. We substitute that into the last equation to get that $2x^2+2x-1=0$. The solutions to the latter are $x=  \displaystyle \frac{ -2\pm \sqrt{2^2-4\cdot 2\cdot(-1)}}{4} = \frac{-1\pm \sqrt{3}}{2}$. The only restriction on the points $(x,y)$ is that they lie on the curve $y^2 + y+ x^2 +x=1$ (a circle). A circle is a bounded and closed set. We recall that a set in space is bounded if is contained in a ball (with finite radius) and a set in space is closed if it contains all of its boundary points. Therefore $f$ must attain both its minimum and its maximum on it. Therefore the two critical points are maximum and minimum of $f$. Substitution of our answer in $f$ shows that $f$ attains its minimum at $\displaystyle  \left(x,y\right)=\left(\frac{-1-\sqrt{3}}{2}, \frac{-1 - \sqrt{ 3}}{2}\right)$ and its maximum at $\displaystyle \left(x, y\right)= \left( \frac{ -1+\sqrt{3}}{2} , \frac{ -1+\sqrt{3}}{2}\right)$. Our final answer is below.

\begin{tabular}{r|l}
$(x,y)$ & max or min\\\hline
$\left(\frac{-1-\sqrt{3}}{2},\frac{-1-\sqrt{3}}{2}\right) $ & minimum\\
$\left(\frac{-1+\sqrt{3}}{2} , \frac{ -1+\sqrt{3}}{2}\right)$ & maximum\\
\end{tabular}
}