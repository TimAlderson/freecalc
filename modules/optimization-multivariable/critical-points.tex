\begin{frame}
\frametitle{Critical Points}
If $\textbf{u}=(\nabla f)(P_0)$ exists and is non-zero, then
\begin{itemize}
\item $f$ increases along $\textbf{u}$;
\item $f$ decreases along $-\textbf{u}$;
\end{itemize}
\pause
If we can move along $\pm\textbf{u}$ and stay in $D$, then $P_0$ is not an extreme.

\pause
If:
%
\begin{itemize}
  \item $P_0$ is a point of extreme (minimum or maximum);
  %
  \item $P_0$ is an \emph{interior point} of $D$, which means that there exists an open disk centered at $P_0$ and completely included in $D$;
  %
  \item directional derivatives at $P_0$ exist in all directions
\end{itemize}
%
then \pause $(\nabla f)(P_0) = \textbf{0}$. In particular, $f_x(P_0) = f_y(P_0) = 0$.\pause

\smallskip

\underline{Geometric Interpretation}: \pause At an interior point of extreme, the tangent plane to the graph surface is horizontal.\pause

\smallskip

\underline{The converse is not true}: \pause if $f_x(P_0) = f_y(P_0) = 0$, then $P_0$ is not necessarily a point of extreme.


\end{frame}



\begin{frame}
  \frametitle{}

Where else can one find extreme points?\pause

\begin{itemize}
  \item At points $P_0$ where some directional derivatives do not exist (suffices that one of $f_x(P_0)$ or $f_y(P_0)$ does not exist.);
  \item At points $P_0$ in $D$ that are not interior points of $D$.
\end{itemize}

\pause
\underline{Important concept}: A point $P$ in $\mathbb{R}^2$ is a \emph{boundary point} for a region $D$ if every open disk centered at $P$ has points both in $D$ and outside of $D$. Similar definition for $\mathbb{R}^3$, but replace open disk with open ball.

\pause
Examples:
\begin{itemize}
  \item $D$=open unit disk $\Longrightarrow$\pause set of boundary points = unit circle;\pause
  \item $D$=closed unit disk $\Longrightarrow$\pause set of boundary points = unit circle;\pause
\end{itemize}
%
Notice that a boundary point may or may not be included in $D$.

\pause
\underline{Strategy for finding extreme points}:
%
\begin{itemize}
  \item Check the \emph{critical points} of $f$:
  \begin{itemize}
    \item Points $P_0$ for which $f_x(P_0)$ or $f_y(P_0)$ does not exist;
    \item Points $P_0$ for which $f_x(P_0)=f_y(P_0)=0$.
  \end{itemize}
  %
  \item Check boundary points included in the domain.
\end{itemize}
\end{frame}
