\begin{frame}
\begin{itemize}
\item Just as with one-variable derivatives, a number of notations are used/accepted.
\item Notations for partial derivatives:
\[
\begin{array}{rcl}
(D_{\,\fcv{i}}f)(x_0,y_0) &=& \frac{\partial f}{\partial x}(x_0,y_0) \\
&=&
f_x(x_0,y_0)\\
&=&(\partial_x f)(x_0, y_0)\\
(D_{\,\fcv{j}}f)(x_0,y_0) &=&  \frac{\partial f}{\partial y}(x_0,y_0) \\
&=& f_y(x_0,y_0)\\
&=&\left(\partial_y f\right) (x_0, y_0)
\end{array}
\]
\item<2-> By convention, the notation $\frac{\diff }{\diff x}$ implies we are working with one variable only; $\frac{\partial }{\partial x}$ implies we are working with more than one. 
\item<3-> If in doubt about the number of variables - for example, if you intend to convert a parameter of the system into a variable - use the $\frac{\partial}{\partial x}$ notation. 
\item<4-> To compute a partial derivative with respect to a variable:
\begin{itemize}
\item consider all other variables as constants and
\item apply the rules for differentiation for single variable functions.
\end{itemize}
\end{itemize}
\end{frame}