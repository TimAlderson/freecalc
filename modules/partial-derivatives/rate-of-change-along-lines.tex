\begin{frame}
  \frametitle{Rates of Change along Lines}
\begin{itemize}
\item Let $L$ be a line through $P_0(\fcv{r}_0)$.
\begin{center}  How does $f(\fcv{r}) = |\fcv{r}|^2$ change \alert<1->{along $L$}?  
\end{center}
\item<2-> Let $\fcv{r} : \RR \to L$: smooth parametrization of $L$, $\fcv{r}(0) = \fcv{r}_0$
$$g\colon \RR \to \RR, \quad g(t) = f(\fcv{r}(t))$$
\begin{center}
    Rate of change of $f$ along $L$ = rate of change of $g$.
  \end{center}
\item<3-> With respect to $t$:
  $$\lim_{t\to 0} \frac{f(\fcv{r}(t))-f(\fcv{r}(0))}{t} =
  \lim_{t\to 0} \frac{g(t)-g(0)}{t} = g'(0)$$
\item<4-> Still ambiguous: depends on the parametrization $\fcv{r}$.
\item<5-> Solution: arclength parametrization.
\item<6-> Almost solves the problem: orientation still matters.
\end{itemize}
\end{frame}
