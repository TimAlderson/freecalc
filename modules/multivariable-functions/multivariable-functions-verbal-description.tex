\begin{frame}\frametitle{Describing multivariable functions}
\begin{itemize}
\item When doing mathematical modeling, there are several ways to define a function of several variables.
\item<2-> Usually: start with \emph{verbal} description, then give specific meanings
to our input and output variables.
\item<3-> We explain by examples.
\end{itemize}
\end{frame}

\begin{frame}\frametitle{Verbal description examples}
\begin{itemize}
\item The apparent temperature, $W$, felt on exposed skin depends on several factors, including the actual temperature, $T$, the wind speed, $v$, and the humidity.  The \emph{wind chill temperature} is a mathematical model  for $W$ under the assumption that the humidity is 0 and  that the only factors influencing $W$ are $T$ and $v$:
\[
W = W(T,v)\; .
\]
The domain of the function $W$ consists of all reasonable pairs $(T,v)$.
\item<2-> The Cobb-Douglas production function models the production output, $P$, under the assumption that the only factors are the amount of labor, $L$, and the amount of capital, $K$:
\[
P=P(L,K) \; .
\]
\end{itemize}

\vskip 8cm %to serve for spacing
\end{frame}
\begin{frame}\frametitle{Verbal description examples}
\begin{itemize}
\item The magnitude $G$ of the attraction force between two mass points depends on the masses $m$ and $M$ of the bodies and the distance $d$ between them:
\[
G=G(m,M,d) \quad .
\]
\item<2-> A set $(\rho, \phi, \theta)$ of spherical coordinates determines the rectangular coordinates $(x,y,z)$ of a point. In this case, both the input and the output are multidimensional:
\[
(x,y,z) = \textbf{F}(\rho, \theta, \phi)\; ,
\]

\end{itemize}

\vskip 8cm %to serve for spacing

\end{frame}
\begin{frame}
\begin{itemize}
\item The wind velocity $\textbf{v}$ at a point $P$ depends on the position $\textbf{r}$ of $P$,
\[
\textbf{v} = \textbf{V} (\textbf{r})\; .
\]
In this case both the input and the output are vectors.

\item<2-> The electric force on a charge $q$ displaced by $\textbf{r}$ from a charge $Q$ depends on the two charges, the displacement, and the medium in which the charges are placed:
\[
\textbf{E} =\textbf{E}(q, Q,\textbf{r})\; .
\]
Note that in this case the output data is a vector and the input data is a mix of scalars and vectors.
\end{itemize}
\vskip 8cm
\end{frame}
