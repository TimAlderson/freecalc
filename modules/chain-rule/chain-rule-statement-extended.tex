% begin module chain-rule-statement
\begin{frame}
\frametitle{The Chain Rule}

Let $g$ and $h$ be functions. Recall that the composite function $f=g\circ h$ is defined via $f(x) = g(h(x))$.
\begin{theorem} Let $h$ be differentiable at $x$ and let $g$ be a differentiable at $h(x)$. Then the composite function $f = g\circ h$  is differentiable at $x$ and $f'$ is given by the product
\[
\begin{array}{rclll}
\alertNoH{4}{f'(x)} &=&\alertNoH{5}{ g'(\alertNoH{3}{ h(x)}) } \cdot \alertNoH{6}{ {\alertNoH{3}{ h}}'\alertNoH{3}{(x)}}&& \text{(notation 1)}  \\
\uncover<2->{&&&& \text{equivalently:}}\\
\uncover<3->{f'(x)=(g(\alertNoH{3}{u}))'&=&g'(\alertNoH{3}{u}) {\alertNoH{3}{u}}'&\text{where } \alertNoH{3,6}{u=h(x)}& \text{(notation 2)}}\\
\uncover<4->{\displaystyle \alertNoH{4}{\frac{\diff y}{\diff x}} &=& \displaystyle \alertNoH{5}{ \frac{\diff y}{\diff u} } \alertNoH{6}{ \frac{\diff u}{\diff x}} &\text{where } \alertNoH{4}{ y=g(u)}& \text{(notation 3)}\quad.}\\
\end{array}
\]
\end{theorem}
\uncover<7->{The last equality uses the Leibniz notation (due to G. Leibniz (1646-1716)).}
\end{frame}
% end module chain-rule-statement
