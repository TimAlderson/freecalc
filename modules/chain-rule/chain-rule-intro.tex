% begin module chain-rule-intro
\begin{frame}
\frametitle{The Chain Rule}
\begin{itemize}
\item  What is the derivative of $f(x) = \only<handout:0|6>{\color{red}} \sqrt{\only<handout:0|6>{ \color{black}}\alertNoH{8}{ x^2 + 1} }$?
\item<2->  The Power Rule doesn't tell us how to find the derivative.
\item<3->  $f$ is a composite function $g\circ h$:
\item<3-| alert@4-5,9,11-12>  $y = \alertNoH{6}{g(}u \alertNoH{6}{)} = \only<handout:0|6>{\color{red}} \fcAnswerUncover{3}{5}{ \sqrt{\only<handout:0|6>{\color{black}} u}}$.
\item<3-| alert@7-9,13-14>  $u = h(x) = \fcAnswerUncover{3}{8}{ x^2+1}$.
\item<3->  Then $y = f(x) = \alertNoH{9}{ \alertNoH{6}{g(} \alertNoH{ 8}{h(x)}\alertNoH{6}{ ) } =} \uncover<9->{\alertNoH{9}{ g(\alertNoH{ 8}{x^2+1}) }}  \uncover<9->{\alertNoH{ 9}{=\sqrt{x^2+1}}.}$
\item<10->  We know the derivatives of $g$ and $h$:
\item<10-| alert@11-12>  $g'(u) = \fcAnswerUncover{10}{12}{ \frac{1 }{2}u^{-\frac{1}{2}}}$.
\item<10-| alert@13-14>  $h'(x) = \fcAnswerUncover{10}{14}{ 2x}$.
\item<15->  It would be nice if we could find the derivative of $f$ in terms of the derivatives of $y$ and $u$.
\item<16->  It turns out that the derivative of the composition $g\circ h$ is the product of the derivative of $g$ and the derivative of $h$.
\item<17->  This important fact is called the Chain Rule.
\end{itemize}
\end{frame}
% end module chain-rule-intro
