% begin module exponential-function-def
\begin{frame}
\frametitle{(1.5) Exponential Functions}
The function $f(x) = 2^x$ is called an exponential function because the variable $x$ is the exponent.
\begin{columns}[c]
\column{.5\textwidth}
\psset{xunit=0.8cm, yunit=0.8cm}
\begin{pspicture}(-5, -5)(5,5) 
\psframe*[linecolor=white](-5,-5)(5,5) 
\psaxes[labels=none]{<->}(0,0)(-2.5,-0.5)(2.5,6.25)
\rput[t](1, -0.2){$1$}
\uncover<3->{
\psFullDot{2}{4}
}
\uncover<5->{
\psFullDot{1}{2}
}
\uncover<7->{
\psFullDot{0}{1}
}
\uncover<9->{
\psFullDot{-1}{0.5}
}
\uncover<11->{
\psFullDot{-2}{0.25}
}
\uncover<12>{
\rput[r](-0.5, 1.5){$y=2^x$}
%Function formula: 2^{x} 
\psplot[linecolor=red, plotpoints=1000]{-2.5}{2.5}{2 x exp }
}
\end{pspicture}
\column{.5\textwidth}
\[
\begin{array}{r|l}
x & y\\
\hline
\alert<handout:0| 2-3>{2} & \alert<handout:0| 3>{\uncover<3->{4}} \\
\alert<handout:0| 4-5>{1} & \alert<handout:0| 5>{\uncover<5->{2}} \\
\alert<handout:0| 6-7>{0} & \alert<handout:0| 7>{\uncover<7->{1}} \\
\alert<handout:0| 8-9>{-1} & \alert<handout:0| 9>{\uncover<9->{1/2}} \\
\alert<handout:0| 10-11>{-2} & \alert<handout:0| 11>{\uncover<11->{1/4}} 
\end{array}
\]
\uncover<13->{
\begin{definition}[Exponential Function]
In general, an exponential function is a function of the form $f(x) = a^x$, where $a$ is a positive constant.
\end{definition}
}
\end{columns}
\end{frame}
% end module exponential-function-def
