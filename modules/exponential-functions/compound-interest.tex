\begin{frame}
\begin{itemize}
\item In finance, \alertNoH{1}{compound interest} is interest on a deposit which gets added automatically to the deposit so it earns additional interest from then on.
\item<2-> The period in which this compounding process occurs is called \alertNoH{2}{compounding period}.
\item<3-> Annual compound interest rate of $k\%$ compounded once a year multiplies the current deposit by a factor of 
$
\displaystyle\alertNoH{3}{\left(1+\frac{k}{100}\right).}
$
\item<4-> Therefore $n$ years of annual compound interest rate of $k\%$ compounded once a year multiplies the original deposit by factor:
\[
\only<handout:0|4-5>{\color{white}}\underbrace{\only<handout:0|5>{\color{black}} \underbrace{\color{black}\underbrace{ \alertNoH{7}{\left(1+\frac{k}{100} \right)}}_{\text{after 1 year}} \uncover<5->{\cdot \alertNoH{7}{\left(1+\frac{k}{100} \right)} }}_{\uncover<5->{\text{after 2 years}}}\uncover<6->{\cdot \dots\cdot \alertNoH{7}{\left(1+\frac{k}{100} \right)}} }_{\text{after }\alertNoH{8}{n} \text{ years}}={\alertNoH{7}{\left(1+\frac{k}{100} \right)}}^{{\alertNoH{8}{n}}}
\]
\end{itemize}
\end{frame}

\begin{frame}
\begin{definition}
The amount of money obtained from principal (original deposit) $P$ after $n$ years of annual compound interest rate of $k\%$, compounded once a year, is given by the formula
\[
P\left(1+\frac{k}{100}\right)^n.
\] 
\end{definition}

\end{frame}