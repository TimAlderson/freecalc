\begin{frame}
\vskip -0.15cm
\begin{example}
In a hypothetical experiment, the number of E. Coli bacteria cells is modeled with a logistic curve $\displaystyle E(t)= \frac{2.6\times 10^{11}}{1+ (3.94\times 10^9) e^{-1.387t }} $, where $t$ measures time in hours since the start of the experiment. 
\begin{itemize}
\only<handout:1|1,2>{\item According to the model, approximately how many cells were there at the start of the experiment?}
\only<handout:2|3->{\item According to the model, how many hours are needed for the number of cells to be approximately $10^{11}$?}
\end{itemize}
\only<handout:1|2>{
According to the model, the number of cells in the beginning was
$\begin{array}{@{}r@{}c@{}l@{}l|l}
\displaystyle E(0)&=&\displaystyle \frac{2.6\times 10^{11}}{1+ (3.94\times 10^9) e^{-1.387\cdot 0 }}=  \frac{2.6\times 10^{11}}{1+ (3.94\times 10^9) e^{0}}\\ &=&\displaystyle  \frac{2.6\times 10^{11}}{1+ (3.94\times 10^9)} \approx  \frac{2.6\times 10^{11}}{3.94\times 10^9}\\
&=& \displaystyle \frac{2.6\times 10^2}{3.94}\approx \frac{260}{3.94}\approx  65.99 \approx 66 \text{ cells.} &&\text{calculator}\end{array} $ 
}
\uncover<handout:2|4->{
To find the hours $t$ needed to get $10^{11}$ cells, we solve $E(t)=10^{11} $.
$\begin{array}{@{}r@{}c@{}l@{}l|l}
\displaystyle 10^{11} &=&\displaystyle \frac{2.6\times 10^{11}}{1+ (3.94\times 10^9) e^{-1.387 t }} \\ \displaystyle  1+3.94\times 10^9e^{-1.387t}&=&2.6\\
\displaystyle e^{-1.387 x }&=& \displaystyle \frac{1.6}{3.94} \times 10^{-9} &&\text{take }\ln \\
-1.387t &=&\displaystyle \ln \left(\frac{1.6}{3.94} \times 10^{-9}\right)\\
t &=&\displaystyle \frac{\ln (1.6)-\ln (3.94)-9\ln \left(10\right)}{-1.387}\\
&\approx& 15.6 \text{ hours}&&\text{calculator}\\

\end{array} $ 


}
\end{example}

\end{frame}