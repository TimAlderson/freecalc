\begin{frame}
\begin{example}
$1$ day after the start of hypothetical experiment a population of fruit flies was measured to have $100$ individuals. $2$ days after the start there were $150$ flies. Write down an exponential growth law that fits this data. According to the model, how may fruit were there at the start of the experiment? After 5 days?
\uncover<2->{
$
\left|\begin{array}{r@{}c@{}l}
\displaystyle b e^{1\cdot a} &=& 100\\
\displaystyle be^a&=&100\\
b&=&\displaystyle 100e^{-a}\\
\end{array}\right.
\left|
\begin{array}{r@{}c@{}l}
\displaystyle b e^{2a} &=&150\\
\displaystyle 100e^{-a} e^{2a}&=&150\\
\displaystyle e^{a}&=&\displaystyle \frac{3}{2}\\
a&=&\displaystyle \ln \left(\frac{3}{2}\right)\approx 0.405\\
b&=&\displaystyle 100 e^{-a}=\frac{100}{e^{ a}}=\frac{100}{ \frac{3}{2}} = \frac{200 }{3}\approx 67.
\end{array}
\right.
$
In the start of the hypothetical experiment there must have been $b e^{a\cdot 0} =b\approx 67 $ flies. Under the exponential growth model, after $5$ one would expect $be^{5a}=67(e^{a})^5=67\left(\frac{3}{2}\right)^5=67 \cdot \frac{243}{32}\approx 509$ flies. 
}
\end{example}
\end{frame}