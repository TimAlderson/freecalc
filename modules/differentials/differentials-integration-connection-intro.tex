%begin module differentials-integration-connection-intro
\begin{frame}
\begin{columns}

\column{0.3\textwidth}
\begin{pspicture}(-0.2,-0.2)(1,1)
\fcRiemannSum{0.2}{1.6}{x x x mul mul 0.2 add sqrt}{7}{0}
\fcAxesStandardNoFrame{-0.2}{-0.2}{2}{2}
\end{pspicture}
\column{0.7\textwidth}
\begin{itemize}
\item<1->  $\displaystyle \only<2>{\color{red} } \only<3>{ \color{black} } \int\limits^{b}_{a} \uncover<2>{\color{black}}  \alert<4>{f( x) }\alert<3>{ \diff x} $ is the definite integral of $f$.
\item<1-> $\displaystyle\alert<2>{ \int} f(x) \diff x$ anti-derivative used for definite integral.
\item<2-> $\alert<2>{\int}$ stands for the \alert<2>{limit of a Riemann sum} (sum of \alert<3,4>{approximating rectangles}).
\end{itemize}
\end{columns}
\begin{itemize}

\item<3-> $\alert<3>{\diff x}$ ``encodes''  \alert<3>{the length of the base} of an ``\alert<5>{infinitesimally small}'' approximating rectangle\uncover<4->{, $\alert<4>{f(x)}$ stands for the \alert<4>{height}.}
\item<6-> Formally, the expression $f(x) \diff x$ is a differential form (the same differential forms discussed in the preceding slides).
\item<7-> We did not give a complete formal definition of a differential form, but we showed how to compute with those. 
\item<8-> Differential forms are consistent with integrals: integrals of equal differential forms are equal (follows from Net Change Theorem (subst. rule)).
\end{itemize}
\end{frame}

%end module differentials-integratino-connection-intro