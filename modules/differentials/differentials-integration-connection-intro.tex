%begin module differentials-integration-connection-intro
\begin{frame}
\begin{itemize}
\item<1-> We defined $\int f(x) dx$ as an anti-derivative of $f(x)$ and $\displaystyle \only<2>{\color{red} }\only<3>{\color{black} } \int\limits^{b}_{a} \uncover<2>{\color{black}}  \alert<4>{f(x)}\alert<3>{ dx} $ as the definite integral of $f$.
\item<2-> The $\alert<2>{\int}$ sign stands for the \alert<2>{limit of a Riemann sum} (sum of \alert<3,4>{approximating rectangles}).
\item<3-> $\alert<3>{dx}$ stands for \alert<3>{the base} of an ``\alert<5>{infinitesimally small}'' approximating rectangle\uncover<4->{, $\alert<4>{f(x)}$ stands for the \alert<4>{height}.}
\item<5-> ``\alert<5>{Infinitesimally small}'' is an informal expression. 
\item<6-> The strict mathematical terminology for the expression $f(x) dx$ is differential form.
\item<7-> Differential forms can be defined and used as stand-alone objects, without integrals.
\item<8-> We do not give such a stand-alone definition in Calc I and II. However we allow ourselves to compute with differential forms as if we had a stand-alone definition. 
\end{itemize}
\end{frame}

%end module differentials-integratino-connection-intro