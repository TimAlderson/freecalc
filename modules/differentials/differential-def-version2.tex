% begin module differential-def
\begin{frame}


\frametitle{Differentials}
\begin{itemize}
\item<1-> From previous slides:
\[ \only<1-2, 7->{\Delta y} \only<3-6>{\alert<3,6> {dy}} \only<1-3, 7->{\approx}\only<4-6>{\alert<4> =} \only<1-5,7->{\frac{dy}{dx}}
\only<6>{{\frac{dy}{\cancel{dx}}}}
\only<1-4, 7->{ \Delta x} 
\only<5>{\alert<5>{dx}} 
\only<6>{\alert<5>{\cancel{dx}}} 
\only<6>{=\alert<6>{dy} }
\]
\item<2-> If in the above we \alert<3-6>{formally substitute} \alert<3>{$\Delta y $ by $dy$}, \alert<4>{$=$ by $\approx$} and \alert<5>{$\Delta x$ by $dx$}, we get a \alert<6>{formal identity}.
\item<7-> Define formally the \emph{differential operator $d$} and the \emph{differential form $dx$} by requesting that $d$ and $dx$ satisfy the transformation law 
\[
d(f(x))=f'(x)dx
\] for any differentiable function $f(x)$. In abbreviated notation:
\[ df = f' dx
\]
\item<8-> Differential forms express the idea of approximating differentiable functions by linear.
\item<9-> The strict definition of differential forms  is outside of the scope of Calc I and II.
\end{itemize}
\end{frame}





%\begin{frame}
%\begin{itemize}
%\item
% The rest of this slide is for your information only; you will not be tested on it.
%\item To define differentials strictly:
%\begin{itemize}
%\item Let $\xi$ be a function mapping functions that have first derivative to functions that have first derivative. 
%\item Define  $\xi$ to be a \emph{differential operator} if it satisfies the Leibniz rule: $\xi (f g)= \xi (f) g+ f\xi (g) $ for any two functions $f,g$ with first derivative evaluated at an arbitrary point $x$.
%\item Define $w $ to be a differential operator 
%\end{itemize} 

%\item  $\diff y$ depends on $x$ and $\diff x$.
%\item  It measures the ``rise'' for the linear approximation $L(x)$:
%\end{itemize}
%\begin{align*}
%\uncover<1->{%
%\diff y%
%}%
%& \uncover<1->{ = } %
%\uncover<1->{%
%L(x+\diff x) - L(x)%
%}%
%\\%
%& \uncover<2->{ = } &%
%\uncover<2->{%
%[f(x) + f'(x)(x+\diff x - x)] - [f(x) + f'(x)(x-x)]%
%}\\%
%& \uncover<2->{ = } &%
%\uncover<2->{%
%f'(x)\diff x%
%}\\%
%\end{align*}
%\end{frame}

% end module differential-def
