% begin module differential-def
\begin{frame}
\frametitle{Differentials}
\begin{itemize}
\item<1-> Recall $\Delta y, \Delta x$ stand for change of $x,y$. Recall: \alertNoH{12}{$\displaystyle \Delta y\approx \frac{\diff y}{\diff x} \Delta x$}
\item $\displaystyle \only<handout:0|1-2>{\Delta y}
\only<3->{\alertNoH{3,6} {\alertNoH{7}{\diff}y}} \only<handout:0|1-3>{\approx}
\only<4->{\alertNoH{3} =} \only<1->{\frac{\diff y}{\alertNoH{5}{\diff x}}}
\only<handout:0|1-3>{ \Delta x}
\only<4->{\alertNoH{4,5,8}{\alertNoH{7}{\diff}x}}
\only<6->{=\alertNoH{6}{\alertNoH{7}{\diff}y} }
$
\item<2-> If we substitute \alertNoH{3}{$\Delta y $ by the formal expression $\diff y$} and \alertNoH{4}{$\Delta x$ by the formal expression $\diff x$}, the expression \alertNoH{5}{$\diff x$ appears to ``cancel''} to give a \alertNoH{6}{formal identity}.
\item<7-> Define the \alertNoH{7,11}{\emph{differential $\diff$}} %\uncover<8->
{ and the \alertNoH{8,10}{\emph{differential forms $\diff x$, $\diff(f(x))$}}} %\uncover<9->
{by requesting that \alertNoH{9}{$\diff$ and $\diff x$ satisfy the transformation law}
\[
\alertNoH{9}{\alertNoH{8}{\alertNoH{7}{\diff}(f(x))}=f'(x) \alertNoH{8}{\alertNoH{7}{\diff}x}}
\]
for any differentiable function $f(x)$.} In abbreviated notation:
\[
\alertNoH{9}{ \alertNoH{7}{\diff}f = f' \alertNoH{8}{\alertNoH{7}{\diff}x}}
\]
\uncover<10->{Expressions containing expression of the form $\alertNoH{10}{\alertNoH{11}{\diff}(something)}$ are called \alertNoH{10}{differential forms}.}
\end{itemize}
\end{frame}
\begin{frame}
\begin{itemize}
\item $\alertNoH{4,9,12}{\alertNoH{3}{ \alertNoH{2}{\diff} f(x)}= f'(x) \alertNoH{3}{\alertNoH{2}{\diff}x}}$.
\item<2-> On the previous slide we stated the \alertNoH{2}{differential $\diff$} and the \alertNoH{3}{differential forms} $\alertNoH{3}{\diff x, \diff f(x)}$ are \alertNoH{4,8}{formal expressions related by a transformation law}.
\item<5-> The precise definitions of differential forms and differentials are outside of the scope of Calculus I and II.
\item<6-> Differential forms ``encode'' linear approximations which in turn ``encode'' ``infinitesimal'' lengths of segments.
\item<7-> Courses such as ``Integration and Manifolds'' or ``Differential geometry'' usually give precise definitions and fill in the details.
\item<8-> Nonetheless, \alertNoH{8}{what we studied} is \alertNoH{9}{completely sufficient} for practical purposes and \alertNoH{9}{carrying out computations}.
\item<10-> \alertNoH{10,11}{\textbf{Do not confuse differentials with derivatives.}} \uncover<12->{\alertNoH{12}{The correct equality is this.}}

\[
\only<handout:0|10>{\alertNoH{10,11}{ \diff f(x) = f'(x)}} \only<11->{\alertNoH{10,11}{ \xcancel{ \diff f(x) = f'(x)}}}
\quad \quad \quad\quad \quad \uncover<12>{\alertNoH{12}{\diff f(x)=f'(x)\diff x}}
\]
\end{itemize}
\end{frame}
