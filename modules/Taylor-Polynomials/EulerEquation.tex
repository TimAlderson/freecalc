% begin module Euler Eqn
\begin{frame}
\frametitle{An amazing application}
\[
e^x = 1+x+\frac{x^2}{2!}+\frac{x^3}{3!}+\frac{x^4}{4!}+\frac{x^5}{5!}\cdots 
\]
So, if $ i^2=-1 $, then we have
\begin{align*}
e^{ix} & \uncover<2->{= 1+ix+\frac{(ix)^2}{2!}+\frac{(ix)^3}{3!}+\frac{(ix)^4}{4!}+\frac{(ix)^5}{5!}\cdots}\\
\uncover<3->{&= 1+ix+\frac{i^2x^2}{2!}+\frac{i^3x^3}{3!}+\frac{i^4x^4}{4!}+\frac{i^5x^5}{5!}\cdots\\} 
\uncover<4->{&= 1+ix+\frac{(-1)x^2}{2!}+\frac{-ix^3}{3!}+\frac{1x^4}{4!}+\frac{ix^5}{5!}\cdots\\} 
\uncover<5->{&= {\uncover<6->{\color{red}}{\left(1-\frac{x^2}{2!}+\frac{x^4}{4!}-\frac{x^6}{6!}+\cdots\right)}} +i{\uncover<6->{\color{blue}}{\left(x-\frac{x^3}{3!}+\frac{x^5}{5!}-\frac{x^7}{7!}+\cdots\right)}}\\} 
\uncover<6->{&= {\color{red}{\cos(x)}} + i {\color{blue}{\sin(x)}}
}
\end{align*}

\end{frame}
% end module power-series-as-function-ex1


\begin{frame}\frametitle{Euler's Identity}
\[
e^{ix} = \cos(x)+i\sin{x}
\]
\pause Let $ x=\pi $, then
\[
e^{i\pi}=\cos(\pi)+i\sin(\pi)
\] \pause 
which gives
\[
e^{i\pi}=-1+i(0)
\] \pause 
so
{\huge{
\[
e^{i\pi}+1=0
\]
}}

A poll conducted by \textit{The Mathematical Intelligencer} magazine named Euler's Identity as the "\textit{most beautiful theorem in mathematics}".\\

Another poll conducted by \textit{Physics World} magazine, in 2004, had Euler's Identity tied with Maxwell equations (of electromagnetism) as the "\textit{greatest equation ever}".


\end{frame}

\begin{frame}

{\huge{
\[
e^{i\pi}+1=0
\]
}}

``It is absolutely paradoxical; we cannot understand it, and we don't know what it means, but we have proved it, and therefore we know it must be the truth."
\begin{flushright}
{\color{blue}{{\textsc{Benjamin Peirce, 19th century philosopher/mathematician, Harvard University
}}}}
\end{flushright}

\end{frame}


\begin{frame}
\begin{itemize}
\item paragon of mathematical beauty

\item  suggests a profound relationship between geometry, analysis, and algebra - apparently distinct branches of math, each with its own developmental history and "\textit{purpose}"

\item  each of the five constants it connects were derived independently in their own areas without explicit reference to the others. Each fits an area of mathematics. To discover that these important mathematical quantities are interdependent has a profoundly beautiful and even mystical quality.

\item Is math "invented" or "discovered"? Euler's identity tends to imply "Mathematical Realism": That mathematics and its truths exist independently from the mind and are "out there", waiting to be discovered. 
\end{itemize}

\end{frame}
