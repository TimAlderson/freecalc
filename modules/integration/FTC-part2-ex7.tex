% begin module FTC-part2-ex7
\begin{frame}
\begin{example}
\begin{columns}
\column{0.4\textwidth}
\psset{xunit=0.7cm, yunit=0.7cm}
\begin{pspicture}(-1.000000, -1.3)(6.2,1.3)
\tiny
\psframe*[linecolor=white](-1.000000,-1.1)(6.2,1.2)
\pscustom*[linecolor=cyan]{ %Function formula: \cos{}x
\psplot[linecolor=\fcColorGraph, plotpoints=1000]{0}{1}{x 57.29578 mul cos }\psline(1.000000, 0)(0.000000, 0)}
%Function formula: \cos{}x
\psplot[linecolor=\fcColorGraph, plotpoints=1000]{0}{6}{x 57.29578 mul cos }
\psaxes[arrows=<->, ticks=none, labels=none](0,0)(-0.6,-1.1)(6,1.1)
\fcLabels{6}{1.1}
\fcXTickWithLabel{1}{$b$}
\end{pspicture}
\column{0.6\textwidth}
Find the area under the cosine curve from $0$ to $b$, where $0 \leq b \leq \frac{\pi}{2}$.
\end{columns}
\begin{itemize}
\item<2->  $\cos x$ is continuous on $[0, \frac{\pi}{2}]$ (in fact, it's continuous everywhere).
\item<3->  \alertNoH{3,4,6}{An antiderivative of $\cos x$ is $\fcAnswer{4}{\sin x.}$}
\[
\uncover<5->{%
{\alertNoH{6}{\int}}_{{\!\!\!\alertNoH{ 5}{0}}}^{{\alertNoH{ 5}{b}}} \alertNoH{6}{\cos x \ \diff x} = \left[ \alertNoH{6}{\sin \alertNoH{7,8}{x}} \right]_{\alertNoH{5,8}{0}}^{\alertNoH{5,7}{b}} %
}%
\uncover<7->{%
= \sin (\alertNoH{7}{b}) - \sin(\alertNoH{8}{0})%
}%
\uncover<9->{%
 = \sin b%
}%
\]
\end{itemize}
\end{example}
\end{frame}
% end module FTC-part2-ex7
