% begin module sigma-notation
\begin{frame}
\[
\mathop{\alert<handout:0| 2>{\sum}}_{\alert<handout:0| 3>{i=1}}^{\alert<handout:0| 4>{n}} f(x_i)\Delta x = f(x_1)\Delta x + f(x_2)\Delta x + \cdots + f(x_n)\Delta x
\]
\begin{itemize}
\item  We use sigma notation to write sums more compactly.
\item<2-| alert@2>  $\sum$ is the Greek letter sigma.  It tells us to add.
\item<3-| alert@3>  The subscript tells us to start at 1.
\item<4-| alert@4>  The superscript tells us to finish at $n$.
\end{itemize}
\uncover<5->{%
\begin{example}
\begin{align*}
\sum_{i=1}^4 2i\Delta x & =  2\Delta x + 4 \Delta x + 6 \Delta x + 8\Delta x\\
\uncover<6->{\alert<handout:0| 6-7>{\sum_{i=3}^7 i^2\Delta x}} & \uncover<6->{\alert<handout:0| 6-7>{=}}  \uncover<7->{\alert<handout:0| 7>{9\Delta x + 16 \Delta x + 25 \Delta x + 36\Delta x + 49\Delta x}}
\end{align*}
\end{example}
}%
\end{frame}
% end module sigma-notation
