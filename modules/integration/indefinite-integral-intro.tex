% begin module indefinite-integral-intro
\begin{frame}\frametitle{Indefinite Integrals}
\begin{itemize}
\item  The Evaluation Theorem establishes a connection between antiderivatives and definite integrals.
\item  It says that $\int_a^b f(x)\diff x$ equals $F(b) - F(a)$, where $F$ is an antiderivative of $f$.
\item  We need convenient notation for writing antiderivatives.
\item  This is what the indefinite integral is.
\end{itemize}
\begin{definition}[Indefinite Integral]
The indefinite integral of $f$ is another way of saying the antiderivative of $f$, and is written $\int f(x) \diff x$.  In other words,
\abovedisplayskip=0pt
\belowdisplayskip=0pt
\[
\int f(x) \diff x = F(x) \qquad \text{means}\qquad F'(x) = f(x).
\]
\end{definition}
\end{frame}

\begin{frame}
\begin{example}
\abovedisplayskip=0pt
\belowdisplayskip=0pt
\[
\int x^4 \diff x = \uncover<2->{\frac{x^5}{5}} \uncover<3->{\alertNoH{ 3}{+ C}}
\]
\uncover<4->{because}
\abovedisplayskip=0pt
\belowdisplayskip=0pt
\[
\uncover<4->{\frac{\diff}{\diff x}\left( \frac{x^5}{5} + C\right) = x^4.}
\]
\end{example}
\begin{itemize}
\item<5->  The indefinite integral represents a whole family of functions.
\item<6->  Example: %1b, p. 318:
the general antiderivative of $\frac{1}{x}$ is
\abovedisplayskip=0pt
\belowdisplayskip=0pt
\[
\uncover<6->{%
F(x) = \left\{ \begin{array}{ccc}
\ln |x| + C_1 & \text{ if } & x > 0\\
\ln |x| + C_2 & \text{ if } & x < 0
\end{array}\right.
}%
\]
\item<7->  We adopt the convention that the constant participating in an indefinite integral is only valid on one interval.
\item<8->  $\int \frac{1}{x} \diff x = \ln |x| + C$, and this is valid either on $(-\infty , 0)$ or $(0, \infty)$.
\end{itemize}
\end{frame}
% end module indefinite-integral-intro
