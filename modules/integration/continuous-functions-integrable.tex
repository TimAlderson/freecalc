% begin module continuous-functions-integrable
\begin{frame}
\begin{theorem}
Let $f$ be a continuous function on $[a,b]$. Then $f$ is integrable over $[a,b]$.
\end{theorem}
\begin{itemize}
\item In particular the integral does not depend the choice of sampling points so long as the intervals containing them shrink.
\item The proof of this theorem is not difficult, but is outside of the scope of Calculus I and II.
\item The only ``difficulty'' in the proof stems from the fact that we have not rigorously constructed the real numbers. 
\item We already (silently) assumed a construction of the real numbers when we defined limits. 
\item Such a construction is also (silently) assumed in most regular high school mathematics courses.
%\item  The proof of the Theorem uses the fact that every set bounded above has a least upper bound. The latter fact is either taken as an axiom of the real numbers, or is proven from equivalent set of axioms. This is the main fact a calculus student is missing to understand/prove on one's own the above theorem.
\item A student interested in a proof of the theorem should google ``Darboux integral''.
\end{itemize}

\end{frame}
% end module continuous-functions-integrable
