% begin module FTC-part2-ex6
\begin{frame}
\begin{example}
\begin{columns}
\column{0.1\textwidth}
\psset{xunit=1cm, yunit=1cm}
\begin{pspicture}(-0.5,-0.5)(1.2,1.2)
\psframe*[linecolor=white](-0.5,-0.5)(1.2,1.2)
\tiny
\pscustom*[linecolor=cyan]{ %Function formula: x^{2}
\psplot[linecolor=\fcColorGraph, plotpoints=1000]{0}{1}{x 2 exp }\psline(1.000000, 0)(0.000000, 0)}
\psplot[linecolor=\fcColorGraph, plotpoints=1000]{0}{1}{x 2 exp }
\psaxes[ticks=none, labels=none]{<->}(0,0)(-0.5,-0.5)(1.1,1.1)
\fcLabels{1.1}{1.1}
\end{pspicture}
\column{0.9\textwidth}
Find the area under the parabola $y = x^2$ from $0$ to $1$.
\begin{itemize}
\item<2->  $x^2$ is continuous on $[0, 1]$ (in fact, it's continuous everywhere).
\item<3-| alert@3-4,6> An antiderivative of $x^2$ is $\fcAnswer{4}{\frac{1}{3}x^3.}$
\end{itemize}
\[
\uncover<5->{%
{\alertNoH{6}{\int}}_{\!\!\!\alertNoH{ 5}{0}}^{\alertNoH{ 5}{1}} \alertNoH{6}{x^2 \ \diff x} = \left[ \alertNoH{6}{\frac{1}{3}\alertNoH{7,8}{x}^3}\right]_{\alertNoH{ 5,8}{0}}^{\alertNoH{ 5,7}{1}} %
}%
\uncover<7->{%
 = \frac{1}{3}(\alertNoH{7}{1})^3 - \frac{1}{3}(\alertNoH{8}{0})^3 %
}%
\uncover<9->{%
 = \frac{1}{3}%
}%
\]
\end{columns}
\end{example}
\end{frame}
% end module FTC-part2-ex6
