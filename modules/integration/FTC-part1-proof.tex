% begin module FTC1-proof
\begin{frame}
\begin{theorem}[The Fundamental Theorem of Calculus part 1]
Let \alertNoH{2}{$f$ be a function continuous on $[a, b]$} and let $G(x) = \displaystyle \int_a^x f(t) \diff t$ for all $x\in [a,b]$. Then $G$ is differentiable and $G'(x) = f(x)$.
\end{theorem}
\begin{proof}[Proof]
\begin{columns}
\column{0.27\textwidth}
\psset{xunit=0.9cm, yunit=0.9cm}

\begin{pspicture}(-0.6,-0.6)(3.3,3.3)%
\psframe*[linecolor=white](-0.6,-0.6)(3.3,3.3)%
\tiny%
%\fcLabels{3}{2.5}%
\pscustom*[linecolor=cyan]{%
\psplot[linecolor=\fcColorGraph, plotpoints=1000]{0.5}{2}{1 -0.5 x add 3 exp 0.333333 mul add -0.5 x add 2 exp -0.5 mul add }%
\psline(2,0)(0.5,0)%
}%
\uncover<handout:0|20,21>{%
\pscustom*[linecolor=red]{%
\psplot[linecolor=\fcColorGraph, plotpoints=1000]{0.5}{2}{1 -0.5 x add 3 exp 0.333333 mul add -0.5 x add 2 exp -0.5 mul add }%
\psline(2,0)(0.5,0)%
}%
}%
\uncover<11->{%
\pscustom*[linecolor=blue]{%
\psplot[linecolor=\fcColorGraph, plotpoints=1000]{2}{2.15}{1 -0.5 x add 3 exp 0.333333 mul add -0.5 x add 2 exp -0.5 mul add }%
\psline(2.15,1.136125)(2.15,0)(2,0)%
}%
}%
\uncover<handout:0|10,20,22>{%
\pscustom*[linecolor=red]{%
\psplot[linecolor=\fcColorGraph, plotpoints=1000]{2}{2.15}{1 -0.5 x add 3 exp 0.333333 mul add -0.5 x add 2 exp -0.5 mul add }%
\psline(2.15,1.136125)(2.15,0)(2,0)%
}%
}%
\uncover<handout:0|12>{%
\pscustom*[linecolor=red]{%
\psplot[linecolor=\fcColorGraph, plotpoints=1000]{2}{2.15}{1 -0.5 x add 3 exp 0.333333 mul add -0.5 x add 2 exp -0.5 mul add }%
\psline(2.15,1.136125)(2.15,1)(2,1)%
}%
}%
%Function formula: -1/2 (x-1/2)^{2}+1/3 (x-1/2)^{3}+1
\psplot[linecolor=\fcColorGraph, plotpoints=1000]{0.5}{3}{1 -0.5 x add 3 exp 0.333333 mul add -0.5 x add 2 exp -0.5 mul add }%
\rput[b](2.3, 2.4){$y=f(x)$}%
\uncover<handout:1|11,14>{%
\psline*[linecolor=red](2,0)(2.15,0)(2.15,1)(2,1)(2,0)%
}%
\uncover<handout:0|13->{%
\psline(2,0)(2.15,0)(2.15,1.136125)(2,1.136125)(2,0)%
}%
\fcXTick{2}%
\uncover<handout:0|3>{ %
\psline[linecolor=red, linewidth=2pt](1.5,0)( 2.5,0 )%
}%
\uncover<handout:0|4-7>{ %
\psline[linewidth=2pt](1.5,0)( 2.5,0 )%
}%
\uncover<handout:0|4>{ %
\psline[linecolor=red, linewidth=2pt](2,0)(2,1)%
}%
\uncover<handout:0|6>{%
\psline(2.4,1.481333)(2.4,0)%
}%
\uncover<handout:0|5>{ %
\psline[linecolor=red, linewidth=2pt](2.4,0)(2.4,1.481333)%
}%
\uncover<5-7>{ %
\psline(2,0)(2,1)%
}%
\uncover<handout:0|5>{ %
\psline[linecolor=red](2.4,0)(2.4,1.481333)%
}%
\uncover<handout:0|6>{ %
\psline[linecolor=red, linewidth=2pt](2,1)(2, 1.481333)%
}%
\fcXTickWithLabel{0.5}{$a$}%
\fcXTick{3}%
\rput[tl](3,-0.2){$b$}%
\uncover<7->{%
\fcXTick{2.15}%
\rput[tl](2.15, -0.2){$x+h$}%
}%
\uncover<handout:0|7,13>{%
\psline[linecolor=red, linewidth=2pt](2,0)(2.15,0)%
}%
\rput[tr](2, -0.235){$x$}%
\psaxes[ticks=none, labels=none]{<->}(0,0)(-0.5,-0.5)(3.2,3.083333)%
\end{pspicture}

\vspace{1.68cm}
\column{0.73\textwidth}

\only<handout:1|1-15>{
\uncover<2->{\alertNoH{2}{Let $\varepsilon >0$. There \alertNoH{3}{exists $\delta$} such that $\alertNoH{8}{\alertNoH{6}{|\alertNoH{5}{f(t)}-\alertNoH{4}{f(x)}|} < \varepsilon}$ for all $t$ for which $|x-t|<\delta$}.} \uncover<3->{\alertNoH{7}{Then for all $0<h<\delta$:} }
\[
\begin{array}{r@{}c@{}l@{}c@{}ll|l}
\uncover<8->{%
\alertNoH{8}{\varepsilon} &\alertNoH{8}{ >}& \hphantom{ \int_{ x}^{ x+h}(} \alertNoH{8}{f(t)-f(x) }  &\alertNoH{8}{ >}& \alertNoH{8}{ - \varepsilon }  &\uncover<9->{&\text{integrate}} 
}\\%
\uncover<9->{%
h\varepsilon  & \alertNoH{0}{>}& \alertNoH{12}{ \alertNoH{10,11}{\int_{x }^{x+h}} ( \alertNoH{10}{ f(t)}-\alertNoH{11}{f(x)})\alertNoH{10,11}{ \diff t} }&\alertNoH{0}{>}&-h\varepsilon &\uncover<13->{& \text{divide~by~}\alertNoH{13}{h }} 
} \\%
\uncover<13->{\varepsilon& \alertNoH{0}{>}&\frac{\alertNoH{14}{\int_{x }^{x+h}}(f(t) -\alertNoH{14}{f(x)}) \alertNoH{14}{\diff t}}{ \alertNoH{13,14}{ h} }& \alertNoH{0}{>}&-\varepsilon}\\
\uncover<14->{%
\alertNoH{15}{\varepsilon}&\alertNoH{15}{>}& \frac{\int_{x }^{x+h}f(t)\diff t }{h}- \alertNoH{14}{\frac{h f(x)}{h}} &\alertNoH{15}{ >}&\alertNoH{15}{{ - \varepsilon}}\\
\uncover<15->{\varepsilon&\alertNoH{15}{>} & \only<handout:0|1->{\color{red}}\left| \color{black} \frac{\int_{x }^{x+h}f(t)\diff t }{h} - f(x)\only<handout:0|1->{\color{red}} \right|}
}\\
\end{array}
\]
}

\only<handout:2|16->{
\uncover<17->{In analogous fashion  \alertNoH{17}{we can handle the case $h<0$}, to prove:} \alertNoH{23}{ for any $\varepsilon >0$ there exists $\delta>0$ so that for \\ all $\only<16>{0<h\phantom{|}} \alertNoH{17}{\only<17->{\phantom{0<}|h|}} \alertNoH{17}{<\delta}$ we have $\left| \frac{\int_{x }^{x+h}f(t)\diff t }{h}-f(x)\right|<\varepsilon$.}

\[
\begin{array}{l}
\uncover<18->{\displaystyle G'(x)=\lim\limits_{h\to 0}\frac{ \alertNoH{20}{ G(x+h)} -\alertNoH{21}{G(x)}}{h} }\\
\uncover<19->{\displaystyle \phantom{ G'(x) }= \lim\limits_{h\to 0}  \frac{ \alertNoH{20}{ \alertNoH{22}{\int_{a}^{x+h}} f(t)\diff t} - \alertNoH{21}{ \alertNoH{22}{\int_{a}^{x}} f(t)\diff t}}{h} }\\
\uncover<22->{\displaystyle \phantom{ G'(x) }= \alertNoH{23}{\lim\limits_{h\to 0}\frac{\alertNoH{22}{\int_{x}^{x+h}}f(t)\diff t }{h} \uncover<23->{=f(x)}}}
\end{array}
\]

}
\end{columns}
\end{proof}
\end{frame}
% end module FTC1-proof
