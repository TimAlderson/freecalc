% begin module FTC1-proof
\begin{frame}
\begin{theorem}[The Fundamental Theorem of Calculus part 1]
Let \alert<2>{$f$ be a function continuous on $[a, b]$} and let $G(x) = \displaystyle \int_a^x f(t) \diff t$ for all $x\in [a,b]$. Then $G$ is differentiable and $G'(x) = f(x)$.  
\end{theorem}
\begin{proof}[Proof]
\begin{columns}
\column{0.27\textwidth}
\psset{xunit=0.9cm, yunit=0.9cm}

\begin{pspicture}(-0.6,-0.6)(3.3,3.3)
\psframe*[linecolor=white](-0.6,-0.6)(3.3,3.3)
\tiny 
%\psLabels{3}{2.5}

\pscustom*[linecolor=cyan]{
\psplot[linecolor=\psColorGraph, plotpoints=1000]{0.5}{2}{1 -0.5 x add 3 exp 0.333333 mul add -0.5 x add 2 exp -0.5 mul add }
\psline(2,0)(0.5,0) 
}
\uncover<20,21>{
\pscustom*[linecolor=red]{
\psplot[linecolor=\psColorGraph, plotpoints=1000]{0.5}{2}{1 -0.5 x add 3 exp 0.333333 mul add -0.5 x add 2 exp -0.5 mul add }
\psline(2,0)(0.5,0) 
}
}

\uncover<11->{
\pscustom*[linecolor=blue]{
\psplot[linecolor=\psColorGraph, plotpoints=1000]{2}{2.15}{1 -0.5 x add 3 exp 0.333333 mul add -0.5 x add 2 exp -0.5 mul add }
\psline(2.15,1.136125)(2.15,0)(2,0)
}
}

\uncover<10,20,22>{
\pscustom*[linecolor=red]{
\psplot[linecolor=\psColorGraph, plotpoints=1000]{2}{2.15}{1 -0.5 x add 3 exp 0.333333 mul add -0.5 x add 2 exp -0.5 mul add }
\psline(2.15,1.136125)(2.15,0)(2,0)
}
}

\uncover<12>{
\pscustom*[linecolor=red]{
\psplot[linecolor=\psColorGraph, plotpoints=1000]{2}{2.15}{1 -0.5 x add 3 exp 0.333333 mul add -0.5 x add 2 exp -0.5 mul add }
\psline(2.15,1.136125)(2.15,1)(2,1)
}
}

\psaxes[ticks=none, labels=none]{<->}(0,0)(-0.5,-0.5)(3.2,3.083333)

%Function formula: -1/2 (x-1/2)^{2}+1/3 (x-1/2)^{3}+1 
\psplot[linecolor=\psColorGraph, plotpoints=1000]{0.5}{3}{1 -0.5 x add 3 exp 0.333333 mul add -0.5 x add 2 exp -0.5 mul add }
\rput[b](2.3, 2.4){$y=f(x)$}
\uncover<11,14>{
\psline*[linecolor=red](2,0)(2.15,0)(2.15,1)(2,1)(2,0)
}
\uncover<13->{
\psline(2,0)(2.15,0)(2.15,1.136125)(2,1.136125)(2,0)
}

\psXTick{2} 

\uncover<3>{ %
\psline[linecolor=red, linewidth=2pt](1.5,0)( 2.5,0 )
}
\uncover<4-7>{ %
\psline[linewidth=2pt](1.5,0)( 2.5,0 )
}

\uncover<4>{ %
\psline[linecolor=red, linewidth=2pt](2,0)(2,1)
}

\uncover<6>{ %
\psline(2.4,1.481333)(2.4,0)
}
\uncover<5>{ %
\psline[linecolor=red, linewidth=2pt](2.4,0)(2.4,1.481333)
}

\uncover<5-7>{ %
\psline(2,0)(2,1)
}
\uncover<5>{ %
\psline[linecolor=red](2.4,0)(2.4,1.481333)
}
\uncover<6>{ %
\psline[linecolor=red, linewidth=2pt](2,1)(2, 1.481333)
}
\psXTickWithLabel{0.5}{$a$}

\psXTick{3}
\rput[tl](3,-0.2){$b$}
\uncover<7->{
\psXTick{2.15}
\rput[tl](2.15, -0.2){$x+h$}
}
\uncover<7,13>{
\psline[linecolor=red, linewidth=2pt](2,0)(2.15,0)
}
\rput[tr](2, -0.235){$x$}
\end{pspicture}

\vspace{1.68cm}
\column{0.73\textwidth}

\only<1-15>{
\uncover<2->{\alert<2>{Let $\varepsilon >0$. There \alert<3>{exists $\delta$} such that $\alert<8>{\alert<6>{|\alert<5>{f(t)}-\alert<4>{f(x)}|} < \varepsilon}$ for all $t$ for which $|x-t|<\delta$}.} \uncover<3->{\alert<7>{Then for all $0<h<\delta$:} }
\[
\begin{array}{rlll|l}
\uncover<8->{\alert<8>{\varepsilon} &\alert<8>{ > \hphantom{ \int_{ x}^{ x+h}(} f(t)-f(x) } & \alert<8>{ > - \varepsilon }  &\uncover<9->{&\text{integrate}} }\\
\uncover<9->{h\varepsilon  & >\alert<12>{ \alert<10,11>{\int_{x }^{x+h}} ( \alert<10>{ f(t)}-\alert<11>{f(x)})\alert<10,11>{ dt} }&>-h\varepsilon &\uncover<13->{& \text{divide~by~}\alert<13>{h }} } \\
\uncover<13->{\varepsilon& >\frac{\alert<14>{\int_{x }^{x+h}}(f(t) -\alert<14>{f(x)}) \alert<14>{dt}}{ \alert<13,14>{ h} }& >-\varepsilon}\\
\uncover<14->{\alert<15>{\varepsilon}&\alert<15>{>} \frac{\int_{x }^{x+h}f(t)dt }{h}- \alert<14>{\frac{h f(x)}{h}} &\alert<15>{ > - \varepsilon}}\\
\uncover<15->{\varepsilon&\color{red} >  \left| \color{black} \frac{\int_{x }^{x+h}f(t)dt }{h} - f(x)\color{red} \right|}\\
\end{array}
\]
}

\only<16->{
\uncover<17->{In analogous fashion  \alert<17>{we can handle the case $h<0$}, to prove:} \alert<23>{ for any $\varepsilon >0$ there exists $\delta>0$ so that for \\ all $\only<16>{0<h\phantom{|}} \alert<17>{\only<17->{\phantom{0<}|h|}} \alert<17>{<\delta}$ we have $\left| \frac{\int_{x }^{x+h}f(t)dt }{h}-f(x)\right|<\varepsilon$.}

\[
\begin{array}{l}
\uncover<18->{\displaystyle G'(x)=\lim\limits_{h\to 0}\frac{ \alert<20>{ G(x+h)} -\alert<21>{G(x)}}{h} }\\ 
\uncover<19->{\displaystyle \phantom{ G'(x) }= \lim\limits_{h\to 0}  \frac{ \alert<20>{ \alert<22>{\int_{a}^{x+h}} f(t)dt} - \alert<21>{ \alert<22>{\int_{a}^{x}} f(t)dt}}{h} }\\
\uncover<22->{\displaystyle \phantom{ G'(x) }= \alert<23>{\lim\limits_{h\to 0}\frac{\alert<22>{\int_{x}^{x+h}}f(t)dt }{h} \uncover<23->{=f(x)}}}
\end{array}
\]

}
\end{columns}
\end{proof} 
\end{frame}
% end module FTC1-proof
