% begin module definite-integral-properties
\begin{frame}
\frametitle{Properties of the Definite Integral}
\begin{itemize}
\item  So far when we have calculated $\int_a^b f(x)\diff x$, we have assumed that $a < b$.
\item  The definition as a limit of Riemann sums will still work even if we don't assume this.
\item<2->  If we reverse $a$ and $b$, then $\Delta x$ changes from $\frac{b-a}{n}$ to $\frac{a-b}{n}$.
\end{itemize}
\[
\uncover<3->{%
\alert<handout:0| 3-4>{%
\int_b^a f(x)\diff x = %
}%
}%
\uncover<4->{%
\alert<handout:0| 3-4>{%
- \int_a^b f(x)\diff x %
}%
}%
\]
\begin{itemize}
\item<5-| alert@5-6>  If $a = b$, then $\Delta x = $ \uncover<6->{$0$.}
\end{itemize}
\[
\uncover<7->{%
\alert<handout:0| 7-8>{%
\int_a^a f(x) \diff x = %
}%
}%
\uncover<8->{%
\alert<handout:0| 7-8>{%
0 %
}%
}%
\]
\end{frame}

\begin{frame}
Properties of the Integral
\begin{enumerate}
\item<1-| alert@2>  $\int_a^b c\diff x = c(b-a)$, where $c$ is any constant.
\item<1-| alert@3>  $\int_a^b [f(x)+g(x)] \diff x = \int_a^b f(x)\diff x + \int_a^b g(x) \diff x$.
\item<1-| alert@4>  $\int_a^b cf(x) \diff x = c\int_a^b f(x)\diff x$, where $c$ is any constant.
\item<1-| alert@5>  $\int_a^b [f(x)-g(x)] \diff x = \int_a^b f(x)\diff x - \int_a^b g(x) \diff x$.
\end{enumerate}
\end{frame}
% end module definite-integral-properties
