Estimate the integral using a Riemann sum using the indicated sample points and interval length.
\begin{enumerate}
% Riemann sums
\item\label{problemRiemannSum-sqrt(8x+1)} $\displaystyle \int_0^4 \left(\sqrt{8x+1}\right)\diff x$. Use four intervals of equal width, choose the sample point to be the left endpoint of each interval. 

\answer{ $\Delta x = 1$ and $f(x) = \sqrt{8x+1}$. ${\displaystyle \int_0^4 f(x) \diff x \approx 9 + \sqrt{17}}$.}

\item $\displaystyle \int_0^6 \frac{1}{x^2+1} \diff x$. Use three intervals of equal width, choose the sample point to be the left endpoint. 

\answer{ $\Delta x = 2$ and $f(x) = \frac{1}{x^2+1}$. ${\displaystyle \int_0^6 f(x) \diff x \approx \frac{214}{85}}$.}

\item $\displaystyle\int_{0}^2 \frac{\diff x}{1+x+x^3}$. Use $\Delta x=\frac{1}2 $ and right endpoint sampling points.

\answer{$ \frac{1}{2}\left(\frac{8}{13}+\frac{1}{3}+\frac{8}{47}+\frac{1}{11}\right)=\frac{12197}{20163}\approx 0.604920$}
\item $\displaystyle\int_{-2}^{0} \frac{\diff x}{1+x+x^2}$. Use $\Delta x=\frac23 $ and left endpoint sampling points.

\answer{$\frac23\left(\frac{1}{3}+\frac{9}{13}+\frac{9}{7}\right)=\frac{1262}{819}\approx 1.540904$}

\end{enumerate}


\solution{
\ref{problemRiemannSum-sqrt(8x+1)}. The interval $[0,4]$ is subdivided into $n=4$ intervals, therefore the length of each is $\Delta x=1$. The intervals are therefore 
\[
[0,1], [1,2], [2,3], [3,4]\quad .
\]
The problem asks us to use the left endpoints of each interval as sampling points. Therefore our sampling points are $0,1,2,3$. Therefore the Riemann sum we are looking for is 
\[
\Delta x\left(f(0)+f(1)+f(2)+f(3) \right)=1\cdot \left(\sqrt{8\cdot 0+1}+\sqrt{8\cdot 1+1}+\sqrt{8\cdot 2+1}+\sqrt{8\cdot 3+1}\right)= 9+\sqrt{17}\approx 13.1231
\]
\psset{xunit=1cm, yunit=1cm}
\begin{pspicture}(-0.9, -0.9)(4.4,6.233433) 
\tiny 
\psline*[linecolor=\psColorAreaUnderGraph, linewidth=0.1pt](0.000000, 0.000000)(0.000000, 1.000000)(1.000000, 1.000000)(1.000000, 0.000000)(0.000000, 0.000000)
\psline*[linecolor=\psColorAreaUnderGraph, linewidth=0.1pt](1.000000, 0.000000)(1.000000, 3.000000)(2.000000, 3.000000)(2.000000, 0.000000)(1.000000, 0.000000)
\psline*[linecolor=\psColorAreaUnderGraph, linewidth=0.1pt](2.000000, 0.000000)(2.000000, 4.123106)(3.000000, 4.123106)(3.000000, 0.000000)(2.000000, 0.000000)
\psline*[linecolor=\psColorAreaUnderGraph, linewidth=0.1pt](3.000000, 0.000000)(3.000000, 5.000000)(4.000000, 5.000000)(4.000000, 0.000000)(3.000000, 0.000000)
\psline[linecolor=blue, linewidth=0.1pt](0.000000, 0.000000)(0.000000, 1.000000)(1.000000, 1.000000)(1.000000, 0.000000)(0.000000, 0.000000)
\psline[linecolor=blue, linewidth=0.1pt](1.000000, 0.000000)(1.000000, 3.000000)(2.000000, 3.000000)(2.000000, 0.000000)(1.000000, 0.000000)
\psline[linecolor=blue, linewidth=0.1pt](2.000000, 0.000000)(2.000000, 4.123106)(3.000000, 4.123106)(3.000000, 0.000000)(2.000000, 0.000000)
\psline[linecolor=blue, linewidth=0.1pt](3.000000, 0.000000)(3.000000, 5.000000)(4.000000, 5.000000)(4.000000, 0.000000)(3.000000, 0.000000)
%Function formula: (8 x+1)^{1/2} 
\psplot[linecolor=\psColorGraph, plotpoints=1000]{0}{4}{ 1 x 8 mul add 0.5 exp }
\psaxes(0,0)(-0.65,-0.65)(4.15,5.883433)
\end{pspicture} 
}