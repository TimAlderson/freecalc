\begin{frame}
\begin{proposition}[Vieta's formulas]
Let $ax^2+bx+c$ be a quadratic functions with zeros $x_1$ and $x_2$. Then:
\[
\begin{array}{rcl}
\displaystyle \alertNoH{2,3,4,5}{a}( \alertNoH{2,3}{x} \alertNoH{4,5}{- x_1})(\alertNoH{2,4}{ x} \alertNoH{3,5}{ -x_2})& \alertNoH{0}{=} &\displaystyle ax^2+ b x +c\\
\displaystyle \uncover<handout:0|2->{ \alertNoH{2}{ a x^2} \alertNoH{3}{\alertNoH{7}{-}\alertNoH{6}{a}\alertNoH{6}{x} \alertNoH{8}{x_2} } \alertNoH{4}{\alertNoH{7}{-}\alertNoH{6}{ a} \alertNoH{8}{x_1} \alertNoH{6}{x}} +\alertNoH{5}{ a(\alertNoH{9}{-}x_1)(\alertNoH{9}{-}x_2)}&=& \displaystyle ax^2+bx+c}\\
\displaystyle \uncover<6->{ ax^2 \alertNoH{12}{\alertNoH{7}{-} \alertNoH{6}{a} (\alertNoH{8}{x_2+ x_1})} \alertNoH{6}{x} \alertNoH{9}{+} \alertNoH{10}{ax_1x_2}&=&\displaystyle ax^2+\alertNoH{12}{b} x+\alertNoH{10}{c}}\\

\displaystyle \uncover<handout:0|10->{ \alertNoH{10}{\alertNoH{11}{a} x_1x_2}&=&\displaystyle \alertNoH{10}{c} }\\
\displaystyle \uncover<11->{ \alertNoH{14}{x_1x_2}&\alertNoH{14}{{=}} &\displaystyle \alertNoH{14}{ \frac{c}{\alertNoH{11}{a}} }}\\
\displaystyle \uncover<handout:0|12->{ \alertNoH{12}{\alertNoH{13}{-a} (x_1+x_2)}&=&\displaystyle \alertNoH{12}{b} }\\
\displaystyle \uncover<13->{\alertNoH{14}{x_1+x_2}&\alertNoH{14}{{=}}&\displaystyle \alertNoH{14}{\alertNoH{13}{-}\frac{b}{\alertNoH{13}{a}} }}
\end{array}
\]
\end{proposition}
\uncover<14->{The last two formulas are called Vieta's formulas (after Fran\c{c}ois Vi\`ete (1540-1603), Latinized name: Franciscus Vieta).}

\end{frame}