\begin{frame}
\begin{theorem}
\begin{columns}
\column{0.3\textwidth}
\psset{xunit=0.6cm, yunit=0.6cm}
\begin{pspicture}(-1,-1.5)(5,3)
\tiny
\fcAxesStandard{-0.5}{-1.5}{5}{3}
\psplot[linecolor=\fcColorGraph]{0.2}{3.8}{x 1 sub x 3 sub mul}
\fcFullDot{1}{0}
\rput[lb](1,0.1){$x_1$}
\fcFullDot{3}{0}
\rput[br](3,0.1){$x_2$}
\end{pspicture}
\column{0.7\textwidth}
The solutions of the quadratic equation 
\[
ax^2+bx+c=0
\]
are the numbers 
\[
x_1=\frac{-b+ \sqrt{b^2-4ac}}{2a} \qquad x_2=\frac{-b- \sqrt{b^2-4ac}}{2a}.
\]

\end{columns}
\end{theorem}
\begin{itemize}
\item Abbreviated as 
\[
x=\frac{-b\pm \sqrt{b^2-4ac}}{2a}=\frac{-b\pm \sqrt{D}}{2a}, \qquad \text{ where }D=b^2-4ac.
\]
\item If $D<0$ then $\sqrt{D}$ is not a real $\Rightarrow$ quadratic has no real solutions.
\item If $D=0$ then $x_1=x_2$, the equation has only one zero (with multiplicity two). The zero is located at the vertex of the parabola.
\end{itemize}
\end{frame}