\begin{frame}
\begin{proposition}
Let $ax^2+bx+c$, $a\neq 0$ be a quadratic with discriminant $D=b^2-4ac$ and roots $x_1$ and $x_2$. \alertNoH{10}{Then $D=\alertNoH{2}{a^2\left( \alertNoH{11}{x_1- x_2}\right)^2}$}.
\end{proposition}
\begin{proof}
\[
\begin{array}{rcl}
\displaystyle \uncover<2->{ \alertNoH{2,10}{ a^2(\alertNoH{3}{x_1} -\alertNoH{3}{ x_2})^2} &=&\displaystyle a^2\left( \alertNoH{3}{\frac{\fcCancel{4}{ -b}+ \alertNoH{5}{\sqrt{D}}}{2a}}\alertNoH{4,5}{-} \alertNoH{3}{\frac{\fcCancel{4}{ -b}\alertNoH{5}{-\sqrt{D}}}{2a}} \right) }\\
\uncover<5->{&=&\displaystyle a^2\left(\frac{\alertNoH{5}{\fcCancel{6}{2} \alertNoH{7}{ \sqrt{D}}} }{\fcCancel{6}{2} \alertNoH{8}{a} } \right)^{\alertNoH{7, 8}{2}}}\\
\uncover<7->{&=&\displaystyle \fcCancel{9}{ a^2} \frac{\alertNoH{7}{ D}}{\alertNoH{8}{\fcCancel{9}{ a^2}}} }  \\
\uncover<9->{&\alertNoH{10}{{=}}&\displaystyle \alertNoH{10}{D}}\uncover<10->{\alertNoH{10}{, \qquad\text{as desired.}}}
\end{array}
\]
\vskip -0.2cm
\end{proof}
\begin{itemize}
\item<11-> Discriminant is zero $\Leftrightarrow$ the quadratic has non-distinct roots\uncover<12->{, hence the discriminant discriminates between the two roots.}
\end{itemize}

\end{frame}