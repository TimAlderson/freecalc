\begin{frame}
\uncover<16->{$ ax^2+bx+c=\alertNoH{18}{a}\left( x- \alertNoH{19,35}{ x_1}\right) \left(x- \alertNoH{20,34}{x_2}\right)$\uncoverAlert{22}{, where $x_1, x_2 = \frac{-b \pm \sqrt{b^2-4ac}}{2a}$.}}
\begin{example}
Factor the polynomial. If possible, guess the factorization.

\hfil\hfil $\begin{array}{rcll|l}
\alertNoH{14,15}{\alertNoH{23}{3}x^2+\alertNoH{24}{8} x\alertNoH{25}{- 11}}\uncover<2->{ &=& (\uncover<3->{\alertNoH{16}{3} x+}  \fcAnswerUncover{2}{15}{11})(\uncover<3->{x} \uncover<3-14>{+} \fcAnswerUncover{2}{15}{-1})}\\
\uncover<16->{&=& \alertNoH{16,18}{3} \left(x \alertNoH{17}{-} \left(\alertNoH{19,35}{ \alertNoH{17}{-} \frac{11}{ \alertNoH{16}{3}}} \right) \right)(x- \alertNoH{20,34}{1}) }\\ 
\end{array}
$

\begin{itemize}
\only<handout:1|1-20>{%
\item<3-> If there is a factorization using integers, it should be of the form 

\hfil \hfil$
\begin{array}{rcl}
3 x^2+\alertNoH{10,12}{ 8}x \alertNoH{11,12}{-11} &\alertNoH{0}{=} &(\alertNoH{4,5}{3x} +\alertNoH{6,7,14,15}{p})(\alertNoH{4,6}{x}+ \alertNoH{5,7,14,15}{q})\\
\uncover<4->{& \alertNoH{0}{=} &\alertNoH{4}{3x^2}+ \alertNoH{5}{ \alertNoH{9}{3}\alertNoH{8}{ x} \alertNoH{9}{q}}+ \alertNoH{6}{ \alertNoH{9}{p}\alertNoH{8}{x}}+\alertNoH{7}{pq}}\\
\uncover<8->{&\alertNoH{0}{=} & 3x^2+\alertNoH{8}{x}( \alertNoH{9,10,12}{ 3q+p})+ \alertNoH{11,12}{ pq} }\\
\multicolumn{3}{c}{\uncoverAlert{12}{\text{(Vieta's formulas) }} \uncover<10->{\text{This means that  :}}}\\
\uncover<10->{\alertNoH{10,12,14}{8}&\alertNoH{12,14}{=}&\alertNoH{10,12,14}{3q +p}}\\
\uncover<10->{\alertNoH{11,12,13,14}{-11}&\alertNoH{12,13,14}{=}&\alertNoH{11,12,13,14}{pq}}\\
\multicolumn{3}{c}{\uncover<13->{\alertNoH{13}{p,q \text{ must be divisors of 11: }\pm 1,\pm 11 }}}\\
\uncover<14->{\alertNoH{14,15}{p}&\alertNoH{14,15}{=}& \fcAnswer{15}{11} } \\
\uncover<14->{\alertNoH{14,15}{q}&\alertNoH{14,15}{=}& \fcAnswer{15}{-1}} \\
\end{array}
$
}
\only<handout:2|21->{
\item<21-> What if we can't guess the factorization?
\item<22-> \alertNoH{22}{Use the formulas for $x_1, x_2$}.

\hfil \hfil$\renewcommand{\arraystretch}{2}
\begin{array}{rcl}
\alertNoH{22,34,35}{x_1, x_2} &=& \displaystyle \frac{-\alertNoH{24}{ b}\pm \sqrt{{ \alertNoH{24}{ b}}^2-4\alertNoH{23}{a}\alertNoH{25}{ c} }}{2\alertNoH{23}{ a}}
\uncover<23->{=\displaystyle \frac{-\alertNoH{24}{8}\pm \sqrt{{ \alertNoH{24}{8}}^2 \alertNoH{27}{-}  \alertNoH{26}{4}\cdot \alertNoH{23,26}{3} \cdot (\alertNoH{25}{\alertNoH{27}{-} \alertNoH{26}{11}})}}{\alertNoH{28}{ 2\cdot \alertNoH{23}{3}}}}\\
\uncover<26->{&=&\displaystyle \frac{-8 \pm \sqrt{\alertNoH{29}{ 64 \alertNoH{27}{+} \alertNoH{26}{132}}}}{\alertNoH{28}{6}} \uncover<29->{=\frac{-8 \pm  \alertNoH{30}{\sqrt{\alertNoH{29}{196 }} } }{6}}}\\
\uncover<30->{&=&\displaystyle \frac{-8 \pm\alertNoH{30}{ 14} }{6}\uncover<31->{= \left\{
\begin{array}{l}
\displaystyle \alertNoH{32}{\frac{-8+14 }{6}} \uncoverAlert{32}{ = \frac{6}{6}= \alertNoH{34}{1}} \\
\displaystyle \alertNoH{33}{\frac{-8-14 }{6}} \uncoverAlert{33}{ =-\frac{22}{6} =\alertNoH{35}{-\frac{11}{3}}}
\end{array}
\right. 
} } 
\end{array}
$
}
\end{itemize}

\end{example}

\vskip 10cm
\end{frame}