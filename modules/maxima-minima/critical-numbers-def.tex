% begin module critical-numbers-def
\begin{frame}
Fermat's Theorem and Example 6 suggest that we should look at three types of points to find local maxima and minima:
\begin{enumerate}
\item  Points $c$ for which $f'(c) = 0$.
\item  Points $c$ for which $f'(c)$ doesn't exist.
\item  Points $c$ at end of intervals where $f$ is defined. Here, we need also that $f$ be defined at $c$.
\end{enumerate}
\uncover<2->{%
\begin{definition}[Critical Number]
A critical number of a function $f$ is a number $c$ in the domain of $f$ such that either $f'(c) = 0$ or $f'(c)$ doesn't exist.
\end{definition}
}%
\uncover<3->{%
Fermat's Theorem says that if $f$ has a local maximum or minimum at $c$, and $c$ is not an endpoint, then $c$ is a critical number for $f$.
}%
\end{frame}
% end module critical-numbers-def
