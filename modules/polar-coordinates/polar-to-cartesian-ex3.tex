% begin module polar-to-cartesian-ex3
\begin{frame}
\begin{example}
\begin{columns}
\column{0.25\textwidth}
\psset{xunit=0.65cm, yunit=0.65cm}
\begin{pspicture}(-1,-1)(2,2)
\tiny
\fcAxesStandard{-1.5}{-1.5}{1.5}{1.5}
\fcLabels{1.5}{1.5}
\uncover<9->{%
\fcFullDot{1}{-1}%
}%
\uncover<11->{% 
\fcAngleDegrees[arrows=->, linecolor=red]{0}{315}{0.2} { $\frac{7\pi}{4}$}%
\psline(0,0)(1,-1)
}%
\uncover<13->{%
\fcAngleDegrees{0}{-45}{0.4}{$~~-\frac{\pi}{4}$}
}%
\end{pspicture}
\column{0.75\textwidth}
Represent the point with Cartesian coordinates $(1,-1)$ in terms of polar coordinates.
\end{columns}
\begin{columns}
\column{.6\textwidth}
\begin{itemize}
\item<3-| alert@3-4>  Suppose $r$ is positive.
\item<7->  $\tan \theta = -1$ for $\theta = \alert<handout:0| 10>{\frac{3\pi}{4}}, \alert<handout:0| 10-11>{\frac{7\pi}{4}}$, and many other angles.
\item<8-| alert@8-9>  $(1,-1)$ is in the \uncover<9->{fourth} quadrant.
\item<10->  Of the two values above, only \alert<handout:0| 10-11>{$\theta = \uncover<11->{\frac{7\pi}{4}}$} gives a point in the fourth quadrant.
\item<12->  Therefore one possible representation of $(1,-1)$ in polar coordinates is $(\sqrt{2}, 7\pi/4)$.
\item<13->  $(\sqrt{2}, -\pi /4)$ is another.
\end{itemize}
\column{.4\textwidth}
\begin{eqnarray*}
\uncover<2->{%
r%
}%
& \uncover<2->{ = } &%
\uncover<2->{%
\uncover<-3>{\alert<handout:0| 3>{\pm}} \sqrt{x^2+y^2}%
}\\%
& \uncover<5->{ = } &%
\uncover<5->{%
\sqrt{1^2 + (-1)^2}%
}\\% = \sqrt{2}%
& \uncover<5->{ = } &%
\uncover<5->{%
\sqrt{2}%
}\\% = \sqrt{2}%
&&\\
\uncover<2->{%
\tan \theta%
}%
& \uncover<2->{ = } &%
\uncover<2->{%
\frac{y}{x}%
}\\%
& \uncover<6->{ = } &%
\uncover<6->{%
-1%
}\\%
\end{eqnarray*}
\end{columns}
\end{example}
\end{frame}
% end module polar-to-cartesian-ex3
