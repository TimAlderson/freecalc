% begin module partial-fractions-cases
\begin{frame}
The second step is to factor the denominator $Q(x)$ as far as possible.  In fact, any polynomial $Q$ can be factored as a product of linear factors (of the form $ax + b$) and irreducible quadratic factors (of the form $ax^2 + bx + c$, where $b^2 - 4ac < 0$).

\uncover<2->{%
The third step is to express the rational function $R(x)/Q(x)$ as a sum of partial fractions of the form
\[
\frac{A}{(ax+b)^i} \qquad \textrm{or}\qquad \frac{Ax+B}{(ax^2+bx+c)^j}
\]
This is always possible to do.
}%

\uncover<3->{%
There are four cases that occur:
\begin{enumerate}
\item  $Q(x)$ is a product of distinct linear factors.
\item  $Q(x)$ is a product of linear factors, some of which are repeated.
\item  $Q(x)$ contains irreducible quadratic factors, none of which is repeated.
\item  $Q(x)$ contains irreducible quadratic factors, some of which are repeated.
\end{enumerate}
}%
\end{frame}
% end module partial-fractions-cases
