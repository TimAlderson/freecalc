%begin module partial-fractions-building-block-3b-example-n-equals-2
\begin{frame}
\frametitle{Building block IIIb: example illustrating main idea}
\begin{example}
Integrate $\int \frac{\diff x}{(x^2+1)^2}$. We start with an already known integral:
\[
\begin{array}{rcl}
\uncover<2->{\alertNoH{12,13}{\alertNoH{15}{\Arctan x}+C}}
\only<handout:1|1-12>{
\uncover<2->{
&=&\displaystyle \int \alertNoH{3}{\frac{1}{x^2+1}}\diff \alertNoH{4}{ x}}\\
\uncover<3->{&=&\displaystyle \alertNoH{3}{\frac{1}{x^2+1}} \alertNoH{4}{ x}-\int \alertNoH{4}{x} \alertNoH{5,6}{ \diff \left(\alertNoH{3}{\frac{1}{x^2+1}}\right)} }\\
\uncover<5->{&=&\displaystyle \frac{x}{x^2+1} \alertNoH{7}{-} \int \alertNoH{7}{ x} \left( \fcAnswer{6}{ \alertNoH{7}{-} \frac{\alertNoH{7}{ 2x} }{ (x^2+1)^2}} \right) \alertNoH{5,6}{ \diff x} }\\
\uncover<7->{&=&\displaystyle \frac{x}{x^2+1}\alertNoH{7}{ +2} \int \frac{ \uncover<8->{\alertNoH{8}{-\alertNoH{10}{1}}+}\alertNoH{9}{ \alertNoH{7}{ x^2} \uncover<8->{\alertNoH{8}{+1}}}}{\alertNoH{9,10}{(x^2+1)^2}}\diff x }\\
\uncover<9->{&=&\displaystyle  \frac{x}{x^2+1}+2\alertNoH{11}{\int \alertNoH{9}{\frac{1}{x^2+1}} \diff x}-2\int \alertNoH{10}{\frac{1}{(x^2+1)^2}}\diff x}\\
} %only<1-12>
\uncover<11->{ &\alertNoH{12,13}{=}&\displaystyle \alertNoH{12,13}{ \frac{x}{x^2+1}+ \alertNoH{15}{2\alertNoH{11}{\Arctan x} }\alertNoH{14}{ -2 \int \frac{\diff x}{(x^2+1)^2}}} {~~~~~~~~~~~~~~~}}
\end{array}
\]
\only<handout:2|13->{
\uncover<14->{Rearrange terms \uncover<16->{and divide by $2$ to get the desired integral:}
\[
\alertNoH{14}{\uncover<handout:0|14,15>{2} \int \frac{\diff x}{(1+x^2)^2}}=\uncover<16->{\frac{1}{2}} \left(\frac{x}{x^2+1}+ \alertNoH{15}{\Arctan x}  \right)+\uncover<handout:0|14,15>{L}\uncover<16->{K}\quad .
\]
}%uncover14
}%uncover13
\end{example}
\vspace{8cm}
\end{frame}
%end module partial-fractions-building-block-3b-example-n-equals-2