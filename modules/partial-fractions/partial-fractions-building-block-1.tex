%begin module partial-fractions-building-block-1
\begin{frame}
\frametitle{Building block I}
The first building block integral is: $\displaystyle \int \frac{1}{x^n }\diff x=\int x^{-n}\diff x$, $n\neq 1$. 
\begin{example} Integrate the first building block integral 
\[
\int \frac{1}{x^n }\diff x\quad , n\neq 1.
\]

\[
\int \frac{1}{x^n}\diff x=\int x^{-n}\diff x = \frac{x^{-n+1}}{-n+1} +C
\]
\end{example}
\end{frame}
\begin{frame}
\frametitle{Linear substitutions leading to building block I}
First building block: $\displaystyle \int \frac{1}{x^n }\diff x=\int x^{-n}\diff x= \frac{x^{-n+1}}{-n+1} +C$, $n\neq 1$. 
\begin{example} Integrate 
\[
\begin{array}{rcll|l}
\displaystyle \int \frac{1}{(3x+5)^3 }\diff x&=&\displaystyle \int \frac{1}{(3x+5)^3 }\frac{\diff (3 x)}{3} \\
&=&\displaystyle \int \frac{1}{(3x+5)^3 }\frac{\diff (3 x+5)}{3} &&\text{Set }u=3x+5\\
&=&\displaystyle \int\frac{1}{u^3}\frac{\diff u}{3}\\
&=&\displaystyle \frac{1}{3}\int u^{-3} \diff u =\frac{1}{3} \frac{u^{-2}}{(-2)}+C\\
&=&\displaystyle -\frac{1}{6(3x+5)^2}+C\quad .
\end{array}
\]

\end{example}
\end{frame}
\begin{frame}
\frametitle{Lin. subst. leading to building block I: general case}
First building block: $\displaystyle \int \frac{1}{x^n }\diff x=\int x^{-n}\diff x= \frac{x^{-n+1}}{-n+1} +C$, $n\neq 1$. 
\begin{example} Let $n\neq 1$. Integrate 
\[
\begin{array}{rcll|l}
\displaystyle \int \frac{1}{(ax+b)^n }\diff x&=&\displaystyle \int \frac{1}{(ax+b)^n }\frac{\diff (a x)}{a} \\
&=&\displaystyle \int \frac{1}{(ax+b)^n }\frac{\diff (a x+b)}{a} &&\text{Set }u=ax+b\\
&=&\displaystyle \int\frac{1}{u^3}\frac{\diff u}{a}\\
&=&\displaystyle \frac{1}{a}\int u^{-n} \diff u =-\frac{1}{a} \frac{u^{-n+1}}{(n-1)}+C\\
&=&\displaystyle -\frac{1}{ a(n-1)(ax+b)^{n-1}}+C\quad .
\end{array}
\]

\end{example}
\end{frame}

%end module partial-fractions-building-block-1