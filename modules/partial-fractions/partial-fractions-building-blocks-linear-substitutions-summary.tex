\begin{frame}
\frametitle{Linear substitutions leading to building blocks}
\begin{itemize}
\item In the previous slides we showed how to solve building block integrals I, II and III:
$\displaystyle \int \frac{1}{x^n }\diff x$, $\displaystyle \int \frac{x}{(1+x^2)^n }\diff x$, $\displaystyle \int \frac{1}{(1+x^2)^n }\diff x$
\item<2-> Every integral of the form
$\displaystyle \int \frac{1}{(ax+b)^n }\diff x$
can be transformed using linear substitution to building block I. We did that in full detail.
\item<3-> Every integral of the form  
$\displaystyle \int \frac{Ax+B}{(ax^2+bx+c)^{n} }\diff x$
for which $b^2-4ac<0$ can be transformed using linear substitutions to a sum of building blocks II and III. 
\begin{itemize}
\item<4-> For $n=1$ (blocks IIa and IIIa) we showed a complete example how to do that. 
\item<5-> For $n>1$ the integrals are transformed to blocks IIb and IIIb in a completely analogous fashion using the same techniques. An extended example can be found in Calculus for beginners, Chapter Techniques of Integration.
\end{itemize}
\end{itemize}

\end{frame}