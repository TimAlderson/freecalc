% begin module from-building-blocks-to-complete-algorithm-intro
\begin{frame}
\frametitle{From building blocks to all rational functions: example}
\begin{itemize}
\item We know how to solve $\displaystyle \int \frac{2}{x-1}\diff x$ and $\displaystyle \int \frac{1}{x+2}\diff x$. 
\item Consider the difference
\[
\alert<5>{\frac{2}{x-1} - \frac{1}{x+2} } = %
\uncover<2->{%
\frac{2(x+2) - (x-1)}{(x-1)(x+2)} = %
}%
\uncover<3->{%
\alert<5>{ \frac{x + 5}{x^2+x-2} }\quad .
}%
\]
\item<4-> 

We can now solve the following integral:
\[
\int \alert<5>{ \frac{x+5}{x^2+x-2}}\diff x = %
\uncover<5->{%
\int \left(\alert<5>{\frac{2}{x-1} - \frac{1}{x+2}} \right) \diff x = %
}%
\uncover<6->{%
2\ln | x - 1| - \ln | x + 2| + C
}%
\]
\item<7-> From  (linear substitutions of) basic building blocks we constructed a larger example, which we can therefore solve. 
\item<8-> We will now learn how to do the reverse procedure: given a rational function, split it into ``partial fractions'' which are transformed by linear substitutions to basic building block integrals.
\end{itemize}
\end{frame}
% end module from-building-blocks-to-complete-algorithm-intro
