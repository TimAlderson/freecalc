% begin module partial-fractions-intro
\begin{frame}
\frametitle{(8.4) Integration of Rational Functions by Partial Fractions}
In this section, we'll show how to integrate any rational function by expressing it as a sum of simpler fractions, called partial fractions.  To illustrate the method, let us put $2/(x-1)$ and $1/(x+2)$ over a common denominator in the following:
\[
\frac{2}{x-1} - \frac{1}{x+2} = %
\uncover<2->{%
\frac{2(x+2) - (x-1)}{(x-1)(x+2)} = %
}%
\uncover<3->{%
\frac{x + 5}{x^2+x-2}%
}%
\]

\uncover<4->{%
This tells us how to integrate the following:
\[
\int \frac{x+5}{x^2+x-2}\diff x = %
\uncover<5->{%
\int \left( \frac{2}{x-1} - \frac{1}{x+2}\right) \diff x = %
}%
\uncover<6->{%
2\ln | x - 1| - \ln | x + 2| + C
}%
\]
}%

\end{frame}
% end module partial-fractions-intro
