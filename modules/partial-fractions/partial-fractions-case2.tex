% begin module partial-fractions-case2
\begin{frame}
\frametitle{$Q(x)$ has linear factors with higher multiplicity}
\begin{itemize}
\item Suppose $Q(x)$ is a product of linear factors, some of which appear with power greater than $1$.
\item<2-> For example suppose the first linear factor has power $r$, that is, $(a_1x+b_1)^r$ occurs in the factorization of $Q(x)$.
\item<3-> Then instead of a single term $\frac{A}{a_1x+b_1}$ we use
\[
\frac{A_1}{a_1x+b_1}%
 + \frac{A_2}{(a_1x+b_1)^2}%
 + \cdots %
 + \frac{A_r}{(a_1x+b_1)^r}%
\]
\item<4-> In a similar fashion we add more partial fractions to account for all other terms of the form $(a_sx+b_s)^{t}$.
\end{itemize}
\end{frame}
% end module partial-fractions-case2
