\begin{frame}
\frametitle{Linear substitutions leading to blocks IIa and IIIa}
Building block IIa: \alert<4>{$ \int \frac{x}{1+x^2}\diff x = \frac{1}{2}\ln(1+x^2)+C$}.

Building block IIIa: \alert<4>{$\int \frac{1}{1+x^2 }\diff x=\Arctan x+C$}.

\begin{itemize}

\item<1-> Let $ax^2+bx+c$ have no real roots.
\item<2-> We can find $p,q$ so that the linear substitution $u=px+q$ transforms the quadratic to:
\[
ax^2+bx+c= r(u^2+1)
\] 
(where $r$ is some number to be determined).
\item<3-> To find $p,q$, we \alert<3>{complete the square}. 
\item<4-> In this way, integrals of the form \alert<4>{$\displaystyle \int \frac{Ax+B}{ax^2+bx+c} \diff x$} are transformed to \alert<4>{combinations of building blocks IIa and IIIa}.

\item<5-> We show examples; the general case is analogous and we leave it to the student.
\end{itemize}
\vspace{5cm}
\end{frame}