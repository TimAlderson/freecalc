% begin module related-rates-intro
\begin{frame}
\frametitle{Related Rates}
\begin{itemize}
\item  Suppose we are pumping a balloon with air.
\item  The balloon's volume is increasing.
\item  The balloon's radius is increasing.
\item  The rates of increase of these quantities are related to one another.
\item It is easier to measure the rate of increase of volume.
\item<2->  In a related rates problem, we compute the rate of change of one quantity in terms of the rate of change of another (which may be more easily measured).
\end{itemize}
\end{frame}

\begin{frame}
\frametitle{Related Rates: Guideline}
\begin{itemize}
\item \textbf{Draw a picture, introduce notation:}  If possible, draw a schematic picture with all the relevant information, introduce variables/labels. 
\item \textbf{Identify:}  Identify quantities whose rates of change are either given or are required.
\item \textbf{Find an equation:} Find an equation that relates only the quantities with rates that are given or required. 
\item \textbf{Differentiate the equation with respect to $ t $ (time):}  This will often involve implicit differentiation.
\item \textbf{Evaluate the appropriate equation at the desired values:}   The known/given values   should allow you solve for the required rate.
\end{itemize}
\end{frame}

