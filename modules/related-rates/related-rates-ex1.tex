% begin module related-rates-ex1
\begin{frame}
\begin{example}[Example 1, p. 256]
Air is being pumped into a balloon such that \alert<handout:0| 5>{its volume changes at a rate of 100 cm$^3$/s}.  \alert<handout:0| 7>{How fast is the radius of the balloon increasing when the diameter is 50 cm?}
\begin{columns}[c]
\column{.5\textwidth}
\begin{itemize}
\item<2->  Let $V$ denote the balloon's volume.
\item<2->  Let $r$ denote its radius.
\item<3-| alert@4-5,16>  Given: $\uncover<5->{\frac{\diff V}{\diff t} = 100}$ \uncover<5->{cm$^3$/s.}
\item<3-| alert@6-7,15> Unknown: \uncover<7->{$\frac{\diff r}{\diff t}$ \alert<handout:0| 16>{when $r = 25$ cm}.}
\end{itemize}
\begin{enumerate}
\item<8-| alert@9-10>  Find an equation relating the two quantities.
\item<8->  \alert<handout:0| 12>{Use the Chain Rule} to \alert<handout:0| 11>{differentiate both sides}.
\end{enumerate}
%\begin{itemize}
%\item<18->  Therefore the radius of the balloon is increasing at a rate of $1/(25\pi )$ cm/s.
%\end{itemize}
\column{.5\textwidth}
\abovedisplayskip=0pt
\belowdisplayskip=0pt
\abovedisplayshortskip=0pt
\belowdisplayshortskip=0pt
\begin{align*}
\uncover<10->{%
\alert<handout:0| 10>{V}%
}%
& \uncover<10->{\alert<handout:0| 10>{ = }} %
\uncover<10->{%
\alert<handout:0| 10>{\frac{4}{3}\pi r^3}%
}\\%
\uncover<11->{%
\frac{\diff V}{\diff t}%
}%
& \uncover<11->{ = } %
\uncover<11->{%
\alert<handout:0| 12>{\frac{\diff}{\diff t}\left( \frac{4}{3}\pi r^3\right)}%
}\\%
\uncover<12->{%
\frac{\diff V}{\diff t}%
}%
& \uncover<12->{ = } %
\uncover<12->{%
\alert<handout:0| 12>{\alert<handout:0| 13-14>{\frac{\diff}{\diff r}\left( \frac{4}{3}\pi r^3\right)}\frac{\diff r}{\diff t}}%
}\\%
\uncover<13->{%
\frac{\diff V}{\diff t}%
}%
& \uncover<13->{ = } %
\uncover<13->{%
\alert<handout:0| 14>{\uncover<14->{ 4\pi r^2}}\frac{\diff r}{\diff t}%
}\\%
\uncover<15->{%
\frac{\diff r}{\diff t}%
}%
& \uncover<15->{ = } %
\uncover<15->{%
\frac{1}{4\pi \alert<handout:0| 16>{r}^2}\alert<handout:0| 16>{\frac{\diff V}{\diff t}}%
}\\%
\uncover<16->{%
\frac{\diff r}{\diff t}%
}%
& \uncover<16->{ = } %
\uncover<16->{%
\frac{1}{4\pi (\alert<handout:0| 16>{25})^2}\alert<handout:0| 16>{100} \uncover<17->{ = \frac{1}{25\pi}}%
}%
\end{align*}
\end{columns}
%\uncover<17->{%
%Therefore the $r$ is increasing at a rate of $1/(25\pi )$ cm/s when $r = 25$ cm.
%}%
\end{example}
\end{frame}
% end module related-rates-ex1
