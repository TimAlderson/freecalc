\begin{frame}
\frametitle{Distances and Angles}
\begin{itemize}
\item In the preceding slides/lectures we saw distinguishing configurations of lines and planes doesn'tnot require the notion of angle or distance. \uncover<2->{Nonetheless the latter are fundamental.}
\item<3-> Distance is a function that assigns to two points $A,B$ the non-negative number $|AB|$ that quantifies/measures how close/far apart are the points. We denote distance also by $d(A,B)$.
\item<4-> From elementary Euclidean geometry: if we know the lengths of the sides of a triangle, we know \uncover<6->{the \alert<6>{magnitude} of} its angles. 
\item<5-> So the notion of \uncover<6->{\alert<6>{magnitude} of} angle follows from that of distance. 
\item<6-> We note that knowing distances determines \alert<6>{magnitudes of angles} but \alert<6>{not their signs}. 
\item<7-> Signs of angles are a manifestation of the fundamental concept of orientation, which we will study later.
\item<8-> We recall two intersecting lines are perpendicular when the angle between them is $\pm \frac{\pi}{2}$. 
\end{itemize}

\end{frame}