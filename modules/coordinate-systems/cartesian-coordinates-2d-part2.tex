\begin{frame}
\frametitle{Rectangular/Cartesian Coordinates}
\begin{columns}
\column{0.3\textwidth}
\psset{xunit=2cm, yunit=2cm}
\begin{pspicture}(-1 ,-1)(1, 1)%
\fcBoundingBox{-1}{-1}{1}{1}%
\tiny%
\psline[arrows=->](0,-1)(0,1)%
\psline[arrows=->](-1,0)(1,0)%
\rput[t](1, -0.1){$x$}%
\rput[l](-0.1, 1){$y$}%
%\fcFullDot{0}{0}
\fcFullDot{0.7}{0.5}%
\rput[b](0.7, 0.6){$P(x,y)$}%
\fcPerpendicular{[0.7 0.5]}{[0 1]}{0.07}%
\fcPerpendicular{[0.7 0.5]}{[1 0]}{0.07}%
\fcLengthIndicator[arrows=->]{0}{-0.1}{0.7}{-0.1}{$x$}%
\fcLengthIndicator[arrows=->]{-0.1}{0}{-0.1}{0.5}{$y$}%
\psline[linecolor=red](-0.05,0)(-0.15,0)
\psline[linecolor=red](-0.05,0.5)(-0.15,0.5)
\psline[linecolor=red](0,-0.05)(0,-0.15)
\psline[linecolor=red](0.7,-0.05)(0.7,-0.15)
\end{pspicture}
\column{0.7\textwidth}
\begin{itemize}
\item<1-> Let $P$ -point. We assign to it a pair of numbers $(x,y)$.
\item<2-> Assignment will be such that distinct points are assigned distinct pairs.
\item<3-> $Q=$ base of perpendicular from $P$ to $x$-axis.
\item<4-> Define $x$ as \alert<5>{signed distance b-n $O$ and $Q$}.
\item<5-> Take distance with \alert<5>{$+$ sign if $OQ$ points in direction of $x$-axis, $-$ sign else}.
\item<6-> To define $y$, do the same with the $y$ axis.
\item<7-> $(x,y)$ = Cartesian coordinates of $P$. 
\item<8-> $x$ is called the $x$-coordinate of $P$, $y$- the $y$ coordinate.
\item<9-> $(x,y)$ = singed lengths of sides of the rectangle indicated in the picture.
\end{itemize}

\vfill
\end{columns}

\vskip 5cm

\end{frame}
