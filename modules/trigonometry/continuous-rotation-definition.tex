\begin{frame}
\vskip -0.2cm
\begin{columns}
\column{0.25\textwidth}
\begin{pspicture}(-1,-1)(6, 4)%
\tiny%
\fcBoundingBox{-1.2}{-1.2}{1.2}{1.2}%
\pstVerb{20 dict begin /endAngle 60 def}%
\only<8->{\pstVerb{/endAngle 120 def}}%
\only<9->{\pstVerb{/endAngle 180 def}}%
\only<10->{\pstVerb{/endAngle 240 def}}%
\only<11->{\pstVerb{/endAngle 300 def}}%
\only<12->{\pstVerb{/endAngle 360 def}}%
\only<13->{\pstVerb{/endAngle 420 def}}%
%\psline[arrows=->](0,0)(2,0)
%\psline[arrows=->](0,0)(! endAngle cos 2 mul endAngle sin 2 mul )
%\parametricplot[arrows=->, linecolor=blue]{0}{endAngle}{t cos 0.2 mul t sin 0.2 mul}
%\parametricplot[linestyle=dashed, linecolor=red!60]{endAngle}{360}{t cos t sin}
\parametricplot[arrows=->, linecolor=red]{0}{endAngle}{t cos t sin}%
\fcFullDot[linecolor=blue]{0}{0}%
\rput[t] (-0.2, -0.2){$O$}%
\fcFullDot{1}{0}%
\rput[t](1, -0.2){$A$}%
\fcFullDot{endAngle cos}{endAngle sin}%
\rput[tr](! endAngle cos endAngle sin ){$B~$}%
\pstVerb{end}%
\end{pspicture}

\column{0.75\textwidth}
\begin{definition}[Continuous rotation]
A continuous rotation about a point, which we call the center of rotation, is a motion of points in space with the following properties.
\begin{itemize}
\item<2-> All points move in a circular fashion around the center of rotation. 
\item<3-> The distance between each rotated point and the center of rotation does not change.
\item<4-> The distance between each pair of rotated points is preserved.
\end{itemize}
\end{definition}
\vskip -0.2cm
\begin{itemize}
\item<5-> The position of a point under a continuous rotation is assumed to be a function of time.
\item<6-> The trajectory of a point is an arc of a circle. 
\item<7-12,13-> A point can traverse a full circle - more than once. In this case the moving point passes through the same positions more than once.
\end{itemize}
\end{columns}
\vskip 10cm
\end{frame}