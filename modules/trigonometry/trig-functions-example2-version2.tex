% begin module trig-functions-example2
\begin{frame}
\begin{example}
If $\cos \theta = \frac{2}{5}$ and $0 < \theta < \pi /2$, find the other five trigonometric functions of $\theta$.
\begin{columns}[c]
\column{.3\textwidth}

\psset{xunit=1cm,yunit=1cm}
\begin{pspicture}(-0.5,0)(2.5,5.5)
\rput[l](2,2.6){ $x={\sqrt{21}} $}
\rput[br](0.95,2.5){ $5$}
\rput[t](1, -0.1){$2$}
\psframe*[linecolor=white, fillcolor=white](-0.5,-0.5)(4,5.5)
\uncover<2->{
\rput[l](2,2.6){ $x=\uncover<5->{\alert<5,7,9,11,15>{\sqrt{21}}} $}
\rput[br](0.95,2.5){ \alert<7,11,13>{$5$}}
\rput[t](1, -0.1){\alert<9,13, 15>{$2$}}
}
\psline(0,0)(2,0)(2,5)(0,0)
\psline(1.8,0)(1.8,0.2)(2,0.2)
\psarc[linecolor=red](0,0){0.3}{0}{68.19859}
\rput(0.4,0.3){$\theta$}
\end{pspicture}

\column{.7\textwidth}
\begin{itemize}
\item<2->  Label the hypotenuse with length 5 and the adjacent side with length 2.
\item<3->  Pythagorean theorem: $x^2 +2^2 = 5^2$.
\item<4->  Therefore $x^2 = \uncover<5->{\alert<5>{21}}$, so $x = \uncover<5->{\alert<5>{\sqrt{21}}}$.
\end{itemize}
\[
\begin{array}{cc}
\alert<handout:0| 6-7>{%
\sin \theta = %
\uncover<7->{%
\frac{\sqrt{21}}{5}%
}}&%
\alert<handout:0| 8-9>{%
\tan \theta = %
\uncover<9->{%
\frac{\sqrt{21}}{2}%
}}\\%
& \\
\alert<handout:0| 10-11>{%
\csc \theta = %
\uncover<11->{%
\frac{5}{\sqrt{21}}%
}}&%
\alert<handout:0| 12-13>{%
\sec \theta = %
\uncover<13->{%
\frac{5}{2}%
}}\\%
& \\
\alert<handout:0| 14-15>{%
\cot \theta = %
\uncover<15->{%
\frac{2}{\sqrt{21}}%
}}&%
\end{array}
\]
\end{columns}
\end{example}
\end{frame}
% end module trig-functions-example2
