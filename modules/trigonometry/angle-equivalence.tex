\begin{frame}
\frametitle{Equivalence of angles}
\begin{columns}
\column{0.25\textwidth}
\begin{pspicture}(-1,-1)(6, 4)%
\tiny%
\fcBoundingBox{-1}{-4}{2}{2}%
\psline[arrows=->](0,0)(2,0)%
\psline[arrows=->](0,0)(! 1 3 sqrt)%
\pstVerb{20 dict begin}%
\pstVerb{%
/DeltaAngle 60 def %
/startAngle1 0 def %
/startAngle2 45 def %
/endAngle1 {startAngle1 DeltaAngle add} def %
/endAngle2 {startAngle2 DeltaAngle add} def %
/coordX1 0 def %
/coordY1 0 def %
/coordX2 0.3 def %
/coordY2 -3 def %
}%
\only<2->{\pstVerb{/coordX2 0.2 def /coordY2 -2 def}}%
\only<3->{\pstVerb{/coordX2 0.1 def /coordY2 -1 def}}%
\only<4->{\pstVerb{/coordX2 0 def /coordY2 0 def}}%
\only<5->{\pstVerb{/startAngle2 30 def}}%
\only<6->{\pstVerb{/startAngle2 15 def}}%
\only<7->{\pstVerb{/startAngle2 0 def}}%
\fcAngleBetweenVectors[arrows=->, linecolor=blue]{[1 0]}{[0.5 3 sqrt 2 div]}{0.3}{}%
\fcFullDot[linecolor=blue]{coordX1}{coordY1}%
\rput[t](! coordX1 -0.2 add coordY1 -0.2 add){$O$}%
\fcFullDot{startAngle1 cos 1.5 mul coordX1 add}{startAngle1 sin 1.5 mul coordY1 add}%
\rput[t](1.5, -0.2){$A$}%
\fcFullDot{endAngle1 cos 1.5 mul coordX1 add }{endAngle1 sin 1.5 mul coordY1 add}%
\rput[rt](0.6, 1.5){$B$}%
\psline[arrows=->](! coordX2 coordY2)(! startAngle2 cos 2 mul coordX2 add startAngle2 sin 2 mul coordY2 add)%
\psline[arrows=->](! coordX2 coordY2)(! endAngle2 cos 2 mul coordX2 add endAngle2 sin 2 mul coordY2 add)%
\parametricplot[arrows=->, linecolor=blue]{startAngle2}{endAngle2}{t cos 0.3 mul coordX2 add t sin 0.3 mul coordY2 add}%
\fcFullDot[linecolor=blue]{coordX2}{coordY2}%
\rput[t] (! coordX2 coordY2 -0.2 add){$O_1$}%
\fcFullDot{startAngle2 cos 1.5 mul coordX2 add}{startAngle2 sin 1.5 mul coordY2 add}%
\rput[lt] (! startAngle2 cos 1.5 mul coordX2 add startAngle2 sin 1.5 mul coordY2 add -0.1 add){$A_1$}%
\fcFullDot{endAngle2 cos 1.5 mul coordX2 add}{endAngle2 sin 1.5 mul coordY2 add}%
\rput[tr] (! endAngle2 cos 1.5 mul coordX2 add endAngle2 sin 1.5 mul coordY2 add){$B_1~~$}%
\pstVerb{end}%
\end{pspicture}
\column{0.75\textwidth}
\begin{definition}[Congruent angles]
Two geometric angles are congruent (equivalent) if they one can be transformed onto the other with a sequence of translations and rotations. \uncover<1-7>{}
\end{definition}
\uncover<8->{
\begin{proposition}
Two geometric angles are congruent if and only if they have equal angle measures.
\end{proposition}
}
\vskip -0.1cm
\begin{itemize}
\item<9-> Recall ``angle'' refers to both geometric angle and angle measure (depending on context). 
\item<10-> The expression \alertNoH{10}{``the two angles are equal''} is to be interpreted as \alertNoH{10}{``the angle measures are equal''} and therefore \alertNoH{11}{``the geometric angles are congruent''}. 
\end{itemize}
\end{columns}
\end{frame}