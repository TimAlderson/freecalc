\begin{frame}
\frametitle{Cofunction identities}
\vskip -0.1cm
\begin{proposition}[Cofunction identities]
\hfil \hfil$
\renewcommand{\arraystretch}{1.4}
\begin{array}{rclrcl}
\displaystyle \sin \left(\frac{\pi}{2}-\alpha\right)&=&\cos \alpha &\displaystyle \sin \left(\frac{\pi}{2}+\alpha\right)&=&\cos \alpha \\
\displaystyle \cos \left(\frac{\pi}{2}- \alpha\right)&=&\sin \alpha &\displaystyle \cos \left(\frac{\pi}{2}+\alpha\right)&=&-\sin \alpha
\end{array}
$
\end{proposition}
\begin{itemize}
\item<2-> The proof each formula is broken into $4$ cases depending on which quadrant contains $\alpha$.
\item<3-> This makes a total of $4 $ formulas $\times 4$ cases per formula = $16$ cases.
\item<4-> We show only a few of the cases.
\item<5-> The proof provides intuition why the formulas are true. \item<6-> The Quadrant I part of the proof serves as a visual aid for memorization. 
\item<7-> There is an algebraically simpler (but theoretically advanced) way to prove the above identities through the angle sum f-las, derived in turn from Euler's formula (studied later/in another course).

\end{itemize}

\vskip 10cm
\end{frame}


