\begin{frame}
\begin{example}
\begin{columns}
\column{0.3\textwidth}
\psset{xunit=3cm, yunit=3cm}
\begin{pspicture}(-0.5,-0.5)(1.5,1.5)
\tiny
\fcBoundingBox{-0.05}{-0.15}{1.15}{1.1}
\psline(0,0)(1,0)(1,1)(0,0)
\fcPerpendicular{[1 1]}{[1 0]}{0.1}
\parametricplot[linecolor=red]{0}{45}{t cos 0.2 mul t sin 0.2 mul}
\parametricplot[linecolor=red]{-90}{-135}{t cos 0.2 mul 1 add t sin 0.2 mul 1 add}
\rput[l](0.2, 0.1){$45^\circ$}
\rput[t](0.9, 0.8){$\gamma=45^\circ$}
\rput[t](0.5, -0.05){$1$}
\rput[l](1.05, 0.5){$1$}
\end{pspicture}
\column{0.7\textwidth}
Find the values of $\sin 45^{\circ}, \cos45^{\circ}, \tan45^{\circ}$.

\uncover<2->{
Realize the $45^\circ$ angle in a right angle triangle so that the adjacent side has length $1$ and let $\gamma$ be the other acute angle of the triangle. The angles in a triangle sum to $180^\circ$, and so 
\[
\begin{array}{rcl}
45^\circ+90^\circ +\gamma &=& 180^\circ\\
\gamma&=&45^\circ.
\end{array}
\]
}
\end{columns}
\uncover<2->{Then the right angle triangle has two equal angles and is therefore isosceles (has two sides of equal length). Thus the two legs are of length $1$ and the hypothenuse is of length $\sqrt{1^2+1^2}=\sqrt{2}$.

Therefore 

\hfil \hfil $ \displaystyle \sin 45^\circ = \frac{\sqrt{2}}{2}\qquad \cos 45^\circ = \frac{\sqrt{2}}{2}\qquad \tan 45^\circ = \frac{\sqrt{2}}{\sqrt{2}}=1.
$
}
\end{example}

\end{frame}