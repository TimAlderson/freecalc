% begin module trig-identities
\begin{frame}
\frametitle{Trigonometric Identities}
\begin{definition}[Trigonometric Identity]
A trigonometric identity is a relationship among the trigonometric functions that is true for any value of the independent variable.
\end{definition}
\end{frame}

\newcommand{\trigIdentitiesPicture}{
\psset{xunit=1cm,yunit=1cm}
\begin{pspicture}(-4,-0.5)(1,4)
\psaxes[labels=none, ticks=none]{<->}(0,0)(-4,-0.5)(1,4)
\pscircle*(-3,2){0.07}
\psline[linecolor=blue](0,0)(-3,2)
\psline[linecolor=blue](0,0)(1,0)
\psarc[linecolor=red](0,0){0.5}{0}{146.3099}
\rput[br](-3, 2){$(x,y)$}
\rput[l](0.1, 0.7){$\theta$}
\rput[lb](-1.55, 1.1){$r$}
\psline[linestyle=dotted](-3, 2)(-3, 0)
\psline[linestyle=dotted](-3, 2)(0, 2)
\psline(-2.7, 0)(-2.7, 0.3)(-3, 0.3)
\psline(0, 1.7)(-0.3, 1.7)(-0.3, 2)
\end{pspicture}
}

\begin{frame}
\begin{columns}[c]
\column{.5\textwidth}
\trigIdentitiesPicture
\[
\begin{array}{cc}
\sin \theta = \frac{ y}{ r} &
\csc \theta = \frac{ r}{ y} \\
\cos \theta = \frac{ x}{ r} &
\sec \theta = \frac{ r}{ x} \\
\tan \theta = \frac{ y}{ x} &
\cot \theta = \frac{ x}{ y} \\
\end{array}
\]
\column{.5\textwidth}
\begin{itemize}
\item $\csc \theta = \frac{1}{\sin \theta}$
\item $\sec \theta = \frac{1}{\cos \theta}$
\item $\cot \theta = \frac{1}{\tan \theta}$
\item $\tan \theta = \frac{\sin \theta}{\cos \theta}$
\item $\cot \theta = \frac{\cos \theta}{\sin \theta}$
\end{itemize}
\end{columns}
\end{frame}


\begin{frame}
\begin{columns}[c]
\column{.5\textwidth}
\trigIdentitiesPicture

\[
\begin{array}{cc}
\sin \theta = \frac{ y}{ r} &
\csc \theta = \frac{ r}{ y} \\
\cos \theta = \frac{ x}{ r} &
\sec \theta = \frac{ r}{ x} \\
\tan \theta = \frac{ y}{ x} &
\cot \theta = \frac{ x}{ y} \\
\end{array}
\]
\column{.5\textwidth}
\begin{eqnarray*}
& & \uncover<2->{\sin^2 \theta + \cos^2 \theta}\\
& \uncover<3->{=} & \uncover<3->{\frac{y^2}{r^2} + \frac{x^2}{r^2}}\\
& \uncover<4->{=} & \uncover<4->{\frac{y^2+x^2}{r^2}}\\
& \uncover<5->{=} & \uncover<5->{\frac{r^2}{r^2}}\\
& \uncover<6->{=} & \uncover<6->{1}
\end{eqnarray*}
\uncover<7->{%
Therefore $\sin^2 \theta + \cos^2 \theta = 1$.%
}%
\end{columns}
\end{frame}


\begin{frame}
\begin{columns}[c]
\column{.5\textwidth}
\trigIdentitiesPicture
\[
\begin{array}{cc}
\sin \theta = \frac{ y}{ r} &
\csc \theta = \frac{ r}{ y} \\
\cos \theta = \frac{ x}{ r} &
\sec \theta = \frac{ r}{ x} \\
\tan \theta = \frac{ y}{ x} &
\cot \theta = \frac{ x}{ y} \\
\end{array}
\]
\column{.5\textwidth}
\begin{example}[$\tan^2 \theta + 1 = \sec^2 \theta$]
Prove the identity $\tan^2 \theta + 1 = \sec^2 \theta$.
\begin{eqnarray*}
\uncover<2->{\sin^2 \theta + \cos^2 \theta} & \uncover<2->{=} & \uncover<2->{1}\\
\uncover<3->{\frac{\sin^2 \theta}{\cos^2\theta} + \frac{\cos^2 \theta}{\cos^2\theta}} & \uncover<3->{=} & \uncover<3->{\frac{1}{\cos^2\theta}}\\
\uncover<4->{\tan^2 \theta + 1} & \uncover<4->{=} & \uncover<4->{\sec^2\theta}
\end{eqnarray*}
\end{example}
\end{columns}
\end{frame}



\begin{frame}
\begin{columns}[c]
\column{.5\textwidth}
\psset{xunit=1cm,yunit=1cm}
\begin{pspicture}(-4,-2.5)(1,4)
\psaxes[labels=none, ticks=none]{<->}(0,0)(-4.2,-2.5)(1,4)
\pscircle*(-3,2){0.07}
\psline[linecolor=blue](0,0)(-3,2)
\psline[linecolor=blue](0,0)(1,0)
\rput[br](-3, 2){$(x,y)$}
\rput[lb](-1.55, 1.1){$r$}

\uncover<1>{
\psarc[linecolor=red, linewidth=2pt]{->}(0,0){0.5}{0}{146.3099}
\rput[l](0.1, 0.7){$\theta$}
}
\uncover<2>{
\psarc[linecolor=red, linewidth=2pt]{<-}(0,0){0.5}{0}{146.3099}
\rput[l](0.1, 0.7){$-\theta$}
}
\uncover<3->{
\psarc[linecolor=red]{->}(0,0){0.5}{0}{146.3099}
\rput[l](0.1, 0.7){$\theta$}
}
\uncover<4->{
\psarc[linecolor=red]{<-}(0,0){0.5}{-146.3099}{0}
\rput[lt](0.1, -0.7){$-\theta$}
\psline[linecolor=blue](0,0)(-3,-2)
\psline[linecolor=blue](0,0)(1,0)
\rput[br](-3, -2){$(x,-y)$}
\rput[lt](-1.55, -1.1){$r$}
}
\uncover<5->{\psline(-2.7, 0)(-2.7, -0.3)(-3, -0.3)
\psline[linestyle=dotted](-3, -2)(-3, 0)
}

\psline[linestyle=dotted](-3, 2)(-3, 0)
\psline[linestyle=dotted](-3, 2)(0, 2)
\psline(-2.7, 0)(-2.7, 0.3)(-3, 0.3)
\psline(0, 1.7)(-0.3, 1.7)(-0.3, 2)
\end{pspicture}
\[
\begin{array}{cc}
\sin \theta = \frac{ y}{ r} &
\csc \theta = \frac{ r}{ y} \\
\cos \theta = \frac{ x}{ r} &
\sec \theta = \frac{ r}{ x} \\
\tan \theta = \frac{ y}{ x} &
\cot \theta = \frac{ x}{ y} \\
\end{array}
\]
\column{.5\textwidth}
\begin{itemize}
\item<1->  Positive angles are obtained by rotating counterclockwise.
\item<2->  Negative angles are obtained by rotating clockwise.
\item<3->  If $(x,y)$ is on the terminal arm of the angle $\theta$, \alert<4>{then $(x, -y)$ is on the terminal arm of $-\theta$}.
\item<5->  $\alert<7>{\sin(-\theta )} = \frac{-y}{r} = -\frac{y}{r} = \alert<7>{-\sin \theta}$.
\item<6->  $\alert<8>{\cos(-\theta )} = \frac{x}{r} =\alert<8>{\cos \theta}$.
\item<7->  $\sin$ is an \alert<7>{odd function}.
\item<8->  $\cos$ is an \alert<8>{even function}.
\end{itemize}
\end{columns}
\end{frame}



\begin{frame}
\begin{columns}[c]
\column{.5\textwidth}
\psset{xunit=1cm,yunit=1cm}
\begin{pspicture}(-4,-2.5)(1,4)
\psaxes[labels=none, ticks=none]{<->}(0,0)(-4.2,-2.5)(1,4)
\pscircle*(-3,2){0.07}
\psline[linecolor=blue](0,0)(-3,2)
\psline[linecolor=blue](0,0)(1,0)
\rput[br](-3, 2){$(x,y)$}
\rput[lb](-1.55, 1.1){$r$}

\psarc[linecolor=red]{->}(0,0){0.5}{0}{146.3099}
\rput[l](0.1, 0.7){$\theta$}

\uncover<2>{
\parametricplot[linecolor=red, plotpoints=1000, arrows=->]{0}{360}{0.25 0.000198413 t mul add t cos mul 0.25 0.000198413 t mul add t sin mul}
}

\uncover<3->{
\parametricplot[linecolor=red, plotpoints=1000, arrows=->]{0}{506.3099}{0.25 0.000198413 t mul add t cos mul 0.25 0.000198413 t mul add t sin mul}
}

\psline[linestyle=dotted](-3, 2)(-3, 0)
\psline[linestyle=dotted](-3, 2)(0, 2)
\psline(-2.7, 0)(-2.7, 0.3)(-3, 0.3)
\psline(0, 1.7)(-0.3, 1.7)(-0.3, 2)
\end{pspicture}

\[
\begin{array}{cc}
\sin \theta = \frac{ y}{ r} &
\csc \theta = \frac{ r}{ y} \\
\cos \theta = \frac{ x}{ r} &
\sec \theta = \frac{ r}{ x} \\
\tan \theta = \frac{ y}{ x} &
\cot \theta = \frac{ x}{ y} \\
\end{array}
\]
\column{.5\textwidth}
\begin{itemize}
\item<2->  $2\pi$ represents a full rotation.
\item<3->  $\theta + 2\pi$ has the same terminal arm as $\theta$.
\item<4->  $\theta + 2\pi$ uses the same point $(x,y)$ and the same length $r$.
\item<5->  $\sin (\theta + 2\pi ) = \sin \theta$.
\item<5->  $\cos (\theta + 2\pi ) = \cos \theta$.
\item<6->  We say $\sin$ and $\cos$ are $2\pi$-periodic.
\end{itemize}
\end{columns}
\end{frame}


\begin{frame}[t]
The remaining identities are consequences of the addition formulas:
\[
\begin{array}{ccccc}
\sin (x + y) & = & \sin x\cos y & + & \cos x \sin y \\
\cos (x + y) & = & \cos x\cos y & - & \sin x \sin y 
\end{array}
\]
\uncover<2->{
Substitute $-y$ for $y$, and use the fact that $\sin(-y) = -\sin y$ and $\cos (-y) = \cos y$:
\[
\begin{array}{ccccc}
\sin (x - y) & = & \sin x\cos y & - & \cos x \sin y \\
\cos (x - y) & = & \cos x\cos y & + & \sin x \sin y 
\end{array}
\]
}
\end{frame}


\begin{frame}[t]
The remaining identities are consequences of the addition formulas:
\[
\begin{array}{ccccc}
\sin (x + y) & = & \sin x\cos y & + & \cos x \sin y \\
\cos (x + y) & = & \cos x\cos y & - & \sin x \sin y 
\end{array}
\]
\uncover<2->{
To get the double angle formulas, substitute $x$ for $y$:
\[
\begin{array}{rcl}
\sin 2x  & = & 2\sin x\cos x \\
\cos 2x  & = & \cos^2 x - \sin^2 x
\end{array}
\]
}
\uncover<3->{
Rewrite the second double angle formula in two ways, using $\cos^2 x = 1 - \sin^2 x$ and $\sin^2 x = 1 - \cos^2 x$:
\[
\begin{array}{rcl}
\cos 2x  & = & 2\cos^2 x  -1\\
\cos 2x  & = & 1 - 2\sin^2 x
\end{array}
\]
}
\uncover<4->{
To get the half-angle formulas, solve these equations for $\cos^2 x$ and $\sin^2 x$ respectively.
\[
\cos^2 x = \frac{1 + \cos 2x}{2}, \qquad \sin^2 x = \frac{1 - \cos 2x}{2} 
\]
}
\end{frame}


\begin{frame}[t]
The remaining identities are consequences of the addition formulas:
\[
\begin{array}{ccccc}
\sin (x + y) & = & \sin x\cos y & + & \cos x \sin y \\
\cos (x + y) & = & \cos x\cos y & - & \sin x \sin y 
\end{array}
\]
\uncover<2->{
Divide the first equation by the second, and then cancel $\cos x \cos y$ from the top and bottom:
\[
\begin{array}{rcl}
\tan (x + y)  & = & \frac{\tan x + \tan y}{1 - \tan x \tan y}
\end{array}
\]
}
\uncover<3->{
Do the same for the subtraction formulas:
\[
\begin{array}{rcl}
\tan (x - y)  & = & \frac{\tan x - \tan y}{1 + \tan x \tan y}
\end{array}
\]
}
\end{frame}
% end module trig-identities
