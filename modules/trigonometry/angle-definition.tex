\begin{frame}
\frametitle{Geomtric angle definition}
\vskip -0.2cm
\begin{columns}
\column{0.22\textwidth}
~\\~\\
~\\~\\
\psset{xunit=2cm, yunit=2cm}
\begin{pspicture}(-0.25,-1)(1.4,1.6)%
\tiny%
\fcBoundingBox{-0.25}{-1}{1.4}{1}%
\pstVerb{20 dict begin /startAngle -30 def /endAngle 30 def /angleLength 1.2 def}%
\only<handout:0|13>{\pstVerb{/startAngle 0 def /endAngle 0 def}}%
\only<1-12,14-23,27->{\fcAngleBetweenVectors[arrows=->, linecolor=blue]{[startAngle cos startAngle sin]}{[endAngle cos endAngle sin]}{0.4}{}}%
\only<handout:0|24-26>{\fcAngleBetweenVectors[arrows=<-, linecolor=blue]{[startAngle cos angleLength mul startAngle sin angleLength mul]}{[endAngle cos angleLength mul endAngle sin angleLength mul]}{0.4}{}}%
\only<handout:0|26>{\fcAngleBetweenVectors[arrows=<-, linecolor=blue, linewidth=2pt]{[startAngle cos angleLength mul startAngle sin angleLength mul]}{[endAngle cos endAngle sin]}{0.4}{}}%
\psline[arrows=->](0,0)(! startAngle cos angleLength mul startAngle sin  angleLength mul)%
\psline[arrows=->](0,0)(! endAngle cos angleLength mul endAngle sin angleLength mul)%
\only<handout:0|2,8,10,22,25>{\psline[arrows=->, linewidth=2pt, linecolor=red](0,0)(! startAngle cos angleLength mul startAngle sin angleLength mul)}%
\only<handout:0|2,8,11,23,24>{\psline[arrows=->, linewidth=2pt, linecolor=red](0,0)(! endAngle cos angleLength mul endAngle sin angleLength mul)}%
\only<handout:0|3,12>{%
\fcAngleBetweenVectors[arrows=->, linecolor=blue, linewidth=1.5pt]{[startAngle cos startAngle sin]}{[endAngle cos endAngle sin]}{0.4}{}%
}%
\only<handout:2|9->{\fcFullDot[linecolor=blue]{0}{0}}%
\uncover<handout:2-|10-13,22,23>{\rput[t]{! startAngle}(! startAngle cos 2 div startAngle sin 2 div 0.1 sub){initial arm}}%
\uncover<handout:2-|11-13,23>{\rput[b]{! endAngle}(! endAngle cos 2 div endAngle sin 2 div 0.1 add){terminal arm}}%
\uncover<handout:0|24-26>{\rput[t]{! startAngle}(! startAngle cos 2 div startAngle sin 2 div 0.1 sub){terminal arm}}%
\uncover<handout:0|24-26>{\rput[b]{! endAngle}(! endAngle cos 2 div endAngle sin 2 div 0.1 add){initial arm}}%
\uncover<15->{%
\fcFullDot{startAngle cos 0.9 mul}{startAngle sin 0.9 mul}%
}%
\uncover<16->{%
\fcFullDot{endAngle cos 1 mul}{endAngle sin 1 mul}%
}%
\uncover<17->{
\rput[t](! startAngle cos 0.9 mul startAngle sin 0.9 mul 0.1 sub){$A$}%
}
\uncover<21>{
\fcFullDot{startAngle cos 0.5 mul}{startAngle sin 0.5 mul}
\rput[t](! startAngle cos 0.5 mul startAngle sin 0.5 mul 0.1 sub){$A'$}%
}
\uncover<18->{
\rput[b](! endAngle cos 1 mul 0 add endAngle sin 1 mul 0.1 add){$B$}%
}
\uncover<19->{
\rput[t](0,-0.1){$O$}
}
\pstVerb{end}%
\end{pspicture}

~\\
~\\
~\\
~\\

\column{0.78\textwidth}
\begin{itemize}
\only<handout:1|1-6>{
\item<1-> The term angle is used to denote two distinct mathematical objects: 
\begin{itemize}
\item<2-> \alertNoH{2}{the angle formed by two rays} and
\item<3-> \alertNoH{3}{the angle-measure of such a figure.}
\end{itemize}
\item<4-> \alertNoH{4}{Both are referred to as ``angle''}: one is expected to use \alertNoH{5}{context} to decide whether ``angle'' means \alertNoH{5}{``angle formed by two rays''} or \alertNoH{5}{``angle measure''}.
\item<6-> We take full advantage of this convention. 
}
\only<handout:2|7-13>{
\begin{definition}
A \alertNoH{7}{\emph{geometric angle}} (\alertNoH{7}{\emph{angle}} for short) is the figure formed by \alertNoH{8}{two rays, called arms}, sharing a common endpoint called the \alertNoH{9}{vertex} of the angle. \uncover<10->{The rays are ordered.}
\end{definition}
\item<10-> The ray that comes first is called the \alertNoH{10}{initial arm (side)} of the angle.
\item<11-> The ray that comes second is called the \alertNoH{10}{terminal arm (side)} of the angle.
\item<12-> Angle measures are depicted as arcs pointing from the initial arm towards the terminal arm.
\item<13-> By convention, the rays are allowed to coincide; the resulting angle is then called the \alertNoH{13}{\emph{zero angle}}.
}
\only<handout:3|14->{
\item<14-> A ray can be identified by its starting point and any other point on the ray.
\item<15-> Therefore an angle can be identified by its vertex and one point on each of its arms. 
\item<17-> \alertNoH{17}{If $A$ is pt. on the first ray} and \alertNoH{18}{$B$ on the second} and \alertNoH{19}{$O$ is the vertex}, we denote the angle by \alertNoH{20}{$\angle AOB $}. 
\item<21-> The choice $A$ and $B$ is not unique - for example $\angle AOB$ and $\angle A'OB$ coincide. 

\item<22-> In $\angle \alertNoH{22}{A}\alertNoH{22,23}{O}\alertNoH{23}{B}$ the ray $\alertNoH{22}{OA}$ is the initial arm and the ray  $\alertNoH{23}{OB}$ is the terminal arm.
\item<24-> In $\angle BOA$ the ray $\alertNoH{24}{OB}$ is the initial arm, the ray $\alertNoH{25}{OA}$ is the terminal arm, and the angle measure \alertNoH{26}{points in the opposite direction}. 
\item<27->{In this way $\angle AOB\neq \angle BOA$.}

}

\end{itemize}


\end{columns}


\vskip 20cm

\end{frame}