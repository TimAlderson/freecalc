\begin{frame}
\frametitle{Proof of a Trigonometric Identity}
Let $F$ and $G$ be expressions that give a trigonometric identity:
$
F(\sin\theta, \cos\theta)=G(\sin\theta, \cos \theta).
$
\begin{itemize}
\item<2-> To prove a trigonometric identity means to show that the two sides of the equality sign are equivalent. 
\item<3-> There are two ways to do this (in the present course the first way will be preferred).
\item<4-> First method: transform the left and right hand sides to an equal expression. In particular:
\begin{itemize}
\item<5-> we can choose to transform the left hand side to the right;
\item<6-> we can choose to transform the right hand side to the left;
\item<7-> we can choose to transform both sides to a third equivalent expression.
\end{itemize}
\item<8-> Second method: start with an already known identity and transform it, by a series of equivalent transformations, to the identity we desire to prove.
\item<9-> The discussion here also applies for trigonometric identities in more than one variables.
\end{itemize}
\end{frame}