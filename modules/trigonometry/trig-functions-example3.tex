\begin{frame}
\begin{example}
\begin{columns}
\column{0.3\textwidth}
\psset{xunit=0.7cm, yunit=0.7cm}
\begin{pspicture}(-1,-1)(4.5,3)
\tiny
\fcBoundingBox{-0.5}{-0.5}{4.5}{3.5}
\fcPerpendicular{[4 3]}{[1 0]}{0.4}
\psline(0,0)(4,0)(4,3)(0,0)
\fcAngleBetweenVectors{[1 0]}{[4 3]}{0.5}{}
\rput[l](0.5, 0.2){$\theta$}
\only<handout:0|10, 12,16>{\psline[linewidth=2pt, linecolor=green](0,0)(4,0)}
\only<handout:0|8, 12, 14,18>{\psline[linewidth=2pt, linecolor=blue](4,0)(4,3)}
\only<handout:0|8, 10, 14,16>{\psline[linewidth=2pt, linecolor=orange](4, 3)(0, 0)}
\rput[t](2, -0.2){$\only<handout:0|10,12,16,18>{\color{green}}4$}
\rput[l](4.2, 1.5){$\only<handout:0|8,12,14,18>{\color{blue}} 3$}
\rput[bl](1.9, 1.7){$\fcAnswerUncover{2}{6}{ \only<handout:0|8, 10,14, 16>{\color{orange}} 5} $}
\end{pspicture}

\column{0.7\textwidth}
Let the angle $\theta$ be as indicated in the figure. Find the values of the six trigonometric functions of $\theta$.
\end{columns}
\uncover<2->{To find the trigonometric functions, we need to know the length of the hypotenuse. } \uncover<3->{
\[
\text{hypotenuse}=\fcAnswer{4}{\sqrt{\alertNoH{5}{4^2+3^2}}}\uncover<5->{=\alertNoH{6}{\sqrt{\alertNoH{5}{25}}}}\uncover<6->{=\alertNoH{6}{5}.}
\]
}
\uncover<7->{Using the right angle triangle ratio interpretations of the trig functions, we can compute:}
\[
\begin{array}{@{}r@{}c@{}lr@{}c@{}lr@{}c@{}l}
\displaystyle \alert<handout:0| 7-8>{\sin\theta} & \alertNoH{7-8}{=}& \displaystyle \fcAnswer{8}{ \frac{3}{5}}  &
\displaystyle \alert<handout:0| 9-10>{\cos\theta} &\alertNoH{9-10}{=}& \displaystyle \fcAnswer{10}{\frac{ 4}{5}} &
\displaystyle \alert<handout:0| 11-12>{\tan \theta} &\alertNoH{11-12}{=}&\displaystyle   \fcAnswer{12}{\frac{3}{4}} \\
\displaystyle  \alert<handout:0| 13-14>{\csc \theta} &\alertNoH{13-14}{=}&\displaystyle   \fcAnswer{14}{\frac{5}{3}} &\displaystyle  
\alert<handout:0| 15-16>{\sec \theta} &\alertNoH{15-16}{=}&\displaystyle   \fcAnswer{16}{\frac{5}{4}} &\displaystyle  
\alert<handout:0| 17-18>{\cot \theta} & \alertNoH{17-18}{ =}&\displaystyle   \fcAnswer{18}{\frac{4}{3}}
\end{array}
\]

\end{example}

\end{frame}