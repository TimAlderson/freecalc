\begin{frame}
\frametitle{Strategy for proving trigonometric identities}
\begin{question}
\alertNoH{1,6}{Is there a general method for proving all trigonometric identities in one variable?}
\end{question}
\only<handout:1|2-5>{
\begin{itemize}
\item<2-> Given a number of variables and relations between them, there is an algorithm to check whether (rational) expressions in those variables are equal under the given relations. 
\item<3-> Thus, if we pick two variables $s$ and $c$, and a single relation 
\[
s^2+c^2=1
\]
\uncover<4->{there is a standard method to verify whether two (rational) expressions in $s$ and $c$ are equal.}
\item<5-> The method is rather cumbersome for a human and is best suited for computers.  
\end{itemize}
}
\only<handout:2|6-10>{
\begin{itemize}
\item<6-> \alertNoH{6}{Yes.}
\item<7-> For expressions that depend only on $\sin \theta$ and $\cos \theta$,  algebra  tells us when two expressions in those are equal.
\item<8-> Problems depending on $\cos\theta$, $\sin \theta$ alone will always be doable via easy ad-hoc tricks using
\[
\sin^2\theta+\cos^2\theta=1.
\]
\item<9-> \alertNoH{10}{The full method will not be needed in this course.}

\begin{itemize}
\item<9->\only<10->{\color{gray}} The full method: set $s=\sin\theta,c=\cos \theta $.
\item<9->\only<10->{\color{gray}} Check whether the two expressions in $s,c$ are equal under the relation $s^2+c^2=1$. (The method lies outside of present scope).
\end{itemize}
%\only<10->{\color{black}}
\end{itemize}
}
\only<handout:3|11->{
\begin{itemize}
\item<11-> To prove a general trigonometric identity:
\begin{itemize}
\item<11-> Use angle sum/double angle sum formulas to convert all formulas to trig expression depending only on $\sin \theta$, $\cos \theta$.
\item<12-> Use $\sin^2 \theta + \cos^2\theta=1$ to show the two formulas are equal (usage: ad-hoc).
\item<13-> You may need to use trig functions of angles smaller than $\theta$, for example $\sin \left(\frac{\theta}{2}\right), \cos \left(\frac{\theta}{2}\right)$.
\item<14-> A fraction of $\theta$ such that all appearing angles are integer multiples of it will always work.
\end{itemize}
\end{itemize}
}

\vskip 10cm
\end{frame}
