\begin{frame}
\frametitle{Strategy for solving trigonometric equations}
\begin{itemize}
\item Suppose we want to solve an algebraic trigonometric equation. 
\item More precisely, the equation should be an algebraic expressions of the trigonometric functions of a single variable. 
\item Here is a general strategy for solving such a problem:
\begin{itemize}
\item Using trig identities, rewrite in terms of $\sin x  $ and $\cos x$ only. 
\item Suppose $x\in [2n\pi, (2n+1)\pi]$. 
\begin{itemize}
\item Set $\cos x= \sqrt{1-\sin^2x} $ (allowed due to restrictions on $x$).
\item Set $\sin x=u$. Solve the resulting algebraic equation for $u$.
\item For the found solutions for $u$, solve $\sin x=u$.
\item Check whether your solutions satisfy $x\in [2n\pi, (2n+1)\pi]$.
\end{itemize}

\item Suppose  $x\in [(2n-1)\pi, 2n\pi]$. 
\begin{itemize}
\item Set $\cos x= -\sqrt{1-\sin^2x} $ (allowed due to restrictions on $x$).
\item Set $\sin x=u$. Solve the resulting algebraic equation for $u$.
\item For the found solutions for $u$, solve $\sin x=u$.
\item Check whether your solutions satisfy $x\in [(2n-1)\pi, 2n\pi]$.
\end{itemize}

\end{itemize}
\item A similar strategy exists with $\cos x$ instead of $\sin x$.
\item Problems requiring full algorithm may be too hard for Calc exams.
\end{itemize}
\end{frame}