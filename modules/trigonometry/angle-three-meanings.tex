\begin{frame}
\frametitle{Three Meanings of the Term Angle}
\vskip -0.2cm
\begin{columns}
\column{0.22\textwidth}
~\\~\\
~\\~\\
\psset{xunit=2cm, yunit=2cm}
\begin{pspicture}(-0.25,-1)(1.4,1.6)%
\tiny%
\fcBoundingBox{-0.25}{-1}{1.4}{1}%
\pstVerb{20 dict begin /startAngle -30 def /endAngle 30 def /angleLength 1.2 def}%
\parametricplot[arrows=->, linecolor=blue]{startAngle}{endAngle}{t cos 0.4 mul t sin 0.4 mul}%
\only<handout:0|3,4>{%
\parametricplot[arrows=->, linecolor=blue, linewidth=1.5pt]{startAngle}{endAngle}{t cos 0.4 mul t sin 0.4 mul}%
}%
\only<handout:0|4>{%
\parametricplot[arrows=->, linecolor=blue, linewidth=1.5pt]{startAngle}{endAngle 360 add}{t cos 0.2 0.0002 t mul add mul t sin 0.2 0.0002 t mul add mul}%
}%
\only<handout:0|5->{%
\parametricplot[arrows=->, linecolor=blue]{startAngle}{endAngle 360 add}{t cos 0.2 0.0002 t mul add mul t sin 0.2 0.0002 t mul add mul}%
}%
\psline[arrows=->](0,0)(! startAngle cos angleLength mul startAngle sin  angleLength mul)%
\psline[arrows=->](0,0)(! endAngle cos angleLength mul endAngle sin angleLength mul)%
\only<handout:0|2>{\psline[arrows=->, linewidth=2pt, linecolor=red](0,0)(! startAngle cos angleLength mul startAngle sin angleLength mul)}%
\only<handout:0|2>{\psline[arrows=->, linewidth=2pt, linecolor=red](0,0)(! endAngle cos angleLength mul endAngle sin angleLength mul)}%
\pstVerb{end}%
\end{pspicture}

~\\
~\\
~\\
~\\

\column{0.78\textwidth}
\begin{itemize}
\item<1-> The term angle is used to denote three distinct mathematical objects: 
\begin{itemize}
\item<2-> \alertNoH{2}{the (geometric) angle formed by two rays,}
\item<3-> \alertNoH{3}{the angle-measure of such a geometric angle}
\item<4-> \alertNoH{4}{the angle-measure of a rotation.}
\end{itemize}
\item<5-> \alertNoH{5}{All three objects are referred to as ``angle''}: one is expected to use \alertNoH{6}{context} to decide whether ``angle'' means \alertNoH{6}{``angle formed by two rays''}, \alertNoH{6}{``angle measure''} or \alertNoH{6}{``angle-measure of a rotation''}.
\item<7-> Except for a few introductory slides, we take full advantage of this convention.
\end{itemize}


\end{columns}


\vskip 20cm

\end{frame}