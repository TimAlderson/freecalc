\begin{frame}
\begin{example}
\begin{columns}
\column{0.3\textwidth}
\psset{xunit=1.5cm, yunit=1.5cm}
\begin{pspicture}(-0.1,-0.2)(2.1,1.8)
\tiny
\fcBoundingBox{-0.1}{-0.2}{2.1}{1.8}
\fcPerpendicular{[1 3 sqrt ]}{[1 0]}{0.1}
\parametricplot[linecolor=red]{0}{60}{t cos 0.3 mul t sin 0.3 mul}
\parametricplot[linecolor=red]{120}{180}{t cos 0.3 mul 2 add t sin 0.3 mul}
\parametricplot[linecolor=red]{-90}{-120}{t cos 0.3 mul 1 add t sin 0.3 mul 3 sqrt add}
\parametricplot[linecolor=red]{-90}{-60}{t cos 0.3 mul 1 add t sin 0.3 mul 3 sqrt add}
\psline(0,0)(1, 0)(! 1 3 sqrt )(0,0)
\psline[linestyle=dashed](1, 0)(2,0)(! 1 3 sqrt )
\rput[t](0.5, -0.1){$1$}
\rput[t](1.5, -0.1){$1$}
\rput[br](0.4, 0.8){$2$}
\rput[bl](1.6, 0.8){$2$}
\rput[l](0.3, 0.2){$60^\circ$}
\rput[r](1.7, 0.2){$60^\circ$}
\rput[tr](1, 1.2){$30^\circ=\gamma$}
\rput[tl](1, 1.2){$~30^\circ$}
\rput[l](! 1 3 sqrt 2 div){$\sqrt{3}$}

\end{pspicture}
\column{0.7\textwidth}
Find the values of $\sin 60^{\circ}, \cos 60^{\circ}, \tan 60^{\circ}$, $\sin 30^{\circ}, \cos30^{\circ}, \tan30^{\circ}$.

\uncover<2->{
Realize the $60^\circ$ angle in right triangle with adjacent side of length $1$. Let $\gamma$ be the indicated angle. The angles in triangle sum to $180^\circ$:

\hfil \hfil$
\begin{array}{rcl}
60^\circ+90^\circ +\gamma &=& 180^\circ\\
\gamma&=&30^\circ.
\end{array}
$
}
\end{columns}
\uncover<2->{Construct a triangle congruent to the original right angle triangle as shown. The larger triangle has three equal angles ($=60^\circ$) and so is equilateral (sides are of equal length). Thus the side of the large triangle equals $1+1=2$. Therefore the large leg in the right angle triangle has length $\sqrt{2^2-1^2}=\sqrt{3}$. Finally we can compute:

\hfil \hfil $ \displaystyle \sin 30^\circ = \frac{1}{2}\qquad \cos 30^\circ = \frac{\sqrt{3}}{2}\qquad \tan 30^\circ= \frac{\frac{1}{2}}{\frac{\sqrt{3}}{2}}=\frac{\sqrt{3}}{3}.
$

\hfil \hfil $ \displaystyle \sin 60^\circ = \frac{1}{2}\qquad \cos 60^\circ = \frac{\sqrt{3}}{2}\qquad \tan 60^\circ  = \frac{\frac{\sqrt{3}}{2}}{\frac{1}{2}}=\sqrt{3}.
$

}
\end{example}

\end{frame}

\begin{frame}
\begin{observation}
\begin{itemize}
\item If the hypotenuse of a right angle triangle is twice larger than one of the sides, then the angle opposite to that side is $30^\circ$. 
\item Conversely, in a right angle triangle with angle $30^\circ$, the hypotenuse is twice longer than the shorter of the two legs.
\end{itemize}
\end{observation}

\hfil \hfil\psset{xunit=4cm, yunit=4cm}
\begin{pspicture}(0,-0.1)(0.6,1.1)
\tiny
\fcPerpendicular{[0.5 3 sqrt 2 div]}{[0.5 0]}{0.1}
\parametricplot[linecolor=red]{0}{60}{t cos 0.1 mul t sin 0.1 mul}
\parametricplot[linecolor=red]{-90}{-120}{t cos 0.1 mul 0.5 add t sin 0.1 mul 3 sqrt 2 div add}
\psline(0,0)(0.5, 0)(! 0.5 3 sqrt 2 div)(0,0)
\rput[t](0.25, -0.05){$x$}
\rput[br](0.2, 0.4){$2x$}
\rput(0.14, 0.07){$60^\circ$}
\rput[tr](0.5, 0.68){$30^\circ$}
\end{pspicture}
\end{frame}