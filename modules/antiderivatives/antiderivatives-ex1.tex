% begin module antiderivatives-ex1
\begin{frame}
\begin{example}[Example 1, p. 318]
Find the most general antiderivative of each of the following functions.
\begin{columns}[c]
\column{.5\textwidth}
\[
f(x) = \sin x%
\]
\begin{itemize}
\item<2->  If \alert<handout:0| 3-4>{$F(x) = \uncover<4->{-\cos x}$}, then $F'(x) = \sin x$.
\item<5->  Therefore the most general antiderivative is $G(x) = -\cos x + C$.
\end{itemize}
\column{.5\textwidth}
\[
f(x) = x^n, n\geq 0%
\]
\begin{itemize}
\item<6->  If \alert<handout:0| 7-8>{$F(x) = \uncover<8->{\frac{x^{n+1}}{n+1}}$}, then $F'(x) = x^n$.
\item<9->  Therefore the most general antiderivative is $G(x) = \frac{x^{n+1}}{n+1} + C$.
\end{itemize}
\end{columns}
\end{example}
\end{frame}


\begin{frame}
\begin{example}[Example 1b, p. 318]
Find the most general antiderivative of $\displaystyle f(x) = \frac{1}{x}$.%
\begin{itemize}
\item<2->  Example 6, p. 223: If \alert<handout:0| 3-4>{$F(x) = \uncover<4->{\ln |x|}$}, then $\displaystyle F'(x) = \frac{1}{x}$.
\item<5->  This is valid for any interval on which $\displaystyle \frac{1}{x}$ is defined.
\item<6-| alert@6-7>  $\displaystyle \frac{1}{x}$ is defined \uncover<7->{everywhere except at $0$.}
\item<8->  The most general answer needs two different constants, one for $(-\infty , 0)$ and one for $(0, \infty )$.
\end{itemize}
\uncover<9->{%
\[
G(x) = \left\{ \begin{array}{lll}
\ln |x| + C_1 & \textrm{if} & x > 0\\
\ln |x| + C_2 & \textrm{if} & x < 0
\end{array}\right.
\]
}%
\end{example}
\end{frame}
% end module antiderivatives-ex1
