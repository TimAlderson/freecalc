% begin module antiderivatives-ex-initial-value
\begin{frame}
\begin{example}
Find $f$ if $f'(x) = \frac{1}{x\sqrt{x}}$ for $x > 0$, and $f(1) = 1$.
\begin{columns}[c]
\column{.4\textwidth}
\begin{eqnarray*}
\uncover<2->{%
f'(x)%
}%
& \uncover<2->{ = } &%
\uncover<2->{%
\frac{1}{x\sqrt{x}} = x^{-3/2}%
}\\%
\uncover<3->{%
\alert<handout:0| 3-4>{f(x)}%
}%
& \uncover<3->{\alert<handout:0| 3-4>{ = }} &%
\uncover<4->{%
\alert<handout:0| 4>{\frac{x^{-1/2}}{-\frac{1}{2}}} \uncover<5->{\alert<handout:0| 5>{+C}}%
}\\%
& \uncover<6->{\alert<handout:0| 3-4>{ = }} &%
\uncover<6->{%
-\frac{2}{\sqrt{x}} +C%
}%
\end{eqnarray*}
\column{.6\textwidth}
\uncover<7->{To find $C$, use the fact that $f(1) = 1$.}%
\begin{eqnarray*}
\uncover<7->{%
f(1)%
}%
& \uncover<7->{ = } &%
\uncover<7->{%
1%
}\\%
\uncover<8->{%
-\frac{2}{\sqrt{1}} +C%
}%
& \uncover<8->{ = } &%
\uncover<8->{%
1%
}\\%
\uncover<9->{%
C%
}%
& \uncover<9->{ = } &%
\uncover<9->{%
3%
}%
\end{eqnarray*}
\end{columns}

\uncover<10->{Therefore
\[
f(x) = -\frac{2}{\sqrt{x}}+3.
\]
}%
\end{example}
\end{frame}
% end module antiderivatives-ex-initial-value
