% begin module limits-piecewise-ex9
\begin{frame}
\begin{example}%[Example 9, p. 82]
If
\[
f(x) = \left\{ \begin{array}{lcr}
\alert<handout:0| 3>{\sqrt{x-4}} & \alert<handout:0| 3>{\text{ if }} & \alert<handout:0| 3>{x > 4} \\
\alert<handout:0| 6>{8 - 2x} & \alert<handout:0| 6>{\text{ if }} & \alert<handout:0| 6>{x < 4} 
\end{array}\right.
\]
determine whether $\lim_{x\rightarrow 4} f(x)$ exists.
\[
\uncover<2->{\lim_{x\rightarrow 4^{\alert<handout:0| 3>{+}}}f(x)} \uncover<3->{ = \lim_{x\rightarrow 4^{\alert<handout:0| 3>{+}}}\alert<handout:0| 3>{\sqrt{x-4}}} \uncover<4->{ = \sqrt{4 - 4} = 0}
\]
\[
\uncover<5->{\lim_{x\rightarrow 4^{\alert<handout:0| 6>{-}}}f(x)} \uncover<6->{ = \lim_{x\rightarrow 4^{\alert<handout:0| 6>{-}}}\alert<handout:0| 6>{(8 - 2x)}} \uncover<7->{ = 8 - 2\cdot 4 = 0}
\]
\uncover<8->{
The left and right hand limits are equal.  Therefore the limit exists and
\[
\lim_{x\rightarrow 4} f(x) = 0.
\]
}
\end{example}
\end{frame}
% end module limits-piecewise-ex9
