% begin module limit-def
\begin{frame}
\frametitle{The Limit of a Function}
\begin{definition}[The Limit of a Function]
We write
\[
\lim_{x\rightarrow a} f(x) = L
\]
and say ``the limit of $f(x)$, as $x$ approaches $a$, equals $L$,'' if we can make the values of $f(x)$ \alert<3>{arbitrarily close to $L$} \alert<4>{by taking $x$ to be sufficiently close to $a$} (on either side of $a$) \alert<5>{but not equal to $a$}.

\uncover<2->{Equivalent formulation.  $\lim_{x\to a} f(x)=L$  if \alert<3>{   for every $\varepsilon>0$}, \alert<4>{there exists $\delta>0$} such that \alert<3>{$|f(x)-L|<\varepsilon$} \alert<4>{for all $x$ with} $\alert<5>{0<} \alert<4-5>{|x-a|} \alert<4>{<\delta}$. }
\end{definition}
\begin{center}
\begin{pspicture}(-0.5,-0.5)(3,2.5)
\tiny
\psaxes[labels=none, ticks=none]{<->}(0,0)(-1,-0.5)(3,2.3)
\psLabels{3}{2.3}
\rput[l](2.8,1.7){\tiny $y=f(x)$}

\psplot[linecolor=red]{-1}{3}{x x mul 4 div}

\psXTickWithLabel{2}{$a$}
%\rput[b](2,0.1){\tiny $a$}
\psYTickWithLabel{1}{$L$}
%\rput[l](0.1,1){\tiny $L$}
\pscircle*(2,1){0.05}
%\pscircle*(2,0){0.05}
%\pscircle*(0,1){0.05}
\uncover<5>{
\psHollowDot{2}{1}
\psFullDot{2}{1.7}
}
\uncover<3->{
\psline[linestyle=dotted](-0.99,0.8)(3, 0.8)
\psline[linestyle=dotted](-0.99,1.2)(3, 1.2)
\psline[linecolor=blue]{<->}(-0.95,0.8)(-0.95,1.2)
\rput[r](-0.6, 1){ $2\varepsilon$}
}

\uncover<4->{
\psline[linestyle=dotted](1.84,-0.5)(1.84, 2.3)
\psline[linestyle=dotted](2.16,-0.5)(2.16, 2.3)
\psline[linecolor=blue]{<->}(1.84,2.3)(2.16,2.3)
\rput[t](2, 2.2){ $2\delta$}
}

\end{pspicture}
\end{center}
\end{frame}
% end module limit-def
