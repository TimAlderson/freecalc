% begin module logarithm-properties-ex2
\begin{frame}
\begin{example}%[Example 2, p. 406]
Use the properties of logarithms to evaluate the following:
\begin{columns}[t]
\column{.5\textwidth}
\begin{align*}
& \invisible{=} \log_4 2 + \log_4 32 \\
&\uncover<2->{=}  \uncover<2-| handout:0>{\log_4 (2\cdot 32 )} \\
&\uncover<3->{=}  \uncover<3-| handout:0>{\log_4 (64)} \\
&\uncover<4->{=}  \uncover<4-| handout:0>{3} \\
& \uncover<4-| handout:0>{\text{(because $4^3 = 64$.)}}
\end{align*}
\column{.5\textwidth}
\begin{align*}
& \invisible{=} \log_2 80 - \log_2 5 \\
&\uncover<5->{=}  \uncover<5-| handout:0>{\log_2 \left( \frac{80}{5}\right) } \\
&\uncover<6->{=}  \uncover<6-| handout:0>{\log_2 (16)} \\
&\uncover<7->{=}  \uncover<7-| handout:0>{4} \\
& \uncover<7-| handout:0>{\text{(because $2^4 = 16$.)}}
\end{align*}
\end{columns}
\end{example}
\end{frame}
% end module logarithm-properties-ex2
