% begin module logarithm-properties-ex2
\begin{frame}
Use the properties of logarithms to evaluate the following.
\begin{example}
\[
\begin{array}{rcl}
\log_{\alertNoH{2}{4}} \alertNoH{3}{2} + \log_{\alertNoH{2}{4}} \alertNoH{3}{32}&\uncover<2->{=}&\displaystyle  \uncover<2->{ \worksheet{ \log_{ \alertNoH{2}{4}} (\alertNoH{4}{ \alertNoH{3}{2}\cdot \alertNoH{3}{32}} )}} \\
&\uncover<4->{=}&\displaystyle  \uncover<4->{\worksheet{\alertNoH{5,6}{ \log_{ \alertNoH{7}{4}} (\alertNoH{4,9}{64})}}} \\
&\uncover<5->{\alertNoH{5,6}{=}}&\displaystyle   \fcAnswer{6}{\alertNoH{8}{3}} \\
&&\displaystyle  \uncover<6->{\worksheet{\text{(because ${\alertNoH{7}{4}}^{\alertNoH{8}{3}} = \alertNoH{9}{64}$.)}}}
\end{array}
\]
\end{example}


\begin{example}
\[
\begin{array}{rcl}
\log_{\alertNoH{10}{2}} 80 - \log_{ \alertNoH{10}{2} } 5 &\uncover<10->{=}& \displaystyle  \uncover<10->{\worksheet{ \log_{\alertNoH{10}{2}} \left(\alertNoH{11}{ \frac{80}{5}} \right) }} \\
&\uncover<11->{=}&\displaystyle   \uncover<11->{\worksheet{\alertNoH{12,13 }{ \log_{\alertNoH{14}{2}} (\alertNoH{11,16}{16})}}} \\
&\uncover<12->{\alertNoH{12,13}{=}}&\displaystyle   \fcAnswer{13}{\alertNoH{15}{4}} \\
&& \uncover<13->{\worksheet{\text{(because $\alertNoH{14}{2}^{\alertNoH{15}{4}} = \alertNoH{16}{16}$.)}}}
\end{array}
\]

\end{example}


\end{frame}
% end module logarithm-properties-ex2
