\begin{frame}
\begin{example}
Compute as a rational number, without using calculator. 
$
\renewcommand{\arraystretch}{2}
\begin{array}{@{}r@{}c@{}l@{}l|l}
\displaystyle \log_{7}{(24)}+\log_{\frac{1}{7}}{(3)}-\log_{49}{(64)}&=&\displaystyle  \log_{7}{(24)}+ \frac{\log_{7}{(3)}}{ \log_{7}{\left(\frac{1}{7}\right)}}- \frac{\log_{7}{ (64)}}{\log_{7}{(49)}} \\
&=&\displaystyle \log_{7}{(24)}+ \frac{\log_{7}{(3)}}{-1} -\frac{\log_{7}{(64)}}{2}\\
&=&\displaystyle \log_{7}{(24)}-\log_{7}{(3)}-\frac{1}{2}\log_{7}{(64)}\\
\renewcommand{\arraystretch}{1.2} \left[ \begin{array}{@{}l} \log_ax-\log_ay=\log_a\left( \frac{x}{y}\right)  \\ \log_ax^r=r\log_ax \end{array}\right] {~~~~~~} &=&\displaystyle \log_{7}{\left(\frac{24}{3}\right)}- \log_{7}{ \left(64^{ \frac{ 1}{2}}\right) }\\
&=&\displaystyle \log_{7}{\left(8\right)}-\log_{7}\left(\sqrt{64} \right)\\
&=&\displaystyle \log_78-\log_78 =0\\
~[\text{alternatively:}] {~~~~~~} &=&\displaystyle \log_7\left(\frac{8}{8}\right)=\displaystyle \log_7(1)=0.
\end{array}
$

\end{example}
\end{frame}