% begin module e-limit
\begin{frame}
Recall that $e=1+\frac{1}{1}+\frac{1}{2!}+\frac{1}{3!}+\dots\approx 2.718281828$.
\begin{theorem}[The Number $e$ as a Limit]
For large $n$ we have that:
\[
\begin{array}{rcl}
e &\approx& \displaystyle  \left(1+\frac{1}{n}\right)^n\\
&\approx&\displaystyle (1 + n)^{\frac{1}{n}} \\
e^x&\approx&\displaystyle  \left(1+\frac{x}{n}\right)^n
\end{array}
\]
All approximations become better as $n$ increases.
\end{theorem}
\begin{itemize}
\item The approximation was discovered by Jacob Bernoulli (1655-1705) in order to apply to compound interest rate computations.
\end{itemize}
\end{frame}
% end module e-limit
