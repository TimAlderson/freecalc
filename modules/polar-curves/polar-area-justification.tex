% begin module polar-area-justification
\begin{frame}[t]
\begin{columns}
\frametitle{Area swept by a polar curve: justification}
\column{0.5\textwidth}
\psset{xunit=1cm, yunit=1cm, algebraic=false}
\begin{pspicture}(-2.65,-1)(1.4,2.3)%
\tiny%
\psline[linecolor=red!1](1.4, 2.3)(1.39, 2.3)%
\psline[linecolor=red!1](-2.65, -1)(-2.649, -1)%
\rput[t](0,-0.1){$O$}%
\pstVerb{%
10 dict begin
/thePolarR {t 2 div 1 add} def
/thePolarX {t 57.295779513 mul cos thePolarR mul} def
/thePolarY {t 57.295779513 mul sin thePolarR mul} def
/thetaMax 3.5 def
}%
\uncover<1>{%
\pscustom*[linecolor=cyan]{%
\fcDrawPolar[linecolor=red, plotpoints=1000]{0}{thetaMax}{thePolarR}%
\pstVerb{1 dict begin /t thetaMax def}%
\psline(! thePolarX thePolarY )(0,0)(1,0)%
\pstVerb{end}%
}%
\pstVerb{1 dict begin /t thetaMax def}%
\psline[linecolor=red](! thePolarX thePolarY )(0,0)(1,0)%
\pstVerb{end}%
}%
\rput[t](1.2,-0.1){$x$}%
\uncover<2-15>{%
\rput[b](-0.832294, 1.85){$\alertNoH{4}{P_2}$}%
\rput[bl](0.84, 1.3){$\alertNoH{4}{P_1}$}%
\pstVerb{1 dict begin /t 1 def}%
\fcFullDot{thePolarX}{thePolarY}%
\pstVerb{/t 2 def}%
\fcFullDot{thePolarX}{thePolarY}%
\pstVerb{end}%
}%
\uncover<5-15>{%
\pstVerb{1 dict begin /t 1 def}%
\psline[linecolor=cyan](0,0)(! thePolarX thePolarY)%
\pstVerb{/t 2 def}%
\psline[linecolor=cyan](0,0)(! thePolarX thePolarY)%
\pstVerb{end}%
}%
\uncover<7-15>{%
\fcPolarWedge{1}{2}{thePolarR}%
}%
\uncover<8,9,10>{%
\pstVerb{1 dict begin /t 1 def}%
\psline[linecolor=red, linewidth=2pt](0,0)(! thePolarX thePolarY)%
\pstVerb{/t 2 def}%
\psline[linecolor=red, linewidth=2pt](0,0)(! thePolarX thePolarY)%
\pstVerb{end}%
}%
\uncover<5-15>{%
\rput[bl](0.50, 0.6){$r_1$}%
\rput[l](-0.45, 1.1){$r_2$}%
}%
\uncover<6-15>{%
\fcAngle{1}{2}{0.7}{}%
\fcAngle{0}{2}{0.45}{}%
\fcAngle{0}{1}{0.15}{}%
\rput[bl](0.15, 0.05){$\alertNoH{6}{\theta_1}$}%
\rput[b](0, 0.46){$\alertNoH{6}{\theta_2}$}%
\rput[b](0, 0.73){$ \alertNoH{9,10}{\Delta}$}%
}%
\uncover<handout:2|11-15>{%
\fcPolarWedgeSequence{0}{1}{3}{thePolarR}%
}%
\uncover<handout:2|16>{%
\fcPolarWedgeSequence{0}{0.75}{4}{thePolarR}%
}%
\uncover<handout:2|17>{%
\fcPolarWedgeSequence{0}{0.5}{6}{thePolarR}%
}%
\uncover<handout:2|18>{%
\fcPolarWedgeSequence{0}{0.3}{10}{thePolarR}%
}%
\uncover<handout:2|19>{%
\fcPolarWedgeSequence{0}{0.2}{15}{thePolarR}%
}%
\uncover<handout:2|20->{%
\fcPolarWedgeSequence{0}{0.1}{30}{thePolarR}%
}%
\fcDrawPolar[linecolor=red, plotpoints=1000]{0}{3.5}{thePolarR}%
\psline[arrows=->](0,0)(1.2, 0)%
\pstVerb{end}%
\end{pspicture}

\column{0.5\textwidth}
\uncover<2->{Split $[a,b]$ into $N$ equal segments via points $a=\theta_0 \leq \theta_1 \leq \dots \leq \theta_{N-1} \leq \theta_N=b$.} \uncover<3->{The length of each segment is $\Delta=\frac{b-a}{N}$.} \uncover<4->{Let $r_i=f(\theta_i)$. Then each $\theta_i$ gives a \alertNoH{4}{point $P_i$} with polar coordinates $(\alertNoH{5}{r_i},\alertNoH{6}{\theta_i})$.}
\end{columns}

\only<handout:1|1-12>{ \uncover<7->{The area swept by the curve is approximated by sum of areas of triangles given by connecting the origin with two consecutive vertices. Consider one such triangle, say, $OP_1P_2$.} \uncover<8->{By Euclidean geometry, the area of $\triangle OP_1P_2 $ is $\alertNoH{8,9}{\frac{|OP_1| |OP_2|  \fcAnswer{9}{ \sin \Delta}}{2}} \uncover<10->{=\frac{ r_1 r_2 \sin \Delta}{2}= \frac{ f(\theta_1) f(\theta_2) \sin \Delta}{2}} $.}
}

\uncover<handout:2| 11->{\alertNoH{12,13}{ Therefore the area swept by the curve \only<handout:0|1-14>{is approximated by} \only<handout:2|15->{\alertNoH{15}{equals \alertNoH{16-}{the limit} of}} the sum:}
\[\begin{array}{rcl}
\uncover<15->{ A&=&}\alertNoH{12,13}{\uncover<15->{\lim\limits_{\Delta\to 0}} \sum\limits_{i=0}^{N-1} \frac{f(\theta_i)f(\theta_{i+1}) \alertNoH{14}{ \sin \Delta} }{2}} \uncover<14->{= \uncover<15->{ \lim \limits_{\Delta\to 0}} \alertNoH{14}{ \frac{ \sin\Delta}{ \Delta}} \sum\limits_{i=0}^{N-1} \frac{ f( \theta_i) f(\theta_{i} + \Delta)\alertNoH{14}{\Delta}}{2}}\\
\uncover<21->{\uncover<24>{\alertNoH{24}{\text{{\tiny(can be proved)}}}}  &=&   \alertNoH{22,23}{ \lim \limits_{\Delta\to 0}\frac{ \sin\Delta}{ \Delta  }}  \lim \limits_{\alertNoH{24}{\Delta\to 0} } \sum\limits_{i=0}^{N-1} \frac{ f( \theta_i) f( \alertNoH{24}{ \theta_{i} + \Delta} )\Delta}{2} \uncover<22->{=  \fcAnswer{23}{1}\cdot \lim \limits_{\Delta\to 0} \sum\limits_{i=0}^{N-1} \frac{\alertNoH{25}{ f( \theta_i) f(\alertNoH{24}{ \theta_{i}} )} \Delta}{2}}} \\
\uncover<25->{\uncover<27->{{\tiny\text{\alertNoH{27}{(Riemann sum)}}}} &=& \lim \limits_{\Delta\to 0} \sum \limits_{i=0}^{N-1} \frac{ \alertNoH{25}{ f^2( \theta_i)}  \Delta }{2}}\uncover<26->{=  \fcAnswer{27}{  \int\limits_{a}^b \frac{ f^2(\theta)}{2}\diff \theta} }
\end{array}
\]
}
\end{frame}
%end module polar-area-justification
