% begin module polar-area-intro
\begin{frame}
\frametitle{Areas in Polar Coordinates}
Suppose we have a polar curve $r = f(\theta )$, $a\leq \theta \leq b$.
\begin{definition} We say that the figure obtained as the union of the segments connecting the origin with the points of the curve is the figure \emph{swept} by the curve as  $\theta$ varies from $a$ to $b$.
\end{definition}
%\begin{center}breaks in Ubuntu
\hfil\hfil \psset{xunit=0.6cm, yunit=0.6cm}
\begin{pspicture}(-0.5,-0.5)(3.1,2.4)%
\tiny%
\pstVerb{/theFunction {3 t sub} def}%
\uncover<2->{%
\fcPolarWedge{0}{0.05}{theFunction}%
}%
\uncover<3->{%
\fcPolarWedge{0.05}{0.1}{theFunction}%
}%
\uncover<4->{%
\fcPolarWedge{0.1}{0.15}{theFunction}%
}%
\uncover<5->{%
\fcPolarWedge{0.15}{0.2}{theFunction}%
}%
\uncover<6->{%
\fcPolarWedge{0.2}{0.25}{theFunction}%
}%
\uncover<7->{%
\fcPolarWedge{0.25}{0.3}{theFunction}%
}%
\uncover<8->{%
\fcPolarWedge{0.3}{0.35}{theFunction}%
}%
\uncover<9->{%
\fcPolarWedge{0.35}{0.4}{theFunction}%
}%
\uncover<10->{%
\fcPolarWedge{0.4}{0.45}{theFunction}%
}%
\uncover<11->{%
\fcPolarWedge{0.45}{0.5}{theFunction}%
}%
\uncover<12->{%
\fcPolarWedge{0.5}{0.55}{theFunction}%
}%
\uncover<13->{%
\fcPolarWedgeSequence{0.55}{0.05}{20}{theFunction}%
}%
\fcAxesStandardNoFrame{-0.5}{-0.5}{3.2}{2.2}%
\rput[bl](1.5, 1.7){$r=f(\theta)$}%
\end{pspicture}%
%\end{center}
\uncover<14->{
\begin{theorem}
Suppose no two points on the curve lie on the same ray from the origin. Then the area swept by the curve equals $\displaystyle A = \int_a^b \frac{1}{2}\left(f(\theta )\right)^2\diff \theta$.
\end{theorem}
}
%A geometric explanation of this formula can be found in the textbook. I am preparing a geometric explanation, no need to refer to Stewart.

\end{frame}
% end module polar-area-intro
