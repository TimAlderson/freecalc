% begin module continuity-ex2
\begin{frame}
\begin{example}
Where is this function discontinuous?
\begin{columns}[c]
\column{.4\textwidth}
\[
f(x) = \frac{x^2 - x - 2}{\alert<handout:0 |3>{x - 2}}
\]
\ \uncover<4->{
\psset{xunit=0.8cm, yunit=0.8cm}
\begin{pspicture}(-3, -2)(3,4)
 \psframe*[linecolor=white](-3,-2)(3,4) \psaxes[labels=none]{<->}(0,0)(-3,-2)(3,4)
 \psplot[linecolor=red, plotpoints=1000]{-3}{3}{x 1 add }
 \psHollowDot{2}{3}
\end{pspicture} %
}
\column{.6\textwidth}
\begin{itemize}
\item<2-| alert@2-3>  $f(2)$ \uncover<3->{doesn't exist.}
\item<4->  Discontinuous at 2.
\item<5->  This is called a removable discontinuity because we could remove it by redefining $f$ at the single number 2.
\end{itemize}
\end{columns}
\end{example}
\end{frame}



\begin{frame}
\begin{example}
Where is this function discontinuous?
\begin{columns}[c]
\column{.4\textwidth}
\[
f(x) = \left\{ \begin{array}{lcl}
\frac{1}{\alert<handout:0 |6>{x^2}} & \text{ if } & x \neq 0 \\
\alert<handout:0 |4>{1} & \alert<handout:0 |4>{\text{ if }} & \alert<handout:0 |4>{x = 0} \\
\end{array}\right.
\]

\psset{xunit=0.8cm, yunit=0.8cm}
\begin{pspicture}(-3.1, -0.5)(5.1,3.1) \psframe*[linecolor=white](-3.1,-0.5)(3,5) 
\psaxes[ticks=x, labels=none]{<->}(0,0)(-3,-0.5) (3,5)
\psplot[linecolor=red, plotpoints=1000]{0.447213595}{3}{1 x 2 exp div }
\psplot[linecolor=red, plotpoints=1000]{-3}{-0.447213595}{1 x 2 exp div}
\rput(2,2){$y=\frac{1}{x^2}$}
\psFullDot{0}{1}
\end{pspicture} %
\column{.6\textwidth}
\begin{itemize}
\item<2-| alert@3-4>  $f(0)$ \uncover<4->{exists ($f(0) = 1$).}
\item<2-| alert@5-6>  $\lim_{x\rightarrow 0} f(x)$ \uncover<6->{doesn't exist ($\infty$).}
\item<7->  Discontinuous at 0.
\item<8->  This is called an infinite discontinuity. 
\end{itemize}
\end{columns}
\end{example}
\end{frame}


\begin{frame}
\begin{example}
Where is this function discontinuous?
\begin{columns}[c]
\column{.4\textwidth}
\[
f(x) = \left\{ \begin{array}{lcl}
\frac{x^2 - x - 2}{x-2} & \text{ if } & x \neq 2 \\
\alert<handout:0 |4>{1} & \alert<handout:0 |4>{\text{ if }} & \alert<handout:0 |4>{x = 2} \\
\end{array}\right.
\]
\psset{xunit=0.8cm, yunit=0.8cm}
\begin{pspicture}(-3, -2)(3,4)
\psframe*[linecolor=white](-3,-2)(3,4) \psaxes[labels=none]{<->}(0,0)(-3,-2)(3,4)
\psplot[linecolor=red, plotpoints=1000]{-3}{3}{x 1 add }
\psHollowDot{2}{3}
\psFullDot{2}{1}
\end{pspicture} %
\column{.6\textwidth}
\begin{itemize}
\item<2-| alert@3-4>  $f(2)$ \uncover<4->{is defined ($f(2) = 1$).}
\item<2-| alert@5-6>  $\lim\limits_{x\rightarrow 2} f(x)$ \uncover<6->{exists ($3$).}
\item<7->  $\lim\limits_{x\rightarrow 2}f(x) \neq f(2)$.
\item<8->  Discontinuous at 2.
\item<9->  This is also called a removable discontinuity.
\end{itemize}
\end{columns}
\end{example}
\end{frame}



\begin{frame}
\begin{example}
Where is this function discontinuous?
\begin{columns}[c]
\column{.4\textwidth}
\[
f(x) = \lfloor x\rfloor
\]
\ \psset{xunit=1cm, yunit=1cm}
\begin{pspicture}(-1.5, -1.5)(3.8,3.8)
\psframe*[linecolor=white](-1.5,-1.5)(3.8,3.8) 
\psaxes[labels=x, ticks=x]{<->}(0,0)(-1.5,-1.5)(3.8,3.8)
\psline(-0.1,1)(0.1,1)
\rput[b](-0.25, 1){$1$}
\psline[linecolor=red](-1,-1)(0,-1)
\psFullDot{-1}{-1}
\psHollowDot{0}{-1}

\psline[linecolor=red](0,0)(1,0)
\psFullDot{0}{0}
\psHollowDot{1}{0}

\psline[linecolor=red](1,1)(2,1)
\psFullDot{1}{1}
\psHollowDot{2}{1}

\psline[linecolor=red](2,2)(3,2)
\psFullDot{2}{2}
\psHollowDot{3}{2}

\psline[linecolor=red](3,3)(3.8,3)
\psFullDot{3}{3}
%\psHollowDot{4}{3}
\end{pspicture}
\column{.6\textwidth}
\begin{itemize}
\item<2-| alert@3-4>  $f(1)$ \uncover<4->{exists ($f(1) = 1$).}
\item<2-| alert@5-6>  $\lim_{x\rightarrow 1^+} f(x)$ \uncover<6->{$ = 1$.}
\item<2-| alert@7-8>  $\lim_{x\rightarrow 1^-} f(x)$ \uncover<8->{$ = 0$.}
\item<2-| alert@9-10>  $\lim_{x\rightarrow 1} f(x)$ \uncover<10->{doesn't exist.}
\item<11->  Discontinuous at 1.
\item<12->  Discontinuous at every integer $n$.
\item<13->  These are called jump discontinuities because the function ``jumps'' at these numbers (i.e., the left limit doesn't equal the right limit).
\end{itemize}
\end{columns}
\end{example}
\end{frame}
% end module continuity-ex2
