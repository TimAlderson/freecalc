% begin module IVT-ex3
\begin{frame}
\begin{example}
A monk starts climbing a mountain at 6am on Monday and reaches the peak at 6pm.  He sleeps on top of the mountain, then on Tuesday at 6am he starts descending and reaches the bottom at 6pm.  Show that there is a time when he is at the same elevation on both days.  

\begin{itemize}
\item<2->  Let $M$ denote the height of the mountain.  
\item<3->  Let $h(t)$ denote the monk's elevation at time $t$ on Monday.  
\item<3->  Let $k(t)$ denote the monk's elevation at time $t$ on Tuesday.  
\item<4->  Let $f(t) = h(t) - k(t)$.  
\item<5->  $f$ is continuous.  
\item<6->  \alert<6-7>{$f(6\text{am}) = $ \uncover<7->{$0 - M = -M$.}}
\item<8->  \alert<8-9>{$f(6\text{pm}) = $ \uncover<9->{$M - 0 = M$.}}
\item<10->  $f(6\text{am}) < 0 < f(6\text{pm})$.  
\item<11->  Therefore there is a $c$ between 6am and 6pm such that $f(c) = 0$.
\item<12->  Then $h(c) -k(c) = 0$, so $h(c) = k(c)$.  
\end{itemize}

\end{example}
\end{frame}
% end module IVT-ex3
