% begin module hyperbolic-functions-def
% % % % % % % % % % % % % % % % % % % % % % % % % %
\begin{frame} 
The derivatives of the hyperbolic functions are easily computed.\\ For example,
\onslide<2->
\begin{align*}
\frac{d}{dx}\sinh x & = \frac{d}{dx}\left(\frac{e^x-e^{-x}}{2}\right)\\
\onslide<3->{& = \frac{e^x+e^{-x}}{2}\\}
\onslide<4->{& = \cosh x}
\end{align*}
\onslide<5->
The differentiation formulas for the hyperbolic functions are as follows:
 

\begin{definition}[Derivatives of Hyperbolic Functions]
\[
\begin{array}{l@{\extracolsep{2cm}}l}
\frac{d}{dx} (\sinh x)  = \cosh x & \frac{d}{dx}(\csch(x))=-\csch x \coth x\\ [3mm]

\frac{d}{dx}(\cosh(x)) = \sinh x  & \frac{d}{dx}(\sech(x))=-\sech x \tanh x\\ [3mm]

\frac{d}{dx}(\tanh(x)) = \sech^2 x & \frac{d}{dx}(\coth(x))=-\csch^2 x\\
\end{array}
\]
\end{definition}


\end{frame}

% % % % % % % % % % % % % % % % % % % % % % % % % % % % %
\begin{frame} 
Of course, any of these differentiation rules can be combined with the Chain Rule. For instance
\[
\frac{d}{dx} \sinh(\sqrt{x}) = \cosh(\sqrt{x})\cdot \frac{d}{dx}\sqrt{x} 
\]

\[
= \frac{\cosh(\sqrt{x})}{2\sqrt{x}}
\]
\end{frame}

% end module
