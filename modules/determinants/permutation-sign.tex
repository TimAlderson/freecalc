\begin{frame}
\frametitle{Sign of permutation}
\begin{itemize}
\item Given two sequences of numbers, define them to be transpositions of one another if one is obtained from the other with a \alertNoH{1}{single swap of \alertNoH{5}{neighboring} numbers}.
\item<2-> \alertNoH{2}{$(2,\alertNoH{3}{3,4},1) $ and $(2,\alertNoH{3}{4,3},1) $ are} \uncover<3->{ \alertNoH{3}{transpositions} of one another.}

\alertNoH{4}{$(2,3,4,1) $ and $(1,3,4,2) $ are} \uncover<5->{\alertNoH{5}{\textbf{not} transpositions} of one another.}
\item<6-> Write the numbers $\left(\sigma(1),\sigma(2), \dots, \sigma(n)\right)$ in a sequence.
\item<7-> Using transpositions, get from $\left(\sigma(1),\sigma(2), \dots, \sigma(n)\right)$ to the properly ordered sequence $1,2,\dots, n$.
\item<8-> Number of transpositions used varies depending how we do it, but parity (even-ness) of \# of transpositions is always the same.
\item<10-> If $sign(\sigma)=1$,  $\sigma$ is called even, if $sign(\sigma)=-1$, $\sigma$ is called odd.
\end{itemize}
\uncover<9->{
\begin{definition}
If we can get from $\left(\sigma(1),\sigma(2), \dots, \sigma(n)\right)$ to $(1,2,\dots, n)$ with even \# of transpositions, define $sign(\sigma)$ to be $1$, else define $sign(\sigma)$ to be $-1$.
\end{definition}
}
\end{frame}
