% begin module inverse-function-def
\begin{frame}
\frametitle{The Definition of the Inverse of $f$}
\begin{definition}[$f^{-1}$]
Let $f$ be a one-to-one function with domain $A$ and range $B$.  Then the inverse of $f$ is the function $f^{-1}$ that has domain $B$ and range $A$ and is defined by
\[
f^{-1}(y) = x \qquad \Leftrightarrow \qquad f(x) = y 
\]
for all $y$ in $B$.
\end{definition}
\begin{columns}[T]
\column{.5\textwidth}
\uncover<2->{Note:}
\begin{itemize}
\item<3->  Only one-to-one functions have inverses.
\item<4->  $f^{-1}$ reverses the effect of $f$.
\item<5->  domain of $f^{-1} = $ range of $f$.
\item<5->  range of $f^{-1} = $ domain of $f$.
\end{itemize}
\column{.5\textwidth}
\uncover<6->{
\begin{example}[$f(x) = x^3$]
The inverse of $f(x) = x^3$ is $f^{-1}(x) = \sqrt[3]{x}$.  This is because if $y = x^3$, then
\[
f^{-1}(y) = \sqrt[3]{y} = \sqrt[3]{x^3} = x .
\]
\end{example}
}
\end{columns}
\end{frame}
% end module inverse-function-def
