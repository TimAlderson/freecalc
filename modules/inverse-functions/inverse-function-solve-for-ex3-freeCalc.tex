%begin module inverse-function-solve-for-ex3-freeCalc
\begin{frame}
\begin{example}
Find $f^{-1}(x)$ where $f(x)=\frac{x+1}{x-1}$.

\begin{columns}
\column{0.35\textwidth}
\psset{xunit=0.35cm, yunit=0.35cm}
\begin{pspicture}(-5.214286, -5)(7.214286,5)
\tiny
\psframe*[linecolor=white](-5.214286,-5.2)(7.214286,7.714285714)
\fcAxesStandard{-4.714286}{-4.8}{6.714286}{6.714286}
\rput[tl](2.3,6){$y=\frac{x+1}{x-1}$}

%Function formula: \frac{x+1}{x-1}
\psplot[linecolor=\fcColorGraph, plotpoints=1000]{1.350000}{6.714286}{1 x add -1 x add div }
%Function formula: \frac{x+1}{x-1}
\psplot[linecolor=\fcColorGraph, plotpoints=1000]{-4.714286}{0.650000}{1 x add -1 x add div }
\uncover<17->{
\psline[linecolor=\fcColorTangent, linestyle=dashed](-4.7,-4.7)(6.7,6.7)
}
\end{pspicture}

\uncover<11->{Answer: $f^{-1}(x)=\frac{x+1}{x-1}$}\uncover<13->{, \alertNoH{13}{$x\neq 1$}.}

\column{0.65\textwidth}

\uncover<2->{We deal with domains and ranges later:}
$
\begin{array}{rcll|l}
\displaystyle \uncover<2->{y}&\uncover<2->{=}&\displaystyle \uncover<2->{ \frac{x+1}{\alertNoH{3}{ x-1} } } \uncover<3->{&& \alertNoH{3}{ \text{mult. by } (x-1)}} \\
\displaystyle  \uncover<3->{\alertNoH{4,6}{y}\alertNoH{3}{ (\alertNoH{4}{ x} \alertNoH{6}{-1})}} &\uncover<3->{=} & \displaystyle  \uncover<3->{ \alertNoH{5}{x} +\alertNoH{7}{1}}\\
\displaystyle \uncover<4->{ \alertNoH{4,5}{x} \alertNoH{8}{(\alertNoH{4}{y}\alertNoH{5}{-1})}} &\uncover<4->{=} &\displaystyle \uncover<4->{ \alertNoH{6}{y}+\alertNoH{7}{1}}  \uncover<8->{ && \text{div. by } \alertNoH{8,12}{ (y-1)}} \\
\displaystyle \uncover<9->{f^{-1}(\alertNoH{10}{ y} )= }\uncover<8->{x}&\uncover<8->{=}&\displaystyle \uncover<8->{ \frac{\alertNoH{10}{ y}+1}{\alertNoH{8}{\alertNoH{10}{y} -1}} } \uncover<10->{ &&\alertNoH{10}{\text{relabel }x, y}} \\
\displaystyle \uncover<10->{ f^{-1}(\alertNoH{10}{ x} )&=&\displaystyle \frac{\alertNoH{10}{ x }+1}{\alertNoH{10}{ x}-1}}
\end{array}
$
\uncover<12->{We divided by \alertNoH{12}{$y-1$} so $y\neq 1$.} \uncover<13->{Therefore the domain of $f^{-1}$ is all real numbers except $1$.}

\medskip

\uncover<14->{ \alertNoH{14}{Can a non-identity function be its own inverse?}} \uncover<15->{\alertNoH{15}{ Yes, $f$ is.} }

\uncover<16->{ \alertNoH{16}{What does it mean for $f$ to be its own inverse?}}\uncover<17->{\alertNoH{17}{ Graph of $f$ is symmetric across $y=x$. }}

\end{columns}
\end{example}
\end{frame}


%end module inverse-function-solve-for-ex3-freeCalc
