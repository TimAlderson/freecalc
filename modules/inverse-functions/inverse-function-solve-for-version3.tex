% begin module inverse-function-solve-for
\begin{frame}
\frametitle{How to Find the Inverse of a One-to-one Function}
\begin{enumerate}
\item<1-| alert@3>  Write $y = f(x)$.
\item<1-| alert@4-5>  Solve this equation for $x$ in terms of $y$ (if possible).
\end{enumerate}
\uncover<2->{
\begin{example}%[Example 4, p. 388]
If $f(x) = x^3 + 2$, find a formula for $f^{-1}(y)$.
\begin{align*}
\uncover<3->{y} & \uncover<3->{=}  \uncover<3->{x^3 + 2}\\
\uncover<4->{x^3} & \uncover<4->{=}  \uncover<4->{y - 2}\\
\uncover<5->{\alert<handout:0| handout:0| 6>{x}} & \uncover<5->{=}  \uncover<5->{\alert<handout:0| 6>{\sqrt[3]{y - 2}}}
\end{align*}
\uncover<6->{
Therefore \alert<handout:0| 6,8>{$x=f^{-1}(y) = \sqrt[3]{y - 2}$}.
}
\uncover<7->{
Usually we relabel $x$ and $y$ and write \only<7>{ $f^{-1}(x)=\sqrt[3]{x-2}$.} \only<8-| handout:0>{\color{gray!50} $f^{-1}(x)=\sqrt[3]{x-2}$.  }\color{black}
}
\uncover<8->{\alert<handout:0| 8>{
Unless asked for $ f^{-1}(x) $, do not relabel anything.
}}
\end{example}
}
\uncover<9->{\alert<handout:0| 8>{
Shoes-socks
}}
\end{frame}
% end module inverse-function-solve-for
