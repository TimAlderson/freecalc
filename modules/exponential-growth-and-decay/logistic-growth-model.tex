\begin{frame}
\frametitle{Logistic growth model}
\begin{itemize}
\item The Logistic growth model is the simplest mathematical model which satisfies the following assumptions.
\begin{itemize}
\item<2-> We assume near exponential growth for small populations (relative to the environment's resources).
\item<3-> We assume negative growth when the population exceeds a threshold number called the environment carrying capacity.
\end{itemize}
\item<6-> Although it's one of the simplest mathematical models, the logistic growth model is  used in practice. 
\end{itemize}

\begin{center}
\psset{xunit=0.6cm, yunit=0.6cm}
\begin{pspicture}(-1,-1)(1,1)%
\tiny %
\pstVerb{
20 dict begin
/theConstA 3 def
/theConstB 20 def
/theConstC 2.8 def
}%
\fcAxesStandard{-1.2}{-1.2}{4}{4}%
\psline[linecolor=blue, linestyle=dashed](! -1.2 theConstC)(! 4 theConstC)%
\uncover<3->{\rput[bl](! -1.2 theConstC 0.1 add){$c$=Carrying capacity}}%
\psplot[linecolor=\fcColorGraph]{-1}{4}{theConstC 2.718281828 x theConstA -1 mul mul exp theConstB mul 1 add  div}%
\rput[tl](0.05,-0.1){\begin{tabular}{l}\text{Logistic curve}\\ $ \begin{array}{l} f(x)=\frac{c}{1+be^{-ax}} \end{array}$ \end{tabular}}
\uncover<4->{%
\fcFullDot{1 theConstB div ln theConstA -1 mul div}{theConstC 2 div}%
\fcLengthIndicatorTwo[l]{1 theConstB div ln theConstA -1 mul div}{0}{1 theConstB div ln theConstA -1 mul div}{theConstC 2 div}{ \begin{tabular}{l}Increasing \\ rate of \\ growth\end{tabular}}%
}%
\uncover<5->{%
\fcLengthIndicatorTwo[r]{1 theConstB div ln theConstA -1 mul div}{theConstC}{1 theConstB div ln theConstA -1 mul div}{theConstC 2 div}{ \begin{tabular}{l}Decreasing \\{rate} of \\ growth\end{tabular}}%
} %
\pstVerb{end}%
\end{pspicture}
\end{center}


\end{frame}