\solution{\ref{problemintsecxdx}. \textbf{Variant I.} 

This variant uses the standard method for solving trigonometric integrals with the substitution $x=\arctan (2t)$.

\noindent 
$\begin{array}{rcll|l}
\displaystyle \int \sec x \diff x&=&\displaystyle \int \sec (2\arctan t) \diff (2\arctan t)&& \text{Set } x= 2\arctan t\\
&=&\displaystyle \int \frac{1}{\cos (2\arctan t)} \frac{2}{1+t^2}\diff t&& \text{Use \refBad{\ref{eqsin2zViaTanzcos2zViaTanz}}{}{\eqref{\label{eqsin2zViaTanzcos2zViaTanz}}:}} \cos (2 z)= \frac{1-\tan^2 z }{1+\tan^2 z}\\
&=&\displaystyle \int \frac{1}{\frac{1 -t^2}{1 +t^2}} \frac{ 2 }{ 1+t^2}\diff t\\
&=&\displaystyle \int \frac{2}{1-t^2}\diff t&&\text{part. fractions}\\
&=&\displaystyle \int \left(\frac{1}{1-t}+ \frac{1}{ 1+t} \right)\diff t\\
&=&\displaystyle  -\ln |1-t|+\ln |1+t|+C \\
&=&\displaystyle \ln \left|\frac{1+t}{1-t}\right| &&\text{Subst. } t=\tan \left(\frac{x}{2}\right) \\
&=&\displaystyle \ln \left|\frac{1+\tan \left( \frac{ x}{2} \right)}{ 1-\tan \left(\frac{x}{2}\right)} \right|+C&& \text{Last step: see below}\\
&=&\displaystyle \ln |\sec x +\tan x |+C\quad .
\end{array}
$

The expression $\ln \left| \frac{1+\tan \left(\frac{x}{2} \right) } {1 -\tan \left(\frac{x}{2}\right)} \right|$ presents a perfectly good answer, which would certainly would qualify for a correct test answer. However, as shown above, it can be rewritten into the shorter form $\ln |\sec x +\tan x |$. Below we quickly prove that $\displaystyle \frac{1+\tan \left(\frac{x}{2} \right) } {1 -\tan \left(\frac{x}{2}\right)}$ equals $\sec x +\tan x$.

$\begin{array}{@{}r@{}c@{}l@{}l@{}|l}
\displaystyle \sec x+\tan x &=&\displaystyle  \frac{1+\sin x}{\cos x}&&\begin{array}{@{}r@{}c@{}l}\text{Use:}\\
\sin x &=& 2\sin \left(\frac{x}{2}\right)\\ \cos x&=& \cos^2 \left(\frac{x}{2}\right) -\sin^2 \left(\frac{x}{2}\right)\\ 1&=&\cos^2 \left(\frac{x}{2}\right)+ \sin^2 \left(\frac{x}{2}\right) \end{array}\\
&=&\displaystyle  \frac{\cos^2\left(\frac{x}{2}\right)+ \sin^2\left(\frac{x}{2}\right)+2\sin \left(\frac{x}{2}\right) \cos \left(\frac{x}{2}\right)}{\cos^2 \left(\frac{x}{2}\right) -\sin^2\left(\frac{x}{2}\right)}\\
&=&\displaystyle \frac{\left(\sin \left(\frac{x}{2}\right)+ \cos\left( \frac{ x}{ 2}\right) \right)^{2}}{ {\left(\cos \left( \frac{ x}{ 2} \right)-\sin \left( \frac{x }{2}\right)\right)} \left(\cos  \left(\frac{x}{2}\right)+\sin \left(\frac{x}{2}\right)\right)}\\
&=&\displaystyle \frac{\left(\sin \left(\frac{x}{2}\right)+ \cos\left( \frac{ x}{ 2}\right) \right)\frac{1}{\cos  \left(\frac{x}{2}\right) } }{ \left(\cos  \left(\frac{x}{2}\right)- \sin \left( \frac{ x}{2} \right)\right)\frac{1}{\cos\left(\frac{x}{2}\right)}}\\
&=&\displaystyle \frac{1+\tan \left(\frac{x}{2} \right)}{1 -\tan\left( \frac{ x}{2} \right)}
\end{array}
$

\ref{problemintsecxdx}. \textbf{Variant II.} This variant is based on the following observation. For an odd number $m>0 $, we studied a quick technique for integrating $\int \sin^n x \cos^m x \diff x $: namely, use the transformation $\cos x \diff x = \diff (\sin x)$ and change variables $u=\sin x $. This trick relies heavily on the fact that $m$ is odd (as we need to express the remaining even power of $\cos x$ via $\sin x$). However, the positivity of $m$ is not essential: by multiplying top and bottom by $\cos x$ we can make this technique work also for odd negative values of $m$. We illustrate the technique in the solution below.

$\begin{array}{rcll|l}
\displaystyle \int \sec x \diff x&=&\displaystyle \int \frac{1}{\cos x}\diff x\\
&=&\displaystyle \int \frac{\cos x}{\cos^2x}\diff x\\
&=&\displaystyle \int \frac{\diff (\sin x)}{1-\sin^2x}&&\text{Set }u=\sin x\\
&=&\displaystyle \int \frac{\diff u}{1-u^2}\\
&=&\displaystyle \int \frac{\diff u}{(1+u)(1-u)}&&\text{part. fractions}\\
&=&\displaystyle \int \left(\frac{\frac{1}{2}}{1+u} +\frac{ \frac{1}{ 2}}{1 -u} \right)\diff u \\
&=&\displaystyle \frac{1}{2}\left(\ln |1+u|-\ln|1-u| \right)+C\\
&=&\displaystyle \frac{1}{2}\ln \left|\frac{1+u}{1-u} \right|+C &&\text{Subst. back } u=\sin x \\
&=&\displaystyle \frac{1}{2} \ln \left|\frac{1+\sin x}{1-\sin x}\right|+C&&\text{Last step: see below}\\
&=&\displaystyle \ln|\sec x+\tan x|+C\quad .
\end{array}
$

The expression $\displaystyle \frac{1}{2} \ln \left|\frac{1+\sin x}{1-\sin x}\right|+C$ gives a perfectly good answer (which may be the preferred answer depending on the textbook). Let us show however that $\displaystyle \frac{1}{2} \ln \left|\frac{1+\sin x}{1-\sin x}\right|$ equals $\displaystyle \ln |\sec x+\tan x|$, the answer given in the other variants.

$\begin{array}{rcll|l}
\displaystyle \frac{1}{2}  \ln \left|\frac{1+\sin x}{1-\sin x}\right|&=&\displaystyle  \frac{1}{2} \ln \left|\frac{1+\sin x}{1-\sin x}\right| &&\text{Mult. \& div by } 1+\sin x\\
&=&\displaystyle \frac{1}{2}\ln \left|\frac{(1+\sin x)^2}{(1-\sin x)(1+\sin x)}\right|\\
&=&\displaystyle \frac{1}{2}\ln \left|\frac{(1+\sin x)^2}{\cos^2x}\right|&& \text{use }\frac{1}{2}\ln |a|= \ln |a|^{\frac{1}{2}}\\
&=&\displaystyle \ln \sqrt{\left|\frac{(1+\sin x)^2}{\cos^2x}\right|}\\
&=&\displaystyle \ln \left|\frac{1+\sin x}{\cos x}\right|\\
&=&\displaystyle \ln |\sec x +\tan x|\quad .
\end{array}
$

\ref{problemintsecxdx}. \textbf{Variant III. } This variant present a quick solution by multiplying and dividing our integrand by the multiplier $\sec x+\tan x$. Of course, the idea of using that multiplier comes from knowing the answer to the problem in advance (which can be obtained, for example, by using the preceding solution variants).

$\begin{array}{rcll|l}
\displaystyle \int \sec x \diff x&=&\displaystyle \int \sec x \frac{\sec x +\tan x }{\sec x+\tan x }\diff x  \\
&=& \displaystyle \int \frac{\sec^2 x+\sec x\tan x }{\sec x +\tan x }\diff x&&
\begin{array}{rcl}
\diff (\tan x )&=&\sec^2 x\diff x\\
\diff (\sec x)&=&\sec x\tan x \diff x
\end{array}\\
&=&\displaystyle  \int \frac{\diff (\sec x+\tan x) }{\sec x +\tan x} &&\text{Set }u=\sec x +\tan x\\
&=&\displaystyle \int \frac{\diff u}{u}\\
&=&\displaystyle \ln |u|+C\\
&=&\displaystyle \ln |\sec x +\tan x|+C\quad .
\end{array}
$
}


\solution{\ref{problemintsec^3xdx} This problem can be solved with the general method by setting $x=2\Arctan t$. However, there are shorter ways to solve the integral, as we show below. 

\textbf{Variant I. }

\noindent $
\begin{array}{rcll|l}
\displaystyle \int \sec^3 x \diff x&=&\displaystyle \int  \frac{1}{\cos^3 x} \diff x \\
&=&\displaystyle \int  \frac{\cos x}{\cos^4 x} \diff x  && \text{use }\diff (\sin x)=\cos x \diff x\\
&=&\displaystyle \int  \frac{1}{\cos^4 x} \diff (\sin x)&& \text{use }\cos^2x=1-\sin^2x \\
&=&\displaystyle \int  \frac{1}{\left(1-\sin^2x\right)^2} \diff (\sin x) &&\text{Set }\sin x=u\\
&=&\displaystyle \int  \frac{1}{\left(1-u^2\right)^2} \diff u&&\text{split in part. frac.}\\
&=&\displaystyle \int\left(\frac{\frac{1}{4}}{u +1}+\frac{\frac{1}{4}}{(u +1)^{2}}+\frac{-\frac{1}{4}}{u -1}+\frac{\frac{1}{4}}{(u -1)^{2}}  \right) \diff u\\
&=&\displaystyle \frac{1}{4}\left(\ln |u+1|-\ln|u-1|-\frac{1}{u+1}-\frac{1}{u-1} \right)+C\\
&=&\displaystyle\frac{1}{4}\left(\ln \left|\frac{u+1}{u-1}\right|-\frac{2u}{u^2-1} \right)+C\\
&=&\displaystyle\frac{1}{4}\left(\ln \left|\frac{\sin x+1}{\sin x-1}\right|+\frac{2\sin x}{\cos^2x} \right)+C.\\
\end{array}
$

\textbf{Variant II. } This variant uses the preceding problem to get to a solution as follows.

\noindent $
\begin{array}{rcll|l}
\displaystyle \int \sec^3 x \diff x&=&\displaystyle \int  \sec x \diff (\tan x) &&\text{int. by parts}\\
&=&\displaystyle  \sec x\tan x - \int \tan x \diff (\sec x)\\
&=&\displaystyle \sec x \tan x - \int \sec x \tan^2 x\diff x &&\tan^2 x =\sec^2x-1\\
&=&\displaystyle \sec x \tan x - \int\sec x (\sec^2x-1)\diff x\\
&=&\displaystyle \sec x \tan x - \int \sec^3 x \diff x+\int \sec x \diff x &&\text{Use Problem }\ref{problemintsecxdx}\\
&=&\displaystyle \sec x \tan x -\int \sec^3x \diff x+\ln |\sec x +\tan x|&& \begin{array}{l} + \int \sec^3x \diff x \\ 
\text{ to both sides}\end{array} \\
\displaystyle 2\int \sec^3x \diff x&=&\displaystyle \left(\sec x\tan x +\ln |\sec x+\tan x| \right) +C\\
\displaystyle \int \sec^3x \diff x&=&\displaystyle \frac{1}{2}\left(\sec x\tan x +\ln |\sec x+\tan x| \right) +K\quad .
\end{array}
$
}