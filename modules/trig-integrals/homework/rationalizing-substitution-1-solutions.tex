\solution{\ref{problemInt1/(3+cos x)dx}
This integral is of none of the forms that can be integrated quickly. Therefore we have to use the standard rationalizing substitution $x=2\Arctan t$, $t=\tan \left(\frac{x}{2}\right)$. We recall that from the double angle formulas it follows that 
\[
\cos (2\Arctan t)=\frac{\cos^2(\Arctan t)- \sin^2(2\Arctan t)}{\cos^{2}(\Arctan t)+\sin^2(\Arctan t)}=\frac{1-t^2}{1+t^2}\quad .
\]
Therefore we can solve the integral as follows.
\[
\begin{array}{rcll|l}
\displaystyle \int \frac{1}{3+\cos x}\diff x&=&\displaystyle \int \frac{1}{3+\cos (2\Arctan t)}\diff \left(2\Arctan t\right) &&\text{Substitute } x=2\Arctan t\\
&=& \displaystyle \int \frac{1}{\left(3+\frac{1- t^2}{ 1+t^2}\right)} \frac{ 2}{\left(1+ t^2\right) } \diff t\\
&=&\displaystyle \int\frac{2}{4+2t^2}\diff t\\
&=&\displaystyle \int \frac{1}{2 +t^2}\diff t\\
&=&\displaystyle \frac{\sqrt{2}}{2} \Arctan\left(\frac{\sqrt{2 }}{2} t \right) +C\\
&=&\displaystyle \frac{\sqrt{2}}{2} \Arctan\left(\frac{\sqrt{2 }}{2} \tan\left(\frac{x}{2}\right) \right) +C\quad .
\end{array}
\]
}


\solution{\ref{problemInt1/(2+tan x)dx}
This integral is of none of the forms that can be integrated quickly. Therefore we can solve it using the standard rationalizing substitution $x=2\Arctan t$, $t=\tan \left(\frac{x}{2}\right)$. This results in somewhat long computations and we invite the reader to try it. 

However, as proposed in the hint, the substitution $x=\Arctan t$ works much faster:
\[
\begin{array}{rcll|l}
\displaystyle \int \frac{1}{2+\tan x}\diff x&=&\displaystyle \int \frac{1}{2+\tan (\Arctan t)}\diff \left(\Arctan t\right) &&\text{Substitute } x=\Arctan t\\
&=& \displaystyle \int \frac{1}{\left(2+t\right)} \frac{ 1}{\left(1+ t^2\right) } \diff t && \text{Decompose into partial fractions}\\
&=&\displaystyle \int \left( \frac{\frac{1}{5}}{(t +2)}+\frac{-\frac{t }{5}+\frac{2}{5}}{(t^{2}+1)}\right)\diff t
\\
&=&\displaystyle \frac{1}{5}\ln (t+2)-\frac{1}{10}\ln (t^2+1)+\frac{2}{5}\arctan t+C &&\text{Substitute back } t=\tan x \\
&=&\displaystyle \frac{1}{5}\ln (\tan x+ 2)-\frac{1}{10}\ln (\tan^2 x+1)+\frac{2}{5}x+C\\
&=&\displaystyle \frac{1}{5}\ln (\tan x+ 2)+\frac{1}{5}\ln (\cos x)+\frac{2}{5}x+C\\
&=&\displaystyle \frac{1}{5} \ln \left((\tan x+2)\cos x\right)+\frac{2}{5}x+C\\
&=&\displaystyle \frac{1}{5} \ln \left(\sin x+2\cos x\right)+\frac{2}{5}x+C.\\
\end{array}
\]
}