\begin{frame}
\frametitle{Tangent Plane}
\begin{columns}
\column{0.4\textwidth}
\begin{pspicture}(-2,-2)(2,2)
\fcStartIIIdScene
\fcSurfaceInScene[linecolor=orange, iterationsU=4, iterationsV=4, colorUV=cyan]{-1}{-1}{1}{1}{
3 dict begin
/basePoint [30 cos 2 mul 0 30 sin 1.5 mul ] def
/tangent1 [0 1 0] def
/tangent2 [30 sin -2 mul 0 30 sin 1.5 mul] def
basePoint tangent1 u \fcVectorTimesScalar tangent2 v \fcVectorTimesScalar \fcVectorPlusVector \fcVectorPlusVector
end
}
\fcEllipsoidInScene[linecolor=red, iterationsU=6, iterationsV=7, colorUV={1 0.5 0.5},  colorVU={1 0.5 0.5}]{0 0 0 2 1.5 1.5}
\fcFinishIIIdScene
\fcAxesIIId{3}{3}{3}
\end{pspicture}
\column{0.6\textwidth}
\begin{itemize}
\item Consider a surface $S$ in space and a point $P$ on the surface.
\item<2-> How should we define the notion of ``a plane tangent to $S$ at $P$'' so that it matches our geometric intuition?
\item<3-> Intuitively, it should include all tangents at $P$ to curves passing through $P$ and contained in the surface.
\item<4-> Therefore it should be the geometric plane
\begin{itemize}
\item passing through $P$;
\item parallel to the directions of all tangent vectors of curves passing through $P$ and contained in the surface.
\end{itemize}
\end{itemize}
\end{columns}
\end{frame}