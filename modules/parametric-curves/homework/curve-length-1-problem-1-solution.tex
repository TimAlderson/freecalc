\solution{\ref{problemlengthy=sqrt(x)from1to2}

\textbf{Solution I.} The curve can be rewritten in the form $x=y^2$, $y\in[1, \sqrt{2}]$. 
\[
\begin{array}{rcll|l}
L&=&\displaystyle \int_{1}^{\sqrt{2}}\sqrt{\left(\frac{\diff x}{\diff y}\right)^2+1 }~\diff y  \\
&=&\displaystyle \int_{y=1}^{y=\sqrt{2}}\sqrt{4y^2+1 }~\diff y && \begin{array}{rcl}\text{Substitute }2y&=&u \\ \diff y&=&\frac{1}{2}\diff u\end{array} \\
&=&\displaystyle \int_{u=2}^{u=2\sqrt{2}}\sqrt{u^2+1 }\left(\frac{1}{2}\diff u\right)  \\
&=&\displaystyle \frac{1}{2}\int \sqrt{u^2+1}\diff u&&
\begin{array}{l}\displaystyle
\int \sqrt{u^2+1}\diff u \\
= \displaystyle \frac{1}{2}\left(u\sqrt{u^2+1}+\ln\left(u+\sqrt{u^2+1}\right) \right)+C\\
\text{previously studied}
\end{array}\\
\\
&=&\displaystyle \frac{1}{4} \left[ u\sqrt{u^2+1}+\ln\left(u+\sqrt{u^2+1}\right)\right]_{2}^{2\sqrt{2}}\\
&=&\displaystyle \frac{3}{2}\sqrt{2}+\frac{1}{4} \ln{}\left(2\sqrt{2}+3\right)-\frac{1}{4} \ln{}\left(\sqrt{5}+2\right)-\frac{\sqrt{5}}{2} \\
&\approx&1.083
\end{array}
\]


\textbf{Solution II. } The length of the parametric curve is given by
\[
\begin{array}{rcll|l}
L&=&\displaystyle \int_{1}^{2}\sqrt{1+\left(\frac{\diff y}{\diff x}\right)^2 }\diff x\\
&=&\displaystyle \int_{1}^{2}\sqrt{1+\left(\frac{1}{2\sqrt{x}}\right)^2 }\diff x\\
&=&\displaystyle \int_{x=1}^{x=2} \sqrt{1 +\frac{1}{4x}}\diff x &&\begin{array}{rcl}\text{Substitute }4x&=&u\\\diff x&=&\frac{1}{4}\diff u\\ \end{array}\\
&=&\displaystyle \int_{u=4}^{u=8} \sqrt{1 +\frac{1}{u}}\left(\frac{1}{4}\diff u\right)\\
&=&\displaystyle \frac{1}{4}\int_{4}^{8}\sqrt{\frac{u+1}{u}}\diff u\\
&=&\displaystyle \frac{1}{4}\int_{4}^{8}\sqrt{\frac{u(u+1)}{u^2}}\diff u\\
&=&\displaystyle \frac{1}{4}\int_{4}^{8}\frac{\sqrt{u^2+u }}{u}\diff u\\
&=&\displaystyle \frac{1}{4}\int_{4}^{8}\frac{\sqrt{u^2+u+\frac{1}{4}-\frac{1}{4} }}{u}\diff u\\
&=&\displaystyle \frac{1}{4}\int_{4}^{8}\frac{\sqrt{\left(u+\frac{1}{2}\right)^2-\frac{1}{4} }}{u}\diff u\\
&=&\displaystyle \frac{1}{4}\int_{4}^{8}\frac{ \sqrt{\frac{1}{4}\left( \left(2u+1\right)^2-1\right) }}{u}\diff u\\
&=&\displaystyle \frac{1}{8}\int_{u=4}^{u=8}\frac{ \sqrt{ \left(2u+1\right)^2-1 }}{u}\diff u &&\begin{array}{rcl}\text{Substitute }2u+1&=&z\\ u&=&\frac{z-1}{2}\\\diff u&=&\frac{1}{2}\diff z \end{array}\\ 
&=&\displaystyle \frac{1}{8}\int_{z=9}^{z=17}\frac{ \sqrt{ z^2-1 }}{\frac{z-1}{2}}\frac{1}{2}\diff z \\
&=&\displaystyle \frac{1}{8}\int_{z=9}^{z=17}\frac{ \sqrt{ z^2-1 }}{z-1}\diff z&&  \begin{array}{rcl}
\text{Trig. subst.: }z&=&\sec\theta \\ 
\sqrt{z^2-1}&=&\tan \theta \\
\diff z &= &\tan \theta\sec\theta \diff \theta
\end{array}\\
&=&\displaystyle \frac{1}{8}\int_{\theta=\Arcsec(9) }^{\theta=\Arcsec(17)}\frac{\tan \theta}{\sec \theta - 1} \sec \theta \tan \theta \diff \theta  && 
\begin{array}{rcl}
\text{Set }\alpha&=&\Arcsec(9)\\ 
\text{Set }\beta&=&\Arcsec(17)\\ 
\end{array}
\\
&=&\displaystyle \frac{1}{8}\int_{\alpha }^{\beta}\frac{\tan^2 \theta}{\sec \theta - 1} \sec \theta \diff \theta && \text{Use} \tan^2\theta=\sec^2\theta-1\\
&=&\displaystyle \frac{1}{8}\int_{\alpha }^{\beta}\frac{\sec^2 \theta-1}{\sec \theta - 1} \sec \theta \diff \theta \\
&=&\displaystyle \frac{1}{8}\int_{\alpha }^{\beta}\frac{(\cancel{ \sec\theta-1})(\sec \theta+1)}{\cancel{\sec \theta - 1}} \sec \theta \diff \theta \\
&=&\displaystyle \frac{1}{8}\int_{\alpha }^{\beta}\left(\sec^2\theta+ \sec \theta \right)\diff \theta && \begin{array}{l}\displaystyle\int\sec \theta\diff \theta= \ln\left|\sec\theta+\tan\theta\right|+C\\ \text{previously studied}\end{array}\\
&=&\displaystyle \frac{1}{8}\left[\tan \theta +\ln \left|\sec\theta+\tan\theta\right| \right]_{\alpha}^{\beta} && \begin{array}{rlc}\tan \theta &=& \sqrt{\sec^2\theta-1}, \theta\in\left[0,\frac{\pi}{2}\right) \\ \tan \alpha&=&\sqrt{9^2-1}=4\sqrt{5}\\ 
\tan \beta&=& \sqrt{17^2-1}=12\sqrt{2} \end{array}  \\
&=&\displaystyle \frac{1}{8}\left( 12\sqrt{2} +\ln (17+12\sqrt{2})-4\sqrt{5}-\ln (9+4\sqrt{5})\right) \\
&=&\displaystyle \frac{1}{8} \ln{}\left(12\sqrt{2}+17\right)-\frac{1}{8} \ln{}\left(4\sqrt{5}+9\right)-\frac{\sqrt{5}}{2}+\frac{3}{2}\sqrt{2} \\
&\approx& 1.083\quad.
\end{array}
\]
The two answers are both approximately 1.083, so that serves to cross verify our two solutions against one another. 

Comparing the two answers we notice that the logarithmic parts in the two answers look different (yet they must be equal). It follows that 
\[
\frac{1}{8} \ln{}\left(12\sqrt{2}+17\right)-\frac{1}{8} \ln{}\left(4\sqrt{5}+9\right) = \frac{1}{4} \ln{}\left(2\sqrt{2}+3\right)-\frac{1}{4} \ln{}\left(\sqrt{5}+2\right).
\]
A short computation (which computation?), left to the reader, confirms that indeed those two expressions are equal.

}