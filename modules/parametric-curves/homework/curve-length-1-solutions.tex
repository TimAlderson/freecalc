\solution{\ref{problemlengthy=sqrt(x)from1to2}
The length of the parametric curve is given by
\[
\begin{array}{rcll|l}
L&=&\displaystyle \int_{1}^{2}\sqrt{1+\left(\frac{\diff y}{\diff x}\right)^2 }\diff x\\
&=&\displaystyle \int_{1}^{2}\sqrt{1+\left(\frac{1}{2\sqrt{x}}\right)^2 }\diff x\\
&=&\displaystyle \int_{x=1}^{x=2} \sqrt{1 +\frac{1}{4x}}\diff x &&\begin{array}{rcl}\text{Substitute }4x&=&u\\\diff x&=&\frac{1}{4}\diff u\\ \end{array}\\
&=&\displaystyle \int_{u=4}^{u=8} \sqrt{1 +\frac{1}{u}}\left(\frac{1}{4}\diff u\right)\\
&=&\displaystyle \frac{1}{4}\int_{4}^{8}\sqrt{\frac{u+1}{u}}\diff u\\
&=&\displaystyle \frac{1}{4}\int_{4}^{8}\sqrt{\frac{u(u+1)}{u^2}}\diff u\\
&=&\displaystyle \frac{1}{4}\int_{4}^{8}\frac{\sqrt{u^2+u }}{u}\diff u\\
&=&\displaystyle \frac{1}{4}\int_{4}^{8}\frac{\sqrt{u^2+u+\frac{1}{4}-\frac{1}{4} }}{u}\diff u\\
&=&\displaystyle \frac{1}{4}\int_{4}^{8}\frac{\sqrt{\left(u+\frac{1}{2}\right)^2-\frac{1}{4} }}{u}\diff u\\
&=&\displaystyle \frac{1}{4}\int_{4}^{8}\frac{ \sqrt{\frac{1}{4}\left( \left(2u+1\right)^2-1\right) }}{u}\diff u\\
&=&\displaystyle \frac{1}{8}\int_{u=4}^{u=8}\frac{ \sqrt{ \left(2u+1\right)^2-1 }}{u}\diff u &&\begin{array}{rcl}\text{Substitute }2u+1&=&z\\ u&=&\frac{z-1}{2}\\\diff u&=&\frac{1}{2}\diff z \end{array}\\ 
&=&\displaystyle \frac{1}{8}\int_{z=9}^{z=17}\frac{ \sqrt{ z^2-1 }}{\frac{z-1}{2}}\frac{1}{2}\diff z \\
&=&\displaystyle \frac{1}{8}\int_{z=9}^{z=17}\frac{ \sqrt{ z^2-1 }}{z-1}\diff z&&  \begin{array}{rcl}
\text{Trig. subst.: }z&=&\sec\theta \\ 
\sqrt{z^2-1}&=&\tan \theta \\
\diff z &= &\tan \theta\sec\theta \diff \theta
\end{array}\\
&=&\displaystyle \frac{1}{8}\int_{\theta=\Arcsec(9) }^{\theta=\Arcsec(17)}\frac{\tan \theta}{\sec \theta - 1} \sec \theta \tan \theta \diff \theta  && 
\begin{array}{rcl}
\text{Set }\alpha&=&\Arcsec(9)\\ 
\text{Set }\beta&=&\Arcsec(17)\\ 
\end{array}
\\
&=&\displaystyle \frac{1}{8}\int_{\alpha }^{\beta}\frac{\tan^2 \theta}{\sec \theta - 1} \sec \theta \diff \theta && \text{Use} \tan^2\theta=\sec^2\theta-1\\
&=&\displaystyle \frac{1}{8}\int_{\alpha }^{\beta}\frac{\sec^2 \theta-1}{\sec \theta - 1} \sec \theta \diff \theta \\
&=&\displaystyle \frac{1}{8}\int_{\alpha }^{\beta}\frac{(\cancel{ \sec\theta-1})(\sec \theta+1)}{\cancel{\sec \theta - 1}} \sec \theta \diff \theta \\
&=&\displaystyle \frac{1}{8}\int_{\alpha }^{\beta}\left(\sec^2\theta+ \sec \theta \right)\diff \theta && \begin{array}{l}\displaystyle\int\sec \theta\diff \theta= \ln\left|\sec\theta+\tan\theta\right|+C\\ \text{previously studied}\end{array}\\
&=&\displaystyle \frac{1}{8}\left[\tan \theta +\ln \left|\sec\theta+\tan\theta\right| \right]_{\alpha}^{\beta} && \begin{array}{rlc}\tan \theta &=& \sqrt{\sec^2\theta-1}, \theta\in\left[0,\frac{\pi}{2}\right) \\ \tan \alpha&=&\sqrt{9^2-1}=4\sqrt{5}\\ 
\tan \beta&=& \sqrt{17^2-1}=12\sqrt{2} \end{array}  \\
&=& \frac{1}{8}\left( 12\sqrt{2} +\ln (17+12\sqrt{2})-4\sqrt{5}-\ln (9+4\sqrt{5})\right) \approx 1.083
\end{array}
\]

}

\solution{\ref{problemlengthy=x^2from1to2}
The length of the parametric curve is given by
\[
\begin{array}{rcll|l}
L&=&\displaystyle \int_{1}^{2}\sqrt{1+\left(\frac{\diff y}{\diff x}\right)^2 }\diff x\\
&=&\displaystyle \int_{x=1}^{x=2}\sqrt{1+4x^2 }\diff x &&\begin{array}{rcl}\text{Substitute }2x&=&u\\\diff x&=&\frac{1}{2}\diff u\\ \end{array}\\
&=&\displaystyle \int_{u=2}^{u=4} \sqrt{u^2+1}\left(\frac{1}{2}\diff u\right)\\
&=&\displaystyle \frac{1}{2}\int_{u=2}^{u=4} \sqrt{u^2+1}\diff u&& 
\begin{array}{l}\displaystyle
\int \sqrt{u^2+1}\diff u \\
= \displaystyle \frac{1}{2}\left(u\sqrt{u^2+1}+\ln\left(u+\sqrt{u^2+1}\right) \right)+C\\
\text{previously studied}
\end{array}
\\
&=&\frac{1}{4} \left[ u\sqrt{u^2+1}+\ln\left(u+\sqrt{u^2+1}\right)\right]_{2}^4\\
&=&\sqrt{17}+\frac{1}{4} \log{}\left(\sqrt{17}+4\right)-\frac{1}{4} \log{}\left(\sqrt{5}+2\right)-\frac{\sqrt{5}}{2} \\
&\approx& 3.167841
\end{array}
\]

}

\solution{ \ref{problemlengthx=sqrt(t)-2t,y=8/3t^(3/4)}.
The length of the parametric curve is given by
\[
L={\displaystyle \int_{1}^4 \sqrt{\left(\frac{\diff x}{\diff t}\right)^2 + \left( \frac{\diff y}{\diff t} \right)^2}  \diff t}\quad .
\]
We have that 
\[
\begin{array}{rclll}
\displaystyle \frac{\diff x}{\diff t} &=&\displaystyle  \frac{1}{2\sqrt{t}} - 2\\
\displaystyle \frac{\diff y}{\diff t} &=&\displaystyle  2t^{-\frac{1}4}\\
\displaystyle \left(\frac{\diff x}{\diff t}\right)^2 &=&\displaystyle  \frac{1}{4t} - \frac{2}{\sqrt{t}} + 4\\
\displaystyle \left(\frac{\diff y}{\diff t}\right)^2 &=&\displaystyle  4t^{-\frac{1}{2}} = \frac{4}{\sqrt{t}}\\
\displaystyle \left(\frac{\diff x}{\diff t}\right)^2+\left(\frac{\diff y}{\diff t}\right)^2 & =&\displaystyle  \frac{1}{4t} + 2\frac{1}{\sqrt{t}} + 4 = \left(\frac{1}{2\sqrt{t}} + 2\right)^2\quad .
\end{array}
\]

$\frac{1}{2\sqrt{t}} +2$ is positive and $\sqrt{\left(\frac{ 1}{2 \sqrt{t}} +2\right)^2} =\frac{1}{2\sqrt{t}} +2$. So the integral becomes 
\[\displaystyle 
L= \int_1^4 \left(\frac{1}{2\sqrt{t}} +2\right)  \diff t=\left[\sqrt{t} + 2t\right]_{t=1}^{t=4}=(2+8)-(1+2)=7\quad .
\]
}