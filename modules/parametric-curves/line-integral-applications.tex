\begin{frame}
%\frametitle{Line Integral Applications}
\begin{example}
An object thrown from initial position $\fcv{r}_0$ with initial velocity $\fcv{v}_0$. Describe the trajectory of the object (ignore air resistance).\pause

Total force: gravity: $m\fcv{a} = \fcv{F} =  -mg \fcv{k} \Longrightarrow \fcv{a} = -g \fcv{k} \Longrightarrow \fcv{r}''(t) = -g \fcv{k}$
\pause
$$\fcv{r}'(t) = \fcv{r}'(0) + \int_{0}^t -g \fcv{k} \; d\tau = \fcv{v}_0 + \left(\int_{0}^t -g\; d\tau \right) \fcv{k} = \fcv{v}_0 -gt \fcv{k}$$
\pause
$$\fcv{r}(t) = \fcv{r}(0) + \int_{0}^t (\fcv{v}_0-g\tau \fcv{k}) \; d\tau = \fcv{r}_0 + \left( \int_0^t d\tau \right) \fcv{v}_0 - \left( \int_0^t g\tau \; d\tau \right) \fcv{k} \Longrightarrow$$
%
$$\boxed{\fcv{r}(t) = \fcv{r}_0  +t  \fcv{v}_0 - \frac{1}{2}gt^2 \fcv{k}}$$
\pause
Parabola in the plane determined by $\fcv{v}_0$ and $\fcv{k}$.
\end{example}
\end{frame}