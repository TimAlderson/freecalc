% begin module parametric-tangents-ex1
\begin{frame}[t]
\begin{example}
\begin{columns}
\column{0.25\textwidth}
\psset{xunit=0.4cm, yunit=0.4cm}
\begin{pspicture}(-0.9, -2.4)(4.4,2.499997)
\tiny
\fcAxesStandard{-0.650000}{-2.150000}{4.150000}{2.149997}
%Calculator input: plotCurve{}(t^{2}, t^{3}-3 t, -2, 2)
\parametricplot[linecolor=\fcColorGraph, plotpoints=1000]{-2}{2}{t 2.0000000 exp t -3.0000000 mul t 3.0000000 exp add }
\end{pspicture}
\column{0.75\textwidth}
A curve $C$ is defined by $x = t^2, y = t^3 - 3t$.
\end{columns}
\begin{enumerate}
\item Show $C$ traverses $(x,y)=(3,0)$ for two values of $t$; find the tangent slopes for both of these values.
\item Find the points on $C$ where the tangents are horizontal or vertical.
\item Find two intervals where we can write $y$ as a function of $x$.
\item Determine concavity intervals of the functions found in item 3.
\end{enumerate}
\end{example}
\vspace{4cm}
\end{frame}




\begin{frame}[t]
\begin{example}
\begin{columns}
\column{0.25\textwidth}
\psset{xunit=0.4cm, yunit=0.4cm}
\begin{pspicture}(-0.9, -2.4)(4.4,2.499997)
\tiny%
\fcAxesStandard{-0.650000}{-2.150000}{4.150000}{2.149997}%
%Calculator input: plotCurve{}(t^{2}, t^{3}-3 t, -2, 2)
\parametricplot[linecolor=\fcColorGraph, plotpoints=1000]{-2}{2}{t 2.0000000 exp t -3.0000000 mul t 3.0000000 exp add }%
\fcFullDotBlue{3}{0}%
\uncover<15->{%
\psline[linecolor=\fcColorTangent](2,1.732050808)(4,-1.732050808)%
\psline[linecolor=\fcColorTangent](2,-1.732050808)(4,1.732050808)%
}%
\end{pspicture}
\column{0.75\textwidth}
A curve $C$ is defined by \alert<handout:0| 2>{$x = t^2, y = t^3 - 3t$}.
\end{columns}
\begin{enumerate}
\item Show $C$ traverses $(x,y)=(3,0)$ for two values of $t$; find the tangent slopes for both of these values.
\end{enumerate}
\begin{itemize}
\item<2-| alert@3-4>  $3 = \alert<handout:0| 2>{x = t^2}$ \ if \ $t = $ \uncover<4->{$\pm \sqrt{3}$.}
\item<2-| alert@5-6>  $0 = \alert<handout:0| 2>{y = t^3 - 3t} = t(t^2-3)$\  if \ $t = $ \uncover<6->{$0$\  or\  $\pm \sqrt{3}$.}
\item<7->  Therefore the point $(3,0)$ is traversed when $t$ equals $\sqrt{3}$ or $-\sqrt{3}$.
\item<8-> $ \uncover<8->{ \frac{\diff y}{\diff x} = \frac{\alert<handout:0| 9-10>{\diff y / \diff t}}{\alert<handout:0| 11-12>{\diff x / \diff t}} %
} \uncover<9->{= \frac{\alert<handout:0| 10>{\uncover<10->{3t^2-3 }}}{\alert<handout:0| 12>{\uncover<12->{2t }}}}$\uncover<12->{\quad .}
\item<13->
Plug in $t = \pm \sqrt{3}$:
$\displaystyle
\uncover<13->{%
\frac{\diff y}{\diff x} _{|t = \pm \sqrt{3}} = \frac{3(\pm \sqrt{3})^2 - 3}{2(\pm \sqrt{3})} = %
}%
\uncover<14->{%
\pm \frac{6}{2\sqrt{3}} = \pm \sqrt{3}%
}%
$
\uncover<15->{%
Therefore the tangents at $(3,0)$ have slopes $\pm \sqrt{3}$.
}%
\end{itemize}
\end{example}
\vspace{4cm}
\end{frame}

\begin{frame}[t]
\begin{example}
\begin{columns}
\column{0.25\textwidth}
\psset{xunit=0.4cm, yunit=0.4cm}
\begin{pspicture}(-0.9, -2.4)(4.4,2.499997)
\tiny
\fcAxesStandard{-0.650000}{-2.150000}{4.150000}{2.149997}
%Calculator input: plotCurve{}(t^{2}, t^{3}-3 t, -2, 2)
\parametricplot[linecolor=\fcColorGraph, plotpoints=1000]{-2}{2}{t 2.0000000 exp t -3.0000000 mul t 3.0000000 exp add }
\uncover<6->{%
\fcFullDotBlue{1}{2}
\fcFullDotBlue{1}{-2}
\psline[linecolor=\fcColorTangent](0.1,2)(1.9,2)
\psline[linecolor=\fcColorTangent](0.1,-2)(1.9,-2)
}%
\uncover<10->{%
\fcFullDotBlue{0}{0}
\psline[linecolor=\fcColorTangent](0,-1)(0,1)
}%
\end{pspicture}
\column{0.75\textwidth}
A curve $C$ is defined by $x = t^2, y = t^3 - 3t$.
\end{columns}
\begin{enumerate}
\setcounter{enumi}{1}
%\item  Show that $C$ has two tangents at $(3,0)$ and find their slopes.
\item  Find the points on $C$ where the tangents are horizontal or vertical.
%\item  Determine where the curve is concave up or down.
\end{enumerate}
\begin{columns}[t]
\column{.5\textwidth}
Horizontal tangent:
\abovedisplayskip=0pt
\belowdisplayskip=0pt
\begin{eqnarray*}
\frac{\diff y}{\diff t} & = & 0\\
\uncover<2->{%
3t^2 - 3%
}%
& \uncover<2->{ = } & %
\uncover<2->{%
0
}\\%
\uncover<3->{%
3(t^2 - 1)%
}%
& \uncover<3->{ = } & %
\uncover<3->{%
0
}\\%
\uncover<4->{%
t%
}%
& \uncover<4->{ = } & %
\uncover<4->{%
\pm 1%
}%
\end{eqnarray*}
\uncover<5->{$\frac{\diff x}{\diff t} \neq 0$ when $t = \pm 1$, so there are horizontal tangents when $t = \pm 1$.}

\uncover<6->{%
The points are $(1, 2)$ and $(1, -2)$.
}%
\column{.5\textwidth}
Vertical tangent:
\abovedisplayskip=0pt
\belowdisplayskip=0pt
\begin{eqnarray*}
\frac{\diff x}{\diff t} & = & 0\\
\uncover<7->{%
2t%
}%
& \uncover<7->{ = } & %
\uncover<7->{%
0%
}\\%
\uncover<8->{%
t%
}%
& \uncover<8->{ = } & %
\uncover<8->{%
0%
}%
\end{eqnarray*}
\uncover<9->{$\frac{\diff y}{\diff t} \neq 0$ when $t =  0$, so there is a vertical tangent when $t = 0$.}

\uncover<10->{%
The points is $(0,0)$.
}%
\end{columns}
\end{example}
\vspace{4cm}
\end{frame}



\begin{frame}[t]
\begin{example} %[Example 1, p. 667]
\begin{columns}
\column{0.25\textwidth}
\psset{xunit=0.4cm, yunit=0.4cm}
\begin{pspicture}(-0.9, -2.4)(4.4,2.499997)
\tiny
\fcAxesStandard{-0.650000}{-2.150000}{4.150000}{2.149997}
\uncover<1-4,6->{%
%Calculator input: plotCurve{}(t^{2}, t^{3}-3 t, -2, 2)
\parametricplot[linecolor=\fcColorGraph, plotpoints=1000]{-2}{0}{t 2.0000000 exp t -3.0000000 mul t 3.0000000 exp add }%
}%
\uncover<1-5>{%
\parametricplot[linecolor=\fcColorGraph, plotpoints=1000]{0}{2}{t 2.0000000 exp t -3.0000000 mul t 3.0000000 exp add }%
}%
\end{pspicture}
\column{0.75\textwidth}
A curve $C$ is defined by $x = t^2, y = t^3 - 3t$.
\end{columns}
\begin{enumerate}
\setcounter{enumi}{2}
%\item  Show that $C$ has two tangents at $(3,0)$ and find their slopes.
%\item  Find the points on $C$ where the tangents are horizontal or vertical.
\item  Find two intervals where we can write $y$ as a function of $x$.
\end{enumerate}
\uncover<2->{From $x=t^2$ we have that $t=\pm \sqrt{x}$.} \uncover<3->{Therefore, when $t>0$, we have that $t=\sqrt{x}$.} \uncover<4->{Since that determines uniquely $t$ via $x$, this means that for $t>0$  $y$ is a function of $x$.} \uncover<5->{In other words, for $t>0$, the curve satisfies the vertical line test. } \uncover<6->{Similarly we conclude that when $t<0$, $y$ is a function of $x$.}
\end{example}
\vspace{5cm}
\end{frame}

\begin{frame}[t]
\begin{example} %[Example 1, p. 667]
\begin{columns}
\column{0.25\textwidth}
\psset{xunit=0.4cm, yunit=0.4cm}
\begin{pspicture}(-0.9, -2.4)(4.4,2.499997)
\tiny
\fcAxesStandard{-0.650000}{-2.150000}{4.150000}{2.149997}
\uncover<1-11,13->{%
%Calculator input: plotCurve{}(t^{2}, t^{3}-3 t, -2, 2)
\parametricplot[linecolor=\fcColorGraph, plotpoints=1000]{-2}{0}{t 2.0000000 exp t -3.0000000 mul t 3.0000000 exp add }%
}%
\uncover<1-12>{%
\parametricplot[linecolor=\fcColorGraph, plotpoints=1000]{0}{2}{t 2.0000000 exp t -3.0000000 mul t 3.0000000 exp add }%
}%
\end{pspicture}
\column{0.75\textwidth}
A curve $C$ is defined by $x = t^2, y = t^3 - 3t$.
\end{columns}
\begin{enumerate}
\setcounter{enumi}{3}
\item  Determine the concavity intervals of the functions found in item 3.
\end{enumerate}
\uncover<2->{Find the second derivative:}%

$\begin{array}{rcl}
\displaystyle \uncover<2->{%
\frac{\diff^2 y}{\diff x^2}%
}%
& \uncover<2->{ = } &%
\displaystyle \uncover<2->{%
\frac{\frac{\diff}{\diff t}\left( \alert<handout:0| 3-4>{\frac{\diff y}{\diff x}}\right)}{\alert<handout:0| 5-6>{\frac{\diff x}{\diff t}}}%
}  \uncover<3->{ = }  \uncover<3->{%
\frac{\frac{\diff}{\diff t}\left( \alert<handout:0| 3-4,7>{\uncover<4->{\frac{3t^2-3}{2t}}}\right)}{\alert<handout:0| 5-6>{\uncover<6->{2t}}}%
}\\%
& \uncover<7->{ = } &\displaystyle
\uncover<7->{%
\frac{\alert<handout:0| 8-9>{\frac{\diff}{\diff t}\left( \alert<handout:0| 7>{\frac{3}{2}\left( t - \frac{1}{t}\right)}\right)}}{2t}%
}  \uncover<8->{ = }  \uncover<8->{%
\frac{\alert<handout:0| 8-9>{\uncover<9->{\frac{3}{2} + \frac{3}{2t^2} }}}{2t}%
}\\%
& \uncover<10->{ = } &\displaystyle
\uncover<10->{%
\frac{\frac{3t^2 + 3}{2t^2}}{2t}%
}  \uncover<11->{ = } \uncover<11->{%
\frac{3(t^2 + 1)}{4t^3}%
}%
\end{array}
$

\uncover<12->{%
Therefore $y$ as a function of $x$ (which is a function of $t$) is concave up when $t > 0$}\uncover<13->{ and concave down when $t < 0$.}%
\end{example}
\vspace{4cm}
\end{frame}
% end module parametric-tangents-ex1
