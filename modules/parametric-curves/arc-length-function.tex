\begin{frame}
\frametitle{Arclength Function}
\begin{itemize}
\item Fix the point $a$ as starting point. For $t \geqslant a$, let
\item $L(t) = $ distance traveled between $a$ and $t$ = length of the piece of the curve corresponding to values of the parameter between $a$ and $t$.
\[
L(t) =  \int_{a}^t |\fcv{r}'(\tau)| \, d\tau 
\]
\item The function $L\colon [a,b]\to \RR$ is called the \emph{arclength function}.
\item \underline{Example}: $\fcv{r}(t) = \langle \cos{t}, \sin{t}, t\rangle$. Then
\[\fcv{r}'(t) = \langle -\sin{t}, \cos{t}, 1 \rangle
\Longrightarrow |\fcv{r}'(t)| = \sqrt{2}\; 
\]
and the arclength function is
\[L(t) = \int_0^t |\fcv{r}'(\tau)| \; d\tau = t\sqrt{2} \; .\]
\end{itemize}

\end{frame}