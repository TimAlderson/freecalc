% begin module parametric-tangents-intro-version2
\begin{frame}
\frametitle{Tangents}
Let $C $ be the curve $C:\left|\begin{array}{rcl}x&=&f(t)\\y&=&g(t)\end{array} \right., t\in [a,b]$.
\begin{definition}
Suppose \alertNoH{4,5}{ $f'(t)$ and $g'(t)$ are not simultaneously equal to $0$.}
\begin{itemize}
\item  We define $(f'(t), g'(t))$ to be the \emph{tangent vector} to $C$ at $t$.
\item<2->  We define the line passing through $(f(t), g(t))$ with direction vector equal to the tangent vector to be \emph{tangent line} to $C$ at $t$. In other words, the tangent line has equation
\[
(x-f(t))g'(t) =(y-g(t))f'(t)\quad .
\]
\item<3->  We say that the tangent to $C$ at $t$ is vertical if $f'(t)=0$ (\alertNoH{4}{and therefore $g'(t)\neq 0$}).
\end{itemize}
\end{definition}
\uncover<5->{\alertNoH{5-}{Note.} When $f'(t)=g'(t)=0$, for curves $C$ with additional properties, natural definition(s) of tangent(s) do exist but are beyond Calc II.}
\end{frame}

