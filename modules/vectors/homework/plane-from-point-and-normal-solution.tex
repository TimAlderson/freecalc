\solution{\ref{problemFindPlaneFromP(1,2,3)andn(4,5,6)}
As studied, an equation passing through $(1,2,3)$ and with normal $(4,5,6)$ has equation:

\[
\begin{array}{rcl}
( x, y, z)\cdot ( 4,5,6)&=& (4,5,6)\cdot ( 1,2,3)\\
4x +5y+6z&=&23
\end{array}
\]
To find parametric equations of the plane, we need to find two directions, $\fcv u, \fcv v$, that can be added to the base point to obtain all points in the plane. This means that a direction vector $\fcv u$ has to be perpendicular to $\fcv n$. Equivalently, a direction vector $\fcv u$ lies in the plane passing through the origin and orthogonal to $\fcv n$. This means $\fcv u( u_1, u_2, u_3)$ satisfies the equation:
\begin{equation}\label{eqproblemFindPlaneFromP(1,2,3)andn(4,5,6)eq1}
\begin{array}{rcl}
\fcv u\cdot \fcv n&=&0\\
4 u_1+5u_2+6u_3&=&0.
\end{array}
\end{equation}
There are infinitely many solutions to that equation - in fact, for each point in the plane passing through the origin and orthogonal to $\fcv n$ there is one solution. However, we only need to find two such non-colinear solutions, and declare them to be our vectors $\fcv u$ and $\fcv v$. It is very easy to do that: if we set $u_1$ and $u_2$ to be arbitrary, then $u_3$ can always be chosen so as to make the equation above hold. There are a number of accepted ways to choose $u_1$ and $u_2$ in a not-so-arbitrary fashion. For reasons outside of the scope of this homework, such ways to choose $u_1$ and $u_2$ may be preferable to the choosing at random. Our scheme for choosing a vector $\fcv u$ will be to choose $u_1=1$ and $u_2=0$ (or the other way round for $\fcv v$), and then to rescale the resulting vector so all coordinates are integers and the first non-zero coordinate is positive. In other words, we select $\fcv u$ to be proportional to $( 1, 0, -\frac{4}{6} )$, and $\fcv v$ to be proportional to $( 0, 1 ,-\frac{5}{6} )$, i.e., we select
\[
\begin{array}{rcl}
\fcv u&=& ( 3,0, -2 ) \\
\fcv v&=& ( 0, 6, -5 )
\end{array}.
\]
Finally we get that a parametric equation of the plane is given by:
\begin{equation}\label{eqproblemFindPlaneFromP(1,2,3)andn(4,5,6)eq2}
( 1,2,3 ) + s( 3,0, -2 ) +t( 0, 6, -5 )\quad .
\end{equation}
The above equations are not unique; therefore our problem has many correct answers.

A question arises: what do we need to do in order to check if two plane parametrizations are equivalent? Equivalently, how do we check that equation \eqref{eqproblemFindPlaneFromP(1,2,3)andn(4,5,6)eq2} gives a plane that coincides with the plane in given in \eqref{eqproblemFindPlaneFromP(1,2,3)andn(4,5,6)eq1}? Here's what we need to do to make sure our answer is correct (we leave the justification for that to the reader):
\begin{itemize}
\item Check that our $\fcv u, \fcv v$ are orthogonal to $\fcv n$.
\item Check that our $\fcv u, \fcv v$ are not proportional to one another.
\item Check that the base point of our equation is in the original plane.
\end{itemize}

}
