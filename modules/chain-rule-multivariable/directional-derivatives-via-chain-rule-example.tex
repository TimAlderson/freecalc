\begin{frame}
\frametitle{Example}

Directional derivative
\begin{itemize}
  \item of function $f(x,y,z) = \ln{(x^2+2y^2-z^2)}$;
  \item at the point $P(2,1,-1)$;
  \item in the direction $\textbf{v}=\langle -1,2,1\rangle$.
\end{itemize}
\pause
A unit vector in the direction of $\textbf{v}$ is
%
$$\textbf{u} = \frac{1}{|\textbf{v}|} \textbf{v} = \frac{1}{\sqrt{6}} \langle -1,2,1\rangle\; .$$
%
\begin{overlayarea}{\textheight}{5cm}
\only<3>{The partial derivatives are
%
\begin{align*}
  \frac{\partial f}{\partial x} = \frac{2x}{x^2+2y^2-z^2} & \Longrightarrow \frac{\partial f}{\partial x}(2,1,-1) = \frac{4}{5} \\
  %
  \frac{\partial f}{\partial y} = \frac{4y}{x^2+2y^2-z^2} & \Longrightarrow \frac{\partial f}{\partial y}(2,1,-1) = \frac{4}{5} \\
  %
  \frac{\partial f}{\partial z} = \frac{-2z}{x^2+2y^2-z^2} & \Longrightarrow \frac{\partial f}{\partial z}(2,1,-1) = \frac{2}{5}
\end{align*}
%
$$\textbf{W}_{f,(2,1,-1)} = \langle \frac{4}{5}, \frac{4}{5}, \frac{2}{5}\rangle$$}
%
\only<4->{
%
$$\textbf{W}_{f,(2,1,-1)} = \langle f_x(2,1,-1), f_y(2,1,-1), f_z(2,1,-1) \rangle = \langle \frac{4}{5}, \frac{4}{5}, \frac{2}{5}\rangle$$
%
$$(D_{\textbf{u}}f)(2,1,-1) = \textbf{W}_{f,(2,1,-1)} \cdot \textbf{u} = \frac{\sqrt{6}}{5}$$

$(D_{\textbf{u}}f)(2,1,-1)>0 \Longrightarrow$ \pause if we start at $(2,1,-1)$ and move in the direction $\textbf{u}$, then $f$ is increasing.}
  \end{overlayarea}
\end{frame}