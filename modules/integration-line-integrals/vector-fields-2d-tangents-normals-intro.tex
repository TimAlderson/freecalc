\begin{frame}
\frametitle{Planar Vector Fields}

\textbf{This slide needs a rewrite.}
\begin{itemize}
\item $\fcv{F}$: smooth vector field defined on $\mathcal{R}$;
\item $\fcv{p}$: point in the interior of $\mathcal{R}$;
\item Question: How does $\fcv{F}$ impact the space around $p$?
\item \pause Method: study the effect of $\fcv{F}$ on the boundary $C =\partial D$ of a small region $D$ around $p$
\end{itemize}
\pause There are two fundamental effects:

\begin{enumerate}
  \item \pause Normal component of $\fcv{F}$ $\Longrightarrow$ moves matter across the boundary. Accumulation: \emph{flux across $C$};

  \item \pause Tangential component of $\fcv{F}$ $\Longrightarrow$ moves matter along the boundary. Accumulation: \emph{circulation along $C$}.
\end{enumerate}

\end{frame}