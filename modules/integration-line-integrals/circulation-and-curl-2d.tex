\begin{frame}
\frametitle{Circulation and Planar Curl}

\begin{columns}
\column{0.2\textwidth}
\psset{xunit=0.75 cm, yunit=0.75cm}
\begin{pspicture}(-1,-1)(1,1)
\tiny
\fcBoundingBox{-2}{-2}{2}{2}
\pstVerb{%
10 dict begin
/theScale 1 def
/theCurve {t 20 sub cos theScale mul t sin theScale mul} def
/theTangent {[t 20 sub sin -1 mul theScale mul t cos theScale mul] \fcVectorNormalize } def
/theNormal {[t 20 sub cos  -1 mul theScale mul t sin -1 mul theScale mul] \fcVectorNormalize} def
}%
\pscustom*[linecolor=pink]{\parametricplot{0}{360}{theCurve}}%
%\fcVectorField[linecolor=blue]{-1.5}{-1.5}{6}{6}{0.6}{x y add 6 div y 3 div }%
\parametricplot[arrows=->, linecolor=\fcColorGraph]{0}{360}{ theCurve}
\pstVerb{1 dict begin /t 60 def}%
\psline[arrows=->](! theCurve )(! [theCurve] theTangent \fcVectorPlusVector \fcArrayToStack )%
\rput[bl](! [theCurve] theTangent \fcVectorPlusVector \fcArrayToStack ){$~\fcv T$}%
\psline[arrows=->](! theCurve )(! [theCurve] theNormal \fcVectorPlusVector \fcArrayToStack )%
\rput[tr](! [theCurve] theNormal \fcVectorPlusVector \fcArrayToStack -0.1 add){$~~\fcv N$}%
\rput(! theCurve){%
\fcAngleBetweenVectors[arrows=->, linecolor=blue]{theTangent }{theNormal}{0.3}{}%
}%
\pstVerb{end}%
\pstVerb{end}
\end{pspicture}
\column{0.8\textwidth}
\begin{itemize}
\item<1-> Fix an orientation of the plane. Convention: a positive orientation in the plane corresponds to the counterclockwise motion.
\item<2-> A regular parametrization of $C$ is \emph{positive} if $C$ is followed in a counterclockwise direction \uncover<3->{$\Leftrightarrow$ $ \fcv{T}, \fcv{N}$ are positively oriented.}
\end{itemize}
\end{columns}
\uncover<4->{
\[
\text{Circulation } = \oint_C \fcv{F} \cdot \fcv{T} \diff s = \oint_C \fcv{F} \cdot \fcv{\diff r}\; .
\]
where $\diff s$ is the element of arclength.
}

\uncover<5->{%
The \emph{planar curl} of $\fcv{F}$ at $p$: the density of circulation
%
\begin{align*}
(\curl_{\,\fcv{k}} \fcv{F})(p)= & \lim_{D \to \{p\}} \frac{\text{Circulation along boundary}}{\text{Area}(D)} =  \\ = &
      \lim_{D\to \{p\}} \frac{1}{\text{Area}(D)} \oint_C \fcv{F} \cdot \fcv{\diff r} \; .
\end{align*}
}%

\end{frame}