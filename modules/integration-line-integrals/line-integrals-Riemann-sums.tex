\begin{frame}
\frametitle{Riemann Sums for Line Integrals}
\begin{columns}

\column{0.3\textwidth}
\begin{pspicture}(-0.15,-2)(3.5,3)
\tiny%
\renewcommand{\fcScreen}{[-1 4 -1] 0}%
\fcBoundingBox{-0.1}{-1}{3}{3}%
%\fcAxesIIId{2}{2}{2}
\newcommand{\numIterationsPlusOne}{6}%
\newcommand{\numIterations}{5}%
\pstVerb{
10 dict begin
/theCurveX {t 2 exp } def
/theCurveY {t 180 mul sin  } def
/tmax 2 def
/tmin 0 def
/indexOfInterest 4 def
/Delta tmax tmin sub \numIterations\space div def
/fOnTheCurve {theCurveY theCurveX 3 div add 1 add} def
}%
\fcCurveIIId{tmax}{tmin}{[theCurveX theCurveY 0]}%
\uncover<12->{%
\multido{\na=0+1}{\numIterations}%
{%
\fcParallelogramIIId[linecolor=cyan, linewidth=1]{1 dict begin /t tmin Delta \na\space mul add def [theCurveX theCurveY 0] end }{1 dict begin /t tmin Delta \na\space 1 add mul add def [theCurveX theCurveY 0] end}{1 dict begin [/t tmin Delta \na\space mul add def theCurveX theCurveY /t tmin Delta \na\space 0.5 add mul add def fOnTheCurve] end }%
}%
}%
\uncover<4->{%
\multido{\na=0+1}{\numIterationsPlusOne}%
{%
\pstVerb{ 1 dict begin /t tmin Delta \na\space mul add def}%
\fcDotIIId[linecolor=blue]{[theCurveX theCurveY 0]}%
\pstVerb{end}%
}%
}%
\uncover<5->{%
\multido{\na=0+1}{\numIterations}%
{%
\pstVerb{ 1 dict begin /t tmin Delta \na\space 0.5 add mul add def}%
\fcDotIIId[linecolor=red]{[theCurveX theCurveY 0]}%
\fcPutIIId[t]{[theCurveX theCurveY -0.2]}{$P_{\fcEvalToInt{\na+1}}$}%
\pstVerb{end}%
}%
}%
\uncover<10->{%
\multido{\na=0+1}{\numIterations}%
{%
\pstVerb{ 1 dict begin /t tmin Delta \na\space 0.5 add mul add def}%
\fcLineIIId{[theCurveX theCurveY 0]}{[theCurveX theCurveY fOnTheCurve]}%
\pstVerb{end}%
}%
}%
\uncover<10->{%
\multido{\na=0+1}{\numIterations}%
{%
\pstVerb{ 1 dict begin /t tmin Delta \na\space 0.5 add mul add def}%
\fcLineIIId[linecolor=red, linewidth=2pt]{[theCurveX theCurveY 0]}{[theCurveX theCurveY fOnTheCurve]}%
\pstVerb{end}%
}%
}%
\uncover<13->{%
\pscustom*[linecolor=orange]{%
\fcCurveIIId{tmin}{tmax}{[theCurveX theCurveY fOnTheCurve]}%
\fcCurveIIId{tmax}{tmin}{[theCurveX theCurveY 0]}%
}%
}%
\uncover<9->{%
\fcCurveIIId[linecolor=brown]{tmin}{tmax}{[theCurveX theCurveY fOnTheCurve]}%
}%
\uncover<10->{%
\multido{\na=0+1}{\numIterations}%
{%
\fcLineIIId[linecolor=blue, linewidth=0.5pt]{1 dict begin /t tmin Delta \na\space mul add def [theCurveX theCurveY 0] end }{1 dict begin /t tmin Delta \na\space 1 add mul add def [theCurveX theCurveY 0] end}%
}%
}%
\uncover<11>{%
\multido{\na=0+1}{\numIterations}%
{%
\fcLineIIId[linecolor=blue, linewidth=2pt]{1 dict begin /t tmin Delta \na\space mul add def [theCurveX theCurveY 0] end }{1 dict begin /t tmin Delta \na\space 1 add mul add def [theCurveX theCurveY 0] end}%
}%
}%
\uncover<8>{%
\fcLineIIId[linecolor=blue, linewidth=2pt]{1 dict begin /t tmin Delta indexOfInterest 1 sub mul add def [theCurveX theCurveY 0] end }{1 dict begin /t tmin Delta indexOfInterest mul add def [theCurveX theCurveY 0] end}%
}%
\pstVerb{ 1 dict begin /t tmin Delta indexOfInterest 0.5 sub mul add def}%
\uncover<8,9>{%
\fcLineIIId[linewidth=1pt]{[theCurveX theCurveY 0]}{[theCurveX theCurveY fOnTheCurve]}%
}%
\uncover<7>{%
\fcLineIIId[linewidth=2pt, linecolor=red]{[theCurveX theCurveY 0]}{[theCurveX theCurveY fOnTheCurve]}%
}%
\uncover<13>{%
\fcLineIIId[linewidth=1pt]{[theCurveX theCurveY 0]}{[theCurveX theCurveY fOnTheCurve]}%
}%
\pstVerb{end}%
\pstVerb{end}
\end{pspicture}

\column{0.7\textwidth}
\begin{itemize}
\item Let $C$ be a piecewise smooth curve (endpoints included) in space.
\item<2-> Let \alert<9>{$f$ be} a scalar or vector-valued \alert<9>{function defined on $C$}.
\item<3-> We aim to define the integral of $f$ on $C$ with respect to arclength.
\item<4-> Divide $C$ into pieces $D_1$, \ldots, $D_N$ with non-overlapping interiors;
\end{itemize}

\end{columns}

\begin{itemize}
\item<5-> Pick a sample point $P_k$ in each $D_k$. 
\item<6-> The accumulation on $D_k$ is approximated by $ \alert<7>{f(P_k)} \cdot \alert<8>{\text{length}(D_k)}$.
\item<10-> \alert<13>{The integral (total accumulation)} is approximated by the Riemann sum
$$\alert<12>{ \sum_{k=1}^N \alert<10>{f(P_k)} \cdot\alert<11>{ \text{length}(D_k)}}$$
\end{itemize}
\end{frame}