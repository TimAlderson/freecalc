\begin{frame}
\frametitle{The Boundary Operator, Closed Curve Orientation}
\begin{columns}
\column{0.25\textwidth}
\psset{xunit=0.75 cm, yunit=0.75cm}
\begin{pspicture}(-2,-2)(2,2)
\tiny
\fcBoundingBox{-2}{-2}{2}{2}
\pstVerb{%
10 dict begin
/theScale 1 def
/theCurve {t 20 sub cos theScale mul t sin theScale mul} def
/theTangent {[t 20 sub sin -1 mul theScale mul t cos theScale mul] \fcVectorNormalize } def
/theNormal {[t 20 sub cos  -1 mul theScale mul t sin -1 mul theScale mul] \fcVectorNormalize} def
}%
\pscustom*[linecolor=pink]{\parametricplot{0}{360}{theCurve}}%
%\fcVectorField[linecolor=blue]{-1.5}{-1.5}{6}{6}{0.6}{x y add 6 div y 3 div }%
\parametricplot[arrows=->, linecolor=\fcColorGraph]{0}{360}{ theCurve}
\pstVerb{1 dict begin /t 60 def}%
\psline[arrows=->](! theCurve )(! [theCurve] theTangent \fcVectorPlusVector \fcArrayToStack )%
\uncover<4->{
\rput[bl](! [theCurve] theTangent \fcVectorPlusVector \fcArrayToStack ){$~\fcv T$}%
\psline[arrows=->](! theCurve )(! [theCurve] theNormal \fcVectorPlusVector \fcArrayToStack )%
\rput[tr](! [theCurve] theNormal \fcVectorPlusVector \fcArrayToStack -0.1 add){$~~\fcv N$}%
}
\uncover<6,7>{
\psline[arrows=->, linewidth=2pt, linecolor=red](! theCurve )(! [theCurve] theNormal \fcVectorPlusVector \fcArrayToStack )%
}
\uncover<5,6>{
\psline[arrows=->](! theCurve )(! [theCurve] theNormal -1 \fcVectorTimesScalar \fcVectorPlusVector \fcArrayToStack )%
\rput[tl](! [theCurve] theNormal -1 \fcVectorTimesScalar \fcVectorPlusVector \fcArrayToStack -0.1 add){$\fcv N'$}%
}
\uncover<5>{
\psline[arrows=->, linewidth=2pt, linecolor=red](! theCurve )(! [theCurve] theNormal -1 \fcVectorTimesScalar \fcVectorPlusVector \fcArrayToStack )%
}
\uncover<8->{%
\rput(! theCurve){%
\fcAngleBetweenVectors[arrows=->, linecolor=blue]{theTangent }{theNormal}{0.3}{}%
}%
}
\pstVerb{end}%
\pstVerb{end}
\end{pspicture}
\column{0.75\textwidth}
\begin{itemize}
\item Let $D$ be an open set and $C$ a closed piecewise smooth curve with parametrization $\fcv r(t)$.
\item<2-> Suppose the boundary of $D$ equals $C$. 
\item<3-> Let $\fcv T =\frac{\fcv r}{|\fcv r|}$, ($\fcv T$ is the unit vector compatible with the orientation of $\fcv C$).
\end{itemize}
\end{columns}
\begin{itemize}
\item<4-> Let $\fcv N$ be a unit vector perpendicular to $\fcv T$. \uncover<5->{\alert<5,6>{There are two choices for $\fcv N$;}} \uncover<7->{we select that which points towards $D$ as indicated.}
\uncover<8->{
\begin{definition}[boundary]
We say that the oriented curve $C$ is the boundary of $D$ if the pair of vectors $(\fcv T,\fcv N)$ is positively oriented. \uncover<9->{We write 

\hfil\hfil$
C=\alert<10>{\partial} D\quad .
$
}

\uncover<10->{The symbol $\alert<10>{\partial}$ above is called \alert<10>{the \emph{boundary operator}}.}
\end{definition}
}
\item<11-> When walking along the boundary $\partial D$, $D$ is to the walker's left.
\end{itemize}


\end{frame}