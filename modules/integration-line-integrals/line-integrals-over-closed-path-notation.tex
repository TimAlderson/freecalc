\begin{frame}
\frametitle{The Notation $\oint$: Closed Path Integrals}
\begin{itemize}
\item Suppose that the curve image $C$, parametrized by $\fcv r(t):(x(t), y(t)), t\in [\alert<3>{a},\alert<4>{b} ]$ is a closed curve.
\item<2-> That is, the \alert<3>{start point} and the \alert<4>{end point} coincide. 
\item<3-> In other words $\alert<3>{(x(a), y(a))}=\alert<4>{(x(b), y(b))}$.
\item<5-> Let $\omega$ be a 1-form.
\item<6-> Then we sometimes use the notation 
\[
\oint_C\omega = \int_C \omega.
\]
\item<7-> The circle around the first integral simply indicates the path is closed.
\item<8-> The notation is mostly useful when we are integrating an closed 1-form. (Definition of closed form is/will be studied separately).
\end{itemize}
\end{frame}