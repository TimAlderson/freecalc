\begin{frame}
\begin{columns}
\column{0.3\textwidth}
\psset{xunit=0.5cm, yunit=0.5cm}
\begin{pspicture}(-1,-1)(2,1)%
\tiny%
\fcAxesStandard{-0.5}{-0.5}{3.2}{2.7}%
\pscustom*[linecolor=\fcColorAreaUnderGraph]{%
\psccurve(1, 0.5)(2, 0.2)(3,0.5)(3, 2.5)(2, 2.8) (1, 2.5 ) (1, 2)(2, 1.7)(2, 1.3)(1, 1)%
}%
\psccurve[linecolor=red, linewidth=1pt](1, 0.5)(2, 0.2)(3,0.5)(3, 2.5)(2, 2.8) (1, 2.5 ) (1, 2)(2, 1.7)(2, 1.3)(1, 1)%
\uncover<2>{%
\psecurve[linecolor=red, arrows=>->](2, 0.2)(3, 0.5)(3, 2.5)(2, 2.8)
\psecurve[linecolor=red, arrows=>-](2, 2.8)(1, 2.5)(1, 2  )(2, 1.7)
\psecurve[linecolor=red, arrows=>-](2, 1.3)(1, 1  )(1, 0.5)(2, 0.2)
\psline[linecolor=red](2.065,0.2)(2.065,2.8)%
\psline[linecolor=red, linewidth=0.3pt](2.165,0.4)(2.165,0.8)%
\psline[linecolor=red, arrows=->, linewidth=0.3pt](2.165,0.8)(2.165,0.5)%
\psline[linecolor=red, linewidth=0.3pt](2.165,2)(2.165,2.4)%
\psline[linecolor=red, arrows=->, linewidth=0.3pt](2.165,2.4)(2.165,2.1)%
\psline[linecolor=red, linewidth=0.3pt](1.965,0.4)(1.965,0.8)%
\psline[linecolor=red, arrows=->, linewidth=0.3pt](1.965,0.4)(1.965,0.7)%
\psline[linecolor=red, linewidth=0.3pt](1.965,2)(1.965,2.4)%
\psline[linecolor=red, arrows=->, linewidth=0.3pt](1.965,2.1)(1.965,2.3)%
}%
\uncover<4>{%
\psecurve[linecolor=orange, arrows=>-](2, 1.3)(1, 1  )(1, 0.5)(2, 0.2)
\psecurve[linecolor=orange](2, 1.3)(1, 1  )(1, 0.5)(2, 0.2)
\psline[linecolor=orange, arrows=->, linewidth=0.3pt](2.165,2.4)(2.165,2.1)%
\psline[linecolor=orange, linewidth=0.3pt](1.965,0.4)(1.965,0.8)%
\psline[linecolor=orange](2.065,0.2)(2.065,1.5)%
}%
\uncover<10>{%
\psecurve[linecolor=red, arrows=>-](3, 0.5)(3, 2.5)(2, 2.8)(1, 2.5)
\psecurve[linecolor=red, arrows=->](3, 2.5)(2, 2.8)(1, 2.5)(1, 2)
\psecurve[linecolor=red, arrows=->](1, 0.5)(2, 0.2)(3, 0.5)(3, 2.5)
\psecurve[linecolor=red, arrows=>-](1, 1  )(1, 0.5)(2, 0.2)(3, 0.5)
\psline[linecolor=red](2.065,1.5)(3.3,1.5)
\psline[linecolor=red, arrows=->, linewidth=0.3pt](2.3,1.6)(2.75,1.6)
\psline[linecolor=red, linewidth=0.3pt](2.3,1.6)(3.06,1.6)
\psline[linecolor=red, arrows=-<,linewidth=0.3pt](2.3,1.4)(2.75,1.4)
\psline[linecolor=red, linewidth=0.3pt](2.3,1.4)(3.06,1.4)
}%


\end{pspicture}
\column{0.7\textwidth}
\vskip -0.2cm
\begin{theorem}[Green]
$\displaystyle  \oint_{\partial D}\left( P \diff x + Q \diff y\right) =  \iint_D \left( \frac{ \partial Q}{\partial x} - \frac{ \partial P}{ \partial y} \right) \diff x \diff y \; .$
\end{theorem}
\end{columns}
\begin{proof} [When $D$= representable by curv. trapezoids in both directions]
So far, we demonstrated that  
$
\begin{array}{rcl|l}
\oint_{\partial D} P \diff x &=& \iint_D\left(- \frac{ \partial P}{ \partial y} \right) \diff x \diff y&  \text{curv. trapezoids vert. bases}\\
\oint_{\partial D} Q \diff y &=& \iint_D  \frac{ \partial Q}{\partial x} \diff x \diff y &\text{curv. trapezoids horiz. bases}
\end{array}
$.
\uncover<2->{
Suppose $D$ can be represented as a union of curvilinear trapezoids with \alert<2>{vertical bases}, pairwise intersecting on their boundaries only. \uncover<3->{ The theorem holds over \alert<4>{each curvilinear trapezoid;}} the contributions of the extra line integrals arising from the subdivision cancel one another.
}
\uncover<2>{}
\end{proof}

\vskip 10cm

\end{frame}