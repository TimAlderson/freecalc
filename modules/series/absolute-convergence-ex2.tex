% begin module absolute-convergence-ex2
\begin{frame}
\begin{example} %[Example 2, p. 751]
The alternating harmonic series
\abovedisplayskip=0pt
\belowdisplayskip=0pt
\[
\sum_{n=1}^\infty \frac{(-1)^{n-1}}{n} = 1 - \frac{1}{2} + \frac{1}{3} - \frac{1}{4} + \cdots%
\]
is convergent (by the alternating series test, as already demonstrated).
\begin{itemize}
\item<2->  Is it absolutely convergent?
\[
\uncover<2->{%
\sum_{n=1}^\infty \left| \frac{(-1)^{n-1}}{n}\right|  = 1 + \frac{1}{2} + \frac{1}{3} + \frac{1}{4} + \cdots%
}%
\]
\item<3->  This is a $p$-series with \alertNoH{ 3-4}{$p = \uncover<4->{1.}$}
\item<5->  Therefore $\sum_{n=1}^\infty \left|\frac{(-1)^{n-1}}{n}\right|$ is \uncover<6->{\alertNoH{ 6}{divergent.}}
\item<5->  Therefore $\sum_{n=1}^\infty \frac{(-1)^{n-1}}{n}$ is \uncover<7->{\alertNoH{ 7}{not absolutely convergent.}}
\end{itemize}
\end{example}
\end{frame}
% end module absolute-convergence-ex2
