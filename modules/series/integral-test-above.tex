% begin module integral-test-above
\begin{frame}
\begin{columns}
\column{.57\textwidth}
\[
\alertNoH{15}{\sum_{n=1}^\infty \alertNoH{5,11}{\frac{1}{n^2}} =} \alertNoH{6-10,15}{\frac{1}{1^2}} \alertNoH{ 7-10,14,15}{+\alertNoH{14}{\frac{1}{2^2}}} \alertNoH{ 8-10, 14-15}{+ \frac{1}{3^2}} \alertNoH{9-10,14-15}{+ \frac{1}{4^2}} \alertNoH{10,14-15}{+ \cdots} 
\]
\begin{itemize}
\item<2->  Use a computer to calculate partial sums.
\item<3->  Appears to be converging.
\item<4->  How do we prove it?
\item<5->  Use $\alertNoH{11}{f(x) = \frac{1}{x^2}}$.
\end{itemize}
\begin{center}
\uncover<5->{%
\psset{xunit=1.2cm, yunit=1.2cm}
\begin{pspicture}(-0.3,-0.8)(5.4,1.2)%
\tiny%
\fcBoundingBox{-0.3}{-0.8}{5.4}{1.2}%
\fcLabels{5.5}{1.5}%
\newcommand{\oneBlock}[2]{%
\pstVerb{1 dict begin /x ####1\space def}%
\psline*[linecolor=####2](! x 0)(!x 1 x x mul div)(!x 1 sub 1 x x mul div)(! x 1 sub 0)(! x 0)%
\psline[linecolor=brown](! x 0)(!x 1 x x mul div)(!x 1 sub 1 x x mul div)(! x 1 sub 0)(! x 0)%
\pstVerb{end}%
}%
\newcommand{\oneBlockLabel}[2]{%
\pstVerb{1 dict begin /x ####1\space def}%
\psline[arrows=->, linewidth=0.5pt](! x 0.5 sub -0.2)(! x 0.5 sub 0.5 x x mul div)%
\rput[t](! x 0.5 sub -0.3 ){####2}%
\pstVerb{end}%
}%
\uncover<handout:1,4|6-11,15->{\oneBlock{1}{orange}}%
\uncover<handout:1|6-11>{\oneBlockLabel{1}{$\alertNoH{6}{a_1=1\vphantom{\frac{1}{4}}}$}}%
\uncover<handout:3|14>{\oneBlock{1}{gray}}%
\uncover<handout:1,3,4|7-11,14-> {\oneBlock{2}{orange}}%
\uncover<handout:1|7-11>{\oneBlockLabel{2}{$\alertNoH{7}{a_2=\frac{1}{4}}$}}%
\uncover<handout:1,3,4|8-11,14->{\oneBlock{3}{orange}}%
\uncover<handout:1|8-11> {\oneBlockLabel{3}{$\alertNoH{8}{a_3=\frac{1}{9}}$}}%
\uncover<handout:1,3,4|9-11,14-> {\oneBlock{4}{orange}}%
\uncover<handout:1|9-11> {\oneBlockLabel{4}{$\alertNoH{9}{a_4=\frac{1}{16}}$}}%
\uncover<handout:1,3,4|10-11,13->{\oneBlock{5}{orange}}%
\uncover<handout:1|10-11>{\oneBlockLabel{5}{$\alertNoH{10}{a_5=\frac{1}{25}}$}}%
\uncover<handout:2|12,13>{%
\pscustom*[linecolor=orange]{
\psplot{1}{5.2}{1 x x mul div}
\psline(5.2, 0)(1,0)(1,1)
}%
}%
\uncover<handout:2,3,4|13->{%
\rput[t](2.6, -0.3){\alertNoH{13,14,15}{finite area}}%
\psline[arrows=->, linewidth=0.5pt](2.6, -0.2)(2.6, 0.1)%
}%
\psplot[linecolor=\fcColorGraph]{1}{5.2}{1 x x mul div}%
\rput(2, 0.8){$y=\frac{1}{x^2}$}%
\fcXTickWithLabel{1}{$1$}%
\fcXTickWithLabel{2}{$2$}%
\fcXTickWithLabel{3}{$3$}%
\fcXTickWithLabel{4}{$4$}%
\fcXTickWithLabel{5}{$5$}%
\fcAxesStandardNoFrame{-0.3}{-0.3}{5.5}{1.5}%
\end{pspicture}
} %

\end{center}
\column{.43\textwidth}
\uncover<2->{%
$
\begin{array}{|r@{\ }|c|}
\hline
n & s_n = \sum_{i=1}^n \frac{1}{i^2}\\
\hline
     5 & 1.4636 \\
    10 & 1.5498 \\
    50 & 1.6251 \\
   100 & 1.6350 \\
   500 & 1.6429 \\
  1000 & 1.6439 \\
  5000 & 1.6447 \\
\hline
\end{array}
$
}%
\begin{itemize}
\item<6->  $\frac{1}{1^2}$ is the area of a rectangle.
\item<7->  So is $\frac{1}{2^2} = \frac{1}{4}$.
\item<handout:2-| 11->  The improper integral \alertNoH{12,13 }{$ \int_1^\infty \alertNoH{11}{\frac{1}{x^2}} \diff x$ is}  \fcAnswerUncover{12}{13}{convergent.}
\item<handout:4-| 15->  Therefore $\alertNoH{14}{ \sum_{n=1}^\infty \frac{1}{n^2}}$ is convergent.
\end{itemize}
\end{columns}
\end{frame}
% end module integral-test-above
