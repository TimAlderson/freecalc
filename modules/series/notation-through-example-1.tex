\begin{frame}
\vskip -0.1cm
\begin{example}[The $\dots$ and $\sum$ notations for series]
Let \alertNoH{3,4}{$A$ be the sum of the positive even integers between 2 and 124}. \uncover<2->{Write $A$ \alertNoH{3,4}{using the $\dots$ notation} and using the $\sum$ notation.}

\vskip -0.2cm
\[
\begin{array}{rcl}
\uncover<3->{\alertNoH{3,4}{A}&\alertNoH{3,4}{=}&\displaystyle \fcAnswer{4}{\alertNoH{6,27}{2+4+6+\dots+\alertNoH{28}{124}}}} \\
\uncover<7->{&=& \alertNoH{9}{2} + \alertNoH{10}{4}+\alertNoH{11}{6}+\dots + \alertNoH{7,8,12}{2n}+\dots +\alertNoH{ 13}{124}} \\
\uncover<9->{&=& \alertNoH{9,17}{2\cdot \alertNoH{18,19,21}{ 1}} +\alertNoH{10,17}{2\cdot \alertNoH{18,22}{2} }+\alertNoH{11,17}{2\cdot \alertNoH{18,23}{3}}+ \dots+\alertNoH{12,17}{ 2\cdot \alertNoH{18,24}{n} }+\dots +\alertNoH{13,17}{2\cdot \alertNoH{ 18,20,25 }{62} } } \\
\uncover<8->{&\uncover<14->{=}& \displaystyle  \uncover<14->{ \only<handout:0 | 16>{\color{red}} \sum \limits_{ {\color{black}  \alertNoH{15}{n} =\alertNoH{19,21}{1} }}^{ {\color{black} \alertNoH{20,25,28}{62}}} }\color{black} \alertNoH{8-13,17, 27,28}{ 2 \alertNoH{15,18,21-25}{n}} \quad .} 
\end{array}
\]

\end{example}
\begin{itemize}
\only<handout:1|1-14>{ 
\item<5-> We aim to introduce the $\sum $ notation for series via this example.

\item<6-> The $\dots$ notation is informal but easier to read.
\item<7-> If the $\dots $ are too ambiguous, we should include the general term.
\item<8-> To make it clearer we should rewrite all elements in the \alertNoH{8-12}{pattern of the general term}.
\item<14-> If that is still ambiguous we should switch to the completely unambiguous $\sum$ notation. 
}
\only<handout:2|15-26>{
\item<15-> The number $\alertNoH{14}{n}$ is the \alertNoH{14}{index (counter)} of the sum.
\item<16-> \alertNoH{16}{$\sum$ tells us to add} \alertNoH{17}{several copies of the summed term}, \alertNoH{18}{where in each term the index is replaced by a concrete value}.
\item<19-> The values taken by the index are determined by the \alertNoH{19,20}{boundaries of summation}. 
\item<21-> The index varies over all integers starting with the lower boundary and ending with upper boundary. 
\item<26-> In programming, what objects are similar to $\sum$?
}
\only<handout:3|27-29>{
\item<27-> To go from $\sum$ to $\dots$ notation: substitute few values for the index. \alertNoH{28}{Make sure to include the last value.}
\item<28-> To go from $\dots$ to $\sum $ notation: 
\begin{itemize}
\item<28-> figure out a pattern for the general term just as with sequences;
\item<29-> select first and last index so that your general term formula reproduces the first and last terms of the sequence.
\end{itemize}
}
\only<handout:4|30->{
\item Bear in mind the $\dots$ notation is informal. 
\begin{itemize}
\item<31-> There are infinitely many formulas that fit any single pattern.
\item<32-> Thus it is acceptable to use the $\dots$ notation only when we believe there is a single completely obvious pattern that will be recognized by every one.
\item<33-> The pattern should be obvious not only to us, but also to our potential readers.
\item<34-> If in doubt or seeking complete rigor we should use the $\sum$ notation. 
\end{itemize}  

}
\end{itemize}

\vskip 4cm
\end{frame}