% begin module formal-series-def
\begin{frame}
\frametitle{Formal Series}
\begin{definition}[Formal Series]
A \alertNoH{2}{formal series is a \alertNoH{3}{list of numbers}} \alertNoH{3}{delimited by the plus sign}.
\vskip -0.1cm
\[
a_1 \alertNoH{3,4}{+} a_2 \alertNoH{3,4}{+} a_3\alertNoH{3,4}{+} a_4\alertNoH{3,4}{+} \cdots \alertNoH{3,4}{+} a_n \alertNoH{3,4}{+} \cdots 
\]
\end{definition}
\begin{itemize}
\item<2-> Recall a \alertNoH{2}{sequence is a \alertNoH{3}{list of numbers}}. 
\vskip -0.1cm
\[a_1\alertNoH{3}{,}\quad  a_2\alertNoH{3}{,}\quad  a_3\alertNoH{3}{,}\quad  a_4\alertNoH{3}{,} \quad  \dots\alertNoH{3}{,} \quad  a_n\alertNoH{3}{,}\quad \dots\]
\item<4-> \alertNoH{4}{The $+$ sign} indicates our \alertNoH{4}{intention to attempt to sum} the elements of the formal series.
\item<5-> Except for the indication of that intention, formal series and sequences are essentially synonymous. 
\item<6-> The \alertNoH{6}{sum of a finite sequence}/finite formal series is studied in the subject of elementary \alertNoH{6}{arithmetics}.
\item<7-> \alertNoH{7}{The sum, if convergent, of an infinite sequence}/infinite formal series \alertNoH{7}{will be defined} in the following slides.
\end{itemize}

\end{frame}
% end module formal-series-def
