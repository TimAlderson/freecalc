% begin module series-geometric-ex4
\begin{frame}
\begin{example}
Write the number $2.3\overline{17} = 2.3171717\ldots$ as a quotient of integers.
\uncover<2->{%
\abovedisplayskip=0pt
\belowdisplayskip=0pt
\[
\alertNoH{ 3}{2.3}%
\alertNoH{ 4}{17}%
\alertNoH{ 5}{17}%
\alertNoH{ 6}{17}%
\ldots =
\alertNoH{ 3}{2.3}%
+ \alertNoH{ 7-11}{\alertNoH{ 4}{\frac{17}{10^3}}%
+ \alertNoH{ 5}{\frac{17}{10^5}}%
+ \alertNoH{ 6}{\frac{17}{10^7}}%
+ \cdots}%
\]
}%
\begin{itemize}
\item<7->  After the first term, we have a geometric series.
\item<8->  \alertNoH{ 8-9,12-13}{$a =$ \uncover<9->{$\frac{17}{10^3}$}} and \alertNoH{ 10-11,14-15}{$r =$ \uncover<11->{$\frac{1}{10^2}$.}}
\end{itemize}
\begin{eqnarray*}
\uncover<12->{%
2.3171717\ldots%
}%
& \uncover<12->{ = } &%
\uncover<12->{%
2.3 + \frac{\uncover<13->{\alertNoH{ 13}{\frac{17}{10^3}}}}{1- \uncover<15->{\alertNoH{ 15}{\frac{1}{10^2}}}}%
}%
 \uncover<16->{ = } %
\uncover<16->{%
2.3 + \frac{\uncover<13->{\alertNoH{ 13}{\frac{17}{1000}}}}{\uncover<15->{\alertNoH{ 15}{\frac{99}{100}}}}%
}\\%
& \uncover<17->{ = } &%
\uncover<17->{%
\frac{23}{10} + \frac{17}{990}%
}%
 \uncover<18->{ = } %
\uncover<18->{%
\frac{1147}{495}%
}%
\end{eqnarray*}
\end{example}
\end{frame}
% end module series-geometric-ex4
