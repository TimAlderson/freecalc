% begin module diff-eq-separable-solution
\begin{frame}
\[
\frac{\diff y}{\diff x} = \frac{g(x)}{h(y)}.
\]
\begin{itemize}
\item<2->  To solve, write this in differential form:
\uncover<2->{%
\abovedisplayskip=0pt
\belowdisplayshortskip=0pt
\belowdisplayskip=0pt
\[
h(y) \diff y = g(x)\diff x
\]
}%
\item<3->  Now integrate:
\uncover<3->{%
\abovedisplayskip=0pt
\belowdisplayshortskip=0pt
\belowdisplayskip=0pt
\[
\int h(y) \diff y = \int g(x)\diff x
\]
}%
\item<4->  This defines $y$ implicitly as a function of $x$.
\item<5->  Sometimes we might be able to solve explicitly for $y$ in terms of $x$.
\end{itemize}
\end{frame}

\begin{frame}
Why does this process yield a function that satisfies the original differential equation?  Suppose that $\int h(y) \diff y = \int g(x) \diff x$.  Then we will use the Chain Rule to show that $y$ satisfies the original equation.
\begin{eqnarray*}
\int h(y) \diff y & = & \int g(x) \diff x\\
\uncover<2->{%
\frac{\diff}{\diff x}\left( \int h(y)\diff y\right)%
}%
& \uncover<2->{ = } &%
\uncover<2->{%
\frac{\diff}{\diff x}\left( \int g(x)\diff x\right)%
}\\%
\uncover<3->{%
\frac{\diff}{\diff y}\left( \int h(y)\diff y\right)\frac{\diff  y}{\diff x}%
}%
& \uncover<3->{ = } &%
\uncover<3->{%
\frac{\diff}{\diff x}\left( \int g(x)\diff x\right)%
}\\%
\uncover<4->{%
h(y) \frac{\diff y}{\diff x}%
}%
& \uncover<4->{ = } &%
\uncover<4->{%
g(x)%
}\\%
\frac{\diff y}{\diff x} & = & \frac{g(x)}{h(y)}
\end{eqnarray*}
\end{frame}
% end module diff-eq-separable-solution
