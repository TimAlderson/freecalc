\begin{frame}
\begin{emptyTheorem}[Elementary Computer algorithm for sketching graphs]
\alertNoH{1}{\alertNoH{14}{Let $H$-continuous}; is there simple algorithm to sketch $\alertNoH{5}{H(x,y)=0}$?} \uncover<2->{\alertNoH{2}{Yes.}}
\end{emptyTheorem}
\begin{columns}
\column{0.3\textwidth}
\only<handout:1-6|-27>{%
\psset{xunit=0.8cm, yunit=0.8cm}
\begin{pspicture}(-1,-1)(1,1)
\tiny
\fcBoundingBox{-1.8}{-1.8}{3}{3}
\newcommand{\theFun}{x x mul 2 y y mul mul add 1 sub}
\fcAxesStandard{-1.6}{-1.6}{1.6}{1.6}
\uncover<7-18>{\fcImplicitIId[linestyle=dashed, dashes={[1 1] 0}, showGridImplicitIId=true]{-1.6}{-1.6}{2}{2}{1.6}{1.6}{-1} }
\uncover<handout:1|19>{\fcImplicitIId[linestyle=solid, linecolor=red, showGridImplicitIId=true]{-1.6}{-1.6}{2}{2}{1.6}{1.6}{\theFun} }
\uncover<handout:2|20>{\fcImplicitIId[linestyle=solid, linecolor=red, showGridImplicitIId=true]{-1.6}{-1.6}{4}{4}{0.8}{0.8}{\theFun} }
\uncover<handout:3|21>{\fcImplicitIId[linestyle=solid, linecolor=red, showGridImplicitIId=true]{-1.6}{-1.6}{8}{8}{0.4}{0.4}{\theFun} }
\uncover<handout:4|22>{\fcImplicitIId[linestyle=solid, linecolor=red, showGridImplicitIId=false]{-1.6}{-1.6}{16}{16}{0.2}{0.2}{\theFun} }
\uncover<handout:0|23>{\fcImplicitIId[linestyle=solid, linecolor=red, showGridImplicitIId=false]{-1.6}{-1.6}{32}{32}{0.1}{0.1}{\theFun} }
\uncover<handout:0|24>{\fcImplicitIId[linestyle=solid, linecolor=red, showGridImplicitIId=false]{-1.6}{-1.6}{64}{64}{0.05}{0.05}{\theFun} }
\uncover<handout:0|25>{\fcImplicitIId[linestyle=solid, linecolor=red, showGridImplicitIId=false]{-1.6}{-1.6}{128}{128}{0.025}{0.025}{\theFun} }
\uncover<handout:5|26>{\fcImplicitIId[linestyle=solid, linecolor=red, showGridImplicitIId=false]{-1.6}{-1.6}{400}{400}{3.2 400 div }{3.2 400 div}{\theFun} }
\uncover<handout:6|27>{\fcImplicitIId[linestyle=solid, linecolor=red, showGridImplicitIId=false]{-1.6}{-1.6}{1000}{1000}{3.2 1000 div }{3.2 1000 div}{\theFun} }
\uncover<handout:0|6>{
\psline[linestyle=dotted](-1.6,-1.6)(1.6,-1.6)(1.6,1.6)(-1.6,1.6)(-1.6,-1.6)
}
\uncover<9-18>{
\fcFullDot{1.6}{1.6}
\fcFullDot{0}{0}
\rput[lt](1.6,1.5){$\begin{array}{l}\alertNoH{9}{P(x_P, y_P)}\\\alertNoH{9}{=(1.6, 1.6)}\end{array} $}
\rput[lt](0,-0.1){$\begin{array}{l}\alertNoH{10}{ Q(x_Q, y_Q)}\\ \alertNoH{10}{=(0,0)}\end{array}$}
}
\uncover<11-18>{\rput[lb](1.6,1.65){$\begin{array}{l}\alertNoH{11,13}{H(1.6,1.6)}\\ \alertNoH{11}{=6.68}\uncover<13->{\alertNoH{13}{ >0}}\end{array} $}}

\uncover<12-18>{\rput[br](0,0){$\begin{array}{l}\alertNoH{12,13}{H(0,0)} \\\alertNoH{12}{=-1}\uncover<13->{\alertNoH{13}{<0}}\end{array}$}}
\uncover<handout:0|14>{\psline[linecolor=red, linewidth=1.5pt](0,0)(1.6,1.6)}
\uncover<16-19>{\fcFullDot{0.8}{0.8}}
\uncover<17-19>{\fcFullDot{0.8}{0}}
\uncover<18>{\psline[linecolor=red](0.8, 0)(0.8, 0.8)}
\uncover<handout:1|19>{
\fcFullDot{0}{-0.8}
\fcFullDot{-0.8}{-0.8}
\fcFullDot{-0.8}{0}
\fcFullDot{0}{0.8}
}
\end{pspicture}
\uncover<3-27>{
We illustrate the algorithm for:
$
\begin{array}{@{}r@{}c@{}l@{}l@{}|l}
\uncover<3->{x^{2}+2y^2&\alertNoH{0}{=}&\alertNoH{4}{1}}\\
\uncover<4->{\alertNoH{5}{x^{2}+2y^2\alertNoH{4}{-1}}&\alertNoH{0}{=}&0}\\
\uncover<5->{\text{Set }\alertNoH{5,11,12}{H(x,y)}&\alertNoH{5,11,12}{=}&\alertNoH{5,11,12}{x^{2}+2y^{2}-1}}
\end{array}
$
}}%only

\only<handout:7|28-39>{
\psset{xunit=0.2cm, yunit=0.2cm}
\begin{pspicture}(-1,-1)(1,1)
\tiny
\fcBoundingBox{-7.2}{-7.2}{12}{12}
\newcommand{\theFun}{ y y mul y y mul 3 sub mul x x mul x x mul 5 sub mul sub}
\fcAxesStandard{-6}{-6}{6}{6}
\uncover<handout:0|28,29>{
\fcImplicitIId[linestyle=solid, linecolor=red, showGridImplicitIId=true]{-6}{-6}{4}{4}{3}{3}{-1} 
}
\uncover<handout:0|30>{
\fcImplicitIId[linestyle=solid, linecolor=red, showGridImplicitIId=true]{-6}{-6}{4}{4}{3}{3}{\theFun} 
}
\uncover<handout:0|31>{
\fcImplicitIId[linestyle=solid, linecolor=red, showGridImplicitIId=true]{-6}{-6}{6}{6}{2}{2}{\theFun} 
}
\uncover<handout:0|32>{
\fcImplicitIId[linestyle=solid, linecolor=red, showGridImplicitIId=true]{-6}{-6}{8}{8}{1.5}{1.5}{\theFun} 
}
\uncover<handout:0|33>{
\fcImplicitIId[linestyle=solid, linecolor=red, showGridImplicitIId=true]{-6}{-6}{12}{12}{1}{1}{\theFun} 
}
\uncover<handout:0|34>{
\fcImplicitIId[linestyle=solid, linecolor=red, showGridImplicitIId=true]{-6}{-6}{24}{24}{0.5}{0.5}{\theFun} 
}
\uncover<handout:0|35>{
\fcImplicitIId[linestyle=solid, linecolor=red, showGridImplicitIId=false]{-6}{-6}{48}{48}{0.25}{0.25}{\theFun} 
}
\uncover<handout:0|36>{
\fcImplicitIId[linestyle=solid, linecolor=red, showGridImplicitIId=false]{-6}{-6}{100}{100}{0.12}{0.12}{\theFun} 
}
\uncover<handout:0|37>{
\fcImplicitIId[linestyle=solid, linecolor=red, showGridImplicitIId=false]{-6}{-6}{200}{200}{0.06}{0.06}{\theFun} 
}
\uncover<handout:0|38>{
\fcImplicitIId[linestyle=solid, linecolor=red, showGridImplicitIId=false]{-6}{-6}{400}{400}{0.03}{0.03}{\theFun} 
}
\uncover<handout:7|39>{
\fcImplicitIId[linestyle=solid, linecolor=red, showGridImplicitIId=false]{-6}{-6}{800}{800}{0.015}{0.015}{\theFun} 
}
\end{pspicture}
Illustrate the algorithm for:
$\begin{array}{@{}r@{}c@{}l@{}l@{}|l}
\alertNoH{28}{y^{2}(y^{2}-3)}&\alertNoH{28}{=}&\alertNoH{28}{x^2(x^2-5)}\\
\alertNoH{29}{H(x,y)}&\alertNoH{29}{=}&\alertNoH{29}{y^{2}(y^{2}-3)}\\
&&\alertNoH{29}{-x^2(x^2-5)}
\end{array}
$
}
\column{0.7\textwidth}
\begin{itemize}
\item<6-> Elementary algorithm: fix  large rectangle. 
\item<7-> Split the grid in triangular mesh. One strategy to do that is shown.
\item<8-> \alertNoH{19}{For each triangle:}
\begin{itemize}
\item<9-> Fix two corners $\alertNoH{9}{P(x_P, y_P)}$ and $\alertNoH{10}{Q(x_Q, y_Q)}$. 
\item<11-> If $\alertNoH{11}{H(x_P, y_P)}$ and $\alertNoH{12}{H(x_Q, y_Q)}$ \alertNoH{13}{have different sign} \alertNoH{14}{then $H$ must become zero somewhere on the segment between $P$ and $Q$}. 
\item<15-> Select a point between $P$ and $Q$ and ``guess'' that $H$ is zero there. 
\begin{itemize}
\item<16-> In our implementation, we select the midpoint (i.e., $\frac{1}{2}P+\frac{1}{2}Q$).
\item<17-> Connect the selected pts. for each triangle.
\end{itemize}
\item<20-> Repeat for ever finer grid.
\end{itemize}

\end{itemize}
\end{columns}

\end{frame}