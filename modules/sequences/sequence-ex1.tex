% begin module sequence-ex1
\begin{frame}
\frametitle{Sequences via formulas}
\begin{itemize}
\item Sequences can be defined by presenting a formula to obtain the $n^{th}$ term $a_n$ as a function of the index $n$.
\item There are two frequently used notations illustrated below.
\item In addition, there is an informal but frequently used notation: list few terms of the sequence and let the reader guess the formula. 
\end{itemize}
\vskip -0.15cm
\begin{example}
$
\begin{array}{lll}
\left( \frac{n}{n+1}\right) &%
a_n = \frac{n}{n+1} &%
\left( \frac{1}{2}, \frac{2}{3}, \frac{3}{4}, \frac{4}{5}, \ldots \right) \\%
&&\\%
\left( \frac{(-1)^n(n+1)}{3^n}\right) &%
a_n = \frac{(-1)^n(n+1)}{3^n} &%
\left( \frac{-2}{3}, \frac{3}{9}, \frac{-4}{27}, \frac{5}{81}, \ldots \right)\\%
&&\\%
\left( \sqrt{n-3}\right)_{n=3}^\infty &%
a_n = \sqrt{n-3}, n\geq 3&%
\left( 0, 1, \sqrt{2}, \sqrt{3}, \ldots \right)\\%
&&\\%
\left( \cos \frac{n\pi}{6}\right)_{n=0}^\infty &%
a_n = \cos \left(\frac{n\pi}{6}\right), n\geq 0&%
\left( 1, \frac{\sqrt{3}}{2}, \frac{1}{2}, 0, \ldots \right)\\%
\end{array}
$
\end{example}
\end{frame}
% end module sequence-ex1
