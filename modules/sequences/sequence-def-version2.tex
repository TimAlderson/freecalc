% begin module sequence-def
\begin{frame}
%\frametitle{Sequences}
\begin{definition}[Sequence \uncover<2->{indexed by the integers}]
A sequence is a list of numbers \uncover<2->{indexed by consecutive integers bounded below and} written in a definite order

\uncover<2->{\hfil\hfil$a_1, a_2, a_3, a_4, \ldots , a_n , \ldots\quad .$}
\end{definition}
\begin{itemize}
\item<2-> In our course/lectures we assume all sequences are indexed by consecutive integers.
\item<3-> {Unless stated/implied otherwise }
\begin{itemize}
\item<3-> \alertNoH{3}{We assume the first index is 1.}
\item<4-> Under above assumption $a_1$ is called the first term, $a_2$ is called the second term, and $a_n$ is the $n$th term.
\end{itemize}
\item<5-> We often denote the sequence of elements $a_1, a_2, \dots$ by 
\[
\begin{array}{lcl}
\{ a_n\} &\textrm{ and more precisely }& \{ a_n\}_{n = 1}^\infty\\
&\text{ or by }&\\
\left(a_n\right) &\textrm{ and more precisely }& \left( a_n\right)_{n = 1}^\infty
\end{array}
\]
\item<6-> The use of $\{\}$ versus $()$ differs between authors and instructors.
 

 
\end{itemize}

\vskip 5cm
	
\end{frame}

\begin{frame}
\begin{definition}[Sequence indexed by the integers]
A sequence is a list of numbers indexed by consecutive integers bounded below and written in a definite order

\hfil\hfil$a_1, a_2, a_3, a_4, \ldots , a_n , \ldots\quad .$
\end{definition}

\begin{itemize}
\item To indicate a sequence labeled so the first index is not $1$ write:
\[
\begin{array}{rcl}
\left(a_n\right)_{n=0}^{\infty}&\text{for}& a_0, a_1, a_2,\dots \\
\left(a_n\right)_{n=2}^{\infty}&\text{for}& a_2, a_3, a_4,\dots \\
\left(a_n\right)_{n=-1}^{\infty}&\text{for}& a_{-1}, a_0, a_1,\dots
\end{array}
\]
\begin{definition}
A sequence is finite if it has a finite number of elements.
\end{definition}
\item To indicate a sequence is finite either write all elements of the sequence or use indices as shown below.
\[
\begin{array}{rcl}
&&1,2,3,4,5,6,7,8,9,10\\
\left(a_n\right)_{n=1}^{5}&\text{for}& a_1,a_2,a_3,a_4,a_5 
\end{array}
\]
\end{itemize}

\vskip 5cm

\end{frame}
% end module sequence-def
