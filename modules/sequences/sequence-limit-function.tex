% begin module sequence-limit-function
\begin{frame}
If you compare the definition of the limit of a sequence with the definition of the infinite limit of a function, you'll see that the only difference between
\[
\lim_{n\to\infty} a_n = L \qquad \textrm{and} \qquad \lim_{x\to\infty}f(x) = L
\]
is that $n$ is required to be an integer.
%\begin{center}breaks in ubuntu

\hfil \hfil\psset{xunit=0.5cm,yunit=0.5cm}
\begin{pspicture}(-0.5,-0.5)(14,4)
\tiny
\fcAxesStandard{-1}{-1}{14}{6}
\rput[b](0,6.2){$a$}
\rput[l](14.2, 0){$n$}
\multido{\na=0+1}{36}{%
\pstVerb{2 dict begin /na \na\space def /x na 0.4 mul def}%
\fcFullDot[scale=0.5, linecolor=red]{x }{x x mul 3 mul x 4 mul add x x mul 3 mul 1 add div x 90 mul sin 3 mul x 1 add div add }%
\pstVerb{end}%
}%
\uncover<2->{\psplot[plotpoints=400, linecolor=\fcColorGraph]{0}{14}{x x mul 3 mul x 4 mul add x x mul 3 mul 1 add div x 90 mul sin 3 mul x 1 add div add }}
\psline[linewidth =0.4pt, linestyle=dashed, linecolor=blue](0, 1)(14, 1)
\rput[r](-0.1, 1){$L$}
\end{pspicture}
%\end{center}

\uncover<3->{%
\begin{theorem}
If $\lim_{x\to\infty}f(x) = L$ and $f(n) = a_n$ for all integers $n$, then $\lim_{n\to\infty} a_n = L$.
\end{theorem}
}%
\end{frame}
% end module sequence-limit-function
