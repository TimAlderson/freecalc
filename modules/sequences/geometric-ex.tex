% begin module geometric-ex
\begin{frame}
\begin{example}[Arithmetic and geometric]
{\renewcommand{\arraystretch}{1.2}
\begin{tabular}{l|c|c|c|c|c}
& \alert<handout:0| 2-3,10-11,18-19,28-29,36-37>{Arithmetic/} &  &  &  & \\
Sequence & \alert<handout:0| 2-3,10-11,18-19,28-29,36-37>{geometric} & \alert<handout:0| 12-13,20-21>{Diff.} & \alert<handout:0| 4-5,22-23,30-31>{Ratio} & \alert<handout:0| 6-7,14-15,24-25,32-33,38-39>{$a_1$} & \alert<handout:0| 8-9,16-17,26-27,34-35,40-41>{$a_n$} \\
\hline
\alert<handout:0| 2-9>{$\frac{2}{3},\frac{4}{9},\frac{8}{27},\frac{16}{81},\ldots$} & \uncover<3-| handout:0>{\alert<handout:0| 3>{geometric}} & \uncover<3-| handout:0>{\alert<handout:0| 3>{---}} & \uncover<5-| handout:0>{\alert<5| handout:0>{$\frac{2}{3}$}} & \uncover<7-| handout:0>{\alert<handout:0| 7>{$\frac{2}{3}$}} & \uncover<9-| handout:0>{\alert<handout:0| 9>{$\big(\frac{2}{3}\big)^{n}\uncover<handout:0| 42->{\alert<handout:0| 42>{ = \frac{2}{3}\big(\frac{2}{3}\big)^{n-1}}}$}} \\
\hline
\alert<handout:0| 10-17>{$7,3,-1,-5,\ldots$} & \uncover<11-| handout:0>{\alert<handout:0| 11>{arithmetic}} & \uncover<13-| handout:0>{\alert<handout:0| 13>{$-4$}} & \uncover<11-| handout:0>{\alert<11| handout:0>{---}} & \uncover<15-| handout:0>{\alert<handout:0| 15>{$7$}} & \uncover<17-| handout:0>{\alert<handout:0| 17>{$7-4n$}} \\
\hline
\alert<handout:0| 18-27>{$4,4,4,4,\ldots$} & \uncover<19-| handout:0>{\alert<handout:0| 19>{both}} & \uncover<21-| handout:0>{\alert<handout:0| 21>{$0$}} & \uncover<23-| handout:0>{\alert<23| handout:0>{$1$}} & \uncover<25-| handout:0>{\alert<handout:0| 25>{$4$}} & \uncover<27-| handout:0>{\alert<handout:0| 27>{$4\uncover<handout:0| 42->{\alert<handout:0| 42>{ = 4(1)^{n-1}}}$}} \\
\hline
\alert<handout:0| 28-35>{$\pi,-\pi^2,\pi^3,-\pi^4,\ldots$} & \uncover<29-| handout:0>{\alert<handout:0| 29>{geometric}} & \uncover<29-| handout:0>{\alert<handout:0| 29>{---}} & \uncover<31-| handout:0>{\alert<31| handout:0>{$-\pi$}} & \uncover<33-| handout:0>{\alert<handout:0| 33>{$\pi$}} & \uncover<35-| handout:0>{\alert<handout:0| 35>{$\pi(-\pi)^{n-1}$}} \\
\hline
\alert<handout:0| 36-41>{$1,1,2,2,3,3,\ldots$} & \uncover<37-| handout:0>{\alert<handout:0| 37>{neither}} & \uncover<37-| handout:0>{\alert<handout:0| 37>{---}} & \uncover<37-| handout:0>{\alert<37| handout:0>{---}} & \uncover<39-| handout:0>{\alert<handout:0| 39>{$1$}} & \uncover<41-| handout:0>{\alert<handout:0| 41>{$\lceil\frac{n}{2}\rceil$}} \\
\end{tabular}
}%
\end{example}

\uncover<42->{%
If a geometric sequence has ratio $r$, then the $n$th term has formula 
\[
a_n = a_1r^{n-1}.
\]
where $a_1$ is the first term.  
}%
\end{frame}
% end module geometric-ex
