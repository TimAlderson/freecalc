% begin module sequence-ex2
\begin{frame}
Not all sequences can be represented by a simple formula.
\begin{example}[Sequences without a simple formula]
\begin{enumerate}
\item  Consider the sequence $( p_n)$, where $p_n$ is the population of the world as of January 1 of year $n$.  This has no simple formula.
\item<2->  Let $a_n$ be the $n$th digit of the number $e$.  Then the first few terms of $( a_n)$ are
\uncover<2->{%
\[
 7, 1, 8, 2, 8, 1, 8, 2, 8, 4, 5, \ldots 
\]
}%
\item<3->  The Fibonacci sequence $( f_n)$ is defined recursively by
\[
f_1 = 1 \qquad f_2 = 1 \qquad f_n = f_{n-1} + f_{n-2}, \quad n\geq 3
\]
\uncover<4->{%
The first few terms are
\abovedisplayskip=1pt
\belowdisplayskip=1pt
\[
\alert<handout:0| 4-5>{1},%
\alert<handout:0| 4-7>{1},%
\uncover<5-| handout:0>{\alert<handout:0| 5-9>{{2},}}%
\uncover<7-| handout:0>{\alert<handout:0| 7-11>{{3},}}%
\uncover<9-| handout:0>{\alert<handout:0| 9-13>{{5},}}%
\uncover<11-| handout:0>{\alert<handout:0| 11-13>{{8},}}%
\uncover<13-| handout:0>{\alert<handout:0| 13>{{13},\ldots }}%
\]
}%
\uncover<14->{%
The Fibonacci sequence can be described by a formula, but not a simple one.%
}%
\end{enumerate}
\end{example}
\end{frame}
% end module sequence-ex2
