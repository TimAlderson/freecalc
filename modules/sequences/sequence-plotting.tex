% begin module sequence-plotting
\begin{frame}
\psset{xunit=0.75cm, yunit=0.75cm}
\begin{pspicture}(-0.5, -0.5)(15, 1.4)
\tiny
\fcAxesStandard{-0.5}{-0.5}{15.2}{1.4}
\fcYTickWithLabel{1}{$1$}
\rput[tl](15.2, -0.1){$n$}
\rput[rb]( -0.1, 1.45){$a$}
\multido{\na=1+1}{9}{%
\FPeval{\naPlusOne}{clip(\na+1)}%
\FPeval{\naPlusTwo}{clip(\na+2)}%
\only<\naPlusOne->{%
\pstVerb{1 dict begin /na \na\space def}%
\fcXTickWithLabel{na}{$\na$}%
}%
\only<\naPlusOne>{%
\rput[l](! na 0.1 add na na 1 add div 2 div ){ $a_{\na}=\frac{\na}{\naPlusOne}$}%
\psline(! na 0)(! na na na 1 add div)%
\fcFullDot[linecolor=red]{na}{na na 1 add div}%
}%
\only<\naPlusTwo->{%
\fcFullDot[scale=0.6, linecolor=red]{na}{na na 1 add div}
}%
\only<\naPlusOne->{%
\pstVerb{end}%
}%
}%
\uncover<11->{%
\multido{\na=10+1}{6}{%
\pstVerb{1 dict begin /na \na\space def}%
\fcXTickWithLabel{na}{$\na$}%
\fcFullDot[scale=0.6, linecolor=red]{na}{na na 1 add div}%
\pstVerb{end}%
}%
\psline[linestyle=dashed, linewidth = 0.4pt, linecolor=blue](0,1)(15,1)%
}%
\end{pspicture}



\psset{xunit=6cm, yunit=6cm}
\begin{pspicture}(-0.05, -0.2)(1.05, 0.2)
\tiny
\fcBoundingBox{-0.05}{-0.2}{1.05}{0.2}
\psline(-0.05,0)(1.05,0)%
\rput[t](0,-0.03){$0$}
\psline(0,-0.02)(0, 0.02)
\rput[t](1,-0.03){$1$}
\psline(1,-0.02)(1, 0.02)
\uncover<2->{
\rput[t](! 1 2 div -0.03){$a_1$}
\psline[linewidth=0.3pt](! 1 2 div -0.02)(! 1 2 div 0.02)
}
\uncover<3->{
\rput[t](! 2 3 div -0.03){$a_2$}
\psline[linewidth=0.3pt](! 2 3 div -0.02)(! 2 3 div 0.02)
}
\uncover<4->{
\rput[t](! 3 4 div -0.03){$a_3$}
\psline[linewidth=0.3pt](! 3 4 div -0.02)(! 3 4 div 0.02)
}
\multido{\na=5+1}{6}{%
\FPeval{\naMinusOne}{clip(\na-1)}%
\only<\na->{%
\psline[linewidth=0.3pt](!\na\space 1 sub \na\space div -0.02)(!\na\space 1 sub \na\space div 0.02)%
}%
\only<\na>{%
\rput[t](!\na\space 1 sub \na\space div -0.03){$a_{\naMinusOne}$}%
}%
}%
\uncover<11->{%
\multido{\na=11+1}{25}{%
\psline[linewidth=0.3pt](!\na\space 1 sub \na\space div -0.02)(!\na\space 1 sub \na\space div 0.02)%
}%
\psline*(! 35 36 div -0.02 )(! 1 -0.02 )(! 1 0.02 )(! 35 36 div 0.02 )(! 35 36 div -0.02 )
}
\end{pspicture}



\begin{itemize}
\item  The sequence $\left\{ \frac{n}{n+1}\right\}$ can be plotted on a number line or using Cartesian coordinates.
\item<12->  From the pictures, the terms in the sequence appear to approach $1$ as $n$ gets larger.
\item<13->  \alert<handout:0| 13-14>{$1 - \frac{n}{n+1} =$ \uncover<14->{$\frac{1}{n+1}$.}}
\item<15->  This can be made arbitrarily small by choosing $n$ large enough.
\item<16->  We express this by writing $\lim\limits_{n\to\infty}\frac{n}{n+1} = 1$.
\end{itemize}
\end{frame}
% end module sequence-plotting
