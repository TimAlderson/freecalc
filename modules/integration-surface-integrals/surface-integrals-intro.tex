\begin{frame}
\frametitle{Surface Integral Motivation}

\begin{columns}
\column{0.3\textwidth}
\psset{xunit=0.7cm, yunit=0.7cm}
\begin{pspicture}(-2.2,-2)(2,2)%
\tiny%
\fcBoundingBox{-2.2}{-1.4}{2.1}{2.2}%
\pstVerb{%
5 dict begin%
/theEllipsoidTop {[u v u 2 div dup mul v 1.5 div dup mul add 1 exch sub sqrt]} def%
/basePoint 2 dict begin /u 1 def /v 0 def theEllipsoidTop end def%
/tangent1 [0 1 0] def%
/tangent2 [1 0 3 sqrt 6 div -1 mul] def%
}%
\fcStartIIIdScene%
\fcSurfaceInScene[forceForeground=false, linecolor=red, iterationsU=7, iterationsV=8, colorUV={1 0.5 0.5}, colorVU={0.7 0.2 0.2} ]{ -1.4}{-1}{1.4 }{1 }{ theEllipsoidTop}{}%
\fcFinishIIIdScene%
\pstVerb{end}%
\end{pspicture}
\column{0.7\textwidth}
\begin{itemize}
\item Let $S$ be a surface in space and let $\diff S$ denote the element of surface area.
\end{itemize}
\end{columns}

\begin{itemize}
\item If $\rho$ is the density of the surface, then $\diff m = \rho \diff S$ is the element of mass, and the total mass is
\[
M = \iint_S \diff m = \iint_S \rho  \diff S \; .
\]
\item If $\fcv p$ is the pressure function - the density of force with respect to surface area - then the element of force is $\diff \fcv{F} = \diff \fcv p \diff S$. The total force exerted by pressure on the surface is then
\[
\fcv{F} = \iint_S \diff \fcv{F} = \iint_S \fcv p \diff S .
\]
\item How do we compute surface integrals?
\end{itemize}
\end{frame}