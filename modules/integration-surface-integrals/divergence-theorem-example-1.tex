\begin{frame}
\begin{example}
\begin{columns}

\column{0.4\textwidth}
\psset{xunit=0.5cm, yunit=0.5cm}
\begin{pspicture}(-1,-1)(1,1)
\tiny
\fcBoundingBox{-2}{-1.4}{3}{4.5}
\fcStartIIIdScene
\fcSurfaceInScene[colorUV=pink, iterationsV=12, iterationsU = 6, linecolor=gray ]{ 0}{0 }{2}{360}{[v cos u mul v sin u mul 4 u u mul sub]}{}
\fcVectorFieldAlongSurfaceInScene[linecolor=blue, arrows=->, iterationsV=6, iterationsU =4 ]{ 2 6 div 2 mul}{ 0}{2}{360}{[v cos u mul v sin u mul 4 u u mul sub]}{}{[v cos u mul v sin u mul 4 u u mul sub add add 4 div dup dup]}
\fcAxesIIIdInScene{3}{3}{4.5}
\fcFinishIIIdScene[fastsort=true]
\end{pspicture}

\column{0.6\textwidth}
Let $S$ be the part of the paraboloid $z=4-x^2-y^2$ above the $xy-$plane, oriented upward, and $\fcv{X} = a \fcv{i} + b \fcv{j} + c \fcv{k}$. Use the Divergence Theorem to compute
$\iint_S \fcv {X} \cdot \fcv{\diff S} \; .$
\end{columns}
The surface $S$ does not enclose a region in space. However, we add the disk $D$ of radius 2 centered at the origin in the plane $z=0$ to make it closed. $R$ orients $D$ with the downward normal, hence
\[
\begin{array}{rcl}
\iint_{S\uparrow \cup D\downarrow} \fcv{X} \cdot \fcv{\diff S} &=& \iiint_R\; \divg \fcv{X} \; \diff V = 0\; ,\\
\iint_{S\uparrow} \fcv{X} \cdot \fcv{\diff S} &=& \iint_{D\uparrow} \fcv{X} \cdot \fcv{\diff S}\quad .
\end{array}
\]

The upward normal to $D$ is $\fcv{k}$, hence $\fcv{X}\cdot \fcv{\diff S}= \fcv{X}\cdot \fcv{k} \, \diff S= c\,\diff S$. Therefore
%
$$\iint_{S} \fcv{X} \cdot \fcv{\diff S} = \iint_{D} \fcv{X} \cdot \fcv{\diff S} = \iint_D c\, \diff S = c\cdot\text{area}(D) = 4\pi c\, .$$
\end{example}
\end{frame}