% begin module function-def
\begin{frame}
\begin{columns}
\column{0.4\textwidth}
\psset{xunit=0.6cm,yunit=0.6cm}
\begin{pspicture}(-2,-3.65)(7,3.1)
\fcBoundingBox{-0.5}{-3.65}{1.1}{3.1}
\rput[r](-1.2,0){$D$}
\uncover<handout:3|6->{
\fcFullDot[scale=2, linecolor=red]{5}{2.5}
\fcFullDot[scale=2, linecolor=red]{0}{2.5}

}
\psellipse*[linecolor=cyan](0,0)(1, 3)
\fcFullDot[scale=1, linecolor=red]{0}{2.5}
\fcFullDot{0}{1.5}
\fcFullDot{0}{0.5}
\fcFullDot{0}{-0.5}
\fcFullDot{0}{-1.5}
\fcFullDot{0}{-2.5}
\uncover<handout:0|4,6>{
\fcFullDot[scale=2, linecolor=red]{0}{-0.5}
}
\uncover<handout:0|4>{
\fcFullDot[scale=2, linecolor=red]{0}{2.5}
\fcFullDot[scale=2, linecolor=red]{0}{1.5}
\fcFullDot[scale=2, linecolor=red]{0}{0.5}
\fcFullDot[scale=2, linecolor=red]{0}{-1.5}
\fcFullDot[scale=2, linecolor=red]{0}{-2.5}
}
\uncover<4->{\rput[t](0, -3.2){\alertNoH{4}
{Domain}}}

\rput[l](6.2,0){$E$}
\psellipse*[linecolor=cyan](5,0)(1, 3)
\fcFullDot[scale=1, linecolor=red]{5}{2.5}
\fcFullDot{5}{1.25}
\fcFullDot{5}{0}
\fcFullDot{5}{-1.25}
\fcFullDot{5}{-2.5}
\fcFullDot{5}{-2.5}
\uncover<handout:0|5,12>{
\fcFullDot[scale=2, linecolor=red]{5}{0}
}
\uncover<handout:0|5,6,9,11>{
\fcFullDot[scale=2, linecolor=red]{5}{2.5}
}
\uncover<handout:0|5,9>{
\fcFullDot[scale=2, linecolor=red]{5}{1.25}
\fcFullDot[scale=2, linecolor=red]{5}{-1.25}
\fcFullDot[scale=2, linecolor=red]{5}{-2.5}
}
\uncover<5->{\rput[t](5, -3.2){\alertNoH{5}{Co-domain}}}
\rput(2.5, 2.5){$f$}
\psline[linestyle=dashed]{->}(0,2.5)(5,1.25)
\psline[linestyle=dashed]{->}(0,1.5)(5,-1.25)
\psline[linestyle=dashed]{->}(0,0.5)(5,-2.5)
\psline[linestyle=dashed]{->}(0,-0.5)(5,2.5)
\psline[linestyle=dashed]{->}(0,-1.5)(5,-1.25)
\psline[linestyle=dashed]{->}(0,-2.5)(5,2.5)
\uncover<handout:3|6->{%
\rput(5.4, 2.5){~~~$f(x)$}
\rput[r](-0.3, -0.5){$x$}
}
\end{pspicture}
\column{0.6\textwidth}
\begin{itemize}
\item<10-> A function has domain $D$ $\Rightarrow$ there is exactly one arrow starting at each element of $D$.
\item<11-> An element of the co-domain can be at the tip of more than one arrow.
\item<12-> It is allowed to have an element in the co-domain without arrows pointing to it.
\end{itemize}
\end{columns}

\begin{definition}[Function]
A function $f$ is a rule that assigns to each element $x$ in a set $D$ exactly one element, called $f(x)$, in a set $E$.
\end{definition}

\only<handout:1| 2-4>{
\begin{itemize}
\item<2-> Functions are also synonymously called ``maps''.
\item<3-> In the picture above, $f$ is represented via the arrows.
\end{itemize}

\uncover<4->{
\begin{definition}[Domain]
The set $D$ in the definition of $f$ is called the domain of $f$. 
\end{definition}
}
}

\only<handout:2| 5>{
\begin{definition}[Co-domain]
The set $E$ in the definition of $f$  is called the co-domain of $f$. 
\end{definition}
}

\only<handout:3| 6-8>{
\begin{definition}[Value of $f$ at $x$]
The number $f(x)$ is called \emph{the value of $f$ at $x$} and is read ``$f$ of $x$''.
\end{definition}

\begin{itemize}
\item<7-> The value of $f$ at $x$ is also called the image of $x$ under the map $f$.
\item<8-> In the expression $f(x)$, $x$ is referred to as the \emph{argument} of $f$. 
\end{itemize}
}
\only<handout:4| 9->{
\begin{definition}[Range]
The set of all possible values taken by $f(x)$ as the element $x$ runs over elements of $D$ is called the range of $f$. 
\end{definition}
}

\vskip 10cm 



\end{frame}
% end module function-def