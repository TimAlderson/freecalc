%begin module trig-substitution-case-1a-tan
\begin{frame}
\frametitle{Trigonometric substitution $x=\tan \theta$  for $\sqrt{ x^2+1}$ }
The trigonometric substitution $ \alert<2,9>{x=\tan \theta}$, $\theta\in \left(-\frac{\pi}{2} , \frac{\pi}{2}\right) $ for $\sqrt{x^2+1}$:
\[
\begin{array}{rcll}
\displaystyle   \sqrt{ x^2+1} &=& \sqrt{ \alert<2>{\tan^2  \theta} +1}&\\
\uncover<3->{ %
&=&\displaystyle \sqrt{\sec^2 \theta} & \\
} %
\uncover<4->{ %
&=&\displaystyle   |\sec\theta|  &\\
} %
\uncover<5->{ %
&=& \displaystyle  \sec \theta& 
     \vspace*{-1cm} \begin{array}{|l}
      \text{When }\theta\in \left(-\frac{\pi}{2} , \frac{\pi}{2}\right) \text{ we have } ~ \sec \theta \geq 0\\
      \text{ so }  |\sec\theta|=\sec\theta 
      \end{array} \\
} 
\end{array}
\]


\uncover<6->{
To summarize:
\begin{definition}The trigonometric substitution $\alert<9>{ x=\tan \theta }$, $\theta\in (-\frac{\pi}{2} , \frac{\pi}{2})$ for $\sqrt{x^2+1} $ is given by:
\[
\begin{array}{rcl}
x &=&\displaystyle \tan \theta \\
\alert<7>{\sqrt{x^2+1}}&\alert<7>{=}&\displaystyle \alert<7>{\sec \theta}\\
\alert<9,10>{\diff x} &\alert<9,10>{=}&\displaystyle \uncover<10->{\alert<10>{  \sec^2 \theta} } \uncover<1-9>{\alert<9>{\textbf{?}}} \alert<9,10>{\diff \theta}\\
\alert<8>{\theta}& \alert<8>{=}& \alert<8>{\Arctan x}\quad .
\end{array}
\]
\end{definition}
}
\end{frame}
%end module trig-substitution-case-1-cot