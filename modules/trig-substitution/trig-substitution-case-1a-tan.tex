%begin module trig-substitution-case-1a-tan
\begin{frame}
\frametitle{Trigonometric substitution $x=\tan \theta$  for $\sqrt{ x^2+1}$ }
The trigonometric substitution $ \alertNoH{2,9}{x=\tan \theta}$, $\theta\in \left(-\frac{\pi}{2} , \frac{\pi}{2}\right) $ for $\sqrt{x^2+1}$:
\[
\begin{array}{rcll}
\displaystyle   \sqrt{ x^2+1} &=& \sqrt{ \alertNoH{2}{\tan^2  \theta} +1}&\\
\uncover<3->{ %
&=&\displaystyle \sqrt{\sec^2 \theta} & \\
} %
\uncover<4->{ %
&=&\displaystyle   |\sec\theta|  &\\
} %
\uncover<5->{ %
&=& \displaystyle  \sec \theta& 
     \vspace*{-1cm} \begin{array}{|l}
      \text{When }\theta\in \left(-\frac{\pi}{2} , \frac{\pi}{2}\right) \text{ we have } ~ \sec \theta \geq 0\\
      \text{ so }  |\sec\theta|=\sec\theta 
      \end{array} \\
} 
\end{array}
\]


\uncover<6->{
To summarize:
\begin{definition}The trigonometric substitution $\alertNoH{9}{ x=\tan \theta }$, $\theta\in (-\frac{\pi}{2} , \frac{\pi}{2})$ for $\sqrt{x^2+1} $ is given by:
\[
\begin{array}{rcl}
x &=&\displaystyle \tan \theta \\
\alertNoH{7}{\sqrt{x^2+1}}&\alertNoH{7}{=}&\displaystyle \alertNoH{7}{\sec \theta}\\
\alertNoH{9,10}{\diff x} &\alertNoH{9,10}{=}&\displaystyle \fcAnswerUncover{1}{10}{\sec^2 \theta} \alertNoH{9,10}{\diff \theta}\\
\alertNoH{8}{\theta}& \alertNoH{8}{=}& \alertNoH{8}{\Arctan x}\quad .
\end{array}
\]
\end{definition}
}
\end{frame}
%end module trig-substitution-case-1-cot