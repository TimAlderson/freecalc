The trigonometric substitution $x=\tan \theta$ is given by
\[
\begin{array}{rcll|l}
\displaystyle \sqrt{ x^2+1}&=&\displaystyle \sqrt{\tan^2 \theta+1}\\
&=&\displaystyle   \sqrt{ \frac{ \sin^2 \theta}{ \cos^2 \theta} +1}\\
&=&\displaystyle \sqrt{ \frac{ \sin^2\theta+\cos^2 \theta}{ \cos^2 \theta}} \\
&=& \displaystyle \sqrt{\frac{1}{\cos^2\theta}} && \begin{array}{l} \displaystyle \text{when }\theta\in \left(-\frac{\pi}{2}, \frac{\pi}{2}\right) \text{ we have }\\ ~ \cos \theta > 0\text{ and so } \sqrt{\cos^2 \theta}=\cos\theta \end{array}\\
&=&\displaystyle  \frac{1}{\cos \theta}= \sec \theta\quad .
\end{array}
\]
The differential $\diff x$ can be expressed via $\diff \theta$ from $x=\tan \theta$. The substitution $x=\tan \theta$ can be now summarized as:
\[
\begin{array}{rcl}
x&=&\displaystyle \tan \theta\\
\sqrt{x^2+1}& =& \displaystyle \frac{1}{\cos \theta}=\sec \theta\\
\diff x &=&\displaystyle \frac{\diff \theta}{\cos^2\theta}= \sec^2 \theta \diff \theta\\
\theta& =& \arctan x\quad .
\end{array}
\]