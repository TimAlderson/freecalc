\solution{\ref{problemIntegratesqrt(x^2+1)dx}. 

This problem can be solved both via the Euler substitution and by transforming to a trigonometric integral and solving the trigonometric integral on its own. We present both variants. 


\noindent\textbf{Variant I.}
We recall the Euler substitution for $ \sqrt{x^2+1 } $\refBad{\ref{eqEulerSub-case1-cot(2arctant)}}{}{  given in  \eqref{eqEulerSub-case1-cot(2arctant)}}:

$\begin{array}{rcl}
\displaystyle x&=&\displaystyle \frac{1}{2}\left(\frac{1}{t}-t\right)\\
\displaystyle \sqrt{x^2+1}&=&\displaystyle \frac{1}2\left(\frac 1 t +t\right)\\
\displaystyle \diff x&=&\displaystyle -\frac{1}{2} \left(\frac1{t^2} +1\right)\diff t\\
\displaystyle t&=&\displaystyle \sqrt{x^2+1}-x\quad .
\end{array}$

Therefore

\noindent$
\begin{array}{@{}r@{}c@{}l@{}l@{}|l}
\displaystyle\int \sqrt{(x^2+1)}\diff x&=&\displaystyle-\int  \frac14 \left(\frac1t +t\right)\left(\frac 1 {t^2} +1\right)\diff t\\
&=&\displaystyle -\frac 1 4\int \left( \frac{1}{t^3}+ 2\frac{1}t +t\right)\diff t\\
&=&\displaystyle -\frac{1}4 \left(- \frac{t^{-2}}{2}+ 2\ln|t|+ \frac{t^2}2 \right)+C\\
&=&\displaystyle\frac{1}{8}\left(t^{-2}-t^2\right) + \frac{1}{2}\ln |t|+C &&\begin{array}{l}a^2-b^2=\\(a-b)(a+b)\end{array}\\
&=&\displaystyle \frac{1}{2} \left( \underbrace {\frac{1}{2}\left(t^{-1}  -t \right) }_{=x} \right) \left(\underbrace{\frac{1}{2}\left(t^{-1}+t\right)}_{=\sqrt{x^2+1} } \right)+\frac{1}{2}\ln |t|+C\\
&=&\displaystyle \frac{1}{2} x\sqrt{x^2+1}-\frac{1}{2} \ln \left|\sqrt{x^2+1}-x\right|+C &&\text{See below}\\
&=&\displaystyle \frac{1}{2} x\sqrt{x^2+1}+\frac{1}{2} \ln \left(\sqrt{x^2+1}+x\right) +C\quad .
\end{array}
$

\noindent Our problem is solved.  

A few comments are in order. In the above expression we would have obtained a perfectly good answer if we plugged in $t=\sqrt{x^2+1}-x$ into the fourth line, however our answer would look much more complicated. Indeed, had we not used the formula $a^2-b^2=(a-b)(a+b)$ in the fourth line, the term $t^{-2}-t^{2}$ would be equal to $\frac{1}{ (\sqrt{x^2+ 1}- x)^2 }- (\sqrt{x^2+1}-x)^2$. In turn, the term $\frac{1}{ (\sqrt{ x^2+1}-x)^2}- (\sqrt{x^2+1}-x)^2$ can be simplified to $4x\sqrt{x^2+1}$ as follows. We carry out the simplifications to illustrate some of the algebraic issues arising when dealing with integrals of radicals.

\noindent$
\begin{array}{@{}r@{}c@{}l@{}}
\displaystyle t^{-2}-t^{2}&=&\displaystyle \frac{1}{(\sqrt{x^2+1}-x)^2}- (\sqrt{x^2+1}-x)^2\\
&=&\displaystyle
\frac{(\sqrt{x^2+1}+x)^2}{(\sqrt{x^2+1}-x)^2  (\sqrt{x^2+1}+x)^2 } \\
&&\displaystyle - (\sqrt{x^2+1}-x)^2 \\
&=&\displaystyle \frac{(\sqrt{x^2+1}+x)^2}{\underbrace{((\sqrt{x^2+1})^2-x^2)^2}_{=1} }- (\sqrt{x^2+1}-x)^2 \\
&=&\displaystyle 4x\sqrt{x^2+1}\quad .
\end{array}
$

Of course, the above computations are unnecessary if we use the formula $a^2-b^2=(a-b) (a+b)$ as done in the original solution. 

\noindent We note that in the last transformation we transformed $\ln \left| \sqrt{x^2+1}-x\right|$ to $\ln \left( \sqrt{ x^2+1}-x\right)$ because the quantity $\sqrt{x^2+1}-x$ is always positive. The proof of that fact we leave for the reader's exercise.

Finally, we note that as a last simplification to our solution, we used the transformation $\ln |t|= \ln \left( \sqrt{x^2+1} -x\right) = -\ln|\frac{1}{t}|= -\ln \left( \sqrt{x^2+1} +x \right)$. This is seen as follows.

\noindent $\begin{array}{rcll|l}
\displaystyle \ln |t|&=&\displaystyle  -\ln \left| \frac{1}{ t} \right| \\
&=&\displaystyle -\ln \left(\frac{1}{ \sqrt{x^2+1} - x} \right) && \text{rationalize}\\
&=&\displaystyle - \ln \left(\frac{ \left( \sqrt{x^2+1}+ x\right) }{\left( \sqrt{x^2+1} - x\right) \left(\sqrt{x^2+1 }+x\right) } \right)\\
&=& \displaystyle -\ln \left(\frac{\sqrt{x^2+1}+x}{x^2 +1 -x^2 }\right)\\
&=& \displaystyle  -\ln \left(\sqrt{x^2+1}+x\right)\quad .\\
\end{array}
$

\noindent\textbf{Variant II.} In this variant we transform to a trigonometric integral and solve it using ad-hoc methods. We recall that if we decided to solve the trigonometric integral using the standard substitution $\theta=2\arctan t$, we would arrive at the Euler substitution given in Variant I.

\noindent $
\begin{array}{@{}r@{~}c@{~}l@{}l@{}|l}
\displaystyle \int \sqrt{x^2+1}\diff x&=&\displaystyle \int \sqrt{\tan^2\theta+1}~\diff (\tan \theta) && \begin{array}{l} \text{Set }\\ x=\tan \theta \\ \theta\in \left(-\frac{\pi}{2}, \frac{\pi }{2}\right)\end{array}\\
&=&\displaystyle \int \sqrt{sec^2\theta}\sec^2\theta\diff \theta && \sec\theta>0\\
&=&\displaystyle \int \sec^3\theta\diff \theta&&\text{Problem } \refBad{\ref{problemintsec^3xdx}}{\text{solved already}}{\ref{problemintsec^3xdx}} \\
&=&\displaystyle \frac{1}{2}\left(\tan \theta \sec \theta+\ln|\sec\theta +\tan \theta | \right)+C&&
\begin{array}{@{}l}
\psset{xunit=0.3cm, yunit=0.3cm}
\begin{pspicture}(-0.8,-0.8)(4.6,3.2)
\tiny
\fcBoundingBox{-1}{-0.8}{4.6}{3.2}
\psline(0,0)(4, 0)(4,3)(0,0)
\psline(3.8,0)(3.8, 0.2)(4,0.2)
\fcAngle{0}{3 4 div ATAN}{0.8}{}
\rput[bl](1, 0.2){$\theta$}
\rput[l](4.2, 1.5){$x$}
\rput[t](2, -0.2){$1$}
\rput[br](2, 1.5){$\sqrt{x^2+1}$}
%bounding box for pdflatex compilation:
\psline[linecolor=red!1](-0.11, -0.3 )(-0.105, -0.3)
\psline[linecolor=red!1](2.3, 1.21)(2.3, 1.205)
\end{pspicture}\\
\sec\theta=\sqrt{x^2+1}\\
\tan \theta=x
\end{array}
\\
&=&\displaystyle \frac{1}{2}\left(x \sqrt{x^2+1} +\ln\left(\sqrt{x^2+1}+x \right) \right)+C
\end{array}
$

}

\solution{\ref{problemIntegrate sqrt(2x^2+2x+1)dx}
\[
\begin{array}{@{}r@{}c@{}l@{}l@{}|l}
\displaystyle \int \sqrt{\left(2x^2+2x+1\right)}\diff x &=&\displaystyle  \int \sqrt{2}\sqrt{\left(\left(x+\frac{1}{2}\right)^2+\frac{1}{4}\right)} \diff x &&\text{complete square}\\
&=&\displaystyle  \frac{\sqrt{2}}{2} \int \sqrt{\left( 4 \left( x + \frac{ 1}{2}\right)^2+1\right)}   \diff {}x\\
&=&\displaystyle  \frac{\sqrt{2}}{2}  \int \sqrt{\left( \left( 2x +1\right)^2+ 1\right)} \frac{1}{2}\diff {}\left(2x+1\right) &&\text{Set }u=2x+1 \\
&=&\displaystyle \frac{\sqrt{2}}{4} \int \sqrt{\left(u^2+1\right)}\diff {}u && \begin{array}{@{}l} \text{Euler subst.: } \\ u = \frac{1}{2}\left(\frac{1}{t}-t\right),\\ t> 0 \\~\\ \diff u = -\frac{1}{2}\left(\frac{1}{t^{2}}+1\right)\diff t \\~\\ \sqrt{u^2+1}= \frac{1}{2}\left(\frac{1}{t}+t\right)\\~\\ t= \sqrt{u^2+1}-u \end{array}\\
&=&\displaystyle - \frac{\sqrt{2} }{16} \int\left(\frac{1}{t}+t\right)\left(\frac{1}{t^2}+1 \right) \diff t\\
&=&\displaystyle - \frac{\sqrt{2} }{16} \int \left( t^{-3}+2t^{-1}+t\right)\diff t\\
&=&\displaystyle - \frac{\sqrt{2} }{16}  \left( -\frac{t^{-2}}{2} + 2\ln |t| + \frac{t^2}{2}\right)+C&& \begin{array}{@{}l}
\text{simplify as} \\
\text{in Problem \ref{problemIntegratesqrt(x^2+1)dx}}
\end{array}
\\
&=&\displaystyle \frac{\sqrt{2}}{8} \left( u\sqrt{u^2+1}+ \ln\left(\sqrt{u^2+1}+u \right) \right) +C \\
&=&\displaystyle \frac{\sqrt{2}}{8} \left( (2x+1)\sqrt{(2x+1)^2+1 }\right.\\
&&\displaystyle ~~~~~~ \left.+\ln \left( \sqrt{(2x+1)^2+1 }+2x+1\right) \right)+C.
\end{array}
\]
}

\solution{\ref{problemintsqrt(x^2+1)/(x+1)dx}

$\begin{array}{@{}r@{}c@{}l@{}l@{}|l}
\displaystyle \int \frac{\sqrt{ x^2+1}}{x+1}\diff x&=& \displaystyle \int \frac{ \frac{1}{2}\left(\frac{1}{t}+t \right) }{ \frac{1}{2}\left(\frac{1}{t}-t \right)+1 }\diff \left(\frac{1}{2} \left(\frac{1}{t}-t \right) \right) &&\begin{array}{l}\text{Euler sub: }\\ \begin{array}{@{\!\!\!\!\!}r@{}c@{}l} x&=&\frac{1}{2}\left(\frac{1}{t}-t\right) \\
\sqrt{x^2+1}&=&\frac{1}{2}\left(\frac{1}{t}+t\right)
\end{array}\end{array}\\
&=&\displaystyle \int \left(\frac{1+t^2}{1-t^2+2t }\right)\frac{1}{2} \left(-t^{-2}-1 \right)\diff t\\
&=&\displaystyle \int \frac{1}{2} \frac{(1+t^2)\left(-t^{-2}-1 \right)}{1-t^2+2t } \diff t\\
&=&\displaystyle \frac{1}{2} \int \frac{ t^{4}+2 t^{2}+1}{ t^{4}-2 t^{3}-t^{2}} \diff t &&\text{pol. long div.}\\
&=&\displaystyle \frac{1}{2}\int \left(1+\frac{ 2t^{3}+ 3t^{2} +1}{ t^2\left(t^{2}-2 t-1\right)}\right) \diff t &&\text{part. fractions}\\
&=&\displaystyle \frac{1}{2}\int \left(1+ \frac{2\sqrt{2}}{t -\sqrt{2}-1}+\frac{-2\sqrt{2}}{t +\sqrt{2}-1}+\frac{2}{t }+\frac{-1}{t^{2}}\right)\diff t\\
&=&\displaystyle -\sqrt{2} \ln{}\left|t+\sqrt{2}-1\right|+\sqrt{2} \ln{}\left|t-\sqrt{2}-1\right|\\
&&\displaystyle+\frac{1}{2} t^{-1}+\ln{}\left|t\right|+\frac{1}{2} t +C&& t=\sqrt{x^2+1}-x\\
&=&\displaystyle -\sqrt{2} \ln{}\left(\sqrt{x^{2}+1}- x+\sqrt{2}-1\right)\\
&&\displaystyle +\sqrt{2} \ln{}\left(\sqrt{x^{2}+1}- x-\sqrt{2}-1\right)\\
&&\displaystyle + \ln{} \left( \sqrt{x^{2}+1}- x\right)\\
&&\displaystyle +\frac{1}{2} \left(\sqrt{x^{2}+1}- x\right)^{-1}+\frac{1}{2} \sqrt{x^{2}+1}-\frac{1}{2} x+C &&\begin{array}{l}\text{Last 3 terms}\\ \text{simplify}\end{array}\\
&=&\displaystyle -\sqrt{2} \ln{}\left(\sqrt{x^{2}+1}- x+\sqrt{2}-1\right) \\
&&\displaystyle +\sqrt{2} \ln{}\left(\sqrt{x^{2}+1}- x-\sqrt{2}-1\right)\\
&&\displaystyle + \ln{} \left( \sqrt{x^{2}+1}- x\right)\\
&&+ \sqrt{x^{2}+1}+C\quad .

\end{array}
$

}