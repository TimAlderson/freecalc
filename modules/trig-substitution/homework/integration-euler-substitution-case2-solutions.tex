\solution{\ref{problemintsqrt(1-x^2)dx} 

\textbf{Variant I.} This integral is possibly fastest to solve directly using a trig substitution. In the next variant of the solution we show the Euler substitution.

\noindent $
\begin{array}{rcl@{}l@{}|l}
\displaystyle \int \sqrt{1-x^2}\diff x &=&\displaystyle  \int \sqrt{1-\cos^2\theta}\diff (\cos \theta) &&\text{Set }x=\cos \theta, \theta\in[0,\pi]\\
&=&\displaystyle \int \sqrt{\sin^2\theta } (-\sin \theta)\diff \theta &&\theta\in [0,\pi]\Rightarrow \sin \theta\geq 0\\
&=&\displaystyle -\int \sin^2\theta \diff \theta && \sin^2\theta= \frac{1+\cos (2\theta) }{2}\\
&=&\displaystyle-\int \frac{1+\cos (2\theta)}{2}\diff \theta\\
&=&\displaystyle - \frac{\theta}{2}-\frac{ \sin (2\theta)}{4}+C&&
\begin{array}{rcl}
\theta &=&\arccos \theta
\end{array}
\end{array}
$


\textbf{Variant II.} We show how to do this integral via the Euler substitution $x=\cos (2\Arctan t)$.

\textbf{Variant III. }
}

\solution{\ref{problemintsqrt(1-x^2)/(1+x)dx} In this problem solution we use the standard Euler substitution $x=\cos (2\Arctan t)$. We recall \refBad{\ref{eqEulerSubx=cos(2arctant)}}{that}{from \eqref{eqEulerSubx=cos(2arctant)} that}

\noindent $\begin{array}{rcl}
\displaystyle x& =&\displaystyle \cos(2\Arctan t)= \frac{1-t^2}{1+t^2}\\
\Arccos (x)&=& 2 \Arctan t\\
\displaystyle \diff x &=&\displaystyle  -\frac{4t}{(1+t^2)^2}\diff t\\
\displaystyle \sqrt{1-x^2}&=&\displaystyle \sin (2\Arctan t)= \frac{2t}{1+t^2}\\
t&=&\displaystyle \frac{\sqrt{1-x^2}}{x+1}\quad .
\end{array}
$

\noindent $
\begin{array}{@{}r@{}c@{}l@{}l@{}|l}
\displaystyle \int  \frac{\sqrt{1-x^2}}{1+x}\diff x&=& \displaystyle \int t \left(-\frac{4t}{\left(1+t^2\right)^2} \right)\diff t&&\begin{array}{l} \text{Set }x=\frac{1-t^2}{1+t^2}\\
\text{Use f-las above}
\end{array}
\\
&=&\displaystyle -4\int\frac{t^2}{\left(1+t^2\right)^2}\diff t\\
&=&\displaystyle -4\int\frac{1+t^2-1}{\left(1+t^2\right)^2 }\diff t\\
&=&\displaystyle -4\int\left(\frac{1}{1+t^2} -\frac{1}{\left(1+t^2\right)^2}\right)\diff t\\
&=&\displaystyle -4\left(\Arctan t - \frac{1}{2}\left(\Arctan t+ \frac{t}{1+t^2} \right) \right)+C\\
&=&\displaystyle -2\left(\Arctan {}t - \frac{t}{1+t^2} \right) +C\\
&=&\displaystyle -2\left( \Arctan {}\left(\frac{\sqrt{1-x^2}}{1+x} \right) - \frac{1}{2}\sqrt{1-x^2} \right) +C\\
&=&\displaystyle -2 \Arctan {}t  +\sqrt{1-x^2} +C &&\text{Use f-las above}
\\
&=&\displaystyle -\Arccos x+ \sqrt{1-x^2}  +C\\
&=&\displaystyle \Arcsin x +\sqrt{1-x^2}+K\quad .
\end{array}
$

We have included the last equality to remind the student that derivatives of $\Arcsin(x)$ and $\Arccos x$ are negatives of one another.


}