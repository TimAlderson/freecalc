%begin module Euler-substitution-case-1-cot
\begin{frame}
\frametitle{Euler subst. for $\sqrt{x^2+1}$ corresponding to $x=\cot \theta$ }
Recall $\theta=2\arctan t$ transforms a trigonometric integral into integral of a rational function. Carry out the substitution $\theta= \theta=2\arctan t$:
\[
\begin{array}{rcll|l}
x&=&\displaystyle \cot \theta \\
&=& \displaystyle \cot \left(2\arctan t\right) &&\displaystyle  \text{Recall: } \cot 2z=\frac{\cos (2z)}{\sin (2z)}=\frac{1-\tan^2z}{2\tan z } \\
&=&\displaystyle \frac{1-\tan^2 (\arctan t)}{2 \tan (\arctan t)} \\
&=&\displaystyle \frac{1-t^2}{2t}\\
&=&\displaystyle \frac{1}{2}\left(\frac{1}t -t\right)\quad .
\end{array}
\]
We can furthermore compute 
\begin{equation} \label{eqsqrtx2plus1Euler2}
\begin{array}{rcll|l}
\displaystyle \sqrt{x^2+1}&=& \displaystyle  \sqrt{ \frac{1}{4} \left(\frac{1}t -t \right)^2 +1}\\
&=&\displaystyle \frac{1}{2} \sqrt{\left( \frac{1}{t} +t \right)^2} & &\displaystyle \sqrt{\left(\frac{1}{t}+t\right)^2} = \frac{1}{t} +t \text{ because }t>0\\
&=&\displaystyle \frac{1}{2}\left(\frac{1}{t}+t\right)\quad .
\end{array}
\end{equation}
The differential $\diff x$ can be expressed via $\diff t$ from $\displaystyle x=\frac{1}{2} \left( \frac{1}{t} - t\right)$. Finally, we can subtract $\displaystyle x=\frac{1}{2} \left( \frac{1}{t} - t\right)$ from  $\displaystyle \sqrt{x^2+1}= \frac{1}{2} \left( \frac{1 }{ t} +t\right)$ to we get that $t=\sqrt{x^2+1}-x $. The Euler substitution $x=\cot \theta= \cot (\arctan 2t)$ can be now summarized as:
\begin{equation}\label{eqEulerSub1.2}
\begin{array}{rcl}
x&=&\displaystyle \frac12\left(\frac{1}{t}- t\right)\\
\displaystyle\sqrt{x^2+1}&=& \displaystyle \frac12 \left(\frac1t +t\right)\\ 
\displaystyle \diff x&=&\displaystyle -\frac12\left(\frac{1}{t^2}+1\right) \diff t\\
t &=&\sqrt{x^2+1}-x\quad .
\end{array}
\end{equation}
As demonstrated in the present section, it is sufficient to memorize the substitution $x=\cot (2\arctan t)$ in order to derive the equalities above. An alternative way to memorize the Euler substitution is through the equality 
\[
\sqrt{x^2+1}=x+t\quad .
\]
\end{frame}

%end module Euler-substitution-case-1-cot