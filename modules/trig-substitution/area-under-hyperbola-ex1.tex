%begin module area-under-hyperbola-ex1

\vskip -0.05cm
\begin{frame}
\begin{example}
Find the area locked b-n the hyperbolas $\alertNoH{2,3}{ y=\pm \sqrt{ x^2+1}}$ and $x=\pm 2\sqrt{ 2}$.
\begin{columns}
\column{.5\textwidth}
\psset{xunit=0.7cm, yunit=0.7cm}%
\begin{pspicture}(-4.15, -3.05)(4.15,3.05)%
\psframe*[linecolor=white](-4.15,-3)(4.15,3)%
\tiny%
\pstVerb{20 dict begin 
/rightX 2 sqrt 2 mul def /theFun {dup mul 1 add sqrt} def 
/rightY rightX theFun def 
/leftX rightX -1 mul def
/leftY leftX theFun def
}%
\uncover<handout:3|31->{%
\pscustom*[linecolor=\fcColorAreaUnderGraph]{%
\psplot[plotpoints = 1000 ] {leftX} {rightX}{x theFun}
\psline(! rightX -3)(! rightX 3)
\psplot[plotpoints=1000] {rightX } {leftX}{x theFun -1 mul}
\psline(! leftX -3)(! leftX 3)
}%
}%
\uncover<handout:0|33->{%
\pscustom*[linecolor=cyan!15]{%
\psplot[plotpoints = 1000 ] {leftX} {0}{x theFun}
\psline(0,1)(0,-1)
\psplot[plotpoints=1000] { 0 } {leftX}{x theFun -1 mul}
\psline(! leftX leftY -1 mul)(! leftX leftY)%
}%
}%
\uncover<handout:0|37->{%
\pscustom*[linecolor=orange]{%
\psline(0,0)(! rightX rightY)(! rightX rightY -1 mul)(0, 0)
}%
}%
\uncover<handout:0|38->{%
\pscustom*[linecolor=green]{%
\psplot[plotpoints = 1000 ] {0} {rightX}{x theFun}%
\psline(! rightX rightY)(0,0)(! 0 0 theFun)%
}%
\pscustom*[linecolor=green]{%
\psplot[plotpoints = 1000 ] {0} {rightX}{x theFun -1 mul}%
\psline(! rightX rightY -1 mul)(0,0)(! 0 0 theFun)%
}%
}%
\uncover<handout:1,3|1-26,28->{%
\psaxes[arrows=<->,ticks=none, labels=none](0,0)(-4,-3)(4,3)%
}%
\psline[linecolor=red!1](3.301,2)(3.302,2)%
\psline[linecolor=red!1](-3.301,2)(-3.302,2)%
%Function formula: - (x^{2}+1)^{1/2}
\psplot[linecolor=\fcColorGraph, plotpoints=1000]{-2.828427}{2.828427}{1 x 2 exp add 0.5 exp -1 mul }%
\uncover<3-4>{\rput[tl](-2.2, -2.4){ \alertNoH{3}{ $y= - \sqrt{ x^2 +1 }$}}}%
%Function formula: (x^{2}+1)^{1/2}
\psplot[linecolor=\fcColorGraph, plotpoints=1000]{-2.828427}{ 2.828427 }{1 x 2 exp add 0.5 exp }%
\uncover<2-4>{\rput[bl](-2.1, 2.4){\alertNoH{2}{ $y=\sqrt{ x^2 +1} $}}}%
\uncover<handout:3|29->{%
\psline[linecolor=\fcColorGraph](-2.828427,3)(-2.828427,-3)%
}%
\uncover<handout:3|30->{%
\psline[linecolor=\fcColorGraph](2.828427,3)(2.828427,-3)%
}%
\uncover<handout:1,2|25-27>{
\psline{<->}(-2.9,2.9)(2.9,-2.9)
\rput[t](-2.1, 1.7){$\begin{array}{l} \alertNoH{25}{v=0} \\\uncover<1-26>{\alertNoH{25}{y+x=0}} \end{array}$}
}
\uncover<handout:1,2|15-27>{
\psline{<->}(-2.9,-2.9)(2.9,2.9)
\rput[b](-2.1, -1.9){$\begin{array}{l} \uncover<1-26>{ \alertNoH{15}{ y-x=0 }}\\\uncover<16->{\alertNoH{16}{u=0}} \end{array}$}
}
\uncover<handout:1|17-26>{
\fcFullDot{1.4}{1.4}
\rput[l]( 1.6, 1.4){$(\frac{y+x}{2},\frac{y+x}{2})$}
}
\uncover<handout:1|14-26>{
\fcFullDot{0.6}{2.2}
\rput[lb](0.65, 2.2){$(x,y)$}
}
\uncover<handout:1|26>{
\psline(0.6,2.2)(-0.8,0.8)
\psline(-0.7, 0.9)(-0.6, 0.8)(-0.7, 0.7)
\rput[rb](-0.3, 1.3){\alertNoH{26}{$v$}}
}
\uncover<handout:1|18-26>{
\psline(0.6,2.2)(1.4, 1.4)
\psline(1.3, 1.5)(1.2,1.4)(1.3, 1.3)
}
\uncover<handout:1|23-26>{
\rput[tr](0.95, 1.8){\alertNoH{23}{$u$}}
}
\uncover<handout:1|14-26>{
\fcFullDot{2.2}{0.6}
\rput[lt]( 2.2, 0.65){$(y,x)$}
}%
\uncover<handout:3|34->{%
\psline[linecolor=black](! rightX rightY)(! leftX leftY -1 mul)%
\psline[linecolor=black](! rightX rightY -1 mul)(! leftX leftY)%
}%
\uncover<handout:0|35>{%
\psline[linecolor=brown, linewidth=2.5pt](0, 0)(! rightX 0)%
}
\uncover<handout:3|36->{%
\psline[linecolor=brown, linewidth=1pt](0, 0)(! rightX 0)%
}%
\uncover<handout:3|35->{%
\rput[t](! rightX 2 div -0.1){$\alertNoH{35}{x}$}%
}%
\uncover<handout:0|36>{%
\fcPerpendicular[linewidth=2.5pt, linecolor=brown]{[rightX rightY]}{[1 0]}{0.2}%
}%
\uncover<handout:3|36->{%
\fcPerpendicular[linewidth=1pt, linecolor=brown]{[rightX rightY]}{[1 0]}{0.2}%
\rput[l](! rightX rightY 2 div ){$\alertNoH{36}{\sqrt{x^2+1}}$}%
}%
\pstVerb{end}%
\end{pspicture}

\vbox to 3.0cm {
\only<handout:1,2|1-27>{
\uncover<18->{\alertNoH{18}{
\uncover<22->{\alertNoH{22}{Signed}} distance b-n $(x,y)$ and line $u=0$ equals}}
\only<handout:1|1-23>{
$\uncover<19->{\uncover<22->{\alertNoH{22}{\pm}} \alertNoH{19}{ \sqrt{ \alertNoH{20}{ \left(x-\frac{(x+y)}{2} \right)^2+ \left( y- \frac{(x+y )}{2} \right)^2}}}}
$
$\uncover<20->{=\uncover<22->{\alertNoH{22}{\pm}} \sqrt{ \alertNoH{20}{ \frac{1}{2}(y-x)^2 }}} \uncover<21->{= \alertNoH{21}{ \uncover<1-21>{\pm} \alertNoH{23}{ \frac{\sqrt{2 }}{ 2 } ( y-x)}}} \uncover<23>{ \alertNoH{23}{=}}$
} %only<1-23>
\uncover<23->{ \alertNoH{23,24}{$u $}.}
\only<handout:2|24->{\uncover<25->{
Similarly compute that \alertNoH{26}{signed distance b-n $(x,y)$ and the \alertNoH{25}{line $v=0$} equals $v$}.
\uncover<27->{$\Rightarrow$ $y^2-x^2=1$ is the \alertNoH{27}{ hyperbola $v=\frac{1/2}{u}$} in the $(u,v)$-plane.}
}}
}%only<1-27>
\vfil
} %vbox

\column {.5\textwidth}
\only<handout:1,2|1-27>{
\uncover<4->{We studied $\alertNoH{27}{v=\frac{\frac{1}{2} }{u }}$ is called a hyperbola:}\uncover<3->{ why do we call $y= \sqrt{ x^2 +1}$ hyperbola?} \uncover<5->{Compute:}
\[
\begin{array}{rcl}
\uncover<5->{\sqrt{x^2+1} &=& y}\\
\uncover<6->{ x^2+1 &=& y^2}\\
\uncover<7->{y^2-x^2&=&1}\\
\uncover<8->{\uncover<handout:0|9>{\alertNoH{9}{\frac{1}{2}}} \uncover<10->{\alertNoH{10,11}{\frac{\sqrt{2}}{2}}} \alertNoH{11}{(y-x)} \uncover<10->{\alertNoH{10,12}{\frac{\sqrt{2}}{2}}} \alertNoH{12}{(y+x)}&=&\uncover<9->{\alertNoH{9}{\frac{1}{2}}} \uncover<handout:0|8>{1}}\\
\uncover<11->{\alertNoH{11}{u}\alertNoH{12}{v}&=& \frac{1}{2}}\\
\uncover<13->{\alertNoH{27}{v}&\alertNoH{27}{=}& \alertNoH{27}{\frac{\frac{1}{2}}{u}},}
\end{array}
\]
\uncover<11->{where $\begin{array}{|l}
\alertNoH{11,16,23}{u=\frac{\sqrt{2}}{2} \left(y-x\right)}\\
\alertNoH{12,25}{v=\frac{\sqrt{2}}{2}\left(y+x\right)}
\end{array}$. } \uncover<14->{Consider an arbitrary point $(x,y)$.}
} %only<1-27>
\only<handout:3|28->{
The area in question is:
$
\begin{array}{l}
\displaystyle\phantom{=} \int \limits^{{{  \fcAnswerUncover{ 28 }{30}{ 2 \sqrt{ 2} } } } }_{  \fcAnswer{29}{ -2 \sqrt{2 }}} 2\sqrt{x^2+1} \diff x \\
\displaystyle \uncover<32->{= \uncover<33->{\alertNoH{33}{2}} \left[ \alertNoH{35,37}{ x} \alertNoH{36,37}{\sqrt{x^2+1}} \vphantom{\ln \left(\sqrt{x^2+1}+x\right) } \right.}\\
\displaystyle \uncover<32->{\left. + \alertNoH{38}{\ln \left(\sqrt{x^2+1}+x \right) }\right]^{2\sqrt{2}}_{\only<33->{\alertNoH{33}{0}} \uncover<handout:0|1-32>{-2\sqrt{2}}}}\\
\uncover<39->{=2\left(2\sqrt{2} \sqrt{(2\sqrt{2})^2+1}\right.} \\
\uncover<40->{\left.+ \ln \left(\sqrt{(2 \sqrt{2 })^2+ 1} + 2 \sqrt{2} \right) \right)}\\
\uncover<41->{=12\sqrt{2} +2\ln \left(3+2\sqrt{2}\right )}\\
\uncover<42->{\approx 20.496}
\end{array}
$
}
\end{columns}

\end{example}

\end{frame}

%end module area-under-hyperbola-ex1
