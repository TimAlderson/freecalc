% begin module transformations-shifts-example
\begin{frame}
\begin{example}
Draw a graph of the function $f(x) = x^2 + 6x + 10$.
\begin{columns}[c]
\column{.5\textwidth}

\psset{xunit=0.7cm, yunit=0.7cm}
\begin{pspicture}(-5.1, -0.6)(3.1,5.1) 
\tiny
\psaxesStandard{-5.03}{-0.5}{3}{5}
\rput[rt](-0.1,-0.1){$(0,0)$}
\psFullDot{0}{0}
%Function formula: (-3+x)^{2}-9+6 (x) 
\only<handout:0| 7>{
\rput(2.5,2){$y=x^2$} 
\psplot[linecolor=red, plotpoints=1000]{-2.23607}{2.23607}{x 6 mul -9 x -3 add 2 exp add add } %Function formula: (x)^{2}+10+6 (x) 
}
\only<handout:0| 8->{
\rput(2.5,2){\color{gray}$y=x^2$} 
\psplot[linecolor=gray, plotpoints=1000]{-2.23607}{2.23607}{x 6 mul -9 x -3 add 2 exp add add } %Function formula: (x)^{2}+10+6 (x) 
\rput[rt](-3.1,0.9){$(-3,1)$}
\psline{->}(0,0)(-3,1)
\psFullDot{-3}{1}

\rput[b](-3,4.2){\alert<8->{$y=x^{2}+6x+10$}} 
\psplot[linecolor=red, plotpoints=1000]{-5}{-1}{x 6 mul 10 x 2 exp add add }
}
\end{pspicture} 

\column{.5\textwidth}
\uncover<2->{
Complete the square:
}
\begin{eqnarray*}
\uncover<3->{f(x)} & \uncover<3->{ = } & \uncover<3->{x^2 + 6x + 10} \\
& \uncover<4->{ = } & \uncover<4->{(x^2 + 6x \uncover<5->{\alert<handout:0| 5>{+ 9}}) + 10 \uncover<5->{\alert<handout:0| 5>{- 9}}} \\
 & \uncover<6->{ = } & \uncover<6->{(x + 3)^2 + 1} \\
\end{eqnarray*}
\end{columns}
\uncover<7-8>{}
\end{example}
\end{frame}
% end module transformations-shifts-example
